%% Generated by Sphinx.
\def\sphinxdocclass{report}
\documentclass[letterpaper,10pt,english]{sphinxmanual}
\ifdefined\pdfpxdimen
   \let\sphinxpxdimen\pdfpxdimen\else\newdimen\sphinxpxdimen
\fi \sphinxpxdimen=.75bp\relax
\ifdefined\pdfimageresolution
    \pdfimageresolution= \numexpr \dimexpr1in\relax/\sphinxpxdimen\relax
\fi
%% let collapsible pdf bookmarks panel have high depth per default
\PassOptionsToPackage{bookmarksdepth=5}{hyperref}
%% turn off hyperref patch of \index as sphinx.xdy xindy module takes care of
%% suitable \hyperpage mark-up, working around hyperref-xindy incompatibility
\PassOptionsToPackage{hyperindex=false}{hyperref}
%% memoir class requires extra handling
\makeatletter\@ifclassloaded{memoir}
{\ifdefined\memhyperindexfalse\memhyperindexfalse\fi}{}\makeatother

\PassOptionsToPackage{warn}{textcomp}

\catcode`^^^^00a0\active\protected\def^^^^00a0{\leavevmode\nobreak\ }
\usepackage{cmap}
\usepackage{fontspec}
\defaultfontfeatures[\rmfamily,\sffamily,\ttfamily]{}
\usepackage{amsmath,amssymb,amstext}
\usepackage{polyglossia}
\setmainlanguage{english}



\setmainfont{FreeSerif}[
  Extension      = .otf,
  UprightFont    = *,
  ItalicFont     = *Italic,
  BoldFont       = *Bold,
  BoldItalicFont = *BoldItalic
]
\setsansfont{FreeSans}[
  Extension      = .otf,
  UprightFont    = *,
  ItalicFont     = *Oblique,
  BoldFont       = *Bold,
  BoldItalicFont = *BoldOblique,
]
\setmonofont{FreeMono}[
  Extension      = .otf,
  UprightFont    = *,
  ItalicFont     = *Oblique,
  BoldFont       = *Bold,
  BoldItalicFont = *BoldOblique,
]



\usepackage[Bjarne]{fncychap}
\usepackage{sphinx}

\fvset{fontsize=\small}
\usepackage{geometry}


% Include hyperref last.
\usepackage{hyperref}
% Fix anchor placement for figures with captions.
\usepackage{hypcap}% it must be loaded after hyperref.
% Set up styles of URL: it should be placed after hyperref.
\urlstyle{same}


\usepackage{sphinxmessages}



% Jupyter Notebook code cell colors
\definecolor{nbsphinxin}{HTML}{307FC1}
\definecolor{nbsphinxout}{HTML}{BF5B3D}
\definecolor{nbsphinx-code-bg}{HTML}{F5F5F5}
\definecolor{nbsphinx-code-border}{HTML}{E0E0E0}
\definecolor{nbsphinx-stderr}{HTML}{FFDDDD}
% ANSI colors for output streams and traceback highlighting
\definecolor{ansi-black}{HTML}{3E424D}
\definecolor{ansi-black-intense}{HTML}{282C36}
\definecolor{ansi-red}{HTML}{E75C58}
\definecolor{ansi-red-intense}{HTML}{B22B31}
\definecolor{ansi-green}{HTML}{00A250}
\definecolor{ansi-green-intense}{HTML}{007427}
\definecolor{ansi-yellow}{HTML}{DDB62B}
\definecolor{ansi-yellow-intense}{HTML}{B27D12}
\definecolor{ansi-blue}{HTML}{208FFB}
\definecolor{ansi-blue-intense}{HTML}{0065CA}
\definecolor{ansi-magenta}{HTML}{D160C4}
\definecolor{ansi-magenta-intense}{HTML}{A03196}
\definecolor{ansi-cyan}{HTML}{60C6C8}
\definecolor{ansi-cyan-intense}{HTML}{258F8F}
\definecolor{ansi-white}{HTML}{C5C1B4}
\definecolor{ansi-white-intense}{HTML}{A1A6B2}
\definecolor{ansi-default-inverse-fg}{HTML}{FFFFFF}
\definecolor{ansi-default-inverse-bg}{HTML}{000000}

% Define an environment for non-plain-text code cell outputs (e.g. images)
\makeatletter
\newenvironment{nbsphinxfancyoutput}{%
    % Avoid fatal error with framed.sty if graphics too long to fit on one page
    \let\sphinxincludegraphics\nbsphinxincludegraphics
    \nbsphinx@image@maxheight\textheight
    \advance\nbsphinx@image@maxheight -2\fboxsep   % default \fboxsep 3pt
    \advance\nbsphinx@image@maxheight -2\fboxrule  % default \fboxrule 0.4pt
    \advance\nbsphinx@image@maxheight -\baselineskip
\def\nbsphinxfcolorbox{\spx@fcolorbox{nbsphinx-code-border}{white}}%
\def\FrameCommand{\nbsphinxfcolorbox\nbsphinxfancyaddprompt\@empty}%
\def\FirstFrameCommand{\nbsphinxfcolorbox\nbsphinxfancyaddprompt\sphinxVerbatim@Continues}%
\def\MidFrameCommand{\nbsphinxfcolorbox\sphinxVerbatim@Continued\sphinxVerbatim@Continues}%
\def\LastFrameCommand{\nbsphinxfcolorbox\sphinxVerbatim@Continued\@empty}%
\MakeFramed{\advance\hsize-\width\@totalleftmargin\z@\linewidth\hsize\@setminipage}%
\lineskip=1ex\lineskiplimit=1ex\raggedright%
}{\par\unskip\@minipagefalse\endMakeFramed}
\makeatother
\newbox\nbsphinxpromptbox
\def\nbsphinxfancyaddprompt{\ifvoid\nbsphinxpromptbox\else
    \kern\fboxrule\kern\fboxsep
    \copy\nbsphinxpromptbox
    \kern-\ht\nbsphinxpromptbox\kern-\dp\nbsphinxpromptbox
    \kern-\fboxsep\kern-\fboxrule\nointerlineskip
    \fi}
\newlength\nbsphinxcodecellspacing
\setlength{\nbsphinxcodecellspacing}{0pt}

% Define support macros for attaching opening and closing lines to notebooks
\newsavebox\nbsphinxbox
\makeatletter
\newcommand{\nbsphinxstartnotebook}[1]{%
    \par
    % measure needed space
    \setbox\nbsphinxbox\vtop{{#1\par}}
    % reserve some space at bottom of page, else start new page
    \needspace{\dimexpr2.5\baselineskip+\ht\nbsphinxbox+\dp\nbsphinxbox}
    % mimick vertical spacing from \section command
      \addpenalty\@secpenalty
      \@tempskipa 3.5ex \@plus 1ex \@minus .2ex\relax
      \addvspace\@tempskipa
      {\Large\@tempskipa\baselineskip
             \advance\@tempskipa-\prevdepth
             \advance\@tempskipa-\ht\nbsphinxbox
             \ifdim\@tempskipa>\z@
               \vskip \@tempskipa
             \fi}
    \unvbox\nbsphinxbox
    % if notebook starts with a \section, prevent it from adding extra space
    \@nobreaktrue\everypar{\@nobreakfalse\everypar{}}%
    % compensate the parskip which will get inserted by next paragraph
    \nobreak\vskip-\parskip
    % do not break here
    \nobreak
}% end of \nbsphinxstartnotebook

\newcommand{\nbsphinxstopnotebook}[1]{%
    \par
    % measure needed space
    \setbox\nbsphinxbox\vbox{{#1\par}}
    \nobreak % it updates page totals
    \dimen@\pagegoal
    \advance\dimen@-\pagetotal \advance\dimen@-\pagedepth
    \advance\dimen@-\ht\nbsphinxbox \advance\dimen@-\dp\nbsphinxbox
    \ifdim\dimen@<\z@
      % little space left
      \unvbox\nbsphinxbox
      \kern-.8\baselineskip
      \nobreak\vskip\z@\@plus1fil
      \penalty100
      \vskip\z@\@plus-1fil
      \kern.8\baselineskip
    \else
      \unvbox\nbsphinxbox
    \fi
}% end of \nbsphinxstopnotebook

% Ensure height of an included graphics fits in nbsphinxfancyoutput frame
\newdimen\nbsphinx@image@maxheight % set in nbsphinxfancyoutput environment
\newcommand*{\nbsphinxincludegraphics}[2][]{%
    \gdef\spx@includegraphics@options{#1}%
    \setbox\spx@image@box\hbox{\includegraphics[#1,draft]{#2}}%
    \in@false
    \ifdim \wd\spx@image@box>\linewidth
      \g@addto@macro\spx@includegraphics@options{,width=\linewidth}%
      \in@true
    \fi
    % no rotation, no need to worry about depth
    \ifdim \ht\spx@image@box>\nbsphinx@image@maxheight
      \g@addto@macro\spx@includegraphics@options{,height=\nbsphinx@image@maxheight}%
      \in@true
    \fi
    \ifin@
      \g@addto@macro\spx@includegraphics@options{,keepaspectratio}%
    \fi
    \setbox\spx@image@box\box\voidb@x % clear memory
    \expandafter\includegraphics\expandafter[\spx@includegraphics@options]{#2}%
}% end of "\MakeFrame"-safe variant of \sphinxincludegraphics
\makeatother

\makeatletter
\renewcommand*\sphinx@verbatim@nolig@list{\do\'\do\`}
\begingroup
\catcode`'=\active
\let\nbsphinx@noligs\@noligs
\g@addto@macro\nbsphinx@noligs{\let'\PYGZsq}
\endgroup
\makeatother
\renewcommand*\sphinxbreaksbeforeactivelist{\do\<\do\"\do\'}
\renewcommand*\sphinxbreaksafteractivelist{\do\.\do\,\do\:\do\;\do\?\do\!\do\/\do\>\do\-}
\makeatletter
\fvset{codes*=\sphinxbreaksattexescapedchars\do\^\^\let\@noligs\nbsphinx@noligs}
\makeatother



\title{fmdtools}
\date{Jan 05, 2022}
\release{0.9.4}
\author{Daniel Hulse, Sequoia Andrade, Hannah Walsh}
\newcommand{\sphinxlogo}{\vbox{}}
\renewcommand{\releasename}{Release}
\makeindex
\begin{document}

\pagestyle{empty}
\sphinxmaketitle
\pagestyle{plain}
\sphinxtableofcontents
\pagestyle{normal}
\phantomsection\label{\detokenize{index::doc}}


\noindent\sphinxincludegraphics[width=800\sphinxpxdimen]{{logo_detailed}.jpg}

\sphinxAtStartPar
\sphinxstylestrong{fmdtools} (Fault Model Design tools) is a toolkit for modelling system resilience in the early design phase. With it, one can simulate the effects of faults in a system to build resilience into the system design at a high level.  To achieve this, fmdtools provides a Python\sphinxhyphen{}based \sphinxstyleemphasis{design environment} where one can represent the system in a model, simulate the resilience of the model to faults, and analyze the resulting model responses to iteratively improve the resilience of the design.


\chapter{Overview}
\label{\detokenize{index:overview}}
\sphinxAtStartPar
The main impetus for the development of the fmdtools project was a lack existing tools to enable early function\sphinxhyphen{}based fault simulation for early functional hazard assessment. Researchers thus had to re\sphinxhyphen{}implement modelling, simulation, and analysis approaches for each new case study or methodological improvement. The fmdtools resolves this problem by separating resilience modelling, simulation, and analysis constructs from the model under study, enabling reuse of methodology between case studies. Towards this end, the fmdtools package provides three major pieces of functionality:
\begin{enumerate}
\sphinxsetlistlabels{\arabic}{enumi}{enumii}{}{.}%
\item {} 
\sphinxAtStartPar
Model definition constructs which enable systematic early specification of the high level structure and behaviors of a system with concise syntax (fmdtools.modeldef).

\item {} 
\sphinxAtStartPar
Simulation methods which enable the quantification of system performance and propagation of hazards over a wide range of operational scenarios over a wide range of model types (fmdtools.faultsim).

\item {} 
\sphinxAtStartPar
Analysis methods for quantifying resilience and summarizing and visualizing behaviors and properties of interest (fmdtools.resultdisp).

\end{enumerate}


\section{Key Features}
\label{\detokenize{index:key-features}}
\sphinxAtStartPar
fmdtools was developed with a number of unique features that differentiate it from existing safety/resilience simulation tools.
\begin{itemize}
\item {} 
\sphinxAtStartPar
fmdtools uses an object\sphinxhyphen{}oriented undirected graph\sphinxhyphen{}based model representation which enables arbitrary propagation of flow states through a model graph. As opposed to a \sphinxstyleemphasis{procedural} \sphinxstyleemphasis{directed} graph\sphinxhyphen{}based model representation (a typical strategy for developing fault models in code in which each function or component is represented by a method, the inputs and outputs are which are connected with connected functions/components in a larger model method), this enables one to:
\sphinxhyphen{} propagate behaviors in multiple directions in a model graph, e.g., closing a valve will not just reduce flow in the downstream pipe but also increase pressure in upstream pipes.
\sphinxhyphen{} define the data structures defining a function/component (e.g. states, faults, timed events) with the behavioral methods in a single logical structure that can be re\sphinxhyphen{}used and modified for similar components and methods (that is, a class, instead of a set of unstructured variables and methods)

\item {} 
\sphinxAtStartPar
fmdtools can represent the system at varying levels of fidelity through the design process so that one can start with a simple model and analysis and make it more detailed as the design is elaborated. A typical process of representing the system (from less to more detail) would involve:
\sphinxhyphen{} Creating a network representation of the model functions and flows to visualize the system and identify structurally\sphinxhyphen{}important parts of the model’s causal structure
\sphinxhyphen{} Elaborating the flow attributes and function failure logic in a static propagation to simulate the timeless effects of faults in the model
\sphinxhyphen{} Adding dynamic states and behaviors to the functions as well as a simulation times and operational phases in a dynamic propagation model to simulate the dynamic effects of faults simulated during different time\sphinxhyphen{}steps
\sphinxhyphen{} Instantiating functions with component architectures to compare the expected resilience and behaviors of each
\sphinxhyphen{} Defining stochastic behavioral and input parameters to simulate and analyze system resilience throughout the operational envelope

\item {} 
\sphinxAtStartPar
fmdtools provides convenience methods for quickly visualizing the results of fault simulations with commonly\sphinxhyphen{}used Python libraries to enable one to quickly assess:
\sphinxhyphen{} effects of faults on functions and flows in the model graph at a given time\sphinxhyphen{}step
\sphinxhyphen{} the behavior of system states over time in nominal and faulty scenarios over a range of operational parameters
\sphinxhyphen{} the effect of model input parameters (e.g., ranges, stochastic inputs) on nominal/faulty operations
\sphinxhyphen{} the high\sphinxhyphen{}level results of a set of simulations in an FMEA\sphinxhyphen{}style table of faults, effects, rates, costs, and overall risk
\sphinxhyphen{} simulation responses over a range or distribution of model and scenario parameters

\end{itemize}

\sphinxAtStartPar
An overview of an earlier version of fmdtools (0.6.2) is provided in the paper:

\sphinxAtStartPar
\sphinxhref{https://doi.org/10.36001/ijphm.2021.v12i3.2954}{Hulse, D., Walsh, H., Dong, A., Hoyle, C., Tumer, I., Kulkarni, C., \& Goebel, K. (2021). fmdtools: A Fault Propagation Toolkit for Resilience Assessment in Early Design. International Journal of Prognostics and Health Management, 12(3).}

\sphinxAtStartPar
fmdtools is a research code and is under active development. As a result, Some use\sphinxhyphen{}cases may not work as desired and may change. If you find a bug or would like to contribute, contact the contributors.


\chapter{Getting Started}
\label{\detokenize{index:getting-started}}
\sphinxAtStartPar
The latest public version of fmdtools can be downloaded from the \sphinxhref{https://github.com/NASA-SW-VnV/fmdtools/}{fmdtools github repository} e.g., using:

\begin{sphinxVerbatim}[commandchars=\\\{\}]
\PYG{n}{git} \PYG{n}{clone} \PYG{n}{https}\PYG{p}{:}\PYG{o}{/}\PYG{o}{/}\PYG{n}{github}\PYG{o}{.}\PYG{n}{com}\PYG{o}{/}\PYG{n}{NASA}\PYG{o}{\PYGZhy{}}\PYG{n}{SW}\PYG{o}{\PYGZhy{}}\PYG{n}{VnV}\PYG{o}{/}\PYG{n}{fmdtools}\PYG{o}{.}\PYG{n}{git}
\end{sphinxVerbatim}

\sphinxAtStartPar
A version of the fmdtools toolkit can also be installed directly from the \sphinxhref{https://pypi.org/project/fmdtools/}{PyPI package repository} using \sphinxtitleref{pip install fmdtools}.


\section{Prerequisites}
\label{\detokenize{index:prerequisites}}
\sphinxAtStartPar
fmdtools requires Python 3 and depends directly on these packages (see requirements.txt):

\begin{sphinxVerbatim}[commandchars=\\\{\}]
\PYG{n}{scipy}
\PYG{n}{tqdm}
\PYG{n}{multiprocessing}
\PYG{n}{networkx}
\PYG{n}{numpy}
\PYG{n}{matplotlib}
\PYG{n}{pandas}
\PYG{n}{ordered}\PYG{o}{\PYGZhy{}}\PYG{n+nb}{set}
\PYG{n}{dill}
\PYG{n}{pickle}
\end{sphinxVerbatim}

\sphinxAtStartPar
These packages are optional but recommended to enable specific fmdtools use\sphinxhyphen{}cases and work with examples in the repository:

\begin{sphinxVerbatim}[commandchars=\\\{\}]
\PYG{n}{jupyter} \PYG{n}{notebook}                     \PYG{c+c1}{\PYGZsh{}(for repository notebooks)}
\PYG{n}{graphviz}                             \PYG{c+c1}{\PYGZsh{}(to plot using graphviz options)}
\PYG{n}{netgraph}                             \PYG{c+c1}{\PYGZsh{}(to adjust node positions in graphs)}
\PYG{n}{pyvis}                                \PYG{c+c1}{\PYGZsh{}(for interactive html views of model graphs)}
\PYG{n}{quadpy}                               \PYG{c+c1}{\PYGZsh{}(for quadrature sampling)}
\PYG{n}{ffmpeg}                               \PYG{c+c1}{\PYGZsh{}(for animations)}
\PYG{n}{shapely}                              \PYG{c+c1}{\PYGZsh{}(for multirotor model)}
\PYG{n}{pycallgraph2}                         \PYG{c+c1}{\PYGZsh{}(for model profiling)}
\end{sphinxVerbatim}

\sphinxAtStartPar
These must be installed (e.g. using \sphinxcode{\sphinxupquote{pip install packagename}} or \sphinxcode{\sphinxupquote{conda install packagename}}) them before running any of the codes in the repository.


\chapter{Contributions}
\label{\detokenize{index:contributions}}
\sphinxAtStartPar
fmdtools is developed primarily by the \sphinxhref{https://ti.arc.nasa.gov/tech/rse/research/rad/}{Resilience Analysis and Design} research project. External contributions are welcome under a Contributor License Agreement.


\section{Contributors}
\label{\detokenize{index:contributors}}
\sphinxAtStartPar
\sphinxhref{https://github.com/hulsed}{Daniel Hulse} : Primary Author and point\sphinxhyphen{}of\sphinxhyphen{}contact

\sphinxAtStartPar
\sphinxhref{https://github.com/walshh}{Hannah Walsh} : Network Analysis Codes

\sphinxAtStartPar
\sphinxhref{https://ti.arc.nasa.gov/profile/andrade/}{Sequoia Andrade} : Graph visualization graphviz options, Code review


\chapter{Licenses}
\label{\detokenize{index:licenses}}
\begin{sphinxVerbatim}[commandchars=\\\{\}]
\PYG{n}{fmdtools} \PYG{n}{version} \PYG{l+m+mf}{0.9}\PYG{l+m+mf}{.4}
\PYG{o}{\PYGZhy{}}\PYG{o}{\PYGZhy{}}\PYG{o}{\PYGZhy{}}\PYG{o}{\PYGZhy{}}\PYG{o}{\PYGZhy{}}\PYG{o}{\PYGZhy{}}\PYG{o}{\PYGZhy{}}\PYG{o}{\PYGZhy{}}\PYG{o}{\PYGZhy{}}\PYG{o}{\PYGZhy{}}\PYG{o}{\PYGZhy{}}\PYG{o}{\PYGZhy{}}\PYG{o}{\PYGZhy{}}\PYG{o}{\PYGZhy{}}\PYG{o}{\PYGZhy{}}\PYG{o}{\PYGZhy{}}\PYG{o}{\PYGZhy{}}\PYG{o}{\PYGZhy{}}\PYG{o}{\PYGZhy{}}\PYG{o}{\PYGZhy{}}\PYG{o}{\PYGZhy{}}\PYG{o}{\PYGZhy{}}\PYG{o}{\PYGZhy{}}\PYG{o}{\PYGZhy{}}\PYG{o}{\PYGZhy{}}\PYG{o}{\PYGZhy{}}\PYG{o}{\PYGZhy{}}
\PYG{n}{Contributions} \PYG{k+kn}{from} \PYG{n+nn}{fmdtools} \PYG{n}{version} \PYG{l+m+mf}{0.6}\PYG{l+m+mf}{.2} \PYG{p}{(}\PYG{o+ow}{and} \PYG{n}{prior} \PYG{n}{releases}\PYG{p}{)} \PYG{n}{require} \PYG{n}{the} \PYG{n}{following} \PYG{n}{notice}\PYG{p}{:}
\PYG{n}{MIT} \PYG{n}{License}

\PYG{n}{Copyright} \PYG{p}{(}\PYG{n}{c}\PYG{p}{)} \PYG{l+m+mi}{2019} \PYG{n}{Design} \PYG{n}{Engineering} \PYG{n}{Lab}

\PYG{n}{Permission} \PYG{o+ow}{is} \PYG{n}{hereby} \PYG{n}{granted}\PYG{p}{,} \PYG{n}{free} \PYG{n}{of} \PYG{n}{charge}\PYG{p}{,} \PYG{n}{to} \PYG{n+nb}{any} \PYG{n}{person} \PYG{n}{obtaining} \PYG{n}{a} \PYG{n}{copy}
\PYG{n}{of} \PYG{n}{this} \PYG{n}{software} \PYG{o+ow}{and} \PYG{n}{associated} \PYG{n}{documentation} \PYG{n}{files} \PYG{p}{(}\PYG{n}{the} \PYG{l+s+s2}{\PYGZdq{}}\PYG{l+s+s2}{Software}\PYG{l+s+s2}{\PYGZdq{}}\PYG{p}{)}\PYG{p}{,} \PYG{n}{to} \PYG{n}{deal}
\PYG{o+ow}{in} \PYG{n}{the} \PYG{n}{Software} \PYG{n}{without} \PYG{n}{restriction}\PYG{p}{,} \PYG{n}{including} \PYG{n}{without} \PYG{n}{limitation} \PYG{n}{the} \PYG{n}{rights}
\PYG{n}{to} \PYG{n}{use}\PYG{p}{,} \PYG{n}{copy}\PYG{p}{,} \PYG{n}{modify}\PYG{p}{,} \PYG{n}{merge}\PYG{p}{,} \PYG{n}{publish}\PYG{p}{,} \PYG{n}{distribute}\PYG{p}{,} \PYG{n}{sublicense}\PYG{p}{,} \PYG{o+ow}{and}\PYG{o}{/}\PYG{o+ow}{or} \PYG{n}{sell}
\PYG{n}{copies} \PYG{n}{of} \PYG{n}{the} \PYG{n}{Software}\PYG{p}{,} \PYG{o+ow}{and} \PYG{n}{to} \PYG{n}{permit} \PYG{n}{persons} \PYG{n}{to} \PYG{n}{whom} \PYG{n}{the} \PYG{n}{Software} \PYG{o+ow}{is}
\PYG{n}{furnished} \PYG{n}{to} \PYG{n}{do} \PYG{n}{so}\PYG{p}{,} \PYG{n}{subject} \PYG{n}{to} \PYG{n}{the} \PYG{n}{following} \PYG{n}{conditions}\PYG{p}{:}

\PYG{n}{The} \PYG{n}{above} \PYG{n}{copyright} \PYG{n}{notice} \PYG{o+ow}{and} \PYG{n}{this} \PYG{n}{permission} \PYG{n}{notice} \PYG{n}{shall} \PYG{n}{be} \PYG{n}{included} \PYG{o+ow}{in} \PYG{n+nb}{all}
\PYG{n}{copies} \PYG{o+ow}{or} \PYG{n}{substantial} \PYG{n}{portions} \PYG{n}{of} \PYG{n}{the} \PYG{n}{Software}\PYG{o}{.}

\PYG{n}{THE} \PYG{n}{SOFTWARE} \PYG{n}{IS} \PYG{n}{PROVIDED} \PYG{l+s+s2}{\PYGZdq{}}\PYG{l+s+s2}{AS IS}\PYG{l+s+s2}{\PYGZdq{}}\PYG{p}{,} \PYG{n}{WITHOUT} \PYG{n}{WARRANTY} \PYG{n}{OF} \PYG{n}{ANY} \PYG{n}{KIND}\PYG{p}{,} \PYG{n}{EXPRESS} \PYG{n}{OR}
\PYG{n}{IMPLIED}\PYG{p}{,} \PYG{n}{INCLUDING} \PYG{n}{BUT} \PYG{n}{NOT} \PYG{n}{LIMITED} \PYG{n}{TO} \PYG{n}{THE} \PYG{n}{WARRANTIES} \PYG{n}{OF} \PYG{n}{MERCHANTABILITY}\PYG{p}{,}
\PYG{n}{FITNESS} \PYG{n}{FOR} \PYG{n}{A} \PYG{n}{PARTICULAR} \PYG{n}{PURPOSE} \PYG{n}{AND} \PYG{n}{NONINFRINGEMENT}\PYG{o}{.} \PYG{n}{IN} \PYG{n}{NO} \PYG{n}{EVENT} \PYG{n}{SHALL} \PYG{n}{THE}
\PYG{n}{AUTHORS} \PYG{n}{OR} \PYG{n}{COPYRIGHT} \PYG{n}{HOLDERS} \PYG{n}{BE} \PYG{n}{LIABLE} \PYG{n}{FOR} \PYG{n}{ANY} \PYG{n}{CLAIM}\PYG{p}{,} \PYG{n}{DAMAGES} \PYG{n}{OR} \PYG{n}{OTHER}
\PYG{n}{LIABILITY}\PYG{p}{,} \PYG{n}{WHETHER} \PYG{n}{IN} \PYG{n}{AN} \PYG{n}{ACTION} \PYG{n}{OF} \PYG{n}{CONTRACT}\PYG{p}{,} \PYG{n}{TORT} \PYG{n}{OR} \PYG{n}{OTHERWISE}\PYG{p}{,} \PYG{n}{ARISING} \PYG{n}{FROM}\PYG{p}{,}
\PYG{n}{OUT} \PYG{n}{OF} \PYG{n}{OR} \PYG{n}{IN} \PYG{n}{CONNECTION} \PYG{n}{WITH} \PYG{n}{THE} \PYG{n}{SOFTWARE} \PYG{n}{OR} \PYG{n}{THE} \PYG{n}{USE} \PYG{n}{OR} \PYG{n}{OTHER} \PYG{n}{DEALINGS} \PYG{n}{IN} \PYG{n}{THE}
\PYG{n}{SOFTWARE}\PYG{o}{.}
\end{sphinxVerbatim}


\chapter{TABLE OF CONTENTS}
\label{\detokenize{index:table-of-contents}}
\sphinxAtStartPar
\sphinxstylestrong{fmdtools} (Fault Model Design tools) is a toolkit for modelling system resilience in the early design phase. With it, one can simulate the effects of faults in a system to build resilience into the system design at a high level.  To achieve this, fmdtools provides a Python\sphinxhyphen{}based \sphinxstyleemphasis{design environment} where one can represent the system in a model, simulate the resilience of the model to faults, and analyze the resulting model responses to iteratively improve the resilience of the design.


\section{Overview}
\label{\detokenize{README:overview}}\label{\detokenize{README::doc}}
\sphinxAtStartPar
The main impetus for the development of the fmdtools project was a lack existing tools to enable early function\sphinxhyphen{}based fault simulation for early functional hazard assessment. Researchers thus had to re\sphinxhyphen{}implement modelling, simulation, and analysis approaches for each new case study or methodological improvement. The fmdtools resolves this problem by separating resilience modelling, simulation, and analysis constructs from the model under study, enabling reuse of methodology between case studies. Towards this end, the fmdtools package provides three major pieces of functionality:
\begin{enumerate}
\sphinxsetlistlabels{\arabic}{enumi}{enumii}{}{.}%
\item {} 
\sphinxAtStartPar
Model definition constructs which enable systematic early specification of the high level structure and behaviors of a system with concise syntax (fmdtools.modeldef).

\item {} 
\sphinxAtStartPar
Simulation methods which enable the quantification of system performance and propagation of hazards over a wide range of operational scenarios over a wide range of model types (fmdtools.faultsim).

\item {} 
\sphinxAtStartPar
Analysis methods for quantifying resilience and summarizing and visualizing behaviors and properties of interest (fmdtools.resultdisp).

\end{enumerate}


\subsection{Key Features}
\label{\detokenize{README:key-features}}
\sphinxAtStartPar
fmdtools was developed with a number of unique features that differentiate it from existing safety/resilience simulation tools.
\begin{itemize}
\item {} 
\sphinxAtStartPar
fmdtools uses an object\sphinxhyphen{}oriented undirected graph\sphinxhyphen{}based model representation which enables arbitrary propagation of flow states through a model graph. As opposed to a \sphinxstyleemphasis{procedural} \sphinxstyleemphasis{directed} graph\sphinxhyphen{}based model representation (a typical strategy for developing fault models in code in which each function or component is represented by a method, the inputs and outputs are which are connected with connected functions/components in a larger model method), this enables one to:
\sphinxhyphen{} propagate behaviors in multiple directions in a model graph, e.g., closing a valve will not just reduce flow in the downstream pipe but also increase pressure in upstream pipes.
\sphinxhyphen{} define the data structures defining a function/component (e.g. states, faults, timed events) with the behavioral methods in a single logical structure that can be re\sphinxhyphen{}used and modified for similar components and methods (that is, a class, instead of a set of unstructured variables and methods)

\item {} 
\sphinxAtStartPar
fmdtools can represent the system at varying levels of fidelity through the design process so that one can start with a simple model and analysis and make it more detailed as the design is elaborated. A typical process of representing the system (from less to more detail) would involve:
\sphinxhyphen{} Creating a network representation of the model functions and flows to visualize the system and identify structurally\sphinxhyphen{}important parts of the model’s causal structure
\sphinxhyphen{} Elaborating the flow attributes and function failure logic in a static propagation to simulate the timeless effects of faults in the model
\sphinxhyphen{} Adding dynamic states and behaviors to the functions as well as a simulation times and operational phases in a dynamic propagation model to simulate the dynamic effects of faults simulated during different time\sphinxhyphen{}steps
\sphinxhyphen{} Instantiating functions with component architectures to compare the expected resilience and behaviors of each
\sphinxhyphen{} Defining stochastic behavioral and input parameters to simulate and analyze system resilience throughout the operational envelope

\item {} 
\sphinxAtStartPar
fmdtools provides convenience methods for quickly visualizing the results of fault simulations with commonly\sphinxhyphen{}used Python libraries to enable one to quickly assess:
\sphinxhyphen{} effects of faults on functions and flows in the model graph at a given time\sphinxhyphen{}step
\sphinxhyphen{} the behavior of system states over time in nominal and faulty scenarios over a range of operational parameters
\sphinxhyphen{} the effect of model input parameters (e.g., ranges, stochastic inputs) on nominal/faulty operations
\sphinxhyphen{} the high\sphinxhyphen{}level results of a set of simulations in an FMEA\sphinxhyphen{}style table of faults, effects, rates, costs, and overall risk
\sphinxhyphen{} simulation responses over a range or distribution of model and scenario parameters

\end{itemize}

\sphinxAtStartPar
An overview of an earlier version of fmdtools (0.6.2) is provided in the paper:

\sphinxAtStartPar
\sphinxhref{https://doi.org/10.36001/ijphm.2021.v12i3.2954}{Hulse, D., Walsh, H., Dong, A., Hoyle, C., Tumer, I., Kulkarni, C., \& Goebel, K. (2021). fmdtools: A Fault Propagation Toolkit for Resilience Assessment in Early Design. International Journal of Prognostics and Health Management, 12(3).}

\sphinxAtStartPar
fmdtools is a research code and is under active development. As a result, Some use\sphinxhyphen{}cases may not work as desired and may change. If you find a bug or would like to contribute, contact the contributors.


\section{Getting Started}
\label{\detokenize{README:getting-started}}
\sphinxAtStartPar
The latest public version of fmdtools can be downloaded from the \sphinxhref{https://github.com/NASA-SW-VnV/fmdtools/}{fmdtools github repository} e.g., using:

\begin{sphinxVerbatim}[commandchars=\\\{\}]
\PYG{n}{git} \PYG{n}{clone} \PYG{n}{https}\PYG{p}{:}\PYG{o}{/}\PYG{o}{/}\PYG{n}{github}\PYG{o}{.}\PYG{n}{com}\PYG{o}{/}\PYG{n}{NASA}\PYG{o}{\PYGZhy{}}\PYG{n}{SW}\PYG{o}{\PYGZhy{}}\PYG{n}{VnV}\PYG{o}{/}\PYG{n}{fmdtools}\PYG{o}{.}\PYG{n}{git}
\end{sphinxVerbatim}

\sphinxAtStartPar
A version of the fmdtools toolkit can also be installed directly from the \sphinxhref{https://pypi.org/project/fmdtools/}{PyPI package repository} using \sphinxtitleref{pip install fmdtools}.


\subsection{Prerequisites}
\label{\detokenize{README:prerequisites}}
\sphinxAtStartPar
fmdtools requires Python 3 and depends directly on these packages (see requirements.txt):

\begin{sphinxVerbatim}[commandchars=\\\{\}]
\PYG{n}{scipy}
\PYG{n}{tqdm}
\PYG{n}{multiprocessing}
\PYG{n}{networkx}
\PYG{n}{numpy}
\PYG{n}{matplotlib}
\PYG{n}{pandas}
\PYG{n}{ordered}\PYG{o}{\PYGZhy{}}\PYG{n+nb}{set}
\PYG{n}{dill}
\PYG{n}{pickle}
\end{sphinxVerbatim}

\sphinxAtStartPar
These packages are optional but recommended to enable specific fmdtools use\sphinxhyphen{}cases and work with examples in the repository:

\begin{sphinxVerbatim}[commandchars=\\\{\}]
\PYG{n}{jupyter} \PYG{n}{notebook}                     \PYG{c+c1}{\PYGZsh{}(for repository notebooks)}
\PYG{n}{graphviz}                             \PYG{c+c1}{\PYGZsh{}(to plot using graphviz options)}
\PYG{n}{netgraph}                             \PYG{c+c1}{\PYGZsh{}(to adjust node positions in graphs)}
\PYG{n}{pyvis}                                \PYG{c+c1}{\PYGZsh{}(for interactive html views of model graphs)}
\PYG{n}{quadpy}                               \PYG{c+c1}{\PYGZsh{}(for quadrature sampling)}
\PYG{n}{ffmpeg}                               \PYG{c+c1}{\PYGZsh{}(for animations)}
\PYG{n}{shapely}                              \PYG{c+c1}{\PYGZsh{}(for multirotor model)}
\PYG{n}{pycallgraph2}                         \PYG{c+c1}{\PYGZsh{}(for model profiling)}
\end{sphinxVerbatim}

\sphinxAtStartPar
These must be installed (e.g. using \sphinxcode{\sphinxupquote{pip install packagename}} or \sphinxcode{\sphinxupquote{conda install packagename}}) them before running any of the codes in the repository.


\section{Contributions}
\label{\detokenize{README:contributions}}
\sphinxAtStartPar
fmdtools is developed primarily by the \sphinxhref{https://ti.arc.nasa.gov/tech/rse/research/rad/}{Resilience Analysis and Design} research project. External contributions are welcome under a Contributor License Agreement.


\subsection{Contributors}
\label{\detokenize{README:contributors}}
\sphinxAtStartPar
\sphinxhref{https://github.com/hulsed}{Daniel Hulse} : Primary Author and point\sphinxhyphen{}of\sphinxhyphen{}contact

\sphinxAtStartPar
\sphinxhref{https://github.com/walshh}{Hannah Walsh} : Network Analysis Codes

\sphinxAtStartPar
\sphinxhref{https://ti.arc.nasa.gov/profile/andrade/}{Sequoia Andrade} : Graph visualization graphviz options, Code review


\section{Examples}
\label{\detokenize{Examples:examples}}\label{\detokenize{Examples::doc}}
\sphinxAtStartPar
This repository provides several resources to get familiar with fmdtools.

\sphinxAtStartPar
To get started with fmdtools, it is reccomended to start with the \sphinxstylestrong{Intro to fmdtools} workshop, which provides a basic high\sphinxhyphen{}level overview of modelling and simulation in fmdtools, with a corresponding model and example notebook to fill out. To start the workshop, first download the workshop slides (\sphinxcode{\sphinxupquote{Intro to fmdtools.pptx}}), tutorial model (\sphinxcode{\sphinxupquote{ex\_pump.py}}), and Unfilled Notebook \sphinxcode{\sphinxupquote{ex\_pump.py}}.  If you cloned fmdtools, you can just navigate to these files in the repository–they are in a directory called \sphinxcode{\sphinxupquote{example\_pump}}. Then, follow along with the slides and the (see: {\hyperref[\detokenize{example_pump/Tutorial_complete::doc}]{\sphinxcrossref{\DUrole{doc}{filled\sphinxhyphen{}in notebook}}}}).

\sphinxAtStartPar
After completing the workshop, it can be helpful to run through the following notebooks to better understand fmdtools modelling, simulation, and analysis basics as well as more advanced features and use\sphinxhyphen{}cases:
\begin{itemize}
\item {} 
\sphinxAtStartPar
{\hyperref[\detokenize{docs/Model_Structure_Visualization_Tutorial::doc}]{\sphinxcrossref{\DUrole{doc}{Defining and Visualizing fmdtools Model Structures}}}} is helpful for understanding how a given model simulates over time. It covers the methods:
\begin{itemize}
\item {} 
\sphinxAtStartPar
{\hyperref[\detokenize{docs/fmdtools.resultdisp:fmdtools.resultdisp.graph.show}]{\sphinxcrossref{\sphinxcode{\sphinxupquote{fmdtools.resultdisp.graph.show()}}}}}

\item {} 
\sphinxAtStartPar
{\hyperref[\detokenize{docs/fmdtools.resultdisp:fmdtools.resultdisp.graph.exec_order}]{\sphinxcrossref{\sphinxcode{\sphinxupquote{fmdtools.resultdisp.graph.exec\_order()}}}}}

\item {} 
\sphinxAtStartPar
\sphinxcode{\sphinxupquote{fmdtools.resultdisp.plot.graph\_order()}}

\end{itemize}

\item {} 
\sphinxAtStartPar
{\hyperref[\detokenize{example_multirotor/Demonstration::doc}]{\sphinxcrossref{\DUrole{doc}{fmdtools Paper Demonstration}}}}  is helpful for understanding network/static/dynamic/hierarchical model types, covering:
\begin{itemize}
\item {} 
\sphinxAtStartPar
The {\hyperref[\detokenize{docs/fmdtools.faultsim:module-fmdtools.faultsim.networks}]{\sphinxcrossref{\sphinxcode{\sphinxupquote{fmdtools.faultsim.networks}}}}} module.

\item {} 
\sphinxAtStartPar
Basic simulation of dynamic and static models using methods in {\hyperref[\detokenize{docs/fmdtools.faultsim:module-fmdtools.faultsim.propagate}]{\sphinxcrossref{\sphinxcode{\sphinxupquote{fmdtools.faultsim.propagate}}}}} and usage of class {\hyperref[\detokenize{docs/fmdtools:fmdtools.modeldef.SampleApproach}]{\sphinxcrossref{\sphinxcode{\sphinxupquote{fmdtools.modeldef.SampleApproach}}}}} for fault sampling

\item {} 
\sphinxAtStartPar
Analysis using {\hyperref[\detokenize{docs/fmdtools.resultdisp:fmdtools.resultdisp.process.hists}]{\sphinxcrossref{\sphinxcode{\sphinxupquote{fmdtools.resultdisp.process.hists()}}}}}, {\hyperref[\detokenize{docs/fmdtools.resultdisp:fmdtools.resultdisp.plot.mdlhistvals}]{\sphinxcrossref{\sphinxcode{\sphinxupquote{fmdtools.resultdisp.plot.mdlhistvals()}}}}} {\hyperref[\detokenize{docs/fmdtools.resultdisp:fmdtools.resultdisp.tabulate.fullfmea}]{\sphinxcrossref{\sphinxcode{\sphinxupquote{fmdtools.resultdisp.tabulate.fullfmea()}}}}}, and {\hyperref[\detokenize{docs/fmdtools.resultdisp:fmdtools.resultdisp.tabulate.simplefmea}]{\sphinxcrossref{\sphinxcode{\sphinxupquote{fmdtools.resultdisp.tabulate.simplefmea()}}}}}

\end{itemize}

\item {} 
\sphinxAtStartPar
{\hyperref[\detokenize{example_pump/Pump_Example_Notebook::doc}]{\sphinxcrossref{\DUrole{doc}{Pump Example Notebook}}}} is helpful for understanding the breadth of fmdtools plotting, tabulation, and visualization capabilities. It covers:
\begin{itemize}
\item {} 
\sphinxAtStartPar
A variety of graphing usecases in {\hyperref[\detokenize{docs/fmdtools.resultdisp:module-fmdtools.resultdisp.graph}]{\sphinxcrossref{\sphinxcode{\sphinxupquote{fmdtools.resultdisp.graph}}}}} functions which enable viewing different graph types, simulation results at individual times, and overall model statistics/results.

\item {} 
\sphinxAtStartPar
{\hyperref[\detokenize{docs/fmdtools.resultdisp:module-fmdtools.resultdisp.tabulate}]{\sphinxcrossref{\sphinxcode{\sphinxupquote{fmdtools.resultdisp.tabulate}}}}} functions for viewing simulation results over time and summarizing run information

\item {} 
\sphinxAtStartPar
Save/load using the \sphinxtitleref{dill} package

\end{itemize}

\item {} 
\sphinxAtStartPar
{\hyperref[\detokenize{docs/Approach_Use-Cases::doc}]{\sphinxcrossref{\DUrole{doc}{Defining Fault Sampling Approaches in fmdtools}}}} covers how to set up a fault sampling approach and use it to simulate a large number of hazardous scenarios in a model. This includes:
\begin{itemize}
\item {} 
\sphinxAtStartPar
Adding fault and operational modes to Model functions using the method {\hyperref[\detokenize{docs/fmdtools:fmdtools.modeldef.Block.assoc_modes}]{\sphinxcrossref{\sphinxcode{\sphinxupquote{fmdtools.modeldef.Block.assoc\_modes()}}}}} and explanation of the \sphinxtitleref{key\_phases\_by} and \sphinxtitleref{exclusive} options.

\item {} 
\sphinxAtStartPar
Using {\hyperref[\detokenize{docs/fmdtools.resultdisp:fmdtools.resultdisp.process.modephases}]{\sphinxcrossref{\sphinxcode{\sphinxupquote{fmdtools.resultdisp.process.modephases()}}}}} to setting a {\hyperref[\detokenize{docs/fmdtools:fmdtools.modeldef.SampleApproach}]{\sphinxcrossref{\sphinxcode{\sphinxupquote{fmdtools.modeldef.SampleApproach}}}}} up which samples individual faults based on the phases of the model and/or function defined by their operational modes.

\item {} 
\sphinxAtStartPar
Using {\hyperref[\detokenize{docs/fmdtools.resultdisp:fmdtools.resultdisp.plot.phases}]{\sphinxcrossref{\sphinxcode{\sphinxupquote{fmdtools.resultdisp.plot.phases()}}}}} to visualize the phases and modes of a model over time and {\hyperref[\detokenize{docs/fmdtools.resultdisp:fmdtools.resultdisp.plot.samplecosts}]{\sphinxcrossref{\sphinxcode{\sphinxupquote{fmdtools.resultdisp.plot.samplecosts()}}}}} to visualize the consequences of each fault scenario in the approach within each phase.

\item {} 
\sphinxAtStartPar
Using the \sphinxtitleref{defaultsamp} option and {\hyperref[\detokenize{docs/fmdtools:fmdtools.modeldef.SampleApproach.prune_scenarios}]{\sphinxcrossref{\sphinxcode{\sphinxupquote{fmdtools.modeldef.SampleApproach.prune\_scenarios()}}}}} in {\hyperref[\detokenize{docs/fmdtools:fmdtools.modeldef.SampleApproach}]{\sphinxcrossref{\sphinxcode{\sphinxupquote{fmdtools.modeldef.SampleApproach}}}}} to control how many time\sphinxhyphen{}steps in each phase are sampled (and when).

\end{itemize}

\item {} 
\sphinxAtStartPar
{\hyperref[\detokenize{example_pump/Parallelism_Tutorial::doc}]{\sphinxcrossref{\DUrole{doc}{Using Parallel Computing in fmdtools}}}} covers how to reduce computational costs for computationally\sphinxhyphen{}expensive simulation use\sphinxhyphen{}cases (e.g., sampling large numbers of fault scenarios or running complex models with large numbers of timesteps). It covers:
\begin{itemize}
\item {} 
\sphinxAtStartPar
Using process pools as arguments to {\color{red}\bfseries{}:module:`fmdtools.faultsim.propagate`} methods to speed up simulation. A comparison of process pools from different external python packages are provided.

\item {} 
\sphinxAtStartPar
Different options for history tracking and staged execution which can reduce computational costs when desired.

\item {} 
\sphinxAtStartPar
Profiling models with \sphinxcode{\sphinxupquote{cProfile}} and \sphinxcode{\sphinxupquote{pycallgraph2}} to discover what parts are most computationally\sphinxhyphen{}expensive.

\end{itemize}

\item {} 
\sphinxAtStartPar
{\hyperref[\detokenize{docs/Nominal_Approach_Use-Cases::doc}]{\sphinxcrossref{\DUrole{doc}{Defining Nominal Approaches in fmdtools}}}} , which covers simulating the model at different parameters in nominal/faulty scenarios. This includes:
\begin{itemize}
\item {} 
\sphinxAtStartPar
Setting up a nominal parameter sampling approach using {\hyperref[\detokenize{docs/fmdtools:fmdtools.modeldef.NominalApproach}]{\sphinxcrossref{\sphinxcode{\sphinxupquote{fmdtools.modeldef.NominalApproach}}}}} and simulating it with {\hyperref[\detokenize{docs/fmdtools.faultsim:fmdtools.faultsim.propagate.nominal_approach}]{\sphinxcrossref{\sphinxcode{\sphinxupquote{fmdtools.faultsim.propagate.nominal\_approach()}}}}} and {\hyperref[\detokenize{docs/fmdtools.faultsim:fmdtools.faultsim.propagate.nested_approach}]{\sphinxcrossref{\sphinxcode{\sphinxupquote{fmdtools.faultsim.propagate.nested\_approach()}}}}} methods for nominal and faulty simulations.

\item {} 
\sphinxAtStartPar
Using analysis functions like \sphinxcode{\sphinxupquote{fmdtools.resultdisp.tabulate.nominal\_vals\_1d()}} and \sphinxcode{\sphinxupquote{fmdtools.resuldisp.plot.nominal\_factor\_comparison()}} to visualize quantities of interest for the simulation over a range of nominal parameters.

\item {} 
\sphinxAtStartPar
Using analysis functions like \sphinxcode{\sphinxupquote{fmdtools.resuldisp.tabulate.resilience\_factor\_comparison()}} and \sphinxcode{\sphinxupquote{fmdtools.resuldisp.plot.resilience\_factor\_comparison()}} to visualize resilience metrics of the model to a set of fault modes over a range of nominal parameters.

\end{itemize}

\item {} 
\sphinxAtStartPar
{\hyperref[\detokenize{example_pump/Stochastic_Modelling::doc}]{\sphinxcrossref{\DUrole{doc}{Stochastic Modelling in fmdtools}}}} , which covers defining and simulating stochastic models–models with random internal behaviors. This includees:
\begin{itemize}
\item {} 
\sphinxAtStartPar
Setting up random states in functions using {\hyperref[\detokenize{docs/fmdtools:fmdtools.modeldef.Block.assoc_rand_state}]{\sphinxcrossref{\sphinxcode{\sphinxupquote{fmdtools.modeldef.Block.assoc\_rand\_state()}}}}}, {\hyperref[\detokenize{docs/fmdtools:fmdtools.modeldef.Block.set_rand}]{\sphinxcrossref{\sphinxcode{\sphinxupquote{fmdtools.modeldef.Block.set\_rand()}}}}}, and {\hyperref[\detokenize{docs/fmdtools:fmdtools.modeldef.Block.to_default}]{\sphinxcrossref{\sphinxcode{\sphinxupquote{fmdtools.modeldef.Block.to\_default()}}}}}.

\item {} 
\sphinxAtStartPar
Simulating stochastic models using the \sphinxtitleref{run\_stochastic} parameter in {\hyperref[\detokenize{docs/fmdtools.faultsim:module-fmdtools.faultsim.propagate}]{\sphinxcrossref{\sphinxcode{\sphinxupquote{fmdtools.faultsim.propagate}}}}} functions, as well as setting up a {\hyperref[\detokenize{docs/fmdtools:fmdtools.modeldef.NominalApproach}]{\sphinxcrossref{\sphinxcode{\sphinxupquote{fmdtools.modeldef.NominalApproach}}}}} with multiple seeds to run a set of random simulations.

\item {} 
\sphinxAtStartPar
Using {\hyperref[\detokenize{docs/fmdtools.resultdisp:fmdtools.resultdisp.plot.mdlhists}]{\sphinxcrossref{\sphinxcode{\sphinxupquote{fmdtools.resultdisp.plot.mdlhists()}}}}} to visualize the results of multiple stochastic simulations over time, and analyze quantities of interest using {\hyperref[\detokenize{docs/fmdtools.resultdisp:fmdtools.resultdisp.tabulate.nested_stats}]{\sphinxcrossref{\sphinxcode{\sphinxupquote{fmdtools.resultdisp.tabulate.nested\_stats()}}}}}, {\hyperref[\detokenize{docs/fmdtools.resultdisp:fmdtools.resultdisp.tabulate.resilience_factor_comparison}]{\sphinxcrossref{\sphinxcode{\sphinxupquote{fmdtools.resultdisp.tabulate.resilience\_factor\_comparison()}}}}}

\end{itemize}

\end{itemize}

\sphinxAtStartPar
There are also two other example models which demonstrate specialized use\sphinxhyphen{}cases:
\begin{itemize}
\item {} 
\sphinxAtStartPar
{\hyperref[\detokenize{example_tank/Tank_Analysis::doc}]{\sphinxcrossref{\DUrole{doc}{Hold\sphinxhyphen{}up Tank Model}}}} uses the {\hyperref[\detokenize{docs/fmdtools:fmdtools.modeldef.SampleApproach}]{\sphinxcrossref{\sphinxcode{\sphinxupquote{fmdtools.modeldef.SampleApproach}}}}} class to model human interactions with the modelled system.

\item {} 
\sphinxAtStartPar
{\hyperref[\detokenize{example_eps/EPS_Example_Notebook::doc}]{\sphinxcrossref{\DUrole{doc}{EPS Example Notebook}}}} shows a simple static modelling use\sphinxhyphen{}case.

\end{itemize}


\subsection{fmdtools Tutorial}
\label{\detokenize{example_pump/Tutorial_complete:fmdtools-Tutorial}}\label{\detokenize{example_pump/Tutorial_complete::doc}}
\sphinxAtStartPar
This tutorial notebook will show some of the basic commands needed to perform resilience analysis in fmdtools.

\sphinxAtStartPar
For some context, it may be helpful to look through the accompanying presentation. This notebook uses the model defined in \sphinxcode{\sphinxupquote{ex\_pump.py}}. In this notebook, we will:
\begin{itemize}
\item {} 
\sphinxAtStartPar
Load an environment and model

\item {} 
\sphinxAtStartPar
Simulate the system in nominal and faulty scenarios

\item {} 
\sphinxAtStartPar
Visualize and quantify the results

\end{itemize}


\subsubsection{1.) Loading the environment and model}
\label{\detokenize{example_pump/Tutorial_complete:1.)-Loading-the-environment-and-model}}
\sphinxAtStartPar
To load the \sphinxcode{\sphinxupquote{fmdtools}} environment, we have to import it.

\sphinxAtStartPar
Since we’re in a subfolder of the repository, we need to add the fmdtools folder to the system path. (This would be unnecessary if we installed using \sphinxcode{\sphinxupquote{pip}})

\begin{sphinxuseclass}{nbinput}
\begin{sphinxuseclass}{nblast}
{
\sphinxsetup{VerbatimColor={named}{nbsphinx-code-bg}}
\sphinxsetup{VerbatimBorderColor={named}{nbsphinx-code-border}}
\begin{sphinxVerbatim}[commandchars=\\\{\}]
\llap{\color{nbsphinxin}[1]:\,\hspace{\fboxrule}\hspace{\fboxsep}}\PYG{c+c1}{\PYGZsh{}First, import the fault propogation library as well as the model}
\PYG{c+c1}{\PYGZsh{}since the package is in a parallel location to examples...}
\PYG{k+kn}{import} \PYG{n+nn}{sys}\PYG{o}{,} \PYG{n+nn}{os}
\PYG{n}{sys}\PYG{o}{.}\PYG{n}{path}\PYG{o}{.}\PYG{n}{insert}\PYG{p}{(}\PYG{l+m+mi}{1}\PYG{p}{,}\PYG{n}{os}\PYG{o}{.}\PYG{n}{path}\PYG{o}{.}\PYG{n}{join}\PYG{p}{(}\PYG{l+s+s2}{\PYGZdq{}}\PYG{l+s+s2}{..}\PYG{l+s+s2}{\PYGZdq{}}\PYG{p}{)}\PYG{p}{)}
\end{sphinxVerbatim}
}

\end{sphinxuseclass}
\end{sphinxuseclass}
\sphinxAtStartPar
There are a number of different syntaxes for importing modules. Because of the long names of the module trees, it is often helpful to load the modules individually and abbreviate (e.g. \sphinxcode{\sphinxupquote{import fmdtools.faultsim.propagate as propagate}}). Below, import the propagate \sphinxcode{\sphinxupquote{fmdtools.faultsim.propagate}} and \sphinxcode{\sphinxupquote{fmdtools.resultdisp}} modules, as well as the \sphinxcode{\sphinxupquote{SampleApproach}} from \sphinxcode{\sphinxupquote{fmdtools.modeldef}}

\begin{sphinxuseclass}{nbinput}
\begin{sphinxuseclass}{nblast}
{
\sphinxsetup{VerbatimColor={named}{nbsphinx-code-bg}}
\sphinxsetup{VerbatimBorderColor={named}{nbsphinx-code-border}}
\begin{sphinxVerbatim}[commandchars=\\\{\}]
\llap{\color{nbsphinxin}[2]:\,\hspace{\fboxrule}\hspace{\fboxsep}}\PYG{k+kn}{import} \PYG{n+nn}{fmdtools}\PYG{n+nn}{.}\PYG{n+nn}{faultsim}\PYG{n+nn}{.}\PYG{n+nn}{propagate} \PYG{k}{as} \PYG{n+nn}{propagate}
\PYG{k+kn}{import} \PYG{n+nn}{fmdtools}\PYG{n+nn}{.}\PYG{n+nn}{resultdisp} \PYG{k}{as} \PYG{n+nn}{rd}
\PYG{k+kn}{from} \PYG{n+nn}{fmdtools}\PYG{n+nn}{.}\PYG{n+nn}{modeldef} \PYG{k+kn}{import} \PYG{n}{SampleApproach}
\end{sphinxVerbatim}
}

\end{sphinxuseclass}
\end{sphinxuseclass}
\sphinxAtStartPar
Now, import the Pump class defined in the ex\_pump module.

\begin{sphinxuseclass}{nbinput}
\begin{sphinxuseclass}{nblast}
{
\sphinxsetup{VerbatimColor={named}{nbsphinx-code-bg}}
\sphinxsetup{VerbatimBorderColor={named}{nbsphinx-code-border}}
\begin{sphinxVerbatim}[commandchars=\\\{\}]
\llap{\color{nbsphinxin}[3]:\,\hspace{\fboxrule}\hspace{\fboxsep}}\PYG{k+kn}{from} \PYG{n+nn}{ex\PYGZus{}pump} \PYG{k+kn}{import} \PYG{n}{Pump}
\end{sphinxVerbatim}
}

\end{sphinxuseclass}
\end{sphinxuseclass}
\sphinxAtStartPar
We can then use that to instantiate a model object. See:

\begin{sphinxuseclass}{nbinput}
\begin{sphinxuseclass}{nblast}
{
\sphinxsetup{VerbatimColor={named}{nbsphinx-code-bg}}
\sphinxsetup{VerbatimBorderColor={named}{nbsphinx-code-border}}
\begin{sphinxVerbatim}[commandchars=\\\{\}]
\llap{\color{nbsphinxin}[4]:\,\hspace{\fboxrule}\hspace{\fboxsep}}\PYG{n}{mdl} \PYG{o}{=} \PYG{n}{Pump}\PYG{p}{(}\PYG{p}{)}
\end{sphinxVerbatim}
}

\end{sphinxuseclass}
\end{sphinxuseclass}
\sphinxAtStartPar
To get started, it can be helpful to view some of the aspects of the model. Try \sphinxcode{\sphinxupquote{dir(mdl)}}, \sphinxcode{\sphinxupquote{mdl.fxns}}, \sphinxcode{\sphinxupquote{mdl.flows}}, \sphinxcode{\sphinxupquote{mdl.graph}}, etc.

\begin{sphinxuseclass}{nbinput}
{
\sphinxsetup{VerbatimColor={named}{nbsphinx-code-bg}}
\sphinxsetup{VerbatimBorderColor={named}{nbsphinx-code-border}}
\begin{sphinxVerbatim}[commandchars=\\\{\}]
\llap{\color{nbsphinxin}[5]:\,\hspace{\fboxrule}\hspace{\fboxsep}}\PYG{n}{mdl}\PYG{o}{.}\PYG{n}{fxns}
\end{sphinxVerbatim}
}

\end{sphinxuseclass}
\begin{sphinxuseclass}{nboutput}
\begin{sphinxuseclass}{nblast}
{

\kern-\sphinxverbatimsmallskipamount\kern-\baselineskip
\kern+\FrameHeightAdjust\kern-\fboxrule
\vspace{\nbsphinxcodecellspacing}

\sphinxsetup{VerbatimColor={named}{white}}
\sphinxsetup{VerbatimBorderColor={named}{nbsphinx-code-border}}
\begin{sphinxuseclass}{output_area}
\begin{sphinxuseclass}{}


\begin{sphinxVerbatim}[commandchars=\\\{\}]
\llap{\color{nbsphinxout}[5]:\,\hspace{\fboxrule}\hspace{\fboxsep}}\{'ImportEE': ImportEE ImportEE function: (\{\}, \{'nom'\}),
 'ImportWater': ImportWater ImportWater function: (\{\}, \{'nom'\}),
 'ImportSignal': ImportSignal ImportSig function: (\{\}, \{'nom'\}),
 'MoveWater': MoveWater MoveWat function: (\{'eff': 1.0\}, \{'nom'\}),
 'ExportWater': ExportWater ExportWater function: (\{\}, \{'nom'\})\}
\end{sphinxVerbatim}



\end{sphinxuseclass}
\end{sphinxuseclass}
}

\end{sphinxuseclass}
\end{sphinxuseclass}
\sphinxAtStartPar
We can also view the run order to see how the model will be simulated.

\begin{sphinxuseclass}{nbinput}
{
\sphinxsetup{VerbatimColor={named}{nbsphinx-code-bg}}
\sphinxsetup{VerbatimBorderColor={named}{nbsphinx-code-border}}
\begin{sphinxVerbatim}[commandchars=\\\{\}]
\llap{\color{nbsphinxin}[6]:\,\hspace{\fboxrule}\hspace{\fboxsep}}\PYG{n}{rd}\PYG{o}{.}\PYG{n}{graph}\PYG{o}{.}\PYG{n}{exec\PYGZus{}order}\PYG{p}{(}\PYG{n}{mdl}\PYG{p}{)}
\end{sphinxVerbatim}
}

\end{sphinxuseclass}
\begin{sphinxuseclass}{nboutput}
{

\kern-\sphinxverbatimsmallskipamount\kern-\baselineskip
\kern+\FrameHeightAdjust\kern-\fboxrule
\vspace{\nbsphinxcodecellspacing}

\sphinxsetup{VerbatimColor={named}{white}}
\sphinxsetup{VerbatimBorderColor={named}{nbsphinx-code-border}}
\begin{sphinxuseclass}{output_area}
\begin{sphinxuseclass}{}


\begin{sphinxVerbatim}[commandchars=\\\{\}]
\llap{\color{nbsphinxout}[6]:\,\hspace{\fboxrule}\hspace{\fboxsep}}(<Figure size 432x288 with 1 Axes>,
 <matplotlib.axes.\_subplots.AxesSubplot at 0x18df0f66a48>)
\end{sphinxVerbatim}



\end{sphinxuseclass}
\end{sphinxuseclass}
}

\end{sphinxuseclass}
\begin{sphinxuseclass}{nboutput}
\begin{sphinxuseclass}{nblast}
\hrule height -\fboxrule\relax
\vspace{\nbsphinxcodecellspacing}

\makeatletter\setbox\nbsphinxpromptbox\box\voidb@x\makeatother

\begin{nbsphinxfancyoutput}

\begin{sphinxuseclass}{output_area}
\begin{sphinxuseclass}{}
\noindent\sphinxincludegraphics[width=349\sphinxpxdimen,height=281\sphinxpxdimen]{{example_pump_Tutorial_complete_12_1}.png}

\end{sphinxuseclass}
\end{sphinxuseclass}
\end{nbsphinxfancyoutput}

\end{sphinxuseclass}
\end{sphinxuseclass}
\sphinxAtStartPar
As shown, because all of the methods were defined as generic behaviors, they are each run in the static propagation step. No order is shown in the static step because the static propagation step iterates between model functions until the values have converged. Nevertheless, one can view the initial static order using:

\begin{sphinxuseclass}{nbinput}
{
\sphinxsetup{VerbatimColor={named}{nbsphinx-code-bg}}
\sphinxsetup{VerbatimBorderColor={named}{nbsphinx-code-border}}
\begin{sphinxVerbatim}[commandchars=\\\{\}]
\llap{\color{nbsphinxin}[7]:\,\hspace{\fboxrule}\hspace{\fboxsep}}\PYG{n}{mdl}\PYG{o}{.}\PYG{n}{staticfxns}
\end{sphinxVerbatim}
}

\end{sphinxuseclass}
\begin{sphinxuseclass}{nboutput}
\begin{sphinxuseclass}{nblast}
{

\kern-\sphinxverbatimsmallskipamount\kern-\baselineskip
\kern+\FrameHeightAdjust\kern-\fboxrule
\vspace{\nbsphinxcodecellspacing}

\sphinxsetup{VerbatimColor={named}{white}}
\sphinxsetup{VerbatimBorderColor={named}{nbsphinx-code-border}}
\begin{sphinxuseclass}{output_area}
\begin{sphinxuseclass}{}


\begin{sphinxVerbatim}[commandchars=\\\{\}]
\llap{\color{nbsphinxout}[7]:\,\hspace{\fboxrule}\hspace{\fboxsep}}OrderedSet(['ImportEE', 'ImportWater', 'ImportSignal', 'MoveWater', 'ExportWater'])
\end{sphinxVerbatim}



\end{sphinxuseclass}
\end{sphinxuseclass}
}

\end{sphinxuseclass}
\end{sphinxuseclass}
\sphinxAtStartPar
and the dynamic step order (if there was one):

\begin{sphinxuseclass}{nbinput}
{
\sphinxsetup{VerbatimColor={named}{nbsphinx-code-bg}}
\sphinxsetup{VerbatimBorderColor={named}{nbsphinx-code-border}}
\begin{sphinxVerbatim}[commandchars=\\\{\}]
\llap{\color{nbsphinxin}[8]:\,\hspace{\fboxrule}\hspace{\fboxsep}}\PYG{n}{mdl}\PYG{o}{.}\PYG{n}{dynamicfxns}
\end{sphinxVerbatim}
}

\end{sphinxuseclass}
\begin{sphinxuseclass}{nboutput}
\begin{sphinxuseclass}{nblast}
{

\kern-\sphinxverbatimsmallskipamount\kern-\baselineskip
\kern+\FrameHeightAdjust\kern-\fboxrule
\vspace{\nbsphinxcodecellspacing}

\sphinxsetup{VerbatimColor={named}{white}}
\sphinxsetup{VerbatimBorderColor={named}{nbsphinx-code-border}}
\begin{sphinxuseclass}{output_area}
\begin{sphinxuseclass}{}


\begin{sphinxVerbatim}[commandchars=\\\{\}]
\llap{\color{nbsphinxout}[8]:\,\hspace{\fboxrule}\hspace{\fboxsep}}OrderedSet()
\end{sphinxVerbatim}



\end{sphinxuseclass}
\end{sphinxuseclass}
}

\end{sphinxuseclass}
\end{sphinxuseclass}
\sphinxAtStartPar
We can also instantiate this model with different model parameters. By default, this model gets \sphinxcode{\sphinxupquote{params=\{'cost':\{'repair', 'water'\}, 'delay':10, 'units':'hrs'\}}} but we can pass any set of costs that is a subset of \sphinxcode{\sphinxupquote{\{'repair', 'water', 'water\_exp', 'ee'\}}} and any int \sphinxcode{\sphinxupquote{delay}} parameter.

\begin{sphinxuseclass}{nbinput}
\begin{sphinxuseclass}{nblast}
{
\sphinxsetup{VerbatimColor={named}{nbsphinx-code-bg}}
\sphinxsetup{VerbatimBorderColor={named}{nbsphinx-code-border}}
\begin{sphinxVerbatim}[commandchars=\\\{\}]
\llap{\color{nbsphinxin}[9]:\,\hspace{\fboxrule}\hspace{\fboxsep}}\PYG{n}{mdl2} \PYG{o}{=} \PYG{n}{Pump}\PYG{p}{(}\PYG{n}{params}\PYG{o}{=}\PYG{p}{\PYGZob{}}\PYG{l+s+s1}{\PYGZsq{}}\PYG{l+s+s1}{cost}\PYG{l+s+s1}{\PYGZsq{}}\PYG{p}{:}\PYG{p}{\PYGZob{}}\PYG{l+s+s1}{\PYGZsq{}}\PYG{l+s+s1}{repair}\PYG{l+s+s1}{\PYGZsq{}}\PYG{p}{,} \PYG{l+s+s1}{\PYGZsq{}}\PYG{l+s+s1}{water}\PYG{l+s+s1}{\PYGZsq{}}\PYG{p}{,} \PYG{l+s+s1}{\PYGZsq{}}\PYG{l+s+s1}{ee}\PYG{l+s+s1}{\PYGZsq{}}\PYG{p}{\PYGZcb{}}\PYG{p}{,} \PYG{l+s+s1}{\PYGZsq{}}\PYG{l+s+s1}{delay}\PYG{l+s+s1}{\PYGZsq{}}\PYG{p}{:}\PYG{l+m+mi}{20}\PYG{p}{,} \PYG{l+s+s1}{\PYGZsq{}}\PYG{l+s+s1}{units}\PYG{l+s+s1}{\PYGZsq{}}\PYG{p}{:}\PYG{l+s+s1}{\PYGZsq{}}\PYG{l+s+s1}{hrs}\PYG{l+s+s1}{\PYGZsq{}}\PYG{p}{\PYGZcb{}}\PYG{p}{)}
\end{sphinxVerbatim}
}

\end{sphinxuseclass}
\end{sphinxuseclass}

\subsubsection{2.) and 3.) Simulate and visualize the results!}
\label{\detokenize{example_pump/Tutorial_complete:2.)-and-3.)-Simulate-and-visualize-the-results!}}
\sphinxAtStartPar
Now, we will use the methods in \sphinxcode{\sphinxupquote{propagate}} and the visualization modules in \sphinxcode{\sphinxupquote{rd}} to simulate the model and visualize the results.


\paragraph{2a.) Simulate nominal}
\label{\detokenize{example_pump/Tutorial_complete:2a.)-Simulate-nominal}}
\sphinxAtStartPar
To simulate the model in the nominal scenario, use the \sphinxcode{\sphinxupquote{propagate.nominal}} method.

\begin{sphinxuseclass}{nbinput}
\begin{sphinxuseclass}{nblast}
{
\sphinxsetup{VerbatimColor={named}{nbsphinx-code-bg}}
\sphinxsetup{VerbatimBorderColor={named}{nbsphinx-code-border}}
\begin{sphinxVerbatim}[commandchars=\\\{\}]
\llap{\color{nbsphinxin}[10]:\,\hspace{\fboxrule}\hspace{\fboxsep}}\PYG{n}{endresults\PYGZus{}nominal}\PYG{p}{,} \PYG{n}{resgraph\PYGZus{}nominal}\PYG{p}{,} \PYG{n}{mdlhist\PYGZus{}nominal}\PYG{o}{=}\PYG{n}{propagate}\PYG{o}{.}\PYG{n}{nominal}\PYG{p}{(}\PYG{n}{mdl}\PYG{p}{,} \PYG{n}{track}\PYG{o}{=}\PYG{l+s+s2}{\PYGZdq{}}\PYG{l+s+s2}{all}\PYG{l+s+s2}{\PYGZdq{}}\PYG{p}{)}
\end{sphinxVerbatim}
}

\end{sphinxuseclass}
\end{sphinxuseclass}
\sphinxAtStartPar
What do the results look like? Explore results structures.

\begin{sphinxuseclass}{nbinput}
{
\sphinxsetup{VerbatimColor={named}{nbsphinx-code-bg}}
\sphinxsetup{VerbatimBorderColor={named}{nbsphinx-code-border}}
\begin{sphinxVerbatim}[commandchars=\\\{\}]
\llap{\color{nbsphinxin}[11]:\,\hspace{\fboxrule}\hspace{\fboxsep}}\PYG{n}{endresults\PYGZus{}nominal}
\end{sphinxVerbatim}
}

\end{sphinxuseclass}
\begin{sphinxuseclass}{nboutput}
\begin{sphinxuseclass}{nblast}
{

\kern-\sphinxverbatimsmallskipamount\kern-\baselineskip
\kern+\FrameHeightAdjust\kern-\fboxrule
\vspace{\nbsphinxcodecellspacing}

\sphinxsetup{VerbatimColor={named}{white}}
\sphinxsetup{VerbatimBorderColor={named}{nbsphinx-code-border}}
\begin{sphinxuseclass}{output_area}
\begin{sphinxuseclass}{}


\begin{sphinxVerbatim}[commandchars=\\\{\}]
\llap{\color{nbsphinxout}[11]:\,\hspace{\fboxrule}\hspace{\fboxsep}}\{'faults': \{\},
 'classification': \{'rate': 1.0, 'cost': 0.0, 'expected cost': 0.0\}\}
\end{sphinxVerbatim}



\end{sphinxuseclass}
\end{sphinxuseclass}
}

\end{sphinxuseclass}
\end{sphinxuseclass}

\paragraph{2b.) Visualize nominal model}
\label{\detokenize{example_pump/Tutorial_complete:2b.)-Visualize-nominal-model}}
\sphinxAtStartPar
First, we can show the model graph using \sphinxcode{\sphinxupquote{rd.graph.show}} to see that it was set up correctly. We can do this both on the model graph itself \sphinxcode{\sphinxupquote{mdl.graph}} and the results of the nominal run \sphinxcode{\sphinxupquote{resgraph\_nominal}} to verify both are fault\sphinxhyphen{}free.

\begin{sphinxuseclass}{nbinput}
{
\sphinxsetup{VerbatimColor={named}{nbsphinx-code-bg}}
\sphinxsetup{VerbatimBorderColor={named}{nbsphinx-code-border}}
\begin{sphinxVerbatim}[commandchars=\\\{\}]
\llap{\color{nbsphinxin}[12]:\,\hspace{\fboxrule}\hspace{\fboxsep}}\PYG{n}{rd}\PYG{o}{.}\PYG{n}{graph}\PYG{o}{.}\PYG{n}{show}\PYG{p}{(}\PYG{n}{mdl}\PYG{p}{)}
\end{sphinxVerbatim}
}

\end{sphinxuseclass}
\begin{sphinxuseclass}{nboutput}
{

\kern-\sphinxverbatimsmallskipamount\kern-\baselineskip
\kern+\FrameHeightAdjust\kern-\fboxrule
\vspace{\nbsphinxcodecellspacing}

\sphinxsetup{VerbatimColor={named}{white}}
\sphinxsetup{VerbatimBorderColor={named}{nbsphinx-code-border}}
\begin{sphinxuseclass}{output_area}
\begin{sphinxuseclass}{}


\begin{sphinxVerbatim}[commandchars=\\\{\}]
\llap{\color{nbsphinxout}[12]:\,\hspace{\fboxrule}\hspace{\fboxsep}}(<Figure size 432x288 with 1 Axes>,
 <matplotlib.axes.\_subplots.AxesSubplot at 0x18df173c048>)
\end{sphinxVerbatim}



\end{sphinxuseclass}
\end{sphinxuseclass}
}

\end{sphinxuseclass}
\begin{sphinxuseclass}{nboutput}
\begin{sphinxuseclass}{nblast}
\hrule height -\fboxrule\relax
\vspace{\nbsphinxcodecellspacing}

\makeatletter\setbox\nbsphinxpromptbox\box\voidb@x\makeatother

\begin{nbsphinxfancyoutput}

\begin{sphinxuseclass}{output_area}
\begin{sphinxuseclass}{}
\noindent\sphinxincludegraphics[width=349\sphinxpxdimen,height=231\sphinxpxdimen]{{example_pump_Tutorial_complete_25_1}.png}

\end{sphinxuseclass}
\end{sphinxuseclass}
\end{nbsphinxfancyoutput}

\end{sphinxuseclass}
\end{sphinxuseclass}
\sphinxAtStartPar
We can also view the flow values of the model using ‘’rd.plot.mdlhistvals’’. It may be helpful to only view flows of interest.

\begin{sphinxuseclass}{nbinput}
{
\sphinxsetup{VerbatimColor={named}{nbsphinx-code-bg}}
\sphinxsetup{VerbatimBorderColor={named}{nbsphinx-code-border}}
\begin{sphinxVerbatim}[commandchars=\\\{\}]
\llap{\color{nbsphinxin}[13]:\,\hspace{\fboxrule}\hspace{\fboxsep}}\PYG{n}{fig} \PYG{o}{=} \PYG{n}{rd}\PYG{o}{.}\PYG{n}{plot}\PYG{o}{.}\PYG{n}{mdlhistvals}\PYG{p}{(}\PYG{n}{mdlhist\PYGZus{}nominal}\PYG{p}{,} \PYG{n}{fxnflowvals}\PYG{o}{=}\PYG{p}{\PYGZob{}}\PYG{l+s+s1}{\PYGZsq{}}\PYG{l+s+s1}{Wat\PYGZus{}1}\PYG{l+s+s1}{\PYGZsq{}}\PYG{p}{:}\PYG{p}{[}\PYG{l+s+s1}{\PYGZsq{}}\PYG{l+s+s1}{flowrate}\PYG{l+s+s1}{\PYGZsq{}}\PYG{p}{,}\PYG{l+s+s1}{\PYGZsq{}}\PYG{l+s+s1}{pressure}\PYG{l+s+s1}{\PYGZsq{}}\PYG{p}{]}\PYG{p}{,} \PYG{l+s+s1}{\PYGZsq{}}\PYG{l+s+s1}{EE\PYGZus{}1}\PYG{l+s+s1}{\PYGZsq{}}\PYG{p}{:}\PYG{p}{[}\PYG{l+s+s1}{\PYGZsq{}}\PYG{l+s+s1}{voltage}\PYG{l+s+s1}{\PYGZsq{}}\PYG{p}{,} \PYG{l+s+s1}{\PYGZsq{}}\PYG{l+s+s1}{current}\PYG{l+s+s1}{\PYGZsq{}}\PYG{p}{]}\PYG{p}{\PYGZcb{}}\PYG{p}{,} \PYG{n}{returnfig}\PYG{o}{=}\PYG{k+kc}{True}\PYG{p}{)}
\end{sphinxVerbatim}
}

\end{sphinxuseclass}
\begin{sphinxuseclass}{nboutput}
\begin{sphinxuseclass}{nblast}
\hrule height -\fboxrule\relax
\vspace{\nbsphinxcodecellspacing}

\makeatletter\setbox\nbsphinxpromptbox\box\voidb@x\makeatother

\begin{nbsphinxfancyoutput}

\begin{sphinxuseclass}{output_area}
\begin{sphinxuseclass}{}
\noindent\sphinxincludegraphics[width=426\sphinxpxdimen,height=250\sphinxpxdimen]{{example_pump_Tutorial_complete_27_0}.png}

\end{sphinxuseclass}
\end{sphinxuseclass}
\end{nbsphinxfancyoutput}

\end{sphinxuseclass}
\end{sphinxuseclass}
\sphinxAtStartPar
Note: for quick access to the syntax and options for these methods, type the \sphinxcode{\sphinxupquote{?method}} or \sphinxcode{\sphinxupquote{help(method)}} in the terminal. For example \sphinxcode{\sphinxupquote{?rd.plot.mdlhistvals}}

\begin{sphinxuseclass}{nbinput}
\begin{sphinxuseclass}{nblast}
{
\sphinxsetup{VerbatimColor={named}{nbsphinx-code-bg}}
\sphinxsetup{VerbatimBorderColor={named}{nbsphinx-code-border}}
\begin{sphinxVerbatim}[commandchars=\\\{\}]
\llap{\color{nbsphinxin}[14]:\,\hspace{\fboxrule}\hspace{\fboxsep}}\PYG{o}{?}rd.plot.mdlhistvals
\end{sphinxVerbatim}
}

\end{sphinxuseclass}
\end{sphinxuseclass}

\paragraph{2b.) Simulate a fault mode}
\label{\detokenize{example_pump/Tutorial_complete:2b.)-Simulate-a-fault-mode}}
\sphinxAtStartPar
To simulate the model in the nominal scenario, use the \sphinxcode{\sphinxupquote{propagate.one\_fault}} method. The set of possible faults is defined in the function definitions in \sphinxcode{\sphinxupquote{ex\_pump.py}}, and we can propagate a fault at any time in the operational interval (0\sphinxhyphen{}55 seconds).

\begin{sphinxuseclass}{nbinput}
\begin{sphinxuseclass}{nblast}
{
\sphinxsetup{VerbatimColor={named}{nbsphinx-code-bg}}
\sphinxsetup{VerbatimBorderColor={named}{nbsphinx-code-border}}
\begin{sphinxVerbatim}[commandchars=\\\{\}]
\llap{\color{nbsphinxin}[15]:\,\hspace{\fboxrule}\hspace{\fboxsep}}\PYG{n}{endresults\PYGZus{}fault}\PYG{p}{,} \PYG{n}{resgraph\PYGZus{}fault}\PYG{p}{,} \PYG{n}{mdlhist\PYGZus{}fault}\PYG{o}{=}\PYG{n}{propagate}\PYG{o}{.}\PYG{n}{one\PYGZus{}fault}\PYG{p}{(}\PYG{n}{mdl}\PYG{p}{,} \PYG{l+s+s1}{\PYGZsq{}}\PYG{l+s+s1}{MoveWater}\PYG{l+s+s1}{\PYGZsq{}}\PYG{p}{,} \PYG{l+s+s1}{\PYGZsq{}}\PYG{l+s+s1}{short}\PYG{l+s+s1}{\PYGZsq{}}\PYG{p}{,} \PYG{n}{time}\PYG{o}{=}\PYG{l+m+mi}{10}\PYG{p}{)}
\end{sphinxVerbatim}
}

\end{sphinxuseclass}
\end{sphinxuseclass}
\sphinxAtStartPar
We can also view the results for from this. In this case \sphinxcode{\sphinxupquote{mdlhist}} gives a history of results for both the nominal and faulty runs.

\begin{sphinxuseclass}{nbinput}
{
\sphinxsetup{VerbatimColor={named}{nbsphinx-code-bg}}
\sphinxsetup{VerbatimBorderColor={named}{nbsphinx-code-border}}
\begin{sphinxVerbatim}[commandchars=\\\{\}]
\llap{\color{nbsphinxin}[16]:\,\hspace{\fboxrule}\hspace{\fboxsep}}\PYG{n}{endresults\PYGZus{}fault}
\end{sphinxVerbatim}
}

\end{sphinxuseclass}
\begin{sphinxuseclass}{nboutput}
\begin{sphinxuseclass}{nblast}
{

\kern-\sphinxverbatimsmallskipamount\kern-\baselineskip
\kern+\FrameHeightAdjust\kern-\fboxrule
\vspace{\nbsphinxcodecellspacing}

\sphinxsetup{VerbatimColor={named}{white}}
\sphinxsetup{VerbatimBorderColor={named}{nbsphinx-code-border}}
\begin{sphinxuseclass}{output_area}
\begin{sphinxuseclass}{}


\begin{sphinxVerbatim}[commandchars=\\\{\}]
\llap{\color{nbsphinxout}[16]:\,\hspace{\fboxrule}\hspace{\fboxsep}}\{'flows': \{'EE\_1': \{'voltage': 0.0\}\},
 'faults': \{'ImportEE': ['no\_v'], 'MoveWater': ['short']\},
 'classification': \{'rate': 0.00055,
  'cost': 29000.000000000007,
  'expected cost': 1595000.0000000005\}\}
\end{sphinxVerbatim}



\end{sphinxuseclass}
\end{sphinxuseclass}
}

\end{sphinxuseclass}
\end{sphinxuseclass}
\begin{sphinxuseclass}{nbinput}
{
\sphinxsetup{VerbatimColor={named}{nbsphinx-code-bg}}
\sphinxsetup{VerbatimBorderColor={named}{nbsphinx-code-border}}
\begin{sphinxVerbatim}[commandchars=\\\{\}]
\llap{\color{nbsphinxin}[17]:\,\hspace{\fboxrule}\hspace{\fboxsep}}\PYG{n+nb}{list}\PYG{p}{(}\PYG{n}{mdlhist\PYGZus{}fault}\PYG{o}{.}\PYG{n}{keys}\PYG{p}{(}\PYG{p}{)}\PYG{p}{)}
\end{sphinxVerbatim}
}

\end{sphinxuseclass}
\begin{sphinxuseclass}{nboutput}
\begin{sphinxuseclass}{nblast}
{

\kern-\sphinxverbatimsmallskipamount\kern-\baselineskip
\kern+\FrameHeightAdjust\kern-\fboxrule
\vspace{\nbsphinxcodecellspacing}

\sphinxsetup{VerbatimColor={named}{white}}
\sphinxsetup{VerbatimBorderColor={named}{nbsphinx-code-border}}
\begin{sphinxuseclass}{output_area}
\begin{sphinxuseclass}{}


\begin{sphinxVerbatim}[commandchars=\\\{\}]
\llap{\color{nbsphinxout}[17]:\,\hspace{\fboxrule}\hspace{\fboxsep}}['nominal', 'faulty']
\end{sphinxVerbatim}



\end{sphinxuseclass}
\end{sphinxuseclass}
}

\end{sphinxuseclass}
\end{sphinxuseclass}

\paragraph{3b.) Visualize fault model states}
\label{\detokenize{example_pump/Tutorial_complete:3b.)-Visualize-fault-model-states}}
\sphinxAtStartPar
\sphinxcode{\sphinxupquote{rd.plot.mdlhistvals}} also works for a mdlhists given from \sphinxcode{\sphinxupquote{propagate.one\_fault}}. We can view these results below. As shown, the function will give the nominal result in a blue dotted line and the faulty result in a red line.

\begin{sphinxuseclass}{nbinput}
{
\sphinxsetup{VerbatimColor={named}{nbsphinx-code-bg}}
\sphinxsetup{VerbatimBorderColor={named}{nbsphinx-code-border}}
\begin{sphinxVerbatim}[commandchars=\\\{\}]
\llap{\color{nbsphinxin}[18]:\,\hspace{\fboxrule}\hspace{\fboxsep}}\PYG{n}{fig} \PYG{o}{=} \PYG{n}{rd}\PYG{o}{.}\PYG{n}{plot}\PYG{o}{.}\PYG{n}{mdlhistvals}\PYG{p}{(}\PYG{n}{mdlhist\PYGZus{}fault}\PYG{p}{,} \PYG{n}{fault}\PYG{o}{=}\PYG{l+s+s1}{\PYGZsq{}}\PYG{l+s+s1}{Motor Short}\PYG{l+s+s1}{\PYGZsq{}}\PYG{p}{,} \PYG{n}{time}\PYG{o}{=}\PYG{l+m+mi}{10}\PYG{p}{,} \PYG{n}{legend}\PYG{o}{=}\PYG{k+kc}{False}\PYG{p}{,} \PYG{n}{fxnflowvals}\PYG{o}{=}\PYG{p}{\PYGZob{}}\PYG{l+s+s1}{\PYGZsq{}}\PYG{l+s+s1}{Wat\PYGZus{}1}\PYG{l+s+s1}{\PYGZsq{}}\PYG{p}{:}\PYG{p}{[}\PYG{l+s+s1}{\PYGZsq{}}\PYG{l+s+s1}{flowrate}\PYG{l+s+s1}{\PYGZsq{}}\PYG{p}{,}\PYG{l+s+s1}{\PYGZsq{}}\PYG{l+s+s1}{pressure}\PYG{l+s+s1}{\PYGZsq{}}\PYG{p}{]}\PYG{p}{,} \PYG{l+s+s1}{\PYGZsq{}}\PYG{l+s+s1}{EE\PYGZus{}1}\PYG{l+s+s1}{\PYGZsq{}}\PYG{p}{:}\PYG{p}{[}\PYG{l+s+s1}{\PYGZsq{}}\PYG{l+s+s1}{voltage}\PYG{l+s+s1}{\PYGZsq{}}\PYG{p}{,} \PYG{l+s+s1}{\PYGZsq{}}\PYG{l+s+s1}{current}\PYG{l+s+s1}{\PYGZsq{}}\PYG{p}{]}\PYG{p}{\PYGZcb{}}\PYG{p}{,} \PYG{n}{returnfig}\PYG{o}{=}\PYG{k+kc}{True}\PYG{p}{)}
\end{sphinxVerbatim}
}

\end{sphinxuseclass}
\begin{sphinxuseclass}{nboutput}
\begin{sphinxuseclass}{nblast}
\hrule height -\fboxrule\relax
\vspace{\nbsphinxcodecellspacing}

\makeatletter\setbox\nbsphinxpromptbox\box\voidb@x\makeatother

\begin{nbsphinxfancyoutput}

\begin{sphinxuseclass}{output_area}
\begin{sphinxuseclass}{}
\noindent\sphinxincludegraphics[width=426\sphinxpxdimen,height=286\sphinxpxdimen]{{example_pump_Tutorial_complete_36_0}.png}

\end{sphinxuseclass}
\end{sphinxuseclass}
\end{nbsphinxfancyoutput}

\end{sphinxuseclass}
\end{sphinxuseclass}
\sphinxAtStartPar
We can also view this result graph using ‘rd.graph.show’. In this case, it shows the state of the model at the final time\sphinxhyphen{}step of the model run. Thus, while the \sphinxcode{\sphinxupquote{EE\_1}} flow is shown in orange (because it is off\sphinxhyphen{}nominal), the Water flows are not, because they have the same state at the final time\sphinxhyphen{}step.

\begin{sphinxuseclass}{nbinput}
{
\sphinxsetup{VerbatimColor={named}{nbsphinx-code-bg}}
\sphinxsetup{VerbatimBorderColor={named}{nbsphinx-code-border}}
\begin{sphinxVerbatim}[commandchars=\\\{\}]
\llap{\color{nbsphinxin}[19]:\,\hspace{\fboxrule}\hspace{\fboxsep}}\PYG{n}{rd}\PYG{o}{.}\PYG{n}{graph}\PYG{o}{.}\PYG{n}{show}\PYG{p}{(}\PYG{n}{resgraph\PYGZus{}fault}\PYG{p}{)}
\end{sphinxVerbatim}
}

\end{sphinxuseclass}
\begin{sphinxuseclass}{nboutput}
{

\kern-\sphinxverbatimsmallskipamount\kern-\baselineskip
\kern+\FrameHeightAdjust\kern-\fboxrule
\vspace{\nbsphinxcodecellspacing}

\sphinxsetup{VerbatimColor={named}{white}}
\sphinxsetup{VerbatimBorderColor={named}{nbsphinx-code-border}}
\begin{sphinxuseclass}{output_area}
\begin{sphinxuseclass}{}


\begin{sphinxVerbatim}[commandchars=\\\{\}]
\llap{\color{nbsphinxout}[19]:\,\hspace{\fboxrule}\hspace{\fboxsep}}(<Figure size 432x288 with 1 Axes>,
 <matplotlib.axes.\_subplots.AxesSubplot at 0x18df288cec8>)
\end{sphinxVerbatim}



\end{sphinxuseclass}
\end{sphinxuseclass}
}

\end{sphinxuseclass}
\begin{sphinxuseclass}{nboutput}
\begin{sphinxuseclass}{nblast}
\hrule height -\fboxrule\relax
\vspace{\nbsphinxcodecellspacing}

\makeatletter\setbox\nbsphinxpromptbox\box\voidb@x\makeatother

\begin{nbsphinxfancyoutput}

\begin{sphinxuseclass}{output_area}
\begin{sphinxuseclass}{}
\noindent\sphinxincludegraphics[width=349\sphinxpxdimen,height=231\sphinxpxdimen]{{example_pump_Tutorial_complete_38_1}.png}

\end{sphinxuseclass}
\end{sphinxuseclass}
\end{nbsphinxfancyoutput}

\end{sphinxuseclass}
\end{sphinxuseclass}
\sphinxAtStartPar
If we want to view the graph at another time\sphinxhyphen{}step, we can use ‘rd.process.hist’ and ‘rd.graph.result\_from’ to first process the model states into results which can be visualized and then map that onto a graph view at a given time.

\begin{sphinxuseclass}{nbinput}
\begin{sphinxuseclass}{nblast}
{
\sphinxsetup{VerbatimColor={named}{nbsphinx-code-bg}}
\sphinxsetup{VerbatimBorderColor={named}{nbsphinx-code-border}}
\begin{sphinxVerbatim}[commandchars=\\\{\}]
\llap{\color{nbsphinxin}[20]:\,\hspace{\fboxrule}\hspace{\fboxsep}}\PYG{n}{reshist\PYGZus{}fault}\PYG{p}{,} \PYG{n}{b}\PYG{p}{,} \PYG{n}{c} \PYG{o}{=} \PYG{n}{rd}\PYG{o}{.}\PYG{n}{process}\PYG{o}{.}\PYG{n}{hist}\PYG{p}{(}\PYG{n}{mdlhist\PYGZus{}fault}\PYG{p}{)}
\end{sphinxVerbatim}
}

\end{sphinxuseclass}
\end{sphinxuseclass}
\begin{sphinxuseclass}{nbinput}
{
\sphinxsetup{VerbatimColor={named}{nbsphinx-code-bg}}
\sphinxsetup{VerbatimBorderColor={named}{nbsphinx-code-border}}
\begin{sphinxVerbatim}[commandchars=\\\{\}]
\llap{\color{nbsphinxin}[21]:\,\hspace{\fboxrule}\hspace{\fboxsep}}\PYG{n}{rd}\PYG{o}{.}\PYG{n}{graph}\PYG{o}{.}\PYG{n}{result\PYGZus{}from}\PYG{p}{(}\PYG{n}{mdl}\PYG{p}{,} \PYG{n}{reshist\PYGZus{}fault}\PYG{p}{,} \PYG{l+m+mi}{20}\PYG{p}{,} \PYG{n}{gtype}\PYG{o}{=}\PYG{l+s+s1}{\PYGZsq{}}\PYG{l+s+s1}{normal}\PYG{l+s+s1}{\PYGZsq{}}\PYG{p}{)}
\end{sphinxVerbatim}
}

\end{sphinxuseclass}
\begin{sphinxuseclass}{nboutput}
\hrule height -\fboxrule\relax
\vspace{\nbsphinxcodecellspacing}

\savebox\nbsphinxpromptbox[0pt][r]{\color{nbsphinxout}\Verb|\strut{[21]:}\,|}

\begin{nbsphinxfancyoutput}

\begin{sphinxuseclass}{output_area}
\begin{sphinxuseclass}{}
\noindent\sphinxincludegraphics[width=349\sphinxpxdimen,height=231\sphinxpxdimen]{{example_pump_Tutorial_complete_41_0}.png}

\end{sphinxuseclass}
\end{sphinxuseclass}
\end{nbsphinxfancyoutput}

\end{sphinxuseclass}
\begin{sphinxuseclass}{nboutput}
\begin{sphinxuseclass}{nblast}
\hrule height -\fboxrule\relax
\vspace{\nbsphinxcodecellspacing}

\makeatletter\setbox\nbsphinxpromptbox\box\voidb@x\makeatother

\begin{nbsphinxfancyoutput}

\begin{sphinxuseclass}{output_area}
\begin{sphinxuseclass}{}
\noindent\sphinxincludegraphics[width=349\sphinxpxdimen,height=231\sphinxpxdimen]{{example_pump_Tutorial_complete_41_1}.png}

\end{sphinxuseclass}
\end{sphinxuseclass}
\end{nbsphinxfancyoutput}

\end{sphinxuseclass}
\end{sphinxuseclass}

\paragraph{4a.) Simulate set of fault modes}
\label{\detokenize{example_pump/Tutorial_complete:4a.)-Simulate-set-of-fault-modes}}
\sphinxAtStartPar
To simulate the set of fault modes, we first choose a \sphinxcode{\sphinxupquote{SampleApproach}}. For simplicity, we can choose default parameters at first.

\begin{sphinxuseclass}{nbinput}
\begin{sphinxuseclass}{nblast}
{
\sphinxsetup{VerbatimColor={named}{nbsphinx-code-bg}}
\sphinxsetup{VerbatimBorderColor={named}{nbsphinx-code-border}}
\begin{sphinxVerbatim}[commandchars=\\\{\}]
\llap{\color{nbsphinxin}[22]:\,\hspace{\fboxrule}\hspace{\fboxsep}}\PYG{n}{app} \PYG{o}{=} \PYG{n}{SampleApproach}\PYG{p}{(}\PYG{n}{mdl}\PYG{p}{)}
\end{sphinxVerbatim}
}

\end{sphinxuseclass}
\end{sphinxuseclass}
\begin{sphinxuseclass}{nbinput}
{
\sphinxsetup{VerbatimColor={named}{nbsphinx-code-bg}}
\sphinxsetup{VerbatimBorderColor={named}{nbsphinx-code-border}}
\begin{sphinxVerbatim}[commandchars=\\\{\}]
\llap{\color{nbsphinxin}[23]:\,\hspace{\fboxrule}\hspace{\fboxsep}}\PYG{n}{endclasses\PYGZus{}app}\PYG{p}{,} \PYG{n}{mdlhists\PYGZus{}app} \PYG{o}{=} \PYG{n}{propagate}\PYG{o}{.}\PYG{n}{approach}\PYG{p}{(}\PYG{n}{mdl}\PYG{p}{,} \PYG{n}{app}\PYG{p}{)}
\end{sphinxVerbatim}
}

\end{sphinxuseclass}
\begin{sphinxuseclass}{nboutput}
\begin{sphinxuseclass}{nblast}
{

\kern-\sphinxverbatimsmallskipamount\kern-\baselineskip
\kern+\FrameHeightAdjust\kern-\fboxrule
\vspace{\nbsphinxcodecellspacing}

\sphinxsetup{VerbatimColor={named}{nbsphinx-stderr}}
\sphinxsetup{VerbatimBorderColor={named}{nbsphinx-code-border}}
\begin{sphinxuseclass}{output_area}
\begin{sphinxuseclass}{stderr}


\begin{sphinxVerbatim}[commandchars=\\\{\}]
SCENARIOS COMPLETE: 100\%|█████████████████████████████████████████████████████████████| 17/17 [00:00<00:00, 208.71it/s]
\end{sphinxVerbatim}



\end{sphinxuseclass}
\end{sphinxuseclass}
}

\end{sphinxuseclass}
\end{sphinxuseclass}
\sphinxAtStartPar
It can be helpful to view what these results look like–a dictionary of faults injected at particular times with their respective results dictionaries.

\begin{sphinxuseclass}{nbinput}
{
\sphinxsetup{VerbatimColor={named}{nbsphinx-code-bg}}
\sphinxsetup{VerbatimBorderColor={named}{nbsphinx-code-border}}
\begin{sphinxVerbatim}[commandchars=\\\{\}]
\llap{\color{nbsphinxin}[24]:\,\hspace{\fboxrule}\hspace{\fboxsep}}\PYG{n}{endclasses\PYGZus{}app}
\end{sphinxVerbatim}
}

\end{sphinxuseclass}
\begin{sphinxuseclass}{nboutput}
\begin{sphinxuseclass}{nblast}
{

\kern-\sphinxverbatimsmallskipamount\kern-\baselineskip
\kern+\FrameHeightAdjust\kern-\fboxrule
\vspace{\nbsphinxcodecellspacing}

\sphinxsetup{VerbatimColor={named}{white}}
\sphinxsetup{VerbatimBorderColor={named}{nbsphinx-code-border}}
\begin{sphinxuseclass}{output_area}
\begin{sphinxuseclass}{}


\begin{sphinxVerbatim}[commandchars=\\\{\}]
\llap{\color{nbsphinxout}[24]:\,\hspace{\fboxrule}\hspace{\fboxsep}}\{'ImportEE no\_v, t=27': \{'rate': 0.0003600000000000001,
  'cost': 15174.999999999998,
  'expected cost': 546300.0\},
 'ImportEE inf\_v, t=27': \{'rate': 9.000000000000002e-05,
  'cost': 20175.0,
  'expected cost': 181575.00000000003\},
 'ImportWater no\_wat, t=27': \{'rate': 0.00015,
  'cost': 6174.999999999998,
  'expected cost': 92624.99999999996\},
 'ImportSignal no\_sig, t=27': \{'rate': 1.2857142857142856e-05,
  'cost': 15174.999999999998,
  'expected cost': 19510.714285714283\},
 'MoveWater mech\_break, t=27': \{'rate': 0.00023142857142857142,
  'cost': 10174.999999999998,
  'expected cost': 235478.5714285714\},
 'MoveWater short, t=27': \{'rate': 0.00012857142857142858,
  'cost': 25175.0,
  'expected cost': 323678.5714285714\},
 'ExportWater block, t=27': \{'rate': 0.00012857142857142858,
  'cost': 15150.25,
  'expected cost': 194788.92857142858\},
 'ImportWater no\_wat, t=2': \{'rate': 1.6666666666666667e-05,
  'cost': 11125.000000000007,
  'expected cost': 18541.66666666668\},
 'ImportSignal no\_sig, t=2': \{'rate': 2.1428571428571427e-06,
  'cost': 20125.000000000007,
  'expected cost': 4312.500000000001\},
 'MoveWater mech\_break, t=2': \{'rate': 2.1428571428571427e-06,
  'cost': 15125.000000000007,
  'expected cost': 3241.07142857143\},
 'MoveWater short, t=2': \{'rate': 2.1428571428571428e-05,
  'cost': 30125.000000000007,
  'expected cost': 64553.57142857144\},
 'ExportWater block, t=2': \{'rate': 2.1428571428571428e-05,
  'cost': 20100.250000000007,
  'expected cost': 43071.9642857143\},
 'ImportWater no\_wat, t=52': \{'rate': 1.6666666666666667e-05,
  'cost': 1000.0,
  'expected cost': 1666.6666666666667\},
 'ImportSignal no\_sig, t=52': \{'rate': 1.4285714285714284e-06,
  'cost': 10000.0,
  'expected cost': 1428.5714285714284\},
 'MoveWater mech\_break, t=52': \{'rate': 2.1428571428571427e-06,
  'cost': 5000.0,
  'expected cost': 1071.4285714285713\},
 'MoveWater short, t=52': \{'rate': 1.4285714285714285e-05,
  'cost': 10000.0,
  'expected cost': 14285.714285714286\},
 'ExportWater block, t=52': \{'rate': 1.4285714285714285e-05,
  'cost': 5000.0,
  'expected cost': 7142.857142857143\},
 'nominal': \{'rate': 1.0, 'cost': 0.0, 'expected cost': 0.0\}\}
\end{sphinxVerbatim}



\end{sphinxuseclass}
\end{sphinxuseclass}
}

\end{sphinxuseclass}
\end{sphinxuseclass}
\sphinxAtStartPar
It can be helpful to explore different approach parameters (e.g. focussing on single faults, different numbers of sample points, etc)

\begin{sphinxuseclass}{nbinput}
\begin{sphinxuseclass}{nblast}
{
\sphinxsetup{VerbatimColor={named}{nbsphinx-code-bg}}
\sphinxsetup{VerbatimBorderColor={named}{nbsphinx-code-border}}
\begin{sphinxVerbatim}[commandchars=\\\{\}]
\llap{\color{nbsphinxin}[25]:\,\hspace{\fboxrule}\hspace{\fboxsep}}\PYG{n}{app2} \PYG{o}{=} \PYG{n}{SampleApproach}\PYG{p}{(}\PYG{n}{mdl}\PYG{p}{,} \PYG{n}{faults}\PYG{o}{=}\PYG{p}{[}\PYG{p}{(}\PYG{l+s+s1}{\PYGZsq{}}\PYG{l+s+s1}{MoveWater}\PYG{l+s+s1}{\PYGZsq{}}\PYG{p}{,} \PYG{l+s+s1}{\PYGZsq{}}\PYG{l+s+s1}{short}\PYG{l+s+s1}{\PYGZsq{}}\PYG{p}{)}\PYG{p}{]}\PYG{p}{)}
\end{sphinxVerbatim}
}

\end{sphinxuseclass}
\end{sphinxuseclass}
\begin{sphinxuseclass}{nbinput}
{
\sphinxsetup{VerbatimColor={named}{nbsphinx-code-bg}}
\sphinxsetup{VerbatimBorderColor={named}{nbsphinx-code-border}}
\begin{sphinxVerbatim}[commandchars=\\\{\}]
\llap{\color{nbsphinxin}[26]:\,\hspace{\fboxrule}\hspace{\fboxsep}}\PYG{n}{app2}\PYG{o}{.}\PYG{n}{list\PYGZus{}modes}\PYG{p}{(}\PYG{p}{)}
\end{sphinxVerbatim}
}

\end{sphinxuseclass}
\begin{sphinxuseclass}{nboutput}
\begin{sphinxuseclass}{nblast}
{

\kern-\sphinxverbatimsmallskipamount\kern-\baselineskip
\kern+\FrameHeightAdjust\kern-\fboxrule
\vspace{\nbsphinxcodecellspacing}

\sphinxsetup{VerbatimColor={named}{white}}
\sphinxsetup{VerbatimBorderColor={named}{nbsphinx-code-border}}
\begin{sphinxuseclass}{output_area}
\begin{sphinxuseclass}{}


\begin{sphinxVerbatim}[commandchars=\\\{\}]
\llap{\color{nbsphinxout}[26]:\,\hspace{\fboxrule}\hspace{\fboxsep}}[('MoveWater', 'short')]
\end{sphinxVerbatim}



\end{sphinxuseclass}
\end{sphinxuseclass}
}

\end{sphinxuseclass}
\end{sphinxuseclass}
\sphinxAtStartPar
There are a number of different ways to sample the scenarios in the approach:

\begin{sphinxuseclass}{nbinput}
\begin{sphinxuseclass}{nblast}
{
\sphinxsetup{VerbatimColor={named}{nbsphinx-code-bg}}
\sphinxsetup{VerbatimBorderColor={named}{nbsphinx-code-border}}
\begin{sphinxVerbatim}[commandchars=\\\{\}]
\llap{\color{nbsphinxin}[27]:\,\hspace{\fboxrule}\hspace{\fboxsep}}\PYG{n}{app3} \PYG{o}{=} \PYG{n}{SampleApproach}\PYG{p}{(}\PYG{n}{mdl}\PYG{p}{,} \PYG{n}{defaultsamp}\PYG{o}{=}\PYG{p}{\PYGZob{}}\PYG{l+s+s1}{\PYGZsq{}}\PYG{l+s+s1}{samp}\PYG{l+s+s1}{\PYGZsq{}}\PYG{p}{:}\PYG{l+s+s1}{\PYGZsq{}}\PYG{l+s+s1}{evenspacing}\PYG{l+s+s1}{\PYGZsq{}}\PYG{p}{,} \PYG{l+s+s1}{\PYGZsq{}}\PYG{l+s+s1}{numpts}\PYG{l+s+s1}{\PYGZsq{}}\PYG{p}{:}\PYG{l+m+mi}{5}\PYG{p}{\PYGZcb{}}\PYG{p}{)}
\end{sphinxVerbatim}
}

\end{sphinxuseclass}
\end{sphinxuseclass}
\begin{sphinxuseclass}{nbinput}
{
\sphinxsetup{VerbatimColor={named}{nbsphinx-code-bg}}
\sphinxsetup{VerbatimBorderColor={named}{nbsphinx-code-border}}
\begin{sphinxVerbatim}[commandchars=\\\{\}]
\llap{\color{nbsphinxin}[28]:\,\hspace{\fboxrule}\hspace{\fboxsep}}\PYG{n}{app3}\PYG{o}{.}\PYG{n}{times}
\end{sphinxVerbatim}
}

\end{sphinxuseclass}
\begin{sphinxuseclass}{nboutput}
\begin{sphinxuseclass}{nblast}
{

\kern-\sphinxverbatimsmallskipamount\kern-\baselineskip
\kern+\FrameHeightAdjust\kern-\fboxrule
\vspace{\nbsphinxcodecellspacing}

\sphinxsetup{VerbatimColor={named}{white}}
\sphinxsetup{VerbatimBorderColor={named}{nbsphinx-code-border}}
\begin{sphinxuseclass}{output_area}
\begin{sphinxuseclass}{}


\begin{sphinxVerbatim}[commandchars=\\\{\}]
\llap{\color{nbsphinxout}[28]:\,\hspace{\fboxrule}\hspace{\fboxsep}}[0, 1, 2, 3, 4, 12, 20, 27, 34, 42, 50, 51, 52, 53, 54]
\end{sphinxVerbatim}



\end{sphinxuseclass}
\end{sphinxuseclass}
}

\end{sphinxuseclass}
\end{sphinxuseclass}

\paragraph{4b.) Visualize set of fault modes}
\label{\detokenize{example_pump/Tutorial_complete:4b.)-Visualize-set-of-fault-modes}}
\sphinxAtStartPar
Using this fault approach, we can now make an FMEA\sphinxhyphen{}like analyses of the different fault modes. \sphinxcode{\sphinxupquote{rd.tabulate.summfmea}} organizes endclasses into a table for each fault.

\begin{sphinxuseclass}{nbinput}
{
\sphinxsetup{VerbatimColor={named}{nbsphinx-code-bg}}
\sphinxsetup{VerbatimBorderColor={named}{nbsphinx-code-border}}
\begin{sphinxVerbatim}[commandchars=\\\{\}]
\llap{\color{nbsphinxin}[29]:\,\hspace{\fboxrule}\hspace{\fboxsep}}\PYG{n}{summary\PYGZus{}fmea} \PYG{o}{=} \PYG{n}{rd}\PYG{o}{.}\PYG{n}{tabulate}\PYG{o}{.}\PYG{n}{summfmea}\PYG{p}{(}\PYG{n}{endclasses\PYGZus{}app}\PYG{p}{,} \PYG{n}{app}\PYG{p}{)}
\PYG{n}{summary\PYGZus{}fmea}
\end{sphinxVerbatim}
}

\end{sphinxuseclass}
\begin{sphinxuseclass}{nboutput}
\begin{sphinxuseclass}{nblast}
{

\kern-\sphinxverbatimsmallskipamount\kern-\baselineskip
\kern+\FrameHeightAdjust\kern-\fboxrule
\vspace{\nbsphinxcodecellspacing}

\sphinxsetup{VerbatimColor={named}{white}}
\sphinxsetup{VerbatimBorderColor={named}{nbsphinx-code-border}}
\begin{sphinxuseclass}{output_area}
\begin{sphinxuseclass}{}


\begin{sphinxVerbatim}[commandchars=\\\{\}]
\llap{\color{nbsphinxout}[29]:\,\hspace{\fboxrule}\hspace{\fboxsep}}                             rate          cost  expected cost
ImportEE     no\_v        0.000360  15175.000000  546300.000000
             inf\_v       0.000090  20175.000000  181575.000000
ImportWater  no\_wat      0.000183   6100.000000  112833.333333
ImportSignal no\_sig      0.000016  15100.000000   25251.785714
MoveWater    mech\_break  0.000236  10100.000000  239791.071429
             short       0.000164  21766.666667  402517.857143
ExportWater  block       0.000164  13416.833333  245003.750000
\end{sphinxVerbatim}



\end{sphinxuseclass}
\end{sphinxuseclass}
}

\end{sphinxuseclass}
\end{sphinxuseclass}
\sphinxAtStartPar
We can also use \sphinxcode{\sphinxupquote{rd.tabulatefullmea}} with the processed results histories to get a better picture of which flows and functions degrade in each scenario

\begin{sphinxuseclass}{nbinput}
\begin{sphinxuseclass}{nblast}
{
\sphinxsetup{VerbatimColor={named}{nbsphinx-code-bg}}
\sphinxsetup{VerbatimBorderColor={named}{nbsphinx-code-border}}
\begin{sphinxVerbatim}[commandchars=\\\{\}]
\llap{\color{nbsphinxin}[30]:\,\hspace{\fboxrule}\hspace{\fboxsep}}\PYG{n}{reshist\PYGZus{}fault}\PYG{p}{,} \PYG{n}{b}\PYG{p}{,} \PYG{n}{summaries} \PYG{o}{=} \PYG{n}{rd}\PYG{o}{.}\PYG{n}{process}\PYG{o}{.}\PYG{n}{hists}\PYG{p}{(}\PYG{n}{mdlhists\PYGZus{}app}\PYG{p}{)}
\end{sphinxVerbatim}
}

\end{sphinxuseclass}
\end{sphinxuseclass}
\begin{sphinxuseclass}{nbinput}
{
\sphinxsetup{VerbatimColor={named}{nbsphinx-code-bg}}
\sphinxsetup{VerbatimBorderColor={named}{nbsphinx-code-border}}
\begin{sphinxVerbatim}[commandchars=\\\{\}]
\llap{\color{nbsphinxin}[31]:\,\hspace{\fboxrule}\hspace{\fboxsep}}\PYG{n}{rd}\PYG{o}{.}\PYG{n}{tabulate}\PYG{o}{.}\PYG{n}{fullfmea}\PYG{p}{(}\PYG{n}{endclasses\PYGZus{}app}\PYG{p}{,} \PYG{n}{summaries}\PYG{p}{)}
\end{sphinxVerbatim}
}

\end{sphinxuseclass}
\begin{sphinxuseclass}{nboutput}
\begin{sphinxuseclass}{nblast}
{

\kern-\sphinxverbatimsmallskipamount\kern-\baselineskip
\kern+\FrameHeightAdjust\kern-\fboxrule
\vspace{\nbsphinxcodecellspacing}

\sphinxsetup{VerbatimColor={named}{white}}
\sphinxsetup{VerbatimBorderColor={named}{nbsphinx-code-border}}
\begin{sphinxuseclass}{output_area}
\begin{sphinxuseclass}{}


\begin{sphinxVerbatim}[commandchars=\\\{\}]
\llap{\color{nbsphinxout}[31]:\,\hspace{\fboxrule}\hspace{\fboxsep}}                                  degraded functions  \textbackslash{}
ImportEE no\_v, t=27                       [ImportEE]
ImportEE inf\_v, t=27                      [ImportEE]
ImportWater no\_wat, t=27               [ImportWater]
ImportSignal no\_sig, t=27             [ImportSignal]
MoveWater mech\_break, t=27               [MoveWater]
MoveWater short, t=27          [ImportEE, MoveWater]
ExportWater block, t=27     [MoveWater, ExportWater]
ImportWater no\_wat, t=2                [ImportWater]
ImportSignal no\_sig, t=2              [ImportSignal]
MoveWater mech\_break, t=2                [MoveWater]
MoveWater short, t=2           [ImportEE, MoveWater]
ExportWater block, t=2      [MoveWater, ExportWater]
ImportWater no\_wat, t=52               [ImportWater]
ImportSignal no\_sig, t=52             [ImportSignal]
MoveWater mech\_break, t=52               [MoveWater]
MoveWater short, t=52                    [MoveWater]
ExportWater block, t=52                [ExportWater]
nominal                                          NaN

                                         degraded flows         rate     cost  \textbackslash{}
ImportEE no\_v, t=27                [EE\_1, Wat\_1, Wat\_2]      0.00036    15175
ImportEE inf\_v, t=27               [EE\_1, Wat\_1, Wat\_2]        9e-05    20175
ImportWater no\_wat, t=27           [EE\_1, Wat\_1, Wat\_2]      0.00015     6175
ImportSignal no\_sig, t=27   [EE\_1, Sig\_1, Wat\_1, Wat\_2]  1.28571e-05    15175
MoveWater mech\_break, t=27         [EE\_1, Wat\_1, Wat\_2]  0.000231429    10175
MoveWater short, t=27              [EE\_1, Wat\_1, Wat\_2]  0.000128571    25175
ExportWater block, t=27            [EE\_1, Wat\_1, Wat\_2]  0.000128571  15150.2
ImportWater no\_wat, t=2            [EE\_1, Wat\_1, Wat\_2]  1.66667e-05    11125
ImportSignal no\_sig, t=2    [EE\_1, Sig\_1, Wat\_1, Wat\_2]  2.14286e-06    20125
MoveWater mech\_break, t=2          [EE\_1, Wat\_1, Wat\_2]  2.14286e-06    15125
MoveWater short, t=2               [EE\_1, Wat\_1, Wat\_2]  2.14286e-05    30125
ExportWater block, t=2             [EE\_1, Wat\_1, Wat\_2]  2.14286e-05  20100.3
ImportWater no\_wat, t=52                        [Wat\_1]  1.66667e-05     1000
ImportSignal no\_sig, t=52                            []  1.42857e-06    10000
MoveWater mech\_break, t=52                           []  2.14286e-06     5000
MoveWater short, t=52                                []  1.42857e-05    10000
ExportWater block, t=52                         [Wat\_2]  1.42857e-05     5000
nominal                                             NaN            1        0

                           expected cost
ImportEE no\_v, t=27               546300
ImportEE inf\_v, t=27              181575
ImportWater no\_wat, t=27           92625
ImportSignal no\_sig, t=27        19510.7
MoveWater mech\_break, t=27        235479
MoveWater short, t=27             323679
ExportWater block, t=27           194789
ImportWater no\_wat, t=2          18541.7
ImportSignal no\_sig, t=2          4312.5
MoveWater mech\_break, t=2        3241.07
MoveWater short, t=2             64553.6
ExportWater block, t=2             43072
ImportWater no\_wat, t=52         1666.67
ImportSignal no\_sig, t=52        1428.57
MoveWater mech\_break, t=52       1071.43
MoveWater short, t=52            14285.7
ExportWater block, t=52          7142.86
nominal                                0
\end{sphinxVerbatim}



\end{sphinxuseclass}
\end{sphinxuseclass}
}

\end{sphinxuseclass}
\end{sphinxuseclass}

\subsubsection{5.) Saving Work}
\label{\detokenize{example_pump/Tutorial_complete:5.)-Saving-Work}}
\sphinxAtStartPar
In detailed simulations, running a lot of computational simulations can take a considerable amount of time. As a result, it becomes impractical to run a new simulation every time one wishes to analyse its data. The \sphinxcode{\sphinxupquote{dill}} package gives one a very simple method of saving and reloading a workspace one is working on: \sphinxhyphen{} \sphinxcode{\sphinxupquote{dill.dump\_session("filename.pkl")}} to save the session, and \sphinxhyphen{} \sphinxcode{\sphinxupquote{dill.load\_session("filename.pk1")}} to load the session.

\sphinxAtStartPar
Below, we demonstrate this by saving the session, clearing the workspace, and then loading the workspace again.

\begin{sphinxuseclass}{nbinput}
\begin{sphinxuseclass}{nblast}
{
\sphinxsetup{VerbatimColor={named}{nbsphinx-code-bg}}
\sphinxsetup{VerbatimBorderColor={named}{nbsphinx-code-border}}
\begin{sphinxVerbatim}[commandchars=\\\{\}]
\llap{\color{nbsphinxin}[32]:\,\hspace{\fboxrule}\hspace{\fboxsep}}\PYG{k+kn}{import} \PYG{n+nn}{dill}
\end{sphinxVerbatim}
}

\end{sphinxuseclass}
\end{sphinxuseclass}
\sphinxAtStartPar
Viewing a workspace variable:

\begin{sphinxuseclass}{nbinput}
{
\sphinxsetup{VerbatimColor={named}{nbsphinx-code-bg}}
\sphinxsetup{VerbatimBorderColor={named}{nbsphinx-code-border}}
\begin{sphinxVerbatim}[commandchars=\\\{\}]
\llap{\color{nbsphinxin}[42]:\,\hspace{\fboxrule}\hspace{\fboxsep}}\PYG{n}{endclasses\PYGZus{}app}
\end{sphinxVerbatim}
}

\end{sphinxuseclass}
\begin{sphinxuseclass}{nboutput}
\begin{sphinxuseclass}{nblast}
{

\kern-\sphinxverbatimsmallskipamount\kern-\baselineskip
\kern+\FrameHeightAdjust\kern-\fboxrule
\vspace{\nbsphinxcodecellspacing}

\sphinxsetup{VerbatimColor={named}{white}}
\sphinxsetup{VerbatimBorderColor={named}{nbsphinx-code-border}}
\begin{sphinxuseclass}{output_area}
\begin{sphinxuseclass}{}


\begin{sphinxVerbatim}[commandchars=\\\{\}]
\textcolor{ansi-red-intense}{\textbf{---------------------------------------------------------------------------}}
\textcolor{ansi-red-intense}{\textbf{NameError}}                                 Traceback (most recent call last)
\textcolor{ansi-green-intense}{\textbf{<ipython-input-42-afce5bd1bf0c>}} in \textcolor{ansi-cyan}{<module>}
\textcolor{ansi-green-intense}{\textbf{----> 1}}\textcolor{ansi-yellow-intense}{\textbf{ }}endclasses\_app

\textcolor{ansi-red-intense}{\textbf{NameError}}: name 'endclasses\_app' is not defined
\end{sphinxVerbatim}



\end{sphinxuseclass}
\end{sphinxuseclass}
}

\end{sphinxuseclass}
\end{sphinxuseclass}
\sphinxAtStartPar
Dumping the session:

\begin{sphinxuseclass}{nbinput}
\begin{sphinxuseclass}{nblast}
{
\sphinxsetup{VerbatimColor={named}{nbsphinx-code-bg}}
\sphinxsetup{VerbatimBorderColor={named}{nbsphinx-code-border}}
\begin{sphinxVerbatim}[commandchars=\\\{\}]
\llap{\color{nbsphinxin}[35]:\,\hspace{\fboxrule}\hspace{\fboxsep}}\PYG{n}{dill}\PYG{o}{.}\PYG{n}{dump\PYGZus{}session}\PYG{p}{(}\PYG{l+s+s1}{\PYGZsq{}}\PYG{l+s+s1}{Tutorial.pkl}\PYG{l+s+s1}{\PYGZsq{}}\PYG{p}{)}
\end{sphinxVerbatim}
}

\end{sphinxuseclass}
\end{sphinxuseclass}
\sphinxAtStartPar
Reseting the workspace and viewing the variables:

\begin{sphinxuseclass}{nbinput}
{
\sphinxsetup{VerbatimColor={named}{nbsphinx-code-bg}}
\sphinxsetup{VerbatimBorderColor={named}{nbsphinx-code-border}}
\begin{sphinxVerbatim}[commandchars=\\\{\}]
\llap{\color{nbsphinxin}[36]:\,\hspace{\fboxrule}\hspace{\fboxsep}}\PYG{o}{\PYGZpc{}}\PYG{k}{reset}
\end{sphinxVerbatim}
}

\end{sphinxuseclass}
\begin{sphinxuseclass}{nboutput}
\begin{sphinxuseclass}{nblast}
{

\kern-\sphinxverbatimsmallskipamount\kern-\baselineskip
\kern+\FrameHeightAdjust\kern-\fboxrule
\vspace{\nbsphinxcodecellspacing}

\sphinxsetup{VerbatimColor={named}{white}}
\sphinxsetup{VerbatimBorderColor={named}{nbsphinx-code-border}}
\begin{sphinxuseclass}{output_area}
\begin{sphinxuseclass}{}


\begin{sphinxVerbatim}[commandchars=\\\{\}]
Once deleted, variables cannot be recovered. Proceed (y/[n])? y
\end{sphinxVerbatim}



\end{sphinxuseclass}
\end{sphinxuseclass}
}

\end{sphinxuseclass}
\end{sphinxuseclass}
\begin{sphinxuseclass}{nbinput}
{
\sphinxsetup{VerbatimColor={named}{nbsphinx-code-bg}}
\sphinxsetup{VerbatimBorderColor={named}{nbsphinx-code-border}}
\begin{sphinxVerbatim}[commandchars=\\\{\}]
\llap{\color{nbsphinxin}[43]:\,\hspace{\fboxrule}\hspace{\fboxsep}}\PYG{n}{endclasses\PYGZus{}app}
\end{sphinxVerbatim}
}

\end{sphinxuseclass}
\begin{sphinxuseclass}{nboutput}
\begin{sphinxuseclass}{nblast}
{

\kern-\sphinxverbatimsmallskipamount\kern-\baselineskip
\kern+\FrameHeightAdjust\kern-\fboxrule
\vspace{\nbsphinxcodecellspacing}

\sphinxsetup{VerbatimColor={named}{white}}
\sphinxsetup{VerbatimBorderColor={named}{nbsphinx-code-border}}
\begin{sphinxuseclass}{output_area}
\begin{sphinxuseclass}{}


\begin{sphinxVerbatim}[commandchars=\\\{\}]
\textcolor{ansi-red-intense}{\textbf{---------------------------------------------------------------------------}}
\textcolor{ansi-red-intense}{\textbf{NameError}}                                 Traceback (most recent call last)
\textcolor{ansi-green-intense}{\textbf{<ipython-input-43-afce5bd1bf0c>}} in \textcolor{ansi-cyan}{<module>}
\textcolor{ansi-green-intense}{\textbf{----> 1}}\textcolor{ansi-yellow-intense}{\textbf{ }}endclasses\_app

\textcolor{ansi-red-intense}{\textbf{NameError}}: name 'endclasses\_app' is not defined
\end{sphinxVerbatim}



\end{sphinxuseclass}
\end{sphinxuseclass}
}

\end{sphinxuseclass}
\end{sphinxuseclass}
\sphinxAtStartPar
Loading Session:

\begin{sphinxuseclass}{nbinput}
\begin{sphinxuseclass}{nblast}
{
\sphinxsetup{VerbatimColor={named}{nbsphinx-code-bg}}
\sphinxsetup{VerbatimBorderColor={named}{nbsphinx-code-border}}
\begin{sphinxVerbatim}[commandchars=\\\{\}]
\llap{\color{nbsphinxin}[45]:\,\hspace{\fboxrule}\hspace{\fboxsep}}\PYG{k+kn}{import} \PYG{n+nn}{dill}
\PYG{n}{dill}\PYG{o}{.}\PYG{n}{load\PYGZus{}session}\PYG{p}{(}\PYG{l+s+s1}{\PYGZsq{}}\PYG{l+s+s1}{Tutorial.pkl}\PYG{l+s+s1}{\PYGZsq{}}\PYG{p}{)}
\end{sphinxVerbatim}
}

\end{sphinxuseclass}
\end{sphinxuseclass}
\begin{sphinxuseclass}{nbinput}
{
\sphinxsetup{VerbatimColor={named}{nbsphinx-code-bg}}
\sphinxsetup{VerbatimBorderColor={named}{nbsphinx-code-border}}
\begin{sphinxVerbatim}[commandchars=\\\{\}]
\llap{\color{nbsphinxin}[46]:\,\hspace{\fboxrule}\hspace{\fboxsep}}\PYG{n}{endclasses\PYGZus{}app}
\end{sphinxVerbatim}
}

\end{sphinxuseclass}
\begin{sphinxuseclass}{nboutput}
\begin{sphinxuseclass}{nblast}
{

\kern-\sphinxverbatimsmallskipamount\kern-\baselineskip
\kern+\FrameHeightAdjust\kern-\fboxrule
\vspace{\nbsphinxcodecellspacing}

\sphinxsetup{VerbatimColor={named}{white}}
\sphinxsetup{VerbatimBorderColor={named}{nbsphinx-code-border}}
\begin{sphinxuseclass}{output_area}
\begin{sphinxuseclass}{}


\begin{sphinxVerbatim}[commandchars=\\\{\}]
\llap{\color{nbsphinxout}[46]:\,\hspace{\fboxrule}\hspace{\fboxsep}}\{'ImportEE no\_v, t=27': \{'rate': 0.0003600000000000001,
  'cost': 15174.999999999998,
  'expected cost': 546300.0\},
 'ImportEE inf\_v, t=27': \{'rate': 9.000000000000002e-05,
  'cost': 20175.0,
  'expected cost': 181575.00000000003\},
 'ImportWater no\_wat, t=27': \{'rate': 0.00015,
  'cost': 6174.999999999998,
  'expected cost': 92624.99999999996\},
 'ImportSignal no\_sig, t=27': \{'rate': 1.2857142857142856e-05,
  'cost': 15174.999999999998,
  'expected cost': 19510.714285714283\},
 'MoveWater mech\_break, t=27': \{'rate': 0.00023142857142857142,
  'cost': 10174.999999999998,
  'expected cost': 235478.5714285714\},
 'MoveWater short, t=27': \{'rate': 0.00012857142857142858,
  'cost': 25175.0,
  'expected cost': 323678.5714285714\},
 'ExportWater block, t=27': \{'rate': 0.00012857142857142858,
  'cost': 15150.25,
  'expected cost': 194788.92857142858\},
 'ImportWater no\_wat, t=2': \{'rate': 1.6666666666666667e-05,
  'cost': 11125.000000000007,
  'expected cost': 18541.66666666668\},
 'ImportSignal no\_sig, t=2': \{'rate': 2.1428571428571427e-06,
  'cost': 20125.000000000007,
  'expected cost': 4312.500000000001\},
 'MoveWater mech\_break, t=2': \{'rate': 2.1428571428571427e-06,
  'cost': 15125.000000000007,
  'expected cost': 3241.07142857143\},
 'MoveWater short, t=2': \{'rate': 2.1428571428571428e-05,
  'cost': 30125.000000000007,
  'expected cost': 64553.57142857144\},
 'ExportWater block, t=2': \{'rate': 2.1428571428571428e-05,
  'cost': 20100.250000000007,
  'expected cost': 43071.9642857143\},
 'ImportWater no\_wat, t=52': \{'rate': 1.6666666666666667e-05,
  'cost': 1000.0,
  'expected cost': 1666.6666666666667\},
 'ImportSignal no\_sig, t=52': \{'rate': 1.4285714285714284e-06,
  'cost': 10000.0,
  'expected cost': 1428.5714285714284\},
 'MoveWater mech\_break, t=52': \{'rate': 2.1428571428571427e-06,
  'cost': 5000.0,
  'expected cost': 1071.4285714285713\},
 'MoveWater short, t=52': \{'rate': 1.4285714285714285e-05,
  'cost': 10000.0,
  'expected cost': 14285.714285714286\},
 'ExportWater block, t=52': \{'rate': 1.4285714285714285e-05,
  'cost': 5000.0,
  'expected cost': 7142.857142857143\},
 'nominal': \{'rate': 1.0, 'cost': 0.0, 'expected cost': 0.0\}\}
\end{sphinxVerbatim}



\end{sphinxuseclass}
\end{sphinxuseclass}
}

\end{sphinxuseclass}
\end{sphinxuseclass}
\sphinxAtStartPar
\sphinxcode{\sphinxupquote{dill}} can also save/load individual variables, which may be more helpful when a notebook contains multiple large simulations. For this, use: \sphinxhyphen{} \sphinxcode{\sphinxupquote{dill.dump(variablename, 'filename.pkl')}} and \sphinxhyphen{} \sphinxcode{\sphinxupquote{variablename = dill.load('filename.pkl)}}, respectively.

\sphinxAtStartPar
See \sphinxcode{\sphinxupquote{dill}} documentation: \sphinxurl{https://dill.readthedocs.io/en/latest/dill.html}

\sphinxAtStartPar
Sometimes \sphinxcode{\sphinxupquote{dill}} may not be able to dump the session, if there are variables in the session which are unpickleable. Frequent offenders are variables which hold iterators (e.g. self.a=\{1:2\}.keys(), so it can be helpful to look out for those. You can debug dill by running \sphinxcode{\sphinxupquote{dill.detect.trace(True)}} or using other methods in the \sphinxcode{\sphinxupquote{dill.detect}} module.

\begin{sphinxuseclass}{nbinput}
\begin{sphinxuseclass}{nblast}
{
\sphinxsetup{VerbatimColor={named}{nbsphinx-code-bg}}
\sphinxsetup{VerbatimBorderColor={named}{nbsphinx-code-border}}
\begin{sphinxVerbatim}[commandchars=\\\{\}]
\llap{\color{nbsphinxin}[ ]:\,\hspace{\fboxrule}\hspace{\fboxsep}}
\end{sphinxVerbatim}
}

\end{sphinxuseclass}
\end{sphinxuseclass}

\subsection{Defining and Visualizing fmdtools Model Structures}
\label{\detokenize{docs/Model_Structure_Visualization_Tutorial:Defining-and-Visualizing-fmdtools-Model-Structures}}\label{\detokenize{docs/Model_Structure_Visualization_Tutorial::doc}}
\sphinxAtStartPar
To ensure that a simulation meets the intent of a modeller, it is important to carefully define the structure of the model and run order. This notebook will demonstrate fmdtools’ interfaces both for setting up model structures and for fault visualization.

\sphinxAtStartPar
NOTE: For some of these visualizations to display properly without re\sphinxhyphen{}running code use, \sphinxcode{\sphinxupquote{File \sphinxhyphen{}> Trust Notebook}}.

\begin{sphinxuseclass}{nbinput}
\begin{sphinxuseclass}{nblast}
{
\sphinxsetup{VerbatimColor={named}{nbsphinx-code-bg}}
\sphinxsetup{VerbatimBorderColor={named}{nbsphinx-code-border}}
\begin{sphinxVerbatim}[commandchars=\\\{\}]
\llap{\color{nbsphinxin}[1]:\,\hspace{\fboxrule}\hspace{\fboxsep}}\PYG{c+c1}{\PYGZsh{} for use in development \PYGZhy{} makes sure git version is used instead of pip\PYGZhy{}installed version}
\PYG{k+kn}{import} \PYG{n+nn}{sys}\PYG{o}{,} \PYG{n+nn}{os}
\PYG{n}{sys}\PYG{o}{.}\PYG{n}{path}\PYG{o}{.}\PYG{n}{insert}\PYG{p}{(}\PYG{l+m+mi}{1}\PYG{p}{,}\PYG{n}{os}\PYG{o}{.}\PYG{n}{path}\PYG{o}{.}\PYG{n}{join}\PYG{p}{(}\PYG{l+s+s2}{\PYGZdq{}}\PYG{l+s+s2}{..}\PYG{l+s+s2}{\PYGZdq{}}\PYG{p}{)}\PYG{p}{)}

\PYG{k+kn}{from} \PYG{n+nn}{fmdtools}\PYG{n+nn}{.}\PYG{n+nn}{modeldef} \PYG{k+kn}{import} \PYG{o}{*}
\PYG{k+kn}{import} \PYG{n+nn}{fmdtools}\PYG{n+nn}{.}\PYG{n+nn}{resultdisp} \PYG{k}{as} \PYG{n+nn}{rd}
\PYG{k+kn}{import} \PYG{n+nn}{fmdtools}\PYG{n+nn}{.}\PYG{n+nn}{faultsim}\PYG{n+nn}{.}\PYG{n+nn}{propagate} \PYG{k}{as} \PYG{n+nn}{prop}
\end{sphinxVerbatim}
}

\end{sphinxuseclass}
\end{sphinxuseclass}

\subsubsection{Basics}
\label{\detokenize{docs/Model_Structure_Visualization_Tutorial:Basics}}
\sphinxAtStartPar
An fmdtools model is made up of functions–model structures with behavioral methods and internal states–and flows–data relationships between functions. These functions and flows are defined in python classes and thus may be instantiated multiple times in a model to create multi\sphinxhyphen{}component models with complex interactions. The structure of these model classes is shown below:

\noindent\sphinxincludegraphics{{docs/files/figures/example_model}.png}

\sphinxAtStartPar
Creating a model thus involves: \sphinxhyphen{} defining the function and flow classes defining the behavior of the model in the model module \sphinxhyphen{} defining the specific structure for the model: function and flow \sphinxstyleemphasis{objects} (instantiations of functions) and their relationships.

\sphinxAtStartPar
Model structure visualization is performed in the \sphinxcode{\sphinxupquote{rd.graph}} module, which has each of the methods required to vizualize the graph structure. By default, these methods (except for set\_pos) use the \sphinxcode{\sphinxupquote{matplotlib}} renderer, which uses the methods built into \sphinxcode{\sphinxupquote{networkx}} to visualize the graph.

\begin{sphinxuseclass}{nbinput}
{
\sphinxsetup{VerbatimColor={named}{nbsphinx-code-bg}}
\sphinxsetup{VerbatimBorderColor={named}{nbsphinx-code-border}}
\begin{sphinxVerbatim}[commandchars=\\\{\}]
\llap{\color{nbsphinxin}[2]:\,\hspace{\fboxrule}\hspace{\fboxsep}}\PYG{n}{help}\PYG{p}{(}\PYG{n}{rd}\PYG{o}{.}\PYG{n}{graph}\PYG{p}{)}
\end{sphinxVerbatim}
}

\end{sphinxuseclass}
\begin{sphinxuseclass}{nboutput}
\begin{sphinxuseclass}{nblast}
{

\kern-\sphinxverbatimsmallskipamount\kern-\baselineskip
\kern+\FrameHeightAdjust\kern-\fboxrule
\vspace{\nbsphinxcodecellspacing}

\sphinxsetup{VerbatimColor={named}{white}}
\sphinxsetup{VerbatimBorderColor={named}{nbsphinx-code-border}}
\begin{sphinxuseclass}{output_area}
\begin{sphinxuseclass}{}


\begin{sphinxVerbatim}[commandchars=\\\{\}]
Help on module fmdtools.resultdisp.graph in fmdtools.resultdisp:

NAME
    fmdtools.resultdisp.graph

DESCRIPTION
    File Name: resultdisp/graph.py
    Contributors: Daniel Hulse and Sequoia Andrade
    Created: November 2019
    Refactored: April 2020
    Added major interfaces: July 2021

    Description: Gives graph-level visualizations of the model using installed renderers.

    Public user-facing methods:
        - set\_pos:                      Set graph node positions manually (uses netgraph)
        - show:                         Plots a single graph object g. Has options for heatmaps/overlays and matplotlib/graphviz/netgraph/pyviz renderers.
        - exec\_order:                   Displays the propagation order and type (dynamic/static) in the model. Works with matplotlib/graphviz/netgraph renderers.
        - history:                      Displays plots of the graph over time given a dict history of graph objects.  Works with matplotlib/graphviz/netgraph renderers.
        - result\_from:                  Plots a representation of the model graph at a specific time in the results history. Works with matplotlib/graphviz/netgraph renderers.
        - results\_from:                 Plots a set of representations of the model graph at given times in the results history. Works with matplotlib/graphviz/netgraph renderers.
        - animation\_from:               Creates an animation of the model graph using results at given times in the results history.  Works with matplotlib/netgraph renderers.

FUNCTIONS
    animation\_from(mdl, reshist, times='all', faultscen=[], gtype='bipartite', figsize=(6, 4), showfaultlabels=True, scale=1, show=False, pos=[], colors=['lightgray', 'orange', 'red'], renderer='matplotlib')
        Creates an animation of the model graph using results at given times in the results history.
        To view, use \%matplotlib qt from spyder or \%matplotlib notebook from jupyter
        To save (or do anything useful)h, make sure ffmpeg is installed  https://www.wikihow.com/Install-FFmpeg-on-Windows

        Parameters
        ----------
        mdl : model
            The model the faults were run in.
        reshist : dict
            A dictionary of results (from process.hists() or process.typehist() for the typegraph option)
        times : list or 'all'
            The times in the history to plot the graph at. If 'all', plots them all
        faultscen : str, optional
            Name of the fault scenario. The default is [].
        gtype : str, optional
            The type of graph to plot (normal or bipartite). The default is 'bipartite'.
        showfaultlabels : bool, optional
            Whether or not to list faults on the plot. The default is True.
        scale : float, optional
            Scale factor for the node/label sizes. The default is 1.
        show : bool, optional
            Whether to show the plot at the end (may be redundant). The default is True.
        pos : dict, optional
            dict of node positions (if re-using positions). The default is [].

    exec\_order(mdl, renderer='matplotlib', gtype='bipartite', colors=['lightgray', 'cyan', 'teal'], show\_dyn\_order=True, title='Function Execution Order', legend=True, **kwargs)
        Displays the execution order/types of the model, where the functions and flows in the
        static step are highlighted and the functions in the dynamic step are listed (with corresponding order)

        Parameters
        ----------
        mdl : fmdtools Model
            Model of the system to visualize.
        renderer : 'matplotlib' or 'graphviz'
            Renderer to use for the graph
        gtype : 'normal'/'bipartite', optional
            Representation of the model to use. The default is 'bipartite'.
        colors : list, optional
            Colors to use for unexecuted functions, static propagation steps, and dynamic functions.
            The default is ['lightgray', 'cyan','teal'].
        show\_dyn\_order : bool, optional
            Whether to label the execution order for dynamic functions. The default is True.
        title : str, optional
            Title for the plot. The default is "Function Execution Order".
        legend : bool, optional
            Whether to show a legend. The default is True.
        **kwargs : see arguments for the respective renderers
        Returns
        -------
        tuple of form (figure, axis)

    get\_graph\_annotations(g, gtype='bipartite')
        Helper method that returns labels/lists degraded nodes for the plot annotations

    get\_graph\_pos(mdl, pos, gtype)
        Helper function for getting the right graph/positions from a model

    get\_plotlabels(g, reshist, t\_ind)
        Assigns labels to a graph g from reshist at time t so that it can be plotted

        Parameters
        ----------
        g : networkx graph
            The graph to get labels for
        reshist : dict
            The dict of results history over time (from process.hists() or process.typehist() for the typegraph option)
        t\_ind : float
            The time in reshist to update the graph at

        Returns
        -------
        labels : dict
            labels for the graph.
        faultfxns : dict
            functions with faults in them
        degfxns : dict
            functions that are degraded
        degflows : dict
            flows that are degraded
        faultlabels : dict
            names of each fault
        faultedges : dict
            edges with faults in them
        faultedgeflows : dict
            names of flows that are degraded on each edge
        edgelabels : dict
            labels of each edge

    gv\_colors(g, gtype, colors, heatmap, cmap, faultnodes, degradednodes, faultedges=[], edgeflows=\{\}, functions=[], flows=[], highlight=[])
        creates dictonary of node/edge colors for a graphviz plot

        Parameters
        ----------
        g : nx graph object or model
            The multigraph to plot
        gtype : string, optional
            Type of graph input to show
            values are 'normal', 'bipartite', or 'typegraph'.
        colors : list, optional
            List of colors to use for nominal, degraded, and faulty functions/flows.
            Default is: ['lightgray','orange', 'red']
        heatmap : dict, optional
            A heatmap dictionary to overlay on the plot. The default is \{\}.
        cmap : mpl colormap
            Colormap to use for heatmap visualization
        faultnodes : list
            list of the nodes with faults
        degradednodes : list
            list of the nodes with degraded functionality
        faultedges : list
            list of edges(flows) that have faults. Only used for 'normal' graph. The default is [].
        edgeflows : dictionary
            dictionary of edges (n1,n2) and edge/flow names. The default is \{\}.
        functions : list, optional
            list of function nodes. Only used for 'bipartite' graph. The default is [].
        flows : list, optional
            list of flow nodes. Only used for 'bipartite' graph. The default is [].

        Returns
        -------
        colors\_dict : dictionary
            dictionary withe keys as nodes/edges and values colors.

    gv\_execute\_order\_legend(colors)

    gv\_import\_check()
        \#\#\#GRAPHVIZ HELPER FUNCTIONS
        \#\#\#\#\#\#\#\#\#\#\#\#\#\#\#\#\#\#\#\#\#\#\#\#\#\#\#\#

    history(ghist, **kwargs)
        Displays plots of the graph over time given a dict history of graph objects

        Parameters
        ----------
        ghist : dict
            A dictionary of the history of the graph over time with structure:
           \{time: graphobject\}, where
               - time is the time where the snapshot of the graph was recorded
               - graphobject is the snapshot of the graph at that time
        **kwargs : kwargs
            keyword arguments for graph.show()
        Returns
        ----------
        figobjs : dict
            Set of graph objects from graph.show() for the given renderer

    plot\_bip\_netgraph(g, labels, faultfxns, degnodes, faultlabels, faultscen=[], time=0, showfaultlabels=True, scale=1, pos=[], show=True, colors=['lightgray', 'orange', 'red'], title=[], functions=[], flows=[], **kwargs)
        Experimental method for plotting with netgraph instead of networkx

    plot\_bipgraph(g, labels, faultfxns, degnodes, faultlabels, faultscen=[], time=0, showfaultlabels=True, scale=1, pos=[], show=True, colors=['lightgray', 'orange', 'red'], title=[], functions=[], flows=[])
        Plots a bipartite graph. Used in other functions

    plot\_gv\_bipartite(g, faultnodes, degradednodes, faultlabels, faultscen, time, showfaultlabels, colors\_dict, functions, flows, edges, dot)

    plot\_gv\_normgraph(g, edgeflows, faultnodes, degradednodes, faultflows, faultlabels, faultedges, faultscen, time, showfaultlabels, colors\_dict, dot)

    plot\_norm\_netgraph(g, labels, faultfxns, degfxns, degflows, faultlabels, faultedges, faultedgeflows, faultscen, time, showfaultlabels, edgeflows, scale=1, pos=[], show=True, colors=['lightgray', 'orange', 'red'], title=[], show\_edgelabels=True, **kwargs)
        Experimental method for plotting with netgraph instead of networkx

    plot\_normgraph(g, labels, faultfxns, degfxns, degflows, faultlabels, faultedges, faultedgeflows, faultscen, time, showfaultlabels, edgeflows, scale=1, pos=[], show=True, colors=['lightgray', 'orange', 'red'], title=[], show\_edgelabels=True)
        Plots a standard graph. Used in other functions

    result\_from(mdl, reshist, time, renderer='matplotlib', gtype='bipartite', **kwargs)
        Plots a representation of the model graph at a specific time in the results history.

        Parameters
        ----------
        mdl : model
            The model the faults were run in.
        reshist : dict
            A dictionary of results (from process.hists() or process.typehist() for the typegraph option)
        time : float
            The time in the history to plot the graph at.
        renderer : 'matplotlib' or 'graphviz' or 'netgraph'
            Renderer to use to plot the graph. Default is 'matplotlib'
        gtype : str, optional
            The type of graph to plot (normal or bipartite). The default is 'bipartite'.
        MATPLOTLIB OPTIONS:
        ----------
        faultscen : str, optional
            Name of the fault scenario. The default is [].
        showfaultlabels : bool, optional
            Whether or not to list faults on the plot. The default is True.
        scale : float, optional
            Scale factor for the node/label sizes. The default is 1.
        pos : dict, optional
            dict of node positions (if re-using positions). The default is [].

    results\_from(mdl, reshist, times, renderer='matplotlib', gtype='bipartite', **kwargs)
        Plots a set of representations of the model graph at given times in the results history.

        Parameters
        ----------
        mdl : model
            The model the faults were run in.
        reshist : dict
            A dictionary of results (from process.hists() or process.typehist() for the typegraph option)
        times : list or 'all'
            The times in the history to plot the graph at. If 'all', plots them all
        renderer : 'matplotlib' or 'graphviz' or 'netgraph'
            Renderer to use to plot the graph. Default is 'matplotlib'
        gtype : str, optional
            The type of graph to plot (normal or bipartite). The default is 'bipartite'.
        MATPLOTLIB OPTIONS:
        ----------
        faultscen : str, optional
            Name of the fault scenario. The default is [].
        showfaultlabels : bool, optional
            Whether or not to list faults on the plot. The default is True.
        scale : float, optional
            Scale factor for the node/label sizes. The default is 1.
        pos : dict, optional
            dict of node positions (if re-using positions). The default is [].
        Returns
        ----------
        frames : Dict
            Dictionary of mpl figures keyed at each time \{time:fig\}

    set\_pos(g, gtype='normal', scale=1, node\_color='lightgray', label\_size=7, initpos=\{\}, figsize=(6, 4))
        Provides graphical interface to set graph node positions. If model is provided, it will also set the positions in the model object.

        To work, this method must be opened in an external window, so change the IPython before use usings \%matplotlib qt' (or '\%matplotlib osx')

        Parameters
        ----------
        g : networkx graph or model
            normal or bipartite graph of the model of interest
        gtype : 'normal' or 'bipartite', optional
            Type of graph to plot. The default is 'normal'.
        scale : float, optional
            scale for the node sizes. The default is 1.
        node\_color : str, optional
            color to use for the nodes. The default is 'lightgray'.
        label\_size : float, optional
            size to use for the labels. The default is 8.
        initpos : dict, optional
            dict of initial positions for the labels (e.g. from nx.spring\_layout). The default is \{\}.
        figsize : tuple, optional
            size of matplotlib frame. Default is (6,4)

        Returns
        -------
        pos: dict
            dict of node positions for use in graph plotting functions

    show(g, gtype='normal', renderer='matplotlib', filename='', **kwargs)
        Plots a single graph object g.

        Parameters
        ----------
        g : networkx graph or model
            The multigraph to plot
        gtype : 'normal' or 'bipartite'
            Type of graph input to show--normal (multgraph) or bipartite
        renderer : 'matplotlib' or 'graphviz' or 'pyviz' or 'netgraph'
            Renderer to use with the drawing. Renderer must be installed. Default is 'matplotlib'
        filename : string, optional
            the filename for the output. The default is '' (in which a file is not saved except in pyviz).
        **kwargs : dictionary
            keyword arguments for the individual methods. See the documentation for
                graph.show\_graphviz
                graph.show\_maplotlib
                graph.show\_pyviz
                graph.show\_netgraph
            for more information on these arguments

    show\_graphviz(g, gtype='normal', faultscen=[], time=[], filename='', filetype='png', showfaultlabels=True, highlight=[], colors=['lightgray', 'orange', 'red'], heatmap=\{\}, cmap=<matplotlib.colors.LinearSegmentedColormap object at 0x000002766E835A88>, **kwargs)
        Translates an existing nx graph to a graphviz graph. Saves the graph output and dot file.
        Called from show() by passing in graphviz=True and filename

        Parameters
        ----------
        g : nx graph object or model
            The multigraph to plot
        gtype : string, optional
            Type of graph input to show
            values are 'normal', 'bipartite', or 'typegraph'. The default is 'normal'.
        filename : string, optional
            the filename for the rendered output (if any). The default is '' (in which the file is not saved).
        filetype : string
            Type of file to save the figure as (if saving)
        faultscen : str, optional
            Name of the fault scenario (for the title). The default is [].
        time : float, optional
            Time of fault injection. The default is [].
        showfaultlabels : bool, optional
            Whether or not to label the faults on the functions. The default is True.
        highlight : list, optional
            Functions/flows to highlight using [faulty functions, degraded functions, degraded flows] labelling scheme.
            Used for custom overlays. Default is []
        colors : list, optional
            List of colors to use for nominal, degraded, and faulty functions/flows.
            Default is: ['lightgray','orange', 'red']
        heatmap : dict, optional
            A heatmap dictionary to overlay on the plot. The default is \{\}.
        cmap : mpl colormap
            Colormap to use for heatmap visualization
        **kwargs : dictionary
            dictionary of graphviz attributes used to customize the output.
            this includes layout, overlap, node padding, node separation, font, fontsize, etc.
            see http://www.graphviz.org/doc/info/attrs.html for all options

        Returns
        -------
        dot: a graphviz object

    show\_matplotlib(g, gtype='normal', filename='', filetype='png', pos=[], scale=1, faultscen=[], time=[], figsize=(6, 4), showfaultlabels=True, highlight=[], colors=['lightgray', 'orange', 'red'], heatmap=\{\}, cmap=<matplotlib.colors.LinearSegmentedColormap object at 0x000002766E835A88>)
        Plots a single graph object g using matplotlib

        Parameters
        ----------
        g : networkx graph or model
            The multigraph to plot
        gtype : 'normal' or 'bipartite'
            Type of graph input to show--normal (multgraph) or bipartite
        filename : string
            Name to give the saved file, if saved. Default is '' (not saving the file)
        filetype : string
            Type of file to save the figure as (if saving)
        pos : dict
            Positions for nodes
        scale: float
            Changes sizes of nodes in bipartite graph
        faultscen : str, optional
            Name of the fault scenario (for the title). The default is [].
        time : float, optional
            Time of fault injection. The default is [].
        showfaultlabels : bool, optional
            Whether or not to label the faults on the functions. The default is True.
        highlight : list, optional
            Functions/flows to highlight using [faulty functions, degraded functions, degraded flows] labelling scheme.
            Used for custom overlays. Default is []
        colors : list, optional
            List of colors to use for nominal, degraded, and faulty functions/flows.
            Default is: ['lightgray','orange', 'red']
        heatmap : dict, optional
            A heatmap dictionary to overlay on the plot. The default is \{\}.
        cmap : mpl colormap
            Colormap to use for heatmap visualizations

        Returns
        -------
        fig, ax : matplotlib figure/axis
            Matplotlib figure object of the drawn graph

    show\_netgraph(g, gtype='normal', filename='', filetype='png', pos=[], scale=1, faultscen=[], time=[], figsize=(6, 4), showfaultlabels=True, highlight=[], colors=['lightgray', 'orange', 'red'], **kwargs)
        Plots a single graph object g using netgraph

        Parameters
        ----------
        g : networkx graph or model
            The multigraph to plot
        gtype : 'normal' or 'bipartite'
            Type of graph input to show--normal (multgraph) or bipartite
        filename : string
            Name to give the saved file, if saved. Default is '' (not saving the file)
        filetype : string
            Type of file to save the figure as (if saving)
        pos : dict
            Positions for nodes
        scale: float
            Changes sizes of nodes in bipartite graph
        faultscen : str, optional
            Name of the fault scenario (for the title). The default is [].
        time : float, optional
            Time of fault injection. The default is [].
        showfaultlabels : bool, optional
            Whether or not to label the faults on the functions. The default is True.
        highlight : list, optional
            Functions/flows to highlight using [faulty functions, degraded functions, degraded flows] labelling scheme.
            Used for custom overlays. Default is []
        colors : list, optional
            List of colors to use for nominal, degraded, and faulty functions/flows.
            Default is: ['lightgray','orange', 'red']
        Returns
        -------
        fig, ax : matplotlib figure/axis
            Matplotlib figure object of the drawn graph
        gra : netgraph Graph
            Netgraph object which can be further manipulated

    show\_pyviz(g, gtype='typegraph', filename='typegraph', width=1000, filt=True, physics=False, notebook=False)
        Method for plotting graphs with pyvis. Produces interactive HTML!

        Parameters
        ----------
        g : networkx graph or model
            Graph to plot or fmdtools model (which will be used to get the graph)
        gtype : 'hierarchical'/'bipartite'/'component', optional
            Type of model graph to plot The default is 'hierarchical'.
        filename : str, optional
            File to save the html to. The default is "typegraph.html".
        width : int, optional
            Width of the frame in px. The default is 1000.
        filt : Dict/Bool, optional
            Whether to display sliders. The default is True.
        physics : Bool, optional
            Whether to use physics during node placement. The default is False.
        Returns
        -------
        n : pyviz object
            pyviz object of the drawn graph

    update\_bipplot(t\_ind, reshist, g, pos, faultscen=[], showfaultlabels=True, scale=1, show=True, colors=['lightgray', 'orange', 'red'], **kwargs)
        Updates a bipartite graph plot at a given timestep t\_ind given the result history reshist

    update\_graphplot(t\_ind, reshist, g, pos, faultscen=[], showfaultlabels=True, scale=1, show=True, colors=['lightgray', 'orange', 'red'], **kwargs)
        Updates a normal graph plot at a given timestep t\_ind given the result history reshist

    update\_gv\_bipplot(t\_ind, reshist, g, faultscen=[], showfaultlabels=True, colors=['lightgray', 'orange', 'red'], heatmap=\{\}, cmap=<matplotlib.colors.LinearSegmentedColormap object at 0x000002766E835A88>, **kwargs)
        graphviz helper: updates a bipartite graph plot at a given timestep t\_ind given the result history reshist

    update\_gv\_graphplot(t\_ind, reshist, g, faultscen=[], showfaultlabels=True, colors=['lightgray', 'orange', 'red'], heatmap=\{\}, cmap=<matplotlib.colors.LinearSegmentedColormap object at 0x000002766E835A88>, **kwargs)
        graphviz helpwer: Updates a normal graph plot at a given timestep t\_ind given the result history reshist

    update\_net\_bipplot(t\_ind, reshist, g, pos, faultscen=[], showfaultlabels=True, scale=1, show=True, colors=['lightgray', 'orange', 'red'], **kwargs)
        Updates a bipartite graph plot at a given timestep t\_ind given the result history reshist

    update\_net\_graphplot(t\_ind, reshist, g, pos, faultscen=[], showfaultlabels=True, scale=1, show=True, colors=['lightgray', 'orange', 'red'], **kwargs)
        Updates a normal graph plot at a given timestep t\_ind given the result history reshist

    update\_net\_typegraphplot(t\_ind, reshist, g, pos, faultscen=[], showfaultlabels=True, scale=1, show=True, colors=['lightgray', 'orange', 'red'], **kwargs)
        Updates a typegraph-stype plot at a given timestep t\_ind given the result history reshist

    update\_typegraphplot(t\_ind, reshist, g, pos, faultscen=[], showfaultlabels=True, scale=1, show=True, colors=['lightgray', 'orange', 'red'], **kwargs)
        Updates a typegraph-stype plot at a given timestep t\_ind given the result history reshist

FILE
    c:\textbackslash{}users\textbackslash{}dhulse\textbackslash{}documents\textbackslash{}github\textbackslash{}fmdtools\textbackslash{}fmdtools\textbackslash{}resultdisp\textbackslash{}graph.py


\end{sphinxVerbatim}



\end{sphinxuseclass}
\end{sphinxuseclass}
}

\end{sphinxuseclass}
\end{sphinxuseclass}
\sphinxAtStartPar
The main function used for displaying model structure is \sphinxcode{\sphinxupquote{graph.show()}}

\begin{sphinxuseclass}{nbinput}
{
\sphinxsetup{VerbatimColor={named}{nbsphinx-code-bg}}
\sphinxsetup{VerbatimBorderColor={named}{nbsphinx-code-border}}
\begin{sphinxVerbatim}[commandchars=\\\{\}]
\llap{\color{nbsphinxin}[3]:\,\hspace{\fboxrule}\hspace{\fboxsep}}\PYG{n}{help}\PYG{p}{(}\PYG{n}{rd}\PYG{o}{.}\PYG{n}{graph}\PYG{o}{.}\PYG{n}{show}\PYG{p}{)}
\end{sphinxVerbatim}
}

\end{sphinxuseclass}
\begin{sphinxuseclass}{nboutput}
\begin{sphinxuseclass}{nblast}
{

\kern-\sphinxverbatimsmallskipamount\kern-\baselineskip
\kern+\FrameHeightAdjust\kern-\fboxrule
\vspace{\nbsphinxcodecellspacing}

\sphinxsetup{VerbatimColor={named}{white}}
\sphinxsetup{VerbatimBorderColor={named}{nbsphinx-code-border}}
\begin{sphinxuseclass}{output_area}
\begin{sphinxuseclass}{}


\begin{sphinxVerbatim}[commandchars=\\\{\}]
Help on function show in module fmdtools.resultdisp.graph:

show(g, gtype='normal', renderer='matplotlib', filename='', **kwargs)
    Plots a single graph object g.

    Parameters
    ----------
    g : networkx graph or model
        The multigraph to plot
    gtype : 'normal' or 'bipartite'
        Type of graph input to show--normal (multgraph) or bipartite
    renderer : 'matplotlib' or 'graphviz' or 'pyviz' or 'netgraph'
        Renderer to use with the drawing. Renderer must be installed. Default is 'matplotlib'
    filename : string, optional
        the filename for the output. The default is '' (in which a file is not saved except in pyviz).
    **kwargs : dictionary
        keyword arguments for the individual methods. See the documentation for
            graph.show\_graphviz
            graph.show\_maplotlib
            graph.show\_pyviz
            graph.show\_netgraph
        for more information on these arguments

\end{sphinxVerbatim}



\end{sphinxuseclass}
\end{sphinxuseclass}
}

\end{sphinxuseclass}
\end{sphinxuseclass}
\sphinxAtStartPar
However, these relationships only define the \sphinxstyleemphasis{structure} of the model–for the model to simulate correctly and efficiently, the run order of fuctions additionally must be defined.

\sphinxAtStartPar
Each timestep of a model simulation can be broken down into two steps:

\noindent\sphinxincludegraphics{{docs/files/figures/dynamic_step}.png}
\begin{itemize}
\item {} 
\sphinxAtStartPar
\sphinxstylestrong{Dynamic Propagation Step:} In the dynamic propagation step, the time\sphinxhyphen{}based behaviors (e.g. accumulations, movement, etc.) are each run \sphinxstyleemphasis{once} in a specified order. These steps are generally quicker to execute because each behavior is only run once and the flows do not need to be tracked to determine which behaviors to execute next. This step is run first at each time\sphinxhyphen{}step of the model.

\end{itemize}

\noindent\sphinxincludegraphics{{docs/files/figures/propogation}.png}
\begin{itemize}
\item {} 
\sphinxAtStartPar
\sphinxstylestrong{Static Propagation Step:} In the static propagation step, behaviors are propagated between functions iteratively until the state of the model converges to a single value. This may require an update of multiple function behaviors until there are no more new behaviors to run. Thus, static behaviors should be ‘’timeless’’ (always give the same output for the same input) and convergent (behaviors in each function should not change each other ad infinitum). This step is run second at each
time\sphinxhyphen{}step of the model.

\end{itemize}

\sphinxAtStartPar
With these different behaviors, one can express a range of different types of models: \sphinxhyphen{} \sphinxstylestrong{static models} where only only one timestep is run, where fault scenarios show the immediate propagation of faults through the system. \sphinxhyphen{} \sphinxstylestrong{dynamic models} where a number of timesteps are run (but behaviors are only run once). \sphinxhyphen{} \sphinxstylestrong{hybrid models} where dynamic behaviors are run once and then a static propagation step is performed at each time\sphinxhyphen{}step.

\sphinxAtStartPar
The main interfaces/functions involved in defining run order are: \sphinxhyphen{} \sphinxcode{\sphinxupquote{FxnBlock.static\_behavior(self, time)}} and \sphinxcode{\sphinxupquote{FxnBlock.behavior(self, time)}}, which define function behaviors which occur during the static propagation step. \sphinxhyphen{} \sphinxcode{\sphinxupquote{FxnBlock.dynamic\_behavior(self, time)}}, which defines function behaviors during the dynamic propagation step. \sphinxhyphen{} \sphinxcode{\sphinxupquote{FxnBlock.condfaults(self, time)}}, does not define run order, but it can be used for behaviors which would at any time during simulation (dynamic or
static) lead to the system entering a fault mode (e.g., conditional faults). \sphinxhyphen{} \sphinxcode{\sphinxupquote{Model.add\_fxn()}}, which when used successively for each function specifies that those functions run in the order they are added.

\sphinxAtStartPar
There is additionally \sphinxcode{\sphinxupquote{Model.build\_model()}}, which enables one to manually specify the order of the dynamic propagation step (and initial static step):

\begin{sphinxuseclass}{nbinput}
{
\sphinxsetup{VerbatimColor={named}{nbsphinx-code-bg}}
\sphinxsetup{VerbatimBorderColor={named}{nbsphinx-code-border}}
\begin{sphinxVerbatim}[commandchars=\\\{\}]
\llap{\color{nbsphinxin}[4]:\,\hspace{\fboxrule}\hspace{\fboxsep}}\PYG{n}{help}\PYG{p}{(}\PYG{n}{Model}\PYG{o}{.}\PYG{n}{build\PYGZus{}model}\PYG{p}{)}
\end{sphinxVerbatim}
}

\end{sphinxuseclass}
\begin{sphinxuseclass}{nboutput}
\begin{sphinxuseclass}{nblast}
{

\kern-\sphinxverbatimsmallskipamount\kern-\baselineskip
\kern+\FrameHeightAdjust\kern-\fboxrule
\vspace{\nbsphinxcodecellspacing}

\sphinxsetup{VerbatimColor={named}{white}}
\sphinxsetup{VerbatimBorderColor={named}{nbsphinx-code-border}}
\begin{sphinxuseclass}{output_area}
\begin{sphinxuseclass}{}


\begin{sphinxVerbatim}[commandchars=\\\{\}]
Help on function build\_model in module fmdtools.modeldef:

build\_model(self, functionorder=[], graph\_pos=\{\}, bipartite\_pos=\{\})
    Builds the model graph after the functions have been added.

    Parameters
    ----------
    functionorder : list, optional
        The order for the functions to be executed in. The default is [].
    graph\_pos : dict, optional
        position of graph nodes. The default is \{\}.
    bipartite\_pos : dict, optional
        position of bipartite graph nodes. The default is \{\}.

\end{sphinxVerbatim}



\end{sphinxuseclass}
\end{sphinxuseclass}
}

\end{sphinxuseclass}
\end{sphinxuseclass}
\sphinxAtStartPar
The overall static/dynamic propagation steps of the model can then be visualized using \sphinxcode{\sphinxupquote{rd.graph.exec\_order}}

\begin{sphinxuseclass}{nbinput}
{
\sphinxsetup{VerbatimColor={named}{nbsphinx-code-bg}}
\sphinxsetup{VerbatimBorderColor={named}{nbsphinx-code-border}}
\begin{sphinxVerbatim}[commandchars=\\\{\}]
\llap{\color{nbsphinxin}[5]:\,\hspace{\fboxrule}\hspace{\fboxsep}}\PYG{n}{help}\PYG{p}{(}\PYG{n}{rd}\PYG{o}{.}\PYG{n}{graph}\PYG{o}{.}\PYG{n}{exec\PYGZus{}order}\PYG{p}{)}
\end{sphinxVerbatim}
}

\end{sphinxuseclass}
\begin{sphinxuseclass}{nboutput}
\begin{sphinxuseclass}{nblast}
{

\kern-\sphinxverbatimsmallskipamount\kern-\baselineskip
\kern+\FrameHeightAdjust\kern-\fboxrule
\vspace{\nbsphinxcodecellspacing}

\sphinxsetup{VerbatimColor={named}{white}}
\sphinxsetup{VerbatimBorderColor={named}{nbsphinx-code-border}}
\begin{sphinxuseclass}{output_area}
\begin{sphinxuseclass}{}


\begin{sphinxVerbatim}[commandchars=\\\{\}]
Help on function exec\_order in module fmdtools.resultdisp.graph:

exec\_order(mdl, renderer='matplotlib', gtype='bipartite', colors=['lightgray', 'cyan', 'teal'], show\_dyn\_order=True, title='Function Execution Order', legend=True, **kwargs)
    Displays the execution order/types of the model, where the functions and flows in the
    static step are highlighted and the functions in the dynamic step are listed (with corresponding order)

    Parameters
    ----------
    mdl : fmdtools Model
        Model of the system to visualize.
    renderer : 'matplotlib' or 'graphviz'
        Renderer to use for the graph
    gtype : 'normal'/'bipartite', optional
        Representation of the model to use. The default is 'bipartite'.
    colors : list, optional
        Colors to use for unexecuted functions, static propagation steps, and dynamic functions.
        The default is ['lightgray', 'cyan','teal'].
    show\_dyn\_order : bool, optional
        Whether to label the execution order for dynamic functions. The default is True.
    title : str, optional
        Title for the plot. The default is "Function Execution Order".
    legend : bool, optional
        Whether to show a legend. The default is True.
    **kwargs : see arguments for the respective renderers
    Returns
    -------
    tuple of form (figure, axis)

\end{sphinxVerbatim}



\end{sphinxuseclass}
\end{sphinxuseclass}
}

\end{sphinxuseclass}
\end{sphinxuseclass}
\sphinxAtStartPar
While the functions and flows updated/executed on in the dynamic propagation step can be visualized using \sphinxcode{\sphinxupquote{rd.plot.dyn\_order}}:

\begin{sphinxuseclass}{nbinput}
{
\sphinxsetup{VerbatimColor={named}{nbsphinx-code-bg}}
\sphinxsetup{VerbatimBorderColor={named}{nbsphinx-code-border}}
\begin{sphinxVerbatim}[commandchars=\\\{\}]
\llap{\color{nbsphinxin}[6]:\,\hspace{\fboxrule}\hspace{\fboxsep}}\PYG{n}{help}\PYG{p}{(}\PYG{n}{rd}\PYG{o}{.}\PYG{n}{plot}\PYG{o}{.}\PYG{n}{dyn\PYGZus{}order}\PYG{p}{)}
\end{sphinxVerbatim}
}

\end{sphinxuseclass}
\begin{sphinxuseclass}{nboutput}
\begin{sphinxuseclass}{nblast}
{

\kern-\sphinxverbatimsmallskipamount\kern-\baselineskip
\kern+\FrameHeightAdjust\kern-\fboxrule
\vspace{\nbsphinxcodecellspacing}

\sphinxsetup{VerbatimColor={named}{white}}
\sphinxsetup{VerbatimBorderColor={named}{nbsphinx-code-border}}
\begin{sphinxuseclass}{output_area}
\begin{sphinxuseclass}{}


\begin{sphinxVerbatim}[commandchars=\\\{\}]
Help on function dyn\_order in module fmdtools.resultdisp.plot:

dyn\_order(mdl, rotateticks=False, title='Dynamic Run Order')
    Plots the run order for the model during the dynamic propagation step used
    by dynamic\_behavior() methods, where the x-direction is the order of each
    function executed and the y are the corresponding flows acted on by the
    given methods.

    Parameters
    ----------
    mdl : Model
        fmdtools model
    rotateticks : Bool, optional
        Whether to rotate the x-ticks (for bigger plots). The default is False.
    title : str, optional
        String to use for the title (if any). The default is "Dynamic Run Order".

    Returns
    -------
    fig : figure
        Matplotlib figure object
    ax : axis
        Corresponding matplotlib axis

\end{sphinxVerbatim}



\end{sphinxuseclass}
\end{sphinxuseclass}
}

\end{sphinxuseclass}
\end{sphinxuseclass}
\sphinxAtStartPar
The next sections will demonstrate these functions using a simple hybrid model.


\subsubsection{Model Setup}
\label{\detokenize{docs/Model_Structure_Visualization_Tutorial:Model-Setup}}
\sphinxAtStartPar
Consider the following (highly simplified) rover electrical/navigation model. We can define the functions of this rover using the classes:

\begin{sphinxuseclass}{nbinput}
\begin{sphinxuseclass}{nblast}
{
\sphinxsetup{VerbatimColor={named}{nbsphinx-code-bg}}
\sphinxsetup{VerbatimBorderColor={named}{nbsphinx-code-border}}
\begin{sphinxVerbatim}[commandchars=\\\{\}]
\llap{\color{nbsphinxin}[7]:\,\hspace{\fboxrule}\hspace{\fboxsep}}\PYG{k}{class} \PYG{n+nc}{Control\PYGZus{}Rover}\PYG{p}{(}\PYG{n}{FxnBlock}\PYG{p}{)}\PYG{p}{:}
    \PYG{k}{def} \PYG{n+nf+fm}{\PYGZus{}\PYGZus{}init\PYGZus{}\PYGZus{}}\PYG{p}{(}\PYG{n+nb+bp}{self}\PYG{p}{,}\PYG{n}{name}\PYG{p}{,} \PYG{n}{flows}\PYG{p}{)}\PYG{p}{:}
        \PYG{n+nb}{super}\PYG{p}{(}\PYG{p}{)}\PYG{o}{.}\PYG{n+nf+fm}{\PYGZus{}\PYGZus{}init\PYGZus{}\PYGZus{}}\PYG{p}{(}\PYG{n}{name}\PYG{p}{,} \PYG{n}{flows}\PYG{p}{)}
        \PYG{n+nb+bp}{self}\PYG{o}{.}\PYG{n}{assoc\PYGZus{}modes}\PYG{p}{(}\PYG{p}{\PYGZob{}}\PYG{l+s+s1}{\PYGZsq{}}\PYG{l+s+s1}{no\PYGZus{}con}\PYG{l+s+s1}{\PYGZsq{}}\PYG{p}{:}\PYG{p}{[}\PYG{l+m+mf}{1e\PYGZhy{}4}\PYG{p}{,} \PYG{l+m+mi}{200}\PYG{p}{]}\PYG{p}{\PYGZcb{}}\PYG{p}{,} \PYG{p}{[}\PYG{l+s+s1}{\PYGZsq{}}\PYG{l+s+s1}{drive}\PYG{l+s+s1}{\PYGZsq{}}\PYG{p}{,}\PYG{l+s+s1}{\PYGZsq{}}\PYG{l+s+s1}{standby}\PYG{l+s+s1}{\PYGZsq{}}\PYG{p}{]}\PYG{p}{,} \PYG{n}{initmode}\PYG{o}{=}\PYG{l+s+s1}{\PYGZsq{}}\PYG{l+s+s1}{standby}\PYG{l+s+s1}{\PYGZsq{}}\PYG{p}{)}
    \PYG{k}{def} \PYG{n+nf}{dynamic\PYGZus{}behavior}\PYG{p}{(}\PYG{n+nb+bp}{self}\PYG{p}{,}\PYG{n}{time}\PYG{p}{)}\PYG{p}{:}
        \PYG{k}{if} \PYG{o+ow}{not} \PYG{n+nb+bp}{self}\PYG{o}{.}\PYG{n}{in\PYGZus{}mode}\PYG{p}{(}\PYG{l+s+s1}{\PYGZsq{}}\PYG{l+s+s1}{no\PYGZus{}con}\PYG{l+s+s1}{\PYGZsq{}}\PYG{p}{)}\PYG{p}{:}
            \PYG{k}{if} \PYG{n}{time} \PYG{o}{==} \PYG{l+m+mi}{5}\PYG{p}{:} \PYG{n+nb+bp}{self}\PYG{o}{.}\PYG{n}{set\PYGZus{}mode}\PYG{p}{(}\PYG{l+s+s1}{\PYGZsq{}}\PYG{l+s+s1}{drive}\PYG{l+s+s1}{\PYGZsq{}}\PYG{p}{)}
            \PYG{k}{if} \PYG{n}{time} \PYG{o}{==} \PYG{l+m+mi}{50}\PYG{p}{:} \PYG{n+nb+bp}{self}\PYG{o}{.}\PYG{n}{set\PYGZus{}mode}\PYG{p}{(}\PYG{l+s+s1}{\PYGZsq{}}\PYG{l+s+s1}{standby}\PYG{l+s+s1}{\PYGZsq{}}\PYG{p}{)}
        \PYG{k}{if} \PYG{n+nb+bp}{self}\PYG{o}{.}\PYG{n}{in\PYGZus{}mode}\PYG{p}{(}\PYG{l+s+s1}{\PYGZsq{}}\PYG{l+s+s1}{drive}\PYG{l+s+s1}{\PYGZsq{}}\PYG{p}{)}\PYG{p}{:}
            \PYG{n+nb+bp}{self}\PYG{o}{.}\PYG{n}{Control}\PYG{o}{.}\PYG{n}{power} \PYG{o}{=} \PYG{l+m+mi}{1}
            \PYG{n+nb+bp}{self}\PYG{o}{.}\PYG{n}{Control}\PYG{o}{.}\PYG{n}{vel} \PYG{o}{=} \PYG{l+m+mi}{1}
        \PYG{k}{elif} \PYG{n+nb+bp}{self}\PYG{o}{.}\PYG{n}{in\PYGZus{}mode}\PYG{p}{(}\PYG{l+s+s1}{\PYGZsq{}}\PYG{l+s+s1}{standby}\PYG{l+s+s1}{\PYGZsq{}}\PYG{p}{)}\PYG{p}{:}
            \PYG{n+nb+bp}{self}\PYG{o}{.}\PYG{n}{Control}\PYG{o}{.}\PYG{n}{vel} \PYG{o}{=} \PYG{l+m+mi}{0}
            \PYG{n+nb+bp}{self}\PYG{o}{.}\PYG{n}{Control}\PYG{o}{.}\PYG{n}{power}\PYG{o}{=}\PYG{l+m+mi}{0}
\end{sphinxVerbatim}
}

\end{sphinxuseclass}
\end{sphinxuseclass}
\sphinxAtStartPar
This function uses \sphinxcode{\sphinxupquote{dynamic\_behavior()}} to define the dynamic behavior of going through different modes depending on what model time it is. While this could also be entered in as a static behavior, because none of the defined behaviors themselves result from external inputs, there is no reason to.

\begin{sphinxuseclass}{nbinput}
\begin{sphinxuseclass}{nblast}
{
\sphinxsetup{VerbatimColor={named}{nbsphinx-code-bg}}
\sphinxsetup{VerbatimBorderColor={named}{nbsphinx-code-border}}
\begin{sphinxVerbatim}[commandchars=\\\{\}]
\llap{\color{nbsphinxin}[8]:\,\hspace{\fboxrule}\hspace{\fboxsep}}\PYG{k}{class} \PYG{n+nc}{Move\PYGZus{}Rover}\PYG{p}{(}\PYG{n}{FxnBlock}\PYG{p}{)}\PYG{p}{:}
    \PYG{k}{def} \PYG{n+nf+fm}{\PYGZus{}\PYGZus{}init\PYGZus{}\PYGZus{}}\PYG{p}{(}\PYG{n+nb+bp}{self}\PYG{p}{,}\PYG{n}{name}\PYG{p}{,} \PYG{n}{flows}\PYG{p}{)}\PYG{p}{:}
        \PYG{n+nb}{super}\PYG{p}{(}\PYG{p}{)}\PYG{o}{.}\PYG{n+nf+fm}{\PYGZus{}\PYGZus{}init\PYGZus{}\PYGZus{}}\PYG{p}{(}\PYG{n}{name}\PYG{p}{,} \PYG{n}{flows}\PYG{p}{,} \PYG{n}{flownames}\PYG{o}{=}\PYG{p}{\PYGZob{}}\PYG{l+s+s2}{\PYGZdq{}}\PYG{l+s+s2}{EE}\PYG{l+s+s2}{\PYGZdq{}}\PYG{p}{:}\PYG{l+s+s2}{\PYGZdq{}}\PYG{l+s+s2}{EE\PYGZus{}in}\PYG{l+s+s2}{\PYGZdq{}}\PYG{p}{\PYGZcb{}}\PYG{p}{,} \PYG{n}{states} \PYG{o}{=} \PYG{p}{\PYGZob{}}\PYG{l+s+s1}{\PYGZsq{}}\PYG{l+s+s1}{power}\PYG{l+s+s1}{\PYGZsq{}}\PYG{p}{:}\PYG{l+m+mi}{0}\PYG{p}{\PYGZcb{}}\PYG{p}{)}
        \PYG{n+nb+bp}{self}\PYG{o}{.}\PYG{n}{assoc\PYGZus{}modes}\PYG{p}{(}\PYG{p}{\PYGZob{}}\PYG{l+s+s2}{\PYGZdq{}}\PYG{l+s+s2}{mech\PYGZus{}loss}\PYG{l+s+s2}{\PYGZdq{}}\PYG{p}{,} \PYG{l+s+s2}{\PYGZdq{}}\PYG{l+s+s2}{elec\PYGZus{}open}\PYG{l+s+s2}{\PYGZdq{}}\PYG{p}{,} \PYG{l+s+s2}{\PYGZdq{}}\PYG{l+s+s2}{short}\PYG{l+s+s2}{\PYGZdq{}}\PYG{p}{\PYGZcb{}}\PYG{p}{)}
    \PYG{k}{def} \PYG{n+nf}{static\PYGZus{}behavior}\PYG{p}{(}\PYG{n+nb+bp}{self}\PYG{p}{,} \PYG{n}{time}\PYG{p}{)}\PYG{p}{:}
        \PYG{n+nb+bp}{self}\PYG{o}{.}\PYG{n}{power} \PYG{o}{=} \PYG{n+nb+bp}{self}\PYG{o}{.}\PYG{n}{EE\PYGZus{}in}\PYG{o}{.}\PYG{n}{v\PYGZus{}supply} \PYG{o}{*} \PYG{n+nb+bp}{self}\PYG{o}{.}\PYG{n}{Control}\PYG{o}{.}\PYG{n}{vel} \PYG{o}{*}\PYG{n+nb+bp}{self}\PYG{o}{.}\PYG{n}{no\PYGZus{}fault}\PYG{p}{(}\PYG{l+s+s2}{\PYGZdq{}}\PYG{l+s+s2}{elec\PYGZus{}open}\PYG{l+s+s2}{\PYGZdq{}}\PYG{p}{)}
        \PYG{n+nb+bp}{self}\PYG{o}{.}\PYG{n}{EE\PYGZus{}in}\PYG{o}{.}\PYG{n}{a\PYGZus{}supply} \PYG{o}{=} \PYG{n+nb+bp}{self}\PYG{o}{.}\PYG{n}{power}\PYG{o}{/}\PYG{p}{(}\PYG{l+m+mi}{12}\PYG{o}{*}\PYG{p}{(}\PYG{n+nb+bp}{self}\PYG{o}{.}\PYG{n}{no\PYGZus{}fault}\PYG{p}{(}\PYG{l+s+s1}{\PYGZsq{}}\PYG{l+s+s1}{short}\PYG{l+s+s1}{\PYGZsq{}}\PYG{p}{)}\PYG{o}{+}\PYG{l+m+mf}{0.001}\PYG{p}{)}\PYG{p}{)}
        \PYG{k}{if} \PYG{n+nb+bp}{self}\PYG{o}{.}\PYG{n}{power} \PYG{o}{\PYGZgt{}}\PYG{l+m+mi}{100}\PYG{p}{:} \PYG{n+nb+bp}{self}\PYG{o}{.}\PYG{n}{add\PYGZus{}fault}\PYG{p}{(}\PYG{l+s+s2}{\PYGZdq{}}\PYG{l+s+s2}{elec\PYGZus{}open}\PYG{l+s+s2}{\PYGZdq{}}\PYG{p}{)}
    \PYG{k}{def} \PYG{n+nf}{dynamic\PYGZus{}behavior}\PYG{p}{(}\PYG{n+nb+bp}{self}\PYG{p}{,} \PYG{n}{time}\PYG{p}{)}\PYG{p}{:}
        \PYG{k}{if} \PYG{o+ow}{not} \PYG{n+nb+bp}{self}\PYG{o}{.}\PYG{n}{has\PYGZus{}faults}\PYG{p}{(}\PYG{p}{[}\PYG{l+s+s2}{\PYGZdq{}}\PYG{l+s+s2}{elec\PYGZus{}open}\PYG{l+s+s2}{\PYGZdq{}}\PYG{p}{,} \PYG{l+s+s2}{\PYGZdq{}}\PYG{l+s+s2}{mech\PYGZus{}loss}\PYG{l+s+s2}{\PYGZdq{}}\PYG{p}{]}\PYG{p}{)}\PYG{p}{:}
            \PYG{n+nb+bp}{self}\PYG{o}{.}\PYG{n}{Ground}\PYG{o}{.}\PYG{n}{x} \PYG{o}{=} \PYG{n+nb+bp}{self}\PYG{o}{.}\PYG{n}{Ground}\PYG{o}{.}\PYG{n}{x} \PYG{o}{+} \PYG{n+nb+bp}{self}\PYG{o}{.}\PYG{n}{power}\PYG{o}{*}\PYG{n+nb+bp}{self}\PYG{o}{.}\PYG{n}{no\PYGZus{}fault}\PYG{p}{(}\PYG{l+s+s2}{\PYGZdq{}}\PYG{l+s+s2}{mech\PYGZus{}loss}\PYG{l+s+s2}{\PYGZdq{}}\PYG{p}{)}
\end{sphinxVerbatim}
}

\end{sphinxuseclass}
\end{sphinxuseclass}
\sphinxAtStartPar
The \sphinxcode{\sphinxupquote{Move\_Rover}} function uses both: \sphinxhyphen{} a static behavior which defines the input/output of electrical power at each instant, and \sphinxhyphen{} a dynamic behavior which defines the movement of the rover over time

\sphinxAtStartPar
In this instance, the static behavior is important for enabling faults to propagate instantaneously in a single time\sphinxhyphen{}step (in this case, a short causing high current load to the battery).

\begin{sphinxuseclass}{nbinput}
\begin{sphinxuseclass}{nblast}
{
\sphinxsetup{VerbatimColor={named}{nbsphinx-code-bg}}
\sphinxsetup{VerbatimBorderColor={named}{nbsphinx-code-border}}
\begin{sphinxVerbatim}[commandchars=\\\{\}]
\llap{\color{nbsphinxin}[9]:\,\hspace{\fboxrule}\hspace{\fboxsep}}\PYG{k}{class} \PYG{n+nc}{Store\PYGZus{}Energy}\PYG{p}{(}\PYG{n}{FxnBlock}\PYG{p}{)}\PYG{p}{:}
    \PYG{k}{def} \PYG{n+nf+fm}{\PYGZus{}\PYGZus{}init\PYGZus{}\PYGZus{}}\PYG{p}{(}\PYG{n+nb+bp}{self}\PYG{p}{,} \PYG{n}{name}\PYG{p}{,} \PYG{n}{flows}\PYG{p}{)}\PYG{p}{:}
        \PYG{n+nb}{super}\PYG{p}{(}\PYG{p}{)}\PYG{o}{.}\PYG{n+nf+fm}{\PYGZus{}\PYGZus{}init\PYGZus{}\PYGZus{}}\PYG{p}{(}\PYG{n}{name}\PYG{p}{,}\PYG{n}{flows}\PYG{p}{,} \PYG{n}{states}\PYG{o}{=}\PYG{p}{\PYGZob{}}\PYG{l+s+s2}{\PYGZdq{}}\PYG{l+s+s2}{charge}\PYG{l+s+s2}{\PYGZdq{}}\PYG{p}{:} \PYG{l+m+mi}{100}\PYG{p}{\PYGZcb{}}\PYG{p}{)}
        \PYG{n+nb+bp}{self}\PYG{o}{.}\PYG{n}{assoc\PYGZus{}modes}\PYG{p}{(}\PYG{p}{\PYGZob{}}\PYG{l+s+s2}{\PYGZdq{}}\PYG{l+s+s2}{no\PYGZus{}charge}\PYG{l+s+s2}{\PYGZdq{}}\PYG{p}{:}\PYG{p}{[}\PYG{l+m+mf}{1e\PYGZhy{}5}\PYG{p}{,} \PYG{p}{\PYGZob{}}\PYG{l+s+s1}{\PYGZsq{}}\PYG{l+s+s1}{standby}\PYG{l+s+s1}{\PYGZsq{}}\PYG{p}{:}\PYG{l+m+mf}{1.0}\PYG{p}{\PYGZcb{}}\PYG{p}{,} \PYG{l+m+mi}{100}\PYG{p}{]}\PYG{p}{,}\PYG{l+s+s2}{\PYGZdq{}}\PYG{l+s+s2}{short}\PYG{l+s+s2}{\PYGZdq{}}\PYG{p}{:}\PYG{p}{[}\PYG{l+m+mf}{1e\PYGZhy{}5}\PYG{p}{,} \PYG{p}{\PYGZob{}}\PYG{l+s+s1}{\PYGZsq{}}\PYG{l+s+s1}{supply}\PYG{l+s+s1}{\PYGZsq{}}\PYG{p}{:}\PYG{l+m+mf}{1.0}\PYG{p}{\PYGZcb{}}\PYG{p}{,} \PYG{l+m+mi}{100}\PYG{p}{]}\PYG{p}{,}\PYG{p}{\PYGZcb{}}\PYG{p}{,} \PYG{p}{[}\PYG{l+s+s2}{\PYGZdq{}}\PYG{l+s+s2}{supply}\PYG{l+s+s2}{\PYGZdq{}}\PYG{p}{,}\PYG{l+s+s2}{\PYGZdq{}}\PYG{l+s+s2}{charge}\PYG{l+s+s2}{\PYGZdq{}}\PYG{p}{,}\PYG{l+s+s2}{\PYGZdq{}}\PYG{l+s+s2}{standby}\PYG{l+s+s2}{\PYGZdq{}}\PYG{p}{]}\PYG{p}{,} \PYG{n}{initmode}\PYG{o}{=}\PYG{l+s+s2}{\PYGZdq{}}\PYG{l+s+s2}{standby}\PYG{l+s+s2}{\PYGZdq{}}\PYG{p}{,} \PYG{n}{exclusive} \PYG{o}{=} \PYG{k+kc}{True}\PYG{p}{,} \PYG{n}{key\PYGZus{}phases\PYGZus{}by}\PYG{o}{=}\PYG{l+s+s1}{\PYGZsq{}}\PYG{l+s+s1}{self}\PYG{l+s+s1}{\PYGZsq{}}\PYG{p}{)}
    \PYG{k}{def} \PYG{n+nf}{static\PYGZus{}behavior}\PYG{p}{(}\PYG{n+nb+bp}{self}\PYG{p}{,}\PYG{n}{time}\PYG{p}{)}\PYG{p}{:}
        \PYG{k}{if} \PYG{n+nb+bp}{self}\PYG{o}{.}\PYG{n}{EE}\PYG{o}{.}\PYG{n}{a\PYGZus{}supply} \PYG{o}{\PYGZgt{}} \PYG{l+m+mi}{5}\PYG{p}{:} \PYG{n+nb+bp}{self}\PYG{o}{.}\PYG{n}{add\PYGZus{}fault}\PYG{p}{(}\PYG{l+s+s2}{\PYGZdq{}}\PYG{l+s+s2}{no\PYGZus{}charge}\PYG{l+s+s2}{\PYGZdq{}}\PYG{p}{)}
    \PYG{k}{def} \PYG{n+nf}{dynamic\PYGZus{}behavior}\PYG{p}{(}\PYG{n+nb+bp}{self}\PYG{p}{,}\PYG{n}{time}\PYG{p}{)}\PYG{p}{:}
        \PYG{k}{if} \PYG{n+nb+bp}{self}\PYG{o}{.}\PYG{n}{in\PYGZus{}mode}\PYG{p}{(}\PYG{l+s+s2}{\PYGZdq{}}\PYG{l+s+s2}{standby}\PYG{l+s+s2}{\PYGZdq{}}\PYG{p}{)}\PYG{p}{:}
            \PYG{n+nb+bp}{self}\PYG{o}{.}\PYG{n}{EE}\PYG{o}{.}\PYG{n}{v\PYGZus{}supply} \PYG{o}{=} \PYG{l+m+mi}{0}\PYG{p}{;} \PYG{n+nb+bp}{self}\PYG{o}{.}\PYG{n}{EE}\PYG{o}{.}\PYG{n}{a\PYGZus{}supply} \PYG{o}{=} \PYG{l+m+mi}{0}
            \PYG{k}{if} \PYG{n+nb+bp}{self}\PYG{o}{.}\PYG{n}{Control}\PYG{o}{.}\PYG{n}{power}\PYG{o}{==}\PYG{l+m+mi}{1}\PYG{p}{:} \PYG{n+nb+bp}{self}\PYG{o}{.}\PYG{n}{set\PYGZus{}mode}\PYG{p}{(}\PYG{l+s+s2}{\PYGZdq{}}\PYG{l+s+s2}{supply}\PYG{l+s+s2}{\PYGZdq{}}\PYG{p}{)}
        \PYG{k}{elif} \PYG{n+nb+bp}{self}\PYG{o}{.}\PYG{n}{in\PYGZus{}mode}\PYG{p}{(}\PYG{l+s+s2}{\PYGZdq{}}\PYG{l+s+s2}{charge}\PYG{l+s+s2}{\PYGZdq{}}\PYG{p}{)}\PYG{p}{:}
            \PYG{n+nb+bp}{self}\PYG{o}{.}\PYG{n}{EE}\PYG{o}{.}\PYG{n}{charge} \PYG{o}{=}\PYG{n+nb}{min}\PYG{p}{(}\PYG{n+nb+bp}{self}\PYG{o}{.}\PYG{n}{EE}\PYG{o}{.}\PYG{n}{charge}\PYG{o}{+}\PYG{n+nb+bp}{self}\PYG{o}{.}\PYG{n}{tstep}\PYG{p}{,} \PYG{l+m+mi}{20}\PYG{p}{)}
        \PYG{k}{elif} \PYG{n+nb+bp}{self}\PYG{o}{.}\PYG{n}{in\PYGZus{}mode}\PYG{p}{(}\PYG{l+s+s2}{\PYGZdq{}}\PYG{l+s+s2}{supply}\PYG{l+s+s2}{\PYGZdq{}}\PYG{p}{)}\PYG{p}{:}
            \PYG{k}{if} \PYG{n+nb+bp}{self}\PYG{o}{.}\PYG{n}{charge} \PYG{o}{\PYGZgt{}} \PYG{l+m+mi}{0}\PYG{p}{:}         \PYG{n+nb+bp}{self}\PYG{o}{.}\PYG{n}{EE}\PYG{o}{.}\PYG{n}{v\PYGZus{}supply} \PYG{o}{=} \PYG{l+m+mi}{12}\PYG{p}{;} \PYG{n+nb+bp}{self}\PYG{o}{.}\PYG{n}{charge} \PYG{o}{\PYGZhy{}}\PYG{o}{=} \PYG{n+nb+bp}{self}\PYG{o}{.}\PYG{n}{tstep}
            \PYG{k}{else}\PYG{p}{:} \PYG{n+nb+bp}{self}\PYG{o}{.}\PYG{n}{set\PYGZus{}mode}\PYG{p}{(}\PYG{l+s+s2}{\PYGZdq{}}\PYG{l+s+s2}{no\PYGZus{}charge}\PYG{l+s+s2}{\PYGZdq{}}\PYG{p}{)}
            \PYG{k}{if} \PYG{n+nb+bp}{self}\PYG{o}{.}\PYG{n}{Control}\PYG{o}{.}\PYG{n}{power}\PYG{o}{==}\PYG{l+m+mi}{0}\PYG{p}{:} \PYG{n+nb+bp}{self}\PYG{o}{.}\PYG{n}{set\PYGZus{}mode}\PYG{p}{(}\PYG{l+s+s2}{\PYGZdq{}}\PYG{l+s+s2}{standby}\PYG{l+s+s2}{\PYGZdq{}}\PYG{p}{)}
        \PYG{k}{elif} \PYG{n+nb+bp}{self}\PYG{o}{.}\PYG{n}{in\PYGZus{}mode}\PYG{p}{(}\PYG{l+s+s2}{\PYGZdq{}}\PYG{l+s+s2}{short}\PYG{l+s+s2}{\PYGZdq{}}\PYG{p}{)}\PYG{p}{:}     \PYG{n+nb+bp}{self}\PYG{o}{.}\PYG{n}{EE}\PYG{o}{.}\PYG{n}{v\PYGZus{}supply} \PYG{o}{=} \PYG{l+m+mi}{100}\PYG{p}{;} \PYG{n+nb+bp}{self}\PYG{o}{.}\PYG{n}{charge} \PYG{o}{=} \PYG{l+m+mi}{0}
        \PYG{k}{elif} \PYG{n+nb+bp}{self}\PYG{o}{.}\PYG{n}{in\PYGZus{}mode}\PYG{p}{(}\PYG{l+s+s2}{\PYGZdq{}}\PYG{l+s+s2}{no\PYGZus{}charge}\PYG{l+s+s2}{\PYGZdq{}}\PYG{p}{)}\PYG{p}{:} \PYG{n+nb+bp}{self}\PYG{o}{.}\PYG{n}{EE}\PYG{o}{.}\PYG{n}{v\PYGZus{}supply}\PYG{o}{=}\PYG{l+m+mi}{0}\PYG{p}{;} \PYG{n+nb+bp}{self}\PYG{o}{.}\PYG{n}{charge} \PYG{o}{=} \PYG{l+m+mi}{0}
\end{sphinxVerbatim}
}

\end{sphinxuseclass}
\end{sphinxuseclass}
\sphinxAtStartPar
The \sphinxcode{\sphinxupquote{Store\_Energy}} function has both a static behavior and a dynamic behavior. In this case, the static behavior enables the propagation of an adverse current from the drive system to damage the battery instantaneously, (instead of over several timesteps).

\begin{sphinxuseclass}{nbinput}
\begin{sphinxuseclass}{nblast}
{
\sphinxsetup{VerbatimColor={named}{nbsphinx-code-bg}}
\sphinxsetup{VerbatimBorderColor={named}{nbsphinx-code-border}}
\begin{sphinxVerbatim}[commandchars=\\\{\}]
\llap{\color{nbsphinxin}[10]:\,\hspace{\fboxrule}\hspace{\fboxsep}}\PYG{k}{class} \PYG{n+nc}{Rover}\PYG{p}{(}\PYG{n}{Model}\PYG{p}{)}\PYG{p}{:}
    \PYG{k}{def} \PYG{n+nf+fm}{\PYGZus{}\PYGZus{}init\PYGZus{}\PYGZus{}}\PYG{p}{(}\PYG{n+nb+bp}{self}\PYG{p}{,} \PYG{n}{params}\PYG{o}{=}\PYG{p}{\PYGZob{}}\PYG{p}{\PYGZcb{}}\PYG{p}{,}\PYGZbs{}
                 \PYG{n}{modelparams}\PYG{o}{=}\PYG{p}{\PYGZob{}}\PYG{l+s+s1}{\PYGZsq{}}\PYG{l+s+s1}{times}\PYG{l+s+s1}{\PYGZsq{}}\PYG{p}{:}\PYG{p}{[}\PYG{l+m+mi}{0}\PYG{p}{,}\PYG{l+m+mi}{60}\PYG{p}{]}\PYG{p}{,} \PYG{l+s+s1}{\PYGZsq{}}\PYG{l+s+s1}{tstep}\PYG{l+s+s1}{\PYGZsq{}}\PYG{p}{:}\PYG{l+m+mi}{1}\PYG{p}{,} \PYG{l+s+s1}{\PYGZsq{}}\PYG{l+s+s1}{phases}\PYG{l+s+s1}{\PYGZsq{}}\PYG{p}{:}\PYG{p}{\PYGZob{}}\PYG{l+s+s1}{\PYGZsq{}}\PYG{l+s+s1}{start}\PYG{l+s+s1}{\PYGZsq{}}\PYG{p}{:}\PYG{p}{[}\PYG{l+m+mi}{1}\PYG{p}{,}\PYG{l+m+mi}{30}\PYG{p}{]}\PYG{p}{,} \PYG{l+s+s1}{\PYGZsq{}}\PYG{l+s+s1}{end}\PYG{l+s+s1}{\PYGZsq{}}\PYG{p}{:}\PYG{p}{[}\PYG{l+m+mi}{31}\PYG{p}{,} \PYG{l+m+mi}{60}\PYG{p}{]}\PYG{p}{\PYGZcb{}}\PYG{p}{\PYGZcb{}}\PYG{p}{,}\PYGZbs{}
                     \PYG{n}{valparams}\PYG{o}{=}\PYG{p}{\PYGZob{}}\PYG{p}{\PYGZcb{}}\PYG{p}{)}\PYG{p}{:}
        \PYG{n+nb}{super}\PYG{p}{(}\PYG{p}{)}\PYG{o}{.}\PYG{n+nf+fm}{\PYGZus{}\PYGZus{}init\PYGZus{}\PYGZus{}}\PYG{p}{(}\PYG{n}{params}\PYG{p}{,} \PYG{n}{modelparams}\PYG{p}{,} \PYG{n}{valparams}\PYG{p}{)}

        \PYG{n+nb+bp}{self}\PYG{o}{.}\PYG{n}{add\PYGZus{}flow}\PYG{p}{(}\PYG{l+s+s1}{\PYGZsq{}}\PYG{l+s+s1}{Ground}\PYG{l+s+s1}{\PYGZsq{}}\PYG{p}{,} \PYG{p}{\PYGZob{}}\PYG{l+s+s1}{\PYGZsq{}}\PYG{l+s+s1}{x}\PYG{l+s+s1}{\PYGZsq{}}\PYG{p}{:}\PYG{l+m+mi}{0}\PYG{p}{,}\PYG{l+s+s1}{\PYGZsq{}}\PYG{l+s+s1}{y}\PYG{l+s+s1}{\PYGZsq{}}\PYG{p}{:}\PYG{l+m+mi}{0}\PYG{p}{,} \PYG{l+s+s1}{\PYGZsq{}}\PYG{l+s+s1}{dir}\PYG{l+s+s1}{\PYGZsq{}}\PYG{p}{:}\PYG{l+m+mi}{0}\PYG{p}{,} \PYG{l+s+s1}{\PYGZsq{}}\PYG{l+s+s1}{vel}\PYG{l+s+s1}{\PYGZsq{}}\PYG{p}{:}\PYG{l+m+mi}{0}\PYG{p}{\PYGZcb{}}\PYG{p}{)}
        \PYG{n+nb+bp}{self}\PYG{o}{.}\PYG{n}{add\PYGZus{}flow}\PYG{p}{(}\PYG{l+s+s1}{\PYGZsq{}}\PYG{l+s+s1}{Force}\PYG{l+s+s1}{\PYGZsq{}}\PYG{p}{,} \PYG{p}{\PYGZob{}}\PYG{l+s+s1}{\PYGZsq{}}\PYG{l+s+s1}{transfer}\PYG{l+s+s1}{\PYGZsq{}}\PYG{p}{:}\PYG{l+m+mi}{1}\PYG{p}{,} \PYG{l+s+s1}{\PYGZsq{}}\PYG{l+s+s1}{magnitude}\PYG{l+s+s1}{\PYGZsq{}}\PYG{p}{:}\PYG{l+m+mi}{1}\PYG{p}{\PYGZcb{}}\PYG{p}{)}
        \PYG{n+nb+bp}{self}\PYG{o}{.}\PYG{n}{add\PYGZus{}flow}\PYG{p}{(}\PYG{l+s+s1}{\PYGZsq{}}\PYG{l+s+s1}{EE}\PYG{l+s+s1}{\PYGZsq{}}\PYG{p}{,} \PYG{p}{\PYGZob{}}\PYG{l+s+s1}{\PYGZsq{}}\PYG{l+s+s1}{v\PYGZus{}supply}\PYG{l+s+s1}{\PYGZsq{}}\PYG{p}{:}\PYG{l+m+mi}{0}\PYG{p}{,} \PYG{l+s+s1}{\PYGZsq{}}\PYG{l+s+s1}{a\PYGZus{}supply}\PYG{l+s+s1}{\PYGZsq{}}\PYG{p}{:}\PYG{l+m+mi}{0}\PYG{p}{\PYGZcb{}}\PYG{p}{)}
        \PYG{n+nb+bp}{self}\PYG{o}{.}\PYG{n}{add\PYGZus{}flow}\PYG{p}{(}\PYG{l+s+s1}{\PYGZsq{}}\PYG{l+s+s1}{Video}\PYG{l+s+s1}{\PYGZsq{}}\PYG{p}{,} \PYG{p}{\PYGZob{}}\PYG{l+s+s1}{\PYGZsq{}}\PYG{l+s+s1}{line}\PYG{l+s+s1}{\PYGZsq{}}\PYG{p}{:}\PYG{l+m+mi}{0}\PYG{p}{,} \PYG{l+s+s1}{\PYGZsq{}}\PYG{l+s+s1}{angle}\PYG{l+s+s1}{\PYGZsq{}}\PYG{p}{:}\PYG{l+m+mi}{0}\PYG{p}{\PYGZcb{}}\PYG{p}{)}
        \PYG{n+nb+bp}{self}\PYG{o}{.}\PYG{n}{add\PYGZus{}flow}\PYG{p}{(}\PYG{l+s+s1}{\PYGZsq{}}\PYG{l+s+s1}{Control}\PYG{l+s+s1}{\PYGZsq{}}\PYG{p}{,} \PYG{p}{\PYGZob{}}\PYG{l+s+s1}{\PYGZsq{}}\PYG{l+s+s1}{dir}\PYG{l+s+s1}{\PYGZsq{}}\PYG{p}{:}\PYG{l+m+mi}{0}\PYG{p}{,} \PYG{l+s+s1}{\PYGZsq{}}\PYG{l+s+s1}{vel}\PYG{l+s+s1}{\PYGZsq{}}\PYG{p}{:}\PYG{l+m+mi}{0}\PYG{p}{,} \PYG{l+s+s1}{\PYGZsq{}}\PYG{l+s+s1}{power}\PYG{l+s+s1}{\PYGZsq{}}\PYG{p}{:}\PYG{l+m+mi}{0}\PYG{p}{\PYGZcb{}}\PYG{p}{)}
        \PYG{n+nb+bp}{self}\PYG{o}{.}\PYG{n}{add\PYGZus{}flow}\PYG{p}{(}\PYG{l+s+s1}{\PYGZsq{}}\PYG{l+s+s1}{Comms}\PYG{l+s+s1}{\PYGZsq{}}\PYG{p}{)} \PYG{c+c1}{\PYGZsh{}\PYGZob{}\PYGZsq{}x\PYGZsq{}:0,\PYGZsq{}y\PYGZsq{}:0\PYGZcb{}}

        \PYG{n+nb+bp}{self}\PYG{o}{.}\PYG{n}{add\PYGZus{}fxn}\PYG{p}{(}\PYG{l+s+s2}{\PYGZdq{}}\PYG{l+s+s2}{Control\PYGZus{}Rover}\PYG{l+s+s2}{\PYGZdq{}}\PYG{p}{,}\PYG{p}{[}\PYG{l+s+s2}{\PYGZdq{}}\PYG{l+s+s2}{Video}\PYG{l+s+s2}{\PYGZdq{}}\PYG{p}{,}\PYG{l+s+s2}{\PYGZdq{}}\PYG{l+s+s2}{Comms}\PYG{l+s+s2}{\PYGZdq{}}\PYG{p}{,} \PYG{l+s+s2}{\PYGZdq{}}\PYG{l+s+s2}{EE}\PYG{l+s+s2}{\PYGZdq{}}\PYG{p}{,} \PYG{l+s+s2}{\PYGZdq{}}\PYG{l+s+s2}{Control}\PYG{l+s+s2}{\PYGZdq{}}\PYG{p}{,}\PYG{l+s+s2}{\PYGZdq{}}\PYG{l+s+s2}{Force}\PYG{l+s+s2}{\PYGZdq{}}\PYG{p}{]}\PYG{p}{,} \PYG{n}{fclass}\PYG{o}{=}\PYG{n}{Control\PYGZus{}Rover}\PYG{p}{)}
        \PYG{n+nb+bp}{self}\PYG{o}{.}\PYG{n}{add\PYGZus{}fxn}\PYG{p}{(}\PYG{l+s+s2}{\PYGZdq{}}\PYG{l+s+s2}{Store\PYGZus{}Energy}\PYG{l+s+s2}{\PYGZdq{}}\PYG{p}{,} \PYG{p}{[}\PYG{l+s+s2}{\PYGZdq{}}\PYG{l+s+s2}{EE}\PYG{l+s+s2}{\PYGZdq{}}\PYG{p}{,} \PYG{l+s+s2}{\PYGZdq{}}\PYG{l+s+s2}{Force}\PYG{l+s+s2}{\PYGZdq{}}\PYG{p}{,} \PYG{l+s+s2}{\PYGZdq{}}\PYG{l+s+s2}{Control}\PYG{l+s+s2}{\PYGZdq{}}\PYG{p}{]}\PYG{p}{,} \PYG{n}{Store\PYGZus{}Energy}\PYG{p}{)}
        \PYG{n+nb+bp}{self}\PYG{o}{.}\PYG{n}{add\PYGZus{}fxn}\PYG{p}{(}\PYG{l+s+s2}{\PYGZdq{}}\PYG{l+s+s2}{Move\PYGZus{}Rover}\PYG{l+s+s2}{\PYGZdq{}}\PYG{p}{,} \PYG{p}{[}\PYG{l+s+s2}{\PYGZdq{}}\PYG{l+s+s2}{Ground}\PYG{l+s+s2}{\PYGZdq{}}\PYG{p}{,}\PYG{l+s+s2}{\PYGZdq{}}\PYG{l+s+s2}{Force}\PYG{l+s+s2}{\PYGZdq{}}\PYG{p}{,}\PYG{l+s+s2}{\PYGZdq{}}\PYG{l+s+s2}{EE}\PYG{l+s+s2}{\PYGZdq{}}\PYG{p}{,} \PYG{l+s+s2}{\PYGZdq{}}\PYG{l+s+s2}{Control}\PYG{l+s+s2}{\PYGZdq{}}\PYG{p}{]}\PYG{p}{,} \PYG{n}{fclass} \PYG{o}{=} \PYG{n}{Move\PYGZus{}Rover}\PYG{p}{)}
        \PYG{n+nb+bp}{self}\PYG{o}{.}\PYG{n}{add\PYGZus{}fxn}\PYG{p}{(}\PYG{l+s+s2}{\PYGZdq{}}\PYG{l+s+s2}{View\PYGZus{}Ground}\PYG{l+s+s2}{\PYGZdq{}}\PYG{p}{,} \PYG{p}{[}\PYG{l+s+s2}{\PYGZdq{}}\PYG{l+s+s2}{Ground}\PYG{l+s+s2}{\PYGZdq{}}\PYG{p}{,} \PYG{l+s+s2}{\PYGZdq{}}\PYG{l+s+s2}{EE}\PYG{l+s+s2}{\PYGZdq{}}\PYG{p}{,} \PYG{l+s+s2}{\PYGZdq{}}\PYG{l+s+s2}{Video}\PYG{l+s+s2}{\PYGZdq{}}\PYG{p}{,}\PYG{l+s+s2}{\PYGZdq{}}\PYG{l+s+s2}{Force}\PYG{l+s+s2}{\PYGZdq{}}\PYG{p}{]}\PYG{p}{)}
        \PYG{n+nb+bp}{self}\PYG{o}{.}\PYG{n}{add\PYGZus{}fxn}\PYG{p}{(}\PYG{l+s+s2}{\PYGZdq{}}\PYG{l+s+s2}{Communicate\PYGZus{}Externally}\PYG{l+s+s2}{\PYGZdq{}}\PYG{p}{,} \PYG{p}{[}\PYG{l+s+s2}{\PYGZdq{}}\PYG{l+s+s2}{Comms}\PYG{l+s+s2}{\PYGZdq{}}\PYG{p}{,} \PYG{l+s+s2}{\PYGZdq{}}\PYG{l+s+s2}{EE}\PYG{l+s+s2}{\PYGZdq{}}\PYG{p}{,}\PYG{l+s+s2}{\PYGZdq{}}\PYG{l+s+s2}{Force}\PYG{l+s+s2}{\PYGZdq{}}\PYG{p}{]}\PYG{p}{)}

        \PYG{n}{pos\PYGZus{}bip} \PYG{o}{=} \PYG{p}{\PYGZob{}}\PYG{l+s+s1}{\PYGZsq{}}\PYG{l+s+s1}{Control\PYGZus{}Rover}\PYG{l+s+s1}{\PYGZsq{}}\PYG{p}{:} \PYG{p}{[}\PYG{o}{\PYGZhy{}}\PYG{l+m+mf}{0.017014983401385075}\PYG{p}{,} \PYG{l+m+mf}{0.8197778602536954}\PYG{p}{]}\PYG{p}{,}
 \PYG{l+s+s1}{\PYGZsq{}}\PYG{l+s+s1}{Move\PYGZus{}Rover}\PYG{l+s+s1}{\PYGZsq{}}\PYG{p}{:} \PYG{p}{[}\PYG{l+m+mf}{0.1943738434915952}\PYG{p}{,} \PYG{o}{\PYGZhy{}}\PYG{l+m+mf}{0.5118219332727401}\PYG{p}{]}\PYG{p}{,}
 \PYG{l+s+s1}{\PYGZsq{}}\PYG{l+s+s1}{Store\PYGZus{}Energy}\PYG{l+s+s1}{\PYGZsq{}}\PYG{p}{:} \PYG{p}{[}\PYG{o}{\PYGZhy{}}\PYG{l+m+mf}{0.256309000069049}\PYG{p}{,} \PYG{o}{\PYGZhy{}}\PYG{l+m+mf}{0.004117688709924516}\PYG{p}{]}\PYG{p}{,}
 \PYG{l+s+s1}{\PYGZsq{}}\PYG{l+s+s1}{View\PYGZus{}Ground}\PYG{l+s+s1}{\PYGZsq{}}\PYG{p}{:} \PYG{p}{[}\PYG{o}{\PYGZhy{}}\PYG{l+m+mf}{0.7869889764273651}\PYG{p}{,} \PYG{l+m+mf}{0.47147713497270827}\PYG{p}{]}\PYG{p}{,}
 \PYG{l+s+s1}{\PYGZsq{}}\PYG{l+s+s1}{Communicate\PYGZus{}Externally}\PYG{l+s+s1}{\PYGZsq{}}\PYG{p}{:} \PYG{p}{[}\PYG{l+m+mf}{0.5107674237596388}\PYG{p}{,} \PYG{l+m+mf}{0.4117119127760298}\PYG{p}{]}\PYG{p}{,}
 \PYG{l+s+s1}{\PYGZsq{}}\PYG{l+s+s1}{Ground}\PYG{l+s+s1}{\PYGZsq{}}\PYG{p}{:} \PYG{p}{[}\PYG{o}{\PYGZhy{}}\PYG{l+m+mf}{0.7803536309752367}\PYG{p}{,} \PYG{o}{\PYGZhy{}}\PYG{l+m+mf}{0.4502200140852195}\PYG{p}{]}\PYG{p}{,}
 \PYG{l+s+s1}{\PYGZsq{}}\PYG{l+s+s1}{Force}\PYG{l+s+s1}{\PYGZsq{}}\PYG{p}{:} \PYG{p}{[}\PYG{l+m+mf}{0.4327741966569625}\PYG{p}{,} \PYG{l+m+mf}{0.13966361395868865}\PYG{p}{]}\PYG{p}{,}
 \PYG{l+s+s1}{\PYGZsq{}}\PYG{l+s+s1}{EE}\PYG{l+s+s1}{\PYGZsq{}}\PYG{p}{:} \PYG{p}{[}\PYG{o}{\PYGZhy{}}\PYG{l+m+mf}{0.6981138376424448}\PYG{p}{,} \PYG{l+m+mf}{0.13829658866345518}\PYG{p}{]}\PYG{p}{,}
 \PYG{l+s+s1}{\PYGZsq{}}\PYG{l+s+s1}{Video}\PYG{l+s+s1}{\PYGZsq{}}\PYG{p}{:} \PYG{p}{[}\PYG{o}{\PYGZhy{}}\PYG{l+m+mf}{0.49486453723245205}\PYG{p}{,} \PYG{l+m+mf}{0.698244546263499}\PYG{p}{]}\PYG{p}{,}
 \PYG{l+s+s1}{\PYGZsq{}}\PYG{l+s+s1}{Control}\PYG{l+s+s1}{\PYGZsq{}}\PYG{p}{:} \PYG{p}{[}\PYG{l+m+mf}{0.11615283552311584}\PYG{p}{,} \PYG{o}{\PYGZhy{}}\PYG{l+m+mf}{0.1842023746850714}\PYG{p}{]}\PYG{p}{,}
 \PYG{l+s+s1}{\PYGZsq{}}\PYG{l+s+s1}{Comms}\PYG{l+s+s1}{\PYGZsq{}}\PYG{p}{:} \PYG{p}{[}\PYG{l+m+mf}{0.3373143873188402}\PYG{p}{,} \PYG{l+m+mf}{0.6507526319915691}\PYG{p}{]}\PYG{p}{\PYGZcb{}}

        \PYG{n+nb+bp}{self}\PYG{o}{.}\PYG{n}{build\PYGZus{}model}\PYG{p}{(}\PYG{n}{bipartite\PYGZus{}pos} \PYG{o}{=} \PYG{n}{pos\PYGZus{}bip}\PYG{p}{)}
\end{sphinxVerbatim}
}

\end{sphinxuseclass}
\end{sphinxuseclass}

\paragraph{Graph Visualization}
\label{\detokenize{docs/Model_Structure_Visualization_Tutorial:Graph-Visualization}}
\sphinxAtStartPar
Without defining anything about the simulation itself, the containment relationships between the model structures can be visualized using the \sphinxcode{\sphinxupquote{gtype='hierarchical'}} option in \sphinxcode{\sphinxupquote{rd.graph.show}}, \sphinxcode{\sphinxupquote{rd.graph.show\_pyvis}}, and \sphinxcode{\sphinxupquote{rd.graph.set\_pos}}.

\begin{sphinxuseclass}{nbinput}
{
\sphinxsetup{VerbatimColor={named}{nbsphinx-code-bg}}
\sphinxsetup{VerbatimBorderColor={named}{nbsphinx-code-border}}
\begin{sphinxVerbatim}[commandchars=\\\{\}]
\llap{\color{nbsphinxin}[15]:\,\hspace{\fboxrule}\hspace{\fboxsep}}\PYG{n}{mdl} \PYG{o}{=} \PYG{n}{Rover}\PYG{p}{(}\PYG{p}{)}
\PYG{n}{rd}\PYG{o}{.}\PYG{n}{graph}\PYG{o}{.}\PYG{n}{show}\PYG{p}{(}\PYG{n}{mdl}\PYG{p}{,} \PYG{n}{gtype}\PYG{o}{=}\PYG{l+s+s1}{\PYGZsq{}}\PYG{l+s+s1}{typegraph}\PYG{l+s+s1}{\PYGZsq{}}\PYG{p}{)}
\end{sphinxVerbatim}
}

\end{sphinxuseclass}
\begin{sphinxuseclass}{nboutput}
{

\kern-\sphinxverbatimsmallskipamount\kern-\baselineskip
\kern+\FrameHeightAdjust\kern-\fboxrule
\vspace{\nbsphinxcodecellspacing}

\sphinxsetup{VerbatimColor={named}{white}}
\sphinxsetup{VerbatimBorderColor={named}{nbsphinx-code-border}}
\begin{sphinxuseclass}{output_area}
\begin{sphinxuseclass}{}


\begin{sphinxVerbatim}[commandchars=\\\{\}]
\llap{\color{nbsphinxout}[15]:\,\hspace{\fboxrule}\hspace{\fboxsep}}(<Figure size 432x288 with 1 Axes>,
 <matplotlib.axes.\_axes.Axes at 0x27671997c88>)
\end{sphinxVerbatim}



\end{sphinxuseclass}
\end{sphinxuseclass}
}

\end{sphinxuseclass}
\begin{sphinxuseclass}{nboutput}
\begin{sphinxuseclass}{nblast}
\hrule height -\fboxrule\relax
\vspace{\nbsphinxcodecellspacing}

\makeatletter\setbox\nbsphinxpromptbox\box\voidb@x\makeatother

\begin{nbsphinxfancyoutput}

\begin{sphinxuseclass}{output_area}
\begin{sphinxuseclass}{}
\noindent\sphinxincludegraphics[width=446\sphinxpxdimen,height=302\sphinxpxdimen]{{docs_Model_Structure_Visualization_Tutorial_24_1}.png}

\end{sphinxuseclass}
\end{sphinxuseclass}
\end{nbsphinxfancyoutput}

\end{sphinxuseclass}
\end{sphinxuseclass}
\sphinxAtStartPar
As shown, because the class for \sphinxcode{\sphinxupquote{View Ground}} and \sphinxcode{\sphinxupquote{Communicate Externally}} are undefined, they are shown here as both instantiations of the \sphinxcode{\sphinxupquote{GenericFxn}} class, which does not simulate.

\sphinxAtStartPar
This same structure can also be visualized using the \sphinxcode{\sphinxupquote{graphviz}}, \sphinxcode{\sphinxupquote{pyviz}}, and \sphinxcode{\sphinxupquote{netgraph}} renderers, as shown below.

\begin{sphinxuseclass}{nbinput}
{
\sphinxsetup{VerbatimColor={named}{nbsphinx-code-bg}}
\sphinxsetup{VerbatimBorderColor={named}{nbsphinx-code-border}}
\begin{sphinxVerbatim}[commandchars=\\\{\}]
\llap{\color{nbsphinxin}[16]:\,\hspace{\fboxrule}\hspace{\fboxsep}}\PYG{n}{\PYGZus{}} \PYG{o}{=} \PYG{n}{rd}\PYG{o}{.}\PYG{n}{graph}\PYG{o}{.}\PYG{n}{show}\PYG{p}{(}\PYG{n}{mdl}\PYG{p}{,} \PYG{n}{gtype}\PYG{o}{=}\PYG{l+s+s1}{\PYGZsq{}}\PYG{l+s+s1}{typegraph}\PYG{l+s+s1}{\PYGZsq{}}\PYG{p}{,} \PYG{n}{renderer}\PYG{o}{=}\PYG{l+s+s1}{\PYGZsq{}}\PYG{l+s+s1}{graphviz}\PYG{l+s+s1}{\PYGZsq{}}\PYG{p}{)}
\end{sphinxVerbatim}
}

\end{sphinxuseclass}
\begin{sphinxuseclass}{nboutput}
\begin{sphinxuseclass}{nblast}
\hrule height -\fboxrule\relax
\vspace{\nbsphinxcodecellspacing}

\makeatletter\setbox\nbsphinxpromptbox\box\voidb@x\makeatother

\begin{nbsphinxfancyoutput}

\begin{sphinxuseclass}{output_area}
\begin{sphinxuseclass}{}
\noindent\sphinxincludegraphics{{docs_Model_Structure_Visualization_Tutorial_26_0}.svg}

\end{sphinxuseclass}
\end{sphinxuseclass}
\end{nbsphinxfancyoutput}

\end{sphinxuseclass}
\end{sphinxuseclass}
\begin{sphinxuseclass}{nbinput}
{
\sphinxsetup{VerbatimColor={named}{nbsphinx-code-bg}}
\sphinxsetup{VerbatimBorderColor={named}{nbsphinx-code-border}}
\begin{sphinxVerbatim}[commandchars=\\\{\}]
\llap{\color{nbsphinxin}[17]:\,\hspace{\fboxrule}\hspace{\fboxsep}}\PYG{n}{fig\PYGZus{}pyvis} \PYG{o}{=} \PYG{n}{rd}\PYG{o}{.}\PYG{n}{graph}\PYG{o}{.}\PYG{n}{show}\PYG{p}{(}\PYG{n}{mdl}\PYG{p}{,} \PYG{n}{gtype}\PYG{o}{=}\PYG{l+s+s1}{\PYGZsq{}}\PYG{l+s+s1}{typegraph}\PYG{l+s+s1}{\PYGZsq{}}\PYG{p}{,} \PYG{n}{renderer}\PYG{o}{=}\PYG{l+s+s1}{\PYGZsq{}}\PYG{l+s+s1}{pyviz}\PYG{l+s+s1}{\PYGZsq{}}\PYG{p}{,} \PYG{n}{notebook}\PYG{o}{=}\PYG{k+kc}{True}\PYG{p}{)}
\PYG{n}{fig\PYGZus{}pyvis}\PYG{o}{.}\PYG{n}{show}\PYG{p}{(}\PYG{l+s+s2}{\PYGZdq{}}\PYG{l+s+s2}{structure.html}\PYG{l+s+s2}{\PYGZdq{}}\PYG{p}{)}
\end{sphinxVerbatim}
}

\end{sphinxuseclass}
\begin{sphinxuseclass}{nboutput}
\begin{sphinxuseclass}{nblast}
{

\kern-\sphinxverbatimsmallskipamount\kern-\baselineskip
\kern+\FrameHeightAdjust\kern-\fboxrule
\vspace{\nbsphinxcodecellspacing}

\sphinxsetup{VerbatimColor={named}{white}}
\sphinxsetup{VerbatimBorderColor={named}{nbsphinx-code-border}}
\begin{sphinxuseclass}{output_area}
\begin{sphinxuseclass}{}


\begin{sphinxVerbatim}[commandchars=\\\{\}]
\llap{\color{nbsphinxout}[17]:\,\hspace{\fboxrule}\hspace{\fboxsep}}<IPython.lib.display.IFrame at 0x27671a16188>
\end{sphinxVerbatim}



\end{sphinxuseclass}
\end{sphinxuseclass}
}

\end{sphinxuseclass}
\end{sphinxuseclass}
\begin{sphinxuseclass}{nbinput}
{
\sphinxsetup{VerbatimColor={named}{nbsphinx-code-bg}}
\sphinxsetup{VerbatimBorderColor={named}{nbsphinx-code-border}}
\begin{sphinxVerbatim}[commandchars=\\\{\}]
\llap{\color{nbsphinxin}[18]:\,\hspace{\fboxrule}\hspace{\fboxsep}}\PYG{n}{\PYGZus{}} \PYG{o}{=} \PYG{n}{rd}\PYG{o}{.}\PYG{n}{graph}\PYG{o}{.}\PYG{n}{show}\PYG{p}{(}\PYG{n}{mdl}\PYG{p}{,} \PYG{n}{gtype}\PYG{o}{=}\PYG{l+s+s1}{\PYGZsq{}}\PYG{l+s+s1}{typegraph}\PYG{l+s+s1}{\PYGZsq{}}\PYG{p}{,} \PYG{n}{renderer}\PYG{o}{=}\PYG{l+s+s1}{\PYGZsq{}}\PYG{l+s+s1}{netgraph}\PYG{l+s+s1}{\PYGZsq{}}\PYG{p}{)}
\end{sphinxVerbatim}
}

\end{sphinxuseclass}
\begin{sphinxuseclass}{nboutput}
\begin{sphinxuseclass}{nblast}
\hrule height -\fboxrule\relax
\vspace{\nbsphinxcodecellspacing}

\makeatletter\setbox\nbsphinxpromptbox\box\voidb@x\makeatother

\begin{nbsphinxfancyoutput}

\begin{sphinxuseclass}{output_area}
\begin{sphinxuseclass}{}
\noindent\sphinxincludegraphics[width=231\sphinxpxdimen,height=231\sphinxpxdimen]{{docs_Model_Structure_Visualization_Tutorial_28_0}.png}

\end{sphinxuseclass}
\end{sphinxuseclass}
\end{nbsphinxfancyoutput}

\end{sphinxuseclass}
\end{sphinxuseclass}
\sphinxAtStartPar
In general, the graphviz renderer is best at producing graphs that “just” look nice without much user tweaking/intervention. However, it requires additional setup (installation of graphviz and pygraphviz).

\sphinxAtStartPar
The \sphinxcode{\sphinxupquote{matplotlib}} renderer is most flexible and has the most implemented use\sphinxhyphen{}cases in the \sphinxcode{\sphinxupquote{graph}} module. However, it often requires some tweaking to make graphs look nice.

\sphinxAtStartPar
The \sphinxcode{\sphinxupquote{pyviz}} renderer only works for looking at model structures and has no function/fault visualization capabilities implemented. However, it produces nice html views which can be quickly modified.

\sphinxAtStartPar
The \sphinxcode{\sphinxupquote{netgraph}} renderer has most of the features of \sphinxcode{\sphinxupquote{matplotlib}}, but often suffers from scaling issues. It is not recommended unless one wishes to modify the returned graph object afterwards.


\subsubsection{Graph Views}
\label{\detokenize{docs/Model_Structure_Visualization_Tutorial:Graph-Views}}
\sphinxAtStartPar
While the “typegraph” view shows the containtment relationships of the \sphinxstyleemphasis{classes}, the \sphinxstyleemphasis{structural} relationships between the model functions and flows can be viewed using the “normal” and “bipartite” graph views, shown below.

\begin{sphinxuseclass}{nbinput}
{
\sphinxsetup{VerbatimColor={named}{nbsphinx-code-bg}}
\sphinxsetup{VerbatimBorderColor={named}{nbsphinx-code-border}}
\begin{sphinxVerbatim}[commandchars=\\\{\}]
\llap{\color{nbsphinxin}[20]:\,\hspace{\fboxrule}\hspace{\fboxsep}}\PYG{n}{rd}\PYG{o}{.}\PYG{n}{graph}\PYG{o}{.}\PYG{n}{show}\PYG{p}{(}\PYG{n}{mdl}\PYG{p}{)}
\end{sphinxVerbatim}
}

\end{sphinxuseclass}
\begin{sphinxuseclass}{nboutput}
{

\kern-\sphinxverbatimsmallskipamount\kern-\baselineskip
\kern+\FrameHeightAdjust\kern-\fboxrule
\vspace{\nbsphinxcodecellspacing}

\sphinxsetup{VerbatimColor={named}{white}}
\sphinxsetup{VerbatimBorderColor={named}{nbsphinx-code-border}}
\begin{sphinxuseclass}{output_area}
\begin{sphinxuseclass}{}


\begin{sphinxVerbatim}[commandchars=\\\{\}]
\llap{\color{nbsphinxout}[20]:\,\hspace{\fboxrule}\hspace{\fboxsep}}(<Figure size 432x288 with 1 Axes>,
 <matplotlib.axes.\_subplots.AxesSubplot at 0x27671a108c8>)
\end{sphinxVerbatim}



\end{sphinxuseclass}
\end{sphinxuseclass}
}

\end{sphinxuseclass}
\begin{sphinxuseclass}{nboutput}
\begin{sphinxuseclass}{nblast}
\hrule height -\fboxrule\relax
\vspace{\nbsphinxcodecellspacing}

\makeatletter\setbox\nbsphinxpromptbox\box\voidb@x\makeatother

\begin{nbsphinxfancyoutput}

\begin{sphinxuseclass}{output_area}
\begin{sphinxuseclass}{}
\noindent\sphinxincludegraphics[width=349\sphinxpxdimen,height=231\sphinxpxdimen]{{docs_Model_Structure_Visualization_Tutorial_32_1}.png}

\end{sphinxuseclass}
\end{sphinxuseclass}
\end{nbsphinxfancyoutput}

\end{sphinxuseclass}
\end{sphinxuseclass}
\begin{sphinxuseclass}{nbinput}
{
\sphinxsetup{VerbatimColor={named}{nbsphinx-code-bg}}
\sphinxsetup{VerbatimBorderColor={named}{nbsphinx-code-border}}
\begin{sphinxVerbatim}[commandchars=\\\{\}]
\llap{\color{nbsphinxin}[21]:\,\hspace{\fboxrule}\hspace{\fboxsep}}\PYG{n}{rd}\PYG{o}{.}\PYG{n}{graph}\PYG{o}{.}\PYG{n}{show}\PYG{p}{(}\PYG{n}{mdl}\PYG{p}{,} \PYG{n}{gtype}\PYG{o}{=}\PYG{l+s+s2}{\PYGZdq{}}\PYG{l+s+s2}{bipartite}\PYG{l+s+s2}{\PYGZdq{}}\PYG{p}{)}
\end{sphinxVerbatim}
}

\end{sphinxuseclass}
\begin{sphinxuseclass}{nboutput}
{

\kern-\sphinxverbatimsmallskipamount\kern-\baselineskip
\kern+\FrameHeightAdjust\kern-\fboxrule
\vspace{\nbsphinxcodecellspacing}

\sphinxsetup{VerbatimColor={named}{white}}
\sphinxsetup{VerbatimBorderColor={named}{nbsphinx-code-border}}
\begin{sphinxuseclass}{output_area}
\begin{sphinxuseclass}{}


\begin{sphinxVerbatim}[commandchars=\\\{\}]
\llap{\color{nbsphinxout}[21]:\,\hspace{\fboxrule}\hspace{\fboxsep}}(<Figure size 432x288 with 1 Axes>,
 <matplotlib.axes.\_subplots.AxesSubplot at 0x27671bfbe08>)
\end{sphinxVerbatim}



\end{sphinxuseclass}
\end{sphinxuseclass}
}

\end{sphinxuseclass}
\begin{sphinxuseclass}{nboutput}
\begin{sphinxuseclass}{nblast}
\hrule height -\fboxrule\relax
\vspace{\nbsphinxcodecellspacing}

\makeatletter\setbox\nbsphinxpromptbox\box\voidb@x\makeatother

\begin{nbsphinxfancyoutput}

\begin{sphinxuseclass}{output_area}
\begin{sphinxuseclass}{}
\noindent\sphinxincludegraphics[width=349\sphinxpxdimen,height=231\sphinxpxdimen]{{docs_Model_Structure_Visualization_Tutorial_33_1}.png}

\end{sphinxuseclass}
\end{sphinxuseclass}
\end{nbsphinxfancyoutput}

\end{sphinxuseclass}
\end{sphinxuseclass}
\sphinxAtStartPar
One of the major issues/limitations with the “normal” view which is often used by default is that flows connect to more than two nodes. This \sphinxhyphen{} makes it difficult to visualize how multiple nodes are connected through the same flow \sphinxhyphen{} makes it difficult to label edges (since each edge may have a number of flows on it) \sphinxhyphen{} leads to many edge overlaps As a result, despite being the default, it is generally reccomended to use the bipartite view instead, unless the model is constructed such that each
flow only connects two functions.

\sphinxAtStartPar
These views are compatible with the other renderers:

\begin{sphinxuseclass}{nbinput}
{
\sphinxsetup{VerbatimColor={named}{nbsphinx-code-bg}}
\sphinxsetup{VerbatimBorderColor={named}{nbsphinx-code-border}}
\begin{sphinxVerbatim}[commandchars=\\\{\}]
\llap{\color{nbsphinxin}[23]:\,\hspace{\fboxrule}\hspace{\fboxsep}}\PYG{n}{\PYGZus{}} \PYG{o}{=} \PYG{n}{rd}\PYG{o}{.}\PYG{n}{graph}\PYG{o}{.}\PYG{n}{show}\PYG{p}{(}\PYG{n}{mdl}\PYG{p}{,}\PYG{n}{renderer}\PYG{o}{=}\PYG{l+s+s1}{\PYGZsq{}}\PYG{l+s+s1}{graphviz}\PYG{l+s+s1}{\PYGZsq{}}\PYG{p}{)}
\PYG{n}{\PYGZus{}} \PYG{o}{=} \PYG{n}{rd}\PYG{o}{.}\PYG{n}{graph}\PYG{o}{.}\PYG{n}{show}\PYG{p}{(}\PYG{n}{mdl}\PYG{p}{,} \PYG{n}{gtype}\PYG{o}{=}\PYG{l+s+s2}{\PYGZdq{}}\PYG{l+s+s2}{bipartite}\PYG{l+s+s2}{\PYGZdq{}}\PYG{p}{,} \PYG{n}{renderer}\PYG{o}{=}\PYG{l+s+s1}{\PYGZsq{}}\PYG{l+s+s1}{graphviz}\PYG{l+s+s1}{\PYGZsq{}}\PYG{p}{)}
\end{sphinxVerbatim}
}

\end{sphinxuseclass}
\begin{sphinxuseclass}{nboutput}
\hrule height -\fboxrule\relax
\vspace{\nbsphinxcodecellspacing}

\makeatletter\setbox\nbsphinxpromptbox\box\voidb@x\makeatother

\begin{nbsphinxfancyoutput}

\begin{sphinxuseclass}{output_area}
\begin{sphinxuseclass}{}
\noindent\sphinxincludegraphics{{docs_Model_Structure_Visualization_Tutorial_35_0}.svg}

\end{sphinxuseclass}
\end{sphinxuseclass}
\end{nbsphinxfancyoutput}

\end{sphinxuseclass}
\begin{sphinxuseclass}{nboutput}
\begin{sphinxuseclass}{nblast}
\hrule height -\fboxrule\relax
\vspace{\nbsphinxcodecellspacing}

\makeatletter\setbox\nbsphinxpromptbox\box\voidb@x\makeatother

\begin{nbsphinxfancyoutput}

\begin{sphinxuseclass}{output_area}
\begin{sphinxuseclass}{}
\noindent\sphinxincludegraphics{{docs_Model_Structure_Visualization_Tutorial_35_1}.svg}

\end{sphinxuseclass}
\end{sphinxuseclass}
\end{nbsphinxfancyoutput}

\end{sphinxuseclass}
\end{sphinxuseclass}
\sphinxAtStartPar
Graphviz output can be customized using options for the renderer (see \sphinxurl{http://www.graphviz.org/doc/info/attrs.html} for all options)

\sphinxAtStartPar
For example:

\begin{sphinxuseclass}{nbinput}
{
\sphinxsetup{VerbatimColor={named}{nbsphinx-code-bg}}
\sphinxsetup{VerbatimBorderColor={named}{nbsphinx-code-border}}
\begin{sphinxVerbatim}[commandchars=\\\{\}]
\llap{\color{nbsphinxin}[30]:\,\hspace{\fboxrule}\hspace{\fboxsep}}\PYG{c+c1}{\PYGZsh{}graphviz\PYGZus{}args = \PYGZob{}\PYGZdq{}layout\PYGZdq{}:\PYGZsq{}neato\PYGZsq{}, \PYGZdq{}overlap\PYGZdq{}:\PYGZdq{}voronoi\PYGZdq{}\PYGZcb{}}
\PYG{n}{\PYGZus{}} \PYG{o}{=} \PYG{n}{rd}\PYG{o}{.}\PYG{n}{graph}\PYG{o}{.}\PYG{n}{show}\PYG{p}{(}\PYG{n}{mdl}\PYG{p}{,} \PYG{n}{gtype}\PYG{o}{=}\PYG{l+s+s2}{\PYGZdq{}}\PYG{l+s+s2}{bipartite}\PYG{l+s+s2}{\PYGZdq{}}\PYG{p}{,}\PYG{n}{renderer}\PYG{o}{=}\PYG{l+s+s1}{\PYGZsq{}}\PYG{l+s+s1}{graphviz}\PYG{l+s+s1}{\PYGZsq{}}\PYG{p}{,} \PYG{n}{layout}\PYG{o}{=}\PYG{l+s+s1}{\PYGZsq{}}\PYG{l+s+s1}{neato}\PYG{l+s+s1}{\PYGZsq{}}\PYG{p}{,} \PYG{n}{overlap}\PYG{o}{=}\PYG{l+s+s2}{\PYGZdq{}}\PYG{l+s+s2}{voronoi}\PYG{l+s+s2}{\PYGZdq{}}\PYG{p}{)}
\end{sphinxVerbatim}
}

\end{sphinxuseclass}
\begin{sphinxuseclass}{nboutput}
\begin{sphinxuseclass}{nblast}
\hrule height -\fboxrule\relax
\vspace{\nbsphinxcodecellspacing}

\makeatletter\setbox\nbsphinxpromptbox\box\voidb@x\makeatother

\begin{nbsphinxfancyoutput}

\begin{sphinxuseclass}{output_area}
\begin{sphinxuseclass}{}
\noindent\sphinxincludegraphics{{docs_Model_Structure_Visualization_Tutorial_37_0}.svg}

\end{sphinxuseclass}
\end{sphinxuseclass}
\end{nbsphinxfancyoutput}

\end{sphinxuseclass}
\end{sphinxuseclass}
\sphinxAtStartPar
Note: the pyviz renderer does not support the “normal” graph view

\begin{sphinxuseclass}{nbinput}
{
\sphinxsetup{VerbatimColor={named}{nbsphinx-code-bg}}
\sphinxsetup{VerbatimBorderColor={named}{nbsphinx-code-border}}
\begin{sphinxVerbatim}[commandchars=\\\{\}]
\llap{\color{nbsphinxin}[27]:\,\hspace{\fboxrule}\hspace{\fboxsep}}\PYG{n}{fig\PYGZus{}pyvis} \PYG{o}{=} \PYG{n}{rd}\PYG{o}{.}\PYG{n}{graph}\PYG{o}{.}\PYG{n}{show}\PYG{p}{(}\PYG{n}{mdl}\PYG{p}{,} \PYG{n}{gtype}\PYG{o}{=}\PYG{l+s+s1}{\PYGZsq{}}\PYG{l+s+s1}{bipartite}\PYG{l+s+s1}{\PYGZsq{}}\PYG{p}{,} \PYG{n}{renderer}\PYG{o}{=}\PYG{l+s+s1}{\PYGZsq{}}\PYG{l+s+s1}{pyviz}\PYG{l+s+s1}{\PYGZsq{}}\PYG{p}{,} \PYG{n}{notebook}\PYG{o}{=}\PYG{k+kc}{True}\PYG{p}{)}
\PYG{n}{fig\PYGZus{}pyvis}\PYG{o}{.}\PYG{n}{show}\PYG{p}{(}\PYG{l+s+s2}{\PYGZdq{}}\PYG{l+s+s2}{structure.html}\PYG{l+s+s2}{\PYGZdq{}}\PYG{p}{)}
\end{sphinxVerbatim}
}

\end{sphinxuseclass}
\begin{sphinxuseclass}{nboutput}
\begin{sphinxuseclass}{nblast}
{

\kern-\sphinxverbatimsmallskipamount\kern-\baselineskip
\kern+\FrameHeightAdjust\kern-\fboxrule
\vspace{\nbsphinxcodecellspacing}

\sphinxsetup{VerbatimColor={named}{white}}
\sphinxsetup{VerbatimBorderColor={named}{nbsphinx-code-border}}
\begin{sphinxuseclass}{output_area}
\begin{sphinxuseclass}{}


\begin{sphinxVerbatim}[commandchars=\\\{\}]
\llap{\color{nbsphinxout}[27]:\,\hspace{\fboxrule}\hspace{\fboxsep}}<IPython.lib.display.IFrame at 0x27671a6ab88>
\end{sphinxVerbatim}



\end{sphinxuseclass}
\end{sphinxuseclass}
}

\end{sphinxuseclass}
\end{sphinxuseclass}
\begin{sphinxuseclass}{nbinput}
{
\sphinxsetup{VerbatimColor={named}{nbsphinx-code-bg}}
\sphinxsetup{VerbatimBorderColor={named}{nbsphinx-code-border}}
\begin{sphinxVerbatim}[commandchars=\\\{\}]
\llap{\color{nbsphinxin}[28]:\,\hspace{\fboxrule}\hspace{\fboxsep}}\PYG{n}{\PYGZus{}} \PYG{o}{=} \PYG{n}{rd}\PYG{o}{.}\PYG{n}{graph}\PYG{o}{.}\PYG{n}{show}\PYG{p}{(}\PYG{n}{mdl}\PYG{p}{,}\PYG{n}{renderer}\PYG{o}{=}\PYG{l+s+s1}{\PYGZsq{}}\PYG{l+s+s1}{netgraph}\PYG{l+s+s1}{\PYGZsq{}}\PYG{p}{)}
\PYG{n}{\PYGZus{}} \PYG{o}{=} \PYG{n}{rd}\PYG{o}{.}\PYG{n}{graph}\PYG{o}{.}\PYG{n}{show}\PYG{p}{(}\PYG{n}{mdl}\PYG{p}{,} \PYG{n}{gtype}\PYG{o}{=}\PYG{l+s+s2}{\PYGZdq{}}\PYG{l+s+s2}{bipartite}\PYG{l+s+s2}{\PYGZdq{}}\PYG{p}{,} \PYG{n}{renderer}\PYG{o}{=}\PYG{l+s+s1}{\PYGZsq{}}\PYG{l+s+s1}{netgraph}\PYG{l+s+s1}{\PYGZsq{}}\PYG{p}{)}
\end{sphinxVerbatim}
}

\end{sphinxuseclass}
\begin{sphinxuseclass}{nboutput}
\hrule height -\fboxrule\relax
\vspace{\nbsphinxcodecellspacing}

\makeatletter\setbox\nbsphinxpromptbox\box\voidb@x\makeatother

\begin{nbsphinxfancyoutput}

\begin{sphinxuseclass}{output_area}
\begin{sphinxuseclass}{}
\noindent\sphinxincludegraphics[width=240\sphinxpxdimen,height=231\sphinxpxdimen]{{docs_Model_Structure_Visualization_Tutorial_40_0}.png}

\end{sphinxuseclass}
\end{sphinxuseclass}
\end{nbsphinxfancyoutput}

\end{sphinxuseclass}
\begin{sphinxuseclass}{nboutput}
\begin{sphinxuseclass}{nblast}
\hrule height -\fboxrule\relax
\vspace{\nbsphinxcodecellspacing}

\makeatletter\setbox\nbsphinxpromptbox\box\voidb@x\makeatother

\begin{nbsphinxfancyoutput}

\begin{sphinxuseclass}{output_area}
\begin{sphinxuseclass}{}
\noindent\sphinxincludegraphics[width=274\sphinxpxdimen,height=231\sphinxpxdimen]{{docs_Model_Structure_Visualization_Tutorial_40_1}.png}

\end{sphinxuseclass}
\end{sphinxuseclass}
\end{nbsphinxfancyoutput}

\end{sphinxuseclass}
\end{sphinxuseclass}

\subsubsection{Run Order}
\label{\detokenize{docs/Model_Structure_Visualization_Tutorial:Run-Order}}
\sphinxAtStartPar
To specify the run order of this model, the \sphinxcode{\sphinxupquote{add\_fxn}} method is used. The order of the call defines the run order of each instantiated function (\sphinxcode{\sphinxupquote{Control\_Rover}} \sphinxhyphen{}> \sphinxcode{\sphinxupquote{Move\_Rover}} \sphinxhyphen{}> \sphinxcode{\sphinxupquote{Store\_Energy}} \sphinxhyphen{}> \sphinxcode{\sphinxupquote{View\_Ground}} \sphinxhyphen{}> \sphinxcode{\sphinxupquote{Communicate\_Externally}}).

\sphinxAtStartPar
In addition to the functions which have been defined here, this model additionally has a number of functions which have not been defined (and will thus not execute). We can visualize using \sphinxcode{\sphinxupquote{rd.graph.exec\_order}}.

\begin{sphinxuseclass}{nbinput}
{
\sphinxsetup{VerbatimColor={named}{nbsphinx-code-bg}}
\sphinxsetup{VerbatimBorderColor={named}{nbsphinx-code-border}}
\begin{sphinxVerbatim}[commandchars=\\\{\}]
\llap{\color{nbsphinxin}[31]:\,\hspace{\fboxrule}\hspace{\fboxsep}}\PYG{n}{rd}\PYG{o}{.}\PYG{n}{graph}\PYG{o}{.}\PYG{n}{exec\PYGZus{}order}\PYG{p}{(}\PYG{n}{mdl}\PYG{p}{)}
\end{sphinxVerbatim}
}

\end{sphinxuseclass}
\begin{sphinxuseclass}{nboutput}
{

\kern-\sphinxverbatimsmallskipamount\kern-\baselineskip
\kern+\FrameHeightAdjust\kern-\fboxrule
\vspace{\nbsphinxcodecellspacing}

\sphinxsetup{VerbatimColor={named}{white}}
\sphinxsetup{VerbatimBorderColor={named}{nbsphinx-code-border}}
\begin{sphinxuseclass}{output_area}
\begin{sphinxuseclass}{}


\begin{sphinxVerbatim}[commandchars=\\\{\}]
\llap{\color{nbsphinxout}[31]:\,\hspace{\fboxrule}\hspace{\fboxsep}}(<Figure size 432x288 with 1 Axes>,
 <matplotlib.axes.\_subplots.AxesSubplot at 0x276720bcc88>)
\end{sphinxVerbatim}



\end{sphinxuseclass}
\end{sphinxuseclass}
}

\end{sphinxuseclass}
\begin{sphinxuseclass}{nboutput}
\begin{sphinxuseclass}{nblast}
\hrule height -\fboxrule\relax
\vspace{\nbsphinxcodecellspacing}

\makeatletter\setbox\nbsphinxpromptbox\box\voidb@x\makeatother

\begin{nbsphinxfancyoutput}

\begin{sphinxuseclass}{output_area}
\begin{sphinxuseclass}{}
\noindent\sphinxincludegraphics[width=349\sphinxpxdimen,height=281\sphinxpxdimen]{{docs_Model_Structure_Visualization_Tutorial_42_1}.png}

\end{sphinxuseclass}
\end{sphinxuseclass}
\end{nbsphinxfancyoutput}

\end{sphinxuseclass}
\end{sphinxuseclass}
\begin{sphinxuseclass}{nbinput}
{
\sphinxsetup{VerbatimColor={named}{nbsphinx-code-bg}}
\sphinxsetup{VerbatimBorderColor={named}{nbsphinx-code-border}}
\begin{sphinxVerbatim}[commandchars=\\\{\}]
\llap{\color{nbsphinxin}[32]:\,\hspace{\fboxrule}\hspace{\fboxsep}}\PYG{n}{\PYGZus{}} \PYG{o}{=} \PYG{n}{rd}\PYG{o}{.}\PYG{n}{graph}\PYG{o}{.}\PYG{n}{exec\PYGZus{}order}\PYG{p}{(}\PYG{n}{mdl}\PYG{p}{,} \PYG{n}{renderer}\PYG{o}{=}\PYG{l+s+s2}{\PYGZdq{}}\PYG{l+s+s2}{graphviz}\PYG{l+s+s2}{\PYGZdq{}}\PYG{p}{)}
\end{sphinxVerbatim}
}

\end{sphinxuseclass}
\begin{sphinxuseclass}{nboutput}
\hrule height -\fboxrule\relax
\vspace{\nbsphinxcodecellspacing}

\makeatletter\setbox\nbsphinxpromptbox\box\voidb@x\makeatother

\begin{nbsphinxfancyoutput}

\begin{sphinxuseclass}{output_area}
\begin{sphinxuseclass}{}
\noindent\sphinxincludegraphics{{docs_Model_Structure_Visualization_Tutorial_43_0}.svg}

\end{sphinxuseclass}
\end{sphinxuseclass}
\end{nbsphinxfancyoutput}

\end{sphinxuseclass}
\begin{sphinxuseclass}{nboutput}
\hrule height -\fboxrule\relax
\vspace{\nbsphinxcodecellspacing}

\makeatletter\setbox\nbsphinxpromptbox\box\voidb@x\makeatother

\begin{nbsphinxfancyoutput}

\begin{sphinxuseclass}{output_area}
\begin{sphinxuseclass}{}
\noindent\sphinxincludegraphics{{docs_Model_Structure_Visualization_Tutorial_43_1}.svg}

\end{sphinxuseclass}
\end{sphinxuseclass}
\end{nbsphinxfancyoutput}

\end{sphinxuseclass}
\begin{sphinxuseclass}{nboutput}
\begin{sphinxuseclass}{nblast}
{

\kern-\sphinxverbatimsmallskipamount\kern-\baselineskip
\kern+\FrameHeightAdjust\kern-\fboxrule
\vspace{\nbsphinxcodecellspacing}

\sphinxsetup{VerbatimColor={named}{white}}
\sphinxsetup{VerbatimBorderColor={named}{nbsphinx-code-border}}
\begin{sphinxuseclass}{output_area}
\begin{sphinxuseclass}{}


\begin{sphinxVerbatim}[commandchars=\\\{\}]
title not implemented in graphviz renderer
\end{sphinxVerbatim}



\end{sphinxuseclass}
\end{sphinxuseclass}
}

\end{sphinxuseclass}
\end{sphinxuseclass}
\sphinxAtStartPar
As shown, functions and flows active in the static propagation step are highlighted in cyan while the functions in the dynamic propagation step are shown (or given a border) in teal. Functions without behaviors are shown in light grey, and the run order of the dynamic propagation step is shown as numbers under the corresponding functions.

\sphinxAtStartPar
In addition, \sphinxcode{\sphinxupquote{rd.plot.dyn\_order}} can be used to visualize the dynamic propagation step.

\begin{sphinxuseclass}{nbinput}
{
\sphinxsetup{VerbatimColor={named}{nbsphinx-code-bg}}
\sphinxsetup{VerbatimBorderColor={named}{nbsphinx-code-border}}
\begin{sphinxVerbatim}[commandchars=\\\{\}]
\llap{\color{nbsphinxin}[33]:\,\hspace{\fboxrule}\hspace{\fboxsep}}\PYG{n}{rd}\PYG{o}{.}\PYG{n}{plot}\PYG{o}{.}\PYG{n}{dyn\PYGZus{}order}\PYG{p}{(}\PYG{n}{mdl}\PYG{p}{)}
\end{sphinxVerbatim}
}

\end{sphinxuseclass}
\begin{sphinxuseclass}{nboutput}
{

\kern-\sphinxverbatimsmallskipamount\kern-\baselineskip
\kern+\FrameHeightAdjust\kern-\fboxrule
\vspace{\nbsphinxcodecellspacing}

\sphinxsetup{VerbatimColor={named}{white}}
\sphinxsetup{VerbatimBorderColor={named}{nbsphinx-code-border}}
\begin{sphinxuseclass}{output_area}
\begin{sphinxuseclass}{}


\begin{sphinxVerbatim}[commandchars=\\\{\}]
\llap{\color{nbsphinxout}[33]:\,\hspace{\fboxrule}\hspace{\fboxsep}}(<Figure size 432x288 with 1 Axes>,
 <matplotlib.axes.\_subplots.AxesSubplot at 0x276719c0b88>)
\end{sphinxVerbatim}



\end{sphinxuseclass}
\end{sphinxuseclass}
}

\end{sphinxuseclass}
\begin{sphinxuseclass}{nboutput}
\begin{sphinxuseclass}{nblast}
\hrule height -\fboxrule\relax
\vspace{\nbsphinxcodecellspacing}

\makeatletter\setbox\nbsphinxpromptbox\box\voidb@x\makeatother

\begin{nbsphinxfancyoutput}

\begin{sphinxuseclass}{output_area}
\begin{sphinxuseclass}{}
\noindent\sphinxincludegraphics[width=393\sphinxpxdimen,height=262\sphinxpxdimen]{{docs_Model_Structure_Visualization_Tutorial_45_1}.png}

\end{sphinxuseclass}
\end{sphinxuseclass}
\end{nbsphinxfancyoutput}

\end{sphinxuseclass}
\end{sphinxuseclass}
\sphinxAtStartPar
This plot shows that the dynamic execution step runs in the order defined in the \sphinxcode{\sphinxupquote{Model}} module: first, Control\_Rover, then, Store\_Energy, and finally Move\_Rover (reading left\sphinxhyphen{}to\sphinxhyphen{}right in the upper axis). The plot additionally shows which flows correspond to these function as it progresses through execution, which enables some understanding of which data structures are used or acted on at each execution time.


\subsubsection{Behavior/Fault Visualization}
\label{\detokenize{docs/Model_Structure_Visualization_Tutorial:Behavior/Fault-Visualization}}
\sphinxAtStartPar
To verify the static propagation of the \sphinxcode{\sphinxupquote{short}} mode in the \sphinxcode{\sphinxupquote{Move\_Rover}} function, we can view the results of that scenario. As was set up, the intention of using the static propagation step was to enable the resulting fault behavior (a spike in current followed by a loss of charge) to occur in a single timestep.

\sphinxAtStartPar
In general, one can plot the effects of faults over time, using methods in \sphinxcode{\sphinxupquote{plot}}, as shown below.

\begin{sphinxuseclass}{nbinput}
\begin{sphinxuseclass}{nblast}
{
\sphinxsetup{VerbatimColor={named}{nbsphinx-code-bg}}
\sphinxsetup{VerbatimBorderColor={named}{nbsphinx-code-border}}
\begin{sphinxVerbatim}[commandchars=\\\{\}]
\llap{\color{nbsphinxin}[63]:\,\hspace{\fboxrule}\hspace{\fboxsep}}\PYG{n}{endresults}\PYG{p}{,} \PYG{n}{resgraph}\PYG{p}{,} \PYG{n}{mdlhist} \PYG{o}{=} \PYG{n}{prop}\PYG{o}{.}\PYG{n}{one\PYGZus{}fault}\PYG{p}{(}\PYG{n}{mdl}\PYG{p}{,} \PYG{l+s+s2}{\PYGZdq{}}\PYG{l+s+s2}{Move\PYGZus{}Rover}\PYG{l+s+s2}{\PYGZdq{}}\PYG{p}{,} \PYG{l+s+s2}{\PYGZdq{}}\PYG{l+s+s2}{short}\PYG{l+s+s2}{\PYGZdq{}}\PYG{p}{,} \PYG{l+m+mi}{10}\PYG{p}{,} \PYG{n}{gtype}\PYG{o}{=}\PYG{l+s+s1}{\PYGZsq{}}\PYG{l+s+s1}{bipartite}\PYG{l+s+s1}{\PYGZsq{}}\PYG{p}{)}
\end{sphinxVerbatim}
}

\end{sphinxuseclass}
\end{sphinxuseclass}
\begin{sphinxuseclass}{nbinput}
{
\sphinxsetup{VerbatimColor={named}{nbsphinx-code-bg}}
\sphinxsetup{VerbatimBorderColor={named}{nbsphinx-code-border}}
\begin{sphinxVerbatim}[commandchars=\\\{\}]
\llap{\color{nbsphinxin}[64]:\,\hspace{\fboxrule}\hspace{\fboxsep}}\PYG{n}{rd}\PYG{o}{.}\PYG{n}{plot}\PYG{o}{.}\PYG{n}{mdlhistvals}\PYG{p}{(}\PYG{n}{mdlhist}\PYG{p}{,} \PYG{n}{fxnflowvals}\PYG{o}{=}\PYG{p}{\PYGZob{}}\PYG{l+s+s1}{\PYGZsq{}}\PYG{l+s+s1}{Move\PYGZus{}Rover}\PYG{l+s+s1}{\PYGZsq{}}\PYG{p}{:}\PYG{l+s+s1}{\PYGZsq{}}\PYG{l+s+s1}{power}\PYG{l+s+s1}{\PYGZsq{}}\PYG{p}{,} \PYG{l+s+s1}{\PYGZsq{}}\PYG{l+s+s1}{Store\PYGZus{}Energy}\PYG{l+s+s1}{\PYGZsq{}}\PYG{p}{:}\PYG{p}{[}\PYG{l+s+s1}{\PYGZsq{}}\PYG{l+s+s1}{mode}\PYG{l+s+s1}{\PYGZsq{}}\PYG{p}{,} \PYG{l+s+s1}{\PYGZsq{}}\PYG{l+s+s1}{charge}\PYG{l+s+s1}{\PYGZsq{}}\PYG{p}{]}\PYG{p}{,} \PYG{l+s+s1}{\PYGZsq{}}\PYG{l+s+s1}{EE}\PYG{l+s+s1}{\PYGZsq{}}\PYG{p}{:}\PYG{p}{[}\PYG{l+s+s1}{\PYGZsq{}}\PYG{l+s+s1}{v\PYGZus{}supply}\PYG{l+s+s1}{\PYGZsq{}}\PYG{p}{,} \PYG{l+s+s1}{\PYGZsq{}}\PYG{l+s+s1}{a\PYGZus{}supply}\PYG{l+s+s1}{\PYGZsq{}}\PYG{p}{]}\PYG{p}{\PYGZcb{}}\PYG{p}{,} \PYG{n}{time}\PYG{o}{=}\PYG{l+m+mi}{10}\PYG{p}{)}
\end{sphinxVerbatim}
}

\end{sphinxuseclass}
\begin{sphinxuseclass}{nboutput}
\begin{sphinxuseclass}{nblast}
\hrule height -\fboxrule\relax
\vspace{\nbsphinxcodecellspacing}

\makeatletter\setbox\nbsphinxpromptbox\box\voidb@x\makeatother

\begin{nbsphinxfancyoutput}

\begin{sphinxuseclass}{output_area}
\begin{sphinxuseclass}{}
\noindent\sphinxincludegraphics[width=426\sphinxpxdimen,height=458\sphinxpxdimen]{{docs_Model_Structure_Visualization_Tutorial_50_0}.png}

\end{sphinxuseclass}
\end{sphinxuseclass}
\end{nbsphinxfancyoutput}

\end{sphinxuseclass}
\end{sphinxuseclass}
\sphinxAtStartPar
As shown, the static propagation step enables the mode to propagate back to the \sphinxcode{\sphinxupquote{Store\_Energy}} function (causing the \sphinxcode{\sphinxupquote{no\_charge}} fault) in the same timestep it is injected, even though it occurs later in the propagation order.

\sphinxAtStartPar
However, because the voltage and current output behaviors for the function are defined in the \sphinxcode{\sphinxupquote{dynamic\_behavior}} method of the \sphinxcode{\sphinxupquote{Store\_Energy}} function, these are only updated to their final value (of zero) at the next step. While this enables some visualization of the current spike, it may keep faults and behaviors from further propagating through the functions as desired. Thus, to enable this, one might reallocate some of the behaviors from the \sphinxcode{\sphinxupquote{dynamic\_behavior}} method to the
\sphinxcode{\sphinxupquote{static\_behavior}} method.


\paragraph{Visualizing time\sphinxhyphen{}slices}
\label{\detokenize{docs/Model_Structure_Visualization_Tutorial:Visualizing-time-slices}}
\sphinxAtStartPar
One of the outputs from \sphinxcode{\sphinxupquote{propagate.one\_fault}} is the \sphinxcode{\sphinxupquote{resgraph}}, which is a graph with the state of the model at the end of the simulation. Note that there are three main classifications for functions/flows visualized in this type of plot: \sphinxhyphen{} red: faulty function. Notes that the function is in a fault mode \sphinxhyphen{} orange: degraded function/flow. Nodes that the values of the flow or states of the function are \sphinxstyleemphasis{different} from the nominal scenario. Note that this is different than saying the values
represent a problem, since contingency actions are also different between the nominal and faulty runs. \sphinxhyphen{} grey: nominal function/flow. This notes that there is nothing different between the nominal and faulty run in that function or flow.

\begin{sphinxuseclass}{nbinput}
{
\sphinxsetup{VerbatimColor={named}{nbsphinx-code-bg}}
\sphinxsetup{VerbatimBorderColor={named}{nbsphinx-code-border}}
\begin{sphinxVerbatim}[commandchars=\\\{\}]
\llap{\color{nbsphinxin}[65]:\,\hspace{\fboxrule}\hspace{\fboxsep}}\PYG{n}{fig} \PYG{o}{=} \PYG{n}{rd}\PYG{o}{.}\PYG{n}{graph}\PYG{o}{.}\PYG{n}{show}\PYG{p}{(}\PYG{n}{resgraph}\PYG{p}{,} \PYG{n}{gtype} \PYG{o}{=} \PYG{l+s+s1}{\PYGZsq{}}\PYG{l+s+s1}{bipartite}\PYG{l+s+s1}{\PYGZsq{}}\PYG{p}{,} \PYG{n}{pos}\PYG{o}{=}\PYG{n}{mdl}\PYG{o}{.}\PYG{n}{bipartite\PYGZus{}pos}\PYG{p}{)}
\end{sphinxVerbatim}
}

\end{sphinxuseclass}
\begin{sphinxuseclass}{nboutput}
\begin{sphinxuseclass}{nblast}
\hrule height -\fboxrule\relax
\vspace{\nbsphinxcodecellspacing}

\makeatletter\setbox\nbsphinxpromptbox\box\voidb@x\makeatother

\begin{nbsphinxfancyoutput}

\begin{sphinxuseclass}{output_area}
\begin{sphinxuseclass}{}
\noindent\sphinxincludegraphics[width=349\sphinxpxdimen,height=231\sphinxpxdimen]{{docs_Model_Structure_Visualization_Tutorial_54_0}.png}

\end{sphinxuseclass}
\end{sphinxuseclass}
\end{nbsphinxfancyoutput}

\end{sphinxuseclass}
\end{sphinxuseclass}
\begin{sphinxuseclass}{nbinput}
{
\sphinxsetup{VerbatimColor={named}{nbsphinx-code-bg}}
\sphinxsetup{VerbatimBorderColor={named}{nbsphinx-code-border}}
\begin{sphinxVerbatim}[commandchars=\\\{\}]
\llap{\color{nbsphinxin}[66]:\,\hspace{\fboxrule}\hspace{\fboxsep}}\PYG{n}{fig} \PYG{o}{=} \PYG{n}{rd}\PYG{o}{.}\PYG{n}{graph}\PYG{o}{.}\PYG{n}{show}\PYG{p}{(}\PYG{n}{resgraph}\PYG{p}{,} \PYG{n}{renderer}\PYG{o}{=}\PYG{l+s+s1}{\PYGZsq{}}\PYG{l+s+s1}{graphviz}\PYG{l+s+s1}{\PYGZsq{}}\PYG{p}{,} \PYG{n}{gtype} \PYG{o}{=} \PYG{l+s+s1}{\PYGZsq{}}\PYG{l+s+s1}{bipartite}\PYG{l+s+s1}{\PYGZsq{}}\PYG{p}{)}
\end{sphinxVerbatim}
}

\end{sphinxuseclass}
\begin{sphinxuseclass}{nboutput}
\begin{sphinxuseclass}{nblast}
\hrule height -\fboxrule\relax
\vspace{\nbsphinxcodecellspacing}

\makeatletter\setbox\nbsphinxpromptbox\box\voidb@x\makeatother

\begin{nbsphinxfancyoutput}

\begin{sphinxuseclass}{output_area}
\begin{sphinxuseclass}{}
\noindent\sphinxincludegraphics{{docs_Model_Structure_Visualization_Tutorial_55_0}.svg}

\end{sphinxuseclass}
\end{sphinxuseclass}
\end{nbsphinxfancyoutput}

\end{sphinxuseclass}
\end{sphinxuseclass}
\sphinxAtStartPar
The \sphinxcode{\sphinxupquote{resgraph}} only gives the state of the model at the final state. We might instead want to visualize the graph at different given times. This is performed with \sphinxcode{\sphinxupquote{rd.graph.result\_from}}:

\begin{sphinxuseclass}{nbinput}
{
\sphinxsetup{VerbatimColor={named}{nbsphinx-code-bg}}
\sphinxsetup{VerbatimBorderColor={named}{nbsphinx-code-border}}
\begin{sphinxVerbatim}[commandchars=\\\{\}]
\llap{\color{nbsphinxin}[67]:\,\hspace{\fboxrule}\hspace{\fboxsep}}\PYG{n}{help}\PYG{p}{(}\PYG{n}{rd}\PYG{o}{.}\PYG{n}{graph}\PYG{o}{.}\PYG{n}{result\PYGZus{}from}\PYG{p}{)}
\end{sphinxVerbatim}
}

\end{sphinxuseclass}
\begin{sphinxuseclass}{nboutput}
\begin{sphinxuseclass}{nblast}
{

\kern-\sphinxverbatimsmallskipamount\kern-\baselineskip
\kern+\FrameHeightAdjust\kern-\fboxrule
\vspace{\nbsphinxcodecellspacing}

\sphinxsetup{VerbatimColor={named}{white}}
\sphinxsetup{VerbatimBorderColor={named}{nbsphinx-code-border}}
\begin{sphinxuseclass}{output_area}
\begin{sphinxuseclass}{}


\begin{sphinxVerbatim}[commandchars=\\\{\}]
Help on function result\_from in module fmdtools.resultdisp.graph:

result\_from(mdl, reshist, time, renderer='matplotlib', gtype='bipartite', **kwargs)
    Plots a representation of the model graph at a specific time in the results history.

    Parameters
    ----------
    mdl : model
        The model the faults were run in.
    reshist : dict
        A dictionary of results (from process.hists() or process.typehist() for the typegraph option)
    time : float
        The time in the history to plot the graph at.
    renderer : 'matplotlib' or 'graphviz' or 'netgraph'
        Renderer to use to plot the graph. Default is 'matplotlib'
    gtype : str, optional
        The type of graph to plot (normal or bipartite). The default is 'bipartite'.
    MATPLOTLIB OPTIONS:
    ----------
    faultscen : str, optional
        Name of the fault scenario. The default is [].
    showfaultlabels : bool, optional
        Whether or not to list faults on the plot. The default is True.
    scale : float, optional
        Scale factor for the node/label sizes. The default is 1.
    pos : dict, optional
        dict of node positions (if re-using positions). The default is [].

\end{sphinxVerbatim}



\end{sphinxuseclass}
\end{sphinxuseclass}
}

\end{sphinxuseclass}
\end{sphinxuseclass}
\begin{sphinxuseclass}{nbinput}
\begin{sphinxuseclass}{nblast}
{
\sphinxsetup{VerbatimColor={named}{nbsphinx-code-bg}}
\sphinxsetup{VerbatimBorderColor={named}{nbsphinx-code-border}}
\begin{sphinxVerbatim}[commandchars=\\\{\}]
\llap{\color{nbsphinxin}[70]:\,\hspace{\fboxrule}\hspace{\fboxsep}}\PYG{n}{reshist}\PYG{p}{,} \PYG{n}{\PYGZus{}}\PYG{p}{,} \PYG{n}{\PYGZus{}} \PYG{o}{=} \PYG{n}{rd}\PYG{o}{.}\PYG{n}{process}\PYG{o}{.}\PYG{n}{hist}\PYG{p}{(}\PYG{n}{mdlhist}\PYG{p}{)}
\end{sphinxVerbatim}
}

\end{sphinxuseclass}
\end{sphinxuseclass}
\begin{sphinxuseclass}{nbinput}
{
\sphinxsetup{VerbatimColor={named}{nbsphinx-code-bg}}
\sphinxsetup{VerbatimBorderColor={named}{nbsphinx-code-border}}
\begin{sphinxVerbatim}[commandchars=\\\{\}]
\llap{\color{nbsphinxin}[74]:\,\hspace{\fboxrule}\hspace{\fboxsep}}\PYG{n}{fig} \PYG{o}{=} \PYG{n}{rd}\PYG{o}{.}\PYG{n}{graph}\PYG{o}{.}\PYG{n}{result\PYGZus{}from}\PYG{p}{(}\PYG{n}{mdl}\PYG{p}{,} \PYG{n}{reshist}\PYG{p}{,} \PYG{l+m+mi}{5}\PYG{p}{,} \PYG{n}{gtype}\PYG{o}{=}\PYG{l+s+s1}{\PYGZsq{}}\PYG{l+s+s1}{bipartite}\PYG{l+s+s1}{\PYGZsq{}}\PYG{p}{)}
\end{sphinxVerbatim}
}

\end{sphinxuseclass}
\begin{sphinxuseclass}{nboutput}
\begin{sphinxuseclass}{nblast}
\hrule height -\fboxrule\relax
\vspace{\nbsphinxcodecellspacing}

\makeatletter\setbox\nbsphinxpromptbox\box\voidb@x\makeatother

\begin{nbsphinxfancyoutput}

\begin{sphinxuseclass}{output_area}
\begin{sphinxuseclass}{}
\noindent\sphinxincludegraphics[width=349\sphinxpxdimen,height=231\sphinxpxdimen]{{docs_Model_Structure_Visualization_Tutorial_59_0}.png}

\end{sphinxuseclass}
\end{sphinxuseclass}
\end{nbsphinxfancyoutput}

\end{sphinxuseclass}
\end{sphinxuseclass}
\begin{sphinxuseclass}{nbinput}
{
\sphinxsetup{VerbatimColor={named}{nbsphinx-code-bg}}
\sphinxsetup{VerbatimBorderColor={named}{nbsphinx-code-border}}
\begin{sphinxVerbatim}[commandchars=\\\{\}]
\llap{\color{nbsphinxin}[76]:\,\hspace{\fboxrule}\hspace{\fboxsep}}\PYG{n}{fig} \PYG{o}{=} \PYG{n}{rd}\PYG{o}{.}\PYG{n}{graph}\PYG{o}{.}\PYG{n}{result\PYGZus{}from}\PYG{p}{(}\PYG{n}{mdl}\PYG{p}{,} \PYG{n}{reshist}\PYG{p}{,} \PYG{l+m+mi}{10}\PYG{p}{,} \PYG{n}{gtype}\PYG{o}{=}\PYG{l+s+s1}{\PYGZsq{}}\PYG{l+s+s1}{bipartite}\PYG{l+s+s1}{\PYGZsq{}}\PYG{p}{)}
\end{sphinxVerbatim}
}

\end{sphinxuseclass}
\begin{sphinxuseclass}{nboutput}
\begin{sphinxuseclass}{nblast}
\hrule height -\fboxrule\relax
\vspace{\nbsphinxcodecellspacing}

\makeatletter\setbox\nbsphinxpromptbox\box\voidb@x\makeatother

\begin{nbsphinxfancyoutput}

\begin{sphinxuseclass}{output_area}
\begin{sphinxuseclass}{}
\noindent\sphinxincludegraphics[width=349\sphinxpxdimen,height=231\sphinxpxdimen]{{docs_Model_Structure_Visualization_Tutorial_60_0}.png}

\end{sphinxuseclass}
\end{sphinxuseclass}
\end{nbsphinxfancyoutput}

\end{sphinxuseclass}
\end{sphinxuseclass}
\begin{sphinxuseclass}{nbinput}
{
\sphinxsetup{VerbatimColor={named}{nbsphinx-code-bg}}
\sphinxsetup{VerbatimBorderColor={named}{nbsphinx-code-border}}
\begin{sphinxVerbatim}[commandchars=\\\{\}]
\llap{\color{nbsphinxin}[89]:\,\hspace{\fboxrule}\hspace{\fboxsep}}\PYG{n}{fig} \PYG{o}{=} \PYG{n}{rd}\PYG{o}{.}\PYG{n}{graph}\PYG{o}{.}\PYG{n}{result\PYGZus{}from}\PYG{p}{(}\PYG{n}{mdl}\PYG{p}{,} \PYG{n}{reshist}\PYG{p}{,} \PYG{l+m+mi}{25}\PYG{p}{,} \PYG{n}{gtype}\PYG{o}{=}\PYG{l+s+s1}{\PYGZsq{}}\PYG{l+s+s1}{bipartite}\PYG{l+s+s1}{\PYGZsq{}}\PYG{p}{,} \PYG{n}{faultscen} \PYG{o}{=} \PYG{l+s+s2}{\PYGZdq{}}\PYG{l+s+s2}{Move\PYGZus{}Rover short}\PYG{l+s+s2}{\PYGZdq{}}\PYG{p}{)}
\end{sphinxVerbatim}
}

\end{sphinxuseclass}
\begin{sphinxuseclass}{nboutput}
\begin{sphinxuseclass}{nblast}
\hrule height -\fboxrule\relax
\vspace{\nbsphinxcodecellspacing}

\makeatletter\setbox\nbsphinxpromptbox\box\voidb@x\makeatother

\begin{nbsphinxfancyoutput}

\begin{sphinxuseclass}{output_area}
\begin{sphinxuseclass}{}
\noindent\sphinxincludegraphics[width=349\sphinxpxdimen,height=247\sphinxpxdimen]{{docs_Model_Structure_Visualization_Tutorial_61_0}.png}

\end{sphinxuseclass}
\end{sphinxuseclass}
\end{nbsphinxfancyoutput}

\end{sphinxuseclass}
\end{sphinxuseclass}
\sphinxAtStartPar
This can can be done automatically over a number of different times using \sphinxcode{\sphinxupquote{rd.graph.results\_from}} to see how the effects propagate over time. Or, one can create an animation using \sphinxcode{\sphinxupquote{rd.graph.animation\_from}}:

\begin{sphinxuseclass}{nbinput}
{
\sphinxsetup{VerbatimColor={named}{nbsphinx-code-bg}}
\sphinxsetup{VerbatimBorderColor={named}{nbsphinx-code-border}}
\begin{sphinxVerbatim}[commandchars=\\\{\}]
\llap{\color{nbsphinxin}[90]:\,\hspace{\fboxrule}\hspace{\fboxsep}}\PYG{n}{help}\PYG{p}{(}\PYG{n}{rd}\PYG{o}{.}\PYG{n}{graph}\PYG{o}{.}\PYG{n}{animation\PYGZus{}from}\PYG{p}{)}
\end{sphinxVerbatim}
}

\end{sphinxuseclass}
\begin{sphinxuseclass}{nboutput}
\begin{sphinxuseclass}{nblast}
{

\kern-\sphinxverbatimsmallskipamount\kern-\baselineskip
\kern+\FrameHeightAdjust\kern-\fboxrule
\vspace{\nbsphinxcodecellspacing}

\sphinxsetup{VerbatimColor={named}{white}}
\sphinxsetup{VerbatimBorderColor={named}{nbsphinx-code-border}}
\begin{sphinxuseclass}{output_area}
\begin{sphinxuseclass}{}


\begin{sphinxVerbatim}[commandchars=\\\{\}]
Help on function animation\_from in module fmdtools.resultdisp.graph:

animation\_from(mdl, reshist, times='all', faultscen=[], gtype='bipartite', figsize=(6, 4), showfaultlabels=True, scale=1, show=False, pos=[], colors=['lightgray', 'orange', 'red'], renderer='matplotlib')
    Creates an animation of the model graph using results at given times in the results history.
    To view, use \%matplotlib qt from spyder or \%matplotlib notebook from jupyter
    To save (or do anything useful)h, make sure ffmpeg is installed  https://www.wikihow.com/Install-FFmpeg-on-Windows

    Parameters
    ----------
    mdl : model
        The model the faults were run in.
    reshist : dict
        A dictionary of results (from process.hists() or process.typehist() for the typegraph option)
    times : list or 'all'
        The times in the history to plot the graph at. If 'all', plots them all
    faultscen : str, optional
        Name of the fault scenario. The default is [].
    gtype : str, optional
        The type of graph to plot (normal or bipartite). The default is 'bipartite'.
    showfaultlabels : bool, optional
        Whether or not to list faults on the plot. The default is True.
    scale : float, optional
        Scale factor for the node/label sizes. The default is 1.
    show : bool, optional
        Whether to show the plot at the end (may be redundant). The default is True.
    pos : dict, optional
        dict of node positions (if re-using positions). The default is [].

\end{sphinxVerbatim}



\end{sphinxuseclass}
\end{sphinxuseclass}
}

\end{sphinxuseclass}
\end{sphinxuseclass}
\begin{sphinxuseclass}{nbinput}
\begin{sphinxuseclass}{nblast}
{
\sphinxsetup{VerbatimColor={named}{nbsphinx-code-bg}}
\sphinxsetup{VerbatimBorderColor={named}{nbsphinx-code-border}}
\begin{sphinxVerbatim}[commandchars=\\\{\}]
\llap{\color{nbsphinxin}[91]:\,\hspace{\fboxrule}\hspace{\fboxsep}}\PYG{k+kn}{from} \PYG{n+nn}{IPython}\PYG{n+nn}{.}\PYG{n+nn}{display} \PYG{k+kn}{import} \PYG{n}{HTML}
\end{sphinxVerbatim}
}

\end{sphinxuseclass}
\end{sphinxuseclass}
\begin{sphinxuseclass}{nbinput}
{
\sphinxsetup{VerbatimColor={named}{nbsphinx-code-bg}}
\sphinxsetup{VerbatimBorderColor={named}{nbsphinx-code-border}}
\begin{sphinxVerbatim}[commandchars=\\\{\}]
\llap{\color{nbsphinxin}[92]:\,\hspace{\fboxrule}\hspace{\fboxsep}}\PYG{n}{ani} \PYG{o}{=} \PYG{n}{rd}\PYG{o}{.}\PYG{n}{graph}\PYG{o}{.}\PYG{n}{animation\PYGZus{}from}\PYG{p}{(}\PYG{n}{mdl}\PYG{p}{,} \PYG{n}{reshist}\PYG{p}{,} \PYG{n}{gtype}\PYG{o}{=}\PYG{l+s+s1}{\PYGZsq{}}\PYG{l+s+s1}{bipartite}\PYG{l+s+s1}{\PYGZsq{}}\PYG{p}{,} \PYG{n}{faultscen} \PYG{o}{=} \PYG{l+s+s2}{\PYGZdq{}}\PYG{l+s+s2}{Move\PYGZus{}Rover short}\PYG{l+s+s2}{\PYGZdq{}}\PYG{p}{)}
\PYG{n}{HTML}\PYG{p}{(}\PYG{n}{ani}\PYG{o}{.}\PYG{n}{to\PYGZus{}jshtml}\PYG{p}{(}\PYG{p}{)}\PYG{p}{)}
\end{sphinxVerbatim}
}

\end{sphinxuseclass}
\begin{sphinxuseclass}{nboutput}
{

\kern-\sphinxverbatimsmallskipamount\kern-\baselineskip
\kern+\FrameHeightAdjust\kern-\fboxrule
\vspace{\nbsphinxcodecellspacing}

\sphinxsetup{VerbatimColor={named}{white}}
\sphinxsetup{VerbatimBorderColor={named}{nbsphinx-code-border}}
\begin{sphinxuseclass}{output_area}
\begin{sphinxuseclass}{}


\begin{sphinxVerbatim}[commandchars=\\\{\}]
\llap{\color{nbsphinxout}[92]:\,\hspace{\fboxrule}\hspace{\fboxsep}}<IPython.core.display.HTML object>
\end{sphinxVerbatim}



\end{sphinxuseclass}
\end{sphinxuseclass}
}

\end{sphinxuseclass}
\begin{sphinxuseclass}{nboutput}
\begin{sphinxuseclass}{nblast}
\hrule height -\fboxrule\relax
\vspace{\nbsphinxcodecellspacing}

\makeatletter\setbox\nbsphinxpromptbox\box\voidb@x\makeatother

\begin{nbsphinxfancyoutput}

\begin{sphinxuseclass}{output_area}
\begin{sphinxuseclass}{}
\noindent\sphinxincludegraphics[width=349\sphinxpxdimen,height=247\sphinxpxdimen]{{docs_Model_Structure_Visualization_Tutorial_65_1}.png}

\end{sphinxuseclass}
\end{sphinxuseclass}
\end{nbsphinxfancyoutput}

\end{sphinxuseclass}
\end{sphinxuseclass}
\begin{sphinxuseclass}{nbinput}
\begin{sphinxuseclass}{nblast}
{
\sphinxsetup{VerbatimColor={named}{nbsphinx-code-bg}}
\sphinxsetup{VerbatimBorderColor={named}{nbsphinx-code-border}}
\begin{sphinxVerbatim}[commandchars=\\\{\}]
\llap{\color{nbsphinxin}[ ]:\,\hspace{\fboxrule}\hspace{\fboxsep}}
\end{sphinxVerbatim}
}

\end{sphinxuseclass}
\end{sphinxuseclass}

\subsection{fmdtools Paper Demonstration}
\label{\detokenize{example_multirotor/Demonstration:fmdtools-Paper-Demonstration}}\label{\detokenize{example_multirotor/Demonstration::doc}}
\sphinxAtStartPar
This notebook shows some basic fmdtools use\sphinxhyphen{}cases presented in the paper:

\sphinxAtStartPar
\sphinxhref{https://doi.org/10.36001/ijphm.2021.v12i3.2954}{Hulse, D., Walsh, H., Dong, A., Hoyle, C., Tumer, I., Kulkarni, C., \& Goebel, K. (2021). fmdtools: A Fault Propagation Toolkit for Resilience Assessment in Early Design. International Journal of Prognostics and Health Management, 12(3)}

\sphinxAtStartPar
This notebook uses a high\sphinxhyphen{}level model of a multirotor drone to illustrate the following fmdtools model types: \sphinxhyphen{} Network Model \sphinxhyphen{} Static Model \sphinxhyphen{} Dynamic Model \sphinxhyphen{} Hierarchical Model

\begin{sphinxuseclass}{nbinput}
\begin{sphinxuseclass}{nblast}
{
\sphinxsetup{VerbatimColor={named}{nbsphinx-code-bg}}
\sphinxsetup{VerbatimBorderColor={named}{nbsphinx-code-border}}
\begin{sphinxVerbatim}[commandchars=\\\{\}]
\llap{\color{nbsphinxin}[1]:\,\hspace{\fboxrule}\hspace{\fboxsep}}\PYG{k+kn}{import} \PYG{n+nn}{sys}\PYG{o}{,} \PYG{n+nn}{os}
\PYG{n}{sys}\PYG{o}{.}\PYG{n}{path}\PYG{o}{.}\PYG{n}{insert}\PYG{p}{(}\PYG{l+m+mi}{1}\PYG{p}{,}\PYG{n}{os}\PYG{o}{.}\PYG{n}{path}\PYG{o}{.}\PYG{n}{join}\PYG{p}{(}\PYG{l+s+s2}{\PYGZdq{}}\PYG{l+s+s2}{..}\PYG{l+s+s2}{\PYGZdq{}}\PYG{p}{)}\PYG{p}{)}

\PYG{k+kn}{import} \PYG{n+nn}{matplotlib}\PYG{n+nn}{.}\PYG{n+nn}{pyplot} \PYG{k}{as} \PYG{n+nn}{plt}
\PYG{k+kn}{import} \PYG{n+nn}{numpy} \PYG{k}{as} \PYG{n+nn}{np}
\PYG{k+kn}{import} \PYG{n+nn}{pandas} \PYG{k}{as} \PYG{n+nn}{pd}



\PYG{k+kn}{import} \PYG{n+nn}{fmdtools}\PYG{n+nn}{.}\PYG{n+nn}{faultsim} \PYG{k}{as} \PYG{n+nn}{fs}
\PYG{k+kn}{import} \PYG{n+nn}{fmdtools}\PYG{n+nn}{.}\PYG{n+nn}{resultdisp} \PYG{k}{as} \PYG{n+nn}{rd}
\PYG{k+kn}{import} \PYG{n+nn}{quadpy}
\PYG{k+kn}{from} \PYG{n+nn}{IPython}\PYG{n+nn}{.}\PYG{n+nn}{display} \PYG{k+kn}{import} \PYG{n}{HTML}
\PYG{k+kn}{from} \PYG{n+nn}{fmdtools}\PYG{n+nn}{.}\PYG{n+nn}{modeldef} \PYG{k+kn}{import} \PYG{n}{SampleApproach}
\PYG{k+kn}{from} \PYG{n+nn}{fmdtools}\PYG{n+nn}{.}\PYG{n+nn}{modeldef} \PYG{k+kn}{import} \PYG{n}{Model}
\end{sphinxVerbatim}
}

\end{sphinxuseclass}
\end{sphinxuseclass}

\subsubsection{Initial Model}
\label{\detokenize{example_multirotor/Demonstration:Initial-Model}}
\sphinxAtStartPar
In our initial model, all we have is the flows, functions, and connections between them. These are set up in a model class as shown:

\begin{sphinxuseclass}{nbinput}
\begin{sphinxuseclass}{nblast}
{
\sphinxsetup{VerbatimColor={named}{nbsphinx-code-bg}}
\sphinxsetup{VerbatimBorderColor={named}{nbsphinx-code-border}}
\begin{sphinxVerbatim}[commandchars=\\\{\}]
\llap{\color{nbsphinxin}[2]:\,\hspace{\fboxrule}\hspace{\fboxsep}}\PYG{k}{class} \PYG{n+nc}{Drone}\PYG{p}{(}\PYG{n}{Model}\PYG{p}{)}\PYG{p}{:}
    \PYG{k}{def} \PYG{n+nf+fm}{\PYGZus{}\PYGZus{}init\PYGZus{}\PYGZus{}}\PYG{p}{(}\PYG{n+nb+bp}{self}\PYG{p}{,} \PYG{n}{params}\PYG{o}{=}\PYG{p}{\PYGZob{}}\PYG{p}{\PYGZcb{}}\PYG{p}{)}\PYG{p}{:}
        \PYG{n+nb}{super}\PYG{p}{(}\PYG{p}{)}\PYG{o}{.}\PYG{n+nf+fm}{\PYGZus{}\PYGZus{}init\PYGZus{}\PYGZus{}}\PYG{p}{(}\PYG{p}{)}
        \PYG{n+nb+bp}{self}\PYG{o}{.}\PYG{n}{params}\PYG{o}{=}\PYG{n}{params}
        \PYG{c+c1}{\PYGZsh{}add flows to the model}
        \PYG{n+nb+bp}{self}\PYG{o}{.}\PYG{n}{add\PYGZus{}flow}\PYG{p}{(}\PYG{l+s+s1}{\PYGZsq{}}\PYG{l+s+s1}{Force\PYGZus{}ST}\PYG{l+s+s1}{\PYGZsq{}}\PYG{p}{,} \PYG{p}{\PYGZob{}}\PYG{p}{\PYGZcb{}}\PYG{p}{)}
        \PYG{n+nb+bp}{self}\PYG{o}{.}\PYG{n}{add\PYGZus{}flow}\PYG{p}{(}\PYG{l+s+s1}{\PYGZsq{}}\PYG{l+s+s1}{Force\PYGZus{}Lin}\PYG{l+s+s1}{\PYGZsq{}}\PYG{p}{,} \PYG{p}{\PYGZob{}}\PYG{p}{\PYGZcb{}}\PYG{p}{)}
        \PYG{n+nb+bp}{self}\PYG{o}{.}\PYG{n}{add\PYGZus{}flow}\PYG{p}{(}\PYG{l+s+s1}{\PYGZsq{}}\PYG{l+s+s1}{Force\PYGZus{}GR}\PYG{l+s+s1}{\PYGZsq{}} \PYG{p}{,} \PYG{p}{\PYGZob{}}\PYG{p}{\PYGZcb{}}\PYG{p}{)}
        \PYG{n+nb+bp}{self}\PYG{o}{.}\PYG{n}{add\PYGZus{}flow}\PYG{p}{(}\PYG{l+s+s1}{\PYGZsq{}}\PYG{l+s+s1}{Force\PYGZus{}LG}\PYG{l+s+s1}{\PYGZsq{}}\PYG{p}{,} \PYG{p}{\PYGZob{}}\PYG{p}{\PYGZcb{}}\PYG{p}{)}
        \PYG{n+nb+bp}{self}\PYG{o}{.}\PYG{n}{add\PYGZus{}flow}\PYG{p}{(}\PYG{l+s+s1}{\PYGZsq{}}\PYG{l+s+s1}{EE\PYGZus{}1}\PYG{l+s+s1}{\PYGZsq{}}\PYG{p}{,} \PYG{p}{\PYGZob{}}\PYG{p}{\PYGZcb{}}\PYG{p}{)}
        \PYG{n+nb+bp}{self}\PYG{o}{.}\PYG{n}{add\PYGZus{}flow}\PYG{p}{(}\PYG{l+s+s1}{\PYGZsq{}}\PYG{l+s+s1}{EEmot}\PYG{l+s+s1}{\PYGZsq{}}\PYG{p}{,} \PYG{p}{\PYGZob{}}\PYG{p}{\PYGZcb{}}\PYG{p}{)}
        \PYG{n+nb+bp}{self}\PYG{o}{.}\PYG{n}{add\PYGZus{}flow}\PYG{p}{(}\PYG{l+s+s1}{\PYGZsq{}}\PYG{l+s+s1}{EEctl}\PYG{l+s+s1}{\PYGZsq{}}\PYG{p}{,} \PYG{p}{\PYGZob{}}\PYG{p}{\PYGZcb{}}\PYG{p}{)}
        \PYG{n+nb+bp}{self}\PYG{o}{.}\PYG{n}{add\PYGZus{}flow}\PYG{p}{(}\PYG{l+s+s1}{\PYGZsq{}}\PYG{l+s+s1}{Ctl1}\PYG{l+s+s1}{\PYGZsq{}}\PYG{p}{,} \PYG{p}{\PYGZob{}}\PYG{p}{\PYGZcb{}}\PYG{p}{)}
        \PYG{n+nb+bp}{self}\PYG{o}{.}\PYG{n}{add\PYGZus{}flow}\PYG{p}{(}\PYG{l+s+s1}{\PYGZsq{}}\PYG{l+s+s1}{DOFs}\PYG{l+s+s1}{\PYGZsq{}}\PYG{p}{,} \PYG{p}{\PYGZob{}}\PYG{p}{\PYGZcb{}}\PYG{p}{)}
        \PYG{n+nb+bp}{self}\PYG{o}{.}\PYG{n}{add\PYGZus{}flow}\PYG{p}{(}\PYG{l+s+s1}{\PYGZsq{}}\PYG{l+s+s1}{Env1}\PYG{l+s+s1}{\PYGZsq{}}\PYG{p}{,} \PYG{p}{\PYGZob{}}\PYG{p}{\PYGZcb{}}\PYG{p}{)}
        \PYG{n+nb+bp}{self}\PYG{o}{.}\PYG{n}{add\PYGZus{}flow}\PYG{p}{(}\PYG{l+s+s1}{\PYGZsq{}}\PYG{l+s+s1}{Dir1}\PYG{l+s+s1}{\PYGZsq{}}\PYG{p}{,} \PYG{p}{\PYGZob{}}\PYG{p}{\PYGZcb{}}\PYG{p}{)}
        \PYG{c+c1}{\PYGZsh{}add functions to the model}
        \PYG{n}{flows}\PYG{o}{=}\PYG{p}{[}\PYG{l+s+s1}{\PYGZsq{}}\PYG{l+s+s1}{EEctl}\PYG{l+s+s1}{\PYGZsq{}}\PYG{p}{,} \PYG{l+s+s1}{\PYGZsq{}}\PYG{l+s+s1}{Force\PYGZus{}ST}\PYG{l+s+s1}{\PYGZsq{}}\PYG{p}{]}
        \PYG{n+nb+bp}{self}\PYG{o}{.}\PYG{n}{add\PYGZus{}fxn}\PYG{p}{(}\PYG{l+s+s1}{\PYGZsq{}}\PYG{l+s+s1}{StoreEE}\PYG{l+s+s1}{\PYGZsq{}}\PYG{p}{,}\PYG{p}{[}\PYG{l+s+s1}{\PYGZsq{}}\PYG{l+s+s1}{EE\PYGZus{}1}\PYG{l+s+s1}{\PYGZsq{}}\PYG{p}{,} \PYG{l+s+s1}{\PYGZsq{}}\PYG{l+s+s1}{Force\PYGZus{}ST}\PYG{l+s+s1}{\PYGZsq{}}\PYG{p}{]}\PYG{p}{)}
        \PYG{n+nb+bp}{self}\PYG{o}{.}\PYG{n}{add\PYGZus{}fxn}\PYG{p}{(}\PYG{l+s+s1}{\PYGZsq{}}\PYG{l+s+s1}{DistEE}\PYG{l+s+s1}{\PYGZsq{}}\PYG{p}{,} \PYG{p}{[}\PYG{l+s+s1}{\PYGZsq{}}\PYG{l+s+s1}{EE\PYGZus{}1}\PYG{l+s+s1}{\PYGZsq{}}\PYG{p}{,}\PYG{l+s+s1}{\PYGZsq{}}\PYG{l+s+s1}{EEmot}\PYG{l+s+s1}{\PYGZsq{}}\PYG{p}{,}\PYG{l+s+s1}{\PYGZsq{}}\PYG{l+s+s1}{EEctl}\PYG{l+s+s1}{\PYGZsq{}}\PYG{p}{,} \PYG{l+s+s1}{\PYGZsq{}}\PYG{l+s+s1}{Force\PYGZus{}ST}\PYG{l+s+s1}{\PYGZsq{}}\PYG{p}{]}\PYG{p}{)}
        \PYG{n+nb+bp}{self}\PYG{o}{.}\PYG{n}{add\PYGZus{}fxn}\PYG{p}{(}\PYG{l+s+s1}{\PYGZsq{}}\PYG{l+s+s1}{AffectDOF}\PYG{l+s+s1}{\PYGZsq{}}\PYG{p}{,}\PYG{p}{[}\PYG{l+s+s1}{\PYGZsq{}}\PYG{l+s+s1}{EEmot}\PYG{l+s+s1}{\PYGZsq{}}\PYG{p}{,}\PYG{l+s+s1}{\PYGZsq{}}\PYG{l+s+s1}{Ctl1}\PYG{l+s+s1}{\PYGZsq{}}\PYG{p}{,}\PYG{l+s+s1}{\PYGZsq{}}\PYG{l+s+s1}{DOFs}\PYG{l+s+s1}{\PYGZsq{}}\PYG{p}{,}\PYG{l+s+s1}{\PYGZsq{}}\PYG{l+s+s1}{Force\PYGZus{}Lin}\PYG{l+s+s1}{\PYGZsq{}}\PYG{p}{]}\PYG{p}{)}
        \PYG{n+nb+bp}{self}\PYG{o}{.}\PYG{n}{add\PYGZus{}fxn}\PYG{p}{(}\PYG{l+s+s1}{\PYGZsq{}}\PYG{l+s+s1}{CtlDOF}\PYG{l+s+s1}{\PYGZsq{}}\PYG{p}{,} \PYG{p}{[}\PYG{l+s+s1}{\PYGZsq{}}\PYG{l+s+s1}{EEctl}\PYG{l+s+s1}{\PYGZsq{}}\PYG{p}{,} \PYG{l+s+s1}{\PYGZsq{}}\PYG{l+s+s1}{Dir1}\PYG{l+s+s1}{\PYGZsq{}}\PYG{p}{,} \PYG{l+s+s1}{\PYGZsq{}}\PYG{l+s+s1}{Ctl1}\PYG{l+s+s1}{\PYGZsq{}}\PYG{p}{,} \PYG{l+s+s1}{\PYGZsq{}}\PYG{l+s+s1}{DOFs}\PYG{l+s+s1}{\PYGZsq{}}\PYG{p}{,} \PYG{l+s+s1}{\PYGZsq{}}\PYG{l+s+s1}{Force\PYGZus{}ST}\PYG{l+s+s1}{\PYGZsq{}}\PYG{p}{]}\PYG{p}{)}
        \PYG{n+nb+bp}{self}\PYG{o}{.}\PYG{n}{add\PYGZus{}fxn}\PYG{p}{(}\PYG{l+s+s1}{\PYGZsq{}}\PYG{l+s+s1}{Planpath}\PYG{l+s+s1}{\PYGZsq{}}\PYG{p}{,} \PYG{p}{[}\PYG{l+s+s1}{\PYGZsq{}}\PYG{l+s+s1}{EEctl}\PYG{l+s+s1}{\PYGZsq{}}\PYG{p}{,} \PYG{l+s+s1}{\PYGZsq{}}\PYG{l+s+s1}{Env1}\PYG{l+s+s1}{\PYGZsq{}}\PYG{p}{,}\PYG{l+s+s1}{\PYGZsq{}}\PYG{l+s+s1}{Dir1}\PYG{l+s+s1}{\PYGZsq{}}\PYG{p}{,} \PYG{l+s+s1}{\PYGZsq{}}\PYG{l+s+s1}{Force\PYGZus{}ST}\PYG{l+s+s1}{\PYGZsq{}}\PYG{p}{]}\PYG{p}{)}
        \PYG{n+nb+bp}{self}\PYG{o}{.}\PYG{n}{add\PYGZus{}fxn}\PYG{p}{(}\PYG{l+s+s1}{\PYGZsq{}}\PYG{l+s+s1}{Trajectory}\PYG{l+s+s1}{\PYGZsq{}}\PYG{p}{,} \PYG{p}{[}\PYG{l+s+s1}{\PYGZsq{}}\PYG{l+s+s1}{Env1}\PYG{l+s+s1}{\PYGZsq{}}\PYG{p}{,}\PYG{l+s+s1}{\PYGZsq{}}\PYG{l+s+s1}{DOFs}\PYG{l+s+s1}{\PYGZsq{}}\PYG{p}{,}\PYG{l+s+s1}{\PYGZsq{}}\PYG{l+s+s1}{Dir1}\PYG{l+s+s1}{\PYGZsq{}}\PYG{p}{,} \PYG{l+s+s1}{\PYGZsq{}}\PYG{l+s+s1}{Force\PYGZus{}GR}\PYG{l+s+s1}{\PYGZsq{}}\PYG{p}{]} \PYG{p}{)}
        \PYG{n+nb+bp}{self}\PYG{o}{.}\PYG{n}{add\PYGZus{}fxn}\PYG{p}{(}\PYG{l+s+s1}{\PYGZsq{}}\PYG{l+s+s1}{EngageLand}\PYG{l+s+s1}{\PYGZsq{}}\PYG{p}{,}\PYG{p}{[}\PYG{l+s+s1}{\PYGZsq{}}\PYG{l+s+s1}{Force\PYGZus{}GR}\PYG{l+s+s1}{\PYGZsq{}}\PYG{p}{,} \PYG{l+s+s1}{\PYGZsq{}}\PYG{l+s+s1}{Force\PYGZus{}LG}\PYG{l+s+s1}{\PYGZsq{}}\PYG{p}{]}\PYG{p}{)}
        \PYG{n+nb+bp}{self}\PYG{o}{.}\PYG{n}{add\PYGZus{}fxn}\PYG{p}{(}\PYG{l+s+s1}{\PYGZsq{}}\PYG{l+s+s1}{HoldPayload}\PYG{l+s+s1}{\PYGZsq{}}\PYG{p}{,}\PYG{p}{[}\PYG{l+s+s1}{\PYGZsq{}}\PYG{l+s+s1}{Force\PYGZus{}LG}\PYG{l+s+s1}{\PYGZsq{}}\PYG{p}{,} \PYG{l+s+s1}{\PYGZsq{}}\PYG{l+s+s1}{Force\PYGZus{}Lin}\PYG{l+s+s1}{\PYGZsq{}}\PYG{p}{,} \PYG{l+s+s1}{\PYGZsq{}}\PYG{l+s+s1}{Force\PYGZus{}ST}\PYG{l+s+s1}{\PYGZsq{}}\PYG{p}{]}\PYG{p}{)}
        \PYG{n+nb+bp}{self}\PYG{o}{.}\PYG{n}{add\PYGZus{}fxn}\PYG{p}{(}\PYG{l+s+s1}{\PYGZsq{}}\PYG{l+s+s1}{ViewEnv}\PYG{l+s+s1}{\PYGZsq{}}\PYG{p}{,} \PYG{p}{[}\PYG{l+s+s1}{\PYGZsq{}}\PYG{l+s+s1}{Env1}\PYG{l+s+s1}{\PYGZsq{}}\PYG{p}{]}\PYG{p}{)}

        \PYG{n+nb+bp}{self}\PYG{o}{.}\PYG{n}{build\PYGZus{}model}\PYG{p}{(}\PYG{p}{)}
\end{sphinxVerbatim}
}

\end{sphinxuseclass}
\end{sphinxuseclass}

\subsubsection{Setting Node Positions}
\label{\detokenize{example_multirotor/Demonstration:Setting-Node-Positions}}
\sphinxAtStartPar
As shown below, it can be difficult to make sense of a model structure using the default shell graph layout. We might instead want to see something that more closely approximates a flow chart of the system.

\begin{sphinxuseclass}{nbinput}
{
\sphinxsetup{VerbatimColor={named}{nbsphinx-code-bg}}
\sphinxsetup{VerbatimBorderColor={named}{nbsphinx-code-border}}
\begin{sphinxVerbatim}[commandchars=\\\{\}]
\llap{\color{nbsphinxin}[3]:\,\hspace{\fboxrule}\hspace{\fboxsep}}\PYG{n}{mdl} \PYG{o}{=} \PYG{n}{Drone}\PYG{p}{(}\PYG{p}{)}
\PYG{n}{rd}\PYG{o}{.}\PYG{n}{graph}\PYG{o}{.}\PYG{n}{show}\PYG{p}{(}\PYG{n}{mdl}\PYG{p}{)}
\end{sphinxVerbatim}
}

\end{sphinxuseclass}
\begin{sphinxuseclass}{nboutput}
{

\kern-\sphinxverbatimsmallskipamount\kern-\baselineskip
\kern+\FrameHeightAdjust\kern-\fboxrule
\vspace{\nbsphinxcodecellspacing}

\sphinxsetup{VerbatimColor={named}{white}}
\sphinxsetup{VerbatimBorderColor={named}{nbsphinx-code-border}}
\begin{sphinxuseclass}{output_area}
\begin{sphinxuseclass}{}


\begin{sphinxVerbatim}[commandchars=\\\{\}]
\llap{\color{nbsphinxout}[3]:\,\hspace{\fboxrule}\hspace{\fboxsep}}(<Figure size 432x288 with 1 Axes>, <AxesSubplot:>)
\end{sphinxVerbatim}



\end{sphinxuseclass}
\end{sphinxuseclass}
}

\end{sphinxuseclass}
\begin{sphinxuseclass}{nboutput}
\begin{sphinxuseclass}{nblast}
\hrule height -\fboxrule\relax
\vspace{\nbsphinxcodecellspacing}

\makeatletter\setbox\nbsphinxpromptbox\box\voidb@x\makeatother

\begin{nbsphinxfancyoutput}

\begin{sphinxuseclass}{output_area}
\begin{sphinxuseclass}{}
\noindent\sphinxincludegraphics[width=349\sphinxpxdimen,height=231\sphinxpxdimen]{{example_multirotor_Demonstration_6_1}.png}

\end{sphinxuseclass}
\end{sphinxuseclass}
\end{nbsphinxfancyoutput}

\end{sphinxuseclass}
\end{sphinxuseclass}
\sphinxAtStartPar
To set node positions, we can use \sphinxcode{\sphinxupquote{rd.graph.set\_pos()}}, which lets one drag the nodes to their desired locations. If a model is sent to \sphinxcode{\sphinxupquote{set\_pos}}, it will set those locations in the model going forward, though it is good practice to save node locations when one is done in the script or the model class file (they can be used as inputs to \sphinxcode{\sphinxupquote{construct\_graph()}}.

\begin{sphinxuseclass}{nbinput}
\begin{sphinxuseclass}{nblast}
{
\sphinxsetup{VerbatimColor={named}{nbsphinx-code-bg}}
\sphinxsetup{VerbatimBorderColor={named}{nbsphinx-code-border}}
\begin{sphinxVerbatim}[commandchars=\\\{\}]
\llap{\color{nbsphinxin}[4]:\,\hspace{\fboxrule}\hspace{\fboxsep}}\PYG{c+c1}{\PYGZsh{}\PYGZpc{}matplotlib qt5}
\PYG{c+c1}{\PYGZsh{}pos = rd.graph.set\PYGZus{}pos(mdl)}
\PYG{c+c1}{\PYGZsh{}pos}
\end{sphinxVerbatim}
}

\end{sphinxuseclass}
\end{sphinxuseclass}
\sphinxAtStartPar
We can use the same process to arrange the bipartite graph:

\begin{sphinxuseclass}{nbinput}
\begin{sphinxuseclass}{nblast}
{
\sphinxsetup{VerbatimColor={named}{nbsphinx-code-bg}}
\sphinxsetup{VerbatimBorderColor={named}{nbsphinx-code-border}}
\begin{sphinxVerbatim}[commandchars=\\\{\}]
\llap{\color{nbsphinxin}[5]:\,\hspace{\fboxrule}\hspace{\fboxsep}}\PYG{c+c1}{\PYGZsh{}\PYGZpc{}matplotlib qt5}
\PYG{c+c1}{\PYGZsh{}pos = rd.graph.set\PYGZus{}pos(mdl, gtype=\PYGZsq{}bipartite\PYGZsq{})}
\end{sphinxVerbatim}
}

\end{sphinxuseclass}
\end{sphinxuseclass}
\sphinxAtStartPar
As shown, in a large model, the Bipartite graph is often easier to arrange to get a good layout. Since the model will be redefined several times going forward, we will use these positions to keep a consistent layout:

\begin{sphinxuseclass}{nbinput}
\begin{sphinxuseclass}{nblast}
{
\sphinxsetup{VerbatimColor={named}{nbsphinx-code-bg}}
\sphinxsetup{VerbatimBorderColor={named}{nbsphinx-code-border}}
\begin{sphinxVerbatim}[commandchars=\\\{\}]
\llap{\color{nbsphinxin}[6]:\,\hspace{\fboxrule}\hspace{\fboxsep}}\PYG{n}{bipartite\PYGZus{}pos} \PYG{o}{=} \PYG{p}{\PYGZob{}}\PYG{l+s+s1}{\PYGZsq{}}\PYG{l+s+s1}{StoreEE}\PYG{l+s+s1}{\PYGZsq{}}\PYG{p}{:} \PYG{p}{[}\PYG{o}{\PYGZhy{}}\PYG{l+m+mf}{1.067135163123663}\PYG{p}{,} \PYG{l+m+mf}{0.32466987344741055}\PYG{p}{]}\PYG{p}{,}
 \PYG{l+s+s1}{\PYGZsq{}}\PYG{l+s+s1}{DistEE}\PYG{l+s+s1}{\PYGZsq{}}\PYG{p}{:} \PYG{p}{[}\PYG{o}{\PYGZhy{}}\PYG{l+m+mf}{0.617149602161968}\PYG{p}{,} \PYG{l+m+mf}{0.3165981670924663}\PYG{p}{]}\PYG{p}{,}
 \PYG{l+s+s1}{\PYGZsq{}}\PYG{l+s+s1}{AffectDOF}\PYG{l+s+s1}{\PYGZsq{}}\PYG{p}{:} \PYG{p}{[}\PYG{l+m+mf}{0.11827439153655106}\PYG{p}{,} \PYG{l+m+mf}{0.10792528450121897}\PYG{p}{]}\PYG{p}{,}
 \PYG{l+s+s1}{\PYGZsq{}}\PYG{l+s+s1}{CtlDOF}\PYG{l+s+s1}{\PYGZsq{}}\PYG{p}{:} \PYG{p}{[}\PYG{o}{\PYGZhy{}}\PYG{l+m+mf}{0.2636856982162134}\PYG{p}{,} \PYG{l+m+mf}{0.42422600969836144}\PYG{p}{]}\PYG{p}{,}
 \PYG{l+s+s1}{\PYGZsq{}}\PYG{l+s+s1}{Planpath}\PYG{l+s+s1}{\PYGZsq{}}\PYG{p}{:} \PYG{p}{[}\PYG{o}{\PYGZhy{}}\PYG{l+m+mf}{0.9347151173753852}\PYG{p}{,} \PYG{l+m+mf}{0.6943421719257798}\PYG{p}{]}\PYG{p}{,}
 \PYG{l+s+s1}{\PYGZsq{}}\PYG{l+s+s1}{Trajectory}\PYG{l+s+s1}{\PYGZsq{}}\PYG{p}{:} \PYG{p}{[}\PYG{l+m+mf}{0.6180477286739998}\PYG{p}{,} \PYG{l+m+mf}{0.32930706399226856}\PYG{p}{]}\PYG{p}{,}
 \PYG{l+s+s1}{\PYGZsq{}}\PYG{l+s+s1}{EngageLand}\PYG{l+s+s1}{\PYGZsq{}}\PYG{p}{:} \PYG{p}{[}\PYG{l+m+mf}{0.0015917696269229786}\PYG{p}{,} \PYG{o}{\PYGZhy{}}\PYG{l+m+mf}{0.2399760932810826}\PYG{p}{]}\PYG{p}{,}
 \PYG{l+s+s1}{\PYGZsq{}}\PYG{l+s+s1}{HoldPayload}\PYG{l+s+s1}{\PYGZsq{}}\PYG{p}{:} \PYG{p}{[}\PYG{o}{\PYGZhy{}}\PYG{l+m+mf}{0.8833099612826893}\PYG{p}{,} \PYG{o}{\PYGZhy{}}\PYG{l+m+mf}{0.247201580673997}\PYG{p}{]}\PYG{p}{,}
 \PYG{l+s+s1}{\PYGZsq{}}\PYG{l+s+s1}{ViewEnv}\PYG{l+s+s1}{\PYGZsq{}}\PYG{p}{:} \PYG{p}{[}\PYG{l+m+mf}{0.5725955705698363}\PYG{p}{,} \PYG{l+m+mf}{0.6901513410348765}\PYG{p}{]}\PYG{p}{,}
 \PYG{l+s+s1}{\PYGZsq{}}\PYG{l+s+s1}{Force\PYGZus{}ST}\PYG{l+s+s1}{\PYGZsq{}}\PYG{p}{:} \PYG{p}{[}\PYG{o}{\PYGZhy{}}\PYG{l+m+mf}{0.8925771348524384}\PYG{p}{,} \PYG{o}{\PYGZhy{}}\PYG{l+m+mf}{0.025638904424547027}\PYG{p}{]}\PYG{p}{,}
 \PYG{l+s+s1}{\PYGZsq{}}\PYG{l+s+s1}{Force\PYGZus{}Lin}\PYG{l+s+s1}{\PYGZsq{}}\PYG{p}{:} \PYG{p}{[}\PYG{o}{\PYGZhy{}}\PYG{l+m+mf}{0.5530952425102891}\PYG{p}{,} \PYG{o}{\PYGZhy{}}\PYG{l+m+mf}{0.10380834289626095}\PYG{p}{]}\PYG{p}{,}
 \PYG{l+s+s1}{\PYGZsq{}}\PYG{l+s+s1}{Force\PYGZus{}GR}\PYG{l+s+s1}{\PYGZsq{}}\PYG{p}{:} \PYG{p}{[}\PYG{l+m+mf}{0.568921162299461}\PYG{p}{,} \PYG{o}{\PYGZhy{}}\PYG{l+m+mf}{0.22991830334765573}\PYG{p}{]}\PYG{p}{,}
 \PYG{l+s+s1}{\PYGZsq{}}\PYG{l+s+s1}{Force\PYGZus{}LG}\PYG{l+s+s1}{\PYGZsq{}}\PYG{p}{:} \PYG{p}{[}\PYG{o}{\PYGZhy{}}\PYG{l+m+mf}{0.37244114591548894}\PYG{p}{,} \PYG{o}{\PYGZhy{}}\PYG{l+m+mf}{0.2355298479531287}\PYG{p}{]}\PYG{p}{,}
 \PYG{l+s+s1}{\PYGZsq{}}\PYG{l+s+s1}{EE\PYGZus{}1}\PYG{l+s+s1}{\PYGZsq{}}\PYG{p}{:} \PYG{p}{[}\PYG{o}{\PYGZhy{}}\PYG{l+m+mf}{0.809433489993954}\PYG{p}{,} \PYG{l+m+mf}{0.319191761486317}\PYG{p}{]}\PYG{p}{,}
 \PYG{l+s+s1}{\PYGZsq{}}\PYG{l+s+s1}{EEmot}\PYG{l+s+s1}{\PYGZsq{}}\PYG{p}{:} \PYG{p}{[}\PYG{o}{\PYGZhy{}}\PYG{l+m+mf}{0.33469985340998853}\PYG{p}{,} \PYG{l+m+mf}{0.1307636433702345}\PYG{p}{]}\PYG{p}{,}
 \PYG{l+s+s1}{\PYGZsq{}}\PYG{l+s+s1}{EEctl}\PYG{l+s+s1}{\PYGZsq{}}\PYG{p}{:} \PYG{p}{[}\PYG{o}{\PYGZhy{}}\PYG{l+m+mf}{0.48751243650229525}\PYG{p}{,} \PYG{l+m+mf}{0.4852032717825657}\PYG{p}{]}\PYG{p}{,}
 \PYG{l+s+s1}{\PYGZsq{}}\PYG{l+s+s1}{Ctl1}\PYG{l+s+s1}{\PYGZsq{}}\PYG{p}{:} \PYG{p}{[}\PYG{o}{\PYGZhy{}}\PYG{l+m+mf}{0.06913038312848868}\PYG{p}{,} \PYG{l+m+mf}{0.2445174568603189}\PYG{p}{]}\PYG{p}{,}
 \PYG{l+s+s1}{\PYGZsq{}}\PYG{l+s+s1}{DOFs}\PYG{l+s+s1}{\PYGZsq{}}\PYG{p}{:} \PYG{p}{[}\PYG{l+m+mf}{0.2606664304933561}\PYG{p}{,} \PYG{l+m+mf}{0.3243482171363975}\PYG{p}{]}\PYG{p}{,}
 \PYG{l+s+s1}{\PYGZsq{}}\PYG{l+s+s1}{Env1}\PYG{l+s+s1}{\PYGZsq{}}\PYG{p}{:} \PYG{p}{[}\PYG{l+m+mf}{0.06157634305459603}\PYG{p}{,} \PYG{l+m+mf}{0.7099922980251693}\PYG{p}{]}\PYG{p}{,}
 \PYG{l+s+s1}{\PYGZsq{}}\PYG{l+s+s1}{Dir1}\PYG{l+s+s1}{\PYGZsq{}}\PYG{p}{:} \PYG{p}{[}\PYG{o}{\PYGZhy{}}\PYG{l+m+mf}{0.13617863906968142}\PYG{p}{,} \PYG{l+m+mf}{0.6037252153639261}\PYG{p}{]}\PYG{p}{\PYGZcb{}}

\PYG{n}{graph\PYGZus{}pos} \PYG{o}{=} \PYG{p}{\PYGZob{}}\PYG{l+s+s1}{\PYGZsq{}}\PYG{l+s+s1}{StoreEE}\PYG{l+s+s1}{\PYGZsq{}}\PYG{p}{:} \PYG{p}{[}\PYG{o}{\PYGZhy{}}\PYG{l+m+mf}{1.0787279392101061}\PYG{p}{,} \PYG{o}{\PYGZhy{}}\PYG{l+m+mf}{0.06903523859088145}\PYG{p}{]}\PYG{p}{,}
 \PYG{l+s+s1}{\PYGZsq{}}\PYG{l+s+s1}{DistEE}\PYG{l+s+s1}{\PYGZsq{}}\PYG{p}{:} \PYG{p}{[}\PYG{o}{\PYGZhy{}}\PYG{l+m+mf}{0.361531174332526}\PYG{p}{,} \PYG{o}{\PYGZhy{}}\PYG{l+m+mf}{0.0935883732235363}\PYG{p}{]}\PYG{p}{,}
 \PYG{l+s+s1}{\PYGZsq{}}\PYG{l+s+s1}{AffectDOF}\PYG{l+s+s1}{\PYGZsq{}}\PYG{p}{:} \PYG{p}{[}\PYG{l+m+mf}{0.36541282312106205}\PYG{p}{,} \PYG{o}{\PYGZhy{}}\PYG{l+m+mf}{0.09674444529230719}\PYG{p}{]}\PYG{p}{,}
 \PYG{l+s+s1}{\PYGZsq{}}\PYG{l+s+s1}{CtlDOF}\PYG{l+s+s1}{\PYGZsq{}}\PYG{p}{:} \PYG{p}{[}\PYG{l+m+mf}{0.4664934329906758}\PYG{p}{,} \PYG{l+m+mf}{0.5822138245848214}\PYG{p}{]}\PYG{p}{,}
 \PYG{l+s+s1}{\PYGZsq{}}\PYG{l+s+s1}{Planpath}\PYG{l+s+s1}{\PYGZsq{}}\PYG{p}{:} \PYG{p}{[}\PYG{o}{\PYGZhy{}}\PYG{l+m+mf}{0.7095750728126631}\PYG{p}{,} \PYG{l+m+mf}{0.8482786785038505}\PYG{p}{]}\PYG{p}{,}
 \PYG{l+s+s1}{\PYGZsq{}}\PYG{l+s+s1}{Trajectory}\PYG{l+s+s1}{\PYGZsq{}}\PYG{p}{:} \PYG{p}{[}\PYG{l+m+mf}{1.1006824683444765}\PYG{p}{,} \PYG{o}{\PYGZhy{}}\PYG{l+m+mf}{0.10423208715241583}\PYG{p}{]}\PYG{p}{,}
 \PYG{l+s+s1}{\PYGZsq{}}\PYG{l+s+s1}{EngageLand}\PYG{l+s+s1}{\PYGZsq{}}\PYG{p}{:} \PYG{p}{[}\PYG{l+m+mf}{0.8423521094741182}\PYG{p}{,} \PYG{o}{\PYGZhy{}}\PYG{l+m+mf}{0.8813666134484857}\PYG{p}{]}\PYG{p}{,}
 \PYG{l+s+s1}{\PYGZsq{}}\PYG{l+s+s1}{HoldPayload}\PYG{l+s+s1}{\PYGZsq{}}\PYG{p}{:} \PYG{p}{[}\PYG{o}{\PYGZhy{}}\PYG{l+m+mf}{0.5857395187723944}\PYG{p}{,} \PYG{o}{\PYGZhy{}}\PYG{l+m+mf}{0.86974898769837}\PYG{p}{]}\PYG{p}{,}
 \PYG{l+s+s1}{\PYGZsq{}}\PYG{l+s+s1}{ViewEnv}\PYG{l+s+s1}{\PYGZsq{}}\PYG{p}{:} \PYG{p}{[}\PYG{l+m+mf}{1.1035500215472247}\PYG{p}{,} \PYG{l+m+mf}{0.9373523025760659}\PYG{p}{]}\PYG{p}{\PYGZcb{}}
\end{sphinxVerbatim}
}

\end{sphinxuseclass}
\end{sphinxuseclass}
\begin{sphinxuseclass}{nbinput}
{
\sphinxsetup{VerbatimColor={named}{nbsphinx-code-bg}}
\sphinxsetup{VerbatimBorderColor={named}{nbsphinx-code-border}}
\begin{sphinxVerbatim}[commandchars=\\\{\}]
\llap{\color{nbsphinxin}[7]:\,\hspace{\fboxrule}\hspace{\fboxsep}}\PYG{o}{\PYGZpc{}}\PYG{k}{matplotlib} inline
\PYG{n}{rd}\PYG{o}{.}\PYG{n}{graph}\PYG{o}{.}\PYG{n}{show}\PYG{p}{(}\PYG{n}{mdl}\PYG{p}{,} \PYG{n}{pos}\PYG{o}{=}\PYG{n}{graph\PYGZus{}pos}\PYG{p}{,} \PYG{n}{gtype}\PYG{o}{=}\PYG{l+s+s1}{\PYGZsq{}}\PYG{l+s+s1}{normal}\PYG{l+s+s1}{\PYGZsq{}}\PYG{p}{)}
\PYG{n}{rd}\PYG{o}{.}\PYG{n}{graph}\PYG{o}{.}\PYG{n}{show}\PYG{p}{(}\PYG{n}{mdl}\PYG{p}{,} \PYG{n}{pos}\PYG{o}{=}\PYG{n}{bipartite\PYGZus{}pos}\PYG{p}{)}
\end{sphinxVerbatim}
}

\end{sphinxuseclass}
\begin{sphinxuseclass}{nboutput}
{

\kern-\sphinxverbatimsmallskipamount\kern-\baselineskip
\kern+\FrameHeightAdjust\kern-\fboxrule
\vspace{\nbsphinxcodecellspacing}

\sphinxsetup{VerbatimColor={named}{white}}
\sphinxsetup{VerbatimBorderColor={named}{nbsphinx-code-border}}
\begin{sphinxuseclass}{output_area}
\begin{sphinxuseclass}{}


\begin{sphinxVerbatim}[commandchars=\\\{\}]
\llap{\color{nbsphinxout}[7]:\,\hspace{\fboxrule}\hspace{\fboxsep}}(<Figure size 432x288 with 1 Axes>, <AxesSubplot:>)
\end{sphinxVerbatim}



\end{sphinxuseclass}
\end{sphinxuseclass}
}

\end{sphinxuseclass}
\begin{sphinxuseclass}{nboutput}
\hrule height -\fboxrule\relax
\vspace{\nbsphinxcodecellspacing}

\makeatletter\setbox\nbsphinxpromptbox\box\voidb@x\makeatother

\begin{nbsphinxfancyoutput}

\begin{sphinxuseclass}{output_area}
\begin{sphinxuseclass}{}
\noindent\sphinxincludegraphics[width=349\sphinxpxdimen,height=231\sphinxpxdimen]{{example_multirotor_Demonstration_13_1}.png}

\end{sphinxuseclass}
\end{sphinxuseclass}
\end{nbsphinxfancyoutput}

\end{sphinxuseclass}
\begin{sphinxuseclass}{nboutput}
\begin{sphinxuseclass}{nblast}
\hrule height -\fboxrule\relax
\vspace{\nbsphinxcodecellspacing}

\makeatletter\setbox\nbsphinxpromptbox\box\voidb@x\makeatother

\begin{nbsphinxfancyoutput}

\begin{sphinxuseclass}{output_area}
\begin{sphinxuseclass}{}
\noindent\sphinxincludegraphics[width=349\sphinxpxdimen,height=231\sphinxpxdimen]{{example_multirotor_Demonstration_13_2}.png}

\end{sphinxuseclass}
\end{sphinxuseclass}
\end{nbsphinxfancyoutput}

\end{sphinxuseclass}
\end{sphinxuseclass}

\subsubsection{Network Model}
\label{\detokenize{example_multirotor/Demonstration:Network-Model}}
\sphinxAtStartPar
A network model can be used to compute network metrics and visualize network vulnerabilities.

\sphinxAtStartPar
We can calculate network metrics using calc\_aspl, calc\_modularity, and calc\_robustness\_coefficient in the \sphinxcode{\sphinxupquote{networks}} module.

\begin{sphinxuseclass}{nbinput}
{
\sphinxsetup{VerbatimColor={named}{nbsphinx-code-bg}}
\sphinxsetup{VerbatimBorderColor={named}{nbsphinx-code-border}}
\begin{sphinxVerbatim}[commandchars=\\\{\}]
\llap{\color{nbsphinxin}[8]:\,\hspace{\fboxrule}\hspace{\fboxsep}}\PYG{n}{aspl} \PYG{o}{=} \PYG{n}{fs}\PYG{o}{.}\PYG{n}{networks}\PYG{o}{.}\PYG{n}{calc\PYGZus{}aspl}\PYG{p}{(}\PYG{n}{mdl}\PYG{p}{,}\PYG{n}{gtype}\PYG{o}{=}\PYG{l+s+s1}{\PYGZsq{}}\PYG{l+s+s1}{normal}\PYG{l+s+s1}{\PYGZsq{}}\PYG{p}{)}
\PYG{n}{q} \PYG{o}{=} \PYG{n}{fs}\PYG{o}{.}\PYG{n}{networks}\PYG{o}{.}\PYG{n}{calc\PYGZus{}modularity}\PYG{p}{(}\PYG{n}{mdl}\PYG{p}{,}\PYG{n}{gtype}\PYG{o}{=}\PYG{l+s+s1}{\PYGZsq{}}\PYG{l+s+s1}{normal}\PYG{l+s+s1}{\PYGZsq{}}\PYG{p}{)}
\PYG{n}{rc} \PYG{o}{=} \PYG{n}{fs}\PYG{o}{.}\PYG{n}{networks}\PYG{o}{.}\PYG{n}{calc\PYGZus{}robustness\PYGZus{}coefficient}\PYG{p}{(}\PYG{n}{mdl}\PYG{p}{,}\PYG{n}{gtype}\PYG{o}{=}\PYG{l+s+s1}{\PYGZsq{}}\PYG{l+s+s1}{normal}\PYG{l+s+s1}{\PYGZsq{}}\PYG{p}{)}

\PYG{n+nb}{print}\PYG{p}{(}\PYG{l+s+s2}{\PYGZdq{}}\PYG{l+s+s2}{ASPL: }\PYG{l+s+si}{\PYGZpc{}.2f}\PYG{l+s+s2}{\PYGZdq{}} \PYG{o}{\PYGZpc{}} \PYG{n+nb}{round}\PYG{p}{(}\PYG{n}{aspl}\PYG{p}{,} \PYG{l+m+mi}{2}\PYG{p}{)}\PYG{p}{)}
\PYG{n+nb}{print}\PYG{p}{(}\PYG{l+s+s2}{\PYGZdq{}}\PYG{l+s+s2}{Modularity: }\PYG{l+s+si}{\PYGZpc{}.2f}\PYG{l+s+s2}{\PYGZdq{}} \PYG{o}{\PYGZpc{}} \PYG{n+nb}{round}\PYG{p}{(}\PYG{n}{q}\PYG{p}{,}\PYG{l+m+mi}{2}\PYG{p}{)}\PYG{p}{)}
\PYG{n+nb}{print}\PYG{p}{(}\PYG{l+s+s2}{\PYGZdq{}}\PYG{l+s+s2}{Robustness Coefficient: }\PYG{l+s+si}{\PYGZpc{}.2f}\PYG{l+s+s2}{\PYGZdq{}} \PYG{o}{\PYGZpc{}} \PYG{n+nb}{round}\PYG{p}{(}\PYG{n}{rc}\PYG{p}{,}\PYG{l+m+mi}{2}\PYG{p}{)}\PYG{p}{)}
\end{sphinxVerbatim}
}

\end{sphinxuseclass}
\begin{sphinxuseclass}{nboutput}
\begin{sphinxuseclass}{nblast}
{

\kern-\sphinxverbatimsmallskipamount\kern-\baselineskip
\kern+\FrameHeightAdjust\kern-\fboxrule
\vspace{\nbsphinxcodecellspacing}

\sphinxsetup{VerbatimColor={named}{white}}
\sphinxsetup{VerbatimBorderColor={named}{nbsphinx-code-border}}
\begin{sphinxuseclass}{output_area}
\begin{sphinxuseclass}{}


\begin{sphinxVerbatim}[commandchars=\\\{\}]
ASPL: 1.44
Modularity: 0.12
Robustness Coefficient: 95.21
\end{sphinxVerbatim}



\end{sphinxuseclass}
\end{sphinxuseclass}
}

\end{sphinxuseclass}
\end{sphinxuseclass}
\sphinxAtStartPar
Next, we visualize network vulnerabilities using find\_bridging\_nodes and find\_high\_degree\_nodes.

\begin{sphinxuseclass}{nbinput}
{
\sphinxsetup{VerbatimColor={named}{nbsphinx-code-bg}}
\sphinxsetup{VerbatimBorderColor={named}{nbsphinx-code-border}}
\begin{sphinxVerbatim}[commandchars=\\\{\}]
\llap{\color{nbsphinxin}[9]:\,\hspace{\fboxrule}\hspace{\fboxsep}}\PYG{p}{[}\PYG{n}{bridging\PYGZus{}nodes}\PYG{p}{,}\PYG{n}{fig\PYGZus{}bridging\PYGZus{}nodes}\PYG{p}{,}\PYG{n}{ax\PYGZus{}bridging\PYGZus{}nodes}\PYG{p}{]} \PYG{o}{=} \PYG{n}{fs}\PYG{o}{.}\PYG{n}{networks}\PYG{o}{.}\PYG{n}{find\PYGZus{}bridging\PYGZus{}nodes}\PYG{p}{(}\PYG{n}{mdl}\PYG{p}{,}\PYG{n}{plot}\PYG{o}{=}\PYG{l+s+s1}{\PYGZsq{}}\PYG{l+s+s1}{on}\PYG{l+s+s1}{\PYGZsq{}}\PYG{p}{,}\PYG{n}{gtype}\PYG{o}{=}\PYG{l+s+s1}{\PYGZsq{}}\PYG{l+s+s1}{normal}\PYG{l+s+s1}{\PYGZsq{}}\PYG{p}{,} \PYG{n}{pos}\PYG{o}{=}\PYG{n}{graph\PYGZus{}pos}\PYG{p}{)}
\PYG{n}{fig\PYGZus{}bridging\PYGZus{}nodes}\PYG{o}{.}\PYG{n}{savefig}\PYG{p}{(}\PYG{l+s+s1}{\PYGZsq{}}\PYG{l+s+s1}{bridgingnodes.pdf}\PYG{l+s+s1}{\PYGZsq{}}\PYG{p}{,} \PYG{n+nb}{format}\PYG{o}{=}\PYG{l+s+s2}{\PYGZdq{}}\PYG{l+s+s2}{pdf}\PYG{l+s+s2}{\PYGZdq{}}\PYG{p}{,} \PYG{n}{bbox\PYGZus{}inches} \PYG{o}{=} \PYG{l+s+s1}{\PYGZsq{}}\PYG{l+s+s1}{tight}\PYG{l+s+s1}{\PYGZsq{}}\PYG{p}{,} \PYG{n}{pad\PYGZus{}inches} \PYG{o}{=} \PYG{l+m+mi}{0}\PYG{p}{)}
\end{sphinxVerbatim}
}

\end{sphinxuseclass}
\begin{sphinxuseclass}{nboutput}
\begin{sphinxuseclass}{nblast}
\hrule height -\fboxrule\relax
\vspace{\nbsphinxcodecellspacing}

\makeatletter\setbox\nbsphinxpromptbox\box\voidb@x\makeatother

\begin{nbsphinxfancyoutput}

\begin{sphinxuseclass}{output_area}
\begin{sphinxuseclass}{}
\noindent\sphinxincludegraphics[width=349\sphinxpxdimen,height=247\sphinxpxdimen]{{example_multirotor_Demonstration_18_0}.png}

\end{sphinxuseclass}
\end{sphinxuseclass}
\end{nbsphinxfancyoutput}

\end{sphinxuseclass}
\end{sphinxuseclass}
\begin{sphinxuseclass}{nbinput}
{
\sphinxsetup{VerbatimColor={named}{nbsphinx-code-bg}}
\sphinxsetup{VerbatimBorderColor={named}{nbsphinx-code-border}}
\begin{sphinxVerbatim}[commandchars=\\\{\}]
\llap{\color{nbsphinxin}[10]:\,\hspace{\fboxrule}\hspace{\fboxsep}}\PYG{p}{[}\PYG{n}{high\PYGZus{}degree\PYGZus{}nodes}\PYG{p}{,}\PYG{n}{fig\PYGZus{}high\PYGZus{}degree\PYGZus{}nodes}\PYG{p}{,}\PYG{n}{ax\PYGZus{}high\PYGZus{}degree\PYGZus{}nodes}\PYG{p}{]} \PYG{o}{=} \PYG{n}{fs}\PYG{o}{.}\PYG{n}{networks}\PYG{o}{.}\PYG{n}{find\PYGZus{}high\PYGZus{}degree\PYGZus{}nodes}\PYG{p}{(}\PYG{n}{mdl}\PYG{p}{,}\PYG{n}{plot}\PYG{o}{=}\PYG{l+s+s1}{\PYGZsq{}}\PYG{l+s+s1}{on}\PYG{l+s+s1}{\PYGZsq{}}\PYG{p}{,}\PYG{n}{gtype}\PYG{o}{=}\PYG{l+s+s1}{\PYGZsq{}}\PYG{l+s+s1}{normal}\PYG{l+s+s1}{\PYGZsq{}}\PYG{p}{,} \PYG{n}{pos}\PYG{o}{=}\PYG{n}{graph\PYGZus{}pos}\PYG{p}{,}\PYG{n}{scale}\PYG{o}{=}\PYG{l+m+mf}{1.5}\PYG{p}{)}
\end{sphinxVerbatim}
}

\end{sphinxuseclass}
\begin{sphinxuseclass}{nboutput}
\begin{sphinxuseclass}{nblast}
\hrule height -\fboxrule\relax
\vspace{\nbsphinxcodecellspacing}

\makeatletter\setbox\nbsphinxpromptbox\box\voidb@x\makeatother

\begin{nbsphinxfancyoutput}

\begin{sphinxuseclass}{output_area}
\begin{sphinxuseclass}{}
\noindent\sphinxincludegraphics[width=349\sphinxpxdimen,height=247\sphinxpxdimen]{{example_multirotor_Demonstration_19_0}.png}

\end{sphinxuseclass}
\end{sphinxuseclass}
\end{nbsphinxfancyoutput}

\end{sphinxuseclass}
\end{sphinxuseclass}
\begin{sphinxuseclass}{nbinput}
{
\sphinxsetup{VerbatimColor={named}{nbsphinx-code-bg}}
\sphinxsetup{VerbatimBorderColor={named}{nbsphinx-code-border}}
\begin{sphinxVerbatim}[commandchars=\\\{\}]
\llap{\color{nbsphinxin}[11]:\,\hspace{\fboxrule}\hspace{\fboxsep}}\PYG{c+c1}{\PYGZsh{}fig\PYGZus{}high\PYGZus{}degree\PYGZus{}nodes.subplots\PYGZus{}adjust(left=0.2, right=1.3, top=1.0, bottom=0.2)}
\PYG{n}{ax\PYGZus{}high\PYGZus{}degree\PYGZus{}nodes}\PYG{o}{.}\PYG{n}{axis}\PYG{p}{(}\PYG{l+s+s2}{\PYGZdq{}}\PYG{l+s+s2}{off}\PYG{l+s+s2}{\PYGZdq{}}\PYG{p}{)}
\PYG{n}{ax\PYGZus{}high\PYGZus{}degree\PYGZus{}nodes}\PYG{o}{.}\PYG{n}{margins}\PYG{p}{(}\PYG{l+m+mf}{0.08}\PYG{p}{)}
\PYG{n}{fig\PYGZus{}high\PYGZus{}degree\PYGZus{}nodes}\PYG{o}{.}\PYG{n}{tight\PYGZus{}layout}\PYG{p}{(}\PYG{p}{)}
\PYG{n}{fig\PYGZus{}high\PYGZus{}degree\PYGZus{}nodes}
\end{sphinxVerbatim}
}

\end{sphinxuseclass}
\begin{sphinxuseclass}{nboutput}
\begin{sphinxuseclass}{nblast}
\hrule height -\fboxrule\relax
\vspace{\nbsphinxcodecellspacing}

\savebox\nbsphinxpromptbox[0pt][r]{\color{nbsphinxout}\Verb|\strut{[11]:}\,|}

\begin{nbsphinxfancyoutput}

\begin{sphinxuseclass}{output_area}
\begin{sphinxuseclass}{}
\noindent\sphinxincludegraphics[width=424\sphinxpxdimen,height=280\sphinxpxdimen]{{example_multirotor_Demonstration_20_0}.png}

\end{sphinxuseclass}
\end{sphinxuseclass}
\end{nbsphinxfancyoutput}

\end{sphinxuseclass}
\end{sphinxuseclass}
\begin{sphinxuseclass}{nbinput}
\begin{sphinxuseclass}{nblast}
{
\sphinxsetup{VerbatimColor={named}{nbsphinx-code-bg}}
\sphinxsetup{VerbatimBorderColor={named}{nbsphinx-code-border}}
\begin{sphinxVerbatim}[commandchars=\\\{\}]
\llap{\color{nbsphinxin}[12]:\,\hspace{\fboxrule}\hspace{\fboxsep}}\PYG{n}{fig\PYGZus{}high\PYGZus{}degree\PYGZus{}nodes}\PYG{o}{.}\PYG{n}{savefig}\PYG{p}{(}\PYG{l+s+s1}{\PYGZsq{}}\PYG{l+s+s1}{highdegreenodes.pdf}\PYG{l+s+s1}{\PYGZsq{}}\PYG{p}{,} \PYG{n+nb}{format}\PYG{o}{=}\PYG{l+s+s2}{\PYGZdq{}}\PYG{l+s+s2}{pdf}\PYG{l+s+s2}{\PYGZdq{}}\PYG{p}{,} \PYG{n}{bbox\PYGZus{}inches} \PYG{o}{=} \PYG{l+s+s1}{\PYGZsq{}}\PYG{l+s+s1}{tight}\PYG{l+s+s1}{\PYGZsq{}}\PYG{p}{,} \PYG{n}{pad\PYGZus{}inches} \PYG{o}{=} \PYG{l+m+mf}{0.0}\PYG{p}{)}
\end{sphinxVerbatim}
}

\end{sphinxuseclass}
\end{sphinxuseclass}
\sphinxAtStartPar
High degree nodes (along with their degrees) and bridging nodes are also obtainable as lists.

\begin{sphinxuseclass}{nbinput}
{
\sphinxsetup{VerbatimColor={named}{nbsphinx-code-bg}}
\sphinxsetup{VerbatimBorderColor={named}{nbsphinx-code-border}}
\begin{sphinxVerbatim}[commandchars=\\\{\}]
\llap{\color{nbsphinxin}[13]:\,\hspace{\fboxrule}\hspace{\fboxsep}}\PYG{n+nb}{print}\PYG{p}{(}\PYG{l+s+s1}{\PYGZsq{}}\PYG{l+s+s1}{Bridging Nodes:}\PYG{l+s+s1}{\PYGZsq{}}\PYG{p}{,}\PYG{n}{bridging\PYGZus{}nodes}\PYG{p}{)}
\PYG{n+nb}{print}\PYG{p}{(}\PYG{l+s+s1}{\PYGZsq{}}\PYG{l+s+s1}{High Degree Nodes:}\PYG{l+s+s1}{\PYGZsq{}}\PYG{p}{,}\PYG{n}{high\PYGZus{}degree\PYGZus{}nodes}\PYG{p}{)}
\end{sphinxVerbatim}
}

\end{sphinxuseclass}
\begin{sphinxuseclass}{nboutput}
\begin{sphinxuseclass}{nblast}
{

\kern-\sphinxverbatimsmallskipamount\kern-\baselineskip
\kern+\FrameHeightAdjust\kern-\fboxrule
\vspace{\nbsphinxcodecellspacing}

\sphinxsetup{VerbatimColor={named}{white}}
\sphinxsetup{VerbatimBorderColor={named}{nbsphinx-code-border}}
\begin{sphinxuseclass}{output_area}
\begin{sphinxuseclass}{}


\begin{sphinxVerbatim}[commandchars=\\\{\}]
Bridging Nodes: ['AffectDOF', 'CtlDOF', 'DistEE', 'EngageLand', 'HoldPayload', 'Planpath', 'StoreEE', 'Trajectory']
High Degree Nodes: [('CtlDOF', 6), ('Planpath', 6), ('HoldPayload', 6)]
\end{sphinxVerbatim}



\end{sphinxuseclass}
\end{sphinxuseclass}
}

\end{sphinxuseclass}
\end{sphinxuseclass}
\sphinxAtStartPar
Finally, we can plot the degree distribution of the network using degree\_dist.

\begin{sphinxuseclass}{nbinput}
{
\sphinxsetup{VerbatimColor={named}{nbsphinx-code-bg}}
\sphinxsetup{VerbatimBorderColor={named}{nbsphinx-code-border}}
\begin{sphinxVerbatim}[commandchars=\\\{\}]
\llap{\color{nbsphinxin}[14]:\,\hspace{\fboxrule}\hspace{\fboxsep}}\PYG{n}{fig} \PYG{o}{=} \PYG{n}{fs}\PYG{o}{.}\PYG{n}{networks}\PYG{o}{.}\PYG{n}{degree\PYGZus{}dist}\PYG{p}{(}\PYG{n}{mdl}\PYG{p}{,}\PYG{n}{gtype}\PYG{o}{=}\PYG{l+s+s1}{\PYGZsq{}}\PYG{l+s+s1}{normal}\PYG{l+s+s1}{\PYGZsq{}}\PYG{p}{)}
\PYG{n}{fig}\PYG{o}{.}\PYG{n}{savefig}\PYG{p}{(}\PYG{l+s+s1}{\PYGZsq{}}\PYG{l+s+s1}{degreedist.pdf}\PYG{l+s+s1}{\PYGZsq{}}\PYG{p}{,} \PYG{n+nb}{format}\PYG{o}{=}\PYG{l+s+s2}{\PYGZdq{}}\PYG{l+s+s2}{pdf}\PYG{l+s+s2}{\PYGZdq{}}\PYG{p}{,} \PYG{n}{bbox\PYGZus{}inches} \PYG{o}{=} \PYG{l+s+s1}{\PYGZsq{}}\PYG{l+s+s1}{tight}\PYG{l+s+s1}{\PYGZsq{}}\PYG{p}{,} \PYG{n}{pad\PYGZus{}inches} \PYG{o}{=} \PYG{l+m+mf}{0.0}\PYG{p}{)}
\end{sphinxVerbatim}
}

\end{sphinxuseclass}
\begin{sphinxuseclass}{nboutput}
\begin{sphinxuseclass}{nblast}
\hrule height -\fboxrule\relax
\vspace{\nbsphinxcodecellspacing}

\makeatletter\setbox\nbsphinxpromptbox\box\voidb@x\makeatother

\begin{nbsphinxfancyoutput}

\begin{sphinxuseclass}{output_area}
\begin{sphinxuseclass}{}
\noindent\sphinxincludegraphics[width=376\sphinxpxdimen,height=278\sphinxpxdimen]{{example_multirotor_Demonstration_25_0}.png}

\end{sphinxuseclass}
\end{sphinxuseclass}
\end{nbsphinxfancyoutput}

\end{sphinxuseclass}
\end{sphinxuseclass}
\sphinxAtStartPar
The above analysis includes only function nodes. It is also possible to treat the bipartite graph (containing both functions and flows) as a unipartite\sphinxhyphen{}like graph and perform similar analysis on both function and flow nodes.

\begin{sphinxuseclass}{nbinput}
{
\sphinxsetup{VerbatimColor={named}{nbsphinx-code-bg}}
\sphinxsetup{VerbatimBorderColor={named}{nbsphinx-code-border}}
\begin{sphinxVerbatim}[commandchars=\\\{\}]
\llap{\color{nbsphinxin}[15]:\,\hspace{\fboxrule}\hspace{\fboxsep}}\PYG{n}{aspl} \PYG{o}{=} \PYG{n}{fs}\PYG{o}{.}\PYG{n}{networks}\PYG{o}{.}\PYG{n}{calc\PYGZus{}aspl}\PYG{p}{(}\PYG{n}{mdl}\PYG{p}{,}\PYG{n}{gtype}\PYG{o}{=}\PYG{l+s+s1}{\PYGZsq{}}\PYG{l+s+s1}{bipartite}\PYG{l+s+s1}{\PYGZsq{}}\PYG{p}{)}
\PYG{n}{q} \PYG{o}{=} \PYG{n}{fs}\PYG{o}{.}\PYG{n}{networks}\PYG{o}{.}\PYG{n}{calc\PYGZus{}modularity}\PYG{p}{(}\PYG{n}{mdl}\PYG{p}{,}\PYG{n}{gtype}\PYG{o}{=}\PYG{l+s+s1}{\PYGZsq{}}\PYG{l+s+s1}{bipartite}\PYG{l+s+s1}{\PYGZsq{}}\PYG{p}{)}
\PYG{n}{rc} \PYG{o}{=} \PYG{n}{fs}\PYG{o}{.}\PYG{n}{networks}\PYG{o}{.}\PYG{n}{calc\PYGZus{}robustness\PYGZus{}coefficient}\PYG{p}{(}\PYG{n}{mdl}\PYG{p}{,}\PYG{n}{gtype}\PYG{o}{=}\PYG{l+s+s1}{\PYGZsq{}}\PYG{l+s+s1}{bipartite}\PYG{l+s+s1}{\PYGZsq{}}\PYG{p}{)}

\PYG{n+nb}{print}\PYG{p}{(}\PYG{l+s+s2}{\PYGZdq{}}\PYG{l+s+s2}{ASPL, functions and flows: }\PYG{l+s+si}{\PYGZpc{}.2f}\PYG{l+s+s2}{\PYGZdq{}} \PYG{o}{\PYGZpc{}} \PYG{n+nb}{round}\PYG{p}{(}\PYG{n}{aspl}\PYG{p}{,} \PYG{l+m+mi}{2}\PYG{p}{)}\PYG{p}{)}
\PYG{n+nb}{print}\PYG{p}{(}\PYG{l+s+s2}{\PYGZdq{}}\PYG{l+s+s2}{Modularity, functions and flows: }\PYG{l+s+si}{\PYGZpc{}.2f}\PYG{l+s+s2}{\PYGZdq{}} \PYG{o}{\PYGZpc{}} \PYG{n+nb}{round}\PYG{p}{(}\PYG{n}{q}\PYG{p}{,}\PYG{l+m+mi}{2}\PYG{p}{)}\PYG{p}{)}
\PYG{n+nb}{print}\PYG{p}{(}\PYG{l+s+s2}{\PYGZdq{}}\PYG{l+s+s2}{Robustness Coefficient, functions and flows: }\PYG{l+s+si}{\PYGZpc{}.2f}\PYG{l+s+s2}{\PYGZdq{}} \PYG{o}{\PYGZpc{}} \PYG{n+nb}{round}\PYG{p}{(}\PYG{n}{rc}\PYG{p}{,}\PYG{l+m+mi}{2}\PYG{p}{)}\PYG{p}{)}
\end{sphinxVerbatim}
}

\end{sphinxuseclass}
\begin{sphinxuseclass}{nboutput}
\begin{sphinxuseclass}{nblast}
{

\kern-\sphinxverbatimsmallskipamount\kern-\baselineskip
\kern+\FrameHeightAdjust\kern-\fboxrule
\vspace{\nbsphinxcodecellspacing}

\sphinxsetup{VerbatimColor={named}{white}}
\sphinxsetup{VerbatimBorderColor={named}{nbsphinx-code-border}}
\begin{sphinxuseclass}{output_area}
\begin{sphinxuseclass}{}


\begin{sphinxVerbatim}[commandchars=\\\{\}]
ASPL, functions and flows: 2.77
Modularity, functions and flows: 0.35
Robustness Coefficient, functions and flows: 79.77
\end{sphinxVerbatim}



\end{sphinxuseclass}
\end{sphinxuseclass}
}

\end{sphinxuseclass}
\end{sphinxuseclass}
\begin{sphinxuseclass}{nbinput}
{
\sphinxsetup{VerbatimColor={named}{nbsphinx-code-bg}}
\sphinxsetup{VerbatimBorderColor={named}{nbsphinx-code-border}}
\begin{sphinxVerbatim}[commandchars=\\\{\}]
\llap{\color{nbsphinxin}[16]:\,\hspace{\fboxrule}\hspace{\fboxsep}}\PYG{p}{[}\PYG{n}{bridging\PYGZus{}nodes}\PYG{p}{,}\PYG{n}{fig\PYGZus{}bridging\PYGZus{}nodes}\PYG{p}{,}\PYG{n}{ax\PYGZus{}bridging\PYGZus{}nodes}\PYG{p}{]} \PYG{o}{=} \PYG{n}{fs}\PYG{o}{.}\PYG{n}{networks}\PYG{o}{.}\PYG{n}{find\PYGZus{}bridging\PYGZus{}nodes}\PYG{p}{(}\PYG{n}{mdl}\PYG{p}{,}\PYG{n}{plot}\PYG{o}{=}\PYG{l+s+s1}{\PYGZsq{}}\PYG{l+s+s1}{on}\PYG{l+s+s1}{\PYGZsq{}}\PYG{p}{,}\PYG{n}{gtype}\PYG{o}{=}\PYG{l+s+s1}{\PYGZsq{}}\PYG{l+s+s1}{bipartite}\PYG{l+s+s1}{\PYGZsq{}}\PYG{p}{,} \PYG{n}{pos}\PYG{o}{=}\PYG{n}{bipartite\PYGZus{}pos}\PYG{p}{)}
\PYG{p}{[}\PYG{n}{high\PYGZus{}degree\PYGZus{}nodes}\PYG{p}{,}\PYG{n}{fig\PYGZus{}high\PYGZus{}degree\PYGZus{}nodes}\PYG{p}{,}\PYG{n}{ax\PYGZus{}high\PYGZus{}degree\PYGZus{}nodes}\PYG{p}{]} \PYG{o}{=} \PYG{n}{fs}\PYG{o}{.}\PYG{n}{networks}\PYG{o}{.}\PYG{n}{find\PYGZus{}high\PYGZus{}degree\PYGZus{}nodes}\PYG{p}{(}\PYG{n}{mdl}\PYG{p}{,}\PYG{n}{plot}\PYG{o}{=}\PYG{l+s+s1}{\PYGZsq{}}\PYG{l+s+s1}{on}\PYG{l+s+s1}{\PYGZsq{}}\PYG{p}{,}\PYG{n}{gtype}\PYG{o}{=}\PYG{l+s+s1}{\PYGZsq{}}\PYG{l+s+s1}{bipartite}\PYG{l+s+s1}{\PYGZsq{}}\PYG{p}{,} \PYG{n}{pos}\PYG{o}{=}\PYG{n}{bipartite\PYGZus{}pos}\PYG{p}{)}
\end{sphinxVerbatim}
}

\end{sphinxuseclass}
\begin{sphinxuseclass}{nboutput}
\hrule height -\fboxrule\relax
\vspace{\nbsphinxcodecellspacing}

\makeatletter\setbox\nbsphinxpromptbox\box\voidb@x\makeatother

\begin{nbsphinxfancyoutput}

\begin{sphinxuseclass}{output_area}
\begin{sphinxuseclass}{}
\noindent\sphinxincludegraphics[width=349\sphinxpxdimen,height=247\sphinxpxdimen]{{example_multirotor_Demonstration_28_0}.png}

\end{sphinxuseclass}
\end{sphinxuseclass}
\end{nbsphinxfancyoutput}

\end{sphinxuseclass}
\begin{sphinxuseclass}{nboutput}
\begin{sphinxuseclass}{nblast}
\hrule height -\fboxrule\relax
\vspace{\nbsphinxcodecellspacing}

\makeatletter\setbox\nbsphinxpromptbox\box\voidb@x\makeatother

\begin{nbsphinxfancyoutput}

\begin{sphinxuseclass}{output_area}
\begin{sphinxuseclass}{}
\noindent\sphinxincludegraphics[width=349\sphinxpxdimen,height=247\sphinxpxdimen]{{example_multirotor_Demonstration_28_1}.png}

\end{sphinxuseclass}
\end{sphinxuseclass}
\end{nbsphinxfancyoutput}

\end{sphinxuseclass}
\end{sphinxuseclass}
\begin{sphinxuseclass}{nbinput}
{
\sphinxsetup{VerbatimColor={named}{nbsphinx-code-bg}}
\sphinxsetup{VerbatimBorderColor={named}{nbsphinx-code-border}}
\begin{sphinxVerbatim}[commandchars=\\\{\}]
\llap{\color{nbsphinxin}[17]:\,\hspace{\fboxrule}\hspace{\fboxsep}}\PYG{n+nb}{print}\PYG{p}{(}\PYG{l+s+s1}{\PYGZsq{}}\PYG{l+s+s1}{Bridging Nodes:}\PYG{l+s+s1}{\PYGZsq{}}\PYG{p}{,}\PYG{n}{bridging\PYGZus{}nodes}\PYG{p}{)}
\PYG{n+nb}{print}\PYG{p}{(}\PYG{l+s+s1}{\PYGZsq{}}\PYG{l+s+s1}{High Degree Nodes:}\PYG{l+s+s1}{\PYGZsq{}}\PYG{p}{,}\PYG{n}{high\PYGZus{}degree\PYGZus{}nodes}\PYG{p}{)}
\end{sphinxVerbatim}
}

\end{sphinxuseclass}
\begin{sphinxuseclass}{nboutput}
\begin{sphinxuseclass}{nblast}
{

\kern-\sphinxverbatimsmallskipamount\kern-\baselineskip
\kern+\FrameHeightAdjust\kern-\fboxrule
\vspace{\nbsphinxcodecellspacing}

\sphinxsetup{VerbatimColor={named}{white}}
\sphinxsetup{VerbatimBorderColor={named}{nbsphinx-code-border}}
\begin{sphinxuseclass}{output_area}
\begin{sphinxuseclass}{}


\begin{sphinxVerbatim}[commandchars=\\\{\}]
Bridging Nodes: ['AffectDOF', 'Ctl1', 'CtlDOF', 'DOFs', 'Dir1', 'DistEE', 'EEctl', 'EEmot', 'Env1', 'Force\_GR', 'Force\_Lin', 'Force\_ST', 'HoldPayload', 'Planpath', 'StoreEE', 'Trajectory']
High Degree Nodes: [('CtlDOF', 5), ('Force\_ST', 5)]
\end{sphinxVerbatim}



\end{sphinxuseclass}
\end{sphinxuseclass}
}

\end{sphinxuseclass}
\end{sphinxuseclass}
\begin{sphinxuseclass}{nbinput}
{
\sphinxsetup{VerbatimColor={named}{nbsphinx-code-bg}}
\sphinxsetup{VerbatimBorderColor={named}{nbsphinx-code-border}}
\begin{sphinxVerbatim}[commandchars=\\\{\}]
\llap{\color{nbsphinxin}[18]:\,\hspace{\fboxrule}\hspace{\fboxsep}}\PYG{n}{fs}\PYG{o}{.}\PYG{n}{networks}\PYG{o}{.}\PYG{n}{degree\PYGZus{}dist}\PYG{p}{(}\PYG{n}{mdl}\PYG{p}{,}\PYG{n}{gtype}\PYG{o}{=}\PYG{l+s+s1}{\PYGZsq{}}\PYG{l+s+s1}{bipartite}\PYG{l+s+s1}{\PYGZsq{}}\PYG{p}{)}
\end{sphinxVerbatim}
}

\end{sphinxuseclass}
\begin{sphinxuseclass}{nboutput}
\hrule height -\fboxrule\relax
\vspace{\nbsphinxcodecellspacing}

\makeatletter\setbox\nbsphinxpromptbox\box\voidb@x\makeatother

\begin{nbsphinxfancyoutput}

\begin{sphinxuseclass}{output_area}
\begin{sphinxuseclass}{}
\noindent\sphinxincludegraphics[width=376\sphinxpxdimen,height=278\sphinxpxdimen]{{example_multirotor_Demonstration_30_0}.png}

\end{sphinxuseclass}
\end{sphinxuseclass}
\end{nbsphinxfancyoutput}

\end{sphinxuseclass}
\begin{sphinxuseclass}{nboutput}
\begin{sphinxuseclass}{nblast}
\hrule height -\fboxrule\relax
\vspace{\nbsphinxcodecellspacing}

\savebox\nbsphinxpromptbox[0pt][r]{\color{nbsphinxout}\Verb|\strut{[18]:}\,|}

\begin{nbsphinxfancyoutput}

\begin{sphinxuseclass}{output_area}
\begin{sphinxuseclass}{}
\noindent\sphinxincludegraphics[width=376\sphinxpxdimen,height=278\sphinxpxdimen]{{example_multirotor_Demonstration_30_1}.png}

\end{sphinxuseclass}
\end{sphinxuseclass}
\end{nbsphinxfancyoutput}

\end{sphinxuseclass}
\end{sphinxuseclass}
\sphinxAtStartPar
The SFF model can be simulated with options for simulation time, infection (failure) rate, and recovery (fix) rate. The start node can be selected or chosen randomly. Plotting includes an option for error bars. This models the system’s response to a failure using an analogy of an epidemic spreading through a network.

\begin{sphinxuseclass}{nbinput}
{
\sphinxsetup{VerbatimColor={named}{nbsphinx-code-bg}}
\sphinxsetup{VerbatimBorderColor={named}{nbsphinx-code-border}}
\begin{sphinxVerbatim}[commandchars=\\\{\}]
\llap{\color{nbsphinxin}[19]:\,\hspace{\fboxrule}\hspace{\fboxsep}}\PYG{n}{fig}\PYG{o}{=}\PYG{n}{fs}\PYG{o}{.}\PYG{n}{networks}\PYG{o}{.}\PYG{n}{sff\PYGZus{}model}\PYG{p}{(}\PYG{n}{mdl}\PYG{p}{,}\PYG{n}{gtype}\PYG{o}{=}\PYG{l+s+s1}{\PYGZsq{}}\PYG{l+s+s1}{normal}\PYG{l+s+s1}{\PYGZsq{}}\PYG{p}{,}\PYG{n}{endtime}\PYG{o}{=}\PYG{l+m+mi}{15}\PYG{p}{,}\PYG{n}{pi}\PYG{o}{=}\PYG{l+m+mf}{.3}\PYG{p}{,}\PYG{n}{pr}\PYG{o}{=}\PYG{l+m+mf}{.1}\PYG{p}{,}\PYG{n}{start\PYGZus{}node}\PYG{o}{=}\PYG{l+s+s1}{\PYGZsq{}}\PYG{l+s+s1}{AffectDOF}\PYG{l+s+s1}{\PYGZsq{}}\PYG{p}{,}\PYG{n}{error\PYGZus{}bar\PYGZus{}option}\PYG{o}{=}\PYG{l+s+s1}{\PYGZsq{}}\PYG{l+s+s1}{on}\PYG{l+s+s1}{\PYGZsq{}}\PYG{p}{)}
\PYG{n}{fig}\PYG{o}{.}\PYG{n}{savefig}\PYG{p}{(}\PYG{l+s+s1}{\PYGZsq{}}\PYG{l+s+s1}{sff\PYGZus{}model.pdf}\PYG{l+s+s1}{\PYGZsq{}}\PYG{p}{,} \PYG{n+nb}{format}\PYG{o}{=}\PYG{l+s+s2}{\PYGZdq{}}\PYG{l+s+s2}{pdf}\PYG{l+s+s2}{\PYGZdq{}}\PYG{p}{,} \PYG{n}{bbox\PYGZus{}inches} \PYG{o}{=} \PYG{l+s+s1}{\PYGZsq{}}\PYG{l+s+s1}{tight}\PYG{l+s+s1}{\PYGZsq{}}\PYG{p}{,} \PYG{n}{pad\PYGZus{}inches} \PYG{o}{=} \PYG{l+m+mi}{0}\PYG{p}{)}
\end{sphinxVerbatim}
}

\end{sphinxuseclass}
\begin{sphinxuseclass}{nboutput}
\begin{sphinxuseclass}{nblast}
\hrule height -\fboxrule\relax
\vspace{\nbsphinxcodecellspacing}

\makeatletter\setbox\nbsphinxpromptbox\box\voidb@x\makeatother

\begin{nbsphinxfancyoutput}

\begin{sphinxuseclass}{output_area}
\begin{sphinxuseclass}{}
\noindent\sphinxincludegraphics[width=376\sphinxpxdimen,height=278\sphinxpxdimen]{{example_multirotor_Demonstration_32_0}.png}

\end{sphinxuseclass}
\end{sphinxuseclass}
\end{nbsphinxfancyoutput}

\end{sphinxuseclass}
\end{sphinxuseclass}
\begin{sphinxuseclass}{nbinput}
{
\sphinxsetup{VerbatimColor={named}{nbsphinx-code-bg}}
\sphinxsetup{VerbatimBorderColor={named}{nbsphinx-code-border}}
\begin{sphinxVerbatim}[commandchars=\\\{\}]
\llap{\color{nbsphinxin}[20]:\,\hspace{\fboxrule}\hspace{\fboxsep}}\PYG{n}{fig}\PYG{o}{=}\PYG{n}{fs}\PYG{o}{.}\PYG{n}{networks}\PYG{o}{.}\PYG{n}{sff\PYGZus{}model}\PYG{p}{(}\PYG{n}{mdl}\PYG{p}{,}\PYG{n}{gtype}\PYG{o}{=}\PYG{l+s+s1}{\PYGZsq{}}\PYG{l+s+s1}{normal}\PYG{l+s+s1}{\PYGZsq{}}\PYG{p}{,}\PYG{n}{endtime}\PYG{o}{=}\PYG{l+m+mi}{15}\PYG{p}{,}\PYG{n}{pi}\PYG{o}{=}\PYG{l+m+mf}{.3}\PYG{p}{,}\PYG{n}{pr}\PYG{o}{=}\PYG{l+m+mf}{.1}\PYG{p}{,}\PYG{n}{start\PYGZus{}node}\PYG{o}{=}\PYG{l+s+s1}{\PYGZsq{}}\PYG{l+s+s1}{random}\PYG{l+s+s1}{\PYGZsq{}}\PYG{p}{,}\PYG{n}{error\PYGZus{}bar\PYGZus{}option}\PYG{o}{=}\PYG{l+s+s1}{\PYGZsq{}}\PYG{l+s+s1}{on}\PYG{l+s+s1}{\PYGZsq{}}\PYG{p}{)}
\end{sphinxVerbatim}
}

\end{sphinxuseclass}
\begin{sphinxuseclass}{nboutput}
\begin{sphinxuseclass}{nblast}
\hrule height -\fboxrule\relax
\vspace{\nbsphinxcodecellspacing}

\makeatletter\setbox\nbsphinxpromptbox\box\voidb@x\makeatother

\begin{nbsphinxfancyoutput}

\begin{sphinxuseclass}{output_area}
\begin{sphinxuseclass}{}
\noindent\sphinxincludegraphics[width=376\sphinxpxdimen,height=278\sphinxpxdimen]{{example_multirotor_Demonstration_33_0}.png}

\end{sphinxuseclass}
\end{sphinxuseclass}
\end{nbsphinxfancyoutput}

\end{sphinxuseclass}
\end{sphinxuseclass}

\subsubsection{Static Model}
\label{\detokenize{example_multirotor/Demonstration:Static-Model}}
\sphinxAtStartPar
In this demonstration, we will use a static representation of the system model to displaygraph views of fault scenarios and produce a static FMEA

\sphinxAtStartPar
The static model is located in \sphinxcode{\sphinxupquote{drone\_mdl\_static.py}}.

\begin{sphinxuseclass}{nbinput}
\begin{sphinxuseclass}{nblast}
{
\sphinxsetup{VerbatimColor={named}{nbsphinx-code-bg}}
\sphinxsetup{VerbatimBorderColor={named}{nbsphinx-code-border}}
\begin{sphinxVerbatim}[commandchars=\\\{\}]
\llap{\color{nbsphinxin}[21]:\,\hspace{\fboxrule}\hspace{\fboxsep}}\PYG{k+kn}{from} \PYG{n+nn}{drone\PYGZus{}mdl\PYGZus{}static} \PYG{k+kn}{import} \PYG{n}{Drone} \PYG{k}{as} \PYG{n}{Drone\PYGZus{}Static}
\end{sphinxVerbatim}
}

\end{sphinxuseclass}
\end{sphinxuseclass}
\sphinxAtStartPar
In design, it often helps to quantify the relative impact of fault scenarios. Here we produce a scenario\sphinxhyphen{}based FMEA to show which scenarios are most important in the model:

\begin{sphinxuseclass}{nbinput}
{
\sphinxsetup{VerbatimColor={named}{nbsphinx-code-bg}}
\sphinxsetup{VerbatimBorderColor={named}{nbsphinx-code-border}}
\begin{sphinxVerbatim}[commandchars=\\\{\}]
\llap{\color{nbsphinxin}[22]:\,\hspace{\fboxrule}\hspace{\fboxsep}}\PYG{n}{static\PYGZus{}mdl} \PYG{o}{=} \PYG{n}{Drone\PYGZus{}Static}\PYG{p}{(}\PYG{n}{params}\PYG{o}{=}\PYG{p}{\PYGZob{}}\PYG{l+s+s1}{\PYGZsq{}}\PYG{l+s+s1}{graph\PYGZus{}pos}\PYG{l+s+s1}{\PYGZsq{}}\PYG{p}{:}\PYG{n}{graph\PYGZus{}pos}\PYG{p}{,} \PYG{l+s+s1}{\PYGZsq{}}\PYG{l+s+s1}{bipartite\PYGZus{}pos}\PYG{l+s+s1}{\PYGZsq{}}\PYG{p}{:}\PYG{n}{bipartite\PYGZus{}pos}\PYG{p}{\PYGZcb{}}\PYG{p}{)}
\PYG{n}{endclasses}\PYG{p}{,} \PYG{n}{mdlhists} \PYG{o}{=} \PYG{n}{fs}\PYG{o}{.}\PYG{n}{propagate}\PYG{o}{.}\PYG{n}{single\PYGZus{}faults}\PYG{p}{(}\PYG{n}{static\PYGZus{}mdl}\PYG{p}{)}
\end{sphinxVerbatim}
}

\end{sphinxuseclass}
\begin{sphinxuseclass}{nboutput}
\begin{sphinxuseclass}{nblast}
{

\kern-\sphinxverbatimsmallskipamount\kern-\baselineskip
\kern+\FrameHeightAdjust\kern-\fboxrule
\vspace{\nbsphinxcodecellspacing}

\sphinxsetup{VerbatimColor={named}{nbsphinx-stderr}}
\sphinxsetup{VerbatimBorderColor={named}{nbsphinx-code-border}}
\begin{sphinxuseclass}{output_area}
\begin{sphinxuseclass}{stderr}


\begin{sphinxVerbatim}[commandchars=\\\{\}]
SCENARIOS COMPLETE: 100\%|█████████████████████████████████████████████████████████████| 25/25 [00:00<00:00, 363.30it/s]
\end{sphinxVerbatim}



\end{sphinxuseclass}
\end{sphinxuseclass}
}

\end{sphinxuseclass}
\end{sphinxuseclass}
\begin{sphinxuseclass}{nbinput}
{
\sphinxsetup{VerbatimColor={named}{nbsphinx-code-bg}}
\sphinxsetup{VerbatimBorderColor={named}{nbsphinx-code-border}}
\begin{sphinxVerbatim}[commandchars=\\\{\}]
\llap{\color{nbsphinxin}[23]:\,\hspace{\fboxrule}\hspace{\fboxsep}}\PYG{n}{reshists}\PYG{p}{,} \PYG{n}{diffs}\PYG{p}{,} \PYG{n}{summaries} \PYG{o}{=} \PYG{n}{rd}\PYG{o}{.}\PYG{n}{process}\PYG{o}{.}\PYG{n}{hists}\PYG{p}{(}\PYG{n}{mdlhists}\PYG{p}{)}
\PYG{n}{static\PYGZus{}fmea} \PYG{o}{=} \PYG{n}{rd}\PYG{o}{.}\PYG{n}{tabulate}\PYG{o}{.}\PYG{n}{fullfmea}\PYG{p}{(}\PYG{n}{endclasses}\PYG{p}{,} \PYG{n}{summaries}\PYG{p}{)}
\PYG{n}{static\PYGZus{}fmea}\PYG{o}{.}\PYG{n}{sort\PYGZus{}values}\PYG{p}{(}\PYG{l+s+s1}{\PYGZsq{}}\PYG{l+s+s1}{expected cost}\PYG{l+s+s1}{\PYGZsq{}}\PYG{p}{,} \PYG{n}{ascending}\PYG{o}{=}\PYG{k+kc}{False}\PYG{p}{)}
\end{sphinxVerbatim}
}

\end{sphinxuseclass}
\begin{sphinxuseclass}{nboutput}
\begin{sphinxuseclass}{nblast}
{

\kern-\sphinxverbatimsmallskipamount\kern-\baselineskip
\kern+\FrameHeightAdjust\kern-\fboxrule
\vspace{\nbsphinxcodecellspacing}

\sphinxsetup{VerbatimColor={named}{white}}
\sphinxsetup{VerbatimBorderColor={named}{nbsphinx-code-border}}
\begin{sphinxuseclass}{output_area}
\begin{sphinxuseclass}{}


\begin{sphinxVerbatim}[commandchars=\\\{\}]
\llap{\color{nbsphinxout}[23]:\,\hspace{\fboxrule}\hspace{\fboxsep}}                                                            degraded functions  \textbackslash{}
StoreEE nocharge, t=1        [StoreEE, DistEE, CtlDOF, Planpath, Trajectory{\ldots}
Planpath degloc, t=1         [DistEE, CtlDOF, Planpath, Trajectory, EngageL{\ldots}
DistEE short, t=1            [DistEE, CtlDOF, Planpath, Trajectory, EngageL{\ldots}
AffectDOF ctlbreak, t=1      [DistEE, AffectDOF, CtlDOF, Planpath, Trajecto{\ldots}
AffectDOF ctlup, t=1         [DistEE, AffectDOF, CtlDOF, Planpath, Trajecto{\ldots}
DistEE break, t=1            [DistEE, CtlDOF, Planpath, Trajectory, EngageL{\ldots}
CtlDOF noctl, t=1            [DistEE, CtlDOF, Planpath, Trajectory, EngageL{\ldots}
AffectDOF short, t=1         [DistEE, AffectDOF, CtlDOF, Planpath, Trajecto{\ldots}
AffectDOF mechbreak, t=1     [DistEE, AffectDOF, CtlDOF, Planpath, Trajecto{\ldots}
AffectDOF openc, t=1         [DistEE, AffectDOF, CtlDOF, Planpath, Trajecto{\ldots}
Planpath noloc, t=1                                     [Planpath, Trajectory]
CtlDOF degctl, t=1                                                    [CtlDOF]
AffectDOF propbreak, t=1     [DistEE, AffectDOF, CtlDOF, Planpath, Trajecto{\ldots}
AffectDOF propstuck, t=1     [DistEE, AffectDOF, CtlDOF, Planpath, Trajecto{\ldots}
HoldPayload break, t=1       [DistEE, CtlDOF, Planpath, Trajectory, EngageL{\ldots}
ViewEnv poorview, t=1                                                [ViewEnv]
EngageLand deform, t=1                                            [EngageLand]
HoldPayload deform, t=1                                          [HoldPayload]
DistEE degr, t=1                                                      [DistEE]
EngageLand break, t=1                                             [EngageLand]
AffectDOF ctldn, t=1                                               [AffectDOF]
AffectDOF mechfriction, t=1                                        [AffectDOF]
AffectDOF propwarp, t=1                                            [AffectDOF]
Trajectory crash, t=1                                             [Trajectory]
Trajectory lost, t=1                                              [Trajectory]

                                                                degraded flows  \textbackslash{}
StoreEE nocharge, t=1        [Force\_ST, Force\_Lin, Force\_GR, Force\_LG, EE\_1{\ldots}
Planpath degloc, t=1         [Force\_ST, Force\_Lin, Force\_GR, Force\_LG, EEmo{\ldots}
DistEE short, t=1            [Force\_ST, Force\_Lin, Force\_GR, Force\_LG, EEmo{\ldots}
AffectDOF ctlbreak, t=1      [Force\_ST, Force\_Lin, Force\_GR, Force\_LG, EEmo{\ldots}
AffectDOF ctlup, t=1         [Force\_ST, Force\_Lin, Force\_GR, Force\_LG, EEmo{\ldots}
DistEE break, t=1            [Force\_ST, Force\_Lin, Force\_GR, Force\_LG, EEmo{\ldots}
CtlDOF noctl, t=1            [Force\_ST, Force\_Lin, Force\_GR, Force\_LG, EEmo{\ldots}
AffectDOF short, t=1         [Force\_ST, Force\_Lin, Force\_GR, Force\_LG, EE\_1{\ldots}
AffectDOF mechbreak, t=1     [Force\_ST, Force\_Lin, Force\_GR, Force\_LG, EEmo{\ldots}
AffectDOF openc, t=1         [Force\_ST, Force\_Lin, Force\_GR, Force\_LG, EEmo{\ldots}
Planpath noloc, t=1                                         [Ctl1, DOFs, Dir1]
CtlDOF degctl, t=1                            [Force\_GR, Force\_LG, Ctl1, DOFs]
AffectDOF propbreak, t=1     [Force\_ST, Force\_Lin, Force\_GR, Force\_LG, EEmo{\ldots}
AffectDOF propstuck, t=1     [Force\_ST, Force\_Lin, Force\_GR, Force\_LG, EE\_1{\ldots}
HoldPayload break, t=1       [Force\_ST, Force\_Lin, Force\_GR, Force\_LG, EEmo{\ldots}
ViewEnv poorview, t=1                                                       []
EngageLand deform, t=1                                                      []
HoldPayload deform, t=1                                  [Force\_ST, Force\_Lin]
DistEE degr, t=1                [Force\_GR, Force\_LG, EEmot, EEctl, Ctl1, DOFs]
EngageLand break, t=1                                                       []
AffectDOF ctldn, t=1                                [Force\_GR, Force\_LG, DOFs]
AffectDOF mechfriction, t=1                                      [EE\_1, EEmot]
AffectDOF propwarp, t=1                             [Force\_GR, Force\_LG, DOFs]
Trajectory crash, t=1                                                       []
Trajectory lost, t=1                                                        []

                                 rate      cost expected cost
StoreEE nocharge, t=1         0.00001  183300.0      183300.0
Planpath degloc, t=1         0.000008  193000.0      154400.0
DistEE short, t=1            0.000003  186000.0       55800.0
AffectDOF ctlbreak, t=1      0.000002  184000.0       36800.0
AffectDOF ctlup, t=1         0.000002  183500.0       36700.0
DistEE break, t=1            0.000002  183000.0       36600.0
CtlDOF noctl, t=1            0.000002  183000.0       36600.0
AffectDOF short, t=1         0.000001  186200.0       18620.0
AffectDOF mechbreak, t=1     0.000001  183500.0       18350.0
AffectDOF openc, t=1         0.000001  183200.0       18320.0
Planpath noloc, t=1          0.000002   60000.0       12000.0
CtlDOF degctl, t=1           0.000008   10000.0        8000.0
AffectDOF propbreak, t=1          0.0  183200.0        5496.0
AffectDOF propstuck, t=1          0.0  186200.0        3724.0
HoldPayload break, t=1            0.0  183000.0        3660.0
ViewEnv poorview, t=1        0.000002   10000.0        2000.0
EngageLand deform, t=1       0.000008    1000.0         800.0
HoldPayload deform, t=1      0.000001   10000.0         800.0
DistEE degr, t=1             0.000005    1000.0         500.0
EngageLand break, t=1        0.000002    1000.0         200.0
AffectDOF ctldn, t=1         0.000002     500.0         100.0
AffectDOF mechfriction, t=1  0.000001     500.0          25.0
AffectDOF propwarp, t=1           0.0     200.0           2.0
Trajectory crash, t=1             0.0  100000.0           0.0
Trajectory lost, t=1              0.0   50000.0           0.0
\end{sphinxVerbatim}



\end{sphinxuseclass}
\end{sphinxuseclass}
}

\end{sphinxuseclass}
\end{sphinxuseclass}
\begin{sphinxuseclass}{nbinput}
{
\sphinxsetup{VerbatimColor={named}{nbsphinx-code-bg}}
\sphinxsetup{VerbatimBorderColor={named}{nbsphinx-code-border}}
\begin{sphinxVerbatim}[commandchars=\\\{\}]
\llap{\color{nbsphinxin}[24]:\,\hspace{\fboxrule}\hspace{\fboxsep}}\PYG{n+nb}{print}\PYG{p}{(}\PYG{n}{static\PYGZus{}fmea}\PYG{o}{.}\PYG{n}{sort\PYGZus{}values}\PYG{p}{(}\PYG{l+s+s1}{\PYGZsq{}}\PYG{l+s+s1}{expected cost}\PYG{l+s+s1}{\PYGZsq{}}\PYG{p}{,} \PYG{n}{ascending}\PYG{o}{=}\PYG{k+kc}{False}\PYG{p}{)}\PYG{o}{.}\PYG{n}{to\PYGZus{}latex}\PYG{p}{(}\PYG{p}{)}\PYG{p}{)}
\end{sphinxVerbatim}
}

\end{sphinxuseclass}
\begin{sphinxuseclass}{nboutput}
\begin{sphinxuseclass}{nblast}
{

\kern-\sphinxverbatimsmallskipamount\kern-\baselineskip
\kern+\FrameHeightAdjust\kern-\fboxrule
\vspace{\nbsphinxcodecellspacing}

\sphinxsetup{VerbatimColor={named}{white}}
\sphinxsetup{VerbatimBorderColor={named}{nbsphinx-code-border}}
\begin{sphinxuseclass}{output_area}
\begin{sphinxuseclass}{}


\begin{sphinxVerbatim}[commandchars=\\\{\}]
\textbackslash{}begin\{tabular\}\{llllll\}
\textbackslash{}toprule
\{\} \&                                 degraded functions \&                                     degraded flows \&      rate \&      cost \& expected cost \textbackslash{}\textbackslash{}
\textbackslash{}midrule
StoreEE nocharge, t=1       \&  [StoreEE, DistEE, CtlDOF, Planpath, Trajectory{\ldots} \&  [Force\textbackslash{}\_ST, Force\textbackslash{}\_Lin, Force\textbackslash{}\_GR, Force\textbackslash{}\_LG, EE\textbackslash{}\_1{\ldots} \&   0.00001 \&  183300.0 \&      183300.0 \textbackslash{}\textbackslash{}
Planpath degloc, t=1        \&  [DistEE, CtlDOF, Planpath, Trajectory, EngageL{\ldots} \&  [Force\textbackslash{}\_ST, Force\textbackslash{}\_Lin, Force\textbackslash{}\_GR, Force\textbackslash{}\_LG, EEmo{\ldots} \&  0.000008 \&  193000.0 \&      154400.0 \textbackslash{}\textbackslash{}
DistEE short, t=1           \&  [DistEE, CtlDOF, Planpath, Trajectory, EngageL{\ldots} \&  [Force\textbackslash{}\_ST, Force\textbackslash{}\_Lin, Force\textbackslash{}\_GR, Force\textbackslash{}\_LG, EEmo{\ldots} \&  0.000003 \&  186000.0 \&       55800.0 \textbackslash{}\textbackslash{}
AffectDOF ctlbreak, t=1     \&  [DistEE, AffectDOF, CtlDOF, Planpath, Trajecto{\ldots} \&  [Force\textbackslash{}\_ST, Force\textbackslash{}\_Lin, Force\textbackslash{}\_GR, Force\textbackslash{}\_LG, EEmo{\ldots} \&  0.000002 \&  184000.0 \&       36800.0 \textbackslash{}\textbackslash{}
AffectDOF ctlup, t=1        \&  [DistEE, AffectDOF, CtlDOF, Planpath, Trajecto{\ldots} \&  [Force\textbackslash{}\_ST, Force\textbackslash{}\_Lin, Force\textbackslash{}\_GR, Force\textbackslash{}\_LG, EEmo{\ldots} \&  0.000002 \&  183500.0 \&       36700.0 \textbackslash{}\textbackslash{}
DistEE break, t=1           \&  [DistEE, CtlDOF, Planpath, Trajectory, EngageL{\ldots} \&  [Force\textbackslash{}\_ST, Force\textbackslash{}\_Lin, Force\textbackslash{}\_GR, Force\textbackslash{}\_LG, EEmo{\ldots} \&  0.000002 \&  183000.0 \&       36600.0 \textbackslash{}\textbackslash{}
CtlDOF noctl, t=1           \&  [DistEE, CtlDOF, Planpath, Trajectory, EngageL{\ldots} \&  [Force\textbackslash{}\_ST, Force\textbackslash{}\_Lin, Force\textbackslash{}\_GR, Force\textbackslash{}\_LG, EEmo{\ldots} \&  0.000002 \&  183000.0 \&       36600.0 \textbackslash{}\textbackslash{}
AffectDOF short, t=1        \&  [DistEE, AffectDOF, CtlDOF, Planpath, Trajecto{\ldots} \&  [Force\textbackslash{}\_ST, Force\textbackslash{}\_Lin, Force\textbackslash{}\_GR, Force\textbackslash{}\_LG, EE\textbackslash{}\_1{\ldots} \&  0.000001 \&  186200.0 \&       18620.0 \textbackslash{}\textbackslash{}
AffectDOF mechbreak, t=1    \&  [DistEE, AffectDOF, CtlDOF, Planpath, Trajecto{\ldots} \&  [Force\textbackslash{}\_ST, Force\textbackslash{}\_Lin, Force\textbackslash{}\_GR, Force\textbackslash{}\_LG, EEmo{\ldots} \&  0.000001 \&  183500.0 \&       18350.0 \textbackslash{}\textbackslash{}
AffectDOF openc, t=1        \&  [DistEE, AffectDOF, CtlDOF, Planpath, Trajecto{\ldots} \&  [Force\textbackslash{}\_ST, Force\textbackslash{}\_Lin, Force\textbackslash{}\_GR, Force\textbackslash{}\_LG, EEmo{\ldots} \&  0.000001 \&  183200.0 \&       18320.0 \textbackslash{}\textbackslash{}
Planpath noloc, t=1         \&                             [Planpath, Trajectory] \&                                 [Ctl1, DOFs, Dir1] \&  0.000002 \&   60000.0 \&       12000.0 \textbackslash{}\textbackslash{}
CtlDOF degctl, t=1          \&                                           [CtlDOF] \&                   [Force\textbackslash{}\_GR, Force\textbackslash{}\_LG, Ctl1, DOFs] \&  0.000008 \&   10000.0 \&        8000.0 \textbackslash{}\textbackslash{}
AffectDOF propbreak, t=1    \&  [DistEE, AffectDOF, CtlDOF, Planpath, Trajecto{\ldots} \&  [Force\textbackslash{}\_ST, Force\textbackslash{}\_Lin, Force\textbackslash{}\_GR, Force\textbackslash{}\_LG, EEmo{\ldots} \&       0.0 \&  183200.0 \&        5496.0 \textbackslash{}\textbackslash{}
AffectDOF propstuck, t=1    \&  [DistEE, AffectDOF, CtlDOF, Planpath, Trajecto{\ldots} \&  [Force\textbackslash{}\_ST, Force\textbackslash{}\_Lin, Force\textbackslash{}\_GR, Force\textbackslash{}\_LG, EE\textbackslash{}\_1{\ldots} \&       0.0 \&  186200.0 \&        3724.0 \textbackslash{}\textbackslash{}
HoldPayload break, t=1      \&  [DistEE, CtlDOF, Planpath, Trajectory, EngageL{\ldots} \&  [Force\textbackslash{}\_ST, Force\textbackslash{}\_Lin, Force\textbackslash{}\_GR, Force\textbackslash{}\_LG, EEmo{\ldots} \&       0.0 \&  183000.0 \&        3660.0 \textbackslash{}\textbackslash{}
ViewEnv poorview, t=1       \&                                          [ViewEnv] \&                                                 [] \&  0.000002 \&   10000.0 \&        2000.0 \textbackslash{}\textbackslash{}
EngageLand deform, t=1      \&                                       [EngageLand] \&                                                 [] \&  0.000008 \&    1000.0 \&         800.0 \textbackslash{}\textbackslash{}
HoldPayload deform, t=1     \&                                      [HoldPayload] \&                              [Force\textbackslash{}\_ST, Force\textbackslash{}\_Lin] \&  0.000001 \&   10000.0 \&         800.0 \textbackslash{}\textbackslash{}
DistEE degr, t=1            \&                                           [DistEE] \&     [Force\textbackslash{}\_GR, Force\textbackslash{}\_LG, EEmot, EEctl, Ctl1, DOFs] \&  0.000005 \&    1000.0 \&         500.0 \textbackslash{}\textbackslash{}
EngageLand break, t=1       \&                                       [EngageLand] \&                                                 [] \&  0.000002 \&    1000.0 \&         200.0 \textbackslash{}\textbackslash{}
AffectDOF ctldn, t=1        \&                                        [AffectDOF] \&                         [Force\textbackslash{}\_GR, Force\textbackslash{}\_LG, DOFs] \&  0.000002 \&     500.0 \&         100.0 \textbackslash{}\textbackslash{}
AffectDOF mechfriction, t=1 \&                                        [AffectDOF] \&                                      [EE\textbackslash{}\_1, EEmot] \&  0.000001 \&     500.0 \&          25.0 \textbackslash{}\textbackslash{}
AffectDOF propwarp, t=1     \&                                        [AffectDOF] \&                         [Force\textbackslash{}\_GR, Force\textbackslash{}\_LG, DOFs] \&       0.0 \&     200.0 \&           2.0 \textbackslash{}\textbackslash{}
Trajectory crash, t=1       \&                                       [Trajectory] \&                                                 [] \&       0.0 \&  100000.0 \&           0.0 \textbackslash{}\textbackslash{}
Trajectory lost, t=1        \&                                       [Trajectory] \&                                                 [] \&       0.0 \&   50000.0 \&           0.0 \textbackslash{}\textbackslash{}
\textbackslash{}bottomrule
\textbackslash{}end\{tabular\}

\end{sphinxVerbatim}



\end{sphinxuseclass}
\end{sphinxuseclass}
}

\end{sphinxuseclass}
\end{sphinxuseclass}
\sphinxAtStartPar
We can in turn visualize these faults on the graph representation of the system. Here we will focus on the break of one of the rotors in the AffectDOF function, the effects of which are shown below:

\begin{sphinxuseclass}{nbinput}
{
\sphinxsetup{VerbatimColor={named}{nbsphinx-code-bg}}
\sphinxsetup{VerbatimBorderColor={named}{nbsphinx-code-border}}
\begin{sphinxVerbatim}[commandchars=\\\{\}]
\llap{\color{nbsphinxin}[25]:\,\hspace{\fboxrule}\hspace{\fboxsep}}\PYG{n}{static\PYGZus{}mdl} \PYG{o}{=} \PYG{n}{Drone\PYGZus{}Static}\PYG{p}{(}\PYG{n}{params}\PYG{o}{=}\PYG{p}{\PYGZob{}}\PYG{l+s+s1}{\PYGZsq{}}\PYG{l+s+s1}{graph\PYGZus{}pos}\PYG{l+s+s1}{\PYGZsq{}}\PYG{p}{:}\PYG{n}{graph\PYGZus{}pos}\PYG{p}{,} \PYG{l+s+s1}{\PYGZsq{}}\PYG{l+s+s1}{bipartite\PYGZus{}pos}\PYG{l+s+s1}{\PYGZsq{}}\PYG{p}{:}\PYG{n}{bipartite\PYGZus{}pos}\PYG{p}{\PYGZcb{}}\PYG{p}{)}
\PYG{n}{endresults}\PYG{p}{,} \PYG{n}{resgraph}\PYG{p}{,} \PYG{n}{mdlhist} \PYG{o}{=} \PYG{n}{fs}\PYG{o}{.}\PYG{n}{propagate}\PYG{o}{.}\PYG{n}{one\PYGZus{}fault}\PYG{p}{(}\PYG{n}{static\PYGZus{}mdl}\PYG{p}{,}\PYG{l+s+s1}{\PYGZsq{}}\PYG{l+s+s1}{AffectDOF}\PYG{l+s+s1}{\PYGZsq{}}\PYG{p}{,} \PYG{l+s+s1}{\PYGZsq{}}\PYG{l+s+s1}{mechbreak}\PYG{l+s+s1}{\PYGZsq{}}\PYG{p}{,} \PYG{n}{gtype}\PYG{o}{=}\PYG{l+s+s1}{\PYGZsq{}}\PYG{l+s+s1}{bipartite}\PYG{l+s+s1}{\PYGZsq{}}\PYG{p}{)}
\PYG{n}{fig}\PYG{p}{,} \PYG{n}{ax} \PYG{o}{=} \PYG{n}{rd}\PYG{o}{.}\PYG{n}{graph}\PYG{o}{.}\PYG{n}{show}\PYG{p}{(}\PYG{n}{resgraph}\PYG{p}{,} \PYG{n}{pos} \PYG{o}{=} \PYG{n}{bipartite\PYGZus{}pos}\PYG{p}{,} \PYG{n}{faultscen}\PYG{o}{=}\PYG{l+s+s1}{\PYGZsq{}}\PYG{l+s+s1}{AffectDOF: Mechbreak}\PYG{l+s+s1}{\PYGZsq{}}\PYG{p}{,} \PYG{n}{time}\PYG{o}{=}\PYG{l+s+s1}{\PYGZsq{}}\PYG{l+s+s1}{NA}\PYG{l+s+s1}{\PYGZsq{}}\PYG{p}{,} \PYG{n}{scale}\PYG{o}{=}\PYG{l+m+mi}{1}\PYG{p}{,} \PYG{n}{gtype}\PYG{o}{=}\PYG{l+s+s1}{\PYGZsq{}}\PYG{l+s+s1}{bipartite}\PYG{l+s+s1}{\PYGZsq{}}\PYG{p}{)}
\PYG{n}{fig}\PYG{o}{.}\PYG{n}{savefig}\PYG{p}{(}\PYG{l+s+s1}{\PYGZsq{}}\PYG{l+s+s1}{static\PYGZus{}propagation.pdf}\PYG{l+s+s1}{\PYGZsq{}}\PYG{p}{,} \PYG{n+nb}{format}\PYG{o}{=}\PYG{l+s+s2}{\PYGZdq{}}\PYG{l+s+s2}{pdf}\PYG{l+s+s2}{\PYGZdq{}}\PYG{p}{,} \PYG{n}{bbox\PYGZus{}inches} \PYG{o}{=} \PYG{l+s+s1}{\PYGZsq{}}\PYG{l+s+s1}{tight}\PYG{l+s+s1}{\PYGZsq{}}\PYG{p}{,} \PYG{n}{pad\PYGZus{}inches} \PYG{o}{=} \PYG{l+m+mi}{0}\PYG{p}{)}
\end{sphinxVerbatim}
}

\end{sphinxuseclass}
\begin{sphinxuseclass}{nboutput}
\begin{sphinxuseclass}{nblast}
\hrule height -\fboxrule\relax
\vspace{\nbsphinxcodecellspacing}

\makeatletter\setbox\nbsphinxpromptbox\box\voidb@x\makeatother

\begin{nbsphinxfancyoutput}

\begin{sphinxuseclass}{output_area}
\begin{sphinxuseclass}{}
\noindent\sphinxincludegraphics[width=349\sphinxpxdimen,height=247\sphinxpxdimen]{{example_multirotor_Demonstration_41_0}.png}

\end{sphinxuseclass}
\end{sphinxuseclass}
\end{nbsphinxfancyoutput}

\end{sphinxuseclass}
\end{sphinxuseclass}

\subsubsection{Dynamic Model}
\label{\detokenize{example_multirotor/Demonstration:Dynamic-Model}}
\sphinxAtStartPar
In the dynamic model, we add time ranges and dynamic behaviors to generate behavior\sphinxhyphen{}over\sphinxhyphen{}time graphs and dynamic/phase\sphinxhyphen{}based FMEAs.

\sphinxAtStartPar
This model is located in \sphinxcode{\sphinxupquote{drone\_mdl\_dynamic.py}}.

\begin{sphinxuseclass}{nbinput}
\begin{sphinxuseclass}{nblast}
{
\sphinxsetup{VerbatimColor={named}{nbsphinx-code-bg}}
\sphinxsetup{VerbatimBorderColor={named}{nbsphinx-code-border}}
\begin{sphinxVerbatim}[commandchars=\\\{\}]
\llap{\color{nbsphinxin}[26]:\,\hspace{\fboxrule}\hspace{\fboxsep}}\PYG{k+kn}{from} \PYG{n+nn}{drone\PYGZus{}mdl\PYGZus{}dynamic} \PYG{k+kn}{import} \PYG{n}{Drone} \PYG{k}{as} \PYG{n}{Drone\PYGZus{}Dynamic}
\end{sphinxVerbatim}
}

\end{sphinxuseclass}
\end{sphinxuseclass}
\sphinxAtStartPar
Here we can see how the system operates over time in the nominal case:

\begin{sphinxuseclass}{nbinput}
{
\sphinxsetup{VerbatimColor={named}{nbsphinx-code-bg}}
\sphinxsetup{VerbatimBorderColor={named}{nbsphinx-code-border}}
\begin{sphinxVerbatim}[commandchars=\\\{\}]
\llap{\color{nbsphinxin}[27]:\,\hspace{\fboxrule}\hspace{\fboxsep}}\PYG{c+c1}{\PYGZsh{} Note: because of the complicated functions, the model must be re\PYGZhy{}instantiated for each function in order to work in this case}
\PYG{n}{dynamic\PYGZus{}mdl} \PYG{o}{=} \PYG{n}{Drone\PYGZus{}Dynamic}\PYG{p}{(}\PYG{n}{params}\PYG{o}{=}\PYG{p}{\PYGZob{}}\PYG{l+s+s1}{\PYGZsq{}}\PYG{l+s+s1}{graph\PYGZus{}pos}\PYG{l+s+s1}{\PYGZsq{}}\PYG{p}{:}\PYG{n}{graph\PYGZus{}pos}\PYG{p}{,} \PYG{l+s+s1}{\PYGZsq{}}\PYG{l+s+s1}{bipartite\PYGZus{}pos}\PYG{l+s+s1}{\PYGZsq{}}\PYG{p}{:}\PYG{n}{bipartite\PYGZus{}pos}\PYG{p}{\PYGZcb{}}\PYG{p}{)}
\PYG{n}{endresults}\PYG{p}{,} \PYG{n}{resgraph}\PYG{p}{,} \PYG{n}{mdlhist} \PYG{o}{=} \PYG{n}{fs}\PYG{o}{.}\PYG{n}{propagate}\PYG{o}{.}\PYG{n}{nominal}\PYG{p}{(}\PYG{n}{dynamic\PYGZus{}mdl}\PYG{p}{)}
\PYG{n}{rd}\PYG{o}{.}\PYG{n}{plot}\PYG{o}{.}\PYG{n}{mdlhistvals}\PYG{p}{(}\PYG{n}{mdlhist}\PYG{p}{)}
\end{sphinxVerbatim}
}

\end{sphinxuseclass}
\begin{sphinxuseclass}{nboutput}
\hrule height -\fboxrule\relax
\vspace{\nbsphinxcodecellspacing}

\savebox\nbsphinxpromptbox[0pt][r]{\color{nbsphinxout}\Verb|\strut{[27]:}\,|}

\begin{nbsphinxfancyoutput}

\begin{sphinxuseclass}{output_area}
\begin{sphinxuseclass}{}
\noindent\sphinxincludegraphics[width=426\sphinxpxdimen,height=2562\sphinxpxdimen]{{example_multirotor_Demonstration_45_0}.png}

\end{sphinxuseclass}
\end{sphinxuseclass}
\end{nbsphinxfancyoutput}

\end{sphinxuseclass}
\begin{sphinxuseclass}{nboutput}
\begin{sphinxuseclass}{nblast}
\hrule height -\fboxrule\relax
\vspace{\nbsphinxcodecellspacing}

\makeatletter\setbox\nbsphinxpromptbox\box\voidb@x\makeatother

\begin{nbsphinxfancyoutput}

\begin{sphinxuseclass}{output_area}
\begin{sphinxuseclass}{}
\noindent\sphinxincludegraphics[width=426\sphinxpxdimen,height=2562\sphinxpxdimen]{{example_multirotor_Demonstration_45_1}.png}

\end{sphinxuseclass}
\end{sphinxuseclass}
\end{nbsphinxfancyoutput}

\end{sphinxuseclass}
\end{sphinxuseclass}
\sphinxAtStartPar
As shown below, in the case of the break in the AffectDOF function, the system crashes:

\begin{sphinxuseclass}{nbinput}
\begin{sphinxuseclass}{nblast}
{
\sphinxsetup{VerbatimColor={named}{nbsphinx-code-bg}}
\sphinxsetup{VerbatimBorderColor={named}{nbsphinx-code-border}}
\begin{sphinxVerbatim}[commandchars=\\\{\}]
\llap{\color{nbsphinxin}[28]:\,\hspace{\fboxrule}\hspace{\fboxsep}}\PYG{n}{dynamic\PYGZus{}mdl} \PYG{o}{=} \PYG{n}{Drone\PYGZus{}Dynamic}\PYG{p}{(}\PYG{n}{params}\PYG{o}{=}\PYG{p}{\PYGZob{}}\PYG{l+s+s1}{\PYGZsq{}}\PYG{l+s+s1}{graph\PYGZus{}pos}\PYG{l+s+s1}{\PYGZsq{}}\PYG{p}{:}\PYG{n}{graph\PYGZus{}pos}\PYG{p}{,} \PYG{l+s+s1}{\PYGZsq{}}\PYG{l+s+s1}{bipartite\PYGZus{}pos}\PYG{l+s+s1}{\PYGZsq{}}\PYG{p}{:}\PYG{n}{bipartite\PYGZus{}pos}\PYG{p}{\PYGZcb{}}\PYG{p}{)}
\PYG{n}{endresults}\PYG{p}{,} \PYG{n}{resgraph}\PYG{p}{,} \PYG{n}{mdlhist} \PYG{o}{=} \PYG{n}{fs}\PYG{o}{.}\PYG{n}{propagate}\PYG{o}{.}\PYG{n}{one\PYGZus{}fault}\PYG{p}{(}\PYG{n}{dynamic\PYGZus{}mdl}\PYG{p}{,}\PYG{l+s+s1}{\PYGZsq{}}\PYG{l+s+s1}{AffectDOF}\PYG{l+s+s1}{\PYGZsq{}}\PYG{p}{,} \PYG{l+s+s1}{\PYGZsq{}}\PYG{l+s+s1}{mechbreak}\PYG{l+s+s1}{\PYGZsq{}}\PYG{p}{,} \PYG{n}{time}\PYG{o}{=}\PYG{l+m+mi}{50}\PYG{p}{)}
\end{sphinxVerbatim}
}

\end{sphinxuseclass}
\end{sphinxuseclass}
\begin{sphinxuseclass}{nbinput}
{
\sphinxsetup{VerbatimColor={named}{nbsphinx-code-bg}}
\sphinxsetup{VerbatimBorderColor={named}{nbsphinx-code-border}}
\begin{sphinxVerbatim}[commandchars=\\\{\}]
\llap{\color{nbsphinxin}[29]:\,\hspace{\fboxrule}\hspace{\fboxsep}}\PYG{n}{fig}\PYG{o}{=} \PYG{n}{rd}\PYG{o}{.}\PYG{n}{plot}\PYG{o}{.}\PYG{n}{mdlhistvals}\PYG{p}{(}\PYG{n}{mdlhist}\PYG{p}{,}\PYG{l+s+s1}{\PYGZsq{}}\PYG{l+s+s1}{AffectDOF mechbreak}\PYG{l+s+s1}{\PYGZsq{}}\PYG{p}{,} \PYG{n}{time}\PYG{o}{=}\PYG{l+m+mi}{50}\PYG{p}{,} \PYG{n}{fxnflowvals}\PYG{o}{=}\PYG{p}{\PYGZob{}}\PYG{l+s+s1}{\PYGZsq{}}\PYG{l+s+s1}{Env1}\PYG{l+s+s1}{\PYGZsq{}}\PYG{p}{:}\PYG{p}{[}\PYG{l+s+s1}{\PYGZsq{}}\PYG{l+s+s1}{x}\PYG{l+s+s1}{\PYGZsq{}}\PYG{p}{,}\PYG{l+s+s1}{\PYGZsq{}}\PYG{l+s+s1}{y}\PYG{l+s+s1}{\PYGZsq{}}\PYG{p}{,}\PYG{l+s+s1}{\PYGZsq{}}\PYG{l+s+s1}{elev}\PYG{l+s+s1}{\PYGZsq{}}\PYG{p}{]}\PYG{p}{,} \PYG{l+s+s1}{\PYGZsq{}}\PYG{l+s+s1}{StoreEE}\PYG{l+s+s1}{\PYGZsq{}}\PYG{p}{:}\PYG{p}{[}\PYG{l+s+s1}{\PYGZsq{}}\PYG{l+s+s1}{soc}\PYG{l+s+s1}{\PYGZsq{}}\PYG{p}{]}\PYG{p}{\PYGZcb{}}\PYG{p}{,} \PYG{n}{units}\PYG{o}{=}\PYG{p}{[}\PYG{l+s+s1}{\PYGZsq{}}\PYG{l+s+s1}{m}\PYG{l+s+s1}{\PYGZsq{}}\PYG{p}{,}\PYG{l+s+s1}{\PYGZsq{}}\PYG{l+s+s1}{m}\PYG{l+s+s1}{\PYGZsq{}}\PYG{p}{,}\PYG{l+s+s1}{\PYGZsq{}}\PYG{l+s+s1}{m}\PYG{l+s+s1}{\PYGZsq{}}\PYG{p}{,}\PYG{l+s+s1}{\PYGZsq{}}\PYG{l+s+s1}{\PYGZpc{}}\PYG{l+s+s1}{\PYGZsq{}}\PYG{p}{]}\PYG{p}{,}\PYG{n}{legend}\PYG{o}{=}\PYG{k+kc}{False}\PYG{p}{,} \PYG{n}{returnfig}\PYG{o}{=}\PYG{k+kc}{True}\PYG{p}{,} \PYG{n}{timelabel}\PYG{o}{=}\PYG{l+s+s1}{\PYGZsq{}}\PYG{l+s+s1}{time (s)}\PYG{l+s+s1}{\PYGZsq{}}\PYG{p}{)}

\PYG{n}{ax} \PYG{o}{=} \PYG{n}{fig}\PYG{o}{.}\PYG{n}{axes}\PYG{p}{[}\PYG{l+m+mi}{1}\PYG{p}{]}
\PYG{n}{ax}\PYG{o}{.}\PYG{n}{legend}\PYG{p}{(}\PYG{p}{[}\PYG{l+s+s1}{\PYGZsq{}}\PYG{l+s+s1}{faulty}\PYG{l+s+s1}{\PYGZsq{}}\PYG{p}{,} \PYG{l+s+s1}{\PYGZsq{}}\PYG{l+s+s1}{nominal}\PYG{l+s+s1}{\PYGZsq{}}\PYG{p}{]}\PYG{p}{,} \PYG{n}{loc}\PYG{o}{=}\PYG{l+s+s1}{\PYGZsq{}}\PYG{l+s+s1}{right}\PYG{l+s+s1}{\PYGZsq{}}\PYG{p}{)}

\PYG{n}{fig}\PYG{o}{.}\PYG{n}{savefig}\PYG{p}{(}\PYG{l+s+s2}{\PYGZdq{}}\PYG{l+s+s2}{fault\PYGZus{}behavior.pdf}\PYG{l+s+s2}{\PYGZdq{}}\PYG{p}{,} \PYG{n+nb}{format}\PYG{o}{=}\PYG{l+s+s2}{\PYGZdq{}}\PYG{l+s+s2}{pdf}\PYG{l+s+s2}{\PYGZdq{}}\PYG{p}{,} \PYG{n}{bbox\PYGZus{}inches} \PYG{o}{=} \PYG{l+s+s1}{\PYGZsq{}}\PYG{l+s+s1}{tight}\PYG{l+s+s1}{\PYGZsq{}}\PYG{p}{,} \PYG{n}{pad\PYGZus{}inches} \PYG{o}{=} \PYG{l+m+mi}{0}\PYG{p}{)}
\end{sphinxVerbatim}
}

\end{sphinxuseclass}
\begin{sphinxuseclass}{nboutput}
\begin{sphinxuseclass}{nblast}
\hrule height -\fboxrule\relax
\vspace{\nbsphinxcodecellspacing}

\makeatletter\setbox\nbsphinxpromptbox\box\voidb@x\makeatother

\begin{nbsphinxfancyoutput}

\begin{sphinxuseclass}{output_area}
\begin{sphinxuseclass}{}
\noindent\sphinxincludegraphics[width=437\sphinxpxdimen,height=286\sphinxpxdimen]{{example_multirotor_Demonstration_48_0}.png}

\end{sphinxuseclass}
\end{sphinxuseclass}
\end{nbsphinxfancyoutput}

\end{sphinxuseclass}
\end{sphinxuseclass}
\sphinxAtStartPar
Finally, we can see how the cost function of this scenario changes over time. As shown, when the fault is injected early, it has a lower cost because it crashes at the landing pad and not in a dangerous area. When it is injected at the end, the cost is minimal because the drone has already landed.

\begin{sphinxuseclass}{nbinput}
{
\sphinxsetup{VerbatimColor={named}{nbsphinx-code-bg}}
\sphinxsetup{VerbatimBorderColor={named}{nbsphinx-code-border}}
\begin{sphinxVerbatim}[commandchars=\\\{\}]
\llap{\color{nbsphinxin}[30]:\,\hspace{\fboxrule}\hspace{\fboxsep}}\PYG{n}{mdl\PYGZus{}quad\PYGZus{}comp} \PYG{o}{=} \PYG{n}{Drone\PYGZus{}Dynamic}\PYG{p}{(}\PYG{n}{params}\PYG{o}{=}\PYG{p}{\PYGZob{}}\PYG{l+s+s1}{\PYGZsq{}}\PYG{l+s+s1}{graph\PYGZus{}pos}\PYG{l+s+s1}{\PYGZsq{}}\PYG{p}{:}\PYG{n}{graph\PYGZus{}pos}\PYG{p}{,} \PYG{l+s+s1}{\PYGZsq{}}\PYG{l+s+s1}{bipartite\PYGZus{}pos}\PYG{l+s+s1}{\PYGZsq{}}\PYG{p}{:}\PYG{n}{bipartite\PYGZus{}pos}\PYG{p}{\PYGZcb{}}\PYG{p}{)}
\PYG{n}{quad\PYGZus{}comp\PYGZus{}app} \PYG{o}{=} \PYG{n}{SampleApproach}\PYG{p}{(}\PYG{n}{mdl\PYGZus{}quad\PYGZus{}comp}\PYG{p}{,} \PYG{n}{faults}\PYG{o}{=}\PYG{p}{[}\PYG{p}{(}\PYG{l+s+s1}{\PYGZsq{}}\PYG{l+s+s1}{AffectDOF}\PYG{l+s+s1}{\PYGZsq{}}\PYG{p}{,} \PYG{l+s+s1}{\PYGZsq{}}\PYG{l+s+s1}{mechbreak}\PYG{l+s+s1}{\PYGZsq{}}\PYG{p}{)}\PYG{p}{]}\PYG{p}{,}\PYG{n}{defaultsamp}\PYG{o}{=}\PYG{p}{\PYGZob{}}\PYG{l+s+s1}{\PYGZsq{}}\PYG{l+s+s1}{samp}\PYG{l+s+s1}{\PYGZsq{}}\PYG{p}{:}\PYG{l+s+s1}{\PYGZsq{}}\PYG{l+s+s1}{evenspacing}\PYG{l+s+s1}{\PYGZsq{}}\PYG{p}{,}\PYG{l+s+s1}{\PYGZsq{}}\PYG{l+s+s1}{numpts}\PYG{l+s+s1}{\PYGZsq{}}\PYG{p}{:}\PYG{l+m+mi}{5}\PYG{p}{\PYGZcb{}}\PYG{p}{)}
\PYG{n}{quad\PYGZus{}comp\PYGZus{}endclasses}\PYG{p}{,} \PYG{n}{quad\PYGZus{}comp\PYGZus{}mdlhists} \PYG{o}{=} \PYG{n}{fs}\PYG{o}{.}\PYG{n}{propagate}\PYG{o}{.}\PYG{n}{approach}\PYG{p}{(}\PYG{n}{mdl\PYGZus{}quad\PYGZus{}comp}\PYG{p}{,} \PYG{n}{quad\PYGZus{}comp\PYGZus{}app}\PYG{p}{,} \PYG{n}{staged}\PYG{o}{=}\PYG{k+kc}{True}\PYG{p}{)}
\end{sphinxVerbatim}
}

\end{sphinxuseclass}
\begin{sphinxuseclass}{nboutput}
\begin{sphinxuseclass}{nblast}
{

\kern-\sphinxverbatimsmallskipamount\kern-\baselineskip
\kern+\FrameHeightAdjust\kern-\fboxrule
\vspace{\nbsphinxcodecellspacing}

\sphinxsetup{VerbatimColor={named}{nbsphinx-stderr}}
\sphinxsetup{VerbatimBorderColor={named}{nbsphinx-code-border}}
\begin{sphinxuseclass}{output_area}
\begin{sphinxuseclass}{stderr}


\begin{sphinxVerbatim}[commandchars=\\\{\}]
SCENARIOS COMPLETE: 100\%|██████████████████████████████████████████████████████████████| 15/15 [00:09<00:00,  1.65it/s]
\end{sphinxVerbatim}



\end{sphinxuseclass}
\end{sphinxuseclass}
}

\end{sphinxuseclass}
\end{sphinxuseclass}
\begin{sphinxuseclass}{nbinput}
{
\sphinxsetup{VerbatimColor={named}{nbsphinx-code-bg}}
\sphinxsetup{VerbatimBorderColor={named}{nbsphinx-code-border}}
\begin{sphinxVerbatim}[commandchars=\\\{\}]
\llap{\color{nbsphinxin}[31]:\,\hspace{\fboxrule}\hspace{\fboxsep}}\PYG{n}{quad\PYGZus{}comp\PYGZus{}mdlhists}\PYG{p}{[}\PYG{l+s+s1}{\PYGZsq{}}\PYG{l+s+s1}{AffectDOF mechbreak, t=20.0}\PYG{l+s+s1}{\PYGZsq{}}\PYG{p}{]}\PYG{p}{[}\PYG{l+s+s1}{\PYGZsq{}}\PYG{l+s+s1}{flows}\PYG{l+s+s1}{\PYGZsq{}}\PYG{p}{]}\PYG{p}{[}\PYG{l+s+s1}{\PYGZsq{}}\PYG{l+s+s1}{Env1}\PYG{l+s+s1}{\PYGZsq{}}\PYG{p}{]}\PYG{p}{[}\PYG{l+s+s1}{\PYGZsq{}}\PYG{l+s+s1}{x}\PYG{l+s+s1}{\PYGZsq{}}\PYG{p}{]}\PYG{p}{[}\PYG{o}{\PYGZhy{}}\PYG{l+m+mi}{1}\PYG{p}{]}
\end{sphinxVerbatim}
}

\end{sphinxuseclass}
\begin{sphinxuseclass}{nboutput}
\begin{sphinxuseclass}{nblast}
{

\kern-\sphinxverbatimsmallskipamount\kern-\baselineskip
\kern+\FrameHeightAdjust\kern-\fboxrule
\vspace{\nbsphinxcodecellspacing}

\sphinxsetup{VerbatimColor={named}{white}}
\sphinxsetup{VerbatimBorderColor={named}{nbsphinx-code-border}}
\begin{sphinxuseclass}{output_area}
\begin{sphinxuseclass}{}


\begin{sphinxVerbatim}[commandchars=\\\{\}]
\llap{\color{nbsphinxout}[31]:\,\hspace{\fboxrule}\hspace{\fboxsep}}9.977765842580308
\end{sphinxVerbatim}



\end{sphinxuseclass}
\end{sphinxuseclass}
}

\end{sphinxuseclass}
\end{sphinxuseclass}
\begin{sphinxuseclass}{nbinput}
{
\sphinxsetup{VerbatimColor={named}{nbsphinx-code-bg}}
\sphinxsetup{VerbatimBorderColor={named}{nbsphinx-code-border}}
\begin{sphinxVerbatim}[commandchars=\\\{\}]
\llap{\color{nbsphinxin}[32]:\,\hspace{\fboxrule}\hspace{\fboxsep}}\PYG{n}{rd}\PYG{o}{.}\PYG{n}{plot}\PYG{o}{.}\PYG{n}{samplecost}\PYG{p}{(}\PYG{n}{quad\PYGZus{}comp\PYGZus{}app}\PYG{p}{,} \PYG{n}{quad\PYGZus{}comp\PYGZus{}endclasses}\PYG{p}{,} \PYG{p}{(}\PYG{l+s+s1}{\PYGZsq{}}\PYG{l+s+s1}{AffectDOF}\PYG{l+s+s1}{\PYGZsq{}}\PYG{p}{,} \PYG{l+s+s1}{\PYGZsq{}}\PYG{l+s+s1}{mechbreak}\PYG{l+s+s1}{\PYGZsq{}}\PYG{p}{)}\PYG{p}{)}
\PYG{n}{fig} \PYG{o}{=} \PYG{n}{plt}\PYG{o}{.}\PYG{n}{gcf}\PYG{p}{(}\PYG{p}{)}
\PYG{n}{fig}\PYG{o}{.}\PYG{n}{savefig}\PYG{p}{(}\PYG{l+s+s2}{\PYGZdq{}}\PYG{l+s+s2}{cost\PYGZus{}over\PYGZus{}time.pdf}\PYG{l+s+s2}{\PYGZdq{}}\PYG{p}{,} \PYG{n+nb}{format}\PYG{o}{=}\PYG{l+s+s2}{\PYGZdq{}}\PYG{l+s+s2}{pdf}\PYG{l+s+s2}{\PYGZdq{}}\PYG{p}{,} \PYG{n}{bbox\PYGZus{}inches} \PYG{o}{=} \PYG{l+s+s1}{\PYGZsq{}}\PYG{l+s+s1}{tight}\PYG{l+s+s1}{\PYGZsq{}}\PYG{p}{,} \PYG{n}{pad\PYGZus{}inches} \PYG{o}{=} \PYG{l+m+mi}{0}\PYG{p}{)}
\end{sphinxVerbatim}
}

\end{sphinxuseclass}
\begin{sphinxuseclass}{nboutput}
\begin{sphinxuseclass}{nblast}
\hrule height -\fboxrule\relax
\vspace{\nbsphinxcodecellspacing}

\makeatletter\setbox\nbsphinxpromptbox\box\voidb@x\makeatother

\begin{nbsphinxfancyoutput}

\begin{sphinxuseclass}{output_area}
\begin{sphinxuseclass}{}
\noindent\sphinxincludegraphics[width=424\sphinxpxdimen,height=280\sphinxpxdimen]{{example_multirotor_Demonstration_52_0}.png}

\end{sphinxuseclass}
\end{sphinxuseclass}
\end{nbsphinxfancyoutput}

\end{sphinxuseclass}
\end{sphinxuseclass}
\begin{sphinxuseclass}{nbinput}
{
\sphinxsetup{VerbatimColor={named}{nbsphinx-code-bg}}
\sphinxsetup{VerbatimBorderColor={named}{nbsphinx-code-border}}
\begin{sphinxVerbatim}[commandchars=\\\{\}]
\llap{\color{nbsphinxin}[33]:\,\hspace{\fboxrule}\hspace{\fboxsep}}\PYG{n}{quad\PYGZus{}comp\PYGZus{}endclasses}
\end{sphinxVerbatim}
}

\end{sphinxuseclass}
\begin{sphinxuseclass}{nboutput}
\begin{sphinxuseclass}{nblast}
{

\kern-\sphinxverbatimsmallskipamount\kern-\baselineskip
\kern+\FrameHeightAdjust\kern-\fboxrule
\vspace{\nbsphinxcodecellspacing}

\sphinxsetup{VerbatimColor={named}{white}}
\sphinxsetup{VerbatimBorderColor={named}{nbsphinx-code-border}}
\begin{sphinxuseclass}{output_area}
\begin{sphinxuseclass}{}


\begin{sphinxVerbatim}[commandchars=\\\{\}]
\llap{\color{nbsphinxout}[33]:\,\hspace{\fboxrule}\hspace{\fboxsep}}\{'AffectDOF mechbreak, t=0.0': \{'rate': 9.259259259259263e-10,
  'cost': 500,
  'expected cost': 0.046296296296296315\},
 'AffectDOF mechbreak, t=1.0': \{'rate': 9.259259259259263e-10,
  'cost': 500,
  'expected cost': 0.046296296296296315\},
 'AffectDOF mechbreak, t=2.0': \{'rate': 9.259259259259263e-10,
  'cost': 133500,
  'expected cost': 12.361111111111114\},
 'AffectDOF mechbreak, t=3.0': \{'rate': 9.259259259259263e-10,
  'cost': 133500,
  'expected cost': 12.361111111111114\},
 'AffectDOF mechbreak, t=4.0': \{'rate': 9.259259259259263e-10,
  'cost': 133500,
  'expected cost': 12.361111111111114\},
 'AffectDOF mechbreak, t=20.0': \{'rate': 1.6666666666666674e-08,
  'cost': 183500,
  'expected cost': 305.8333333333335\},
 'AffectDOF mechbreak, t=35.0': \{'rate': 1.6666666666666674e-08,
  'cost': 183500,
  'expected cost': 305.8333333333335\},
 'AffectDOF mechbreak, t=49.0': \{'rate': 1.6666666666666674e-08,
  'cost': 183500,
  'expected cost': 305.8333333333335\},
 'AffectDOF mechbreak, t=64.0': \{'rate': 1.6666666666666674e-08,
  'cost': 183500,
  'expected cost': 305.8333333333335\},
 'AffectDOF mechbreak, t=79.0': \{'rate': 1.6666666666666674e-08,
  'cost': 133500,
  'expected cost': 222.50000000000009\},
 'AffectDOF mechbreak, t=95.0': \{'rate': 9.259259259259263e-10,
  'cost': 500,
  'expected cost': 0.046296296296296315\},
 'AffectDOF mechbreak, t=96.0': \{'rate': 9.259259259259263e-10,
  'cost': 500,
  'expected cost': 0.046296296296296315\},
 'AffectDOF mechbreak, t=97.0': \{'rate': 9.259259259259263e-10,
  'cost': 500,
  'expected cost': 0.046296296296296315\},
 'AffectDOF mechbreak, t=98.0': \{'rate': 9.259259259259263e-10,
  'cost': 500,
  'expected cost': 0.046296296296296315\},
 'AffectDOF mechbreak, t=99.0': \{'rate': 9.259259259259263e-10,
  'cost': 500,
  'expected cost': 0.046296296296296315\},
 'nominal': \{'rate': 1.0, 'cost': 0, 'expected cost': 0.0\}\}
\end{sphinxVerbatim}



\end{sphinxuseclass}
\end{sphinxuseclass}
}

\end{sphinxuseclass}
\end{sphinxuseclass}

\subsubsection{Hierarchical Model}
\label{\detokenize{example_multirotor/Demonstration:Hierarchical-Model}}
\sphinxAtStartPar
In the hierarchical model, we can use the simulation to compare system architectures. First by seeing how faults effect the behaviors in each architechture, then by seing how it affects the overall system resilience.

\sphinxAtStartPar
This model is located in \sphinxcode{\sphinxupquote{drone\_mdl\_hierarchical.py}}.

\begin{sphinxuseclass}{nbinput}
\begin{sphinxuseclass}{nblast}
{
\sphinxsetup{VerbatimColor={named}{nbsphinx-code-bg}}
\sphinxsetup{VerbatimBorderColor={named}{nbsphinx-code-border}}
\begin{sphinxVerbatim}[commandchars=\\\{\}]
\llap{\color{nbsphinxin}[34]:\,\hspace{\fboxrule}\hspace{\fboxsep}}\PYG{k+kn}{from} \PYG{n+nn}{drone\PYGZus{}mdl\PYGZus{}hierarchical} \PYG{k+kn}{import} \PYG{n}{Drone} \PYG{k}{as} \PYG{n}{Drone\PYGZus{}Hierarchical}
\end{sphinxVerbatim}
}

\end{sphinxuseclass}
\end{sphinxuseclass}
\sphinxAtStartPar
First, we can model how the quadrotor architecture behaves under faults–in this case, identically to the non\sphinxhyphen{}hierarchical model:

\begin{sphinxuseclass}{nbinput}
\begin{sphinxuseclass}{nblast}
{
\sphinxsetup{VerbatimColor={named}{nbsphinx-code-bg}}
\sphinxsetup{VerbatimBorderColor={named}{nbsphinx-code-border}}
\begin{sphinxVerbatim}[commandchars=\\\{\}]
\llap{\color{nbsphinxin}[35]:\,\hspace{\fboxrule}\hspace{\fboxsep}}\PYG{n}{hierarchical\PYGZus{}model} \PYG{o}{=} \PYG{n}{Drone\PYGZus{}Hierarchical}\PYG{p}{(}\PYG{n}{params}\PYG{o}{=}\PYG{p}{\PYGZob{}}\PYG{l+s+s1}{\PYGZsq{}}\PYG{l+s+s1}{graph\PYGZus{}pos}\PYG{l+s+s1}{\PYGZsq{}}\PYG{p}{:}\PYG{n}{graph\PYGZus{}pos}\PYG{p}{,} \PYG{l+s+s1}{\PYGZsq{}}\PYG{l+s+s1}{bipartite\PYGZus{}pos}\PYG{l+s+s1}{\PYGZsq{}}\PYG{p}{:}\PYG{n}{bipartite\PYGZus{}pos}\PYG{p}{,}\PYG{l+s+s1}{\PYGZsq{}}\PYG{l+s+s1}{arch}\PYG{l+s+s1}{\PYGZsq{}}\PYG{p}{:}\PYG{l+s+s1}{\PYGZsq{}}\PYG{l+s+s1}{quad}\PYG{l+s+s1}{\PYGZsq{}}\PYG{p}{\PYGZcb{}}\PYG{p}{)}
\PYG{n}{endresults}\PYG{p}{,} \PYG{n}{resgraph}\PYG{p}{,} \PYG{n}{mdlhist} \PYG{o}{=} \PYG{n}{fs}\PYG{o}{.}\PYG{n}{propagate}\PYG{o}{.}\PYG{n}{one\PYGZus{}fault}\PYG{p}{(}\PYG{n}{hierarchical\PYGZus{}model}\PYG{p}{,}\PYG{l+s+s1}{\PYGZsq{}}\PYG{l+s+s1}{AffectDOF}\PYG{l+s+s1}{\PYGZsq{}}\PYG{p}{,} \PYG{l+s+s1}{\PYGZsq{}}\PYG{l+s+s1}{RFmechbreak}\PYG{l+s+s1}{\PYGZsq{}}\PYG{p}{,} \PYG{n}{time}\PYG{o}{=}\PYG{l+m+mi}{50}\PYG{p}{)}
\end{sphinxVerbatim}
}

\end{sphinxuseclass}
\end{sphinxuseclass}
\begin{sphinxuseclass}{nbinput}
{
\sphinxsetup{VerbatimColor={named}{nbsphinx-code-bg}}
\sphinxsetup{VerbatimBorderColor={named}{nbsphinx-code-border}}
\begin{sphinxVerbatim}[commandchars=\\\{\}]
\llap{\color{nbsphinxin}[36]:\,\hspace{\fboxrule}\hspace{\fboxsep}}\PYG{n}{rd}\PYG{o}{.}\PYG{n}{plot}\PYG{o}{.}\PYG{n}{mdlhistvals}\PYG{p}{(}\PYG{n}{mdlhist}\PYG{p}{,}\PYG{l+s+s1}{\PYGZsq{}}\PYG{l+s+s1}{AffectDOF mechbreak}\PYG{l+s+s1}{\PYGZsq{}}\PYG{p}{,} \PYG{n}{time}\PYG{o}{=}\PYG{l+m+mi}{50}\PYG{p}{,} \PYG{n}{fxnflowvals}\PYG{o}{=}\PYG{p}{\PYGZob{}}\PYG{l+s+s1}{\PYGZsq{}}\PYG{l+s+s1}{Env1}\PYG{l+s+s1}{\PYGZsq{}}\PYG{p}{:}\PYG{p}{[}\PYG{l+s+s1}{\PYGZsq{}}\PYG{l+s+s1}{x}\PYG{l+s+s1}{\PYGZsq{}}\PYG{p}{,}\PYG{l+s+s1}{\PYGZsq{}}\PYG{l+s+s1}{y}\PYG{l+s+s1}{\PYGZsq{}}\PYG{p}{,}\PYG{l+s+s1}{\PYGZsq{}}\PYG{l+s+s1}{elev}\PYG{l+s+s1}{\PYGZsq{}}\PYG{p}{]}\PYG{p}{,} \PYG{l+s+s1}{\PYGZsq{}}\PYG{l+s+s1}{StoreEE}\PYG{l+s+s1}{\PYGZsq{}}\PYG{p}{:}\PYG{p}{[}\PYG{l+s+s1}{\PYGZsq{}}\PYG{l+s+s1}{soc}\PYG{l+s+s1}{\PYGZsq{}}\PYG{p}{]}\PYG{p}{\PYGZcb{}}\PYG{p}{,} \PYG{n}{legend}\PYG{o}{=}\PYG{k+kc}{False}\PYG{p}{)}
\end{sphinxVerbatim}
}

\end{sphinxuseclass}
\begin{sphinxuseclass}{nboutput}
\hrule height -\fboxrule\relax
\vspace{\nbsphinxcodecellspacing}

\savebox\nbsphinxpromptbox[0pt][r]{\color{nbsphinxout}\Verb|\strut{[36]:}\,|}

\begin{nbsphinxfancyoutput}

\begin{sphinxuseclass}{output_area}
\begin{sphinxuseclass}{}
\noindent\sphinxincludegraphics[width=437\sphinxpxdimen,height=286\sphinxpxdimen]{{example_multirotor_Demonstration_58_0}.png}

\end{sphinxuseclass}
\end{sphinxuseclass}
\end{nbsphinxfancyoutput}

\end{sphinxuseclass}
\begin{sphinxuseclass}{nboutput}
\begin{sphinxuseclass}{nblast}
\hrule height -\fboxrule\relax
\vspace{\nbsphinxcodecellspacing}

\makeatletter\setbox\nbsphinxpromptbox\box\voidb@x\makeatother

\begin{nbsphinxfancyoutput}

\begin{sphinxuseclass}{output_area}
\begin{sphinxuseclass}{}
\noindent\sphinxincludegraphics[width=437\sphinxpxdimen,height=286\sphinxpxdimen]{{example_multirotor_Demonstration_58_1}.png}

\end{sphinxuseclass}
\end{sphinxuseclass}
\end{nbsphinxfancyoutput}

\end{sphinxuseclass}
\end{sphinxuseclass}
\sphinxAtStartPar
Then we can see how the octorotor architecture performs in the same case:

\begin{sphinxuseclass}{nbinput}
\begin{sphinxuseclass}{nblast}
{
\sphinxsetup{VerbatimColor={named}{nbsphinx-code-bg}}
\sphinxsetup{VerbatimBorderColor={named}{nbsphinx-code-border}}
\begin{sphinxVerbatim}[commandchars=\\\{\}]
\llap{\color{nbsphinxin}[37]:\,\hspace{\fboxrule}\hspace{\fboxsep}}\PYG{n}{hierarchical\PYGZus{}model} \PYG{o}{=} \PYG{n}{Drone\PYGZus{}Hierarchical}\PYG{p}{(}\PYG{n}{params}\PYG{o}{=}\PYG{p}{\PYGZob{}}\PYG{l+s+s1}{\PYGZsq{}}\PYG{l+s+s1}{graph\PYGZus{}pos}\PYG{l+s+s1}{\PYGZsq{}}\PYG{p}{:}\PYG{n}{graph\PYGZus{}pos}\PYG{p}{,} \PYG{l+s+s1}{\PYGZsq{}}\PYG{l+s+s1}{bipartite\PYGZus{}pos}\PYG{l+s+s1}{\PYGZsq{}}\PYG{p}{:}\PYG{n}{bipartite\PYGZus{}pos}\PYG{p}{,}\PYG{l+s+s1}{\PYGZsq{}}\PYG{l+s+s1}{arch}\PYG{l+s+s1}{\PYGZsq{}}\PYG{p}{:}\PYG{l+s+s1}{\PYGZsq{}}\PYG{l+s+s1}{oct}\PYG{l+s+s1}{\PYGZsq{}}\PYG{p}{\PYGZcb{}}\PYG{p}{)}
\PYG{n}{endresults}\PYG{p}{,} \PYG{n}{resgraph}\PYG{p}{,} \PYG{n}{mdlhist} \PYG{o}{=} \PYG{n}{fs}\PYG{o}{.}\PYG{n}{propagate}\PYG{o}{.}\PYG{n}{one\PYGZus{}fault}\PYG{p}{(}\PYG{n}{hierarchical\PYGZus{}model}\PYG{p}{,}\PYG{l+s+s1}{\PYGZsq{}}\PYG{l+s+s1}{AffectDOF}\PYG{l+s+s1}{\PYGZsq{}}\PYG{p}{,} \PYG{l+s+s1}{\PYGZsq{}}\PYG{l+s+s1}{RFmechbreak}\PYG{l+s+s1}{\PYGZsq{}}\PYG{p}{,} \PYG{n}{time}\PYG{o}{=}\PYG{l+m+mi}{50}\PYG{p}{)}
\end{sphinxVerbatim}
}

\end{sphinxuseclass}
\end{sphinxuseclass}
\begin{sphinxuseclass}{nbinput}
{
\sphinxsetup{VerbatimColor={named}{nbsphinx-code-bg}}
\sphinxsetup{VerbatimBorderColor={named}{nbsphinx-code-border}}
\begin{sphinxVerbatim}[commandchars=\\\{\}]
\llap{\color{nbsphinxin}[38]:\,\hspace{\fboxrule}\hspace{\fboxsep}}\PYG{n}{fig}\PYG{o}{=} \PYG{n}{rd}\PYG{o}{.}\PYG{n}{plot}\PYG{o}{.}\PYG{n}{mdlhistvals}\PYG{p}{(}\PYG{n}{mdlhist}\PYG{p}{,}\PYG{l+s+s1}{\PYGZsq{}}\PYG{l+s+s1}{AffectDOF: RFmechbreak}\PYG{l+s+s1}{\PYGZsq{}}\PYG{p}{,} \PYG{n}{time}\PYG{o}{=}\PYG{l+m+mi}{50}\PYG{p}{,} \PYG{n}{fxnflowvals}\PYG{o}{=}\PYG{p}{\PYGZob{}}\PYG{l+s+s1}{\PYGZsq{}}\PYG{l+s+s1}{Env1}\PYG{l+s+s1}{\PYGZsq{}}\PYG{p}{:}\PYG{p}{[}\PYG{l+s+s1}{\PYGZsq{}}\PYG{l+s+s1}{x}\PYG{l+s+s1}{\PYGZsq{}}\PYG{p}{,}\PYG{l+s+s1}{\PYGZsq{}}\PYG{l+s+s1}{y}\PYG{l+s+s1}{\PYGZsq{}}\PYG{p}{,}\PYG{l+s+s1}{\PYGZsq{}}\PYG{l+s+s1}{elev}\PYG{l+s+s1}{\PYGZsq{}}\PYG{p}{]}\PYG{p}{,} \PYG{l+s+s1}{\PYGZsq{}}\PYG{l+s+s1}{StoreEE}\PYG{l+s+s1}{\PYGZsq{}}\PYG{p}{:}\PYG{p}{[}\PYG{l+s+s1}{\PYGZsq{}}\PYG{l+s+s1}{soc}\PYG{l+s+s1}{\PYGZsq{}}\PYG{p}{]}\PYG{p}{\PYGZcb{}}\PYG{p}{,} \PYG{n}{units}\PYG{o}{=}\PYG{p}{[}\PYG{l+s+s1}{\PYGZsq{}}\PYG{l+s+s1}{m}\PYG{l+s+s1}{\PYGZsq{}}\PYG{p}{,}\PYG{l+s+s1}{\PYGZsq{}}\PYG{l+s+s1}{m}\PYG{l+s+s1}{\PYGZsq{}}\PYG{p}{,}\PYG{l+s+s1}{\PYGZsq{}}\PYG{l+s+s1}{m}\PYG{l+s+s1}{\PYGZsq{}}\PYG{p}{,}\PYG{l+s+s1}{\PYGZsq{}}\PYG{l+s+s1}{\PYGZpc{}}\PYG{l+s+s1}{\PYGZsq{}}\PYG{p}{]}\PYG{p}{,} \PYG{n}{legend}\PYG{o}{=}\PYG{k+kc}{False}\PYG{p}{,} \PYG{n}{returnfig}\PYG{o}{=}\PYG{k+kc}{True}\PYG{p}{)}

\PYG{n}{ax} \PYG{o}{=} \PYG{n}{fig}\PYG{o}{.}\PYG{n}{axes}\PYG{p}{[}\PYG{l+m+mi}{1}\PYG{p}{]}
\PYG{n}{ax}\PYG{o}{.}\PYG{n}{legend}\PYG{p}{(}\PYG{p}{[}\PYG{l+s+s1}{\PYGZsq{}}\PYG{l+s+s1}{faulty}\PYG{l+s+s1}{\PYGZsq{}}\PYG{p}{,} \PYG{l+s+s1}{\PYGZsq{}}\PYG{l+s+s1}{nominal}\PYG{l+s+s1}{\PYGZsq{}}\PYG{p}{]}\PYG{p}{,} \PYG{n}{loc}\PYG{o}{=}\PYG{l+s+s1}{\PYGZsq{}}\PYG{l+s+s1}{right}\PYG{l+s+s1}{\PYGZsq{}}\PYG{p}{)}

\PYG{n}{fig}\PYG{o}{.}\PYG{n}{savefig}\PYG{p}{(}\PYG{l+s+s2}{\PYGZdq{}}\PYG{l+s+s2}{red\PYGZus{}fault\PYGZus{}behavior.pdf}\PYG{l+s+s2}{\PYGZdq{}}\PYG{p}{,} \PYG{n+nb}{format}\PYG{o}{=}\PYG{l+s+s2}{\PYGZdq{}}\PYG{l+s+s2}{pdf}\PYG{l+s+s2}{\PYGZdq{}}\PYG{p}{,} \PYG{n}{bbox\PYGZus{}inches} \PYG{o}{=} \PYG{l+s+s1}{\PYGZsq{}}\PYG{l+s+s1}{tight}\PYG{l+s+s1}{\PYGZsq{}}\PYG{p}{,} \PYG{n}{pad\PYGZus{}inches} \PYG{o}{=} \PYG{l+m+mi}{0}\PYG{p}{)}
\end{sphinxVerbatim}
}

\end{sphinxuseclass}
\begin{sphinxuseclass}{nboutput}
\begin{sphinxuseclass}{nblast}
\hrule height -\fboxrule\relax
\vspace{\nbsphinxcodecellspacing}

\makeatletter\setbox\nbsphinxpromptbox\box\voidb@x\makeatother

\begin{nbsphinxfancyoutput}

\begin{sphinxuseclass}{output_area}
\begin{sphinxuseclass}{}
\noindent\sphinxincludegraphics[width=456\sphinxpxdimen,height=286\sphinxpxdimen]{{example_multirotor_Demonstration_61_0}.png}

\end{sphinxuseclass}
\end{sphinxuseclass}
\end{nbsphinxfancyoutput}

\end{sphinxuseclass}
\end{sphinxuseclass}
\sphinxAtStartPar
As shown, the octorotor architecture enables the quadrotor to recover from the fault and land.

\sphinxAtStartPar
Next, we can compare how each architecture mitigates the set of faults that originiate in each function: \#\#\# Quadcopter Resilience

\sphinxAtStartPar
Here we quantify the expected costs of faults originiating in the quadcopter architecture:

\begin{sphinxuseclass}{nbinput}
\begin{sphinxuseclass}{nblast}
{
\sphinxsetup{VerbatimColor={named}{nbsphinx-code-bg}}
\sphinxsetup{VerbatimBorderColor={named}{nbsphinx-code-border}}
\begin{sphinxVerbatim}[commandchars=\\\{\}]
\llap{\color{nbsphinxin}[39]:\,\hspace{\fboxrule}\hspace{\fboxsep}}\PYG{n}{mdl\PYGZus{}quad} \PYG{o}{=} \PYG{n}{Drone\PYGZus{}Hierarchical}\PYG{p}{(}\PYG{n}{params}\PYG{o}{=}\PYG{p}{\PYGZob{}}\PYG{l+s+s1}{\PYGZsq{}}\PYG{l+s+s1}{graph\PYGZus{}pos}\PYG{l+s+s1}{\PYGZsq{}}\PYG{p}{:}\PYG{n}{graph\PYGZus{}pos}\PYG{p}{,} \PYG{l+s+s1}{\PYGZsq{}}\PYG{l+s+s1}{bipartite\PYGZus{}pos}\PYG{l+s+s1}{\PYGZsq{}}\PYG{p}{:}\PYG{n}{bipartite\PYGZus{}pos}\PYG{p}{,}\PYG{l+s+s1}{\PYGZsq{}}\PYG{l+s+s1}{arch}\PYG{l+s+s1}{\PYGZsq{}}\PYG{p}{:}\PYG{l+s+s1}{\PYGZsq{}}\PYG{l+s+s1}{quad}\PYG{l+s+s1}{\PYGZsq{}}\PYG{p}{\PYGZcb{}}\PYG{p}{)}
\PYG{n}{mdl\PYGZus{}quad}\PYG{o}{.}\PYG{n}{fxns}\PYG{p}{[}\PYG{l+s+s1}{\PYGZsq{}}\PYG{l+s+s1}{AffectDOF}\PYG{l+s+s1}{\PYGZsq{}}\PYG{p}{]}\PYG{o}{.}\PYG{n}{faultmodes}
\PYG{n}{quad\PYGZus{}faults} \PYG{o}{=} \PYG{p}{[}\PYG{p}{(}\PYG{l+s+s1}{\PYGZsq{}}\PYG{l+s+s1}{AffectDOF}\PYG{l+s+s1}{\PYGZsq{}}\PYG{p}{,} \PYG{n}{fault}\PYG{p}{)} \PYG{k}{for} \PYG{n}{fault} \PYG{o+ow}{in} \PYG{n+nb}{list}\PYG{p}{(}\PYG{n}{mdl\PYGZus{}quad}\PYG{o}{.}\PYG{n}{fxns}\PYG{p}{[}\PYG{l+s+s1}{\PYGZsq{}}\PYG{l+s+s1}{AffectDOF}\PYG{l+s+s1}{\PYGZsq{}}\PYG{p}{]}\PYG{o}{.}\PYG{n}{faultmodes}\PYG{o}{.}\PYG{n}{keys}\PYG{p}{(}\PYG{p}{)}\PYG{p}{)}\PYG{p}{]}
\end{sphinxVerbatim}
}

\end{sphinxuseclass}
\end{sphinxuseclass}
\begin{sphinxuseclass}{nbinput}
{
\sphinxsetup{VerbatimColor={named}{nbsphinx-code-bg}}
\sphinxsetup{VerbatimBorderColor={named}{nbsphinx-code-border}}
\begin{sphinxVerbatim}[commandchars=\\\{\}]
\llap{\color{nbsphinxin}[40]:\,\hspace{\fboxrule}\hspace{\fboxsep}}\PYG{n}{quad\PYGZus{}app} \PYG{o}{=} \PYG{n}{SampleApproach}\PYG{p}{(}\PYG{n}{mdl\PYGZus{}quad}\PYG{p}{,} \PYG{n}{faults}\PYG{o}{=}\PYG{n}{quad\PYGZus{}faults}\PYG{p}{)}
\PYG{n}{quad\PYGZus{}endclasses}\PYG{p}{,} \PYG{n}{quad\PYGZus{}mdlhists} \PYG{o}{=} \PYG{n}{fs}\PYG{o}{.}\PYG{n}{propagate}\PYG{o}{.}\PYG{n}{approach}\PYG{p}{(}\PYG{n}{mdl\PYGZus{}quad}\PYG{p}{,} \PYG{n}{quad\PYGZus{}app}\PYG{p}{,} \PYG{n}{staged}\PYG{o}{=}\PYG{k+kc}{True}\PYG{p}{)}
\end{sphinxVerbatim}
}

\end{sphinxuseclass}
\begin{sphinxuseclass}{nboutput}
\begin{sphinxuseclass}{nblast}
{

\kern-\sphinxverbatimsmallskipamount\kern-\baselineskip
\kern+\FrameHeightAdjust\kern-\fboxrule
\vspace{\nbsphinxcodecellspacing}

\sphinxsetup{VerbatimColor={named}{nbsphinx-stderr}}
\sphinxsetup{VerbatimBorderColor={named}{nbsphinx-code-border}}
\begin{sphinxuseclass}{output_area}
\begin{sphinxuseclass}{stderr}


\begin{sphinxVerbatim}[commandchars=\\\{\}]
SCENARIOS COMPLETE: 100\%|████████████████████████████████████████████████████████████| 104/104 [01:07<00:00,  1.54it/s]
\end{sphinxVerbatim}



\end{sphinxuseclass}
\end{sphinxuseclass}
}

\end{sphinxuseclass}
\end{sphinxuseclass}
\begin{sphinxuseclass}{nbinput}
{
\sphinxsetup{VerbatimColor={named}{nbsphinx-code-bg}}
\sphinxsetup{VerbatimBorderColor={named}{nbsphinx-code-border}}
\begin{sphinxVerbatim}[commandchars=\\\{\}]
\llap{\color{nbsphinxin}[41]:\,\hspace{\fboxrule}\hspace{\fboxsep}}\PYG{n}{quad\PYGZus{}tab} \PYG{o}{=} \PYG{n}{rd}\PYG{o}{.}\PYG{n}{tabulate}\PYG{o}{.}\PYG{n}{simplefmea}\PYG{p}{(}\PYG{n}{quad\PYGZus{}endclasses}\PYG{p}{)}
\PYG{n}{quad\PYGZus{}tab}\PYG{o}{.}\PYG{n}{sort\PYGZus{}values}\PYG{p}{(}\PYG{l+s+s1}{\PYGZsq{}}\PYG{l+s+s1}{expected cost}\PYG{l+s+s1}{\PYGZsq{}}\PYG{p}{,} \PYG{n}{ascending}\PYG{o}{=}\PYG{k+kc}{False}\PYG{p}{)}
\end{sphinxVerbatim}
}

\end{sphinxuseclass}
\begin{sphinxuseclass}{nboutput}
\begin{sphinxuseclass}{nblast}
{

\kern-\sphinxverbatimsmallskipamount\kern-\baselineskip
\kern+\FrameHeightAdjust\kern-\fboxrule
\vspace{\nbsphinxcodecellspacing}

\sphinxsetup{VerbatimColor={named}{white}}
\sphinxsetup{VerbatimBorderColor={named}{nbsphinx-code-border}}
\begin{sphinxuseclass}{output_area}
\begin{sphinxuseclass}{}


\begin{sphinxVerbatim}[commandchars=\\\{\}]
\llap{\color{nbsphinxout}[41]:\,\hspace{\fboxrule}\hspace{\fboxsep}}                                      rate      cost  expected cost
AffectDOF LRctlbreak, t=49.0  1.666667e-07  186800.0    3113.333333
AffectDOF RRctlbreak, t=49.0  1.666667e-07  186800.0    3113.333333
AffectDOF LFctlbreak, t=49.0  1.666667e-07  186800.0    3113.333333
AffectDOF RFctlbreak, t=49.0  1.666667e-07  186800.0    3113.333333
AffectDOF LFctlup, t=49.0     1.666667e-07  186300.0    3105.000000
{\ldots}                                    {\ldots}       {\ldots}            {\ldots}
AffectDOF RRpropwarp, t=97.0  6.944444e-10     200.0       0.013889
AffectDOF RFpropwarp, t=97.0  6.944444e-10     200.0       0.013889
AffectDOF LRpropwarp, t=97.0  6.944444e-10     200.0       0.013889
AffectDOF LFpropwarp, t=97.0  6.944444e-10     200.0       0.013889
nominal                       1.000000e+00       0.0       0.000000

[105 rows x 3 columns]
\end{sphinxVerbatim}



\end{sphinxuseclass}
\end{sphinxuseclass}
}

\end{sphinxuseclass}
\end{sphinxuseclass}
\sphinxAtStartPar
Based on this model, we can calculate some metrics that quantify how resilient the system was to the set of faults, such as the cost of resilience:

\begin{sphinxuseclass}{nbinput}
{
\sphinxsetup{VerbatimColor={named}{nbsphinx-code-bg}}
\sphinxsetup{VerbatimBorderColor={named}{nbsphinx-code-border}}
\begin{sphinxVerbatim}[commandchars=\\\{\}]
\llap{\color{nbsphinxin}[42]:\,\hspace{\fboxrule}\hspace{\fboxsep}}\PYG{n}{quad\PYGZus{}res} \PYG{o}{=} \PYG{n+nb}{sum}\PYG{p}{(}\PYG{n}{quad\PYGZus{}tab}\PYG{p}{[}\PYG{l+s+s1}{\PYGZsq{}}\PYG{l+s+s1}{expected cost}\PYG{l+s+s1}{\PYGZsq{}}\PYG{p}{]}\PYG{p}{)}
\PYG{n}{quad\PYGZus{}res}
\end{sphinxVerbatim}
}

\end{sphinxuseclass}
\begin{sphinxuseclass}{nboutput}
\begin{sphinxuseclass}{nblast}
{

\kern-\sphinxverbatimsmallskipamount\kern-\baselineskip
\kern+\FrameHeightAdjust\kern-\fboxrule
\vspace{\nbsphinxcodecellspacing}

\sphinxsetup{VerbatimColor={named}{white}}
\sphinxsetup{VerbatimBorderColor={named}{nbsphinx-code-border}}
\begin{sphinxuseclass}{output_area}
\begin{sphinxuseclass}{}


\begin{sphinxVerbatim}[commandchars=\\\{\}]
\llap{\color{nbsphinxout}[42]:\,\hspace{\fboxrule}\hspace{\fboxsep}}45564.95370370373
\end{sphinxVerbatim}



\end{sphinxuseclass}
\end{sphinxuseclass}
}

\end{sphinxuseclass}
\end{sphinxuseclass}
\sphinxAtStartPar
The overall rate of crashes:

\begin{sphinxuseclass}{nbinput}
{
\sphinxsetup{VerbatimColor={named}{nbsphinx-code-bg}}
\sphinxsetup{VerbatimBorderColor={named}{nbsphinx-code-border}}
\begin{sphinxVerbatim}[commandchars=\\\{\}]
\llap{\color{nbsphinxin}[43]:\,\hspace{\fboxrule}\hspace{\fboxsep}}\PYG{n}{quad\PYGZus{}crashes} \PYG{o}{=} \PYG{n}{quad\PYGZus{}tab}\PYG{p}{[}\PYG{n}{quad\PYGZus{}tab}\PYG{p}{[}\PYG{l+s+s1}{\PYGZsq{}}\PYG{l+s+s1}{cost}\PYG{l+s+s1}{\PYGZsq{}}\PYG{p}{]}\PYG{o}{\PYGZgt{}}\PYG{l+m+mi}{100000}\PYG{p}{]}
\PYG{n}{quad\PYGZus{}rate} \PYG{o}{=} \PYG{n+nb}{sum}\PYG{p}{(}\PYG{n}{quad\PYGZus{}crashes}\PYG{p}{[}\PYG{l+s+s1}{\PYGZsq{}}\PYG{l+s+s1}{rate}\PYG{l+s+s1}{\PYGZsq{}}\PYG{p}{]}\PYG{p}{)}
\PYG{n}{quad\PYGZus{}rate}
\end{sphinxVerbatim}
}

\end{sphinxuseclass}
\begin{sphinxuseclass}{nboutput}
\begin{sphinxuseclass}{nblast}
{

\kern-\sphinxverbatimsmallskipamount\kern-\baselineskip
\kern+\FrameHeightAdjust\kern-\fboxrule
\vspace{\nbsphinxcodecellspacing}

\sphinxsetup{VerbatimColor={named}{white}}
\sphinxsetup{VerbatimBorderColor={named}{nbsphinx-code-border}}
\begin{sphinxuseclass}{output_area}
\begin{sphinxuseclass}{}


\begin{sphinxVerbatim}[commandchars=\\\{\}]
\llap{\color{nbsphinxout}[43]:\,\hspace{\fboxrule}\hspace{\fboxsep}}2.3796296296296308e-06
\end{sphinxVerbatim}



\end{sphinxuseclass}
\end{sphinxuseclass}
}

\end{sphinxuseclass}
\end{sphinxuseclass}
\sphinxAtStartPar
The number of crashes:

\begin{sphinxuseclass}{nbinput}
{
\sphinxsetup{VerbatimColor={named}{nbsphinx-code-bg}}
\sphinxsetup{VerbatimBorderColor={named}{nbsphinx-code-border}}
\begin{sphinxVerbatim}[commandchars=\\\{\}]
\llap{\color{nbsphinxin}[44]:\,\hspace{\fboxrule}\hspace{\fboxsep}}\PYG{n}{quad\PYGZus{}num\PYGZus{}crashes} \PYG{o}{=} \PYG{n+nb}{len}\PYG{p}{(}\PYG{n}{quad\PYGZus{}crashes}\PYG{p}{[}\PYG{l+s+s1}{\PYGZsq{}}\PYG{l+s+s1}{rate}\PYG{l+s+s1}{\PYGZsq{}}\PYG{p}{]}\PYG{p}{)}
\PYG{n}{quad\PYGZus{}num\PYGZus{}crashes}
\end{sphinxVerbatim}
}

\end{sphinxuseclass}
\begin{sphinxuseclass}{nboutput}
\begin{sphinxuseclass}{nblast}
{

\kern-\sphinxverbatimsmallskipamount\kern-\baselineskip
\kern+\FrameHeightAdjust\kern-\fboxrule
\vspace{\nbsphinxcodecellspacing}

\sphinxsetup{VerbatimColor={named}{white}}
\sphinxsetup{VerbatimBorderColor={named}{nbsphinx-code-border}}
\begin{sphinxuseclass}{output_area}
\begin{sphinxuseclass}{}


\begin{sphinxVerbatim}[commandchars=\\\{\}]
\llap{\color{nbsphinxout}[44]:\,\hspace{\fboxrule}\hspace{\fboxsep}}46
\end{sphinxVerbatim}



\end{sphinxuseclass}
\end{sphinxuseclass}
}

\end{sphinxuseclass}
\end{sphinxuseclass}
\sphinxAtStartPar
The percentage of crashes:

\begin{sphinxuseclass}{nbinput}
{
\sphinxsetup{VerbatimColor={named}{nbsphinx-code-bg}}
\sphinxsetup{VerbatimBorderColor={named}{nbsphinx-code-border}}
\begin{sphinxVerbatim}[commandchars=\\\{\}]
\llap{\color{nbsphinxin}[45]:\,\hspace{\fboxrule}\hspace{\fboxsep}}\PYG{n}{quad\PYGZus{}perc\PYGZus{}crashes} \PYG{o}{=} \PYG{n+nb}{len}\PYG{p}{(}\PYG{n}{quad\PYGZus{}crashes}\PYG{p}{[}\PYG{l+s+s1}{\PYGZsq{}}\PYG{l+s+s1}{rate}\PYG{l+s+s1}{\PYGZsq{}}\PYG{p}{]}\PYG{p}{)}\PYG{o}{/}\PYG{n+nb}{len}\PYG{p}{(}\PYG{n}{quad\PYGZus{}tab}\PYG{p}{[}\PYG{l+s+s1}{\PYGZsq{}}\PYG{l+s+s1}{rate}\PYG{l+s+s1}{\PYGZsq{}}\PYG{p}{]}\PYG{p}{)}
\PYG{n}{quad\PYGZus{}perc\PYGZus{}crashes}
\end{sphinxVerbatim}
}

\end{sphinxuseclass}
\begin{sphinxuseclass}{nboutput}
\begin{sphinxuseclass}{nblast}
{

\kern-\sphinxverbatimsmallskipamount\kern-\baselineskip
\kern+\FrameHeightAdjust\kern-\fboxrule
\vspace{\nbsphinxcodecellspacing}

\sphinxsetup{VerbatimColor={named}{white}}
\sphinxsetup{VerbatimBorderColor={named}{nbsphinx-code-border}}
\begin{sphinxuseclass}{output_area}
\begin{sphinxuseclass}{}


\begin{sphinxVerbatim}[commandchars=\\\{\}]
\llap{\color{nbsphinxout}[45]:\,\hspace{\fboxrule}\hspace{\fboxsep}}0.4380952380952381
\end{sphinxVerbatim}



\end{sphinxuseclass}
\end{sphinxuseclass}
}

\end{sphinxuseclass}
\end{sphinxuseclass}

\paragraph{Octocopter Resilience}
\label{\detokenize{example_multirotor/Demonstration:Octocopter-Resilience}}
\sphinxAtStartPar
Here we quantify the expected costs of faults originiating in the octocopter architecture:

\begin{sphinxuseclass}{nbinput}
\begin{sphinxuseclass}{nblast}
{
\sphinxsetup{VerbatimColor={named}{nbsphinx-code-bg}}
\sphinxsetup{VerbatimBorderColor={named}{nbsphinx-code-border}}
\begin{sphinxVerbatim}[commandchars=\\\{\}]
\llap{\color{nbsphinxin}[46]:\,\hspace{\fboxrule}\hspace{\fboxsep}}\PYG{n}{mdl\PYGZus{}oct} \PYG{o}{=} \PYG{n}{Drone\PYGZus{}Hierarchical}\PYG{p}{(}\PYG{n}{params}\PYG{o}{=}\PYG{p}{\PYGZob{}}\PYG{l+s+s1}{\PYGZsq{}}\PYG{l+s+s1}{graph\PYGZus{}pos}\PYG{l+s+s1}{\PYGZsq{}}\PYG{p}{:}\PYG{n}{graph\PYGZus{}pos}\PYG{p}{,} \PYG{l+s+s1}{\PYGZsq{}}\PYG{l+s+s1}{bipartite\PYGZus{}pos}\PYG{l+s+s1}{\PYGZsq{}}\PYG{p}{:}\PYG{n}{bipartite\PYGZus{}pos}\PYG{p}{,}\PYG{l+s+s1}{\PYGZsq{}}\PYG{l+s+s1}{arch}\PYG{l+s+s1}{\PYGZsq{}}\PYG{p}{:}\PYG{l+s+s1}{\PYGZsq{}}\PYG{l+s+s1}{oct}\PYG{l+s+s1}{\PYGZsq{}}\PYG{p}{\PYGZcb{}}\PYG{p}{)}
\PYG{n}{mdl\PYGZus{}oct}\PYG{o}{.}\PYG{n}{fxns}\PYG{p}{[}\PYG{l+s+s1}{\PYGZsq{}}\PYG{l+s+s1}{AffectDOF}\PYG{l+s+s1}{\PYGZsq{}}\PYG{p}{]}\PYG{o}{.}\PYG{n}{faultmodes}
\PYG{n}{oct\PYGZus{}faults} \PYG{o}{=} \PYG{p}{[}\PYG{p}{(}\PYG{l+s+s1}{\PYGZsq{}}\PYG{l+s+s1}{AffectDOF}\PYG{l+s+s1}{\PYGZsq{}}\PYG{p}{,} \PYG{n}{fault}\PYG{p}{)} \PYG{k}{for} \PYG{n}{fault} \PYG{o+ow}{in} \PYG{n+nb}{list}\PYG{p}{(}\PYG{n}{mdl\PYGZus{}oct}\PYG{o}{.}\PYG{n}{fxns}\PYG{p}{[}\PYG{l+s+s1}{\PYGZsq{}}\PYG{l+s+s1}{AffectDOF}\PYG{l+s+s1}{\PYGZsq{}}\PYG{p}{]}\PYG{o}{.}\PYG{n}{faultmodes}\PYG{o}{.}\PYG{n}{keys}\PYG{p}{(}\PYG{p}{)}\PYG{p}{)}\PYG{p}{]}
\end{sphinxVerbatim}
}

\end{sphinxuseclass}
\end{sphinxuseclass}
\begin{sphinxuseclass}{nbinput}
{
\sphinxsetup{VerbatimColor={named}{nbsphinx-code-bg}}
\sphinxsetup{VerbatimBorderColor={named}{nbsphinx-code-border}}
\begin{sphinxVerbatim}[commandchars=\\\{\}]
\llap{\color{nbsphinxin}[47]:\,\hspace{\fboxrule}\hspace{\fboxsep}}\PYG{n}{oct\PYGZus{}app} \PYG{o}{=} \PYG{n}{SampleApproach}\PYG{p}{(}\PYG{n}{mdl\PYGZus{}oct}\PYG{p}{,} \PYG{n}{faults}\PYG{o}{=}\PYG{n}{oct\PYGZus{}faults}\PYG{p}{)}
\PYG{n}{oct\PYGZus{}endclasses}\PYG{p}{,} \PYG{n}{oct\PYGZus{}mdlhists} \PYG{o}{=} \PYG{n}{fs}\PYG{o}{.}\PYG{n}{propagate}\PYG{o}{.}\PYG{n}{approach}\PYG{p}{(}\PYG{n}{mdl\PYGZus{}oct}\PYG{p}{,} \PYG{n}{oct\PYGZus{}app}\PYG{p}{,} \PYG{n}{staged}\PYG{o}{=}\PYG{k+kc}{True}\PYG{p}{)}
\end{sphinxVerbatim}
}

\end{sphinxuseclass}
\begin{sphinxuseclass}{nboutput}
\begin{sphinxuseclass}{nblast}
{

\kern-\sphinxverbatimsmallskipamount\kern-\baselineskip
\kern+\FrameHeightAdjust\kern-\fboxrule
\vspace{\nbsphinxcodecellspacing}

\sphinxsetup{VerbatimColor={named}{nbsphinx-stderr}}
\sphinxsetup{VerbatimBorderColor={named}{nbsphinx-code-border}}
\begin{sphinxuseclass}{output_area}
\begin{sphinxuseclass}{stderr}


\begin{sphinxVerbatim}[commandchars=\\\{\}]
SCENARIOS COMPLETE: 100\%|████████████████████████████████████████████████████████████| 208/208 [02:58<00:00,  1.17it/s]
\end{sphinxVerbatim}



\end{sphinxuseclass}
\end{sphinxuseclass}
}

\end{sphinxuseclass}
\end{sphinxuseclass}
\begin{sphinxuseclass}{nbinput}
{
\sphinxsetup{VerbatimColor={named}{nbsphinx-code-bg}}
\sphinxsetup{VerbatimBorderColor={named}{nbsphinx-code-border}}
\begin{sphinxVerbatim}[commandchars=\\\{\}]
\llap{\color{nbsphinxin}[48]:\,\hspace{\fboxrule}\hspace{\fboxsep}}\PYG{n}{oct\PYGZus{}tab} \PYG{o}{=} \PYG{n}{rd}\PYG{o}{.}\PYG{n}{tabulate}\PYG{o}{.}\PYG{n}{simplefmea}\PYG{p}{(}\PYG{n}{oct\PYGZus{}endclasses}\PYG{p}{)}
\PYG{n}{oct\PYGZus{}tab}\PYG{o}{.}\PYG{n}{sort\PYGZus{}values}\PYG{p}{(}\PYG{l+s+s1}{\PYGZsq{}}\PYG{l+s+s1}{expected cost}\PYG{l+s+s1}{\PYGZsq{}}\PYG{p}{,} \PYG{n}{ascending}\PYG{o}{=}\PYG{k+kc}{False}\PYG{p}{)}
\end{sphinxVerbatim}
}

\end{sphinxuseclass}
\begin{sphinxuseclass}{nboutput}
\begin{sphinxuseclass}{nblast}
{

\kern-\sphinxverbatimsmallskipamount\kern-\baselineskip
\kern+\FrameHeightAdjust\kern-\fboxrule
\vspace{\nbsphinxcodecellspacing}

\sphinxsetup{VerbatimColor={named}{white}}
\sphinxsetup{VerbatimBorderColor={named}{nbsphinx-code-border}}
\begin{sphinxuseclass}{output_area}
\begin{sphinxuseclass}{}


\begin{sphinxVerbatim}[commandchars=\\\{\}]
\llap{\color{nbsphinxout}[48]:\,\hspace{\fboxrule}\hspace{\fboxsep}}                                       rate      cost  expected cost
AffectDOF LRshort, t=49.0      8.333333e-08  191800.0    1598.333333
AffectDOF RF2short, t=49.0     8.333333e-08  191800.0    1598.333333
AffectDOF LR2short, t=49.0     8.333333e-08  191800.0    1598.333333
AffectDOF LFshort, t=49.0      8.333333e-08  191800.0    1598.333333
AffectDOF RRshort, t=49.0      8.333333e-08  191800.0    1598.333333
{\ldots}                                     {\ldots}       {\ldots}            {\ldots}
AffectDOF LF2propwarp, t=97.0  6.944444e-10     200.0       0.013889
AffectDOF RRpropwarp, t=97.0   6.944444e-10     200.0       0.013889
AffectDOF RR2propwarp, t=97.0  6.944444e-10     200.0       0.013889
AffectDOF RF2propwarp, t=97.0  6.944444e-10     200.0       0.013889
nominal                        1.000000e+00       0.0       0.000000

[209 rows x 3 columns]
\end{sphinxVerbatim}



\end{sphinxuseclass}
\end{sphinxuseclass}
}

\end{sphinxuseclass}
\end{sphinxuseclass}
\sphinxAtStartPar
Based on this model, we can calculate some metrics that quantify how resilient the system was to the set of faults, such as the cost of resilience:

\begin{sphinxuseclass}{nbinput}
{
\sphinxsetup{VerbatimColor={named}{nbsphinx-code-bg}}
\sphinxsetup{VerbatimBorderColor={named}{nbsphinx-code-border}}
\begin{sphinxVerbatim}[commandchars=\\\{\}]
\llap{\color{nbsphinxin}[49]:\,\hspace{\fboxrule}\hspace{\fboxsep}}\PYG{n}{oct\PYGZus{}res} \PYG{o}{=} \PYG{n+nb}{sum}\PYG{p}{(}\PYG{n}{oct\PYGZus{}tab}\PYG{p}{[}\PYG{l+s+s1}{\PYGZsq{}}\PYG{l+s+s1}{expected cost}\PYG{l+s+s1}{\PYGZsq{}}\PYG{p}{]}\PYG{p}{)}
\PYG{n}{oct\PYGZus{}res}
\end{sphinxVerbatim}
}

\end{sphinxuseclass}
\begin{sphinxuseclass}{nboutput}
\begin{sphinxuseclass}{nblast}
{

\kern-\sphinxverbatimsmallskipamount\kern-\baselineskip
\kern+\FrameHeightAdjust\kern-\fboxrule
\vspace{\nbsphinxcodecellspacing}

\sphinxsetup{VerbatimColor={named}{white}}
\sphinxsetup{VerbatimBorderColor={named}{nbsphinx-code-border}}
\begin{sphinxuseclass}{output_area}
\begin{sphinxuseclass}{}


\begin{sphinxVerbatim}[commandchars=\\\{\}]
\llap{\color{nbsphinxout}[49]:\,\hspace{\fboxrule}\hspace{\fboxsep}}19358.870370370398
\end{sphinxVerbatim}



\end{sphinxuseclass}
\end{sphinxuseclass}
}

\end{sphinxuseclass}
\end{sphinxuseclass}
\sphinxAtStartPar
The overall rate of crashes:

\begin{sphinxuseclass}{nbinput}
{
\sphinxsetup{VerbatimColor={named}{nbsphinx-code-bg}}
\sphinxsetup{VerbatimBorderColor={named}{nbsphinx-code-border}}
\begin{sphinxVerbatim}[commandchars=\\\{\}]
\llap{\color{nbsphinxin}[50]:\,\hspace{\fboxrule}\hspace{\fboxsep}}\PYG{n}{oct\PYGZus{}crashes} \PYG{o}{=} \PYG{n}{oct\PYGZus{}tab}\PYG{p}{[}\PYG{n}{oct\PYGZus{}tab}\PYG{p}{[}\PYG{l+s+s1}{\PYGZsq{}}\PYG{l+s+s1}{cost}\PYG{l+s+s1}{\PYGZsq{}}\PYG{p}{]}\PYG{o}{\PYGZgt{}}\PYG{l+m+mi}{100000}\PYG{p}{]}
\PYG{n}{oct\PYGZus{}rate} \PYG{o}{=} \PYG{n+nb}{sum}\PYG{p}{(}\PYG{n}{oct\PYGZus{}crashes}\PYG{p}{[}\PYG{l+s+s1}{\PYGZsq{}}\PYG{l+s+s1}{rate}\PYG{l+s+s1}{\PYGZsq{}}\PYG{p}{]}\PYG{p}{)}
\PYG{n}{oct\PYGZus{}rate}
\end{sphinxVerbatim}
}

\end{sphinxuseclass}
\begin{sphinxuseclass}{nboutput}
\begin{sphinxuseclass}{nblast}
{

\kern-\sphinxverbatimsmallskipamount\kern-\baselineskip
\kern+\FrameHeightAdjust\kern-\fboxrule
\vspace{\nbsphinxcodecellspacing}

\sphinxsetup{VerbatimColor={named}{white}}
\sphinxsetup{VerbatimBorderColor={named}{nbsphinx-code-border}}
\begin{sphinxuseclass}{output_area}
\begin{sphinxuseclass}{}


\begin{sphinxVerbatim}[commandchars=\\\{\}]
\llap{\color{nbsphinxout}[50]:\,\hspace{\fboxrule}\hspace{\fboxsep}}7.037037037037039e-07
\end{sphinxVerbatim}



\end{sphinxuseclass}
\end{sphinxuseclass}
}

\end{sphinxuseclass}
\end{sphinxuseclass}
\sphinxAtStartPar
Number of crashes:

\begin{sphinxuseclass}{nbinput}
{
\sphinxsetup{VerbatimColor={named}{nbsphinx-code-bg}}
\sphinxsetup{VerbatimBorderColor={named}{nbsphinx-code-border}}
\begin{sphinxVerbatim}[commandchars=\\\{\}]
\llap{\color{nbsphinxin}[51]:\,\hspace{\fboxrule}\hspace{\fboxsep}}\PYG{n}{oct\PYGZus{}num\PYGZus{}crashes} \PYG{o}{=} \PYG{n+nb}{len}\PYG{p}{(}\PYG{n}{oct\PYGZus{}crashes}\PYG{p}{[}\PYG{l+s+s1}{\PYGZsq{}}\PYG{l+s+s1}{rate}\PYG{l+s+s1}{\PYGZsq{}}\PYG{p}{]}\PYG{p}{)}
\PYG{n}{oct\PYGZus{}num\PYGZus{}crashes}
\end{sphinxVerbatim}
}

\end{sphinxuseclass}
\begin{sphinxuseclass}{nboutput}
\begin{sphinxuseclass}{nblast}
{

\kern-\sphinxverbatimsmallskipamount\kern-\baselineskip
\kern+\FrameHeightAdjust\kern-\fboxrule
\vspace{\nbsphinxcodecellspacing}

\sphinxsetup{VerbatimColor={named}{white}}
\sphinxsetup{VerbatimBorderColor={named}{nbsphinx-code-border}}
\begin{sphinxuseclass}{output_area}
\begin{sphinxuseclass}{}


\begin{sphinxVerbatim}[commandchars=\\\{\}]
\llap{\color{nbsphinxout}[51]:\,\hspace{\fboxrule}\hspace{\fboxsep}}16
\end{sphinxVerbatim}



\end{sphinxuseclass}
\end{sphinxuseclass}
}

\end{sphinxuseclass}
\end{sphinxuseclass}
\sphinxAtStartPar
Percent of crashes:

\begin{sphinxuseclass}{nbinput}
{
\sphinxsetup{VerbatimColor={named}{nbsphinx-code-bg}}
\sphinxsetup{VerbatimBorderColor={named}{nbsphinx-code-border}}
\begin{sphinxVerbatim}[commandchars=\\\{\}]
\llap{\color{nbsphinxin}[52]:\,\hspace{\fboxrule}\hspace{\fboxsep}}\PYG{n}{oct\PYGZus{}perc\PYGZus{}crashes} \PYG{o}{=} \PYG{n+nb}{len}\PYG{p}{(}\PYG{n}{oct\PYGZus{}crashes}\PYG{p}{[}\PYG{l+s+s1}{\PYGZsq{}}\PYG{l+s+s1}{rate}\PYG{l+s+s1}{\PYGZsq{}}\PYG{p}{]}\PYG{p}{)}\PYG{o}{/}\PYG{n+nb}{len}\PYG{p}{(}\PYG{n}{oct\PYGZus{}tab}\PYG{p}{[}\PYG{l+s+s1}{\PYGZsq{}}\PYG{l+s+s1}{rate}\PYG{l+s+s1}{\PYGZsq{}}\PYG{p}{]}\PYG{p}{)}
\PYG{n}{oct\PYGZus{}perc\PYGZus{}crashes}
\end{sphinxVerbatim}
}

\end{sphinxuseclass}
\begin{sphinxuseclass}{nboutput}
\begin{sphinxuseclass}{nblast}
{

\kern-\sphinxverbatimsmallskipamount\kern-\baselineskip
\kern+\FrameHeightAdjust\kern-\fboxrule
\vspace{\nbsphinxcodecellspacing}

\sphinxsetup{VerbatimColor={named}{white}}
\sphinxsetup{VerbatimBorderColor={named}{nbsphinx-code-border}}
\begin{sphinxuseclass}{output_area}
\begin{sphinxuseclass}{}


\begin{sphinxVerbatim}[commandchars=\\\{\}]
\llap{\color{nbsphinxout}[52]:\,\hspace{\fboxrule}\hspace{\fboxsep}}0.07655502392344497
\end{sphinxVerbatim}



\end{sphinxuseclass}
\end{sphinxuseclass}
}

\end{sphinxuseclass}
\end{sphinxuseclass}
\begin{sphinxuseclass}{nbinput}
{
\sphinxsetup{VerbatimColor={named}{nbsphinx-code-bg}}
\sphinxsetup{VerbatimBorderColor={named}{nbsphinx-code-border}}
\begin{sphinxVerbatim}[commandchars=\\\{\}]
\llap{\color{nbsphinxin}[53]:\,\hspace{\fboxrule}\hspace{\fboxsep}}\PYG{n}{oct\PYGZus{}crashes}\PYG{p}{[}\PYG{l+m+mi}{35}\PYG{p}{:}\PYG{p}{]}
\end{sphinxVerbatim}
}

\end{sphinxuseclass}
\begin{sphinxuseclass}{nboutput}
\begin{sphinxuseclass}{nblast}
{

\kern-\sphinxverbatimsmallskipamount\kern-\baselineskip
\kern+\FrameHeightAdjust\kern-\fboxrule
\vspace{\nbsphinxcodecellspacing}

\sphinxsetup{VerbatimColor={named}{white}}
\sphinxsetup{VerbatimBorderColor={named}{nbsphinx-code-border}}
\begin{sphinxuseclass}{output_area}
\begin{sphinxuseclass}{}


\begin{sphinxVerbatim}[commandchars=\\\{\}]
\llap{\color{nbsphinxout}[53]:\,\hspace{\fboxrule}\hspace{\fboxsep}}Empty DataFrame
Columns: [rate, cost, expected cost]
Index: []
\end{sphinxVerbatim}



\end{sphinxuseclass}
\end{sphinxuseclass}
}

\end{sphinxuseclass}
\end{sphinxuseclass}
\begin{sphinxuseclass}{nbinput}
\begin{sphinxuseclass}{nblast}
{
\sphinxsetup{VerbatimColor={named}{nbsphinx-code-bg}}
\sphinxsetup{VerbatimBorderColor={named}{nbsphinx-code-border}}
\begin{sphinxVerbatim}[commandchars=\\\{\}]
\llap{\color{nbsphinxin}[ ]:\,\hspace{\fboxrule}\hspace{\fboxsep}}
\end{sphinxVerbatim}
}

\end{sphinxuseclass}
\end{sphinxuseclass}

\subsection{Pump Example Notebook}
\label{\detokenize{example_pump/Pump_Example_Notebook:Pump-Example-Notebook}}\label{\detokenize{example_pump/Pump_Example_Notebook::doc}}
\sphinxAtStartPar
This script shows basic I/O operations that can be performed with this toolkit, as well as some of the basic model and simulation visualization and analysis features.

\sphinxAtStartPar
This script runs these basic operations on the simple model defined in ex\_pump.py.

\begin{sphinxuseclass}{nbinput}
\begin{sphinxuseclass}{nblast}
{
\sphinxsetup{VerbatimColor={named}{nbsphinx-code-bg}}
\sphinxsetup{VerbatimBorderColor={named}{nbsphinx-code-border}}
\begin{sphinxVerbatim}[commandchars=\\\{\}]
\llap{\color{nbsphinxin}[1]:\,\hspace{\fboxrule}\hspace{\fboxsep}}\PYG{c+c1}{\PYGZsh{}First, import the fault propogation library as well as the model}
\PYG{c+c1}{\PYGZsh{}since the package is in a parallel location to examples...}
\PYG{k+kn}{import} \PYG{n+nn}{sys}\PYG{o}{,} \PYG{n+nn}{os}
\PYG{n}{sys}\PYG{o}{.}\PYG{n}{path}\PYG{o}{.}\PYG{n}{insert}\PYG{p}{(}\PYG{l+m+mi}{1}\PYG{p}{,}\PYG{n}{os}\PYG{o}{.}\PYG{n}{path}\PYG{o}{.}\PYG{n}{join}\PYG{p}{(}\PYG{l+s+s2}{\PYGZdq{}}\PYG{l+s+s2}{..}\PYG{l+s+s2}{\PYGZdq{}}\PYG{p}{)}\PYG{p}{)}

\PYG{k+kn}{import} \PYG{n+nn}{fmdtools}\PYG{n+nn}{.}\PYG{n+nn}{faultsim}\PYG{n+nn}{.}\PYG{n+nn}{propagate} \PYG{k}{as} \PYG{n+nn}{propagate}
\PYG{k+kn}{import} \PYG{n+nn}{fmdtools}\PYG{n+nn}{.}\PYG{n+nn}{resultdisp} \PYG{k}{as} \PYG{n+nn}{rd}
\PYG{k+kn}{from} \PYG{n+nn}{ex\PYGZus{}pump} \PYG{k+kn}{import} \PYG{o}{*}
\PYG{k+kn}{from} \PYG{n+nn}{IPython}\PYG{n+nn}{.}\PYG{n+nn}{display} \PYG{k+kn}{import} \PYG{n}{HTML}
\PYG{n}{mdl} \PYG{o}{=} \PYG{n}{Pump}\PYG{p}{(}\PYG{p}{)}
\end{sphinxVerbatim}
}

\end{sphinxuseclass}
\end{sphinxuseclass}

\subsubsection{Initial Model Checks}
\label{\detokenize{example_pump/Pump_Example_Notebook:Initial-Model-Checks}}
\sphinxAtStartPar
Before seeing how faults propogate, it’s useful to see how the model performs in the nominal state to check to see that the model has been defined correctly. Some things worth checking: \sphinxhyphen{} are all functions on the graph? \sphinxhyphen{} are the functions connected with the correct flows? \sphinxhyphen{} do any faults occur in the nominal state? \sphinxhyphen{} do all the flow states proceed as desired over time?

\sphinxAtStartPar
The following code runs the model with no faults to let us do that. The inputs are: \sphinxhyphen{} mdl (the model we imported at the start of the script) \sphinxhyphen{} track (which model aspects to track)

\sphinxAtStartPar
The outputs are: \sphinxhyphen{} endresults (a dictionary of the degraded flows, resulting faults, and fault classification at final t) \sphinxhyphen{} resgraph (the results superimposed on the graph at final t) \sphinxhyphen{} mdlhist (the states of the functions and flows over time)

\begin{sphinxuseclass}{nbinput}
\begin{sphinxuseclass}{nblast}
{
\sphinxsetup{VerbatimColor={named}{nbsphinx-code-bg}}
\sphinxsetup{VerbatimBorderColor={named}{nbsphinx-code-border}}
\begin{sphinxVerbatim}[commandchars=\\\{\}]
\llap{\color{nbsphinxin}[2]:\,\hspace{\fboxrule}\hspace{\fboxsep}}\PYG{n}{endresults}\PYG{p}{,} \PYG{n}{resgraph}\PYG{p}{,} \PYG{n}{mdlhist}\PYG{o}{=}\PYG{n}{propagate}\PYG{o}{.}\PYG{n}{nominal}\PYG{p}{(}\PYG{n}{mdl}\PYG{p}{,} \PYG{n}{track}\PYG{o}{=}\PYG{l+s+s1}{\PYGZsq{}}\PYG{l+s+s1}{all}\PYG{l+s+s1}{\PYGZsq{}}\PYG{p}{)}
\end{sphinxVerbatim}
}

\end{sphinxuseclass}
\end{sphinxuseclass}
\sphinxAtStartPar
With these results, we can now plot the graph of results resgraph using:

\begin{sphinxuseclass}{nbinput}
{
\sphinxsetup{VerbatimColor={named}{nbsphinx-code-bg}}
\sphinxsetup{VerbatimBorderColor={named}{nbsphinx-code-border}}
\begin{sphinxVerbatim}[commandchars=\\\{\}]
\llap{\color{nbsphinxin}[3]:\,\hspace{\fboxrule}\hspace{\fboxsep}}\PYG{n}{rd}\PYG{o}{.}\PYG{n}{graph}\PYG{o}{.}\PYG{n}{show}\PYG{p}{(}\PYG{n}{resgraph}\PYG{p}{)}
\PYG{n}{fig} \PYG{o}{=} \PYG{n}{rd}\PYG{o}{.}\PYG{n}{graph}\PYG{o}{.}\PYG{n}{show}\PYG{p}{(}\PYG{n}{resgraph}\PYG{p}{,} \PYG{n}{renderer}\PYG{o}{=}\PYG{l+s+s1}{\PYGZsq{}}\PYG{l+s+s1}{graphviz}\PYG{l+s+s1}{\PYGZsq{}}\PYG{p}{)}
\PYG{n}{fig} \PYG{o}{=} \PYG{n}{rd}\PYG{o}{.}\PYG{n}{graph}\PYG{o}{.}\PYG{n}{show}\PYG{p}{(}\PYG{n}{resgraph}\PYG{p}{,} \PYG{n}{renderer}\PYG{o}{=}\PYG{l+s+s1}{\PYGZsq{}}\PYG{l+s+s1}{netgraph}\PYG{l+s+s1}{\PYGZsq{}}\PYG{p}{)}
\end{sphinxVerbatim}
}

\end{sphinxuseclass}
\begin{sphinxuseclass}{nboutput}
\hrule height -\fboxrule\relax
\vspace{\nbsphinxcodecellspacing}

\makeatletter\setbox\nbsphinxpromptbox\box\voidb@x\makeatother

\begin{nbsphinxfancyoutput}

\begin{sphinxuseclass}{output_area}
\begin{sphinxuseclass}{}
\noindent\sphinxincludegraphics{{example_pump_Pump_Example_Notebook_5_0}.svg}

\end{sphinxuseclass}
\end{sphinxuseclass}
\end{nbsphinxfancyoutput}

\end{sphinxuseclass}
\begin{sphinxuseclass}{nboutput}
\hrule height -\fboxrule\relax
\vspace{\nbsphinxcodecellspacing}

\makeatletter\setbox\nbsphinxpromptbox\box\voidb@x\makeatother

\begin{nbsphinxfancyoutput}

\begin{sphinxuseclass}{output_area}
\begin{sphinxuseclass}{}
\noindent\sphinxincludegraphics[width=349\sphinxpxdimen,height=231\sphinxpxdimen]{{example_pump_Pump_Example_Notebook_5_1}.png}

\end{sphinxuseclass}
\end{sphinxuseclass}
\end{nbsphinxfancyoutput}

\end{sphinxuseclass}
\begin{sphinxuseclass}{nboutput}
\begin{sphinxuseclass}{nblast}
\hrule height -\fboxrule\relax
\vspace{\nbsphinxcodecellspacing}

\makeatletter\setbox\nbsphinxpromptbox\box\voidb@x\makeatother

\begin{nbsphinxfancyoutput}

\begin{sphinxuseclass}{output_area}
\begin{sphinxuseclass}{}
\noindent\sphinxincludegraphics[width=255\sphinxpxdimen,height=231\sphinxpxdimen]{{example_pump_Pump_Example_Notebook_5_2}.png}

\end{sphinxuseclass}
\end{sphinxuseclass}
\end{nbsphinxfancyoutput}

\end{sphinxuseclass}
\end{sphinxuseclass}
\sphinxAtStartPar
As can be seen, this gives a graphical representation of the functional model with the various flows. Since all of the functions are \sphinxstyleemphasis{green}, no faults were accidentally introduced in this run.

\sphinxAtStartPar
For more complicated models (where any flow connects to more than two functions), it can be helpful to plot the model with a bipartite representation:

\begin{sphinxuseclass}{nbinput}
{
\sphinxsetup{VerbatimColor={named}{nbsphinx-code-bg}}
\sphinxsetup{VerbatimBorderColor={named}{nbsphinx-code-border}}
\begin{sphinxVerbatim}[commandchars=\\\{\}]
\llap{\color{nbsphinxin}[4]:\,\hspace{\fboxrule}\hspace{\fboxsep}}\PYG{n}{rd}\PYG{o}{.}\PYG{n}{graph}\PYG{o}{.}\PYG{n}{show}\PYG{p}{(}\PYG{n}{mdl}\PYG{p}{,} \PYG{n}{gtype}\PYG{o}{=}\PYG{l+s+s1}{\PYGZsq{}}\PYG{l+s+s1}{bipartite}\PYG{l+s+s1}{\PYGZsq{}}\PYG{p}{)}
\PYG{n}{fig}\PYG{o}{=}\PYG{n}{rd}\PYG{o}{.}\PYG{n}{graph}\PYG{o}{.}\PYG{n}{show}\PYG{p}{(}\PYG{n}{mdl}\PYG{p}{,} \PYG{n}{gtype}\PYG{o}{=}\PYG{l+s+s1}{\PYGZsq{}}\PYG{l+s+s1}{bipartite}\PYG{l+s+s1}{\PYGZsq{}}\PYG{p}{,} \PYG{n}{renderer}\PYG{o}{=}\PYG{l+s+s1}{\PYGZsq{}}\PYG{l+s+s1}{graphviz}\PYG{l+s+s1}{\PYGZsq{}}\PYG{p}{)}
\end{sphinxVerbatim}
}

\end{sphinxuseclass}
\begin{sphinxuseclass}{nboutput}
\hrule height -\fboxrule\relax
\vspace{\nbsphinxcodecellspacing}

\makeatletter\setbox\nbsphinxpromptbox\box\voidb@x\makeatother

\begin{nbsphinxfancyoutput}

\begin{sphinxuseclass}{output_area}
\begin{sphinxuseclass}{}
\noindent\sphinxincludegraphics{{example_pump_Pump_Example_Notebook_8_0}.svg}

\end{sphinxuseclass}
\end{sphinxuseclass}
\end{nbsphinxfancyoutput}

\end{sphinxuseclass}
\begin{sphinxuseclass}{nboutput}
\begin{sphinxuseclass}{nblast}
\hrule height -\fboxrule\relax
\vspace{\nbsphinxcodecellspacing}

\makeatletter\setbox\nbsphinxpromptbox\box\voidb@x\makeatother

\begin{nbsphinxfancyoutput}

\begin{sphinxuseclass}{output_area}
\begin{sphinxuseclass}{}
\noindent\sphinxincludegraphics[width=349\sphinxpxdimen,height=231\sphinxpxdimen]{{example_pump_Pump_Example_Notebook_8_1}.png}

\end{sphinxuseclass}
\end{sphinxuseclass}
\end{nbsphinxfancyoutput}

\end{sphinxuseclass}
\end{sphinxuseclass}
\sphinxAtStartPar
Or, if the structure becomes very complex and more than one object instantiates each function/flow, it can be helpful to plot the type/class relationships with a directed typegraph representation:

\begin{sphinxuseclass}{nbinput}
{
\sphinxsetup{VerbatimColor={named}{nbsphinx-code-bg}}
\sphinxsetup{VerbatimBorderColor={named}{nbsphinx-code-border}}
\begin{sphinxVerbatim}[commandchars=\\\{\}]
\llap{\color{nbsphinxin}[5]:\,\hspace{\fboxrule}\hspace{\fboxsep}}\PYG{n}{rd}\PYG{o}{.}\PYG{n}{graph}\PYG{o}{.}\PYG{n}{show}\PYG{p}{(}\PYG{n}{mdl}\PYG{p}{,} \PYG{n}{gtype}\PYG{o}{=}\PYG{l+s+s1}{\PYGZsq{}}\PYG{l+s+s1}{typegraph}\PYG{l+s+s1}{\PYGZsq{}}\PYG{p}{)}
\end{sphinxVerbatim}
}

\end{sphinxuseclass}
\begin{sphinxuseclass}{nboutput}
{

\kern-\sphinxverbatimsmallskipamount\kern-\baselineskip
\kern+\FrameHeightAdjust\kern-\fboxrule
\vspace{\nbsphinxcodecellspacing}

\sphinxsetup{VerbatimColor={named}{white}}
\sphinxsetup{VerbatimBorderColor={named}{nbsphinx-code-border}}
\begin{sphinxuseclass}{output_area}
\begin{sphinxuseclass}{}


\begin{sphinxVerbatim}[commandchars=\\\{\}]
\llap{\color{nbsphinxout}[5]:\,\hspace{\fboxrule}\hspace{\fboxsep}}(<Figure size 432x288 with 1 Axes>, <Axes:>)
\end{sphinxVerbatim}



\end{sphinxuseclass}
\end{sphinxuseclass}
}

\end{sphinxuseclass}
\begin{sphinxuseclass}{nboutput}
\begin{sphinxuseclass}{nblast}
\hrule height -\fboxrule\relax
\vspace{\nbsphinxcodecellspacing}

\makeatletter\setbox\nbsphinxpromptbox\box\voidb@x\makeatother

\begin{nbsphinxfancyoutput}

\begin{sphinxuseclass}{output_area}
\begin{sphinxuseclass}{}
\noindent\sphinxincludegraphics[width=446\sphinxpxdimen,height=302\sphinxpxdimen]{{example_pump_Pump_Example_Notebook_10_1}.png}

\end{sphinxuseclass}
\end{sphinxuseclass}
\end{nbsphinxfancyoutput}

\end{sphinxuseclass}
\end{sphinxuseclass}
\sphinxAtStartPar
We can also render these graphs using graphviz:

\begin{sphinxuseclass}{nbinput}
{
\sphinxsetup{VerbatimColor={named}{nbsphinx-code-bg}}
\sphinxsetup{VerbatimBorderColor={named}{nbsphinx-code-border}}
\begin{sphinxVerbatim}[commandchars=\\\{\}]
\llap{\color{nbsphinxin}[6]:\,\hspace{\fboxrule}\hspace{\fboxsep}}\PYG{n}{dot} \PYG{o}{=} \PYG{n}{rd}\PYG{o}{.}\PYG{n}{graph}\PYG{o}{.}\PYG{n}{show}\PYG{p}{(}\PYG{n}{mdl}\PYG{p}{,} \PYG{n}{renderer}\PYG{o}{=}\PYG{l+s+s1}{\PYGZsq{}}\PYG{l+s+s1}{graphviz}\PYG{l+s+s1}{\PYGZsq{}}\PYG{p}{)}
\end{sphinxVerbatim}
}

\end{sphinxuseclass}
\begin{sphinxuseclass}{nboutput}
\begin{sphinxuseclass}{nblast}
\hrule height -\fboxrule\relax
\vspace{\nbsphinxcodecellspacing}

\makeatletter\setbox\nbsphinxpromptbox\box\voidb@x\makeatother

\begin{nbsphinxfancyoutput}

\begin{sphinxuseclass}{output_area}
\begin{sphinxuseclass}{}
\noindent\sphinxincludegraphics{{example_pump_Pump_Example_Notebook_12_0}.svg}

\end{sphinxuseclass}
\end{sphinxuseclass}
\end{nbsphinxfancyoutput}

\end{sphinxuseclass}
\end{sphinxuseclass}
\sphinxAtStartPar
We can further look at the state of the model using:

\begin{sphinxuseclass}{nbinput}
{
\sphinxsetup{VerbatimColor={named}{nbsphinx-code-bg}}
\sphinxsetup{VerbatimBorderColor={named}{nbsphinx-code-border}}
\begin{sphinxVerbatim}[commandchars=\\\{\}]
\llap{\color{nbsphinxin}[7]:\,\hspace{\fboxrule}\hspace{\fboxsep}}\PYG{n}{rd}\PYG{o}{.}\PYG{n}{plot}\PYG{o}{.}\PYG{n}{mdlhist}\PYG{p}{(}\PYG{n}{mdlhist}\PYG{p}{,} \PYG{l+s+s1}{\PYGZsq{}}\PYG{l+s+s1}{Nominal}\PYG{l+s+s1}{\PYGZsq{}}\PYG{p}{)}
\end{sphinxVerbatim}
}

\end{sphinxuseclass}
\begin{sphinxuseclass}{nboutput}
{

\kern-\sphinxverbatimsmallskipamount\kern-\baselineskip
\kern+\FrameHeightAdjust\kern-\fboxrule
\vspace{\nbsphinxcodecellspacing}

\sphinxsetup{VerbatimColor={named}{white}}
\sphinxsetup{VerbatimBorderColor={named}{nbsphinx-code-border}}
\begin{sphinxuseclass}{output_area}
\begin{sphinxuseclass}{}


\begin{sphinxVerbatim}[commandchars=\\\{\}]
<Figure size 432x144 with 0 Axes>
\end{sphinxVerbatim}



\end{sphinxuseclass}
\end{sphinxuseclass}
}

\end{sphinxuseclass}
\begin{sphinxuseclass}{nboutput}
{

\kern-\sphinxverbatimsmallskipamount\kern-\baselineskip
\kern+\FrameHeightAdjust\kern-\fboxrule
\vspace{\nbsphinxcodecellspacing}

\sphinxsetup{VerbatimColor={named}{white}}
\sphinxsetup{VerbatimBorderColor={named}{nbsphinx-code-border}}
\begin{sphinxuseclass}{output_area}
\begin{sphinxuseclass}{}


\begin{sphinxVerbatim}[commandchars=\\\{\}]
<Figure size 432x144 with 0 Axes>
\end{sphinxVerbatim}



\end{sphinxuseclass}
\end{sphinxuseclass}
}

\end{sphinxuseclass}
\begin{sphinxuseclass}{nboutput}
{

\kern-\sphinxverbatimsmallskipamount\kern-\baselineskip
\kern+\FrameHeightAdjust\kern-\fboxrule
\vspace{\nbsphinxcodecellspacing}

\sphinxsetup{VerbatimColor={named}{white}}
\sphinxsetup{VerbatimBorderColor={named}{nbsphinx-code-border}}
\begin{sphinxuseclass}{output_area}
\begin{sphinxuseclass}{}


\begin{sphinxVerbatim}[commandchars=\\\{\}]
<Figure size 432x144 with 0 Axes>
\end{sphinxVerbatim}



\end{sphinxuseclass}
\end{sphinxuseclass}
}

\end{sphinxuseclass}
\begin{sphinxuseclass}{nboutput}
\hrule height -\fboxrule\relax
\vspace{\nbsphinxcodecellspacing}

\makeatletter\setbox\nbsphinxpromptbox\box\voidb@x\makeatother

\begin{nbsphinxfancyoutput}

\begin{sphinxuseclass}{output_area}
\begin{sphinxuseclass}{}
\noindent\sphinxincludegraphics[width=230\sphinxpxdimen,height=128\sphinxpxdimen]{{example_pump_Pump_Example_Notebook_14_3}.png}

\end{sphinxuseclass}
\end{sphinxuseclass}
\end{nbsphinxfancyoutput}

\end{sphinxuseclass}
\begin{sphinxuseclass}{nboutput}
{

\kern-\sphinxverbatimsmallskipamount\kern-\baselineskip
\kern+\FrameHeightAdjust\kern-\fboxrule
\vspace{\nbsphinxcodecellspacing}

\sphinxsetup{VerbatimColor={named}{white}}
\sphinxsetup{VerbatimBorderColor={named}{nbsphinx-code-border}}
\begin{sphinxuseclass}{output_area}
\begin{sphinxuseclass}{}


\begin{sphinxVerbatim}[commandchars=\\\{\}]
<Figure size 432x144 with 0 Axes>
\end{sphinxVerbatim}



\end{sphinxuseclass}
\end{sphinxuseclass}
}

\end{sphinxuseclass}
\begin{sphinxuseclass}{nboutput}
\hrule height -\fboxrule\relax
\vspace{\nbsphinxcodecellspacing}

\makeatletter\setbox\nbsphinxpromptbox\box\voidb@x\makeatother

\begin{nbsphinxfancyoutput}

\begin{sphinxuseclass}{output_area}
\begin{sphinxuseclass}{}
\noindent\sphinxincludegraphics[width=426\sphinxpxdimen,height=128\sphinxpxdimen]{{example_pump_Pump_Example_Notebook_14_5}.png}

\end{sphinxuseclass}
\end{sphinxuseclass}
\end{nbsphinxfancyoutput}

\end{sphinxuseclass}
\begin{sphinxuseclass}{nboutput}
\hrule height -\fboxrule\relax
\vspace{\nbsphinxcodecellspacing}

\makeatletter\setbox\nbsphinxpromptbox\box\voidb@x\makeatother

\begin{nbsphinxfancyoutput}

\begin{sphinxuseclass}{output_area}
\begin{sphinxuseclass}{}
\noindent\sphinxincludegraphics[width=223\sphinxpxdimen,height=135\sphinxpxdimen]{{example_pump_Pump_Example_Notebook_14_6}.png}

\end{sphinxuseclass}
\end{sphinxuseclass}
\end{nbsphinxfancyoutput}

\end{sphinxuseclass}
\begin{sphinxuseclass}{nboutput}
\hrule height -\fboxrule\relax
\vspace{\nbsphinxcodecellspacing}

\makeatletter\setbox\nbsphinxpromptbox\box\voidb@x\makeatother

\begin{nbsphinxfancyoutput}

\begin{sphinxuseclass}{output_area}
\begin{sphinxuseclass}{}
\noindent\sphinxincludegraphics[width=426\sphinxpxdimen,height=250\sphinxpxdimen]{{example_pump_Pump_Example_Notebook_14_7}.png}

\end{sphinxuseclass}
\end{sphinxuseclass}
\end{nbsphinxfancyoutput}

\end{sphinxuseclass}
\begin{sphinxuseclass}{nboutput}
\begin{sphinxuseclass}{nblast}
\hrule height -\fboxrule\relax
\vspace{\nbsphinxcodecellspacing}

\makeatletter\setbox\nbsphinxpromptbox\box\voidb@x\makeatother

\begin{nbsphinxfancyoutput}

\begin{sphinxuseclass}{output_area}
\begin{sphinxuseclass}{}
\noindent\sphinxincludegraphics[width=426\sphinxpxdimen,height=250\sphinxpxdimen]{{example_pump_Pump_Example_Notebook_14_8}.png}

\end{sphinxuseclass}
\end{sphinxuseclass}
\end{nbsphinxfancyoutput}

\end{sphinxuseclass}
\end{sphinxuseclass}
\sphinxAtStartPar
As we can see, the state of these flows does exactly what we would expect–when the switch turns on at \(t=5\), the pump switches on and there is a flow of water in and out of the model.


\subsubsection{Tables}
\label{\detokenize{example_pump/Pump_Example_Notebook:Tables}}
\sphinxAtStartPar
If we want to see this data in tabular form, we can use \sphinxcode{\sphinxupquote{fp.tabulate.hist()}}:

\begin{sphinxuseclass}{nbinput}
{
\sphinxsetup{VerbatimColor={named}{nbsphinx-code-bg}}
\sphinxsetup{VerbatimBorderColor={named}{nbsphinx-code-border}}
\begin{sphinxVerbatim}[commandchars=\\\{\}]
\llap{\color{nbsphinxin}[8]:\,\hspace{\fboxrule}\hspace{\fboxsep}}\PYG{n}{nominal\PYGZus{}histtable} \PYG{o}{=} \PYG{n}{rd}\PYG{o}{.}\PYG{n}{tabulate}\PYG{o}{.}\PYG{n}{hist}\PYG{p}{(}\PYG{n}{mdlhist}\PYG{p}{)}
\PYG{n}{nominal\PYGZus{}histtable}\PYG{p}{[}\PYG{p}{:}\PYG{l+m+mi}{10}\PYG{p}{]} \PYG{c+c1}{\PYGZsh{}only displaying 10}
\end{sphinxVerbatim}
}

\end{sphinxuseclass}
\begin{sphinxuseclass}{nboutput}
\begin{sphinxuseclass}{nblast}
{

\kern-\sphinxverbatimsmallskipamount\kern-\baselineskip
\kern+\FrameHeightAdjust\kern-\fboxrule
\vspace{\nbsphinxcodecellspacing}

\sphinxsetup{VerbatimColor={named}{white}}
\sphinxsetup{VerbatimBorderColor={named}{nbsphinx-code-border}}
\begin{sphinxuseclass}{output_area}
\begin{sphinxuseclass}{}


\begin{sphinxVerbatim}[commandchars=\\\{\}]
\llap{\color{nbsphinxout}[8]:\,\hspace{\fboxrule}\hspace{\fboxsep}}  time   ImportEE              ImportWater ImportSignal MoveWater  \textbackslash{}
     t no\_v fault inf\_v fault no\_wat fault no\_sig fault       eff
0    0          0           0            0            0       1.0
1    1          0           0            0            0       1.0
2    2          0           0            0            0       1.0
3    3          0           0            0            0       1.0
4    4          0           0            0            0       1.0
5    5          0           0            0            0       1.0
6    6          0           0            0            0       1.0
7    7          0           0            0            0       1.0
8    8          0           0            0            0       1.0
9    9          0           0            0            0       1.0

                               ExportWater    EE\_1         Sig\_1    Wat\_1  \textbackslash{}
  mech\_break fault short fault block fault current voltage power flowrate
0                0           0           0     0.0   500.0   0.0      0.0
1                0           0           0     0.0   500.0   0.0      0.0
2                0           0           0     0.0   500.0   0.0      0.0
3                0           0           0     0.0   500.0   0.0      0.0
4                0           0           0     0.0   500.0   0.0      0.0
5                0           0           0    10.0   500.0   1.0      0.3
6                0           0           0    10.0   500.0   1.0      0.3
7                0           0           0    10.0   500.0   1.0      0.3
8                0           0           0    10.0   500.0   1.0      0.3
9                0           0           0    10.0   500.0   1.0      0.3

                         Wat\_2
  pressure area level flowrate pressure area level
0      0.0  1.0   1.0      0.0      0.0  1.0   1.0
1      0.0  1.0   1.0      0.0      0.0  1.0   1.0
2      0.0  1.0   1.0      0.0      0.0  1.0   1.0
3      0.0  1.0   1.0      0.0      0.0  1.0   1.0
4      0.0  1.0   1.0      0.0      0.0  1.0   1.0
5     10.0  1.0   1.0      0.3     10.0  1.0   1.0
6     10.0  1.0   1.0      0.3     10.0  1.0   1.0
7     10.0  1.0   1.0      0.3     10.0  1.0   1.0
8     10.0  1.0   1.0      0.3     10.0  1.0   1.0
9     10.0  1.0   1.0      0.3     10.0  1.0   1.0
\end{sphinxVerbatim}



\end{sphinxuseclass}
\end{sphinxuseclass}
}

\end{sphinxuseclass}
\end{sphinxuseclass}
\sphinxAtStartPar
This table is a pandas dataframe. We can save this dataframe to a .csv using \sphinxcode{\sphinxupquote{nominal\_histtable.to\_csv("filename.csv")}}


\subsubsection{Propagating and Viewing Results for Individual Faults}
\label{\detokenize{example_pump/Pump_Example_Notebook:Propagating-and-Viewing-Results-for-Individual-Faults}}
\sphinxAtStartPar
It is often necessary to see how the system reacts to individual faults. This can gives us better understanding of how the system behaves under individual faults and can let us iterate with the model better.

\sphinxAtStartPar
The following code runs the model with a single fault in a single function. In this case, we are initiating a short in the ‘Move Water’ function at 10 hours into the system’s operation.

\sphinxAtStartPar
The inputs are: \sphinxhyphen{} mdl (the model we imported at the start of the script) \sphinxhyphen{} Function (the function the fault we’re interested in propagating occurs in) \sphinxhyphen{} faultmode (the fault to initiate) \sphinxhyphen{} time (the time when the fault is initiated) \sphinxhyphen{} track (whether or not we want to track flows)

\sphinxAtStartPar
The outputs are (the same as propogate.nominal): \sphinxhyphen{} endresults (a dictionary of the degraged flows, resulting faults, and fault classification at final t) \sphinxhyphen{} resgraph (the results superimposed on the graph at final t) \sphinxhyphen{} mdlhist (the states of the model over time)

\begin{sphinxuseclass}{nbinput}
\begin{sphinxuseclass}{nblast}
{
\sphinxsetup{VerbatimColor={named}{nbsphinx-code-bg}}
\sphinxsetup{VerbatimBorderColor={named}{nbsphinx-code-border}}
\begin{sphinxVerbatim}[commandchars=\\\{\}]
\llap{\color{nbsphinxin}[9]:\,\hspace{\fboxrule}\hspace{\fboxsep}}\PYG{n}{endresults}\PYG{p}{,} \PYG{n}{resgraph}\PYG{p}{,} \PYG{n}{mdlhist}\PYG{o}{=}\PYG{n}{propagate}\PYG{o}{.}\PYG{n}{one\PYGZus{}fault}\PYG{p}{(}\PYG{n}{mdl}\PYG{p}{,} \PYG{l+s+s1}{\PYGZsq{}}\PYG{l+s+s1}{MoveWater}\PYG{l+s+s1}{\PYGZsq{}}\PYG{p}{,} \PYG{l+s+s1}{\PYGZsq{}}\PYG{l+s+s1}{short}\PYG{l+s+s1}{\PYGZsq{}}\PYG{p}{,} \PYG{n}{time}\PYG{o}{=}\PYG{l+m+mi}{10}\PYG{p}{)}
\end{sphinxVerbatim}
}

\end{sphinxuseclass}
\end{sphinxuseclass}
\sphinxAtStartPar
\sphinxcode{\sphinxupquote{rp.process.hist(mdlhist)}} compares the results over time so we can see what functions and flows were degraded over time. We can then use the summary to view a list of the functions and flows that were impacted over time.

\begin{sphinxuseclass}{nbinput}
{
\sphinxsetup{VerbatimColor={named}{nbsphinx-code-bg}}
\sphinxsetup{VerbatimBorderColor={named}{nbsphinx-code-border}}
\begin{sphinxVerbatim}[commandchars=\\\{\}]
\llap{\color{nbsphinxin}[10]:\,\hspace{\fboxrule}\hspace{\fboxsep}}\PYG{n}{reshist}\PYG{p}{,}\PYG{n}{diff}\PYG{p}{,} \PYG{n}{summary} \PYG{o}{=} \PYG{n}{rd}\PYG{o}{.}\PYG{n}{process}\PYG{o}{.}\PYG{n}{hist}\PYG{p}{(}\PYG{n}{mdlhist}\PYG{p}{)}
\PYG{c+c1}{\PYGZsh{}summarytable = fp.makesummarytable(summary)}
\PYG{n}{tab} \PYG{o}{=} \PYG{n}{rd}\PYG{o}{.}\PYG{n}{tabulate}\PYG{o}{.}\PYG{n}{result}\PYG{p}{(}\PYG{n}{endresults}\PYG{p}{,} \PYG{n}{summary}\PYG{p}{)}
\PYG{n}{tab}
\end{sphinxVerbatim}
}

\end{sphinxuseclass}
\begin{sphinxuseclass}{nboutput}
\begin{sphinxuseclass}{nblast}
{

\kern-\sphinxverbatimsmallskipamount\kern-\baselineskip
\kern+\FrameHeightAdjust\kern-\fboxrule
\vspace{\nbsphinxcodecellspacing}

\sphinxsetup{VerbatimColor={named}{white}}
\sphinxsetup{VerbatimBorderColor={named}{nbsphinx-code-border}}
\begin{sphinxuseclass}{output_area}
\begin{sphinxuseclass}{}


\begin{sphinxVerbatim}[commandchars=\\\{\}]
\llap{\color{nbsphinxout}[10]:\,\hspace{\fboxrule}\hspace{\fboxsep}}      rate     cost  expected cost     degraded functions  \textbackslash{}
0  0.00055  29000.0      1595000.0  [ImportEE, MoveWater]

         degraded flows
0  [EE\_1, Wat\_1, Wat\_2]
\end{sphinxVerbatim}



\end{sphinxuseclass}
\end{sphinxuseclass}
}

\end{sphinxuseclass}
\end{sphinxuseclass}
\sphinxAtStartPar
We can also see what happens with the graph view:

\begin{sphinxuseclass}{nbinput}
{
\sphinxsetup{VerbatimColor={named}{nbsphinx-code-bg}}
\sphinxsetup{VerbatimBorderColor={named}{nbsphinx-code-border}}
\begin{sphinxVerbatim}[commandchars=\\\{\}]
\llap{\color{nbsphinxin}[11]:\,\hspace{\fboxrule}\hspace{\fboxsep}}\PYG{n}{rd}\PYG{o}{.}\PYG{n}{graph}\PYG{o}{.}\PYG{n}{show}\PYG{p}{(}\PYG{n}{resgraph}\PYG{p}{)}
\end{sphinxVerbatim}
}

\end{sphinxuseclass}
\begin{sphinxuseclass}{nboutput}
{

\kern-\sphinxverbatimsmallskipamount\kern-\baselineskip
\kern+\FrameHeightAdjust\kern-\fboxrule
\vspace{\nbsphinxcodecellspacing}

\sphinxsetup{VerbatimColor={named}{white}}
\sphinxsetup{VerbatimBorderColor={named}{nbsphinx-code-border}}
\begin{sphinxuseclass}{output_area}
\begin{sphinxuseclass}{}


\begin{sphinxVerbatim}[commandchars=\\\{\}]
\llap{\color{nbsphinxout}[11]:\,\hspace{\fboxrule}\hspace{\fboxsep}}(<Figure size 432x288 with 1 Axes>, <AxesSubplot:>)
\end{sphinxVerbatim}



\end{sphinxuseclass}
\end{sphinxuseclass}
}

\end{sphinxuseclass}
\begin{sphinxuseclass}{nboutput}
\begin{sphinxuseclass}{nblast}
\hrule height -\fboxrule\relax
\vspace{\nbsphinxcodecellspacing}

\makeatletter\setbox\nbsphinxpromptbox\box\voidb@x\makeatother

\begin{nbsphinxfancyoutput}

\begin{sphinxuseclass}{output_area}
\begin{sphinxuseclass}{}
\noindent\sphinxincludegraphics[width=349\sphinxpxdimen,height=231\sphinxpxdimen]{{example_pump_Pump_Example_Notebook_23_1}.png}

\end{sphinxuseclass}
\end{sphinxuseclass}
\end{nbsphinxfancyoutput}

\end{sphinxuseclass}
\end{sphinxuseclass}
\begin{sphinxuseclass}{nbinput}
\begin{sphinxuseclass}{nblast}
{
\sphinxsetup{VerbatimColor={named}{nbsphinx-code-bg}}
\sphinxsetup{VerbatimBorderColor={named}{nbsphinx-code-border}}
\begin{sphinxVerbatim}[commandchars=\\\{\}]
\llap{\color{nbsphinxin}[12]:\,\hspace{\fboxrule}\hspace{\fboxsep}}\PYG{n}{\PYGZus{}} \PYG{o}{=} \PYG{n}{rd}\PYG{o}{.}\PYG{n}{graph}\PYG{o}{.}\PYG{n}{show}\PYG{p}{(}\PYG{n}{resgraph}\PYG{p}{,} \PYG{n}{renderer}\PYG{o}{=}\PYG{l+s+s1}{\PYGZsq{}}\PYG{l+s+s1}{graphviz}\PYG{l+s+s1}{\PYGZsq{}}\PYG{p}{,} \PYG{n}{filename}\PYG{o}{=}\PYG{l+s+s1}{\PYGZsq{}}\PYG{l+s+s1}{fault}\PYG{l+s+s1}{\PYGZsq{}}\PYG{p}{)}
\end{sphinxVerbatim}
}

\end{sphinxuseclass}
\end{sphinxuseclass}
\sphinxAtStartPar
As can be seen, at the final t, the short causes a degraded flow of electricity as well as a fault in the Import EE function.

\sphinxAtStartPar
However, we would imagine that the short would cause the water to stop moving also–so why is it green?

\sphinxAtStartPar
The answer is that the results graph gives the values of the variables at the final time, which is the same both for the failed model and the nominal model, since the pump is switched “off.” In this case we might be more interested in looking at how the graph looks in operation, rather than at the end. We can do that that by constructing graphs based on the history of the plot. Below we use \sphinxcode{\sphinxupquote{reshist}} to plot the state of the graph at a particular time.

\begin{sphinxuseclass}{nbinput}
{
\sphinxsetup{VerbatimColor={named}{nbsphinx-code-bg}}
\sphinxsetup{VerbatimBorderColor={named}{nbsphinx-code-border}}
\begin{sphinxVerbatim}[commandchars=\\\{\}]
\llap{\color{nbsphinxin}[13]:\,\hspace{\fboxrule}\hspace{\fboxsep}}\PYG{n}{\PYGZus{}} \PYG{o}{=} \PYG{n}{rd}\PYG{o}{.}\PYG{n}{graph}\PYG{o}{.}\PYG{n}{result\PYGZus{}from}\PYG{p}{(}\PYG{n}{mdl}\PYG{p}{,} \PYG{n}{reshist}\PYG{p}{,} \PYG{l+m+mi}{20}\PYG{p}{,} \PYG{n}{gtype} \PYG{o}{=} \PYG{l+s+s1}{\PYGZsq{}}\PYG{l+s+s1}{normal}\PYG{l+s+s1}{\PYGZsq{}}\PYG{p}{)}
\end{sphinxVerbatim}
}

\end{sphinxuseclass}
\begin{sphinxuseclass}{nboutput}
\begin{sphinxuseclass}{nblast}
\hrule height -\fboxrule\relax
\vspace{\nbsphinxcodecellspacing}

\makeatletter\setbox\nbsphinxpromptbox\box\voidb@x\makeatother

\begin{nbsphinxfancyoutput}

\begin{sphinxuseclass}{output_area}
\begin{sphinxuseclass}{}
\noindent\sphinxincludegraphics[width=349\sphinxpxdimen,height=231\sphinxpxdimen]{{example_pump_Pump_Example_Notebook_26_0}.png}

\end{sphinxuseclass}
\end{sphinxuseclass}
\end{nbsphinxfancyoutput}

\end{sphinxuseclass}
\end{sphinxuseclass}
\sphinxAtStartPar
We can view an animation over time using:

\begin{sphinxuseclass}{nbinput}
\begin{sphinxuseclass}{nblast}
{
\sphinxsetup{VerbatimColor={named}{nbsphinx-code-bg}}
\sphinxsetup{VerbatimBorderColor={named}{nbsphinx-code-border}}
\begin{sphinxVerbatim}[commandchars=\\\{\}]
\llap{\color{nbsphinxin}[14]:\,\hspace{\fboxrule}\hspace{\fboxsep}}\PYG{c+c1}{\PYGZsh{}ani = rd.graph.animation\PYGZus{}from(mdl, reshist, \PYGZsq{}all\PYGZsq{}, faultscen=\PYGZsq{}MoveWater Short (10)\PYGZsq{}, gtype=\PYGZsq{}normal\PYGZsq{})}
\PYG{c+c1}{\PYGZsh{}HTML(ani.to\PYGZus{}jshtml())}
\PYG{c+c1}{\PYGZsh{}saving plot (if desired\PYGZhy{}\PYGZhy{}.gif does not seem to work)}
\PYG{c+c1}{\PYGZsh{}ani.save(\PYGZsq{}test.mp4\PYGZsq{})}
\end{sphinxVerbatim}
}

\end{sphinxuseclass}
\end{sphinxuseclass}
\sphinxAtStartPar
Bipartite representations of the graphs can also be made, see:

\begin{sphinxuseclass}{nbinput}
{
\sphinxsetup{VerbatimColor={named}{nbsphinx-code-bg}}
\sphinxsetup{VerbatimBorderColor={named}{nbsphinx-code-border}}
\begin{sphinxVerbatim}[commandchars=\\\{\}]
\llap{\color{nbsphinxin}[15]:\,\hspace{\fboxrule}\hspace{\fboxsep}}\PYG{n}{pos}\PYG{o}{=}\PYG{n}{nx}\PYG{o}{.}\PYG{n}{spring\PYGZus{}layout}\PYG{p}{(}\PYG{n}{mdl}\PYG{o}{.}\PYG{n}{bipartite}\PYG{p}{)} \PYG{c+c1}{\PYGZsh{}(use this option to keep node locations consistent)}
\PYG{n}{\PYGZus{}} \PYG{o}{=} \PYG{n}{rd}\PYG{o}{.}\PYG{n}{graph}\PYG{o}{.}\PYG{n}{result\PYGZus{}from}\PYG{p}{(}\PYG{n}{mdl}\PYG{p}{,} \PYG{n}{reshist}\PYG{p}{,} \PYG{l+m+mi}{20}\PYG{p}{)}
\end{sphinxVerbatim}
}

\end{sphinxuseclass}
\begin{sphinxuseclass}{nboutput}
\begin{sphinxuseclass}{nblast}
\hrule height -\fboxrule\relax
\vspace{\nbsphinxcodecellspacing}

\makeatletter\setbox\nbsphinxpromptbox\box\voidb@x\makeatother

\begin{nbsphinxfancyoutput}

\begin{sphinxuseclass}{output_area}
\begin{sphinxuseclass}{}
\noindent\sphinxincludegraphics[width=349\sphinxpxdimen,height=231\sphinxpxdimen]{{example_pump_Pump_Example_Notebook_30_0}.png}

\end{sphinxuseclass}
\end{sphinxuseclass}
\end{nbsphinxfancyoutput}

\end{sphinxuseclass}
\end{sphinxuseclass}
\sphinxAtStartPar
We can also plot the states of this against the nominal run using:

\begin{sphinxuseclass}{nbinput}
{
\sphinxsetup{VerbatimColor={named}{nbsphinx-code-bg}}
\sphinxsetup{VerbatimBorderColor={named}{nbsphinx-code-border}}
\begin{sphinxVerbatim}[commandchars=\\\{\}]
\llap{\color{nbsphinxin}[16]:\,\hspace{\fboxrule}\hspace{\fboxsep}}\PYG{n}{rd}\PYG{o}{.}\PYG{n}{plot}\PYG{o}{.}\PYG{n}{mdlhist}\PYG{p}{(}\PYG{n}{mdlhist}\PYG{p}{,} \PYG{l+s+s1}{\PYGZsq{}}\PYG{l+s+s1}{short}\PYG{l+s+s1}{\PYGZsq{}}\PYG{p}{,} \PYG{n}{time}\PYG{o}{=}\PYG{l+m+mi}{10}\PYG{p}{,} \PYG{n}{legend}\PYG{o}{=}\PYG{k+kc}{False}\PYG{p}{)}
\end{sphinxVerbatim}
}

\end{sphinxuseclass}
\begin{sphinxuseclass}{nboutput}
{

\kern-\sphinxverbatimsmallskipamount\kern-\baselineskip
\kern+\FrameHeightAdjust\kern-\fboxrule
\vspace{\nbsphinxcodecellspacing}

\sphinxsetup{VerbatimColor={named}{nbsphinx-stderr}}
\sphinxsetup{VerbatimBorderColor={named}{nbsphinx-code-border}}
\begin{sphinxuseclass}{output_area}
\begin{sphinxuseclass}{stderr}


\begin{sphinxVerbatim}[commandchars=\\\{\}]
C:\textbackslash{}Users\textbackslash{}dhulse\textbackslash{}Documents\textbackslash{}GitHub\textbackslash{}fmdtools\textbackslash{}example\_pump\textbackslash{}..\textbackslash{}fmdtools\textbackslash{}resultdisp\textbackslash{}plot.py:176: UserWarning: Tight layout not applied. The bottom and top margins cannot be made large enough to accommodate all axes decorations.
  plt.tight\_layout(pad=1)
\end{sphinxVerbatim}



\end{sphinxuseclass}
\end{sphinxuseclass}
}

\end{sphinxuseclass}
\begin{sphinxuseclass}{nboutput}
{

\kern-\sphinxverbatimsmallskipamount\kern-\baselineskip
\kern+\FrameHeightAdjust\kern-\fboxrule
\vspace{\nbsphinxcodecellspacing}

\sphinxsetup{VerbatimColor={named}{white}}
\sphinxsetup{VerbatimBorderColor={named}{nbsphinx-code-border}}
\begin{sphinxuseclass}{output_area}
\begin{sphinxuseclass}{}


\begin{sphinxVerbatim}[commandchars=\\\{\}]
<Figure size 432x72 with 0 Axes>
\end{sphinxVerbatim}



\end{sphinxuseclass}
\end{sphinxuseclass}
}

\end{sphinxuseclass}
\begin{sphinxuseclass}{nboutput}
{

\kern-\sphinxverbatimsmallskipamount\kern-\baselineskip
\kern+\FrameHeightAdjust\kern-\fboxrule
\vspace{\nbsphinxcodecellspacing}

\sphinxsetup{VerbatimColor={named}{white}}
\sphinxsetup{VerbatimBorderColor={named}{nbsphinx-code-border}}
\begin{sphinxuseclass}{output_area}
\begin{sphinxuseclass}{}


\begin{sphinxVerbatim}[commandchars=\\\{\}]
<Figure size 432x72 with 0 Axes>
\end{sphinxVerbatim}



\end{sphinxuseclass}
\end{sphinxuseclass}
}

\end{sphinxuseclass}
\begin{sphinxuseclass}{nboutput}
{

\kern-\sphinxverbatimsmallskipamount\kern-\baselineskip
\kern+\FrameHeightAdjust\kern-\fboxrule
\vspace{\nbsphinxcodecellspacing}

\sphinxsetup{VerbatimColor={named}{white}}
\sphinxsetup{VerbatimBorderColor={named}{nbsphinx-code-border}}
\begin{sphinxuseclass}{output_area}
\begin{sphinxuseclass}{}


\begin{sphinxVerbatim}[commandchars=\\\{\}]
<Figure size 432x72 with 0 Axes>
\end{sphinxVerbatim}



\end{sphinxuseclass}
\end{sphinxuseclass}
}

\end{sphinxuseclass}
\begin{sphinxuseclass}{nboutput}
\hrule height -\fboxrule\relax
\vspace{\nbsphinxcodecellspacing}

\makeatletter\setbox\nbsphinxpromptbox\box\voidb@x\makeatother

\begin{nbsphinxfancyoutput}

\begin{sphinxuseclass}{output_area}
\begin{sphinxuseclass}{}
\noindent\sphinxincludegraphics[width=355\sphinxpxdimen,height=145\sphinxpxdimen]{{example_pump_Pump_Example_Notebook_32_4}.png}

\end{sphinxuseclass}
\end{sphinxuseclass}
\end{nbsphinxfancyoutput}

\end{sphinxuseclass}
\begin{sphinxuseclass}{nboutput}
{

\kern-\sphinxverbatimsmallskipamount\kern-\baselineskip
\kern+\FrameHeightAdjust\kern-\fboxrule
\vspace{\nbsphinxcodecellspacing}

\sphinxsetup{VerbatimColor={named}{white}}
\sphinxsetup{VerbatimBorderColor={named}{nbsphinx-code-border}}
\begin{sphinxuseclass}{output_area}
\begin{sphinxuseclass}{}


\begin{sphinxVerbatim}[commandchars=\\\{\}]
<Figure size 432x72 with 0 Axes>
\end{sphinxVerbatim}



\end{sphinxuseclass}
\end{sphinxuseclass}
}

\end{sphinxuseclass}
\begin{sphinxuseclass}{nboutput}
\hrule height -\fboxrule\relax
\vspace{\nbsphinxcodecellspacing}

\makeatletter\setbox\nbsphinxpromptbox\box\voidb@x\makeatother

\begin{nbsphinxfancyoutput}

\begin{sphinxuseclass}{output_area}
\begin{sphinxuseclass}{}
\noindent\sphinxincludegraphics[width=416\sphinxpxdimen,height=145\sphinxpxdimen]{{example_pump_Pump_Example_Notebook_32_6}.png}

\end{sphinxuseclass}
\end{sphinxuseclass}
\end{nbsphinxfancyoutput}

\end{sphinxuseclass}
\begin{sphinxuseclass}{nboutput}
\hrule height -\fboxrule\relax
\vspace{\nbsphinxcodecellspacing}

\makeatletter\setbox\nbsphinxpromptbox\box\voidb@x\makeatother

\begin{nbsphinxfancyoutput}

\begin{sphinxuseclass}{output_area}
\begin{sphinxuseclass}{}
\noindent\sphinxincludegraphics[width=313\sphinxpxdimen,height=106\sphinxpxdimen]{{example_pump_Pump_Example_Notebook_32_7}.png}

\end{sphinxuseclass}
\end{sphinxuseclass}
\end{nbsphinxfancyoutput}

\end{sphinxuseclass}
\begin{sphinxuseclass}{nboutput}
\hrule height -\fboxrule\relax
\vspace{\nbsphinxcodecellspacing}

\makeatletter\setbox\nbsphinxpromptbox\box\voidb@x\makeatother

\begin{nbsphinxfancyoutput}

\begin{sphinxuseclass}{output_area}
\begin{sphinxuseclass}{}
\noindent\sphinxincludegraphics[width=426\sphinxpxdimen,height=286\sphinxpxdimen]{{example_pump_Pump_Example_Notebook_32_8}.png}

\end{sphinxuseclass}
\end{sphinxuseclass}
\end{nbsphinxfancyoutput}

\end{sphinxuseclass}
\begin{sphinxuseclass}{nboutput}
\begin{sphinxuseclass}{nblast}
\hrule height -\fboxrule\relax
\vspace{\nbsphinxcodecellspacing}

\makeatletter\setbox\nbsphinxpromptbox\box\voidb@x\makeatother

\begin{nbsphinxfancyoutput}

\begin{sphinxuseclass}{output_area}
\begin{sphinxuseclass}{}
\noindent\sphinxincludegraphics[width=426\sphinxpxdimen,height=286\sphinxpxdimen]{{example_pump_Pump_Example_Notebook_32_9}.png}

\end{sphinxuseclass}
\end{sphinxuseclass}
\end{nbsphinxfancyoutput}

\end{sphinxuseclass}
\end{sphinxuseclass}
\sphinxAtStartPar
As you can see, the system begins nominal until the fault is injected at \(t=10\). At this moment, not only are the electrical energy flows degraded, the flow of water is degraded also. However, at \(t=55\) when the system is supposed to be turned off, this flow of water is no longer “degraded” because it is in the same state as the nominal system.

\sphinxAtStartPar
We can look at a table of to see more precisely what happened (and export, if needed). Note that we need to give the plotting function the mode (‘short’) and the time for it to plot properly.

\sphinxAtStartPar
Here we can see that the short dropped the voltage to zero, (this was because an open circuit resulted in the Import EE function), causing the water to stop flowing. Below, we use the processed model history to show the faults and \sphinxstyleemphasis{degradation} of states over time. In this case, 1 means nominal while 0 means degraded.

\begin{sphinxuseclass}{nbinput}
{
\sphinxsetup{VerbatimColor={named}{nbsphinx-code-bg}}
\sphinxsetup{VerbatimBorderColor={named}{nbsphinx-code-border}}
\begin{sphinxVerbatim}[commandchars=\\\{\}]
\llap{\color{nbsphinxin}[17]:\,\hspace{\fboxrule}\hspace{\fboxsep}}\PYG{n}{short\PYGZus{}histtable} \PYG{o}{=} \PYG{n}{rd}\PYG{o}{.}\PYG{n}{tabulate}\PYG{o}{.}\PYG{n}{hist}\PYG{p}{(}\PYG{n}{reshist}\PYG{p}{)}
\PYG{n}{short\PYGZus{}histtable}
\end{sphinxVerbatim}
}

\end{sphinxuseclass}
\begin{sphinxuseclass}{nboutput}
\begin{sphinxuseclass}{nblast}
{

\kern-\sphinxverbatimsmallskipamount\kern-\baselineskip
\kern+\FrameHeightAdjust\kern-\fboxrule
\vspace{\nbsphinxcodecellspacing}

\sphinxsetup{VerbatimColor={named}{white}}
\sphinxsetup{VerbatimBorderColor={named}{nbsphinx-code-border}}
\begin{sphinxuseclass}{output_area}
\begin{sphinxuseclass}{}


\begin{sphinxVerbatim}[commandchars=\\\{\}]
\llap{\color{nbsphinxout}[17]:\,\hspace{\fboxrule}\hspace{\fboxsep}}   time  ImportEE                               ImportWater         \textbackslash{}
      t numfaults status no\_v fault inf\_v fault   numfaults status
0     0         0      1          0           0           0      1
1     1         0      1          0           0           0      1
2     2         0      1          0           0           0      1
3     3         0      1          0           0           0      1
4     4         0      1          0           0           0      1
5     5         0      1          0           0           0      1
6     6         0      1          0           0           0      1
7     7         0      1          0           0           0      1
8     8         0      1          0           0           0      1
9     9         0      1          0           0           0      1
10   10         1      0          1           0           0      1
11   11         1      0          1           0           0      1
12   12         1      0          1           0           0      1
13   13         1      0          1           0           0      1
14   14         1      0          1           0           0      1
15   15         1      0          1           0           0      1
16   16         1      0          1           0           0      1
17   17         1      0          1           0           0      1
18   18         1      0          1           0           0      1
19   19         1      0          1           0           0      1
20   20         1      0          1           0           0      1
21   21         1      0          1           0           0      1
22   22         1      0          1           0           0      1
23   23         1      0          1           0           0      1
24   24         1      0          1           0           0      1
25   25         1      0          1           0           0      1
26   26         1      0          1           0           0      1
27   27         1      0          1           0           0      1
28   28         1      0          1           0           0      1
29   29         1      0          1           0           0      1
30   30         1      0          1           0           0      1
31   31         1      0          1           0           0      1
32   32         1      0          1           0           0      1
33   33         1      0          1           0           0      1
34   34         1      0          1           0           0      1
35   35         1      0          1           0           0      1
36   36         1      0          1           0           0      1
37   37         1      0          1           0           0      1
38   38         1      0          1           0           0      1
39   39         1      0          1           0           0      1
40   40         1      0          1           0           0      1
41   41         1      0          1           0           0      1
42   42         1      0          1           0           0      1
43   43         1      0          1           0           0      1
44   44         1      0          1           0           0      1
45   45         1      0          1           0           0      1
46   46         1      0          1           0           0      1
47   47         1      0          1           0           0      1
48   48         1      0          1           0           0      1
49   49         1      0          1           0           0      1
50   50         1      0          1           0           0      1
51   51         1      0          1           0           0      1
52   52         1      0          1           0           0      1
53   53         1      0          1           0           0      1
54   54         1      0          1           0           0      1
55   55         1      0          1           0           0      1

                ImportSignal         {\ldots}    EE\_1 Sig\_1    Wat\_1                \textbackslash{}
   no\_wat fault    numfaults status  {\ldots} voltage power flowrate pressure area
0             0            0      1  {\ldots}       1     1        1        1    1
1             0            0      1  {\ldots}       1     1        1        1    1
2             0            0      1  {\ldots}       1     1        1        1    1
3             0            0      1  {\ldots}       1     1        1        1    1
4             0            0      1  {\ldots}       1     1        1        1    1
5             0            0      1  {\ldots}       1     1        1        1    1
6             0            0      1  {\ldots}       1     1        1        1    1
7             0            0      1  {\ldots}       1     1        1        1    1
8             0            0      1  {\ldots}       1     1        1        1    1
9             0            0      1  {\ldots}       1     1        1        1    1
10            0            0      1  {\ldots}       0     1        0        0    1
11            0            0      1  {\ldots}       0     1        0        0    1
12            0            0      1  {\ldots}       0     1        0        0    1
13            0            0      1  {\ldots}       0     1        0        0    1
14            0            0      1  {\ldots}       0     1        0        0    1
15            0            0      1  {\ldots}       0     1        0        0    1
16            0            0      1  {\ldots}       0     1        0        0    1
17            0            0      1  {\ldots}       0     1        0        0    1
18            0            0      1  {\ldots}       0     1        0        0    1
19            0            0      1  {\ldots}       0     1        0        0    1
20            0            0      1  {\ldots}       0     1        0        0    1
21            0            0      1  {\ldots}       0     1        0        0    1
22            0            0      1  {\ldots}       0     1        0        0    1
23            0            0      1  {\ldots}       0     1        0        0    1
24            0            0      1  {\ldots}       0     1        0        0    1
25            0            0      1  {\ldots}       0     1        0        0    1
26            0            0      1  {\ldots}       0     1        0        0    1
27            0            0      1  {\ldots}       0     1        0        0    1
28            0            0      1  {\ldots}       0     1        0        0    1
29            0            0      1  {\ldots}       0     1        0        0    1
30            0            0      1  {\ldots}       0     1        0        0    1
31            0            0      1  {\ldots}       0     1        0        0    1
32            0            0      1  {\ldots}       0     1        0        0    1
33            0            0      1  {\ldots}       0     1        0        0    1
34            0            0      1  {\ldots}       0     1        0        0    1
35            0            0      1  {\ldots}       0     1        0        0    1
36            0            0      1  {\ldots}       0     1        0        0    1
37            0            0      1  {\ldots}       0     1        0        0    1
38            0            0      1  {\ldots}       0     1        0        0    1
39            0            0      1  {\ldots}       0     1        0        0    1
40            0            0      1  {\ldots}       0     1        0        0    1
41            0            0      1  {\ldots}       0     1        0        0    1
42            0            0      1  {\ldots}       0     1        0        0    1
43            0            0      1  {\ldots}       0     1        0        0    1
44            0            0      1  {\ldots}       0     1        0        0    1
45            0            0      1  {\ldots}       0     1        0        0    1
46            0            0      1  {\ldots}       0     1        0        0    1
47            0            0      1  {\ldots}       0     1        0        0    1
48            0            0      1  {\ldots}       0     1        0        0    1
49            0            0      1  {\ldots}       0     1        0        0    1
50            0            0      1  {\ldots}       0     1        1        1    1
51            0            0      1  {\ldots}       0     1        1        1    1
52            0            0      1  {\ldots}       0     1        1        1    1
53            0            0      1  {\ldots}       0     1        1        1    1
54            0            0      1  {\ldots}       0     1        1        1    1
55            0            0      1  {\ldots}       0     1        1        1    1

            Wat\_2
   level flowrate pressure area level
0      1        1        1    1     1
1      1        1        1    1     1
2      1        1        1    1     1
3      1        1        1    1     1
4      1        1        1    1     1
5      1        1        1    1     1
6      1        1        1    1     1
7      1        1        1    1     1
8      1        1        1    1     1
9      1        1        1    1     1
10     1        0        0    1     1
11     1        0        0    1     1
12     1        0        0    1     1
13     1        0        0    1     1
14     1        0        0    1     1
15     1        0        0    1     1
16     1        0        0    1     1
17     1        0        0    1     1
18     1        0        0    1     1
19     1        0        0    1     1
20     1        0        0    1     1
21     1        0        0    1     1
22     1        0        0    1     1
23     1        0        0    1     1
24     1        0        0    1     1
25     1        0        0    1     1
26     1        0        0    1     1
27     1        0        0    1     1
28     1        0        0    1     1
29     1        0        0    1     1
30     1        0        0    1     1
31     1        0        0    1     1
32     1        0        0    1     1
33     1        0        0    1     1
34     1        0        0    1     1
35     1        0        0    1     1
36     1        0        0    1     1
37     1        0        0    1     1
38     1        0        0    1     1
39     1        0        0    1     1
40     1        0        0    1     1
41     1        0        0    1     1
42     1        0        0    1     1
43     1        0        0    1     1
44     1        0        0    1     1
45     1        0        0    1     1
46     1        0        0    1     1
47     1        0        0    1     1
48     1        0        0    1     1
49     1        0        0    1     1
50     1        1        1    1     1
51     1        1        1    1     1
52     1        1        1    1     1
53     1        1        1    1     1
54     1        1        1    1     1
55     1        1        1    1     1

[56 rows x 30 columns]
\end{sphinxVerbatim}



\end{sphinxuseclass}
\end{sphinxuseclass}
}

\end{sphinxuseclass}
\end{sphinxuseclass}
\sphinxAtStartPar
If we want a simpler view of just the degraded faults and flows (rather than the specific faults, etc), we can use:

\begin{sphinxuseclass}{nbinput}
{
\sphinxsetup{VerbatimColor={named}{nbsphinx-code-bg}}
\sphinxsetup{VerbatimBorderColor={named}{nbsphinx-code-border}}
\begin{sphinxVerbatim}[commandchars=\\\{\}]
\llap{\color{nbsphinxin}[18]:\,\hspace{\fboxrule}\hspace{\fboxsep}}\PYG{n}{short\PYGZus{}deghisttable} \PYG{o}{=} \PYG{n}{rd}\PYG{o}{.}\PYG{n}{tabulate}\PYG{o}{.}\PYG{n}{deghist}\PYG{p}{(}\PYG{n}{reshist}\PYG{p}{)}
\PYG{n}{short\PYGZus{}deghisttable}\PYG{p}{[}\PYG{l+m+mi}{1}\PYG{p}{:}\PYG{l+m+mi}{20}\PYG{p}{]}
\end{sphinxVerbatim}
}

\end{sphinxuseclass}
\begin{sphinxuseclass}{nboutput}
\begin{sphinxuseclass}{nblast}
{

\kern-\sphinxverbatimsmallskipamount\kern-\baselineskip
\kern+\FrameHeightAdjust\kern-\fboxrule
\vspace{\nbsphinxcodecellspacing}

\sphinxsetup{VerbatimColor={named}{white}}
\sphinxsetup{VerbatimBorderColor={named}{nbsphinx-code-border}}
\begin{sphinxuseclass}{output_area}
\begin{sphinxuseclass}{}


\begin{sphinxVerbatim}[commandchars=\\\{\}]
\llap{\color{nbsphinxout}[18]:\,\hspace{\fboxrule}\hspace{\fboxsep}}    time  ImportEE  ImportWater  ImportSignal  MoveWater  ExportWater  EE\_1  \textbackslash{}
1      1         1            1             1          1            1     1
2      2         1            1             1          1            1     1
3      3         1            1             1          1            1     1
4      4         1            1             1          1            1     1
5      5         1            1             1          1            1     1
6      6         1            1             1          1            1     1
7      7         1            1             1          1            1     1
8      8         1            1             1          1            1     1
9      9         1            1             1          1            1     1
10    10         0            1             1          0            1     0
11    11         0            1             1          0            1     0
12    12         0            1             1          0            1     0
13    13         0            1             1          0            1     0
14    14         0            1             1          0            1     0
15    15         0            1             1          0            1     0
16    16         0            1             1          0            1     0
17    17         0            1             1          0            1     0
18    18         0            1             1          0            1     0
19    19         0            1             1          0            1     0

    Sig\_1  Wat\_1  Wat\_2
1       1      1      1
2       1      1      1
3       1      1      1
4       1      1      1
5       1      1      1
6       1      1      1
7       1      1      1
8       1      1      1
9       1      1      1
10      1      0      0
11      1      0      0
12      1      0      0
13      1      0      0
14      1      0      0
15      1      0      0
16      1      0      0
17      1      0      0
18      1      0      0
19      1      0      0
\end{sphinxVerbatim}



\end{sphinxuseclass}
\end{sphinxuseclass}
}

\end{sphinxuseclass}
\end{sphinxuseclass}
\sphinxAtStartPar
We can also look at statistics of degradation over time using:

\begin{sphinxuseclass}{nbinput}
{
\sphinxsetup{VerbatimColor={named}{nbsphinx-code-bg}}
\sphinxsetup{VerbatimBorderColor={named}{nbsphinx-code-border}}
\begin{sphinxVerbatim}[commandchars=\\\{\}]
\llap{\color{nbsphinxin}[19]:\,\hspace{\fboxrule}\hspace{\fboxsep}}\PYG{n}{short\PYGZus{}statstable} \PYG{o}{=} \PYG{n}{rd}\PYG{o}{.}\PYG{n}{tabulate}\PYG{o}{.}\PYG{n}{stats}\PYG{p}{(}\PYG{n}{reshist}\PYG{p}{)}
\PYG{n}{short\PYGZus{}statstable}\PYG{p}{[}\PYG{p}{:}\PYG{l+m+mi}{20}\PYG{p}{]}
\end{sphinxVerbatim}
}

\end{sphinxuseclass}
\begin{sphinxuseclass}{nboutput}
\begin{sphinxuseclass}{nblast}
{

\kern-\sphinxverbatimsmallskipamount\kern-\baselineskip
\kern+\FrameHeightAdjust\kern-\fboxrule
\vspace{\nbsphinxcodecellspacing}

\sphinxsetup{VerbatimColor={named}{white}}
\sphinxsetup{VerbatimBorderColor={named}{nbsphinx-code-border}}
\begin{sphinxuseclass}{output_area}
\begin{sphinxuseclass}{}


\begin{sphinxVerbatim}[commandchars=\\\{\}]
\llap{\color{nbsphinxout}[19]:\,\hspace{\fboxrule}\hspace{\fboxsep}}    time  degraded flows  degraded functions  total faults
0      0               0                   0             0
1      1               0                   0             0
2      2               0                   0             0
3      3               0                   0             0
4      4               0                   0             0
5      5               0                   0             0
6      6               0                   0             0
7      7               0                   0             0
8      8               0                   0             0
9      9               0                   0             0
10    10               3                   2             2
11    11               3                   2             2
12    12               3                   2             2
13    13               3                   2             2
14    14               3                   2             2
15    15               3                   2             2
16    16               3                   2             2
17    17               3                   2             2
18    18               3                   2             2
19    19               3                   2             2
\end{sphinxVerbatim}



\end{sphinxuseclass}
\end{sphinxuseclass}
}

\end{sphinxuseclass}
\end{sphinxuseclass}
\sphinxAtStartPar
We can also look at other faults. The results below are for a blockage of the pipe. In this case we’re only interested in the effect on the water going through, so only those flows are tracked.

\begin{sphinxuseclass}{nbinput}
{
\sphinxsetup{VerbatimColor={named}{nbsphinx-code-bg}}
\sphinxsetup{VerbatimBorderColor={named}{nbsphinx-code-border}}
\begin{sphinxVerbatim}[commandchars=\\\{\}]
\llap{\color{nbsphinxin}[20]:\,\hspace{\fboxrule}\hspace{\fboxsep}}\PYG{n}{endresults2}\PYG{p}{,} \PYG{n}{resgraph2}\PYG{p}{,} \PYG{n}{mdlhist2}\PYG{o}{=}\PYG{n}{propagate}\PYG{o}{.}\PYG{n}{one\PYGZus{}fault}\PYG{p}{(}\PYG{n}{mdl}\PYG{p}{,} \PYG{l+s+s1}{\PYGZsq{}}\PYG{l+s+s1}{ExportWater}\PYG{l+s+s1}{\PYGZsq{}}\PYG{p}{,} \PYG{l+s+s1}{\PYGZsq{}}\PYG{l+s+s1}{block}\PYG{l+s+s1}{\PYGZsq{}}\PYG{p}{,} \PYG{n}{time}\PYG{o}{=}\PYG{l+m+mi}{10}\PYG{p}{)}
\PYG{n}{reshist2}\PYG{p}{,}\PYG{n}{diff2}\PYG{p}{,} \PYG{n}{summary2} \PYG{o}{=} \PYG{n}{rd}\PYG{o}{.}\PYG{n}{process}\PYG{o}{.}\PYG{n}{hist}\PYG{p}{(}\PYG{n}{mdlhist2}\PYG{p}{)}
\PYG{n}{restab2} \PYG{o}{=} \PYG{n}{rd}\PYG{o}{.}\PYG{n}{tabulate}\PYG{o}{.}\PYG{n}{result}\PYG{p}{(}\PYG{n}{endresults2}\PYG{p}{,} \PYG{n}{summary2}\PYG{p}{)}
\PYG{n}{restab2}
\end{sphinxVerbatim}
}

\end{sphinxuseclass}
\begin{sphinxuseclass}{nboutput}
\begin{sphinxuseclass}{nblast}
{

\kern-\sphinxverbatimsmallskipamount\kern-\baselineskip
\kern+\FrameHeightAdjust\kern-\fboxrule
\vspace{\nbsphinxcodecellspacing}

\sphinxsetup{VerbatimColor={named}{white}}
\sphinxsetup{VerbatimBorderColor={named}{nbsphinx-code-border}}
\begin{sphinxuseclass}{output_area}
\begin{sphinxuseclass}{}


\begin{sphinxVerbatim}[commandchars=\\\{\}]
\llap{\color{nbsphinxout}[20]:\,\hspace{\fboxrule}\hspace{\fboxsep}}      rate     cost  expected cost        degraded functions  \textbackslash{}
0  0.00055  18977.5      1043762.5  [MoveWater, ExportWater]

         degraded flows
0  [EE\_1, Wat\_1, Wat\_2]
\end{sphinxVerbatim}



\end{sphinxuseclass}
\end{sphinxuseclass}
}

\end{sphinxuseclass}
\end{sphinxuseclass}
\begin{sphinxuseclass}{nbinput}
{
\sphinxsetup{VerbatimColor={named}{nbsphinx-code-bg}}
\sphinxsetup{VerbatimBorderColor={named}{nbsphinx-code-border}}
\begin{sphinxVerbatim}[commandchars=\\\{\}]
\llap{\color{nbsphinxin}[21]:\,\hspace{\fboxrule}\hspace{\fboxsep}}\PYG{n}{rd}\PYG{o}{.}\PYG{n}{graph}\PYG{o}{.}\PYG{n}{show}\PYG{p}{(}\PYG{n}{resgraph2}\PYG{p}{)}
\end{sphinxVerbatim}
}

\end{sphinxuseclass}
\begin{sphinxuseclass}{nboutput}
{

\kern-\sphinxverbatimsmallskipamount\kern-\baselineskip
\kern+\FrameHeightAdjust\kern-\fboxrule
\vspace{\nbsphinxcodecellspacing}

\sphinxsetup{VerbatimColor={named}{white}}
\sphinxsetup{VerbatimBorderColor={named}{nbsphinx-code-border}}
\begin{sphinxuseclass}{output_area}
\begin{sphinxuseclass}{}


\begin{sphinxVerbatim}[commandchars=\\\{\}]
\llap{\color{nbsphinxout}[21]:\,\hspace{\fboxrule}\hspace{\fboxsep}}(<Figure size 432x288 with 1 Axes>, <AxesSubplot:>)
\end{sphinxVerbatim}



\end{sphinxuseclass}
\end{sphinxuseclass}
}

\end{sphinxuseclass}
\begin{sphinxuseclass}{nboutput}
\begin{sphinxuseclass}{nblast}
\hrule height -\fboxrule\relax
\vspace{\nbsphinxcodecellspacing}

\makeatletter\setbox\nbsphinxpromptbox\box\voidb@x\makeatother

\begin{nbsphinxfancyoutput}

\begin{sphinxuseclass}{output_area}
\begin{sphinxuseclass}{}
\noindent\sphinxincludegraphics[width=349\sphinxpxdimen,height=231\sphinxpxdimen]{{example_pump_Pump_Example_Notebook_41_1}.png}

\end{sphinxuseclass}
\end{sphinxuseclass}
\end{nbsphinxfancyoutput}

\end{sphinxuseclass}
\end{sphinxuseclass}
\begin{sphinxuseclass}{nbinput}
{
\sphinxsetup{VerbatimColor={named}{nbsphinx-code-bg}}
\sphinxsetup{VerbatimBorderColor={named}{nbsphinx-code-border}}
\begin{sphinxVerbatim}[commandchars=\\\{\}]
\llap{\color{nbsphinxin}[22]:\,\hspace{\fboxrule}\hspace{\fboxsep}}\PYG{n}{rd}\PYG{o}{.}\PYG{n}{plot}\PYG{o}{.}\PYG{n}{mdlhist}\PYG{p}{(}\PYG{n}{mdlhist2}\PYG{p}{,} \PYG{l+s+s1}{\PYGZsq{}}\PYG{l+s+s1}{blockage}\PYG{l+s+s1}{\PYGZsq{}}\PYG{p}{,} \PYG{n}{time}\PYG{o}{=}\PYG{l+m+mi}{10}\PYG{p}{,} \PYG{n}{legend}\PYG{o}{=}\PYG{k+kc}{False}\PYG{p}{)}
\end{sphinxVerbatim}
}

\end{sphinxuseclass}
\begin{sphinxuseclass}{nboutput}
{

\kern-\sphinxverbatimsmallskipamount\kern-\baselineskip
\kern+\FrameHeightAdjust\kern-\fboxrule
\vspace{\nbsphinxcodecellspacing}

\sphinxsetup{VerbatimColor={named}{nbsphinx-stderr}}
\sphinxsetup{VerbatimBorderColor={named}{nbsphinx-code-border}}
\begin{sphinxuseclass}{output_area}
\begin{sphinxuseclass}{stderr}


\begin{sphinxVerbatim}[commandchars=\\\{\}]
C:\textbackslash{}Users\textbackslash{}dhulse\textbackslash{}Documents\textbackslash{}GitHub\textbackslash{}fmdtools\textbackslash{}example\_pump\textbackslash{}..\textbackslash{}fmdtools\textbackslash{}resultdisp\textbackslash{}plot.py:176: UserWarning: Tight layout not applied. The bottom and top margins cannot be made large enough to accommodate all axes decorations.
  plt.tight\_layout(pad=1)
\end{sphinxVerbatim}



\end{sphinxuseclass}
\end{sphinxuseclass}
}

\end{sphinxuseclass}
\begin{sphinxuseclass}{nboutput}
{

\kern-\sphinxverbatimsmallskipamount\kern-\baselineskip
\kern+\FrameHeightAdjust\kern-\fboxrule
\vspace{\nbsphinxcodecellspacing}

\sphinxsetup{VerbatimColor={named}{white}}
\sphinxsetup{VerbatimBorderColor={named}{nbsphinx-code-border}}
\begin{sphinxuseclass}{output_area}
\begin{sphinxuseclass}{}


\begin{sphinxVerbatim}[commandchars=\\\{\}]
<Figure size 432x72 with 0 Axes>
\end{sphinxVerbatim}



\end{sphinxuseclass}
\end{sphinxuseclass}
}

\end{sphinxuseclass}
\begin{sphinxuseclass}{nboutput}
{

\kern-\sphinxverbatimsmallskipamount\kern-\baselineskip
\kern+\FrameHeightAdjust\kern-\fboxrule
\vspace{\nbsphinxcodecellspacing}

\sphinxsetup{VerbatimColor={named}{white}}
\sphinxsetup{VerbatimBorderColor={named}{nbsphinx-code-border}}
\begin{sphinxuseclass}{output_area}
\begin{sphinxuseclass}{}


\begin{sphinxVerbatim}[commandchars=\\\{\}]
<Figure size 432x72 with 0 Axes>
\end{sphinxVerbatim}



\end{sphinxuseclass}
\end{sphinxuseclass}
}

\end{sphinxuseclass}
\begin{sphinxuseclass}{nboutput}
{

\kern-\sphinxverbatimsmallskipamount\kern-\baselineskip
\kern+\FrameHeightAdjust\kern-\fboxrule
\vspace{\nbsphinxcodecellspacing}

\sphinxsetup{VerbatimColor={named}{white}}
\sphinxsetup{VerbatimBorderColor={named}{nbsphinx-code-border}}
\begin{sphinxuseclass}{output_area}
\begin{sphinxuseclass}{}


\begin{sphinxVerbatim}[commandchars=\\\{\}]
<Figure size 432x72 with 0 Axes>
\end{sphinxVerbatim}



\end{sphinxuseclass}
\end{sphinxuseclass}
}

\end{sphinxuseclass}
\begin{sphinxuseclass}{nboutput}
\hrule height -\fboxrule\relax
\vspace{\nbsphinxcodecellspacing}

\makeatletter\setbox\nbsphinxpromptbox\box\voidb@x\makeatother

\begin{nbsphinxfancyoutput}

\begin{sphinxuseclass}{output_area}
\begin{sphinxuseclass}{}
\noindent\sphinxincludegraphics[width=367\sphinxpxdimen,height=145\sphinxpxdimen]{{example_pump_Pump_Example_Notebook_42_4}.png}

\end{sphinxuseclass}
\end{sphinxuseclass}
\end{nbsphinxfancyoutput}

\end{sphinxuseclass}
\begin{sphinxuseclass}{nboutput}
{

\kern-\sphinxverbatimsmallskipamount\kern-\baselineskip
\kern+\FrameHeightAdjust\kern-\fboxrule
\vspace{\nbsphinxcodecellspacing}

\sphinxsetup{VerbatimColor={named}{white}}
\sphinxsetup{VerbatimBorderColor={named}{nbsphinx-code-border}}
\begin{sphinxuseclass}{output_area}
\begin{sphinxuseclass}{}


\begin{sphinxVerbatim}[commandchars=\\\{\}]
<Figure size 432x72 with 0 Axes>
\end{sphinxVerbatim}



\end{sphinxuseclass}
\end{sphinxuseclass}
}

\end{sphinxuseclass}
\begin{sphinxuseclass}{nboutput}
\hrule height -\fboxrule\relax
\vspace{\nbsphinxcodecellspacing}

\makeatletter\setbox\nbsphinxpromptbox\box\voidb@x\makeatother

\begin{nbsphinxfancyoutput}

\begin{sphinxuseclass}{output_area}
\begin{sphinxuseclass}{}
\noindent\sphinxincludegraphics[width=426\sphinxpxdimen,height=145\sphinxpxdimen]{{example_pump_Pump_Example_Notebook_42_6}.png}

\end{sphinxuseclass}
\end{sphinxuseclass}
\end{nbsphinxfancyoutput}

\end{sphinxuseclass}
\begin{sphinxuseclass}{nboutput}
\hrule height -\fboxrule\relax
\vspace{\nbsphinxcodecellspacing}

\makeatletter\setbox\nbsphinxpromptbox\box\voidb@x\makeatother

\begin{nbsphinxfancyoutput}

\begin{sphinxuseclass}{output_area}
\begin{sphinxuseclass}{}
\noindent\sphinxincludegraphics[width=325\sphinxpxdimen,height=106\sphinxpxdimen]{{example_pump_Pump_Example_Notebook_42_7}.png}

\end{sphinxuseclass}
\end{sphinxuseclass}
\end{nbsphinxfancyoutput}

\end{sphinxuseclass}
\begin{sphinxuseclass}{nboutput}
\hrule height -\fboxrule\relax
\vspace{\nbsphinxcodecellspacing}

\makeatletter\setbox\nbsphinxpromptbox\box\voidb@x\makeatother

\begin{nbsphinxfancyoutput}

\begin{sphinxuseclass}{output_area}
\begin{sphinxuseclass}{}
\noindent\sphinxincludegraphics[width=426\sphinxpxdimen,height=286\sphinxpxdimen]{{example_pump_Pump_Example_Notebook_42_8}.png}

\end{sphinxuseclass}
\end{sphinxuseclass}
\end{nbsphinxfancyoutput}

\end{sphinxuseclass}
\begin{sphinxuseclass}{nboutput}
\begin{sphinxuseclass}{nblast}
\hrule height -\fboxrule\relax
\vspace{\nbsphinxcodecellspacing}

\makeatletter\setbox\nbsphinxpromptbox\box\voidb@x\makeatother

\begin{nbsphinxfancyoutput}

\begin{sphinxuseclass}{output_area}
\begin{sphinxuseclass}{}
\noindent\sphinxincludegraphics[width=426\sphinxpxdimen,height=286\sphinxpxdimen]{{example_pump_Pump_Example_Notebook_42_9}.png}

\end{sphinxuseclass}
\end{sphinxuseclass}
\end{nbsphinxfancyoutput}

\end{sphinxuseclass}
\end{sphinxuseclass}

\subsubsection{Visualization of resilience metrics}
\label{\detokenize{example_pump/Pump_Example_Notebook:Visualization-of-resilience-metrics}}
\sphinxAtStartPar
We can use the processed time history to now make visualizations of the resilience of the system over time. Below we use the “makeheatmaps” function, which calculates the following metrics of interest: \sphinxhyphen{} “degtime,” the percentage of the time the function/flow of the system was degraded \sphinxhyphen{} “maxdeg,” the number of flow values that were degraded at a given time \sphinxhyphen{} “intdeg,” the number of flow values that were degraded at a given time * the time degraded \sphinxhyphen{} “maxfaults,” the maximum number of faults in
each function at any given time \sphinxhyphen{} “maxdiff,” the max distance between states of functions/flows and the nominal \sphinxhyphen{} “intdiff,” the distance between states of functions/flows and the nominal * the time off\sphinxhyphen{}nominal

\begin{sphinxuseclass}{nbinput}
{
\sphinxsetup{VerbatimColor={named}{nbsphinx-code-bg}}
\sphinxsetup{VerbatimBorderColor={named}{nbsphinx-code-border}}
\begin{sphinxVerbatim}[commandchars=\\\{\}]
\llap{\color{nbsphinxin}[23]:\,\hspace{\fboxrule}\hspace{\fboxsep}}\PYG{n}{heatmaps} \PYG{o}{=} \PYG{n}{rd}\PYG{o}{.}\PYG{n}{process}\PYG{o}{.}\PYG{n}{heatmaps}\PYG{p}{(}\PYG{n}{reshist2}\PYG{p}{,} \PYG{n}{diff2}\PYG{p}{)}
\PYG{n}{heatmapstable} \PYG{o}{=} \PYG{n}{rd}\PYG{o}{.}\PYG{n}{tabulate}\PYG{o}{.}\PYG{n}{heatmaps}\PYG{p}{(}\PYG{n}{heatmaps}\PYG{p}{)}
\PYG{n}{heatmapstable}
\end{sphinxVerbatim}
}

\end{sphinxuseclass}
\begin{sphinxuseclass}{nboutput}
\begin{sphinxuseclass}{nblast}
{

\kern-\sphinxverbatimsmallskipamount\kern-\baselineskip
\kern+\FrameHeightAdjust\kern-\fboxrule
\vspace{\nbsphinxcodecellspacing}

\sphinxsetup{VerbatimColor={named}{white}}
\sphinxsetup{VerbatimBorderColor={named}{nbsphinx-code-border}}
\begin{sphinxuseclass}{output_area}
\begin{sphinxuseclass}{}


\begin{sphinxVerbatim}[commandchars=\\\{\}]
\llap{\color{nbsphinxout}[23]:\,\hspace{\fboxrule}\hspace{\fboxsep}}           ImportEE  ImportWater  ImportSignal  MoveWater  ExportWater  \textbackslash{}
degtime         0.0          0.0           0.0   0.642857     0.821429
maxdeg          NaN          NaN           NaN        NaN          NaN
intdeg          NaN          NaN           NaN        NaN          NaN
maxfaults       0.0          0.0           0.0   1.000000     1.000000
intdiff         NaN          NaN           NaN   0.642857          NaN
maxdiff         NaN          NaN           NaN   0.017857          NaN

               EE\_1  Sig\_1      Wat\_1      Wat\_2
degtime    0.714286    0.0   0.714286   0.821429
maxdeg     2.000000    1.0   4.000000   4.000000
intdeg     1.285714    1.0   2.571429   1.750000
maxfaults       NaN    NaN        NaN        NaN
intdiff    2.357143    0.0 -42.803705 -42.600402
maxdiff    0.087500    0.0   0.045982   0.050402
\end{sphinxVerbatim}



\end{sphinxuseclass}
\end{sphinxuseclass}
}

\end{sphinxuseclass}
\end{sphinxuseclass}
\sphinxAtStartPar
Note: not all of these maps will display values for all functions and flows, as shown by the NaN’s in the table. I’ll use “degtime” to illustrate.

\begin{sphinxuseclass}{nbinput}
{
\sphinxsetup{VerbatimColor={named}{nbsphinx-code-bg}}
\sphinxsetup{VerbatimBorderColor={named}{nbsphinx-code-border}}
\begin{sphinxVerbatim}[commandchars=\\\{\}]
\llap{\color{nbsphinxin}[24]:\,\hspace{\fboxrule}\hspace{\fboxsep}}\PYG{n}{a} \PYG{o}{=} \PYG{n}{rd}\PYG{o}{.}\PYG{n}{graph}\PYG{o}{.}\PYG{n}{show}\PYG{p}{(}\PYG{n}{mdl}\PYG{o}{.}\PYG{n}{bipartite}\PYG{p}{,}\PYG{n}{gtype}\PYG{o}{=}\PYG{l+s+s1}{\PYGZsq{}}\PYG{l+s+s1}{bipartite}\PYG{l+s+s1}{\PYGZsq{}}\PYG{p}{,} \PYG{n}{heatmap}\PYG{o}{=}\PYG{n}{heatmaps}\PYG{p}{[}\PYG{l+s+s1}{\PYGZsq{}}\PYG{l+s+s1}{degtime}\PYG{l+s+s1}{\PYGZsq{}}\PYG{p}{]}\PYG{p}{,} \PYG{n}{scale}\PYG{o}{=}\PYG{l+m+mi}{2}\PYG{p}{,} \PYG{n}{pos}\PYG{o}{=}\PYG{n}{pos}\PYG{p}{)}
\PYG{n}{dot} \PYG{o}{=} \PYG{n}{rd}\PYG{o}{.}\PYG{n}{graph}\PYG{o}{.}\PYG{n}{show}\PYG{p}{(}\PYG{n}{mdl}\PYG{o}{.}\PYG{n}{bipartite}\PYG{p}{,}\PYG{n}{gtype}\PYG{o}{=}\PYG{l+s+s1}{\PYGZsq{}}\PYG{l+s+s1}{bipartite}\PYG{l+s+s1}{\PYGZsq{}}\PYG{p}{,} \PYG{n}{heatmap}\PYG{o}{=}\PYG{n}{heatmaps}\PYG{p}{[}\PYG{l+s+s1}{\PYGZsq{}}\PYG{l+s+s1}{degtime}\PYG{l+s+s1}{\PYGZsq{}}\PYG{p}{]}\PYG{p}{,} \PYG{n}{renderer}\PYG{o}{=}\PYG{l+s+s1}{\PYGZsq{}}\PYG{l+s+s1}{graphviz}\PYG{l+s+s1}{\PYGZsq{}}\PYG{p}{)}
\end{sphinxVerbatim}
}

\end{sphinxuseclass}
\begin{sphinxuseclass}{nboutput}
\hrule height -\fboxrule\relax
\vspace{\nbsphinxcodecellspacing}

\makeatletter\setbox\nbsphinxpromptbox\box\voidb@x\makeatother

\begin{nbsphinxfancyoutput}

\begin{sphinxuseclass}{output_area}
\begin{sphinxuseclass}{}
\noindent\sphinxincludegraphics{{example_pump_Pump_Example_Notebook_46_0}.svg}

\end{sphinxuseclass}
\end{sphinxuseclass}
\end{nbsphinxfancyoutput}

\end{sphinxuseclass}
\begin{sphinxuseclass}{nboutput}
\begin{sphinxuseclass}{nblast}
\hrule height -\fboxrule\relax
\vspace{\nbsphinxcodecellspacing}

\makeatletter\setbox\nbsphinxpromptbox\box\voidb@x\makeatother

\begin{nbsphinxfancyoutput}

\begin{sphinxuseclass}{output_area}
\begin{sphinxuseclass}{}
\noindent\sphinxincludegraphics[width=349\sphinxpxdimen,height=231\sphinxpxdimen]{{example_pump_Pump_Example_Notebook_46_1}.png}

\end{sphinxuseclass}
\end{sphinxuseclass}
\end{nbsphinxfancyoutput}

\end{sphinxuseclass}
\end{sphinxuseclass}
\sphinxAtStartPar
These maps can also be plotted on the graph view, where only those for funcions will be shown. Here the maximum number of faults is plotted.

\begin{sphinxuseclass}{nbinput}
{
\sphinxsetup{VerbatimColor={named}{nbsphinx-code-bg}}
\sphinxsetup{VerbatimBorderColor={named}{nbsphinx-code-border}}
\begin{sphinxVerbatim}[commandchars=\\\{\}]
\llap{\color{nbsphinxin}[25]:\,\hspace{\fboxrule}\hspace{\fboxsep}}\PYG{n}{rd}\PYG{o}{.}\PYG{n}{graph}\PYG{o}{.}\PYG{n}{show}\PYG{p}{(}\PYG{n}{mdl}\PYG{o}{.}\PYG{n}{graph}\PYG{p}{,} \PYG{n}{heatmap}\PYG{o}{=}\PYG{n}{heatmaps}\PYG{p}{[}\PYG{l+s+s1}{\PYGZsq{}}\PYG{l+s+s1}{maxfaults}\PYG{l+s+s1}{\PYGZsq{}}\PYG{p}{]}\PYG{p}{)}
\PYG{n}{dot} \PYG{o}{=} \PYG{n}{rd}\PYG{o}{.}\PYG{n}{graph}\PYG{o}{.}\PYG{n}{show}\PYG{p}{(}\PYG{n}{mdl}\PYG{o}{.}\PYG{n}{graph}\PYG{p}{,} \PYG{n}{heatmap}\PYG{o}{=}\PYG{n}{heatmaps}\PYG{p}{[}\PYG{l+s+s1}{\PYGZsq{}}\PYG{l+s+s1}{maxfaults}\PYG{l+s+s1}{\PYGZsq{}}\PYG{p}{]}\PYG{p}{,} \PYG{n}{renderer}\PYG{o}{=}\PYG{l+s+s1}{\PYGZsq{}}\PYG{l+s+s1}{graphviz}\PYG{l+s+s1}{\PYGZsq{}}\PYG{p}{)}
\end{sphinxVerbatim}
}

\end{sphinxuseclass}
\begin{sphinxuseclass}{nboutput}
\hrule height -\fboxrule\relax
\vspace{\nbsphinxcodecellspacing}

\makeatletter\setbox\nbsphinxpromptbox\box\voidb@x\makeatother

\begin{nbsphinxfancyoutput}

\begin{sphinxuseclass}{output_area}
\begin{sphinxuseclass}{}
\noindent\sphinxincludegraphics{{example_pump_Pump_Example_Notebook_48_0}.svg}

\end{sphinxuseclass}
\end{sphinxuseclass}
\end{nbsphinxfancyoutput}

\end{sphinxuseclass}
\begin{sphinxuseclass}{nboutput}
\begin{sphinxuseclass}{nblast}
\hrule height -\fboxrule\relax
\vspace{\nbsphinxcodecellspacing}

\makeatletter\setbox\nbsphinxpromptbox\box\voidb@x\makeatother

\begin{nbsphinxfancyoutput}

\begin{sphinxuseclass}{output_area}
\begin{sphinxuseclass}{}
\noindent\sphinxincludegraphics[width=349\sphinxpxdimen,height=231\sphinxpxdimen]{{example_pump_Pump_Example_Notebook_48_1}.png}

\end{sphinxuseclass}
\end{sphinxuseclass}
\end{nbsphinxfancyoutput}

\end{sphinxuseclass}
\end{sphinxuseclass}

\subsubsection{Running a List of Faults}
\label{\detokenize{example_pump/Pump_Example_Notebook:Running-a-List-of-Faults}}
\sphinxAtStartPar
Finally, to get the results of all of the single\sphinxhyphen{}fault scenarios defined in the model, we can run them all at once using the \sphinxcode{\sphinxupquote{single\_faults()}} function. Note that this will propagate faults based on the times vector put in the model, e.g. if mdl.times={[}0,3,15,55{]}, it will propogate the faults at the begining, end, and at t=15 and t=15. This function only takes in the model mdl and outputs two similar kinds of output–resultsdict (the results in a python dictionary) and resultstab (the results
in a nice tabular form).

\sphinxAtStartPar
Note that the rates provide for this table do not use the opportunity vector information, instead using the assumption that the fault scenario has the rate provided over the entire simulation.

\sphinxAtStartPar
See below:

\begin{sphinxuseclass}{nbinput}
{
\sphinxsetup{VerbatimColor={named}{nbsphinx-code-bg}}
\sphinxsetup{VerbatimBorderColor={named}{nbsphinx-code-border}}
\begin{sphinxVerbatim}[commandchars=\\\{\}]
\llap{\color{nbsphinxin}[26]:\,\hspace{\fboxrule}\hspace{\fboxsep}}\PYG{n}{endclasses}\PYG{p}{,} \PYG{n}{mdlhists}\PYG{o}{=}\PYG{n}{propagate}\PYG{o}{.}\PYG{n}{single\PYGZus{}faults}\PYG{p}{(}\PYG{n}{mdl}\PYG{p}{,} \PYG{n}{staged}\PYG{o}{=}\PYG{k+kc}{True}\PYG{p}{)}
\PYG{n}{simplefmea} \PYG{o}{=} \PYG{n}{rd}\PYG{o}{.}\PYG{n}{tabulate}\PYG{o}{.}\PYG{n}{simplefmea}\PYG{p}{(}\PYG{n}{endclasses}\PYG{p}{)}
\PYG{n}{simplefmea}\PYG{p}{[}\PYG{p}{:}\PYG{l+m+mi}{5}\PYG{p}{]}
\end{sphinxVerbatim}
}

\end{sphinxuseclass}
\begin{sphinxuseclass}{nboutput}
{

\kern-\sphinxverbatimsmallskipamount\kern-\baselineskip
\kern+\FrameHeightAdjust\kern-\fboxrule
\vspace{\nbsphinxcodecellspacing}

\sphinxsetup{VerbatimColor={named}{nbsphinx-stderr}}
\sphinxsetup{VerbatimBorderColor={named}{nbsphinx-code-border}}
\begin{sphinxuseclass}{output_area}
\begin{sphinxuseclass}{stderr}


\begin{sphinxVerbatim}[commandchars=\\\{\}]
SCENARIOS COMPLETE: 100\%|█████████████████████████████████████████████████████████████| 21/21 [00:00<00:00, 309.77it/s]
\end{sphinxVerbatim}



\end{sphinxuseclass}
\end{sphinxuseclass}
}

\end{sphinxuseclass}
\begin{sphinxuseclass}{nboutput}
\begin{sphinxuseclass}{nblast}
{

\kern-\sphinxverbatimsmallskipamount\kern-\baselineskip
\kern+\FrameHeightAdjust\kern-\fboxrule
\vspace{\nbsphinxcodecellspacing}

\sphinxsetup{VerbatimColor={named}{white}}
\sphinxsetup{VerbatimBorderColor={named}{nbsphinx-code-border}}
\begin{sphinxuseclass}{output_area}
\begin{sphinxuseclass}{}


\begin{sphinxVerbatim}[commandchars=\\\{\}]
\llap{\color{nbsphinxout}[26]:\,\hspace{\fboxrule}\hspace{\fboxsep}}                               rate     cost  expected cost
ImportEE no\_v, t=0         0.000448  20125.0       901600.0
ImportEE inf\_v, t=0        0.000112  25125.0       281400.0
ImportWater no\_wat, t=0    0.000560  11125.0       623000.0
ImportSignal no\_sig, t=0   0.000056  20125.0       112700.0
MoveWater mech\_break, t=0  0.000336  15125.0       508200.0
\end{sphinxVerbatim}



\end{sphinxuseclass}
\end{sphinxuseclass}
}

\end{sphinxuseclass}
\end{sphinxuseclass}
\sphinxAtStartPar
To process these results, use \sphinxcode{\sphinxupquote{rd.process.hists(mdlhists)}}, which will calculate the degradation of the system over time in the model for all scenarios.

\begin{sphinxuseclass}{nbinput}
{
\sphinxsetup{VerbatimColor={named}{nbsphinx-code-bg}}
\sphinxsetup{VerbatimBorderColor={named}{nbsphinx-code-border}}
\begin{sphinxVerbatim}[commandchars=\\\{\}]
\llap{\color{nbsphinxin}[27]:\,\hspace{\fboxrule}\hspace{\fboxsep}}\PYG{n}{reshists}\PYG{p}{,} \PYG{n}{diffs}\PYG{p}{,} \PYG{n}{summaries} \PYG{o}{=} \PYG{n}{rd}\PYG{o}{.}\PYG{n}{process}\PYG{o}{.}\PYG{n}{hists}\PYG{p}{(}\PYG{n}{mdlhists}\PYG{p}{)}
\PYG{n}{fullfmea} \PYG{o}{=} \PYG{n}{rd}\PYG{o}{.}\PYG{n}{tabulate}\PYG{o}{.}\PYG{n}{fullfmea}\PYG{p}{(}\PYG{n}{endclasses}\PYG{p}{,} \PYG{n}{summaries}\PYG{p}{)}
\PYG{n}{fullfmea}\PYG{p}{[}\PYG{p}{:}\PYG{l+m+mi}{10}\PYG{p}{]}
\end{sphinxVerbatim}
}

\end{sphinxuseclass}
\begin{sphinxuseclass}{nboutput}
\begin{sphinxuseclass}{nblast}
{

\kern-\sphinxverbatimsmallskipamount\kern-\baselineskip
\kern+\FrameHeightAdjust\kern-\fboxrule
\vspace{\nbsphinxcodecellspacing}

\sphinxsetup{VerbatimColor={named}{white}}
\sphinxsetup{VerbatimBorderColor={named}{nbsphinx-code-border}}
\begin{sphinxuseclass}{output_area}
\begin{sphinxuseclass}{}


\begin{sphinxVerbatim}[commandchars=\\\{\}]
\llap{\color{nbsphinxout}[27]:\,\hspace{\fboxrule}\hspace{\fboxsep}}                                 degraded functions  \textbackslash{}
ImportEE no\_v, t=0                       [ImportEE]
ImportEE inf\_v, t=0                      [ImportEE]
ImportWater no\_wat, t=0               [ImportWater]
ImportSignal no\_sig, t=0             [ImportSignal]
MoveWater mech\_break, t=0               [MoveWater]
MoveWater short, t=0          [ImportEE, MoveWater]
ExportWater block, t=0     [MoveWater, ExportWater]
ImportEE no\_v, t=20                      [ImportEE]
ImportEE inf\_v, t=20                     [ImportEE]
ImportWater no\_wat, t=20              [ImportWater]

                                        degraded flows      rate     cost  \textbackslash{}
ImportEE no\_v, t=0                [EE\_1, Wat\_1, Wat\_2]  0.000448  20125.0
ImportEE inf\_v, t=0               [EE\_1, Wat\_1, Wat\_2]  0.000112  25125.0
ImportWater no\_wat, t=0           [EE\_1, Wat\_1, Wat\_2]   0.00056  11125.0
ImportSignal no\_sig, t=0   [EE\_1, Sig\_1, Wat\_1, Wat\_2]  0.000056  20125.0
MoveWater mech\_break, t=0         [EE\_1, Wat\_1, Wat\_2]  0.000336  15125.0
MoveWater short, t=0              [EE\_1, Wat\_1, Wat\_2]   0.00056  30125.0
ExportWater block, t=0            [EE\_1, Wat\_1, Wat\_2]   0.00056  20102.5
ImportEE no\_v, t=20               [EE\_1, Wat\_1, Wat\_2]  0.000448  16750.0
ImportEE inf\_v, t=20              [EE\_1, Wat\_1, Wat\_2]  0.000112  21750.0
ImportWater no\_wat, t=20          [EE\_1, Wat\_1, Wat\_2]   0.00056   7750.0

                          expected cost
ImportEE no\_v, t=0             901600.0
ImportEE inf\_v, t=0            281400.0
ImportWater no\_wat, t=0        623000.0
ImportSignal no\_sig, t=0       112700.0
MoveWater mech\_break, t=0      508200.0
MoveWater short, t=0          1687000.0
ExportWater block, t=0        1125740.0
ImportEE no\_v, t=20            750400.0
ImportEE inf\_v, t=20           243600.0
ImportWater no\_wat, t=20       434000.0
\end{sphinxVerbatim}



\end{sphinxuseclass}
\end{sphinxuseclass}
}

\end{sphinxuseclass}
\end{sphinxuseclass}

\subsubsection{Running a Fault Sampling Approach}
\label{\detokenize{example_pump/Pump_Example_Notebook:Running-a-Fault-Sampling-Approach}}
\sphinxAtStartPar
Note that only gives accurate results for costs and fault responses–in order to get an accurate idea of \sphinxstyleemphasis{expected cost}, we instead run an Approach, which develops an underlying probability model for faults. See below.

\begin{sphinxuseclass}{nbinput}
\begin{sphinxuseclass}{nblast}
{
\sphinxsetup{VerbatimColor={named}{nbsphinx-code-bg}}
\sphinxsetup{VerbatimBorderColor={named}{nbsphinx-code-border}}
\begin{sphinxVerbatim}[commandchars=\\\{\}]
\llap{\color{nbsphinxin}[28]:\,\hspace{\fboxrule}\hspace{\fboxsep}}\PYG{n}{app} \PYG{o}{=} \PYG{n}{SampleApproach}\PYG{p}{(}\PYG{n}{mdl}\PYG{p}{)} \PYG{c+c1}{\PYGZsh{}using default parameters\PYGZhy{}\PYGZhy{}note there are a variety of options for this appraoch}
\end{sphinxVerbatim}
}

\end{sphinxuseclass}
\end{sphinxuseclass}
\begin{sphinxuseclass}{nbinput}
{
\sphinxsetup{VerbatimColor={named}{nbsphinx-code-bg}}
\sphinxsetup{VerbatimBorderColor={named}{nbsphinx-code-border}}
\begin{sphinxVerbatim}[commandchars=\\\{\}]
\llap{\color{nbsphinxin}[29]:\,\hspace{\fboxrule}\hspace{\fboxsep}}\PYG{n}{endclasses}\PYG{p}{,} \PYG{n}{mdlhists}\PYG{o}{=}\PYG{n}{propagate}\PYG{o}{.}\PYG{n}{approach}\PYG{p}{(}\PYG{n}{mdl}\PYG{p}{,} \PYG{n}{app}\PYG{p}{,} \PYG{n}{staged}\PYG{o}{=}\PYG{k+kc}{True}\PYG{p}{)}
\PYG{n}{simplefmea} \PYG{o}{=} \PYG{n}{rd}\PYG{o}{.}\PYG{n}{tabulate}\PYG{o}{.}\PYG{n}{simplefmea}\PYG{p}{(}\PYG{n}{endclasses}\PYG{p}{)} \PYG{c+c1}{\PYGZsh{}note the costs are the same, but the rates and expected costs are not}
\PYG{n}{simplefmea}\PYG{p}{[}\PYG{p}{:}\PYG{l+m+mi}{5}\PYG{p}{]}
\end{sphinxVerbatim}
}

\end{sphinxuseclass}
\begin{sphinxuseclass}{nboutput}
{

\kern-\sphinxverbatimsmallskipamount\kern-\baselineskip
\kern+\FrameHeightAdjust\kern-\fboxrule
\vspace{\nbsphinxcodecellspacing}

\sphinxsetup{VerbatimColor={named}{nbsphinx-stderr}}
\sphinxsetup{VerbatimBorderColor={named}{nbsphinx-code-border}}
\begin{sphinxuseclass}{output_area}
\begin{sphinxuseclass}{stderr}


\begin{sphinxVerbatim}[commandchars=\\\{\}]
SCENARIOS COMPLETE: 100\%|█████████████████████████████████████████████████████████████| 17/17 [00:00<00:00, 347.81it/s]
\end{sphinxVerbatim}



\end{sphinxuseclass}
\end{sphinxuseclass}
}

\end{sphinxuseclass}
\begin{sphinxuseclass}{nboutput}
\begin{sphinxuseclass}{nblast}
{

\kern-\sphinxverbatimsmallskipamount\kern-\baselineskip
\kern+\FrameHeightAdjust\kern-\fboxrule
\vspace{\nbsphinxcodecellspacing}

\sphinxsetup{VerbatimColor={named}{white}}
\sphinxsetup{VerbatimBorderColor={named}{nbsphinx-code-border}}
\begin{sphinxuseclass}{output_area}
\begin{sphinxuseclass}{}


\begin{sphinxVerbatim}[commandchars=\\\{\}]
\llap{\color{nbsphinxout}[29]:\,\hspace{\fboxrule}\hspace{\fboxsep}}                                rate     cost  expected cost
ImportEE no\_v, t=27         0.000360  15175.0  546300.000000
ImportEE inf\_v, t=27        0.000090  20175.0  181575.000000
ImportWater no\_wat, t=27    0.000150   6175.0   92625.000000
ImportSignal no\_sig, t=27   0.000013  15175.0   19510.714286
MoveWater mech\_break, t=27  0.000231  10175.0  235478.571429
\end{sphinxVerbatim}



\end{sphinxuseclass}
\end{sphinxuseclass}
}

\end{sphinxuseclass}
\end{sphinxuseclass}
\sphinxAtStartPar
We can now summarize the risks of faults over the operational phases and overall.

\begin{sphinxuseclass}{nbinput}
{
\sphinxsetup{VerbatimColor={named}{nbsphinx-code-bg}}
\sphinxsetup{VerbatimBorderColor={named}{nbsphinx-code-border}}
\begin{sphinxVerbatim}[commandchars=\\\{\}]
\llap{\color{nbsphinxin}[30]:\,\hspace{\fboxrule}\hspace{\fboxsep}}\PYG{n}{phasefmea} \PYG{o}{=} \PYG{n}{rd}\PYG{o}{.}\PYG{n}{tabulate}\PYG{o}{.}\PYG{n}{phasefmea}\PYG{p}{(}\PYG{n}{endclasses}\PYG{p}{,} \PYG{n}{app}\PYG{p}{)}
\PYG{n}{phasefmea}
\end{sphinxVerbatim}
}

\end{sphinxuseclass}
\begin{sphinxuseclass}{nboutput}
\begin{sphinxuseclass}{nblast}
{

\kern-\sphinxverbatimsmallskipamount\kern-\baselineskip
\kern+\FrameHeightAdjust\kern-\fboxrule
\vspace{\nbsphinxcodecellspacing}

\sphinxsetup{VerbatimColor={named}{white}}
\sphinxsetup{VerbatimBorderColor={named}{nbsphinx-code-border}}
\begin{sphinxuseclass}{output_area}
\begin{sphinxuseclass}{}


\begin{sphinxVerbatim}[commandchars=\\\{\}]
\llap{\color{nbsphinxout}[30]:\,\hspace{\fboxrule}\hspace{\fboxsep}}                                             rate     cost  expected cost
(ImportEE, no\_v)        (global, on)     0.000360  15175.0  546300.000000
(ImportEE, inf\_v)       (global, on)     0.000090  20175.0  181575.000000
(ImportWater, no\_wat)   (global, on)     0.000150   6175.0   92625.000000
(ImportSignal, no\_sig)  (global, on)     0.000013  15175.0   19510.714286
(MoveWater, mech\_break) (global, on)     0.000231  10175.0  235478.571429
(MoveWater, short)      (global, on)     0.000129  25175.0  323678.571429
(ExportWater, block)    (global, on)     0.000129  15152.5  194817.857143
(ImportWater, no\_wat)   (global, start)  0.000017  11125.0   18541.666667
(ImportSignal, no\_sig)  (global, start)  0.000002  20125.0    4312.500000
(MoveWater, mech\_break) (global, start)  0.000002  15125.0    3241.071429
(MoveWater, short)      (global, start)  0.000021  30125.0   64553.571429
(ExportWater, block)    (global, start)  0.000021  20102.5   43076.785714
(ImportWater, no\_wat)   (global, end)    0.000017   1000.0    1666.666667
(ImportSignal, no\_sig)  (global, end)    0.000001  10000.0    1428.571429
(MoveWater, mech\_break) (global, end)    0.000002   5000.0    1071.428571
(MoveWater, short)      (global, end)    0.000014  10000.0   14285.714286
(ExportWater, block)    (global, end)    0.000014   5000.0    7142.857143
\end{sphinxVerbatim}



\end{sphinxuseclass}
\end{sphinxuseclass}
}

\end{sphinxuseclass}
\end{sphinxuseclass}
\begin{sphinxuseclass}{nbinput}
{
\sphinxsetup{VerbatimColor={named}{nbsphinx-code-bg}}
\sphinxsetup{VerbatimBorderColor={named}{nbsphinx-code-border}}
\begin{sphinxVerbatim}[commandchars=\\\{\}]
\llap{\color{nbsphinxin}[31]:\,\hspace{\fboxrule}\hspace{\fboxsep}}\PYG{n}{summfmea} \PYG{o}{=} \PYG{n}{rd}\PYG{o}{.}\PYG{n}{tabulate}\PYG{o}{.}\PYG{n}{summfmea}\PYG{p}{(}\PYG{n}{endclasses}\PYG{p}{,} \PYG{n}{app}\PYG{p}{)}
\PYG{n}{summfmea}
\end{sphinxVerbatim}
}

\end{sphinxuseclass}
\begin{sphinxuseclass}{nboutput}
\begin{sphinxuseclass}{nblast}
{

\kern-\sphinxverbatimsmallskipamount\kern-\baselineskip
\kern+\FrameHeightAdjust\kern-\fboxrule
\vspace{\nbsphinxcodecellspacing}

\sphinxsetup{VerbatimColor={named}{white}}
\sphinxsetup{VerbatimBorderColor={named}{nbsphinx-code-border}}
\begin{sphinxuseclass}{output_area}
\begin{sphinxuseclass}{}


\begin{sphinxVerbatim}[commandchars=\\\{\}]
\llap{\color{nbsphinxout}[31]:\,\hspace{\fboxrule}\hspace{\fboxsep}}                             rate          cost  expected cost
ImportEE     no\_v        0.000360  15175.000000  546300.000000
             inf\_v       0.000090  20175.000000  181575.000000
ImportWater  no\_wat      0.000183   6100.000000  112833.333333
ImportSignal no\_sig      0.000016  15100.000000   25251.785714
MoveWater    mech\_break  0.000236  10100.000000  239791.071429
             short       0.000164  21766.666667  402517.857143
ExportWater  block       0.000164  13418.333333  245037.500000
\end{sphinxVerbatim}



\end{sphinxuseclass}
\end{sphinxuseclass}
}

\end{sphinxuseclass}
\end{sphinxuseclass}
\sphinxAtStartPar
To visualize the results, the histories need to be processed.

\begin{sphinxuseclass}{nbinput}
\begin{sphinxuseclass}{nblast}
{
\sphinxsetup{VerbatimColor={named}{nbsphinx-code-bg}}
\sphinxsetup{VerbatimBorderColor={named}{nbsphinx-code-border}}
\begin{sphinxVerbatim}[commandchars=\\\{\}]
\llap{\color{nbsphinxin}[32]:\,\hspace{\fboxrule}\hspace{\fboxsep}}\PYG{n}{reshists}\PYG{p}{,} \PYG{n}{diffs}\PYG{p}{,} \PYG{n}{summaries} \PYG{o}{=} \PYG{n}{rd}\PYG{o}{.}\PYG{n}{process}\PYG{o}{.}\PYG{n}{hists}\PYG{p}{(}\PYG{n}{mdlhists}\PYG{p}{)}
\end{sphinxVerbatim}
}

\end{sphinxuseclass}
\end{sphinxuseclass}
\sphinxAtStartPar
Now that these results have been processed, we can use them to visualize the expected resilience of the model to the fault scenarios. Here we will use the average percentage of time degraded

\begin{sphinxuseclass}{nbinput}
{
\sphinxsetup{VerbatimColor={named}{nbsphinx-code-bg}}
\sphinxsetup{VerbatimBorderColor={named}{nbsphinx-code-border}}
\begin{sphinxVerbatim}[commandchars=\\\{\}]
\llap{\color{nbsphinxin}[33]:\,\hspace{\fboxrule}\hspace{\fboxsep}}\PYG{n}{heatmap1} \PYG{o}{=} \PYG{n}{rd}\PYG{o}{.}\PYG{n}{process}\PYG{o}{.}\PYG{n}{avg\PYGZus{}degtime\PYGZus{}heatmap}\PYG{p}{(}\PYG{n}{reshists}\PYG{p}{)}
\PYG{n}{rd}\PYG{o}{.}\PYG{n}{tabulate}\PYG{o}{.}\PYG{n}{dicttab}\PYG{p}{(}\PYG{n}{heatmap1}\PYG{p}{)}
\end{sphinxVerbatim}
}

\end{sphinxuseclass}
\begin{sphinxuseclass}{nboutput}
\begin{sphinxuseclass}{nblast}
{

\kern-\sphinxverbatimsmallskipamount\kern-\baselineskip
\kern+\FrameHeightAdjust\kern-\fboxrule
\vspace{\nbsphinxcodecellspacing}

\sphinxsetup{VerbatimColor={named}{white}}
\sphinxsetup{VerbatimBorderColor={named}{nbsphinx-code-border}}
\begin{sphinxuseclass}{output_area}
\begin{sphinxuseclass}{}


\begin{sphinxVerbatim}[commandchars=\\\{\}]
\llap{\color{nbsphinxout}[33]:\,\hspace{\fboxrule}\hspace{\fboxsep}}   ImportEE  ImportWater  ImportSignal  MoveWater  ExportWater      EE\_1  \textbackslash{}
0  0.144958     0.091387      0.091387   0.245798     0.091387  0.430672

      Sig\_1    Wat\_1    Wat\_2
0  0.071429  0.42542  0.42542
\end{sphinxVerbatim}



\end{sphinxuseclass}
\end{sphinxuseclass}
}

\end{sphinxuseclass}
\end{sphinxuseclass}
\begin{sphinxuseclass}{nbinput}
{
\sphinxsetup{VerbatimColor={named}{nbsphinx-code-bg}}
\sphinxsetup{VerbatimBorderColor={named}{nbsphinx-code-border}}
\begin{sphinxVerbatim}[commandchars=\\\{\}]
\llap{\color{nbsphinxin}[34]:\,\hspace{\fboxrule}\hspace{\fboxsep}}\PYG{n}{rd}\PYG{o}{.}\PYG{n}{graph}\PYG{o}{.}\PYG{n}{show}\PYG{p}{(}\PYG{n}{mdl}\PYG{o}{.}\PYG{n}{bipartite}\PYG{p}{,}  \PYG{n}{heatmap}\PYG{o}{=}\PYG{n}{heatmap1}\PYG{p}{,} \PYG{n}{scale}\PYG{o}{=}\PYG{l+m+mi}{2}\PYG{p}{)}
\end{sphinxVerbatim}
}

\end{sphinxuseclass}
\begin{sphinxuseclass}{nboutput}
{

\kern-\sphinxverbatimsmallskipamount\kern-\baselineskip
\kern+\FrameHeightAdjust\kern-\fboxrule
\vspace{\nbsphinxcodecellspacing}

\sphinxsetup{VerbatimColor={named}{white}}
\sphinxsetup{VerbatimBorderColor={named}{nbsphinx-code-border}}
\begin{sphinxuseclass}{output_area}
\begin{sphinxuseclass}{}


\begin{sphinxVerbatim}[commandchars=\\\{\}]
\llap{\color{nbsphinxout}[34]:\,\hspace{\fboxrule}\hspace{\fboxsep}}(<Figure size 432x288 with 1 Axes>, <AxesSubplot:>)
\end{sphinxVerbatim}



\end{sphinxuseclass}
\end{sphinxuseclass}
}

\end{sphinxuseclass}
\begin{sphinxuseclass}{nboutput}
\begin{sphinxuseclass}{nblast}
\hrule height -\fboxrule\relax
\vspace{\nbsphinxcodecellspacing}

\makeatletter\setbox\nbsphinxpromptbox\box\voidb@x\makeatother

\begin{nbsphinxfancyoutput}

\begin{sphinxuseclass}{output_area}
\begin{sphinxuseclass}{}
\noindent\sphinxincludegraphics[width=349\sphinxpxdimen,height=231\sphinxpxdimen]{{example_pump_Pump_Example_Notebook_63_1}.png}

\end{sphinxuseclass}
\end{sphinxuseclass}
\end{nbsphinxfancyoutput}

\end{sphinxuseclass}
\end{sphinxuseclass}
\sphinxAtStartPar
Using this table (and the visualization) we would conclude that in our set of fault scenarios the Wat\_1, Wat\_2, and EE\_1 flows degrade as often as each other. However, this does not tell us which flows are most likely to be degraded based on our simulations. In order to determine that, rate information must be used to get the \sphinxstyleemphasis{expected} degradation of each node.

\begin{sphinxuseclass}{nbinput}
{
\sphinxsetup{VerbatimColor={named}{nbsphinx-code-bg}}
\sphinxsetup{VerbatimBorderColor={named}{nbsphinx-code-border}}
\begin{sphinxVerbatim}[commandchars=\\\{\}]
\llap{\color{nbsphinxin}[35]:\,\hspace{\fboxrule}\hspace{\fboxsep}}\PYG{n}{heatmap2} \PYG{o}{=} \PYG{n}{rd}\PYG{o}{.}\PYG{n}{process}\PYG{o}{.}\PYG{n}{exp\PYGZus{}degtime\PYGZus{}heatmap}\PYG{p}{(}\PYG{n}{reshists}\PYG{p}{,} \PYG{n}{endclasses}\PYG{p}{)}
\PYG{n}{rd}\PYG{o}{.}\PYG{n}{tabulate}\PYG{o}{.}\PYG{n}{dicttab}\PYG{p}{(}\PYG{n}{heatmap2}\PYG{p}{)}
\end{sphinxVerbatim}
}

\end{sphinxuseclass}
\begin{sphinxuseclass}{nboutput}
\begin{sphinxuseclass}{nblast}
{

\kern-\sphinxverbatimsmallskipamount\kern-\baselineskip
\kern+\FrameHeightAdjust\kern-\fboxrule
\vspace{\nbsphinxcodecellspacing}

\sphinxsetup{VerbatimColor={named}{white}}
\sphinxsetup{VerbatimBorderColor={named}{nbsphinx-code-border}}
\begin{sphinxuseclass}{output_area}
\begin{sphinxuseclass}{}


\begin{sphinxVerbatim}[commandchars=\\\{\}]
\llap{\color{nbsphinxout}[35]:\,\hspace{\fboxrule}\hspace{\fboxsep}}   ImportEE  ImportWater  ImportSignal  MoveWater  ExportWater      EE\_1  \textbackslash{}
0  0.000319     0.000095      0.000009    0.00027     0.000088  0.000568

      Sig\_1     Wat\_1     Wat\_2
0  0.000007  0.000524  0.000522
\end{sphinxVerbatim}



\end{sphinxuseclass}
\end{sphinxuseclass}
}

\end{sphinxuseclass}
\end{sphinxuseclass}
\sphinxAtStartPar
The results here are roughly the same, though. The expected degradation of the EE\_1 flow is less here than the Wat\_1 and Wat\_2 flows.

\begin{sphinxuseclass}{nbinput}
{
\sphinxsetup{VerbatimColor={named}{nbsphinx-code-bg}}
\sphinxsetup{VerbatimBorderColor={named}{nbsphinx-code-border}}
\begin{sphinxVerbatim}[commandchars=\\\{\}]
\llap{\color{nbsphinxin}[36]:\,\hspace{\fboxrule}\hspace{\fboxsep}}\PYG{n}{rd}\PYG{o}{.}\PYG{n}{graph}\PYG{o}{.}\PYG{n}{show}\PYG{p}{(}\PYG{n}{mdl}\PYG{o}{.}\PYG{n}{bipartite}\PYG{p}{,} \PYG{n}{heatmap}\PYG{o}{=}\PYG{n}{heatmap2}\PYG{p}{,} \PYG{n}{scale}\PYG{o}{=}\PYG{l+m+mi}{2}\PYG{p}{)}
\end{sphinxVerbatim}
}

\end{sphinxuseclass}
\begin{sphinxuseclass}{nboutput}
{

\kern-\sphinxverbatimsmallskipamount\kern-\baselineskip
\kern+\FrameHeightAdjust\kern-\fboxrule
\vspace{\nbsphinxcodecellspacing}

\sphinxsetup{VerbatimColor={named}{white}}
\sphinxsetup{VerbatimBorderColor={named}{nbsphinx-code-border}}
\begin{sphinxuseclass}{output_area}
\begin{sphinxuseclass}{}


\begin{sphinxVerbatim}[commandchars=\\\{\}]
\llap{\color{nbsphinxout}[36]:\,\hspace{\fboxrule}\hspace{\fboxsep}}(<Figure size 432x288 with 1 Axes>, <AxesSubplot:>)
\end{sphinxVerbatim}



\end{sphinxuseclass}
\end{sphinxuseclass}
}

\end{sphinxuseclass}
\begin{sphinxuseclass}{nboutput}
\begin{sphinxuseclass}{nblast}
\hrule height -\fboxrule\relax
\vspace{\nbsphinxcodecellspacing}

\makeatletter\setbox\nbsphinxpromptbox\box\voidb@x\makeatother

\begin{nbsphinxfancyoutput}

\begin{sphinxuseclass}{output_area}
\begin{sphinxuseclass}{}
\noindent\sphinxincludegraphics[width=349\sphinxpxdimen,height=231\sphinxpxdimen]{{example_pump_Pump_Example_Notebook_67_1}.png}

\end{sphinxuseclass}
\end{sphinxuseclass}
\end{nbsphinxfancyoutput}

\end{sphinxuseclass}
\end{sphinxuseclass}
\sphinxAtStartPar
We can do the same looking at the maximum number of faults occuring in each scenario.

\begin{sphinxuseclass}{nbinput}
{
\sphinxsetup{VerbatimColor={named}{nbsphinx-code-bg}}
\sphinxsetup{VerbatimBorderColor={named}{nbsphinx-code-border}}
\begin{sphinxVerbatim}[commandchars=\\\{\}]
\llap{\color{nbsphinxin}[37]:\,\hspace{\fboxrule}\hspace{\fboxsep}}\PYG{n}{heatmap3}\PYG{o}{=} \PYG{n}{rd}\PYG{o}{.}\PYG{n}{process}\PYG{o}{.}\PYG{n}{faults\PYGZus{}heatmap}\PYG{p}{(}\PYG{n}{reshists}\PYG{p}{)}
\PYG{n}{rd}\PYG{o}{.}\PYG{n}{tabulate}\PYG{o}{.}\PYG{n}{dicttab}\PYG{p}{(}\PYG{n}{heatmap3}\PYG{p}{)}
\end{sphinxVerbatim}
}

\end{sphinxuseclass}
\begin{sphinxuseclass}{nboutput}
\begin{sphinxuseclass}{nblast}
{

\kern-\sphinxverbatimsmallskipamount\kern-\baselineskip
\kern+\FrameHeightAdjust\kern-\fboxrule
\vspace{\nbsphinxcodecellspacing}

\sphinxsetup{VerbatimColor={named}{white}}
\sphinxsetup{VerbatimBorderColor={named}{nbsphinx-code-border}}
\begin{sphinxuseclass}{output_area}
\begin{sphinxuseclass}{}


\begin{sphinxVerbatim}[commandchars=\\\{\}]
\llap{\color{nbsphinxout}[37]:\,\hspace{\fboxrule}\hspace{\fboxsep}}   ImportEE  ImportWater  ImportSignal  MoveWater  ExportWater
0  0.294118     0.176471      0.176471   0.470588     0.176471
\end{sphinxVerbatim}



\end{sphinxuseclass}
\end{sphinxuseclass}
}

\end{sphinxuseclass}
\end{sphinxuseclass}
\begin{sphinxuseclass}{nbinput}
{
\sphinxsetup{VerbatimColor={named}{nbsphinx-code-bg}}
\sphinxsetup{VerbatimBorderColor={named}{nbsphinx-code-border}}
\begin{sphinxVerbatim}[commandchars=\\\{\}]
\llap{\color{nbsphinxin}[38]:\,\hspace{\fboxrule}\hspace{\fboxsep}}\PYG{n}{rd}\PYG{o}{.}\PYG{n}{graph}\PYG{o}{.}\PYG{n}{show}\PYG{p}{(}\PYG{n}{mdl}\PYG{o}{.}\PYG{n}{graph}\PYG{p}{,} \PYG{n}{heatmap}\PYG{o}{=}\PYG{n}{heatmap3}\PYG{p}{,} \PYG{n}{gtype} \PYG{o}{=} \PYG{l+s+s1}{\PYGZsq{}}\PYG{l+s+s1}{normal}\PYG{l+s+s1}{\PYGZsq{}}\PYG{p}{)}
\end{sphinxVerbatim}
}

\end{sphinxuseclass}
\begin{sphinxuseclass}{nboutput}
{

\kern-\sphinxverbatimsmallskipamount\kern-\baselineskip
\kern+\FrameHeightAdjust\kern-\fboxrule
\vspace{\nbsphinxcodecellspacing}

\sphinxsetup{VerbatimColor={named}{white}}
\sphinxsetup{VerbatimBorderColor={named}{nbsphinx-code-border}}
\begin{sphinxuseclass}{output_area}
\begin{sphinxuseclass}{}


\begin{sphinxVerbatim}[commandchars=\\\{\}]
\llap{\color{nbsphinxout}[38]:\,\hspace{\fboxrule}\hspace{\fboxsep}}(<Figure size 432x288 with 1 Axes>, <AxesSubplot:>)
\end{sphinxVerbatim}



\end{sphinxuseclass}
\end{sphinxuseclass}
}

\end{sphinxuseclass}
\begin{sphinxuseclass}{nboutput}
\begin{sphinxuseclass}{nblast}
\hrule height -\fboxrule\relax
\vspace{\nbsphinxcodecellspacing}

\makeatletter\setbox\nbsphinxpromptbox\box\voidb@x\makeatother

\begin{nbsphinxfancyoutput}

\begin{sphinxuseclass}{output_area}
\begin{sphinxuseclass}{}
\noindent\sphinxincludegraphics[width=349\sphinxpxdimen,height=231\sphinxpxdimen]{{example_pump_Pump_Example_Notebook_70_1}.png}

\end{sphinxuseclass}
\end{sphinxuseclass}
\end{nbsphinxfancyoutput}

\end{sphinxuseclass}
\end{sphinxuseclass}
\sphinxAtStartPar
MoveWater and ImportEE most commonly have a high number of faults in the list of scenarios. Again, we may be more interested in the expected number, however.

\begin{sphinxuseclass}{nbinput}
{
\sphinxsetup{VerbatimColor={named}{nbsphinx-code-bg}}
\sphinxsetup{VerbatimBorderColor={named}{nbsphinx-code-border}}
\begin{sphinxVerbatim}[commandchars=\\\{\}]
\llap{\color{nbsphinxin}[39]:\,\hspace{\fboxrule}\hspace{\fboxsep}}\PYG{n}{heatmap4}\PYG{o}{=} \PYG{n}{rd}\PYG{o}{.}\PYG{n}{process}\PYG{o}{.}\PYG{n}{exp\PYGZus{}faults\PYGZus{}heatmap}\PYG{p}{(}\PYG{n}{reshists}\PYG{p}{,} \PYG{n}{endclasses}\PYG{p}{)}
\PYG{n}{rd}\PYG{o}{.}\PYG{n}{tabulate}\PYG{o}{.}\PYG{n}{dicttab}\PYG{p}{(}\PYG{n}{heatmap4}\PYG{p}{)}
\end{sphinxVerbatim}
}

\end{sphinxuseclass}
\begin{sphinxuseclass}{nboutput}
\begin{sphinxuseclass}{nblast}
{

\kern-\sphinxverbatimsmallskipamount\kern-\baselineskip
\kern+\FrameHeightAdjust\kern-\fboxrule
\vspace{\nbsphinxcodecellspacing}

\sphinxsetup{VerbatimColor={named}{white}}
\sphinxsetup{VerbatimBorderColor={named}{nbsphinx-code-border}}
\begin{sphinxuseclass}{output_area}
\begin{sphinxuseclass}{}


\begin{sphinxVerbatim}[commandchars=\\\{\}]
\llap{\color{nbsphinxout}[39]:\,\hspace{\fboxrule}\hspace{\fboxsep}}   ImportEE  ImportWater  ImportSignal  MoveWater  ExportWater
0  0.000041     0.000011  9.663866e-07   0.000032      0.00001
\end{sphinxVerbatim}



\end{sphinxuseclass}
\end{sphinxuseclass}
}

\end{sphinxuseclass}
\end{sphinxuseclass}
\begin{sphinxuseclass}{nbinput}
{
\sphinxsetup{VerbatimColor={named}{nbsphinx-code-bg}}
\sphinxsetup{VerbatimBorderColor={named}{nbsphinx-code-border}}
\begin{sphinxVerbatim}[commandchars=\\\{\}]
\llap{\color{nbsphinxin}[40]:\,\hspace{\fboxrule}\hspace{\fboxsep}}\PYG{n}{rd}\PYG{o}{.}\PYG{n}{graph}\PYG{o}{.}\PYG{n}{show}\PYG{p}{(}\PYG{n}{mdl}\PYG{o}{.}\PYG{n}{graph}\PYG{p}{,} \PYG{n}{heatmap}\PYG{o}{=}\PYG{n}{heatmap4}\PYG{p}{,} \PYG{n}{gtype}\PYG{o}{=}\PYG{l+s+s1}{\PYGZsq{}}\PYG{l+s+s1}{normal}\PYG{l+s+s1}{\PYGZsq{}}\PYG{p}{)}
\end{sphinxVerbatim}
}

\end{sphinxuseclass}
\begin{sphinxuseclass}{nboutput}
{

\kern-\sphinxverbatimsmallskipamount\kern-\baselineskip
\kern+\FrameHeightAdjust\kern-\fboxrule
\vspace{\nbsphinxcodecellspacing}

\sphinxsetup{VerbatimColor={named}{white}}
\sphinxsetup{VerbatimBorderColor={named}{nbsphinx-code-border}}
\begin{sphinxuseclass}{output_area}
\begin{sphinxuseclass}{}


\begin{sphinxVerbatim}[commandchars=\\\{\}]
\llap{\color{nbsphinxout}[40]:\,\hspace{\fboxrule}\hspace{\fboxsep}}(<Figure size 432x288 with 1 Axes>, <AxesSubplot:>)
\end{sphinxVerbatim}



\end{sphinxuseclass}
\end{sphinxuseclass}
}

\end{sphinxuseclass}
\begin{sphinxuseclass}{nboutput}
\begin{sphinxuseclass}{nblast}
\hrule height -\fboxrule\relax
\vspace{\nbsphinxcodecellspacing}

\makeatletter\setbox\nbsphinxpromptbox\box\voidb@x\makeatother

\begin{nbsphinxfancyoutput}

\begin{sphinxuseclass}{output_area}
\begin{sphinxuseclass}{}
\noindent\sphinxincludegraphics[width=349\sphinxpxdimen,height=231\sphinxpxdimen]{{example_pump_Pump_Example_Notebook_73_1}.png}

\end{sphinxuseclass}
\end{sphinxuseclass}
\end{nbsphinxfancyoutput}

\end{sphinxuseclass}
\end{sphinxuseclass}
\sphinxAtStartPar
So even though ImportEE has faults very commonly in the set of scenarios, when weighted by the occurence of scenarios, the MoveWater function has the most faults.


\paragraph{Save/Load}
\label{\detokenize{example_pump/Pump_Example_Notebook:Save/Load}}
\sphinxAtStartPar
Since fmdtools simulations of detailed models over a large number of scenarios can take a long time, it can be helpful to save and load the results. The easiest way to do this is to dump the entire workspace using the dill package.

\begin{sphinxuseclass}{nbinput}
\begin{sphinxuseclass}{nblast}
{
\sphinxsetup{VerbatimColor={named}{nbsphinx-code-bg}}
\sphinxsetup{VerbatimBorderColor={named}{nbsphinx-code-border}}
\begin{sphinxVerbatim}[commandchars=\\\{\}]
\llap{\color{nbsphinxin}[41]:\,\hspace{\fboxrule}\hspace{\fboxsep}}\PYG{k+kn}{import} \PYG{n+nn}{dill}
\end{sphinxVerbatim}
}

\end{sphinxuseclass}
\end{sphinxuseclass}
\begin{sphinxuseclass}{nbinput}
{
\sphinxsetup{VerbatimColor={named}{nbsphinx-code-bg}}
\sphinxsetup{VerbatimBorderColor={named}{nbsphinx-code-border}}
\begin{sphinxVerbatim}[commandchars=\\\{\}]
\llap{\color{nbsphinxin}[42]:\,\hspace{\fboxrule}\hspace{\fboxsep}}\PYG{n}{check\PYGZus{}model\PYGZus{}pickleability}\PYG{p}{(}\PYG{n}{mdl}\PYG{p}{)}
\end{sphinxVerbatim}
}

\end{sphinxuseclass}
\begin{sphinxuseclass}{nboutput}
\begin{sphinxuseclass}{nblast}
{

\kern-\sphinxverbatimsmallskipamount\kern-\baselineskip
\kern+\FrameHeightAdjust\kern-\fboxrule
\vspace{\nbsphinxcodecellspacing}

\sphinxsetup{VerbatimColor={named}{white}}
\sphinxsetup{VerbatimBorderColor={named}{nbsphinx-code-border}}
\begin{sphinxuseclass}{output_area}
\begin{sphinxuseclass}{}


\begin{sphinxVerbatim}[commandchars=\\\{\}]
The object is pickleable
\end{sphinxVerbatim}



\end{sphinxuseclass}
\end{sphinxuseclass}
}

\end{sphinxuseclass}
\end{sphinxuseclass}
\sphinxAtStartPar
This saves the session to a file:

\begin{sphinxuseclass}{nbinput}
\begin{sphinxuseclass}{nblast}
{
\sphinxsetup{VerbatimColor={named}{nbsphinx-code-bg}}
\sphinxsetup{VerbatimBorderColor={named}{nbsphinx-code-border}}
\begin{sphinxVerbatim}[commandchars=\\\{\}]
\llap{\color{nbsphinxin}[44]:\,\hspace{\fboxrule}\hspace{\fboxsep}}\PYG{n}{dill}\PYG{o}{.}\PYG{n}{dump\PYGZus{}session}\PYG{p}{(}\PYG{l+s+s2}{\PYGZdq{}}\PYG{l+s+s2}{Pump\PYGZus{}Example\PYGZus{}Notebook.pkl}\PYG{l+s+s2}{\PYGZdq{}}\PYG{p}{)}
\end{sphinxVerbatim}
}

\end{sphinxuseclass}
\end{sphinxuseclass}
\sphinxAtStartPar
Now we will simulate starting a new session.

\begin{sphinxuseclass}{nbinput}
{
\sphinxsetup{VerbatimColor={named}{nbsphinx-code-bg}}
\sphinxsetup{VerbatimBorderColor={named}{nbsphinx-code-border}}
\begin{sphinxVerbatim}[commandchars=\\\{\}]
\llap{\color{nbsphinxin}[45]:\,\hspace{\fboxrule}\hspace{\fboxsep}}\PYG{o}{\PYGZpc{}}\PYG{k}{reset}
\end{sphinxVerbatim}
}

\end{sphinxuseclass}
\begin{sphinxuseclass}{nboutput}
\begin{sphinxuseclass}{nblast}
{

\kern-\sphinxverbatimsmallskipamount\kern-\baselineskip
\kern+\FrameHeightAdjust\kern-\fboxrule
\vspace{\nbsphinxcodecellspacing}

\sphinxsetup{VerbatimColor={named}{white}}
\sphinxsetup{VerbatimBorderColor={named}{nbsphinx-code-border}}
\begin{sphinxuseclass}{output_area}
\begin{sphinxuseclass}{}


\begin{sphinxVerbatim}[commandchars=\\\{\}]
Once deleted, variables cannot be recovered. Proceed (y/[n])? y
\end{sphinxVerbatim}



\end{sphinxuseclass}
\end{sphinxuseclass}
}

\end{sphinxuseclass}
\end{sphinxuseclass}
\begin{sphinxuseclass}{nbinput}
{
\sphinxsetup{VerbatimColor={named}{nbsphinx-code-bg}}
\sphinxsetup{VerbatimBorderColor={named}{nbsphinx-code-border}}
\begin{sphinxVerbatim}[commandchars=\\\{\}]
\llap{\color{nbsphinxin}[46]:\,\hspace{\fboxrule}\hspace{\fboxsep}}\PYG{n}{summfmea}
\end{sphinxVerbatim}
}

\end{sphinxuseclass}
\begin{sphinxuseclass}{nboutput}
\begin{sphinxuseclass}{nblast}
{

\kern-\sphinxverbatimsmallskipamount\kern-\baselineskip
\kern+\FrameHeightAdjust\kern-\fboxrule
\vspace{\nbsphinxcodecellspacing}

\sphinxsetup{VerbatimColor={named}{white}}
\sphinxsetup{VerbatimBorderColor={named}{nbsphinx-code-border}}
\begin{sphinxuseclass}{output_area}
\begin{sphinxuseclass}{}


\begin{sphinxVerbatim}[commandchars=\\\{\}]
\textcolor{ansi-red-intense}{\textbf{---------------------------------------------------------------------------}}
\textcolor{ansi-red-intense}{\textbf{NameError}}                                 Traceback (most recent call last)
\textcolor{ansi-green-intense}{\textbf{\textasciitilde{}\textbackslash{}AppData\textbackslash{}Local\textbackslash{}Temp\textbackslash{}2/ipykernel\_3012/133845665.py}} in \textcolor{ansi-cyan}{<module>}
\textcolor{ansi-green-intense}{\textbf{----> 1}}\textcolor{ansi-yellow-intense}{\textbf{ }}summfmea

\textcolor{ansi-red-intense}{\textbf{NameError}}: name 'summfmea' is not defined
\end{sphinxVerbatim}



\end{sphinxuseclass}
\end{sphinxuseclass}
}

\end{sphinxuseclass}
\end{sphinxuseclass}
\sphinxAtStartPar
As shown, the variables are now cleared. We can bring them back using load\_session.

\begin{sphinxuseclass}{nbinput}
\begin{sphinxuseclass}{nblast}
{
\sphinxsetup{VerbatimColor={named}{nbsphinx-code-bg}}
\sphinxsetup{VerbatimBorderColor={named}{nbsphinx-code-border}}
\begin{sphinxVerbatim}[commandchars=\\\{\}]
\llap{\color{nbsphinxin}[47]:\,\hspace{\fboxrule}\hspace{\fboxsep}}\PYG{k+kn}{import} \PYG{n+nn}{dill}
\PYG{n}{dill}\PYG{o}{.}\PYG{n}{load\PYGZus{}session}\PYG{p}{(}\PYG{l+s+s2}{\PYGZdq{}}\PYG{l+s+s2}{Pump\PYGZus{}Example\PYGZus{}Notebook.pkl}\PYG{l+s+s2}{\PYGZdq{}}\PYG{p}{)}
\end{sphinxVerbatim}
}

\end{sphinxuseclass}
\end{sphinxuseclass}
\begin{sphinxuseclass}{nbinput}
{
\sphinxsetup{VerbatimColor={named}{nbsphinx-code-bg}}
\sphinxsetup{VerbatimBorderColor={named}{nbsphinx-code-border}}
\begin{sphinxVerbatim}[commandchars=\\\{\}]
\llap{\color{nbsphinxin}[48]:\,\hspace{\fboxrule}\hspace{\fboxsep}}\PYG{n}{summfmea}
\end{sphinxVerbatim}
}

\end{sphinxuseclass}
\begin{sphinxuseclass}{nboutput}
\begin{sphinxuseclass}{nblast}
{

\kern-\sphinxverbatimsmallskipamount\kern-\baselineskip
\kern+\FrameHeightAdjust\kern-\fboxrule
\vspace{\nbsphinxcodecellspacing}

\sphinxsetup{VerbatimColor={named}{white}}
\sphinxsetup{VerbatimBorderColor={named}{nbsphinx-code-border}}
\begin{sphinxuseclass}{output_area}
\begin{sphinxuseclass}{}


\begin{sphinxVerbatim}[commandchars=\\\{\}]
\llap{\color{nbsphinxout}[48]:\,\hspace{\fboxrule}\hspace{\fboxsep}}                             rate          cost  expected cost
ImportEE     no\_v        0.000360  15175.000000  546300.000000
             inf\_v       0.000090  20175.000000  181575.000000
ImportWater  no\_wat      0.000183   6100.000000  112833.333333
ImportSignal no\_sig      0.000016  15100.000000   25251.785714
MoveWater    mech\_break  0.000236  10100.000000  239791.071429
             short       0.000164  21766.666667  402517.857143
ExportWater  block       0.000164  13418.333333  245037.500000
\end{sphinxVerbatim}



\end{sphinxuseclass}
\end{sphinxuseclass}
}

\end{sphinxuseclass}
\end{sphinxuseclass}
\begin{sphinxuseclass}{nbinput}
{
\sphinxsetup{VerbatimColor={named}{nbsphinx-code-bg}}
\sphinxsetup{VerbatimBorderColor={named}{nbsphinx-code-border}}
\begin{sphinxVerbatim}[commandchars=\\\{\}]
\llap{\color{nbsphinxin}[49]:\,\hspace{\fboxrule}\hspace{\fboxsep}}\PYG{n}{rd}\PYG{o}{.}\PYG{n}{graph}\PYG{o}{.}\PYG{n}{show}\PYG{p}{(}\PYG{n}{mdl}\PYG{o}{.}\PYG{n}{bipartite}\PYG{p}{,}  \PYG{n}{heatmap}\PYG{o}{=}\PYG{n}{heatmap1}\PYG{p}{,} \PYG{n}{scale}\PYG{o}{=}\PYG{l+m+mi}{2}\PYG{p}{)}
\end{sphinxVerbatim}
}

\end{sphinxuseclass}
\begin{sphinxuseclass}{nboutput}
{

\kern-\sphinxverbatimsmallskipamount\kern-\baselineskip
\kern+\FrameHeightAdjust\kern-\fboxrule
\vspace{\nbsphinxcodecellspacing}

\sphinxsetup{VerbatimColor={named}{white}}
\sphinxsetup{VerbatimBorderColor={named}{nbsphinx-code-border}}
\begin{sphinxuseclass}{output_area}
\begin{sphinxuseclass}{}


\begin{sphinxVerbatim}[commandchars=\\\{\}]
\llap{\color{nbsphinxout}[49]:\,\hspace{\fboxrule}\hspace{\fboxsep}}(<Figure size 432x288 with 1 Axes>, <AxesSubplot:>)
\end{sphinxVerbatim}



\end{sphinxuseclass}
\end{sphinxuseclass}
}

\end{sphinxuseclass}
\begin{sphinxuseclass}{nboutput}
\begin{sphinxuseclass}{nblast}
\hrule height -\fboxrule\relax
\vspace{\nbsphinxcodecellspacing}

\makeatletter\setbox\nbsphinxpromptbox\box\voidb@x\makeatother

\begin{nbsphinxfancyoutput}

\begin{sphinxuseclass}{output_area}
\begin{sphinxuseclass}{}
\noindent\sphinxincludegraphics[width=349\sphinxpxdimen,height=231\sphinxpxdimen]{{example_pump_Pump_Example_Notebook_86_1}.png}

\end{sphinxuseclass}
\end{sphinxuseclass}
\end{nbsphinxfancyoutput}

\end{sphinxuseclass}
\end{sphinxuseclass}
\sphinxAtStartPar
Since dill saves/loads the entire workspace (and not individual sims), it is important to make sure each simulation results have a different variable name so they aren’t overwritten. In this case, resgraph was generated multiple times, so it may be ambiguous in the future where this simulation fesult came from.

\begin{sphinxuseclass}{nbinput}
{
\sphinxsetup{VerbatimColor={named}{nbsphinx-code-bg}}
\sphinxsetup{VerbatimBorderColor={named}{nbsphinx-code-border}}
\begin{sphinxVerbatim}[commandchars=\\\{\}]
\llap{\color{nbsphinxin}[50]:\,\hspace{\fboxrule}\hspace{\fboxsep}}\PYG{n}{rd}\PYG{o}{.}\PYG{n}{graph}\PYG{o}{.}\PYG{n}{show}\PYG{p}{(}\PYG{n}{resgraph}\PYG{p}{)}
\end{sphinxVerbatim}
}

\end{sphinxuseclass}
\begin{sphinxuseclass}{nboutput}
{

\kern-\sphinxverbatimsmallskipamount\kern-\baselineskip
\kern+\FrameHeightAdjust\kern-\fboxrule
\vspace{\nbsphinxcodecellspacing}

\sphinxsetup{VerbatimColor={named}{white}}
\sphinxsetup{VerbatimBorderColor={named}{nbsphinx-code-border}}
\begin{sphinxuseclass}{output_area}
\begin{sphinxuseclass}{}


\begin{sphinxVerbatim}[commandchars=\\\{\}]
\llap{\color{nbsphinxout}[50]:\,\hspace{\fboxrule}\hspace{\fboxsep}}(<Figure size 432x288 with 1 Axes>, <AxesSubplot:>)
\end{sphinxVerbatim}



\end{sphinxuseclass}
\end{sphinxuseclass}
}

\end{sphinxuseclass}
\begin{sphinxuseclass}{nboutput}
\begin{sphinxuseclass}{nblast}
\hrule height -\fboxrule\relax
\vspace{\nbsphinxcodecellspacing}

\makeatletter\setbox\nbsphinxpromptbox\box\voidb@x\makeatother

\begin{nbsphinxfancyoutput}

\begin{sphinxuseclass}{output_area}
\begin{sphinxuseclass}{}
\noindent\sphinxincludegraphics[width=349\sphinxpxdimen,height=231\sphinxpxdimen]{{example_pump_Pump_Example_Notebook_88_1}.png}

\end{sphinxuseclass}
\end{sphinxuseclass}
\end{nbsphinxfancyoutput}

\end{sphinxuseclass}
\end{sphinxuseclass}
\begin{sphinxuseclass}{nbinput}
\begin{sphinxuseclass}{nblast}
{
\sphinxsetup{VerbatimColor={named}{nbsphinx-code-bg}}
\sphinxsetup{VerbatimBorderColor={named}{nbsphinx-code-border}}
\begin{sphinxVerbatim}[commandchars=\\\{\}]
\llap{\color{nbsphinxin}[ ]:\,\hspace{\fboxrule}\hspace{\fboxsep}}
\end{sphinxVerbatim}
}

\end{sphinxuseclass}
\end{sphinxuseclass}

\subsection{Defining Fault Sampling Approaches in fmdtools}
\label{\detokenize{docs/Approach_Use-Cases:Defining-Fault-Sampling-Approaches-in-fmdtools}}\label{\detokenize{docs/Approach_Use-Cases::doc}}
\sphinxAtStartPar
Fault Sampling is used to evaluate the overall resilience of a system to a set of faults and the corresponding risks associated with these faults. There is no single best way to define the set of scenarios to evaluate resilience with, because a given resilience analysis may need more or less detail, support more or less computational time, or be interested in specific scenario types of interest.

\sphinxAtStartPar
Thus, there are a number of use\sphinxhyphen{}cases supported by fmdtools for different sampling models. This document will demonstrate and showcase a few of them.

\begin{sphinxuseclass}{nbinput}
\begin{sphinxuseclass}{nblast}
{
\sphinxsetup{VerbatimColor={named}{nbsphinx-code-bg}}
\sphinxsetup{VerbatimBorderColor={named}{nbsphinx-code-border}}
\begin{sphinxVerbatim}[commandchars=\\\{\}]
\llap{\color{nbsphinxin}[1]:\,\hspace{\fboxrule}\hspace{\fboxsep}}\PYG{c+c1}{\PYGZsh{} for use in development \PYGZhy{} makes sure git version is used instead of pip\PYGZhy{}installed version}
\PYG{k+kn}{import} \PYG{n+nn}{sys}\PYG{o}{,} \PYG{n+nn}{os}
\PYG{n}{sys}\PYG{o}{.}\PYG{n}{path}\PYG{o}{.}\PYG{n}{insert}\PYG{p}{(}\PYG{l+m+mi}{1}\PYG{p}{,}\PYG{n}{os}\PYG{o}{.}\PYG{n}{path}\PYG{o}{.}\PYG{n}{join}\PYG{p}{(}\PYG{l+s+s2}{\PYGZdq{}}\PYG{l+s+s2}{..}\PYG{l+s+s2}{\PYGZdq{}}\PYG{p}{)}\PYG{p}{)}

\PYG{k+kn}{from} \PYG{n+nn}{fmdtools}\PYG{n+nn}{.}\PYG{n+nn}{modeldef} \PYG{k+kn}{import} \PYG{o}{*}
\PYG{k+kn}{import} \PYG{n+nn}{fmdtools}\PYG{n+nn}{.}\PYG{n+nn}{resultdisp} \PYG{k}{as} \PYG{n+nn}{rd}
\PYG{k+kn}{import} \PYG{n+nn}{fmdtools}\PYG{n+nn}{.}\PYG{n+nn}{faultsim}\PYG{n+nn}{.}\PYG{n+nn}{propagate} \PYG{k}{as} \PYG{n+nn}{prop}

\end{sphinxVerbatim}
}

\end{sphinxuseclass}
\end{sphinxuseclass}

\subsubsection{Basics}
\label{\detokenize{docs/Approach_Use-Cases:Basics}}
\sphinxAtStartPar
Fault sampling involves: \sphinxhyphen{} Defining faults and fault models within each function/component of the model (using the \sphinxcode{\sphinxupquote{.assoc\_modes()}} method) \sphinxhyphen{} Defining a fault sampling approach (using the \sphinxcode{\sphinxupquote{SampleApproach}} class) \sphinxhyphen{} Propagating faults through the model (using the \sphinxcode{\sphinxupquote{propagate.approach}} method) Before proceeding, it can be helpful to look through their respective documentation.

\begin{sphinxuseclass}{nbinput}
{
\sphinxsetup{VerbatimColor={named}{nbsphinx-code-bg}}
\sphinxsetup{VerbatimBorderColor={named}{nbsphinx-code-border}}
\begin{sphinxVerbatim}[commandchars=\\\{\}]
\llap{\color{nbsphinxin}[2]:\,\hspace{\fboxrule}\hspace{\fboxsep}}\PYG{n}{help}\PYG{p}{(}\PYG{n}{FxnBlock}\PYG{o}{.}\PYG{n}{assoc\PYGZus{}modes}\PYG{p}{)}
\end{sphinxVerbatim}
}

\end{sphinxuseclass}
\begin{sphinxuseclass}{nboutput}
\begin{sphinxuseclass}{nblast}
{

\kern-\sphinxverbatimsmallskipamount\kern-\baselineskip
\kern+\FrameHeightAdjust\kern-\fboxrule
\vspace{\nbsphinxcodecellspacing}

\sphinxsetup{VerbatimColor={named}{white}}
\sphinxsetup{VerbatimBorderColor={named}{nbsphinx-code-border}}
\begin{sphinxuseclass}{output_area}
\begin{sphinxuseclass}{}


\begin{sphinxVerbatim}[commandchars=\\\{\}]
Help on function assoc\_modes in module fmdtools.modeldef:

assoc\_modes(self, faultmodes=\{\}, opermodes=[], initmode='nom', name='', probtype='rate', units='hr', exclusive=False, key\_phases\_by='global', longnames=\{\})
    Associates fault and operational modes with the block when called in the function or component.

    Parameters
    ----------
    faultmodes : dict, optional
        Dictionary/Set of faultmodes with structure, which can have the forms:
            - set \{'fault1', 'fault2', 'fault3'\} (just the respective faults)
            - dict \{'fault1': faultattributes, 'fault2': faultattributes\}, where faultattributes is:
                - float: rate for the mode
                - [float, float]: rate and repair cost for the mode
                - float, oppvect, float]: rate, opportunity vector, and repair cost for the mode
                opportunity vector can be specified as:
                    [float1, float2,{\ldots}], a vector of relative likelihoods for each phase, or
                    \{opermode:float1, opermode:float1\}, a dict of relative likelihoods for each phase/mode
                    the phases/modes to key by are defined in "key\_phases\_by"
    opermodes : list, optional
        List of operational modes
    initmode : str, optional
        Initial operational mode. Default is 'nom'
    name : str, optional
        (for components only) Name of the component. The default is ''.
    probtype : str, optional
        Type of probability in the probability model, a per-time 'rate' or per-run 'prob'.
        The default is 'rate'
    units : str, optional
        Type of units ('sec'/'min'/'hr'/'day') used for the rates. Default is 'hr'
    exclusive : True/False
        Whether fault modes are exclusive of each other or not. Default is False (i.e. more than one can be present).
    key\_phases\_by : 'self'/'none'/'global'/'fxnname'
        Phases to key the faultmodes by (using local, global, or an external function's modes'). Default is 'global'
    longnames : dict
        Longer names for the faults (if desired). \{faultname: longname\}

\end{sphinxVerbatim}



\end{sphinxuseclass}
\end{sphinxuseclass}
}

\end{sphinxuseclass}
\end{sphinxuseclass}
\begin{sphinxuseclass}{nbinput}
{
\sphinxsetup{VerbatimColor={named}{nbsphinx-code-bg}}
\sphinxsetup{VerbatimBorderColor={named}{nbsphinx-code-border}}
\begin{sphinxVerbatim}[commandchars=\\\{\}]
\llap{\color{nbsphinxin}[3]:\,\hspace{\fboxrule}\hspace{\fboxsep}}\PYG{n}{help}\PYG{p}{(}\PYG{n}{SampleApproach}\PYG{p}{)}
\end{sphinxVerbatim}
}

\end{sphinxuseclass}
\begin{sphinxuseclass}{nboutput}
\begin{sphinxuseclass}{nblast}
{

\kern-\sphinxverbatimsmallskipamount\kern-\baselineskip
\kern+\FrameHeightAdjust\kern-\fboxrule
\vspace{\nbsphinxcodecellspacing}

\sphinxsetup{VerbatimColor={named}{white}}
\sphinxsetup{VerbatimBorderColor={named}{nbsphinx-code-border}}
\begin{sphinxuseclass}{output_area}
\begin{sphinxuseclass}{}


\begin{sphinxVerbatim}[commandchars=\\\{\}]
Help on class SampleApproach in module fmdtools.modeldef:

class SampleApproach(builtins.object)
 |  SampleApproach(mdl, faults='all', phases='global', modephases=\{\}, jointfaults=\{'faults': 'None'\}, sampparams=\{\}, defaultsamp=\{'samp': 'evenspacing', 'numpts': 1\})
 |
 |  Class for defining the sample approach to be used for a set of faults.
 |
 |  Attributes
 |  ----------
 |  phases : dict
 |      phases given to sample the fault modes in
 |  globalphases : dict
 |      phases defined in the model
 |  modephases : dict
 |      Dictionary of modes associated with each state
 |  mode\_phase\_map : dict
 |      Mapping of modes to their corresponding phases
 |  tstep : float
 |      timestep defined in the model
 |  fxnrates : dict
 |      overall failure rates for each function
 |  comprates : dict
 |      overall failure rates for each component
 |  jointmodes : list
 |      (if any) joint fault modes to be injected in the approach
 |  rates/comprates/rates\_timeless : dict
 |      rates of each mode (fxn, mode) in each model phase, structured \{fxnmode: \{phaseid:rate\}\}
 |  sampletimes : dict
 |      faults to inject at each time in each phase, structured \{phaseid:time:fnxmode\}
 |  weights : dict
 |      weight to put on each time each fault was injected, structured \{fxnmode:phaseid:time:weight\}
 |  sampparams : dict
 |      parameters used to sample each mode
 |  scenlist : list
 |      list of fault scenarios (dicts of faults and properties) that fault propagation iterates through
 |  scenids : dict
 |      a list of scenario ids associated with a given fault in a given phase, structured \{(fxnmode,phaseid):listofnames\}
 |  mode\_phase\_map : dict
 |      a dict of modes and their respective phases to inject with structure \{fxnmode:\{mode\_phase\_map:[starttime, endtime]\}\}
 |  units : str
 |      time-units to use in the approach probability model
 |  unit\_factors : dict
 |      multiplication factors for converting some time units to others.
 |
 |  Methods defined here:
 |
 |  \_\_init\_\_(self, mdl, faults='all', phases='global', modephases=\{\}, jointfaults=\{'faults': 'None'\}, sampparams=\{\}, defaultsamp=\{'samp': 'evenspacing', 'numpts': 1\})
 |      Initializes the sample approach for a given model
 |
 |      Parameters
 |      ----------
 |      mdl : Model
 |          Model to sample.
 |      faults : str/list/tuple, optional
 |          - The default is 'all', which gets all fault modes from the model.
 |          - 'single-components' uses faults from a single component to represent faults form all components
 |          - passing the function name only includes modes from that function
 |          - List of faults of form [(fxn, mode)] to inject in the model.
 |          -Tuple arguments
 |              - ('mode type', 'mode','notmode'), gets all modes with 'mode' as a string (e.g. "mech", "comms", "loss" faults). 'notmode' (if given) specifies strings to remove
 |              - ('mode types', ('mode1', 'mode2')), gets all modes with the listed strings (e.g. "mech", "comms", "loss" faults)
 |              - ('mode name', 'mode'), gets all modes with the exact name 'mode'
 |              - ('mode names', ('mode1', 'mode2')), gets all modes with the exact names defined in the tuple
 |              - ('function class', 'Classname'), which gets all modes from a function with class 'Classname'
 |              - ('function classes', ('Classname1', 'Classname2')), which gets all modes from a function with the names in the tuple
 |      phases: dict or 'global' or list
 |          Local phases in the model to sample.
 |              Dict has structure: \{'Function':\{'phase':[starttime, endtime]\}\}
 |              List has structure: ['phase1', 'phase2'] where phases are phases in mdl.phases
 |          Defaults to 'global',here only the phases defined in mdl.phases are used.
 |          Phases and modephases can be gotten from process.modephases(mdlhist)
 |      modephases: dict
 |          Dictionary of modes associated with each phase.
 |          For use when the opportunity vector is keyed to modes and each mode is
 |          entered multiple times in a simulation, resulting in
 |          multiple phases associated with that mode. Has structure:
 |              \{'Function':\{'mode':\{'phase','phase1', 'phase2'{\ldots}\}\}\}
 |              Phases and modephases can be gotten from process.modephases(mdlhist)
 |      jointfaults : dict, optional
 |          Defines how the approach considers joint faults. The default is \{'faults':'None'\}. Has structure:
 |              - faults : float
 |                  \# of joint faults to inject. 'all' specifies all faults at the same time
 |              - jointfuncs :  bool
 |                  determines whether more than one mode can be injected in a single function
 |              - pcond (optional) : float in range (0,1)
 |                  conditional probabilities for joint faults. If not give, independence is assumed.
 |              - inclusive (optional) : bool
 |                  specifies whether the fault set includes all joint faults up to the given level, or only the given level
 |                  (e.g., True with 'all' means SampleApproach includes every combination of joint fault modes while
 |                         False with 'all' means SampleApproach only includes the joint fault mode with all faults)
 |      sampparams : dict, optional
 |          Defines how specific modes in the model will be sampled over time. The default is \{\}.
 |          Has structure: \{(fxnmode,phase): sampparam\}, where sampparam has structure:
 |              - 'samp' : str ('quad', 'fullint', 'evenspacing','randtimes','symrandtimes')
 |                  sample strategy to use (quadrature, full integral, even spacing, random times, likeliest, or symmetric random times)
 |              - 'numpts' : float
 |                  number of points to use (for evenspacing, randtimes, and symrandtimes only)
 |              - 'quad' : quadpy quadrature
 |                  quadrature object if the quadrature option is selected.
 |      defaultsamp : TYPE, optional
 |          Defines how the model will be sampled over time by default. The default is \{'samp':'evenspacing','numpts':1\}. Has structure:
 |              - 'samp' : str ('quad', 'fullint', 'evenspacing','randtimes','symrandtimes')
 |                  sample strategy to use (quadrature, full integral, even spacing, random times,likeliest, or symmetric random times)
 |              - 'numpts' : float
 |                  number of points to use (for evenspacing, randtimes, and symrandtimes only)
 |              - 'quad' : quadpy quadrature
 |                  quadrature object if the quadrature option is selected.
 |
 |  add\_phasetimes(self, fxnmode, phaseid, phasetimes, weights=[])
 |      Adds a set of times for a given mode to sampletimes
 |
 |  create\_nomscen(self, mdl)
 |      Creates a nominal scenario
 |
 |  create\_sampletimes(self, mdl, params=\{\}, default=\{'samp': 'evenspacing', 'numpts': 1\})
 |      Initializes weights and sampletimes
 |
 |  create\_scenarios(self)
 |      Creates list of scenarios to be iterated over in fault injection. Added as scenlist and scenids
 |
 |  init\_modelist(self, mdl, faults, jointfaults=\{'faults': 'None'\})
 |      Initializes comprates, jointmodes internal list of modes
 |
 |  init\_rates(self, mdl, jointfaults=\{'faults': 'None'\}, modephases=\{\})
 |      Initializes rates, rates\_timeless
 |
 |  list\_moderates(self)
 |      Returns the rates for each mode
 |
 |  list\_modes(self, joint=False)
 |      Returns a list of modes in the approach
 |
 |  prune\_scenarios(self, endclasses, samptype='piecewise', threshold=0.1, sampparam=\{'samp': 'evenspacing', 'numpts': 1\})
 |      Finds the best sample approach to approximate the full integral (given the approach was the full integral).
 |
 |      Parameters
 |      ----------
 |      endclasses : dict
 |          dict of results (cost, rate, expected cost) for the model run indexed by scenid
 |      samptype : str ('piecewise' or 'bestpt'), optional
 |          Method to use.
 |          If 'bestpt', finds the point in the interval that gives the average cost.
 |          If 'piecewise', attempts to split the inverval into sub-intervals of continuity
 |          The default is 'piecewise'.
 |      threshold : float, optional
 |          If 'piecewise,' the threshold for detecting a discontinuity based on deviation from linearity. The default is 0.1.
 |      sampparam : float, optional
 |          If 'piecewise,' the sampparam sampparam to prune to. The default is \{'samp':'evenspacing','numpts':1\}, which would be a single point (optimal for linear).
 |
 |  select\_points(self, param, possible\_pts)
 |      Selects points in the list possible\_points according to a given sample strategy.
 |
 |      Parameters
 |      ----------
 |      param : dict
 |          Sample parameter. Has structure:
 |              - 'samp' : str ('quad', 'fullint', 'evenspacing','randtimes','symrandtimes')
 |                  sample strategy to use (quadrature, full integral, even spacing, random times, or symmetric random times)
 |              - 'numpts' : float
 |                  number of points to use (for evenspacing, randtimes, and symrandtimes only)
 |              - 'quad' : quadpy quadrature
 |                  quadrature object if the quadrature option is selected.
 |      possible\_pts :
 |          list of possible points in time.
 |
 |      Returns
 |      -------
 |      pts : list
 |          selected points
 |      weights : list
 |          weights for each point
 |
 |  ----------------------------------------------------------------------
 |  Data descriptors defined here:
 |
 |  \_\_dict\_\_
 |      dictionary for instance variables (if defined)
 |
 |  \_\_weakref\_\_
 |      list of weak references to the object (if defined)

\end{sphinxVerbatim}



\end{sphinxuseclass}
\end{sphinxuseclass}
}

\end{sphinxuseclass}
\end{sphinxuseclass}
\begin{sphinxuseclass}{nbinput}
{
\sphinxsetup{VerbatimColor={named}{nbsphinx-code-bg}}
\sphinxsetup{VerbatimBorderColor={named}{nbsphinx-code-border}}
\begin{sphinxVerbatim}[commandchars=\\\{\}]
\llap{\color{nbsphinxin}[4]:\,\hspace{\fboxrule}\hspace{\fboxsep}}\PYG{n}{help}\PYG{p}{(}\PYG{n}{prop}\PYG{o}{.}\PYG{n}{approach}\PYG{p}{)}
\end{sphinxVerbatim}
}

\end{sphinxuseclass}
\begin{sphinxuseclass}{nboutput}
\begin{sphinxuseclass}{nblast}
{

\kern-\sphinxverbatimsmallskipamount\kern-\baselineskip
\kern+\FrameHeightAdjust\kern-\fboxrule
\vspace{\nbsphinxcodecellspacing}

\sphinxsetup{VerbatimColor={named}{white}}
\sphinxsetup{VerbatimBorderColor={named}{nbsphinx-code-border}}
\begin{sphinxuseclass}{output_area}
\begin{sphinxuseclass}{}


\begin{sphinxVerbatim}[commandchars=\\\{\}]
Help on function approach in module fmdtools.faultsim.propagate:

approach(mdl, app, staged=False, track='all', pool=False, showprogress=True, track\_times='all', protect=True, run\_stochastic=False, **kwargs)
    Injects and propagates faults in the model defined by a given sample approach

    Parameters
    ----------
    mdl : model
        The model to inject faults in.
    app : sampleapproach
        SampleApproach used to define the list of faults and sample time for the model.
    staged : bool, optional
        Whether to inject the fault in a copy of the nominal model at the fault time (True) or instantiate a new model for the fault (False). Setting to True roughly halves execution time. The default is False.
    track : str ('all', 'functions', 'flows', 'valparams', dict, 'none'), optional
        Which model states to track over time, which can be given as 'functions', 'flows',
        'all', 'none', 'valparams' (model states specified in mdl.valparams),
        or a dict of form \{'functions':\{'fxn1':'att1'\}, 'flows':\{'flow1':'att1'\}\}
        The default is 'all'.
    pool : process pool, optional
        Process Pool Object from multiprocessing or pathos packages. Pathos is recommended.
        e.g. parallelpool = mp.pool(n) for n cores (multiprocessing)
        or parallelpool = ProcessPool(nodes=n) for n cores (pathos)
        If False, the set of scenarios is run serially. The default is False
    showprogress: bool, optional
        whether to show a progress bar during execution. default is true
    track\_times : str/tuple
        Defines what times to include in the history. Options are:
            'all'--all simulated times
            ('interval', n)--includes every nth time in the history
            ('times', [t1, {\ldots} tn])--only includes times defined in the vector [t1 {\ldots} tn]
    protect : bool
        Whether or not to protect the model object via copying
            True (default) - re-instances the model (safe)
            False - model is not re-instantiated (unsafe--do not use model afterwards)
    run\_stochastic : bool
        Whether to run stochastic behaviors or use default values for stochastic variables. Default is False.
    **kwargs: kwargs (params, modelparams, and/or valparams)
        passing parameter dictionaries (e.g., params, modelparams, valparams) instantiates the model
        to be simulated with the given parameters. Parameter dictionaries do not
        need to be complete (if incomplete)
    Returns
    -------
    endclasses : dict
        A dictionary with the rate, cost, and expected cost of each scenario run with structure \{scenname:\{expected cost, cost, rate\}\}
    mdlhists : dict
        A dictionary with the history of all model states for each scenario (including the nominal)

\end{sphinxVerbatim}



\end{sphinxuseclass}
\end{sphinxuseclass}
}

\end{sphinxuseclass}
\end{sphinxuseclass}

\subsubsection{Model Setup}
\label{\detokenize{docs/Approach_Use-Cases:Model-Setup}}
\sphinxAtStartPar
Consider the following (highly simplified) rover electrical/navigation model. We can define the functions of this rover using the classes:

\begin{sphinxuseclass}{nbinput}
\begin{sphinxuseclass}{nblast}
{
\sphinxsetup{VerbatimColor={named}{nbsphinx-code-bg}}
\sphinxsetup{VerbatimBorderColor={named}{nbsphinx-code-border}}
\begin{sphinxVerbatim}[commandchars=\\\{\}]
\llap{\color{nbsphinxin}[5]:\,\hspace{\fboxrule}\hspace{\fboxsep}}\PYG{k}{class} \PYG{n+nc}{Move\PYGZus{}Rover}\PYG{p}{(}\PYG{n}{FxnBlock}\PYG{p}{)}\PYG{p}{:}
    \PYG{k}{def} \PYG{n+nf+fm}{\PYGZus{}\PYGZus{}init\PYGZus{}\PYGZus{}}\PYG{p}{(}\PYG{n+nb+bp}{self}\PYG{p}{,}\PYG{n}{name}\PYG{p}{,} \PYG{n}{flows}\PYG{p}{)}\PYG{p}{:}
        \PYG{n+nb}{super}\PYG{p}{(}\PYG{p}{)}\PYG{o}{.}\PYG{n+nf+fm}{\PYGZus{}\PYGZus{}init\PYGZus{}\PYGZus{}}\PYG{p}{(}\PYG{n}{name}\PYG{p}{,} \PYG{n}{flows}\PYG{p}{,} \PYG{n}{flownames}\PYG{o}{=}\PYG{p}{\PYGZob{}}\PYG{l+s+s2}{\PYGZdq{}}\PYG{l+s+s2}{EE}\PYG{l+s+s2}{\PYGZdq{}}\PYG{p}{:}\PYG{l+s+s2}{\PYGZdq{}}\PYG{l+s+s2}{EE\PYGZus{}in}\PYG{l+s+s2}{\PYGZdq{}}\PYG{p}{\PYGZcb{}}\PYG{p}{)}
        \PYG{n+nb+bp}{self}\PYG{o}{.}\PYG{n}{assoc\PYGZus{}modes}\PYG{p}{(}\PYG{p}{\PYGZob{}}\PYG{l+s+s2}{\PYGZdq{}}\PYG{l+s+s2}{mech\PYGZus{}loss}\PYG{l+s+s2}{\PYGZdq{}}\PYG{p}{,} \PYG{l+s+s2}{\PYGZdq{}}\PYG{l+s+s2}{elec\PYGZus{}open}\PYG{l+s+s2}{\PYGZdq{}}\PYG{p}{\PYGZcb{}}\PYG{p}{)}
    \PYG{k}{def} \PYG{n+nf}{behavior}\PYG{p}{(}\PYG{n+nb+bp}{self}\PYG{p}{,} \PYG{n}{time}\PYG{p}{)}\PYG{p}{:}
        \PYG{k}{if} \PYG{n+nb+bp}{self}\PYG{o}{.}\PYG{n}{time} \PYG{o}{\PYGZlt{}} \PYG{n}{time}\PYG{p}{:}
            \PYG{n}{power} \PYG{o}{=} \PYG{n+nb+bp}{self}\PYG{o}{.}\PYG{n}{EE\PYGZus{}in}\PYG{o}{.}\PYG{n}{v\PYGZus{}supply} \PYG{o}{*} \PYG{n+nb+bp}{self}\PYG{o}{.}\PYG{n}{Control}\PYG{o}{.}\PYG{n}{vel} \PYG{o}{*}\PYG{n+nb+bp}{self}\PYG{o}{.}\PYG{n}{no\PYGZus{}fault}\PYG{p}{(}\PYG{l+s+s2}{\PYGZdq{}}\PYG{l+s+s2}{elec\PYGZus{}open}\PYG{l+s+s2}{\PYGZdq{}}\PYG{p}{)}
            \PYG{n+nb+bp}{self}\PYG{o}{.}\PYG{n}{EE\PYGZus{}in}\PYG{o}{.}\PYG{n}{a\PYGZus{}supply} \PYG{o}{=} \PYG{n}{power}\PYG{o}{/}\PYG{l+m+mi}{12}
            \PYG{k}{if} \PYG{n}{power} \PYG{o}{\PYGZgt{}}\PYG{l+m+mi}{100}\PYG{p}{:} \PYG{n+nb+bp}{self}\PYG{o}{.}\PYG{n}{add\PYGZus{}fault}\PYG{p}{(}\PYG{l+s+s2}{\PYGZdq{}}\PYG{l+s+s2}{elec\PYGZus{}open}\PYG{l+s+s2}{\PYGZdq{}}\PYG{p}{)}
            \PYG{k}{else}\PYG{p}{:}          \PYG{n+nb+bp}{self}\PYG{o}{.}\PYG{n}{Ground}\PYG{o}{.}\PYG{n}{x} \PYG{o}{=} \PYG{n+nb+bp}{self}\PYG{o}{.}\PYG{n}{Ground}\PYG{o}{.}\PYG{n}{x} \PYG{o}{+} \PYG{n}{power}\PYG{o}{*}\PYG{n+nb+bp}{self}\PYG{o}{.}\PYG{n}{no\PYGZus{}fault}\PYG{p}{(}\PYG{l+s+s2}{\PYGZdq{}}\PYG{l+s+s2}{mech\PYGZus{}loss}\PYG{l+s+s2}{\PYGZdq{}}\PYG{p}{)}
\end{sphinxVerbatim}
}

\end{sphinxuseclass}
\end{sphinxuseclass}
\sphinxAtStartPar
This use of \sphinxcode{\sphinxupquote{assoc\_modes}} specifies that there are two modes to inject, “mech\_loss”, and “elec\_open”, with no more information given for each mode. This is a syntax one might use as one is constructing the model and wishing to verify behaviors, or where modes are only caused behaviorally due to external scenarios. (in this case an open circuit is caused by too much power supply)

\begin{sphinxuseclass}{nbinput}
\begin{sphinxuseclass}{nblast}
{
\sphinxsetup{VerbatimColor={named}{nbsphinx-code-bg}}
\sphinxsetup{VerbatimBorderColor={named}{nbsphinx-code-border}}
\begin{sphinxVerbatim}[commandchars=\\\{\}]
\llap{\color{nbsphinxin}[6]:\,\hspace{\fboxrule}\hspace{\fboxsep}}\PYG{k}{class} \PYG{n+nc}{Control\PYGZus{}Rover}\PYG{p}{(}\PYG{n}{FxnBlock}\PYG{p}{)}\PYG{p}{:}
    \PYG{k}{def} \PYG{n+nf+fm}{\PYGZus{}\PYGZus{}init\PYGZus{}\PYGZus{}}\PYG{p}{(}\PYG{n+nb+bp}{self}\PYG{p}{,}\PYG{n}{name}\PYG{p}{,} \PYG{n}{flows}\PYG{p}{)}\PYG{p}{:}
        \PYG{n+nb}{super}\PYG{p}{(}\PYG{p}{)}\PYG{o}{.}\PYG{n+nf+fm}{\PYGZus{}\PYGZus{}init\PYGZus{}\PYGZus{}}\PYG{p}{(}\PYG{n}{name}\PYG{p}{,} \PYG{n}{flows}\PYG{p}{)}
        \PYG{n+nb+bp}{self}\PYG{o}{.}\PYG{n}{assoc\PYGZus{}modes}\PYG{p}{(}\PYG{p}{\PYGZob{}}\PYG{l+s+s1}{\PYGZsq{}}\PYG{l+s+s1}{no\PYGZus{}con}\PYG{l+s+s1}{\PYGZsq{}}\PYG{p}{:}\PYG{p}{[}\PYG{l+m+mf}{1e\PYGZhy{}4}\PYG{p}{,} \PYG{l+m+mi}{200}\PYG{p}{]}\PYG{p}{\PYGZcb{}}\PYG{p}{,} \PYG{p}{[}\PYG{l+s+s1}{\PYGZsq{}}\PYG{l+s+s1}{drive}\PYG{l+s+s1}{\PYGZsq{}}\PYG{p}{,}\PYG{l+s+s1}{\PYGZsq{}}\PYG{l+s+s1}{standby}\PYG{l+s+s1}{\PYGZsq{}}\PYG{p}{]}\PYG{p}{,} \PYG{n}{initmode}\PYG{o}{=}\PYG{l+s+s1}{\PYGZsq{}}\PYG{l+s+s1}{standby}\PYG{l+s+s1}{\PYGZsq{}}\PYG{p}{)}
    \PYG{k}{def} \PYG{n+nf}{behavior}\PYG{p}{(}\PYG{n+nb+bp}{self}\PYG{p}{,}\PYG{n}{time}\PYG{p}{)}\PYG{p}{:}
        \PYG{k}{if} \PYG{o+ow}{not} \PYG{n+nb+bp}{self}\PYG{o}{.}\PYG{n}{in\PYGZus{}mode}\PYG{p}{(}\PYG{l+s+s1}{\PYGZsq{}}\PYG{l+s+s1}{no\PYGZus{}con}\PYG{l+s+s1}{\PYGZsq{}}\PYG{p}{)}\PYG{p}{:}
            \PYG{k}{if} \PYG{n}{time} \PYG{o}{==} \PYG{l+m+mi}{5}\PYG{p}{:} \PYG{n+nb+bp}{self}\PYG{o}{.}\PYG{n}{set\PYGZus{}mode}\PYG{p}{(}\PYG{l+s+s1}{\PYGZsq{}}\PYG{l+s+s1}{drive}\PYG{l+s+s1}{\PYGZsq{}}\PYG{p}{)}
            \PYG{k}{if} \PYG{n}{time} \PYG{o}{==} \PYG{l+m+mi}{50}\PYG{p}{:} \PYG{n+nb+bp}{self}\PYG{o}{.}\PYG{n}{set\PYGZus{}mode}\PYG{p}{(}\PYG{l+s+s1}{\PYGZsq{}}\PYG{l+s+s1}{standby}\PYG{l+s+s1}{\PYGZsq{}}\PYG{p}{)}
        \PYG{k}{if} \PYG{n}{time}\PYG{o}{\PYGZgt{}}\PYG{n+nb+bp}{self}\PYG{o}{.}\PYG{n}{time}\PYG{p}{:}
            \PYG{k}{if} \PYG{n+nb+bp}{self}\PYG{o}{.}\PYG{n}{in\PYGZus{}mode}\PYG{p}{(}\PYG{l+s+s1}{\PYGZsq{}}\PYG{l+s+s1}{drive}\PYG{l+s+s1}{\PYGZsq{}}\PYG{p}{)}\PYG{p}{:}
                \PYG{n+nb+bp}{self}\PYG{o}{.}\PYG{n}{Control}\PYG{o}{.}\PYG{n}{power} \PYG{o}{=} \PYG{l+m+mi}{1}
                \PYG{n+nb+bp}{self}\PYG{o}{.}\PYG{n}{Control}\PYG{o}{.}\PYG{n}{vel} \PYG{o}{=} \PYG{l+m+mi}{1}
            \PYG{k}{elif} \PYG{n+nb+bp}{self}\PYG{o}{.}\PYG{n}{in\PYGZus{}mode}\PYG{p}{(}\PYG{l+s+s1}{\PYGZsq{}}\PYG{l+s+s1}{standby}\PYG{l+s+s1}{\PYGZsq{}}\PYG{p}{)}\PYG{p}{:}
                \PYG{n+nb+bp}{self}\PYG{o}{.}\PYG{n}{Control}\PYG{o}{.}\PYG{n}{vel} \PYG{o}{=} \PYG{l+m+mi}{0}
                \PYG{n+nb+bp}{self}\PYG{o}{.}\PYG{n}{Control}\PYG{o}{.}\PYG{n}{power}\PYG{o}{=}\PYG{l+m+mi}{0}
\end{sphinxVerbatim}
}

\end{sphinxuseclass}
\end{sphinxuseclass}
\sphinxAtStartPar
This use of \sphinxcode{\sphinxupquote{assoc\_modes}} specifies modes (\sphinxcode{\sphinxupquote{no\_con}}) with a rate (1e\sphinxhyphen{}4) and a repair cost (200), as well as a set of operational modes that the system proceeds through. Specifying operational modes enables one to define different \sphinxstyleemphasis{behaviors} for the system depending on the configuration of the system at a given time. For example, in this function, the system goes into a drive mode at t=5, which powers on the system and outputs a command to move forward.

\sphinxAtStartPar
When operational modes are specified in the model, an initial mode must also be specified–in this case \sphinxcode{\sphinxupquote{initmode='standby'}} specifies that the function starts in the standby mode.

\begin{sphinxuseclass}{nbinput}
\begin{sphinxuseclass}{nblast}
{
\sphinxsetup{VerbatimColor={named}{nbsphinx-code-bg}}
\sphinxsetup{VerbatimBorderColor={named}{nbsphinx-code-border}}
\begin{sphinxVerbatim}[commandchars=\\\{\}]
\llap{\color{nbsphinxin}[7]:\,\hspace{\fboxrule}\hspace{\fboxsep}}\PYG{k}{class} \PYG{n+nc}{Store\PYGZus{}Energy}\PYG{p}{(}\PYG{n}{FxnBlock}\PYG{p}{)}\PYG{p}{:}
    \PYG{k}{def} \PYG{n+nf+fm}{\PYGZus{}\PYGZus{}init\PYGZus{}\PYGZus{}}\PYG{p}{(}\PYG{n+nb+bp}{self}\PYG{p}{,} \PYG{n}{name}\PYG{p}{,} \PYG{n}{flows}\PYG{p}{)}\PYG{p}{:}
        \PYG{n+nb}{super}\PYG{p}{(}\PYG{p}{)}\PYG{o}{.}\PYG{n+nf+fm}{\PYGZus{}\PYGZus{}init\PYGZus{}\PYGZus{}}\PYG{p}{(}\PYG{n}{name}\PYG{p}{,}\PYG{n}{flows}\PYG{p}{,} \PYG{n}{states}\PYG{o}{=}\PYG{p}{\PYGZob{}}\PYG{l+s+s2}{\PYGZdq{}}\PYG{l+s+s2}{charge}\PYG{l+s+s2}{\PYGZdq{}}\PYG{p}{:} \PYG{l+m+mi}{100}\PYG{p}{\PYGZcb{}}\PYG{p}{)}
        \PYG{n+nb+bp}{self}\PYG{o}{.}\PYG{n}{assoc\PYGZus{}modes}\PYG{p}{(}\PYG{p}{\PYGZob{}}\PYG{l+s+s2}{\PYGZdq{}}\PYG{l+s+s2}{no\PYGZus{}charge}\PYG{l+s+s2}{\PYGZdq{}}\PYG{p}{:}\PYG{p}{[}\PYG{l+m+mf}{1e\PYGZhy{}5}\PYG{p}{,} \PYG{p}{\PYGZob{}}\PYG{l+s+s1}{\PYGZsq{}}\PYG{l+s+s1}{standby}\PYG{l+s+s1}{\PYGZsq{}}\PYG{p}{:}\PYG{l+m+mf}{1.0}\PYG{p}{\PYGZcb{}}\PYG{p}{,} \PYG{l+m+mi}{100}\PYG{p}{]}\PYG{p}{,}\PYG{l+s+s2}{\PYGZdq{}}\PYG{l+s+s2}{short}\PYG{l+s+s2}{\PYGZdq{}}\PYG{p}{:}\PYG{p}{[}\PYG{l+m+mf}{1e\PYGZhy{}5}\PYG{p}{,} \PYG{p}{\PYGZob{}}\PYG{l+s+s1}{\PYGZsq{}}\PYG{l+s+s1}{supply}\PYG{l+s+s1}{\PYGZsq{}}\PYG{p}{:}\PYG{l+m+mf}{1.0}\PYG{p}{\PYGZcb{}}\PYG{p}{,} \PYG{l+m+mi}{100}\PYG{p}{]}\PYG{p}{,}\PYG{p}{\PYGZcb{}}\PYG{p}{,} \PYG{p}{[}\PYG{l+s+s2}{\PYGZdq{}}\PYG{l+s+s2}{supply}\PYG{l+s+s2}{\PYGZdq{}}\PYG{p}{,}\PYG{l+s+s2}{\PYGZdq{}}\PYG{l+s+s2}{charge}\PYG{l+s+s2}{\PYGZdq{}}\PYG{p}{,}\PYG{l+s+s2}{\PYGZdq{}}\PYG{l+s+s2}{standby}\PYG{l+s+s2}{\PYGZdq{}}\PYG{p}{]}\PYG{p}{,} \PYG{n}{initmode}\PYG{o}{=}\PYG{l+s+s2}{\PYGZdq{}}\PYG{l+s+s2}{standby}\PYG{l+s+s2}{\PYGZdq{}}\PYG{p}{,} \PYG{n}{exclusive} \PYG{o}{=} \PYG{k+kc}{True}\PYG{p}{,} \PYG{n}{key\PYGZus{}phases\PYGZus{}by}\PYG{o}{=}\PYG{l+s+s1}{\PYGZsq{}}\PYG{l+s+s1}{self}\PYG{l+s+s1}{\PYGZsq{}}\PYG{p}{)}
    \PYG{k}{def} \PYG{n+nf}{behavior}\PYG{p}{(}\PYG{n+nb+bp}{self}\PYG{p}{,}\PYG{n}{time}\PYG{p}{)}\PYG{p}{:}
        \PYG{k}{if} \PYG{n}{time} \PYG{o}{\PYGZgt{}} \PYG{n+nb+bp}{self}\PYG{o}{.}\PYG{n}{time}\PYG{p}{:}
            \PYG{k}{if} \PYG{n+nb+bp}{self}\PYG{o}{.}\PYG{n}{in\PYGZus{}mode}\PYG{p}{(}\PYG{l+s+s2}{\PYGZdq{}}\PYG{l+s+s2}{standby}\PYG{l+s+s2}{\PYGZdq{}}\PYG{p}{)}\PYG{p}{:}
                \PYG{n+nb+bp}{self}\PYG{o}{.}\PYG{n}{EE}\PYG{o}{.}\PYG{n}{v\PYGZus{}supply} \PYG{o}{=} \PYG{l+m+mi}{0}\PYG{p}{;} \PYG{n+nb+bp}{self}\PYG{o}{.}\PYG{n}{EE}\PYG{o}{.}\PYG{n}{a\PYGZus{}supply} \PYG{o}{=} \PYG{l+m+mi}{0}
                \PYG{k}{if} \PYG{n+nb+bp}{self}\PYG{o}{.}\PYG{n}{Control}\PYG{o}{.}\PYG{n}{power}\PYG{o}{==}\PYG{l+m+mi}{1}\PYG{p}{:} \PYG{n+nb+bp}{self}\PYG{o}{.}\PYG{n}{set\PYGZus{}mode}\PYG{p}{(}\PYG{l+s+s2}{\PYGZdq{}}\PYG{l+s+s2}{supply}\PYG{l+s+s2}{\PYGZdq{}}\PYG{p}{)}
            \PYG{k}{elif} \PYG{n+nb+bp}{self}\PYG{o}{.}\PYG{n}{in\PYGZus{}mode}\PYG{p}{(}\PYG{l+s+s2}{\PYGZdq{}}\PYG{l+s+s2}{charge}\PYG{l+s+s2}{\PYGZdq{}}\PYG{p}{)}\PYG{p}{:}
                \PYG{n+nb+bp}{self}\PYG{o}{.}\PYG{n}{EE}\PYG{o}{.}\PYG{n}{charge} \PYG{o}{=}\PYG{n+nb}{min}\PYG{p}{(}\PYG{n+nb+bp}{self}\PYG{o}{.}\PYG{n}{EE}\PYG{o}{.}\PYG{n}{charge}\PYG{o}{+}\PYG{n+nb+bp}{self}\PYG{o}{.}\PYG{n}{tstep}\PYG{p}{,} \PYG{l+m+mi}{20}\PYG{p}{)}
            \PYG{k}{elif} \PYG{n+nb+bp}{self}\PYG{o}{.}\PYG{n}{in\PYGZus{}mode}\PYG{p}{(}\PYG{l+s+s2}{\PYGZdq{}}\PYG{l+s+s2}{supply}\PYG{l+s+s2}{\PYGZdq{}}\PYG{p}{)}\PYG{p}{:}
                \PYG{k}{if} \PYG{n+nb+bp}{self}\PYG{o}{.}\PYG{n}{charge} \PYG{o}{\PYGZgt{}} \PYG{l+m+mi}{0}\PYG{p}{:}         \PYG{n+nb+bp}{self}\PYG{o}{.}\PYG{n}{EE}\PYG{o}{.}\PYG{n}{v\PYGZus{}supply} \PYG{o}{=} \PYG{l+m+mi}{12}\PYG{p}{;} \PYG{n+nb+bp}{self}\PYG{o}{.}\PYG{n}{charge} \PYG{o}{\PYGZhy{}}\PYG{o}{=} \PYG{n+nb+bp}{self}\PYG{o}{.}\PYG{n}{tstep}
                \PYG{k}{else}\PYG{p}{:} \PYG{n+nb+bp}{self}\PYG{o}{.}\PYG{n}{set\PYGZus{}mode}\PYG{p}{(}\PYG{l+s+s2}{\PYGZdq{}}\PYG{l+s+s2}{no\PYGZus{}charge}\PYG{l+s+s2}{\PYGZdq{}}\PYG{p}{)}
                \PYG{k}{if} \PYG{n+nb+bp}{self}\PYG{o}{.}\PYG{n}{Control}\PYG{o}{.}\PYG{n}{power}\PYG{o}{==}\PYG{l+m+mi}{0}\PYG{p}{:} \PYG{n+nb+bp}{self}\PYG{o}{.}\PYG{n}{set\PYGZus{}mode}\PYG{p}{(}\PYG{l+s+s2}{\PYGZdq{}}\PYG{l+s+s2}{standby}\PYG{l+s+s2}{\PYGZdq{}}\PYG{p}{)}
            \PYG{k}{elif} \PYG{n+nb+bp}{self}\PYG{o}{.}\PYG{n}{in\PYGZus{}mode}\PYG{p}{(}\PYG{l+s+s2}{\PYGZdq{}}\PYG{l+s+s2}{short}\PYG{l+s+s2}{\PYGZdq{}}\PYG{p}{)}\PYG{p}{:}     \PYG{n+nb+bp}{self}\PYG{o}{.}\PYG{n}{EE}\PYG{o}{.}\PYG{n}{v\PYGZus{}supply} \PYG{o}{=} \PYG{l+m+mi}{100}\PYG{p}{;} \PYG{n+nb+bp}{self}\PYG{o}{.}\PYG{n}{charge} \PYG{o}{=} \PYG{l+m+mi}{0}
            \PYG{k}{elif} \PYG{n+nb+bp}{self}\PYG{o}{.}\PYG{n}{in\PYGZus{}mode}\PYG{p}{(}\PYG{l+s+s2}{\PYGZdq{}}\PYG{l+s+s2}{no\PYGZus{}charge}\PYG{l+s+s2}{\PYGZdq{}}\PYG{p}{)}\PYG{p}{:} \PYG{n+nb+bp}{self}\PYG{o}{.}\PYG{n}{EE}\PYG{o}{.}\PYG{n}{v\PYGZus{}supply}\PYG{o}{=}\PYG{l+m+mi}{0}
\end{sphinxVerbatim}
}

\end{sphinxuseclass}
\end{sphinxuseclass}
\sphinxAtStartPar
In addition to operational modes, the Battery Function additionally specifies an \sphinxstyleemphasis{opportunity vector} which defines when each fault mode is likely to occur (and thus be injected in the model). In this case, \sphinxcode{\sphinxupquote{no\_charge}} can only occur during standby mode, while a \sphinxcode{\sphinxupquote{short}} can only occur during the supply mode. The \sphinxcode{\sphinxupquote{key\_phases\_by='self'}} option specifies that these phases are \sphinxstyleemphasis{internal} to the function. Opporunity vectors can also be keyed to phases in other functions using
\sphinxcode{\sphinxupquote{key\_phases\_by='fxnname'}} and \sphinxcode{\sphinxupquote{key\_phases\_by='global'}}. There are several ways to define this opportunity vector, which will be described later.

\sphinxAtStartPar
This model also uses the \sphinxcode{\sphinxupquote{exclusive}} option, which specifies that fault modes cannot co\sphinxhyphen{}occur with operational modes. That is, instead of \sphinxstyleemphasis{modifying} individual mode behaviors, fault modes in this function instead cause the system to enter a new mode with different defined mode behaviors.

\begin{sphinxuseclass}{nbinput}
\begin{sphinxuseclass}{nblast}
{
\sphinxsetup{VerbatimColor={named}{nbsphinx-code-bg}}
\sphinxsetup{VerbatimBorderColor={named}{nbsphinx-code-border}}
\begin{sphinxVerbatim}[commandchars=\\\{\}]
\llap{\color{nbsphinxin}[8]:\,\hspace{\fboxrule}\hspace{\fboxsep}}\PYG{k+kn}{import} \PYG{n+nn}{fmdtools}\PYG{n+nn}{.}\PYG{n+nn}{resultdisp} \PYG{k}{as} \PYG{n+nn}{rd}
\PYG{k}{class} \PYG{n+nc}{Rover}\PYG{p}{(}\PYG{n}{Model}\PYG{p}{)}\PYG{p}{:}
    \PYG{k}{def} \PYG{n+nf+fm}{\PYGZus{}\PYGZus{}init\PYGZus{}\PYGZus{}}\PYG{p}{(}\PYG{n+nb+bp}{self}\PYG{p}{,} \PYG{n}{params}\PYG{o}{=}\PYG{p}{\PYGZob{}}\PYG{p}{\PYGZcb{}}\PYG{p}{,}\PYGZbs{}
                 \PYG{n}{modelparams}\PYG{o}{=}\PYG{p}{\PYGZob{}}\PYG{l+s+s1}{\PYGZsq{}}\PYG{l+s+s1}{times}\PYG{l+s+s1}{\PYGZsq{}}\PYG{p}{:}\PYG{p}{[}\PYG{l+m+mi}{0}\PYG{p}{,}\PYG{l+m+mi}{60}\PYG{p}{]}\PYG{p}{,} \PYG{l+s+s1}{\PYGZsq{}}\PYG{l+s+s1}{tstep}\PYG{l+s+s1}{\PYGZsq{}}\PYG{p}{:}\PYG{l+m+mi}{1}\PYG{p}{,} \PYG{l+s+s1}{\PYGZsq{}}\PYG{l+s+s1}{phases}\PYG{l+s+s1}{\PYGZsq{}}\PYG{p}{:} \PYG{p}{\PYGZob{}}\PYG{l+s+s1}{\PYGZsq{}}\PYG{l+s+s1}{firsthalf}\PYG{l+s+s1}{\PYGZsq{}}\PYG{p}{:}\PYG{p}{[}\PYG{l+m+mi}{0}\PYG{p}{,}\PYG{l+m+mi}{30}\PYG{p}{]}\PYG{p}{,} \PYG{l+s+s1}{\PYGZsq{}}\PYG{l+s+s1}{secondhalf}\PYG{l+s+s1}{\PYGZsq{}}\PYG{p}{:}\PYG{p}{[}\PYG{l+m+mi}{31}\PYG{p}{,}\PYG{l+m+mi}{60}\PYG{p}{]}\PYG{p}{\PYGZcb{}}\PYG{p}{\PYGZcb{}}\PYG{p}{,}\PYGZbs{}
                 \PYG{n}{valparams}\PYG{o}{=}\PYG{p}{\PYGZob{}}\PYG{p}{\PYGZcb{}}\PYG{p}{)}\PYG{p}{:}
        \PYG{n+nb}{super}\PYG{p}{(}\PYG{p}{)}\PYG{o}{.}\PYG{n+nf+fm}{\PYGZus{}\PYGZus{}init\PYGZus{}\PYGZus{}}\PYG{p}{(}\PYG{n}{params}\PYG{p}{,} \PYG{n}{modelparams}\PYG{p}{,} \PYG{n}{valparams}\PYG{p}{)}

        \PYG{n+nb+bp}{self}\PYG{o}{.}\PYG{n}{add\PYGZus{}flow}\PYG{p}{(}\PYG{l+s+s1}{\PYGZsq{}}\PYG{l+s+s1}{Ground}\PYG{l+s+s1}{\PYGZsq{}}\PYG{p}{,} \PYG{p}{\PYGZob{}}\PYG{l+s+s1}{\PYGZsq{}}\PYG{l+s+s1}{x}\PYG{l+s+s1}{\PYGZsq{}}\PYG{p}{:}\PYG{l+m+mf}{0.0}\PYG{p}{,}\PYG{l+s+s1}{\PYGZsq{}}\PYG{l+s+s1}{y}\PYG{l+s+s1}{\PYGZsq{}}\PYG{p}{:}\PYG{l+m+mf}{0.0}\PYG{p}{,} \PYG{l+s+s1}{\PYGZsq{}}\PYG{l+s+s1}{dir}\PYG{l+s+s1}{\PYGZsq{}}\PYG{p}{:}\PYG{l+m+mf}{0.0}\PYG{p}{,} \PYG{l+s+s1}{\PYGZsq{}}\PYG{l+s+s1}{vel}\PYG{l+s+s1}{\PYGZsq{}}\PYG{p}{:}\PYG{l+m+mf}{0.0}\PYG{p}{\PYGZcb{}}\PYG{p}{)}
        \PYG{n+nb+bp}{self}\PYG{o}{.}\PYG{n}{add\PYGZus{}flow}\PYG{p}{(}\PYG{l+s+s1}{\PYGZsq{}}\PYG{l+s+s1}{EE}\PYG{l+s+s1}{\PYGZsq{}}\PYG{p}{,} \PYG{p}{\PYGZob{}}\PYG{l+s+s1}{\PYGZsq{}}\PYG{l+s+s1}{v\PYGZus{}supply}\PYG{l+s+s1}{\PYGZsq{}}\PYG{p}{:}\PYG{l+m+mf}{0.0}\PYG{p}{,} \PYG{l+s+s1}{\PYGZsq{}}\PYG{l+s+s1}{a\PYGZus{}supply}\PYG{l+s+s1}{\PYGZsq{}}\PYG{p}{:}\PYG{l+m+mf}{0.0}\PYG{p}{\PYGZcb{}}\PYG{p}{)}
        \PYG{n+nb+bp}{self}\PYG{o}{.}\PYG{n}{add\PYGZus{}flow}\PYG{p}{(}\PYG{l+s+s1}{\PYGZsq{}}\PYG{l+s+s1}{Control}\PYG{l+s+s1}{\PYGZsq{}}\PYG{p}{,} \PYG{p}{\PYGZob{}}\PYG{l+s+s1}{\PYGZsq{}}\PYG{l+s+s1}{dir}\PYG{l+s+s1}{\PYGZsq{}}\PYG{p}{:}\PYG{l+m+mf}{0.0}\PYG{p}{,} \PYG{l+s+s1}{\PYGZsq{}}\PYG{l+s+s1}{vel}\PYG{l+s+s1}{\PYGZsq{}}\PYG{p}{:}\PYG{l+m+mf}{0.0}\PYG{p}{,} \PYG{l+s+s1}{\PYGZsq{}}\PYG{l+s+s1}{power}\PYG{l+s+s1}{\PYGZsq{}}\PYG{p}{:}\PYG{l+m+mf}{0.0}\PYG{p}{\PYGZcb{}}\PYG{p}{)}

        \PYG{n+nb+bp}{self}\PYG{o}{.}\PYG{n}{add\PYGZus{}fxn}\PYG{p}{(}\PYG{l+s+s2}{\PYGZdq{}}\PYG{l+s+s2}{Control\PYGZus{}Rover}\PYG{l+s+s2}{\PYGZdq{}}\PYG{p}{,}\PYG{p}{[}\PYG{l+s+s2}{\PYGZdq{}}\PYG{l+s+s2}{EE}\PYG{l+s+s2}{\PYGZdq{}}\PYG{p}{,} \PYG{l+s+s2}{\PYGZdq{}}\PYG{l+s+s2}{Control}\PYG{l+s+s2}{\PYGZdq{}}\PYG{p}{]}\PYG{p}{,} \PYG{n}{fclass}\PYG{o}{=}\PYG{n}{Control\PYGZus{}Rover}\PYG{p}{)}
        \PYG{n+nb+bp}{self}\PYG{o}{.}\PYG{n}{add\PYGZus{}fxn}\PYG{p}{(}\PYG{l+s+s2}{\PYGZdq{}}\PYG{l+s+s2}{Move\PYGZus{}Rover}\PYG{l+s+s2}{\PYGZdq{}}\PYG{p}{,} \PYG{p}{[}\PYG{l+s+s2}{\PYGZdq{}}\PYG{l+s+s2}{Ground}\PYG{l+s+s2}{\PYGZdq{}}\PYG{p}{,}\PYG{l+s+s2}{\PYGZdq{}}\PYG{l+s+s2}{EE}\PYG{l+s+s2}{\PYGZdq{}}\PYG{p}{,} \PYG{l+s+s2}{\PYGZdq{}}\PYG{l+s+s2}{Control}\PYG{l+s+s2}{\PYGZdq{}}\PYG{p}{]}\PYG{p}{,} \PYG{n}{fclass} \PYG{o}{=} \PYG{n}{Move\PYGZus{}Rover}\PYG{p}{)}
        \PYG{n+nb+bp}{self}\PYG{o}{.}\PYG{n}{add\PYGZus{}fxn}\PYG{p}{(}\PYG{l+s+s2}{\PYGZdq{}}\PYG{l+s+s2}{Store\PYGZus{}Energy}\PYG{l+s+s2}{\PYGZdq{}}\PYG{p}{,} \PYG{p}{[}\PYG{l+s+s2}{\PYGZdq{}}\PYG{l+s+s2}{EE}\PYG{l+s+s2}{\PYGZdq{}}\PYG{p}{,} \PYG{l+s+s2}{\PYGZdq{}}\PYG{l+s+s2}{Control}\PYG{l+s+s2}{\PYGZdq{}}\PYG{p}{]}\PYG{p}{,} \PYG{n}{Store\PYGZus{}Energy}\PYG{p}{)}

        \PYG{n+nb+bp}{self}\PYG{o}{.}\PYG{n}{build\PYGZus{}model}\PYG{p}{(}\PYG{p}{)}
    \PYG{k}{def} \PYG{n+nf}{find\PYGZus{}classification}\PYG{p}{(}\PYG{n+nb+bp}{self}\PYG{p}{,} \PYG{n}{scen}\PYG{p}{,} \PYG{n}{mdlhists}\PYG{p}{)}\PYG{p}{:}
        \PYG{n}{repcost} \PYG{o}{=} \PYG{n+nb+bp}{self}\PYG{o}{.}\PYG{n}{calc\PYGZus{}repaircost}\PYG{p}{(}\PYG{p}{)}
        \PYG{n}{reshist}\PYG{p}{,} \PYG{n}{diff}\PYG{p}{,} \PYG{n}{summary} \PYG{o}{=} \PYG{n}{rd}\PYG{o}{.}\PYG{n}{process}\PYG{o}{.}\PYG{n}{hist}\PYG{p}{(}\PYG{n}{mdlhists}\PYG{p}{)}
        \PYG{n}{losscost} \PYG{o}{=} \PYG{n+nb}{sum}\PYG{p}{(}\PYG{p}{[}\PYG{n+nb}{bool}\PYG{p}{(}\PYG{n}{i}\PYG{p}{)} \PYG{k}{for} \PYG{n}{i} \PYG{o+ow}{in} \PYG{n}{reshist}\PYG{p}{[}\PYG{l+s+s1}{\PYGZsq{}}\PYG{l+s+s1}{stats}\PYG{l+s+s1}{\PYGZsq{}}\PYG{p}{]}\PYG{p}{[}\PYG{l+s+s1}{\PYGZsq{}}\PYG{l+s+s1}{total faults}\PYG{l+s+s1}{\PYGZsq{}}\PYG{p}{]}\PYG{p}{]}\PYG{p}{)}
        \PYG{n}{totcost}\PYG{o}{=}\PYG{n}{repcost}\PYG{o}{+}\PYG{n}{losscost}
        \PYG{k}{return} \PYG{p}{\PYGZob{}}\PYG{l+s+s1}{\PYGZsq{}}\PYG{l+s+s1}{cost}\PYG{l+s+s1}{\PYGZsq{}}\PYG{p}{:}\PYG{n}{totcost}\PYG{p}{,} \PYG{l+s+s1}{\PYGZsq{}}\PYG{l+s+s1}{rate}\PYG{l+s+s1}{\PYGZsq{}}\PYG{p}{:}\PYG{n}{scen}\PYG{p}{[}\PYG{l+s+s1}{\PYGZsq{}}\PYG{l+s+s1}{properties}\PYG{l+s+s1}{\PYGZsq{}}\PYG{p}{]}\PYG{p}{[}\PYG{l+s+s1}{\PYGZsq{}}\PYG{l+s+s1}{rate}\PYG{l+s+s1}{\PYGZsq{}}\PYG{p}{]}\PYG{p}{,} \PYG{l+s+s1}{\PYGZsq{}}\PYG{l+s+s1}{expected cost}\PYG{l+s+s1}{\PYGZsq{}}\PYG{p}{:}\PYG{n}{totcost}\PYG{o}{*}\PYG{n}{scen}\PYG{p}{[}\PYG{l+s+s1}{\PYGZsq{}}\PYG{l+s+s1}{properties}\PYG{l+s+s1}{\PYGZsq{}}\PYG{p}{]}\PYG{p}{[}\PYG{l+s+s1}{\PYGZsq{}}\PYG{l+s+s1}{rate}\PYG{l+s+s1}{\PYGZsq{}}\PYG{p}{]}\PYG{p}{\PYGZcb{}}
\end{sphinxVerbatim}
}

\end{sphinxuseclass}
\end{sphinxuseclass}
\sphinxAtStartPar
\sphinxcode{\sphinxupquote{"global"}} phases in the model are defined using the ‘phases’ string in model parameters. Phases define distinct periods or activities of operation when the model may be in a different state (resulting in different potential failure mode effects and opportunities for failure modes to be entered).


\subsubsection{Setting up an approach}
\label{\detokenize{docs/Approach_Use-Cases:Setting-up-an-approach}}
\sphinxAtStartPar
Because this model has an opportunity vector which is keyed by its operational modes, the history of modes needs to be generated for the model to set up the sample approach–otherwise this information cannot be used, as shown below:

\begin{sphinxuseclass}{nbinput}
\begin{sphinxuseclass}{nblast}
{
\sphinxsetup{VerbatimColor={named}{nbsphinx-code-bg}}
\sphinxsetup{VerbatimBorderColor={named}{nbsphinx-code-border}}
\begin{sphinxVerbatim}[commandchars=\\\{\}]
\llap{\color{nbsphinxin}[9]:\,\hspace{\fboxrule}\hspace{\fboxsep}}\PYG{n}{mdl} \PYG{o}{=} \PYG{n}{Rover}\PYG{p}{(}\PYG{p}{)}
\PYG{n}{app} \PYG{o}{=} \PYG{n}{SampleApproach}\PYG{p}{(}\PYG{n}{mdl}\PYG{p}{)}
\end{sphinxVerbatim}
}

\end{sphinxuseclass}
\end{sphinxuseclass}
\sphinxAtStartPar
To form this history, the model is run in the nominal scenario, as shown.

\begin{sphinxuseclass}{nbinput}
{
\sphinxsetup{VerbatimColor={named}{nbsphinx-code-bg}}
\sphinxsetup{VerbatimBorderColor={named}{nbsphinx-code-border}}
\begin{sphinxVerbatim}[commandchars=\\\{\}]
\llap{\color{nbsphinxin}[10]:\,\hspace{\fboxrule}\hspace{\fboxsep}}\PYG{n}{endclass} \PYG{p}{,} \PYG{n}{resgraph}\PYG{p}{,}\PYG{n}{mdlhist} \PYG{o}{=} \PYG{n}{prop}\PYG{o}{.}\PYG{n}{nominal}\PYG{p}{(}\PYG{n}{mdl}\PYG{p}{)}
\PYG{n}{mdlhist}\PYG{p}{[}\PYG{l+s+s1}{\PYGZsq{}}\PYG{l+s+s1}{functions}\PYG{l+s+s1}{\PYGZsq{}}\PYG{p}{]}\PYG{p}{[}\PYG{l+s+s1}{\PYGZsq{}}\PYG{l+s+s1}{Store\PYGZus{}Energy}\PYG{l+s+s1}{\PYGZsq{}}\PYG{p}{]}
\end{sphinxVerbatim}
}

\end{sphinxuseclass}
\begin{sphinxuseclass}{nboutput}
\begin{sphinxuseclass}{nblast}
{

\kern-\sphinxverbatimsmallskipamount\kern-\baselineskip
\kern+\FrameHeightAdjust\kern-\fboxrule
\vspace{\nbsphinxcodecellspacing}

\sphinxsetup{VerbatimColor={named}{white}}
\sphinxsetup{VerbatimBorderColor={named}{nbsphinx-code-border}}
\begin{sphinxuseclass}{output_area}
\begin{sphinxuseclass}{}


\begin{sphinxVerbatim}[commandchars=\\\{\}]
\llap{\color{nbsphinxout}[10]:\,\hspace{\fboxrule}\hspace{\fboxsep}}\{'faults': array(['nom', 'nom', 'nom', 'nom', 'nom', 'nom', 'nom', 'nom', 'nom',
        'nom', 'nom', 'nom', 'nom', 'nom', 'nom', 'nom', 'nom', 'nom',
        'nom', 'nom', 'nom', 'nom', 'nom', 'nom', 'nom', 'nom', 'nom',
        'nom', 'nom', 'nom', 'nom', 'nom', 'nom', 'nom', 'nom', 'nom',
        'nom', 'nom', 'nom', 'nom', 'nom', 'nom', 'nom', 'nom', 'nom',
        'nom', 'nom', 'nom', 'nom', 'nom', 'nom', 'nom', 'nom', 'nom',
        'nom', 'nom', 'nom', 'nom', 'nom', 'nom', 'nom'], dtype='<U9'),
 'charge': array([100, 100, 100, 100, 100, 100,  99,  98,  97,  96,  95,  94,  93,
         92,  91,  90,  89,  88,  87,  86,  85,  84,  83,  82,  81,  80,
         79,  78,  77,  76,  75,  74,  73,  72,  71,  70,  69,  68,  67,
         66,  65,  64,  63,  62,  61,  60,  59,  58,  57,  56,  55,  55,
         55,  55,  55,  55,  55,  55,  55,  55,  55]),
 'mode': array(['standby', 'standby', 'standby', 'standby', 'standby', 'supply',
        'supply', 'supply', 'supply', 'supply', 'supply', 'supply',
        'supply', 'supply', 'supply', 'supply', 'supply', 'supply',
        'supply', 'supply', 'supply', 'supply', 'supply', 'supply',
        'supply', 'supply', 'supply', 'supply', 'supply', 'supply',
        'supply', 'supply', 'supply', 'supply', 'supply', 'supply',
        'supply', 'supply', 'supply', 'supply', 'supply', 'supply',
        'supply', 'supply', 'supply', 'supply', 'supply', 'supply',
        'supply', 'supply', 'standby', 'standby', 'standby', 'standby',
        'standby', 'standby', 'standby', 'standby', 'standby', 'standby',
        'standby'], dtype='<U9')\}
\end{sphinxVerbatim}



\end{sphinxuseclass}
\end{sphinxuseclass}
}

\end{sphinxuseclass}
\end{sphinxuseclass}
\sphinxAtStartPar
To get the phase information, the ‘process.modephases’ method is used.

\begin{sphinxuseclass}{nbinput}
\begin{sphinxuseclass}{nblast}
{
\sphinxsetup{VerbatimColor={named}{nbsphinx-code-bg}}
\sphinxsetup{VerbatimBorderColor={named}{nbsphinx-code-border}}
\begin{sphinxVerbatim}[commandchars=\\\{\}]
\llap{\color{nbsphinxin}[11]:\,\hspace{\fboxrule}\hspace{\fboxsep}}\PYG{n}{phases}\PYG{p}{,} \PYG{n}{modephases} \PYG{o}{=} \PYG{n}{rd}\PYG{o}{.}\PYG{n}{process}\PYG{o}{.}\PYG{n}{modephases}\PYG{p}{(}\PYG{n}{mdlhist}\PYG{p}{)}
\end{sphinxVerbatim}
}

\end{sphinxuseclass}
\end{sphinxuseclass}
\sphinxAtStartPar
Which returns \sphinxcode{\sphinxupquote{phases}}, the phases of operation for each function where operational modes were defined:

\begin{sphinxuseclass}{nbinput}
{
\sphinxsetup{VerbatimColor={named}{nbsphinx-code-bg}}
\sphinxsetup{VerbatimBorderColor={named}{nbsphinx-code-border}}
\begin{sphinxVerbatim}[commandchars=\\\{\}]
\llap{\color{nbsphinxin}[12]:\,\hspace{\fboxrule}\hspace{\fboxsep}}\PYG{n}{phases}
\end{sphinxVerbatim}
}

\end{sphinxuseclass}
\begin{sphinxuseclass}{nboutput}
\begin{sphinxuseclass}{nblast}
{

\kern-\sphinxverbatimsmallskipamount\kern-\baselineskip
\kern+\FrameHeightAdjust\kern-\fboxrule
\vspace{\nbsphinxcodecellspacing}

\sphinxsetup{VerbatimColor={named}{white}}
\sphinxsetup{VerbatimBorderColor={named}{nbsphinx-code-border}}
\begin{sphinxuseclass}{output_area}
\begin{sphinxuseclass}{}


\begin{sphinxVerbatim}[commandchars=\\\{\}]
\llap{\color{nbsphinxout}[12]:\,\hspace{\fboxrule}\hspace{\fboxsep}}\{'Control\_Rover': \{'standby': [0, 4], 'drive': [5, 49], 'standby1': [50, 60]\},
 'Store\_Energy': \{'standby': [0, 4], 'supply': [5, 49], 'standby1': [50, 60]\}\}
\end{sphinxVerbatim}



\end{sphinxuseclass}
\end{sphinxuseclass}
}

\end{sphinxuseclass}
\end{sphinxuseclass}
\sphinxAtStartPar
as well as \sphinxcode{\sphinxupquote{modephases}}, which lists the phases associated with each mode.

\begin{sphinxuseclass}{nbinput}
{
\sphinxsetup{VerbatimColor={named}{nbsphinx-code-bg}}
\sphinxsetup{VerbatimBorderColor={named}{nbsphinx-code-border}}
\begin{sphinxVerbatim}[commandchars=\\\{\}]
\llap{\color{nbsphinxin}[13]:\,\hspace{\fboxrule}\hspace{\fboxsep}}\PYG{n}{modephases}
\end{sphinxVerbatim}
}

\end{sphinxuseclass}
\begin{sphinxuseclass}{nboutput}
\begin{sphinxuseclass}{nblast}
{

\kern-\sphinxverbatimsmallskipamount\kern-\baselineskip
\kern+\FrameHeightAdjust\kern-\fboxrule
\vspace{\nbsphinxcodecellspacing}

\sphinxsetup{VerbatimColor={named}{white}}
\sphinxsetup{VerbatimBorderColor={named}{nbsphinx-code-border}}
\begin{sphinxuseclass}{output_area}
\begin{sphinxuseclass}{}


\begin{sphinxVerbatim}[commandchars=\\\{\}]
\llap{\color{nbsphinxout}[13]:\,\hspace{\fboxrule}\hspace{\fboxsep}}\{'Control\_Rover': \{'standby': \{'standby', 'standby1'\}, 'drive': \{'drive'\}\},
 'Store\_Energy': \{'standby': \{'standby', 'standby1'\}, 'supply': \{'supply'\}\}\}
\end{sphinxVerbatim}



\end{sphinxuseclass}
\end{sphinxuseclass}
}

\end{sphinxuseclass}
\end{sphinxuseclass}
\sphinxAtStartPar
These phases can be visualized with \sphinxcode{\sphinxupquote{rd.plot.phases}}

\begin{sphinxuseclass}{nbinput}
{
\sphinxsetup{VerbatimColor={named}{nbsphinx-code-bg}}
\sphinxsetup{VerbatimBorderColor={named}{nbsphinx-code-border}}
\begin{sphinxVerbatim}[commandchars=\\\{\}]
\llap{\color{nbsphinxin}[14]:\,\hspace{\fboxrule}\hspace{\fboxsep}}\PYG{n}{fig} \PYG{o}{=} \PYG{n}{rd}\PYG{o}{.}\PYG{n}{plot}\PYG{o}{.}\PYG{n}{phases}\PYG{p}{(}\PYG{n}{phases}\PYG{p}{,}\PYG{n}{modephases}\PYG{p}{)}
\end{sphinxVerbatim}
}

\end{sphinxuseclass}
\begin{sphinxuseclass}{nboutput}
\begin{sphinxuseclass}{nblast}
\hrule height -\fboxrule\relax
\vspace{\nbsphinxcodecellspacing}

\makeatletter\setbox\nbsphinxpromptbox\box\voidb@x\makeatother

\begin{nbsphinxfancyoutput}

\begin{sphinxuseclass}{output_area}
\begin{sphinxuseclass}{}
\noindent\sphinxincludegraphics[width=426\sphinxpxdimen,height=286\sphinxpxdimen]{{docs_Approach_Use-Cases_26_0}.png}

\end{sphinxuseclass}
\end{sphinxuseclass}
\end{nbsphinxfancyoutput}

\end{sphinxuseclass}
\end{sphinxuseclass}
\sphinxAtStartPar
To correctly sample according to the fault model’s intent (that each fault can occur within each \sphinxcode{\sphinxupquote{mode}}), SampleApproach is given both \sphinxcode{\sphinxupquote{phases}} and \sphinxcode{\sphinxupquote{modephases}}, as shown below.

\begin{sphinxuseclass}{nbinput}
{
\sphinxsetup{VerbatimColor={named}{nbsphinx-code-bg}}
\sphinxsetup{VerbatimBorderColor={named}{nbsphinx-code-border}}
\begin{sphinxVerbatim}[commandchars=\\\{\}]
\llap{\color{nbsphinxin}[15]:\,\hspace{\fboxrule}\hspace{\fboxsep}}\PYG{n}{app\PYGZus{}correct} \PYG{o}{=} \PYG{n}{SampleApproach}\PYG{p}{(}\PYG{n}{mdl}\PYG{p}{,} \PYG{n}{phases}\PYG{o}{=}\PYG{n}{phases}\PYG{p}{,} \PYG{n}{modephases}\PYG{o}{=}\PYG{n}{modephases}\PYG{p}{)}
\PYG{n}{app\PYGZus{}correct}\PYG{o}{.}\PYG{n}{sampletimes}
\end{sphinxVerbatim}
}

\end{sphinxuseclass}
\begin{sphinxuseclass}{nboutput}
\begin{sphinxuseclass}{nblast}
{

\kern-\sphinxverbatimsmallskipamount\kern-\baselineskip
\kern+\FrameHeightAdjust\kern-\fboxrule
\vspace{\nbsphinxcodecellspacing}

\sphinxsetup{VerbatimColor={named}{white}}
\sphinxsetup{VerbatimBorderColor={named}{nbsphinx-code-border}}
\begin{sphinxuseclass}{output_area}
\begin{sphinxuseclass}{}


\begin{sphinxVerbatim}[commandchars=\\\{\}]
\llap{\color{nbsphinxout}[15]:\,\hspace{\fboxrule}\hspace{\fboxsep}}\{('global',
  'firsthalf'): \{14: [('Control\_Rover', 'no\_con'),
   ('Move\_Rover', 'mech\_loss'),
   ('Move\_Rover', 'elec\_open')]\},
 ('global',
  'secondhalf'): \{45: [('Control\_Rover', 'no\_con'),
   ('Move\_Rover', 'mech\_loss'),
   ('Move\_Rover', 'elec\_open')]\},
 ('Store\_Energy', 'standby'): \{2: [('Store\_Energy', 'no\_charge')]\},
 ('Store\_Energy', 'standby1'): \{54: [('Store\_Energy', 'no\_charge')]\},
 ('Store\_Energy', 'supply'): \{27: [('Store\_Energy', 'short')]\}\}
\end{sphinxVerbatim}



\end{sphinxuseclass}
\end{sphinxuseclass}
}

\end{sphinxuseclass}
\end{sphinxuseclass}
\begin{sphinxuseclass}{nbinput}
{
\sphinxsetup{VerbatimColor={named}{nbsphinx-code-bg}}
\sphinxsetup{VerbatimBorderColor={named}{nbsphinx-code-border}}
\begin{sphinxVerbatim}[commandchars=\\\{\}]
\llap{\color{nbsphinxin}[16]:\,\hspace{\fboxrule}\hspace{\fboxsep}}\PYG{n+nb}{type}\PYG{p}{(}\PYG{n+nb}{tuple}\PYG{p}{(}\PYG{n}{phases}\PYG{o}{.}\PYG{n}{values}\PYG{p}{(}\PYG{p}{)}\PYG{p}{)}\PYG{p}{[}\PYG{l+m+mi}{0}\PYG{p}{]}\PYG{p}{)}
\end{sphinxVerbatim}
}

\end{sphinxuseclass}
\begin{sphinxuseclass}{nboutput}
\begin{sphinxuseclass}{nblast}
{

\kern-\sphinxverbatimsmallskipamount\kern-\baselineskip
\kern+\FrameHeightAdjust\kern-\fboxrule
\vspace{\nbsphinxcodecellspacing}

\sphinxsetup{VerbatimColor={named}{white}}
\sphinxsetup{VerbatimBorderColor={named}{nbsphinx-code-border}}
\begin{sphinxuseclass}{output_area}
\begin{sphinxuseclass}{}


\begin{sphinxVerbatim}[commandchars=\\\{\}]
\llap{\color{nbsphinxout}[16]:\,\hspace{\fboxrule}\hspace{\fboxsep}}dict
\end{sphinxVerbatim}



\end{sphinxuseclass}
\end{sphinxuseclass}
}

\end{sphinxuseclass}
\end{sphinxuseclass}
\begin{sphinxuseclass}{nbinput}
{
\sphinxsetup{VerbatimColor={named}{nbsphinx-code-bg}}
\sphinxsetup{VerbatimBorderColor={named}{nbsphinx-code-border}}
\begin{sphinxVerbatim}[commandchars=\\\{\}]
\llap{\color{nbsphinxin}[17]:\,\hspace{\fboxrule}\hspace{\fboxsep}}\PYG{n}{app\PYGZus{}incorrect} \PYG{o}{=} \PYG{n}{SampleApproach}\PYG{p}{(}\PYG{n}{mdl}\PYG{p}{,} \PYG{n}{phases}\PYG{o}{=}\PYG{n}{phases}\PYG{p}{)}
\PYG{n}{app\PYGZus{}incorrect}\PYG{o}{.}\PYG{n}{sampletimes}
\end{sphinxVerbatim}
}

\end{sphinxuseclass}
\begin{sphinxuseclass}{nboutput}
\begin{sphinxuseclass}{nblast}
{

\kern-\sphinxverbatimsmallskipamount\kern-\baselineskip
\kern+\FrameHeightAdjust\kern-\fboxrule
\vspace{\nbsphinxcodecellspacing}

\sphinxsetup{VerbatimColor={named}{white}}
\sphinxsetup{VerbatimBorderColor={named}{nbsphinx-code-border}}
\begin{sphinxuseclass}{output_area}
\begin{sphinxuseclass}{}


\begin{sphinxVerbatim}[commandchars=\\\{\}]
\llap{\color{nbsphinxout}[17]:\,\hspace{\fboxrule}\hspace{\fboxsep}}\{('global',
  'firsthalf'): \{14: [('Control\_Rover', 'no\_con'),
   ('Move\_Rover', 'mech\_loss'),
   ('Move\_Rover', 'elec\_open')]\},
 ('global',
  'secondhalf'): \{45: [('Control\_Rover', 'no\_con'),
   ('Move\_Rover', 'mech\_loss'),
   ('Move\_Rover', 'elec\_open')]\},
 ('Store\_Energy', 'standby'): \{2: [('Store\_Energy', 'no\_charge')]\},
 ('Store\_Energy', 'supply'): \{27: [('Store\_Energy', 'short')]\}\}
\end{sphinxVerbatim}



\end{sphinxuseclass}
\end{sphinxuseclass}
}

\end{sphinxuseclass}
\end{sphinxuseclass}
\sphinxAtStartPar
As shown, in the first sampleapproach, only the first phases of the mode are sampled, and not the second phases, because there is nothing to tell the approach which phases are associated with which mode, and the opportunity vector was keyed by modes. However, there are cases where it may not be necessary to provide \sphinxcode{\sphinxupquote{modephases}} when providing , specifically: \sphinxhyphen{} If the opportunity vector is defined in terms of local \sphinxcode{\sphinxupquote{phases}} directly instead of \sphinxcode{\sphinxupquote{modes}}, (e.g. by \sphinxcode{\sphinxupquote{\{standby}}:rate,
\sphinxcode{\sphinxupquote{supply}}:rate, \sphinxcode{\sphinxupquote{standby1}}:rate\})) \sphinxhyphen{} If only the first phase is of interest \sphinxhyphen{} If all fault modes are keyed by \sphinxcode{\sphinxupquote{global}} phases

\sphinxAtStartPar
Note how different opportunity vectors lead to different sample approaches:

\begin{sphinxuseclass}{nbinput}
{
\sphinxsetup{VerbatimColor={named}{nbsphinx-code-bg}}
\sphinxsetup{VerbatimBorderColor={named}{nbsphinx-code-border}}
\begin{sphinxVerbatim}[commandchars=\\\{\}]
\llap{\color{nbsphinxin}[18]:\,\hspace{\fboxrule}\hspace{\fboxsep}}\PYG{n}{app\PYGZus{}correct}\PYG{o}{.}\PYG{n}{sampletimes}
\end{sphinxVerbatim}
}

\end{sphinxuseclass}
\begin{sphinxuseclass}{nboutput}
\begin{sphinxuseclass}{nblast}
{

\kern-\sphinxverbatimsmallskipamount\kern-\baselineskip
\kern+\FrameHeightAdjust\kern-\fboxrule
\vspace{\nbsphinxcodecellspacing}

\sphinxsetup{VerbatimColor={named}{white}}
\sphinxsetup{VerbatimBorderColor={named}{nbsphinx-code-border}}
\begin{sphinxuseclass}{output_area}
\begin{sphinxuseclass}{}


\begin{sphinxVerbatim}[commandchars=\\\{\}]
\llap{\color{nbsphinxout}[18]:\,\hspace{\fboxrule}\hspace{\fboxsep}}\{('global',
  'firsthalf'): \{14: [('Control\_Rover', 'no\_con'),
   ('Move\_Rover', 'mech\_loss'),
   ('Move\_Rover', 'elec\_open')]\},
 ('global',
  'secondhalf'): \{45: [('Control\_Rover', 'no\_con'),
   ('Move\_Rover', 'mech\_loss'),
   ('Move\_Rover', 'elec\_open')]\},
 ('Store\_Energy', 'standby'): \{2: [('Store\_Energy', 'no\_charge')]\},
 ('Store\_Energy', 'standby1'): \{54: [('Store\_Energy', 'no\_charge')]\},
 ('Store\_Energy', 'supply'): \{27: [('Store\_Energy', 'short')]\}\}
\end{sphinxVerbatim}



\end{sphinxuseclass}
\end{sphinxuseclass}
}

\end{sphinxuseclass}
\end{sphinxuseclass}\begin{itemize}
\item {} 
\sphinxAtStartPar
\sphinxcode{\sphinxupquote{Move\_Rover}}, and \sphinxcode{\sphinxupquote{Control\_Rover}}, where no opportunity vector was provided, are sampled once in the middle of an \sphinxcode{\sphinxupquote{operating}} phase

\item {} 
\sphinxAtStartPar
\sphinxcode{\sphinxupquote{Store\_Energy}} is sampled once in each given phase (defined itself).

\item {} 
\sphinxAtStartPar
None of the modes are sampled during the global phases (although they could be, if that was provided in the function defintion)

\end{itemize}

\sphinxAtStartPar
Joint fault modes can additionally be inserted in the model. These fault modes are given their own phases for rates and injection times by finding the overlap between the operational phases of their constituent modes:

\begin{sphinxuseclass}{nbinput}
{
\sphinxsetup{VerbatimColor={named}{nbsphinx-code-bg}}
\sphinxsetup{VerbatimBorderColor={named}{nbsphinx-code-border}}
\begin{sphinxVerbatim}[commandchars=\\\{\}]
\llap{\color{nbsphinxin}[19]:\,\hspace{\fboxrule}\hspace{\fboxsep}}\PYG{n}{app\PYGZus{}joint} \PYG{o}{=} \PYG{n}{SampleApproach}\PYG{p}{(}\PYG{n}{mdl}\PYG{p}{,} \PYG{n}{phases}\PYG{o}{=}\PYG{n}{phases}\PYG{p}{,} \PYG{n}{modephases}\PYG{o}{=}\PYG{n}{modephases}\PYG{p}{,} \PYG{n}{jointfaults} \PYG{o}{=} \PYG{p}{\PYGZob{}}\PYG{l+s+s1}{\PYGZsq{}}\PYG{l+s+s1}{faults}\PYG{l+s+s1}{\PYGZsq{}}\PYG{p}{:}\PYG{l+m+mi}{2}\PYG{p}{\PYGZcb{}}\PYG{p}{)}
\PYG{n}{app\PYGZus{}joint}\PYG{o}{.}\PYG{n}{sampletimes}
\end{sphinxVerbatim}
}

\end{sphinxuseclass}
\begin{sphinxuseclass}{nboutput}
\begin{sphinxuseclass}{nblast}
{

\kern-\sphinxverbatimsmallskipamount\kern-\baselineskip
\kern+\FrameHeightAdjust\kern-\fboxrule
\vspace{\nbsphinxcodecellspacing}

\sphinxsetup{VerbatimColor={named}{white}}
\sphinxsetup{VerbatimBorderColor={named}{nbsphinx-code-border}}
\begin{sphinxuseclass}{output_area}
\begin{sphinxuseclass}{}


\begin{sphinxVerbatim}[commandchars=\\\{\}]
\llap{\color{nbsphinxout}[19]:\,\hspace{\fboxrule}\hspace{\fboxsep}}\{('global',
  'firsthalf'): \{14: [('Control\_Rover', 'no\_con'),
   ('Move\_Rover', 'mech\_loss'),
   ('Move\_Rover', 'elec\_open'),
   (('Control\_Rover', 'no\_con'), ('Move\_Rover', 'mech\_loss')),
   (('Control\_Rover', 'no\_con'), ('Move\_Rover', 'elec\_open'))]\},
 ('global',
  'secondhalf'): \{45: [('Control\_Rover', 'no\_con'),
   ('Move\_Rover', 'mech\_loss'),
   ('Move\_Rover', 'elec\_open'),
   (('Control\_Rover', 'no\_con'), ('Move\_Rover', 'mech\_loss')),
   (('Control\_Rover', 'no\_con'), ('Move\_Rover', 'elec\_open'))]\},
 ('Store\_Energy', 'standby'): \{2: [('Store\_Energy', 'no\_charge')]\},
 ('Store\_Energy', 'standby1'): \{54: [('Store\_Energy', 'no\_charge')]\},
 ('Store\_Energy', 'supply'): \{27: [('Store\_Energy', 'short')]\},
 (('global', 'firsthalf'),
  ('Store\_Energy',
   'standby')): \{2: [(('Control\_Rover', 'no\_con'),
    ('Store\_Energy', 'no\_charge')), (('Move\_Rover', 'mech\_loss'),
    ('Store\_Energy', 'no\_charge')), (('Move\_Rover', 'elec\_open'),
    ('Store\_Energy', 'no\_charge'))]\},
 (('global', 'secondhalf'),
  ('Store\_Energy',
   'standby1')): \{54: [(('Control\_Rover', 'no\_con'),
    ('Store\_Energy', 'no\_charge')), (('Move\_Rover', 'mech\_loss'),
    ('Store\_Energy', 'no\_charge')), (('Move\_Rover', 'elec\_open'),
    ('Store\_Energy', 'no\_charge'))]\},
 (('Store\_Energy', 'supply'),
  ('global',
   'firsthalf')): \{17: [(('Control\_Rover', 'no\_con'),
    ('Store\_Energy', 'short')), (('Move\_Rover', 'mech\_loss'),
    ('Store\_Energy', 'short')), (('Move\_Rover', 'elec\_open'),
    ('Store\_Energy', 'short'))]\},
 (('Store\_Energy', 'supply'),
  ('global',
   'secondhalf')): \{39: [(('Control\_Rover', 'no\_con'),
    ('Store\_Energy', 'short')), (('Move\_Rover', 'mech\_loss'),
    ('Store\_Energy', 'short')), (('Move\_Rover', 'elec\_open'),
    ('Store\_Energy', 'short'))]\}\}
\end{sphinxVerbatim}



\end{sphinxuseclass}
\end{sphinxuseclass}
}

\end{sphinxuseclass}
\end{sphinxuseclass}

\subsubsection{Propagating Faults}
\label{\detokenize{docs/Approach_Use-Cases:Propagating-Faults}}
\sphinxAtStartPar
Given the fault sampling approach, the faults can then be propagated through the model to get results. Note that these faults can be sampled in parallel if desired using a user\sphinxhyphen{}provided pool (see the parallel pool tutorial in the \sphinxcode{\sphinxupquote{\textbackslash{}pump example}} folder).

\begin{sphinxuseclass}{nbinput}
{
\sphinxsetup{VerbatimColor={named}{nbsphinx-code-bg}}
\sphinxsetup{VerbatimBorderColor={named}{nbsphinx-code-border}}
\begin{sphinxVerbatim}[commandchars=\\\{\}]
\llap{\color{nbsphinxin}[20]:\,\hspace{\fboxrule}\hspace{\fboxsep}}\PYG{n}{endclasses}\PYG{p}{,} \PYG{n}{mdlhists} \PYG{o}{=} \PYG{n}{prop}\PYG{o}{.}\PYG{n}{approach}\PYG{p}{(}\PYG{n}{mdl}\PYG{p}{,} \PYG{n}{app\PYGZus{}correct}\PYG{p}{)}
\end{sphinxVerbatim}
}

\end{sphinxuseclass}
\begin{sphinxuseclass}{nboutput}
\begin{sphinxuseclass}{nblast}
{

\kern-\sphinxverbatimsmallskipamount\kern-\baselineskip
\kern+\FrameHeightAdjust\kern-\fboxrule
\vspace{\nbsphinxcodecellspacing}

\sphinxsetup{VerbatimColor={named}{nbsphinx-stderr}}
\sphinxsetup{VerbatimBorderColor={named}{nbsphinx-code-border}}
\begin{sphinxuseclass}{output_area}
\begin{sphinxuseclass}{stderr}


\begin{sphinxVerbatim}[commandchars=\\\{\}]
SCENARIOS COMPLETE: 100\%|███████████████████████████████████████████████████████████████| 9/9 [00:00<00:00, 191.87it/s]
\end{sphinxVerbatim}



\end{sphinxuseclass}
\end{sphinxuseclass}
}

\end{sphinxuseclass}
\end{sphinxuseclass}
\begin{sphinxuseclass}{nbinput}
{
\sphinxsetup{VerbatimColor={named}{nbsphinx-code-bg}}
\sphinxsetup{VerbatimBorderColor={named}{nbsphinx-code-border}}
\begin{sphinxVerbatim}[commandchars=\\\{\}]
\llap{\color{nbsphinxin}[21]:\,\hspace{\fboxrule}\hspace{\fboxsep}}\PYG{n}{endclasses}
\end{sphinxVerbatim}
}

\end{sphinxuseclass}
\begin{sphinxuseclass}{nboutput}
\begin{sphinxuseclass}{nblast}
{

\kern-\sphinxverbatimsmallskipamount\kern-\baselineskip
\kern+\FrameHeightAdjust\kern-\fboxrule
\vspace{\nbsphinxcodecellspacing}

\sphinxsetup{VerbatimColor={named}{white}}
\sphinxsetup{VerbatimBorderColor={named}{nbsphinx-code-border}}
\begin{sphinxuseclass}{output_area}
\begin{sphinxuseclass}{}


\begin{sphinxVerbatim}[commandchars=\\\{\}]
\llap{\color{nbsphinxout}[21]:\,\hspace{\fboxrule}\hspace{\fboxsep}}\{'Control\_Rover no\_con, t=14': \{'cost': 247.0,
  'rate': 0.0015,
  'expected cost': 0.3705\},
 'Move\_Rover mech\_loss, t=14': \{'cost': 47.0,
  'rate': 7.5,
  'expected cost': 352.5\},
 'Move\_Rover elec\_open, t=14': \{'cost': 47.0,
  'rate': 7.5,
  'expected cost': 352.5\},
 'Control\_Rover no\_con, t=45': \{'cost': 216.0,
  'rate': 0.0014500000000000001,
  'expected cost': 0.31320000000000003\},
 'Move\_Rover mech\_loss, t=45': \{'cost': 16.0,
  'rate': 7.25,
  'expected cost': 116.0\},
 'Move\_Rover elec\_open, t=45': \{'cost': 16.0,
  'rate': 7.25,
  'expected cost': 116.0\},
 'Store\_Energy no\_charge, t=2': \{'cost': 159.0,
  'rate': 2e-05,
  'expected cost': 0.00318\},
 'Store\_Energy no\_charge, t=54': \{'cost': 107.0,
  'rate': 5e-05,
  'expected cost': 0.005350000000000001\},
 'Store\_Energy short, t=27': \{'cost': 134.0,
  'rate': 0.00044,
  'expected cost': 0.058960000000000005\},
 'nominal': \{'cost': 0.0, 'rate': 1.0, 'expected cost': 0.0\}\}
\end{sphinxVerbatim}



\end{sphinxuseclass}
\end{sphinxuseclass}
}

\end{sphinxuseclass}
\end{sphinxuseclass}
\sphinxAtStartPar
Note that the results here reflect the lack of information given for rates in the \sphinxcode{\sphinxupquote{Control\_Rover}} and \sphinxcode{\sphinxupquote{Move\_Rover}} functions (which default to a value of 1) and the rates do not correspond \sphinxstyleemphasis{direct} to the given rates because rates shown here are per\sphinxhyphen{}run rates which are spread over the model. Additionally, all scenarios default to a cost of 1 because of the lack of a classification function.

\sphinxAtStartPar
These responses can be visualized:

\begin{sphinxuseclass}{nbinput}
{
\sphinxsetup{VerbatimColor={named}{nbsphinx-code-bg}}
\sphinxsetup{VerbatimBorderColor={named}{nbsphinx-code-border}}
\begin{sphinxVerbatim}[commandchars=\\\{\}]
\llap{\color{nbsphinxin}[22]:\,\hspace{\fboxrule}\hspace{\fboxsep}}\PYG{n}{rd}\PYG{o}{.}\PYG{n}{plot}\PYG{o}{.}\PYG{n}{samplecosts}\PYG{p}{(}\PYG{n}{app\PYGZus{}correct}\PYG{p}{,} \PYG{n}{endclasses}\PYG{p}{)}
\end{sphinxVerbatim}
}

\end{sphinxuseclass}
\begin{sphinxuseclass}{nboutput}
\hrule height -\fboxrule\relax
\vspace{\nbsphinxcodecellspacing}

\makeatletter\setbox\nbsphinxpromptbox\box\voidb@x\makeatother

\begin{nbsphinxfancyoutput}

\begin{sphinxuseclass}{output_area}
\begin{sphinxuseclass}{}
\noindent\sphinxincludegraphics[width=424\sphinxpxdimen,height=280\sphinxpxdimen]{{docs_Approach_Use-Cases_42_0}.png}

\end{sphinxuseclass}
\end{sphinxuseclass}
\end{nbsphinxfancyoutput}

\end{sphinxuseclass}
\begin{sphinxuseclass}{nboutput}
\hrule height -\fboxrule\relax
\vspace{\nbsphinxcodecellspacing}

\makeatletter\setbox\nbsphinxpromptbox\box\voidb@x\makeatother

\begin{nbsphinxfancyoutput}

\begin{sphinxuseclass}{output_area}
\begin{sphinxuseclass}{}
\noindent\sphinxincludegraphics[width=424\sphinxpxdimen,height=280\sphinxpxdimen]{{docs_Approach_Use-Cases_42_1}.png}

\end{sphinxuseclass}
\end{sphinxuseclass}
\end{nbsphinxfancyoutput}

\end{sphinxuseclass}
\begin{sphinxuseclass}{nboutput}
\hrule height -\fboxrule\relax
\vspace{\nbsphinxcodecellspacing}

\makeatletter\setbox\nbsphinxpromptbox\box\voidb@x\makeatother

\begin{nbsphinxfancyoutput}

\begin{sphinxuseclass}{output_area}
\begin{sphinxuseclass}{}
\noindent\sphinxincludegraphics[width=424\sphinxpxdimen,height=280\sphinxpxdimen]{{docs_Approach_Use-Cases_42_2}.png}

\end{sphinxuseclass}
\end{sphinxuseclass}
\end{nbsphinxfancyoutput}

\end{sphinxuseclass}
\begin{sphinxuseclass}{nboutput}
\hrule height -\fboxrule\relax
\vspace{\nbsphinxcodecellspacing}

\makeatletter\setbox\nbsphinxpromptbox\box\voidb@x\makeatother

\begin{nbsphinxfancyoutput}

\begin{sphinxuseclass}{output_area}
\begin{sphinxuseclass}{}
\noindent\sphinxincludegraphics[width=424\sphinxpxdimen,height=280\sphinxpxdimen]{{docs_Approach_Use-Cases_42_3}.png}

\end{sphinxuseclass}
\end{sphinxuseclass}
\end{nbsphinxfancyoutput}

\end{sphinxuseclass}
\begin{sphinxuseclass}{nboutput}
\begin{sphinxuseclass}{nblast}
\hrule height -\fboxrule\relax
\vspace{\nbsphinxcodecellspacing}

\makeatletter\setbox\nbsphinxpromptbox\box\voidb@x\makeatother

\begin{nbsphinxfancyoutput}

\begin{sphinxuseclass}{output_area}
\begin{sphinxuseclass}{}
\noindent\sphinxincludegraphics[width=424\sphinxpxdimen,height=280\sphinxpxdimen]{{docs_Approach_Use-Cases_42_4}.png}

\end{sphinxuseclass}
\end{sphinxuseclass}
\end{nbsphinxfancyoutput}

\end{sphinxuseclass}
\end{sphinxuseclass}

\paragraph{Time Sampling Options}
\label{\detokenize{docs/Approach_Use-Cases:Time-Sampling-Options}}
\sphinxAtStartPar
Above shows the default time\sphinxhyphen{}sampling option–a single fault sample in the middle of the operational phase. However, this approach can be varied to include multiple points for a more accurate sample. This is varied using the options: \sphinxhyphen{} defaultsamp (a dict), which specifies the default sampling approach for all faults/phases \sphinxhyphen{} sampparams (a dict), which specifies the sampling approach for individual faults/phases not included in the default

\sphinxAtStartPar
For example, in the following approach, we specify that each phase is represented by three evenly\sphinxhyphen{}spaced points

\begin{sphinxuseclass}{nbinput}
\begin{sphinxuseclass}{nblast}
{
\sphinxsetup{VerbatimColor={named}{nbsphinx-code-bg}}
\sphinxsetup{VerbatimBorderColor={named}{nbsphinx-code-border}}
\begin{sphinxVerbatim}[commandchars=\\\{\}]
\llap{\color{nbsphinxin}[23]:\,\hspace{\fboxrule}\hspace{\fboxsep}}\PYG{n}{app\PYGZus{}three} \PYG{o}{=} \PYG{n}{SampleApproach}\PYG{p}{(}\PYG{n}{mdl}\PYG{p}{,} \PYG{n}{phases}\PYG{o}{=}\PYG{n}{phases}\PYG{p}{,} \PYG{n}{modephases}\PYG{o}{=}\PYG{n}{modephases}\PYG{p}{,} \PYG{n}{defaultsamp}\PYG{o}{=}\PYG{p}{\PYGZob{}}\PYG{l+s+s1}{\PYGZsq{}}\PYG{l+s+s1}{samp}\PYG{l+s+s1}{\PYGZsq{}}\PYG{p}{:}\PYG{l+s+s1}{\PYGZsq{}}\PYG{l+s+s1}{evenspacing}\PYG{l+s+s1}{\PYGZsq{}}\PYG{p}{,} \PYG{l+s+s1}{\PYGZsq{}}\PYG{l+s+s1}{numpts}\PYG{l+s+s1}{\PYGZsq{}}\PYG{p}{:}\PYG{l+m+mi}{3}\PYG{p}{\PYGZcb{}}\PYG{p}{)}
\end{sphinxVerbatim}
}

\end{sphinxuseclass}
\end{sphinxuseclass}
\begin{sphinxuseclass}{nbinput}
{
\sphinxsetup{VerbatimColor={named}{nbsphinx-code-bg}}
\sphinxsetup{VerbatimBorderColor={named}{nbsphinx-code-border}}
\begin{sphinxVerbatim}[commandchars=\\\{\}]
\llap{\color{nbsphinxin}[24]:\,\hspace{\fboxrule}\hspace{\fboxsep}}\PYG{n}{app\PYGZus{}three}\PYG{o}{.}\PYG{n}{times}
\end{sphinxVerbatim}
}

\end{sphinxuseclass}
\begin{sphinxuseclass}{nboutput}
\begin{sphinxuseclass}{nblast}
{

\kern-\sphinxverbatimsmallskipamount\kern-\baselineskip
\kern+\FrameHeightAdjust\kern-\fboxrule
\vspace{\nbsphinxcodecellspacing}

\sphinxsetup{VerbatimColor={named}{white}}
\sphinxsetup{VerbatimBorderColor={named}{nbsphinx-code-border}}
\begin{sphinxuseclass}{output_area}
\begin{sphinxuseclass}{}


\begin{sphinxVerbatim}[commandchars=\\\{\}]
\llap{\color{nbsphinxout}[24]:\,\hspace{\fboxrule}\hspace{\fboxsep}}[0, 1, 2, 3, 7, 14, 16, 22, 27, 37, 38, 45, 52, 52, 54, 57]
\end{sphinxVerbatim}



\end{sphinxuseclass}
\end{sphinxuseclass}
}

\end{sphinxuseclass}
\end{sphinxuseclass}
\begin{sphinxuseclass}{nbinput}
{
\sphinxsetup{VerbatimColor={named}{nbsphinx-code-bg}}
\sphinxsetup{VerbatimBorderColor={named}{nbsphinx-code-border}}
\begin{sphinxVerbatim}[commandchars=\\\{\}]
\llap{\color{nbsphinxin}[25]:\,\hspace{\fboxrule}\hspace{\fboxsep}}\PYG{n}{endclasses\PYGZus{}three}\PYG{p}{,} \PYG{n}{mdlhists\PYGZus{}three} \PYG{o}{=} \PYG{n}{prop}\PYG{o}{.}\PYG{n}{approach}\PYG{p}{(}\PYG{n}{mdl}\PYG{p}{,} \PYG{n}{app\PYGZus{}three}\PYG{p}{)}
\end{sphinxVerbatim}
}

\end{sphinxuseclass}
\begin{sphinxuseclass}{nboutput}
\begin{sphinxuseclass}{nblast}
{

\kern-\sphinxverbatimsmallskipamount\kern-\baselineskip
\kern+\FrameHeightAdjust\kern-\fboxrule
\vspace{\nbsphinxcodecellspacing}

\sphinxsetup{VerbatimColor={named}{nbsphinx-stderr}}
\sphinxsetup{VerbatimBorderColor={named}{nbsphinx-code-border}}
\begin{sphinxuseclass}{output_area}
\begin{sphinxuseclass}{stderr}


\begin{sphinxVerbatim}[commandchars=\\\{\}]
SCENARIOS COMPLETE: 100\%|█████████████████████████████████████████████████████████████| 28/28 [00:00<00:00, 206.38it/s]
\end{sphinxVerbatim}



\end{sphinxuseclass}
\end{sphinxuseclass}
}

\end{sphinxuseclass}
\end{sphinxuseclass}
\begin{sphinxuseclass}{nbinput}
{
\sphinxsetup{VerbatimColor={named}{nbsphinx-code-bg}}
\sphinxsetup{VerbatimBorderColor={named}{nbsphinx-code-border}}
\begin{sphinxVerbatim}[commandchars=\\\{\}]
\llap{\color{nbsphinxin}[26]:\,\hspace{\fboxrule}\hspace{\fboxsep}}\PYG{n}{rd}\PYG{o}{.}\PYG{n}{plot}\PYG{o}{.}\PYG{n}{samplecosts}\PYG{p}{(}\PYG{n}{app\PYGZus{}three}\PYG{p}{,} \PYG{n}{endclasses\PYGZus{}three}\PYG{p}{)}
\end{sphinxVerbatim}
}

\end{sphinxuseclass}
\begin{sphinxuseclass}{nboutput}
\hrule height -\fboxrule\relax
\vspace{\nbsphinxcodecellspacing}

\makeatletter\setbox\nbsphinxpromptbox\box\voidb@x\makeatother

\begin{nbsphinxfancyoutput}

\begin{sphinxuseclass}{output_area}
\begin{sphinxuseclass}{}
\noindent\sphinxincludegraphics[width=424\sphinxpxdimen,height=280\sphinxpxdimen]{{docs_Approach_Use-Cases_48_0}.png}

\end{sphinxuseclass}
\end{sphinxuseclass}
\end{nbsphinxfancyoutput}

\end{sphinxuseclass}
\begin{sphinxuseclass}{nboutput}
\hrule height -\fboxrule\relax
\vspace{\nbsphinxcodecellspacing}

\makeatletter\setbox\nbsphinxpromptbox\box\voidb@x\makeatother

\begin{nbsphinxfancyoutput}

\begin{sphinxuseclass}{output_area}
\begin{sphinxuseclass}{}
\noindent\sphinxincludegraphics[width=424\sphinxpxdimen,height=280\sphinxpxdimen]{{docs_Approach_Use-Cases_48_1}.png}

\end{sphinxuseclass}
\end{sphinxuseclass}
\end{nbsphinxfancyoutput}

\end{sphinxuseclass}
\begin{sphinxuseclass}{nboutput}
\hrule height -\fboxrule\relax
\vspace{\nbsphinxcodecellspacing}

\makeatletter\setbox\nbsphinxpromptbox\box\voidb@x\makeatother

\begin{nbsphinxfancyoutput}

\begin{sphinxuseclass}{output_area}
\begin{sphinxuseclass}{}
\noindent\sphinxincludegraphics[width=424\sphinxpxdimen,height=280\sphinxpxdimen]{{docs_Approach_Use-Cases_48_2}.png}

\end{sphinxuseclass}
\end{sphinxuseclass}
\end{nbsphinxfancyoutput}

\end{sphinxuseclass}
\begin{sphinxuseclass}{nboutput}
\hrule height -\fboxrule\relax
\vspace{\nbsphinxcodecellspacing}

\makeatletter\setbox\nbsphinxpromptbox\box\voidb@x\makeatother

\begin{nbsphinxfancyoutput}

\begin{sphinxuseclass}{output_area}
\begin{sphinxuseclass}{}
\noindent\sphinxincludegraphics[width=424\sphinxpxdimen,height=280\sphinxpxdimen]{{docs_Approach_Use-Cases_48_3}.png}

\end{sphinxuseclass}
\end{sphinxuseclass}
\end{nbsphinxfancyoutput}

\end{sphinxuseclass}
\begin{sphinxuseclass}{nboutput}
\begin{sphinxuseclass}{nblast}
\hrule height -\fboxrule\relax
\vspace{\nbsphinxcodecellspacing}

\makeatletter\setbox\nbsphinxpromptbox\box\voidb@x\makeatother

\begin{nbsphinxfancyoutput}

\begin{sphinxuseclass}{output_area}
\begin{sphinxuseclass}{}
\noindent\sphinxincludegraphics[width=424\sphinxpxdimen,height=280\sphinxpxdimen]{{docs_Approach_Use-Cases_48_4}.png}

\end{sphinxuseclass}
\end{sphinxuseclass}
\end{nbsphinxfancyoutput}

\end{sphinxuseclass}
\end{sphinxuseclass}
\sphinxAtStartPar
We can also supply a quadrature externally from the optional quadpy package. Documentation is provided at: \sphinxurl{https://github.com/nschloe/quadpy}

\begin{sphinxuseclass}{nbinput}
\begin{sphinxuseclass}{nblast}
{
\sphinxsetup{VerbatimColor={named}{nbsphinx-code-bg}}
\sphinxsetup{VerbatimBorderColor={named}{nbsphinx-code-border}}
\begin{sphinxVerbatim}[commandchars=\\\{\}]
\llap{\color{nbsphinxin}[27]:\,\hspace{\fboxrule}\hspace{\fboxsep}}\PYG{k+kn}{import} \PYG{n+nn}{quadpy}
\end{sphinxVerbatim}
}

\end{sphinxuseclass}
\end{sphinxuseclass}
\sphinxAtStartPar
Here we use the gauss patterson quadrature of degree one (with three nodes)–all quadratures from the c1 line segment package can be used if needed. Shown are both the node positions and the relative weights the quadrature gives those positions.

\begin{sphinxuseclass}{nbinput}
{
\sphinxsetup{VerbatimColor={named}{nbsphinx-code-bg}}
\sphinxsetup{VerbatimBorderColor={named}{nbsphinx-code-border}}
\begin{sphinxVerbatim}[commandchars=\\\{\}]
\llap{\color{nbsphinxin}[28]:\,\hspace{\fboxrule}\hspace{\fboxsep}}\PYG{n}{quadrature} \PYG{o}{=}\PYG{n}{quadpy}\PYG{o}{.}\PYG{n}{c1}\PYG{o}{.}\PYG{n}{gauss\PYGZus{}patterson}\PYG{p}{(}\PYG{l+m+mi}{1}\PYG{p}{)}
\PYG{n}{quadrature}\PYG{o}{.}\PYG{n}{show}\PYG{p}{(}\PYG{p}{)}
\end{sphinxVerbatim}
}

\end{sphinxuseclass}
\begin{sphinxuseclass}{nboutput}
\begin{sphinxuseclass}{nblast}
\hrule height -\fboxrule\relax
\vspace{\nbsphinxcodecellspacing}

\makeatletter\setbox\nbsphinxpromptbox\box\voidb@x\makeatother

\begin{nbsphinxfancyoutput}

\begin{sphinxuseclass}{output_area}
\begin{sphinxuseclass}{}
\noindent\sphinxincludegraphics[width=349\sphinxpxdimen,height=231\sphinxpxdimen]{{docs_Approach_Use-Cases_52_0}.png}

\end{sphinxuseclass}
\end{sphinxuseclass}
\end{nbsphinxfancyoutput}

\end{sphinxuseclass}
\end{sphinxuseclass}
\begin{sphinxuseclass}{nbinput}
{
\sphinxsetup{VerbatimColor={named}{nbsphinx-code-bg}}
\sphinxsetup{VerbatimBorderColor={named}{nbsphinx-code-border}}
\begin{sphinxVerbatim}[commandchars=\\\{\}]
\llap{\color{nbsphinxin}[29]:\,\hspace{\fboxrule}\hspace{\fboxsep}}\PYG{n}{app\PYGZus{}quad} \PYG{o}{=} \PYG{n}{SampleApproach}\PYG{p}{(}\PYG{n}{mdl}\PYG{p}{,} \PYG{n}{phases}\PYG{o}{=}\PYG{n}{phases}\PYG{p}{,} \PYG{n}{modephases}\PYG{o}{=}\PYG{n}{modephases}\PYG{p}{,} \PYG{n}{defaultsamp}\PYG{o}{=}\PYG{p}{\PYGZob{}}\PYG{l+s+s1}{\PYGZsq{}}\PYG{l+s+s1}{samp}\PYG{l+s+s1}{\PYGZsq{}}\PYG{p}{:}\PYG{l+s+s1}{\PYGZsq{}}\PYG{l+s+s1}{quadrature}\PYG{l+s+s1}{\PYGZsq{}}\PYG{p}{,} \PYG{l+s+s1}{\PYGZsq{}}\PYG{l+s+s1}{quad}\PYG{l+s+s1}{\PYGZsq{}}\PYG{p}{:} \PYG{n}{quadrature}\PYG{p}{\PYGZcb{}}\PYG{p}{)}
\PYG{n}{endclasses\PYGZus{}quad}\PYG{p}{,} \PYG{n}{mdlhists\PYGZus{}quad} \PYG{o}{=} \PYG{n}{prop}\PYG{o}{.}\PYG{n}{approach}\PYG{p}{(}\PYG{n}{mdl}\PYG{p}{,} \PYG{n}{app\PYGZus{}quad}\PYG{p}{)}
\end{sphinxVerbatim}
}

\end{sphinxuseclass}
\begin{sphinxuseclass}{nboutput}
\begin{sphinxuseclass}{nblast}
{

\kern-\sphinxverbatimsmallskipamount\kern-\baselineskip
\kern+\FrameHeightAdjust\kern-\fboxrule
\vspace{\nbsphinxcodecellspacing}

\sphinxsetup{VerbatimColor={named}{nbsphinx-stderr}}
\sphinxsetup{VerbatimBorderColor={named}{nbsphinx-code-border}}
\begin{sphinxuseclass}{output_area}
\begin{sphinxuseclass}{stderr}


\begin{sphinxVerbatim}[commandchars=\\\{\}]
SCENARIOS COMPLETE: 100\%|█████████████████████████████████████████████████████████████| 27/27 [00:00<00:00, 202.59it/s]
\end{sphinxVerbatim}



\end{sphinxuseclass}
\end{sphinxuseclass}
}

\end{sphinxuseclass}
\end{sphinxuseclass}
\begin{sphinxuseclass}{nbinput}
{
\sphinxsetup{VerbatimColor={named}{nbsphinx-code-bg}}
\sphinxsetup{VerbatimBorderColor={named}{nbsphinx-code-border}}
\begin{sphinxVerbatim}[commandchars=\\\{\}]
\llap{\color{nbsphinxin}[30]:\,\hspace{\fboxrule}\hspace{\fboxsep}}\PYG{n}{app\PYGZus{}quad}\PYG{o}{.}\PYG{n}{times}
\end{sphinxVerbatim}
}

\end{sphinxuseclass}
\begin{sphinxuseclass}{nboutput}
\begin{sphinxuseclass}{nblast}
{

\kern-\sphinxverbatimsmallskipamount\kern-\baselineskip
\kern+\FrameHeightAdjust\kern-\fboxrule
\vspace{\nbsphinxcodecellspacing}

\sphinxsetup{VerbatimColor={named}{white}}
\sphinxsetup{VerbatimBorderColor={named}{nbsphinx-code-border}}
\begin{sphinxuseclass}{output_area}
\begin{sphinxuseclass}{}


\begin{sphinxVerbatim}[commandchars=\\\{\}]
\llap{\color{nbsphinxout}[30]:\,\hspace{\fboxrule}\hspace{\fboxsep}}[0, 2, 3, 3, 10, 14, 26, 27, 34, 43, 45, 51, 54, 56, 58]
\end{sphinxVerbatim}



\end{sphinxuseclass}
\end{sphinxuseclass}
}

\end{sphinxuseclass}
\end{sphinxuseclass}
\sphinxAtStartPar
As shown, while the times are \sphinxstyleemphasis{similar} to an evenly\sphinxhyphen{}spaced quadrature, they are not quite the same. Additionally, the weights of the quadrature mean that some points matter more when calculating the overall sum of sample costs. This can be visualized below:

\begin{sphinxuseclass}{nbinput}
{
\sphinxsetup{VerbatimColor={named}{nbsphinx-code-bg}}
\sphinxsetup{VerbatimBorderColor={named}{nbsphinx-code-border}}
\begin{sphinxVerbatim}[commandchars=\\\{\}]
\llap{\color{nbsphinxin}[31]:\,\hspace{\fboxrule}\hspace{\fboxsep}}\PYG{n}{rd}\PYG{o}{.}\PYG{n}{plot}\PYG{o}{.}\PYG{n}{samplecosts}\PYG{p}{(}\PYG{n}{app\PYGZus{}quad}\PYG{p}{,} \PYG{n}{endclasses\PYGZus{}quad}\PYG{p}{)}
\end{sphinxVerbatim}
}

\end{sphinxuseclass}
\begin{sphinxuseclass}{nboutput}
\hrule height -\fboxrule\relax
\vspace{\nbsphinxcodecellspacing}

\makeatletter\setbox\nbsphinxpromptbox\box\voidb@x\makeatother

\begin{nbsphinxfancyoutput}

\begin{sphinxuseclass}{output_area}
\begin{sphinxuseclass}{}
\noindent\sphinxincludegraphics[width=424\sphinxpxdimen,height=280\sphinxpxdimen]{{docs_Approach_Use-Cases_56_0}.png}

\end{sphinxuseclass}
\end{sphinxuseclass}
\end{nbsphinxfancyoutput}

\end{sphinxuseclass}
\begin{sphinxuseclass}{nboutput}
\hrule height -\fboxrule\relax
\vspace{\nbsphinxcodecellspacing}

\makeatletter\setbox\nbsphinxpromptbox\box\voidb@x\makeatother

\begin{nbsphinxfancyoutput}

\begin{sphinxuseclass}{output_area}
\begin{sphinxuseclass}{}
\noindent\sphinxincludegraphics[width=424\sphinxpxdimen,height=280\sphinxpxdimen]{{docs_Approach_Use-Cases_56_1}.png}

\end{sphinxuseclass}
\end{sphinxuseclass}
\end{nbsphinxfancyoutput}

\end{sphinxuseclass}
\begin{sphinxuseclass}{nboutput}
\hrule height -\fboxrule\relax
\vspace{\nbsphinxcodecellspacing}

\makeatletter\setbox\nbsphinxpromptbox\box\voidb@x\makeatother

\begin{nbsphinxfancyoutput}

\begin{sphinxuseclass}{output_area}
\begin{sphinxuseclass}{}
\noindent\sphinxincludegraphics[width=424\sphinxpxdimen,height=280\sphinxpxdimen]{{docs_Approach_Use-Cases_56_2}.png}

\end{sphinxuseclass}
\end{sphinxuseclass}
\end{nbsphinxfancyoutput}

\end{sphinxuseclass}
\begin{sphinxuseclass}{nboutput}
\hrule height -\fboxrule\relax
\vspace{\nbsphinxcodecellspacing}

\makeatletter\setbox\nbsphinxpromptbox\box\voidb@x\makeatother

\begin{nbsphinxfancyoutput}

\begin{sphinxuseclass}{output_area}
\begin{sphinxuseclass}{}
\noindent\sphinxincludegraphics[width=424\sphinxpxdimen,height=280\sphinxpxdimen]{{docs_Approach_Use-Cases_56_3}.png}

\end{sphinxuseclass}
\end{sphinxuseclass}
\end{nbsphinxfancyoutput}

\end{sphinxuseclass}
\begin{sphinxuseclass}{nboutput}
\begin{sphinxuseclass}{nblast}
\hrule height -\fboxrule\relax
\vspace{\nbsphinxcodecellspacing}

\makeatletter\setbox\nbsphinxpromptbox\box\voidb@x\makeatother

\begin{nbsphinxfancyoutput}

\begin{sphinxuseclass}{output_area}
\begin{sphinxuseclass}{}
\noindent\sphinxincludegraphics[width=424\sphinxpxdimen,height=280\sphinxpxdimen]{{docs_Approach_Use-Cases_56_4}.png}

\end{sphinxuseclass}
\end{sphinxuseclass}
\end{nbsphinxfancyoutput}

\end{sphinxuseclass}
\end{sphinxuseclass}
\sphinxAtStartPar
We can also sample every time–this is costly (often prohibitively). However, one has more assurance with a full approach that the overall integration will be accurate, an important consideration when the resilience loss function is nonlinear (e.g., systems where there are delays or step\sphinxhyphen{}changes in the faulty behavior)

\begin{sphinxuseclass}{nbinput}
{
\sphinxsetup{VerbatimColor={named}{nbsphinx-code-bg}}
\sphinxsetup{VerbatimBorderColor={named}{nbsphinx-code-border}}
\begin{sphinxVerbatim}[commandchars=\\\{\}]
\llap{\color{nbsphinxin}[32]:\,\hspace{\fboxrule}\hspace{\fboxsep}}\PYG{n}{app\PYGZus{}full} \PYG{o}{=} \PYG{n}{SampleApproach}\PYG{p}{(}\PYG{n}{mdl}\PYG{p}{,} \PYG{n}{phases}\PYG{o}{=}\PYG{n}{phases}\PYG{p}{,} \PYG{n}{modephases}\PYG{o}{=}\PYG{n}{modephases}\PYG{p}{,} \PYG{n}{defaultsamp}\PYG{o}{=}\PYG{p}{\PYGZob{}}\PYG{l+s+s1}{\PYGZsq{}}\PYG{l+s+s1}{samp}\PYG{l+s+s1}{\PYGZsq{}}\PYG{p}{:}\PYG{l+s+s1}{\PYGZsq{}}\PYG{l+s+s1}{fullint}\PYG{l+s+s1}{\PYGZsq{}}\PYG{p}{\PYGZcb{}}\PYG{p}{)}
\PYG{n}{endclasses\PYGZus{}full}\PYG{p}{,} \PYG{n}{mdlhists\PYGZus{}full} \PYG{o}{=} \PYG{n}{prop}\PYG{o}{.}\PYG{n}{approach}\PYG{p}{(}\PYG{n}{mdl}\PYG{p}{,} \PYG{n}{app\PYGZus{}full}\PYG{p}{)}
\end{sphinxVerbatim}
}

\end{sphinxuseclass}
\begin{sphinxuseclass}{nboutput}
\begin{sphinxuseclass}{nblast}
{

\kern-\sphinxverbatimsmallskipamount\kern-\baselineskip
\kern+\FrameHeightAdjust\kern-\fboxrule
\vspace{\nbsphinxcodecellspacing}

\sphinxsetup{VerbatimColor={named}{nbsphinx-stderr}}
\sphinxsetup{VerbatimBorderColor={named}{nbsphinx-code-border}}
\begin{sphinxuseclass}{output_area}
\begin{sphinxuseclass}{stderr}


\begin{sphinxVerbatim}[commandchars=\\\{\}]
SCENARIOS COMPLETE: 100\%|███████████████████████████████████████████████████████████| 235/235 [00:01<00:00, 200.33it/s]
\end{sphinxVerbatim}



\end{sphinxuseclass}
\end{sphinxuseclass}
}

\end{sphinxuseclass}
\end{sphinxuseclass}
\begin{sphinxuseclass}{nbinput}
{
\sphinxsetup{VerbatimColor={named}{nbsphinx-code-bg}}
\sphinxsetup{VerbatimBorderColor={named}{nbsphinx-code-border}}
\begin{sphinxVerbatim}[commandchars=\\\{\}]
\llap{\color{nbsphinxin}[33]:\,\hspace{\fboxrule}\hspace{\fboxsep}}\PYG{n}{rd}\PYG{o}{.}\PYG{n}{plot}\PYG{o}{.}\PYG{n}{samplecosts}\PYG{p}{(}\PYG{n}{app\PYGZus{}full}\PYG{p}{,} \PYG{n}{endclasses\PYGZus{}full}\PYG{p}{)}
\end{sphinxVerbatim}
}

\end{sphinxuseclass}
\begin{sphinxuseclass}{nboutput}
\hrule height -\fboxrule\relax
\vspace{\nbsphinxcodecellspacing}

\makeatletter\setbox\nbsphinxpromptbox\box\voidb@x\makeatother

\begin{nbsphinxfancyoutput}

\begin{sphinxuseclass}{output_area}
\begin{sphinxuseclass}{}
\noindent\sphinxincludegraphics[width=424\sphinxpxdimen,height=280\sphinxpxdimen]{{docs_Approach_Use-Cases_59_0}.png}

\end{sphinxuseclass}
\end{sphinxuseclass}
\end{nbsphinxfancyoutput}

\end{sphinxuseclass}
\begin{sphinxuseclass}{nboutput}
\hrule height -\fboxrule\relax
\vspace{\nbsphinxcodecellspacing}

\makeatletter\setbox\nbsphinxpromptbox\box\voidb@x\makeatother

\begin{nbsphinxfancyoutput}

\begin{sphinxuseclass}{output_area}
\begin{sphinxuseclass}{}
\noindent\sphinxincludegraphics[width=424\sphinxpxdimen,height=280\sphinxpxdimen]{{docs_Approach_Use-Cases_59_1}.png}

\end{sphinxuseclass}
\end{sphinxuseclass}
\end{nbsphinxfancyoutput}

\end{sphinxuseclass}
\begin{sphinxuseclass}{nboutput}
\hrule height -\fboxrule\relax
\vspace{\nbsphinxcodecellspacing}

\makeatletter\setbox\nbsphinxpromptbox\box\voidb@x\makeatother

\begin{nbsphinxfancyoutput}

\begin{sphinxuseclass}{output_area}
\begin{sphinxuseclass}{}
\noindent\sphinxincludegraphics[width=424\sphinxpxdimen,height=280\sphinxpxdimen]{{docs_Approach_Use-Cases_59_2}.png}

\end{sphinxuseclass}
\end{sphinxuseclass}
\end{nbsphinxfancyoutput}

\end{sphinxuseclass}
\begin{sphinxuseclass}{nboutput}
\hrule height -\fboxrule\relax
\vspace{\nbsphinxcodecellspacing}

\makeatletter\setbox\nbsphinxpromptbox\box\voidb@x\makeatother

\begin{nbsphinxfancyoutput}

\begin{sphinxuseclass}{output_area}
\begin{sphinxuseclass}{}
\noindent\sphinxincludegraphics[width=424\sphinxpxdimen,height=280\sphinxpxdimen]{{docs_Approach_Use-Cases_59_3}.png}

\end{sphinxuseclass}
\end{sphinxuseclass}
\end{nbsphinxfancyoutput}

\end{sphinxuseclass}
\begin{sphinxuseclass}{nboutput}
\begin{sphinxuseclass}{nblast}
\hrule height -\fboxrule\relax
\vspace{\nbsphinxcodecellspacing}

\makeatletter\setbox\nbsphinxpromptbox\box\voidb@x\makeatother

\begin{nbsphinxfancyoutput}

\begin{sphinxuseclass}{output_area}
\begin{sphinxuseclass}{}
\noindent\sphinxincludegraphics[width=424\sphinxpxdimen,height=280\sphinxpxdimen]{{docs_Approach_Use-Cases_59_4}.png}

\end{sphinxuseclass}
\end{sphinxuseclass}
\end{nbsphinxfancyoutput}

\end{sphinxuseclass}
\end{sphinxuseclass}
\sphinxAtStartPar
As shown, the cost function is effectively linear over time. This is expected, since the costs we put in \sphinxcode{\sphinxupquote{find\_classification}} are based on the amount of time the fault is present in the simulation.


\subsubsection{Pruning}
\label{\detokenize{docs/Approach_Use-Cases:Pruning}}
\sphinxAtStartPar
SampleApproach pruning can be used to reduce the number of points used to represent the overall costs. A pruned approach could then be used to re\sphinxhyphen{}run the set of faults at reduced computational costs, provided the time\sphinxhyphen{}based behavior has not changed significantly. This is performed using the SampleApproach.prune\_scenarios method.

\begin{sphinxuseclass}{nbinput}
{
\sphinxsetup{VerbatimColor={named}{nbsphinx-code-bg}}
\sphinxsetup{VerbatimBorderColor={named}{nbsphinx-code-border}}
\begin{sphinxVerbatim}[commandchars=\\\{\}]
\llap{\color{nbsphinxin}[34]:\,\hspace{\fboxrule}\hspace{\fboxsep}}\PYG{n}{help}\PYG{p}{(}\PYG{n}{app\PYGZus{}full}\PYG{o}{.}\PYG{n}{prune\PYGZus{}scenarios}\PYG{p}{)}
\end{sphinxVerbatim}
}

\end{sphinxuseclass}
\begin{sphinxuseclass}{nboutput}
\begin{sphinxuseclass}{nblast}
{

\kern-\sphinxverbatimsmallskipamount\kern-\baselineskip
\kern+\FrameHeightAdjust\kern-\fboxrule
\vspace{\nbsphinxcodecellspacing}

\sphinxsetup{VerbatimColor={named}{white}}
\sphinxsetup{VerbatimBorderColor={named}{nbsphinx-code-border}}
\begin{sphinxuseclass}{output_area}
\begin{sphinxuseclass}{}


\begin{sphinxVerbatim}[commandchars=\\\{\}]
Help on method prune\_scenarios in module fmdtools.modeldef:

prune\_scenarios(endclasses, samptype='piecewise', threshold=0.1, sampparam=\{'samp': 'evenspacing', 'numpts': 1\}) method of fmdtools.modeldef.SampleApproach instance
    Finds the best sample approach to approximate the full integral (given the approach was the full integral).

    Parameters
    ----------
    endclasses : dict
        dict of results (cost, rate, expected cost) for the model run indexed by scenid
    samptype : str ('piecewise' or 'bestpt'), optional
        Method to use.
        If 'bestpt', finds the point in the interval that gives the average cost.
        If 'piecewise', attempts to split the inverval into sub-intervals of continuity
        The default is 'piecewise'.
    threshold : float, optional
        If 'piecewise,' the threshold for detecting a discontinuity based on deviation from linearity. The default is 0.1.
    sampparam : float, optional
        If 'piecewise,' the sampparam sampparam to prune to. The default is \{'samp':'evenspacing','numpts':1\}, which would be a single point (optimal for linear).

\end{sphinxVerbatim}



\end{sphinxuseclass}
\end{sphinxuseclass}
}

\end{sphinxuseclass}
\end{sphinxuseclass}
\begin{sphinxuseclass}{nbinput}
\begin{sphinxuseclass}{nblast}
{
\sphinxsetup{VerbatimColor={named}{nbsphinx-code-bg}}
\sphinxsetup{VerbatimBorderColor={named}{nbsphinx-code-border}}
\begin{sphinxVerbatim}[commandchars=\\\{\}]
\llap{\color{nbsphinxin}[35]:\,\hspace{\fboxrule}\hspace{\fboxsep}}\PYG{n}{app\PYGZus{}full}\PYG{o}{.}\PYG{n}{prune\PYGZus{}scenarios}\PYG{p}{(}\PYG{n}{endclasses\PYGZus{}full}\PYG{p}{)}
\end{sphinxVerbatim}
}

\end{sphinxuseclass}
\end{sphinxuseclass}
\begin{sphinxuseclass}{nbinput}
{
\sphinxsetup{VerbatimColor={named}{nbsphinx-code-bg}}
\sphinxsetup{VerbatimBorderColor={named}{nbsphinx-code-border}}
\begin{sphinxVerbatim}[commandchars=\\\{\}]
\llap{\color{nbsphinxin}[40]:\,\hspace{\fboxrule}\hspace{\fboxsep}}\PYG{n}{app\PYGZus{}full}\PYG{o}{.}\PYG{n}{times}
\end{sphinxVerbatim}
}

\end{sphinxuseclass}
\begin{sphinxuseclass}{nboutput}
\begin{sphinxuseclass}{nblast}
{

\kern-\sphinxverbatimsmallskipamount\kern-\baselineskip
\kern+\FrameHeightAdjust\kern-\fboxrule
\vspace{\nbsphinxcodecellspacing}

\sphinxsetup{VerbatimColor={named}{white}}
\sphinxsetup{VerbatimBorderColor={named}{nbsphinx-code-border}}
\begin{sphinxuseclass}{output_area}
\begin{sphinxuseclass}{}


\begin{sphinxVerbatim}[commandchars=\\\{\}]
\llap{\color{nbsphinxout}[40]:\,\hspace{\fboxrule}\hspace{\fboxsep}}[2, 14, 27, 45, 54]
\end{sphinxVerbatim}



\end{sphinxuseclass}
\end{sphinxuseclass}
}

\end{sphinxuseclass}
\end{sphinxuseclass}
\begin{sphinxuseclass}{nbinput}
{
\sphinxsetup{VerbatimColor={named}{nbsphinx-code-bg}}
\sphinxsetup{VerbatimBorderColor={named}{nbsphinx-code-border}}
\begin{sphinxVerbatim}[commandchars=\\\{\}]
\llap{\color{nbsphinxin}[42]:\,\hspace{\fboxrule}\hspace{\fboxsep}}\PYG{n}{endclasses\PYGZus{}pruned}\PYG{p}{,} \PYG{n}{mdlhists\PYGZus{}pruned} \PYG{o}{=} \PYG{n}{prop}\PYG{o}{.}\PYG{n}{approach}\PYG{p}{(}\PYG{n}{mdl}\PYG{p}{,} \PYG{n}{app\PYGZus{}full}\PYG{p}{)}
\end{sphinxVerbatim}
}

\end{sphinxuseclass}
\begin{sphinxuseclass}{nboutput}
\begin{sphinxuseclass}{nblast}
{

\kern-\sphinxverbatimsmallskipamount\kern-\baselineskip
\kern+\FrameHeightAdjust\kern-\fboxrule
\vspace{\nbsphinxcodecellspacing}

\sphinxsetup{VerbatimColor={named}{nbsphinx-stderr}}
\sphinxsetup{VerbatimBorderColor={named}{nbsphinx-code-border}}
\begin{sphinxuseclass}{output_area}
\begin{sphinxuseclass}{stderr}


\begin{sphinxVerbatim}[commandchars=\\\{\}]
SCENARIOS COMPLETE: 100\%|███████████████████████████████████████████████████████████████| 9/9 [00:00<00:00, 191.92it/s]
\end{sphinxVerbatim}



\end{sphinxuseclass}
\end{sphinxuseclass}
}

\end{sphinxuseclass}
\end{sphinxuseclass}
\sphinxAtStartPar
As shown, with the default options the pruned sampling approach applies a centered single\sphinxhyphen{}point, as would happen using the default options. This would change if there were nonlinearities in the resilience loss function.

\begin{sphinxuseclass}{nbinput}
{
\sphinxsetup{VerbatimColor={named}{nbsphinx-code-bg}}
\sphinxsetup{VerbatimBorderColor={named}{nbsphinx-code-border}}
\begin{sphinxVerbatim}[commandchars=\\\{\}]
\llap{\color{nbsphinxin}[43]:\,\hspace{\fboxrule}\hspace{\fboxsep}}\PYG{n}{rd}\PYG{o}{.}\PYG{n}{plot}\PYG{o}{.}\PYG{n}{samplecosts}\PYG{p}{(}\PYG{n}{app\PYGZus{}full}\PYG{p}{,} \PYG{n}{endclasses\PYGZus{}pruned}\PYG{p}{)}
\end{sphinxVerbatim}
}

\end{sphinxuseclass}
\begin{sphinxuseclass}{nboutput}
\hrule height -\fboxrule\relax
\vspace{\nbsphinxcodecellspacing}

\makeatletter\setbox\nbsphinxpromptbox\box\voidb@x\makeatother

\begin{nbsphinxfancyoutput}

\begin{sphinxuseclass}{output_area}
\begin{sphinxuseclass}{}
\noindent\sphinxincludegraphics[width=424\sphinxpxdimen,height=280\sphinxpxdimen]{{docs_Approach_Use-Cases_67_0}.png}

\end{sphinxuseclass}
\end{sphinxuseclass}
\end{nbsphinxfancyoutput}

\end{sphinxuseclass}
\begin{sphinxuseclass}{nboutput}
\hrule height -\fboxrule\relax
\vspace{\nbsphinxcodecellspacing}

\makeatletter\setbox\nbsphinxpromptbox\box\voidb@x\makeatother

\begin{nbsphinxfancyoutput}

\begin{sphinxuseclass}{output_area}
\begin{sphinxuseclass}{}
\noindent\sphinxincludegraphics[width=424\sphinxpxdimen,height=280\sphinxpxdimen]{{docs_Approach_Use-Cases_67_1}.png}

\end{sphinxuseclass}
\end{sphinxuseclass}
\end{nbsphinxfancyoutput}

\end{sphinxuseclass}
\begin{sphinxuseclass}{nboutput}
\hrule height -\fboxrule\relax
\vspace{\nbsphinxcodecellspacing}

\makeatletter\setbox\nbsphinxpromptbox\box\voidb@x\makeatother

\begin{nbsphinxfancyoutput}

\begin{sphinxuseclass}{output_area}
\begin{sphinxuseclass}{}
\noindent\sphinxincludegraphics[width=424\sphinxpxdimen,height=280\sphinxpxdimen]{{docs_Approach_Use-Cases_67_2}.png}

\end{sphinxuseclass}
\end{sphinxuseclass}
\end{nbsphinxfancyoutput}

\end{sphinxuseclass}
\begin{sphinxuseclass}{nboutput}
\hrule height -\fboxrule\relax
\vspace{\nbsphinxcodecellspacing}

\makeatletter\setbox\nbsphinxpromptbox\box\voidb@x\makeatother

\begin{nbsphinxfancyoutput}

\begin{sphinxuseclass}{output_area}
\begin{sphinxuseclass}{}
\noindent\sphinxincludegraphics[width=424\sphinxpxdimen,height=280\sphinxpxdimen]{{docs_Approach_Use-Cases_67_3}.png}

\end{sphinxuseclass}
\end{sphinxuseclass}
\end{nbsphinxfancyoutput}

\end{sphinxuseclass}
\begin{sphinxuseclass}{nboutput}
\begin{sphinxuseclass}{nblast}
\hrule height -\fboxrule\relax
\vspace{\nbsphinxcodecellspacing}

\makeatletter\setbox\nbsphinxpromptbox\box\voidb@x\makeatother

\begin{nbsphinxfancyoutput}

\begin{sphinxuseclass}{output_area}
\begin{sphinxuseclass}{}
\noindent\sphinxincludegraphics[width=424\sphinxpxdimen,height=280\sphinxpxdimen]{{docs_Approach_Use-Cases_67_4}.png}

\end{sphinxuseclass}
\end{sphinxuseclass}
\end{nbsphinxfancyoutput}

\end{sphinxuseclass}
\end{sphinxuseclass}
\sphinxAtStartPar
A more detailed investigation of sampling approaches is provided in \sphinxcode{\sphinxupquote{/example\_pump/IDETC\_Results/IDETC\_Figures.ipynb}} and its corresponding paper:

\sphinxAtStartPar
\sphinxstyleemphasis{Hulse, D., Hoyle, C., Tumer, I. Y., Goebel, K., \& Kulkarni, C. (2020, August). Temporal Fault Injection Considerations in Resilience Quantification. In International Design Engineering Technical Conferences and Computers and Information in Engineering Conference (Vol. 84003, p. V11AT11A040). American Society of Mechanical Engineers.}


\subsection{Using Parallel Computing in fmdtools}
\label{\detokenize{example_pump/Parallelism_Tutorial:Using-Parallel-Computing-in-fmdtools}}\label{\detokenize{example_pump/Parallelism_Tutorial::doc}}
\sphinxAtStartPar
This notebook will discuss how to use parallel programming in fmdtools, including: \sphinxhyphen{} how to set up a model for parallelism \sphinxhyphen{} syntax for using parallelism in simulation functions \sphinxhyphen{} considerations for optimizing computational performance in a model

\begin{sphinxuseclass}{nbinput}
\begin{sphinxuseclass}{nblast}
{
\sphinxsetup{VerbatimColor={named}{nbsphinx-code-bg}}
\sphinxsetup{VerbatimBorderColor={named}{nbsphinx-code-border}}
\begin{sphinxVerbatim}[commandchars=\\\{\}]
\llap{\color{nbsphinxin}[1]:\,\hspace{\fboxrule}\hspace{\fboxsep}}\PYG{k+kn}{import} \PYG{n+nn}{sys}\PYG{o}{,} \PYG{n+nn}{os}
\PYG{n}{sys}\PYG{o}{.}\PYG{n}{path}\PYG{o}{.}\PYG{n}{insert}\PYG{p}{(}\PYG{l+m+mi}{1}\PYG{p}{,}\PYG{n}{os}\PYG{o}{.}\PYG{n}{path}\PYG{o}{.}\PYG{n}{join}\PYG{p}{(}\PYG{l+s+s2}{\PYGZdq{}}\PYG{l+s+s2}{..}\PYG{l+s+s2}{\PYGZdq{}}\PYG{p}{)}\PYG{p}{)}

\PYG{k+kn}{from} \PYG{n+nn}{ex\PYGZus{}pump} \PYG{k+kn}{import} \PYG{o}{*}
\PYG{k+kn}{from} \PYG{n+nn}{fmdtools}\PYG{n+nn}{.}\PYG{n+nn}{modeldef} \PYG{k+kn}{import} \PYG{n}{SampleApproach}
\PYG{k+kn}{import} \PYG{n+nn}{fmdtools}\PYG{n+nn}{.}\PYG{n+nn}{faultsim}\PYG{n+nn}{.}\PYG{n+nn}{propagate} \PYG{k}{as} \PYG{n+nn}{propagate}
\PYG{k+kn}{import} \PYG{n+nn}{fmdtools}\PYG{n+nn}{.}\PYG{n+nn}{resultdisp} \PYG{k}{as} \PYG{n+nn}{rd}
\end{sphinxVerbatim}
}

\end{sphinxuseclass}
\end{sphinxuseclass}
\sphinxAtStartPar
This notebook uses the pump example (see ex\_pump.py) to illustrate the use of parallelism in fmdtools. This is fairly simple model, and thus it should be noted that there may be considerations with more complex models which may not be adequately covered here.

\begin{sphinxuseclass}{nbinput}
\begin{sphinxuseclass}{nblast}
{
\sphinxsetup{VerbatimColor={named}{nbsphinx-code-bg}}
\sphinxsetup{VerbatimBorderColor={named}{nbsphinx-code-border}}
\begin{sphinxVerbatim}[commandchars=\\\{\}]
\llap{\color{nbsphinxin}[2]:\,\hspace{\fboxrule}\hspace{\fboxsep}}\PYG{n}{mdl} \PYG{o}{=} \PYG{n}{Pump}\PYG{p}{(}\PYG{p}{)}
\end{sphinxVerbatim}
}

\end{sphinxuseclass}
\end{sphinxuseclass}
\begin{sphinxuseclass}{nbinput}
\begin{sphinxuseclass}{nblast}
{
\sphinxsetup{VerbatimColor={named}{nbsphinx-code-bg}}
\sphinxsetup{VerbatimBorderColor={named}{nbsphinx-code-border}}
\begin{sphinxVerbatim}[commandchars=\\\{\}]
\llap{\color{nbsphinxin}[3]:\,\hspace{\fboxrule}\hspace{\fboxsep}}\PYG{n}{endresults}\PYG{p}{,} \PYG{n}{resgraph}\PYG{p}{,} \PYG{n}{mdlhist} \PYG{o}{=} \PYG{n}{propagate}\PYG{o}{.}\PYG{n}{nominal}\PYG{p}{(}\PYG{n}{mdl}\PYG{p}{)}
\end{sphinxVerbatim}
}

\end{sphinxuseclass}
\end{sphinxuseclass}
\begin{sphinxuseclass}{nbinput}
{
\sphinxsetup{VerbatimColor={named}{nbsphinx-code-bg}}
\sphinxsetup{VerbatimBorderColor={named}{nbsphinx-code-border}}
\begin{sphinxVerbatim}[commandchars=\\\{\}]
\llap{\color{nbsphinxin}[4]:\,\hspace{\fboxrule}\hspace{\fboxsep}}\PYG{n}{rd}\PYG{o}{.}\PYG{n}{graph}\PYG{o}{.}\PYG{n}{show}\PYG{p}{(}\PYG{n}{resgraph}\PYG{p}{)}
\end{sphinxVerbatim}
}

\end{sphinxuseclass}
\begin{sphinxuseclass}{nboutput}
{

\kern-\sphinxverbatimsmallskipamount\kern-\baselineskip
\kern+\FrameHeightAdjust\kern-\fboxrule
\vspace{\nbsphinxcodecellspacing}

\sphinxsetup{VerbatimColor={named}{white}}
\sphinxsetup{VerbatimBorderColor={named}{nbsphinx-code-border}}
\begin{sphinxuseclass}{output_area}
\begin{sphinxuseclass}{}


\begin{sphinxVerbatim}[commandchars=\\\{\}]
\llap{\color{nbsphinxout}[4]:\,\hspace{\fboxrule}\hspace{\fboxsep}}(<Figure size 432x288 with 1 Axes>, <AxesSubplot:>)
\end{sphinxVerbatim}



\end{sphinxuseclass}
\end{sphinxuseclass}
}

\end{sphinxuseclass}
\begin{sphinxuseclass}{nboutput}
\begin{sphinxuseclass}{nblast}
\hrule height -\fboxrule\relax
\vspace{\nbsphinxcodecellspacing}

\makeatletter\setbox\nbsphinxpromptbox\box\voidb@x\makeatother

\begin{nbsphinxfancyoutput}

\begin{sphinxuseclass}{output_area}
\begin{sphinxuseclass}{}
\noindent\sphinxincludegraphics[width=349\sphinxpxdimen,height=231\sphinxpxdimen]{{example_pump_Parallelism_Tutorial_5_1}.png}

\end{sphinxuseclass}
\end{sphinxuseclass}
\end{nbsphinxfancyoutput}

\end{sphinxuseclass}
\end{sphinxuseclass}
\begin{sphinxuseclass}{nbinput}
{
\sphinxsetup{VerbatimColor={named}{nbsphinx-code-bg}}
\sphinxsetup{VerbatimBorderColor={named}{nbsphinx-code-border}}
\begin{sphinxVerbatim}[commandchars=\\\{\}]
\llap{\color{nbsphinxin}[5]:\,\hspace{\fboxrule}\hspace{\fboxsep}}\PYG{n}{rd}\PYG{o}{.}\PYG{n}{plot}\PYG{o}{.}\PYG{n}{mdlhistvals}\PYG{p}{(}\PYG{n}{mdlhist}\PYG{p}{)}
\end{sphinxVerbatim}
}

\end{sphinxuseclass}
\begin{sphinxuseclass}{nboutput}
\hrule height -\fboxrule\relax
\vspace{\nbsphinxcodecellspacing}

\savebox\nbsphinxpromptbox[0pt][r]{\color{nbsphinxout}\Verb|\strut{[5]:}\,|}

\begin{nbsphinxfancyoutput}

\begin{sphinxuseclass}{output_area}
\begin{sphinxuseclass}{}
\noindent\sphinxincludegraphics[width=426\sphinxpxdimen,height=788\sphinxpxdimen]{{example_pump_Parallelism_Tutorial_6_0}.png}

\end{sphinxuseclass}
\end{sphinxuseclass}
\end{nbsphinxfancyoutput}

\end{sphinxuseclass}
\begin{sphinxuseclass}{nboutput}
\begin{sphinxuseclass}{nblast}
\hrule height -\fboxrule\relax
\vspace{\nbsphinxcodecellspacing}

\makeatletter\setbox\nbsphinxpromptbox\box\voidb@x\makeatother

\begin{nbsphinxfancyoutput}

\begin{sphinxuseclass}{output_area}
\begin{sphinxuseclass}{}
\noindent\sphinxincludegraphics[width=426\sphinxpxdimen,height=788\sphinxpxdimen]{{example_pump_Parallelism_Tutorial_6_1}.png}

\end{sphinxuseclass}
\end{sphinxuseclass}
\end{nbsphinxfancyoutput}

\end{sphinxuseclass}
\end{sphinxuseclass}

\subsubsection{Model Checks}
\label{\detokenize{example_pump/Parallelism_Tutorial:Model-Checks}}
\sphinxAtStartPar
Before attempting to leverage parallelism in model execution, it can be helpful to check whether a model is compatible with python parallel computing libraries. In order for a model to be parallelized, it must be compatible with \sphinxhref{https://docs.python.org/3/library/pickle.html\#:~:text=\%E2\%80\%9CPickling\%E2\%80\%9D\%20is\%20the\%20process\%20whereby,back\%20into\%20an\%20object\%20hierarchy.}{pickling}–python’s method of data serialization. This is used in parallel programming methods to copy the model
from the main process thread to the seperate processes of the pool.

\sphinxAtStartPar
fmdtools has two methods to check whether a model can be pickled, \sphinxcode{\sphinxupquote{check\_pickleability}} and \sphinxcode{\sphinxupquote{check\_model\_pickleability}}. The main difference between these is that \sphinxcode{\sphinxupquote{check\_pickleability}} works for all objects (e.g. functions and flows), while \sphinxcode{\sphinxupquote{check\_model\_pickleability}} gives more information for an overall model structure

\begin{sphinxuseclass}{nbinput}
\begin{sphinxuseclass}{nblast}
{
\sphinxsetup{VerbatimColor={named}{nbsphinx-code-bg}}
\sphinxsetup{VerbatimBorderColor={named}{nbsphinx-code-border}}
\begin{sphinxVerbatim}[commandchars=\\\{\}]
\llap{\color{nbsphinxin}[6]:\,\hspace{\fboxrule}\hspace{\fboxsep}}\PYG{k+kn}{from} \PYG{n+nn}{fmdtools}\PYG{n+nn}{.}\PYG{n+nn}{modeldef} \PYG{k+kn}{import} \PYG{n}{check\PYGZus{}pickleability}\PYG{p}{,} \PYG{n}{check\PYGZus{}model\PYGZus{}pickleability}
\end{sphinxVerbatim}
}

\end{sphinxuseclass}
\end{sphinxuseclass}
\begin{sphinxuseclass}{nbinput}
{
\sphinxsetup{VerbatimColor={named}{nbsphinx-code-bg}}
\sphinxsetup{VerbatimBorderColor={named}{nbsphinx-code-border}}
\begin{sphinxVerbatim}[commandchars=\\\{\}]
\llap{\color{nbsphinxin}[7]:\,\hspace{\fboxrule}\hspace{\fboxsep}}\PYG{n}{unpickleable\PYGZus{}attributes} \PYG{o}{=} \PYG{n}{check\PYGZus{}pickleability}\PYG{p}{(}\PYG{n}{mdl}\PYG{p}{)}
\end{sphinxVerbatim}
}

\end{sphinxuseclass}
\begin{sphinxuseclass}{nboutput}
\begin{sphinxuseclass}{nblast}
{

\kern-\sphinxverbatimsmallskipamount\kern-\baselineskip
\kern+\FrameHeightAdjust\kern-\fboxrule
\vspace{\nbsphinxcodecellspacing}

\sphinxsetup{VerbatimColor={named}{white}}
\sphinxsetup{VerbatimBorderColor={named}{nbsphinx-code-border}}
\begin{sphinxuseclass}{output_area}
\begin{sphinxuseclass}{}


\begin{sphinxVerbatim}[commandchars=\\\{\}]
The object is pickleable
\end{sphinxVerbatim}



\end{sphinxuseclass}
\end{sphinxuseclass}
}

\end{sphinxuseclass}
\end{sphinxuseclass}
\begin{sphinxuseclass}{nbinput}
{
\sphinxsetup{VerbatimColor={named}{nbsphinx-code-bg}}
\sphinxsetup{VerbatimBorderColor={named}{nbsphinx-code-border}}
\begin{sphinxVerbatim}[commandchars=\\\{\}]
\llap{\color{nbsphinxin}[8]:\,\hspace{\fboxrule}\hspace{\fboxsep}}\PYG{n}{check\PYGZus{}model\PYGZus{}pickleability}\PYG{p}{(}\PYG{n}{mdl}\PYG{p}{)}
\end{sphinxVerbatim}
}

\end{sphinxuseclass}
\begin{sphinxuseclass}{nboutput}
\begin{sphinxuseclass}{nblast}
{

\kern-\sphinxverbatimsmallskipamount\kern-\baselineskip
\kern+\FrameHeightAdjust\kern-\fboxrule
\vspace{\nbsphinxcodecellspacing}

\sphinxsetup{VerbatimColor={named}{white}}
\sphinxsetup{VerbatimBorderColor={named}{nbsphinx-code-border}}
\begin{sphinxuseclass}{output_area}
\begin{sphinxuseclass}{}


\begin{sphinxVerbatim}[commandchars=\\\{\}]
The object is pickleable
\end{sphinxVerbatim}



\end{sphinxuseclass}
\end{sphinxuseclass}
}

\end{sphinxuseclass}
\end{sphinxuseclass}
\sphinxAtStartPar
As you can see, this model is pickleable. However, this may not be the case for all structures if they rely on unpickleable data structures. For example:

\begin{sphinxuseclass}{nbinput}
\begin{sphinxuseclass}{nblast}
{
\sphinxsetup{VerbatimColor={named}{nbsphinx-code-bg}}
\sphinxsetup{VerbatimBorderColor={named}{nbsphinx-code-border}}
\begin{sphinxVerbatim}[commandchars=\\\{\}]
\llap{\color{nbsphinxin}[9]:\,\hspace{\fboxrule}\hspace{\fboxsep}}\PYG{n}{mdl}\PYG{o}{.}\PYG{n}{fxns}\PYG{p}{[}\PYG{l+s+s1}{\PYGZsq{}}\PYG{l+s+s1}{ImportEE}\PYG{l+s+s1}{\PYGZsq{}}\PYG{p}{]}\PYG{o}{.}\PYG{n}{bad\PYGZus{}fxn\PYGZus{}attribute} \PYG{o}{=} \PYG{p}{\PYGZob{}}\PYG{l+m+mi}{1}\PYG{p}{:}\PYG{l+m+mi}{2}\PYG{p}{,}\PYG{l+m+mi}{3}\PYG{p}{:}\PYG{l+m+mi}{4}\PYG{p}{\PYGZcb{}}\PYG{o}{.}\PYG{n}{keys}\PYG{p}{(}\PYG{p}{)}
\PYG{n}{mdl}\PYG{o}{.}\PYG{n}{flows}\PYG{p}{[}\PYG{l+s+s1}{\PYGZsq{}}\PYG{l+s+s1}{EE\PYGZus{}1}\PYG{l+s+s1}{\PYGZsq{}}\PYG{p}{]}\PYG{o}{.}\PYG{n}{bad\PYGZus{}flow\PYGZus{}attribute} \PYG{o}{=} \PYG{p}{\PYGZob{}}\PYG{l+m+mi}{1}\PYG{p}{:}\PYG{l+m+mi}{2}\PYG{p}{,}\PYG{l+m+mi}{3}\PYG{p}{:}\PYG{l+m+mi}{4}\PYG{p}{\PYGZcb{}}\PYG{o}{.}\PYG{n}{keys}\PYG{p}{(}\PYG{p}{)}
\end{sphinxVerbatim}
}

\end{sphinxuseclass}
\end{sphinxuseclass}
\begin{sphinxuseclass}{nbinput}
{
\sphinxsetup{VerbatimColor={named}{nbsphinx-code-bg}}
\sphinxsetup{VerbatimBorderColor={named}{nbsphinx-code-border}}
\begin{sphinxVerbatim}[commandchars=\\\{\}]
\llap{\color{nbsphinxin}[10]:\,\hspace{\fboxrule}\hspace{\fboxsep}}\PYG{n}{check\PYGZus{}model\PYGZus{}pickleability}\PYG{p}{(}\PYG{n}{mdl}\PYG{p}{)}
\end{sphinxVerbatim}
}

\end{sphinxuseclass}
\begin{sphinxuseclass}{nboutput}
\begin{sphinxuseclass}{nblast}
{

\kern-\sphinxverbatimsmallskipamount\kern-\baselineskip
\kern+\FrameHeightAdjust\kern-\fboxrule
\vspace{\nbsphinxcodecellspacing}

\sphinxsetup{VerbatimColor={named}{white}}
\sphinxsetup{VerbatimBorderColor={named}{nbsphinx-code-border}}
\begin{sphinxuseclass}{output_area}
\begin{sphinxuseclass}{}


\begin{sphinxVerbatim}[commandchars=\\\{\}]
The following attributes will not pickle: ['flows', 'fxns', 'graph']
FLOWS
EE\_1
The following attributes will not pickle: ['bad\_flow\_attribute']
Sig\_1
The object is pickleable
Wat\_1
The object is pickleable
Wat\_2
The object is pickleable
FUNCTIONS
ImportEE
The following attributes will not pickle: ['flows', 'EEout', 'bad\_fxn\_attribute']
ImportWater
The object is pickleable
ImportSignal
The object is pickleable
MoveWater
The following attributes will not pickle: ['flows', 'EEin']
ExportWater
The object is pickleable
\end{sphinxVerbatim}



\end{sphinxuseclass}
\end{sphinxuseclass}
}

\end{sphinxuseclass}
\end{sphinxuseclass}
\sphinxAtStartPar
In this case, because the \sphinxcode{\sphinxupquote{ImportEE}} function and \sphinxcode{\sphinxupquote{EE\_1}} flow was given a an unpicklable attribute (a \sphinxcode{\sphinxupquote{dict\_keys()}} object) and thus the interfacing functions and overall model will not pickle. To fix this, one could easily convert these attributes into lists, sets, or dictionaries.

\begin{sphinxuseclass}{nbinput}
\begin{sphinxuseclass}{nblast}
{
\sphinxsetup{VerbatimColor={named}{nbsphinx-code-bg}}
\sphinxsetup{VerbatimBorderColor={named}{nbsphinx-code-border}}
\begin{sphinxVerbatim}[commandchars=\\\{\}]
\llap{\color{nbsphinxin}[11]:\,\hspace{\fboxrule}\hspace{\fboxsep}}\PYG{n}{mdl} \PYG{o}{=} \PYG{n}{Pump}\PYG{p}{(}\PYG{p}{)}
\end{sphinxVerbatim}
}

\end{sphinxuseclass}
\end{sphinxuseclass}

\subsubsection{Using Parallelism in Simulation}
\label{\detokenize{example_pump/Parallelism_Tutorial:Using-Parallelism-in-Simulation}}
\sphinxAtStartPar
Parallelism generally requires using some external parallel processing toolkit. The syntax used by fmdtools methods is compatible with: \sphinxhyphen{} \sphinxhref{https://docs.python.org/3/library/multiprocessing.html}{multiprocessing}, python’s default parallel computing library \sphinxhyphen{} \sphinxhref{https://pypi.org/project/multiprocess/}{multiprocess}, a fork of multiprocessing developed by The UQ Foundation \sphinxhyphen{} \sphinxhref{https://github.com/uqfoundation/pathos}{pathos}, a broader parallel computing package developed by The UQ Foundation

\sphinxAtStartPar
And any other package that emulates multiprocessing.Pool

\begin{sphinxuseclass}{nbinput}
\begin{sphinxuseclass}{nblast}
{
\sphinxsetup{VerbatimColor={named}{nbsphinx-code-bg}}
\sphinxsetup{VerbatimBorderColor={named}{nbsphinx-code-border}}
\begin{sphinxVerbatim}[commandchars=\\\{\}]
\llap{\color{nbsphinxin}[12]:\,\hspace{\fboxrule}\hspace{\fboxsep}}\PYG{k+kn}{import} \PYG{n+nn}{multiprocessing} \PYG{k}{as} \PYG{n+nn}{mp}
\PYG{k+kn}{import} \PYG{n+nn}{multiprocess} \PYG{k}{as} \PYG{n+nn}{ms}
\PYG{k+kn}{from} \PYG{n+nn}{pathos}\PYG{n+nn}{.}\PYG{n+nn}{pools} \PYG{k+kn}{import} \PYG{n}{ParallelPool}\PYG{p}{,} \PYG{n}{ProcessPool}\PYG{p}{,} \PYG{n}{SerialPool}\PYG{p}{,} \PYG{n}{ThreadPool}
\end{sphinxVerbatim}
}

\end{sphinxuseclass}
\end{sphinxuseclass}
\sphinxAtStartPar
Parallelism can speed up simulation time when there is a large number of independent simulations to run. The prefered methods for using parallelism are to use a \sphinxcode{\sphinxupquote{NominalApproach}} or \sphinxcode{\sphinxupquote{SampleApproach}} with the methods: \sphinxhyphen{} propagate.singlefaults (for all single\sphinxhyphen{}fault scenarios in a static model with no approach) \sphinxhyphen{} propagate.approach (for sampling a set of faults) \sphinxhyphen{} propagate.nominal\_approach (for simulating the model nominally over a set of parameters) \sphinxhyphen{} propagate.nested\_approach (for sampling a
set of faults over a set of model parameters)

\sphinxAtStartPar
These methods can be run in parallel by sending them a \sphinxcode{\sphinxupquote{pool}} object from one of these modules as the optional \sphinxcode{\sphinxupquote{pool}} argument. Further details on setting up and running an approach are provided in \sphinxcode{\sphinxupquote{docs/Approach Use\sphinxhyphen{}Cases.ipynb}}

\begin{sphinxuseclass}{nbinput}
{
\sphinxsetup{VerbatimColor={named}{nbsphinx-code-bg}}
\sphinxsetup{VerbatimBorderColor={named}{nbsphinx-code-border}}
\begin{sphinxVerbatim}[commandchars=\\\{\}]
\llap{\color{nbsphinxin}[13]:\,\hspace{\fboxrule}\hspace{\fboxsep}}\PYG{n}{pool} \PYG{o}{=} \PYG{n}{mp}\PYG{o}{.}\PYG{n}{Pool}\PYG{p}{(}\PYG{l+m+mi}{4}\PYG{p}{)}
\PYG{n}{app} \PYG{o}{=} \PYG{n}{SampleApproach}\PYG{p}{(}\PYG{n}{mdl}\PYG{p}{)}
\PYG{n}{endclasses}\PYG{p}{,} \PYG{n}{mdlhists} \PYG{o}{=} \PYG{n}{propagate}\PYG{o}{.}\PYG{n}{approach}\PYG{p}{(}\PYG{n}{mdl}\PYG{p}{,}\PYG{n}{app}\PYG{p}{,} \PYG{n}{pool}\PYG{o}{=}\PYG{n}{pool}\PYG{p}{)}
\PYG{n}{rd}\PYG{o}{.}\PYG{n}{tabulate}\PYG{o}{.}\PYG{n}{simplefmea}\PYG{p}{(}\PYG{n}{endclasses}\PYG{p}{)}
\end{sphinxVerbatim}
}

\end{sphinxuseclass}
\begin{sphinxuseclass}{nboutput}
{

\kern-\sphinxverbatimsmallskipamount\kern-\baselineskip
\kern+\FrameHeightAdjust\kern-\fboxrule
\vspace{\nbsphinxcodecellspacing}

\sphinxsetup{VerbatimColor={named}{nbsphinx-stderr}}
\sphinxsetup{VerbatimBorderColor={named}{nbsphinx-code-border}}
\begin{sphinxuseclass}{output_area}
\begin{sphinxuseclass}{stderr}


\begin{sphinxVerbatim}[commandchars=\\\{\}]
SCENARIOS COMPLETE: 100\%|██████████████████████████████████████████████████████████████| 17/17 [00:01<00:00,  9.81it/s]
\end{sphinxVerbatim}



\end{sphinxuseclass}
\end{sphinxuseclass}
}

\end{sphinxuseclass}
\begin{sphinxuseclass}{nboutput}
\begin{sphinxuseclass}{nblast}
{

\kern-\sphinxverbatimsmallskipamount\kern-\baselineskip
\kern+\FrameHeightAdjust\kern-\fboxrule
\vspace{\nbsphinxcodecellspacing}

\sphinxsetup{VerbatimColor={named}{white}}
\sphinxsetup{VerbatimBorderColor={named}{nbsphinx-code-border}}
\begin{sphinxuseclass}{output_area}
\begin{sphinxuseclass}{}


\begin{sphinxVerbatim}[commandchars=\\\{\}]
\llap{\color{nbsphinxout}[13]:\,\hspace{\fboxrule}\hspace{\fboxsep}}                                rate     cost  expected cost
ImportEE no\_v, t=27         0.000360  15175.0  546300.000000
ImportEE inf\_v, t=27        0.000090  20175.0  181575.000000
ImportWater no\_wat, t=27    0.000150   6175.0   92625.000000
ImportSignal no\_sig, t=27   0.000013  15175.0   19510.714286
MoveWater mech\_break, t=27  0.000231  10175.0  235478.571429
MoveWater short, t=27       0.000129  25175.0  323678.571429
ExportWater block, t=27     0.000129  15152.5  194817.857143
ImportWater no\_wat, t=2     0.000017  11125.0   18541.666667
ImportSignal no\_sig, t=2    0.000002  20125.0    4312.500000
MoveWater mech\_break, t=2   0.000002  15125.0    3241.071429
MoveWater short, t=2        0.000021  30125.0   64553.571429
ExportWater block, t=2      0.000021  20102.5   43076.785714
ImportWater no\_wat, t=52    0.000017   1000.0    1666.666667
ImportSignal no\_sig, t=52   0.000001  10000.0    1428.571429
MoveWater mech\_break, t=52  0.000002   5000.0    1071.428571
MoveWater short, t=52       0.000014  10000.0   14285.714286
ExportWater block, t=52     0.000014   5000.0    7142.857143
nominal                     1.000000      0.0       0.000000
\end{sphinxVerbatim}



\end{sphinxuseclass}
\end{sphinxuseclass}
}

\end{sphinxuseclass}
\end{sphinxuseclass}
\begin{sphinxuseclass}{nbinput}
{
\sphinxsetup{VerbatimColor={named}{nbsphinx-code-bg}}
\sphinxsetup{VerbatimBorderColor={named}{nbsphinx-code-border}}
\begin{sphinxVerbatim}[commandchars=\\\{\}]
\llap{\color{nbsphinxin}[14]:\,\hspace{\fboxrule}\hspace{\fboxsep}}\PYG{n}{pool} \PYG{o}{=} \PYG{n}{mp}\PYG{o}{.}\PYG{n}{Pool}\PYG{p}{(}\PYG{l+m+mi}{4}\PYG{p}{)}
\PYG{n}{endclasses}\PYG{p}{,} \PYG{n}{mdlhists} \PYG{o}{=} \PYG{n}{propagate}\PYG{o}{.}\PYG{n}{single\PYGZus{}faults}\PYG{p}{(}\PYG{n}{mdl}\PYG{p}{,}\PYG{n}{app}\PYG{p}{,} \PYG{n}{pool}\PYG{o}{=}\PYG{n}{pool}\PYG{p}{)}
\PYG{n}{rd}\PYG{o}{.}\PYG{n}{tabulate}\PYG{o}{.}\PYG{n}{simplefmea}\PYG{p}{(}\PYG{n}{endclasses}\PYG{p}{)}
\end{sphinxVerbatim}
}

\end{sphinxuseclass}
\begin{sphinxuseclass}{nboutput}
{

\kern-\sphinxverbatimsmallskipamount\kern-\baselineskip
\kern+\FrameHeightAdjust\kern-\fboxrule
\vspace{\nbsphinxcodecellspacing}

\sphinxsetup{VerbatimColor={named}{nbsphinx-stderr}}
\sphinxsetup{VerbatimBorderColor={named}{nbsphinx-code-border}}
\begin{sphinxuseclass}{output_area}
\begin{sphinxuseclass}{stderr}


\begin{sphinxVerbatim}[commandchars=\\\{\}]
SCENARIOS COMPLETE: 100\%|██████████████████████████████████████████████████████████████| 21/21 [00:01<00:00, 12.06it/s]
\end{sphinxVerbatim}



\end{sphinxuseclass}
\end{sphinxuseclass}
}

\end{sphinxuseclass}
\begin{sphinxuseclass}{nboutput}
\begin{sphinxuseclass}{nblast}
{

\kern-\sphinxverbatimsmallskipamount\kern-\baselineskip
\kern+\FrameHeightAdjust\kern-\fboxrule
\vspace{\nbsphinxcodecellspacing}

\sphinxsetup{VerbatimColor={named}{white}}
\sphinxsetup{VerbatimBorderColor={named}{nbsphinx-code-border}}
\begin{sphinxuseclass}{output_area}
\begin{sphinxuseclass}{}


\begin{sphinxVerbatim}[commandchars=\\\{\}]
\llap{\color{nbsphinxout}[14]:\,\hspace{\fboxrule}\hspace{\fboxsep}}                                rate     cost  expected cost
ImportEE no\_v, t=0          0.000448  20125.0       901600.0
ImportEE inf\_v, t=0         0.000112  25125.0       281400.0
ImportWater no\_wat, t=0     0.000560  11125.0       623000.0
ImportSignal no\_sig, t=0    0.000056  20125.0       112700.0
MoveWater mech\_break, t=0   0.000336  15125.0       508200.0
MoveWater short, t=0        0.000560  30125.0      1687000.0
ExportWater block, t=0      0.000560  20102.5      1125740.0
ImportEE no\_v, t=20         0.000448  16750.0       750400.0
ImportEE inf\_v, t=20        0.000112  21750.0       243600.0
ImportWater no\_wat, t=20    0.000560   7750.0       434000.0
ImportSignal no\_sig, t=20   0.000056  16750.0        93800.0
MoveWater mech\_break, t=20  0.000336  11750.0       394800.0
MoveWater short, t=20       0.000560  26750.0      1498000.0
ExportWater block, t=20     0.000560  16727.5       936740.0
ImportEE no\_v, t=55         0.000448  10000.0       448000.0
ImportEE inf\_v, t=55        0.000112   5000.0        56000.0
ImportWater no\_wat, t=55    0.000560   1000.0        56000.0
ImportSignal no\_sig, t=55   0.000056  10000.0        56000.0
MoveWater mech\_break, t=55  0.000336   5000.0       168000.0
MoveWater short, t=55       0.000560  10000.0       560000.0
ExportWater block, t=55     0.000560   5000.0       280000.0
\end{sphinxVerbatim}



\end{sphinxuseclass}
\end{sphinxuseclass}
}

\end{sphinxuseclass}
\end{sphinxuseclass}
\sphinxAtStartPar
It can also be helpful to verify that the results of parallel simulation and normal serial execution are the same:

\begin{sphinxuseclass}{nbinput}
{
\sphinxsetup{VerbatimColor={named}{nbsphinx-code-bg}}
\sphinxsetup{VerbatimBorderColor={named}{nbsphinx-code-border}}
\begin{sphinxVerbatim}[commandchars=\\\{\}]
\llap{\color{nbsphinxin}[15]:\,\hspace{\fboxrule}\hspace{\fboxsep}}\PYG{n}{endclasses\PYGZus{}par}\PYG{p}{,} \PYG{n}{mdlhists} \PYG{o}{=} \PYG{n}{propagate}\PYG{o}{.}\PYG{n}{single\PYGZus{}faults}\PYG{p}{(}\PYG{n}{mdl}\PYG{p}{,} \PYG{n}{pool}\PYG{o}{=}\PYG{n}{pool}\PYG{p}{)}
\PYG{n}{tab\PYGZus{}par} \PYG{o}{=} \PYG{n}{rd}\PYG{o}{.}\PYG{n}{tabulate}\PYG{o}{.}\PYG{n}{simplefmea}\PYG{p}{(}\PYG{n}{endclasses\PYGZus{}par}\PYG{p}{)}
\PYG{n}{endclasses}\PYG{p}{,} \PYG{n}{mdlhists} \PYG{o}{=} \PYG{n}{propagate}\PYG{o}{.}\PYG{n}{single\PYGZus{}faults}\PYG{p}{(}\PYG{n}{mdl}\PYG{p}{)}
\PYG{n}{tab} \PYG{o}{=} \PYG{n}{rd}\PYG{o}{.}\PYG{n}{tabulate}\PYG{o}{.}\PYG{n}{simplefmea}\PYG{p}{(}\PYG{n}{endclasses}\PYG{p}{)}
\PYG{n}{tab} \PYG{o}{\PYGZhy{}} \PYG{n}{tab\PYGZus{}par}
\end{sphinxVerbatim}
}

\end{sphinxuseclass}
\begin{sphinxuseclass}{nboutput}
{

\kern-\sphinxverbatimsmallskipamount\kern-\baselineskip
\kern+\FrameHeightAdjust\kern-\fboxrule
\vspace{\nbsphinxcodecellspacing}

\sphinxsetup{VerbatimColor={named}{nbsphinx-stderr}}
\sphinxsetup{VerbatimBorderColor={named}{nbsphinx-code-border}}
\begin{sphinxuseclass}{output_area}
\begin{sphinxuseclass}{stderr}


\begin{sphinxVerbatim}[commandchars=\\\{\}]
SCENARIOS COMPLETE: 100\%|█████████████████████████████████████████████████████████████| 21/21 [00:00<00:00, 455.48it/s]
SCENARIOS COMPLETE: 100\%|█████████████████████████████████████████████████████████████| 21/21 [00:00<00:00, 226.33it/s]
\end{sphinxVerbatim}



\end{sphinxuseclass}
\end{sphinxuseclass}
}

\end{sphinxuseclass}
\begin{sphinxuseclass}{nboutput}
\begin{sphinxuseclass}{nblast}
{

\kern-\sphinxverbatimsmallskipamount\kern-\baselineskip
\kern+\FrameHeightAdjust\kern-\fboxrule
\vspace{\nbsphinxcodecellspacing}

\sphinxsetup{VerbatimColor={named}{white}}
\sphinxsetup{VerbatimBorderColor={named}{nbsphinx-code-border}}
\begin{sphinxuseclass}{output_area}
\begin{sphinxuseclass}{}


\begin{sphinxVerbatim}[commandchars=\\\{\}]
\llap{\color{nbsphinxout}[15]:\,\hspace{\fboxrule}\hspace{\fboxsep}}                            rate  cost  expected cost
ImportEE no\_v, t=0           0.0   0.0            0.0
ImportEE inf\_v, t=0          0.0   0.0            0.0
ImportWater no\_wat, t=0      0.0   0.0            0.0
ImportSignal no\_sig, t=0     0.0   0.0            0.0
MoveWater mech\_break, t=0    0.0   0.0            0.0
MoveWater short, t=0         0.0   0.0            0.0
ExportWater block, t=0       0.0   0.0            0.0
ImportEE no\_v, t=20          0.0   0.0            0.0
ImportEE inf\_v, t=20         0.0   0.0            0.0
ImportWater no\_wat, t=20     0.0   0.0            0.0
ImportSignal no\_sig, t=20    0.0   0.0            0.0
MoveWater mech\_break, t=20   0.0   0.0            0.0
MoveWater short, t=20        0.0   0.0            0.0
ExportWater block, t=20      0.0   0.0            0.0
ImportEE no\_v, t=55          0.0   0.0            0.0
ImportEE inf\_v, t=55         0.0   0.0            0.0
ImportWater no\_wat, t=55     0.0   0.0            0.0
ImportSignal no\_sig, t=55    0.0   0.0            0.0
MoveWater mech\_break, t=55   0.0   0.0            0.0
MoveWater short, t=55        0.0   0.0            0.0
ExportWater block, t=55      0.0   0.0            0.0
\end{sphinxVerbatim}



\end{sphinxuseclass}
\end{sphinxuseclass}
}

\end{sphinxuseclass}
\end{sphinxuseclass}
\begin{sphinxuseclass}{nbinput}
{
\sphinxsetup{VerbatimColor={named}{nbsphinx-code-bg}}
\sphinxsetup{VerbatimBorderColor={named}{nbsphinx-code-border}}
\begin{sphinxVerbatim}[commandchars=\\\{\}]
\llap{\color{nbsphinxin}[16]:\,\hspace{\fboxrule}\hspace{\fboxsep}}\PYG{n}{endclasses\PYGZus{}par}\PYG{p}{,} \PYG{n}{mdlhists} \PYG{o}{=} \PYG{n}{propagate}\PYG{o}{.}\PYG{n}{approach}\PYG{p}{(}\PYG{n}{mdl}\PYG{p}{,} \PYG{n}{app}\PYG{p}{,} \PYG{n}{pool}\PYG{o}{=}\PYG{n}{pool}\PYG{p}{)}
\PYG{n}{tab\PYGZus{}par} \PYG{o}{=} \PYG{n}{rd}\PYG{o}{.}\PYG{n}{tabulate}\PYG{o}{.}\PYG{n}{simplefmea}\PYG{p}{(}\PYG{n}{endclasses\PYGZus{}par}\PYG{p}{)}
\PYG{n}{endclasses}\PYG{p}{,} \PYG{n}{mdlhists} \PYG{o}{=} \PYG{n}{propagate}\PYG{o}{.}\PYG{n}{approach}\PYG{p}{(}\PYG{n}{mdl}\PYG{p}{,} \PYG{n}{app}\PYG{p}{)}
\PYG{n}{tab} \PYG{o}{=} \PYG{n}{rd}\PYG{o}{.}\PYG{n}{tabulate}\PYG{o}{.}\PYG{n}{simplefmea}\PYG{p}{(}\PYG{n}{endclasses}\PYG{p}{)}
\PYG{n}{tab} \PYG{o}{\PYGZhy{}} \PYG{n}{tab\PYGZus{}par}
\end{sphinxVerbatim}
}

\end{sphinxuseclass}
\begin{sphinxuseclass}{nboutput}
{

\kern-\sphinxverbatimsmallskipamount\kern-\baselineskip
\kern+\FrameHeightAdjust\kern-\fboxrule
\vspace{\nbsphinxcodecellspacing}

\sphinxsetup{VerbatimColor={named}{nbsphinx-stderr}}
\sphinxsetup{VerbatimBorderColor={named}{nbsphinx-code-border}}
\begin{sphinxuseclass}{output_area}
\begin{sphinxuseclass}{stderr}


\begin{sphinxVerbatim}[commandchars=\\\{\}]
SCENARIOS COMPLETE: 100\%|█████████████████████████████████████████████████████████████| 17/17 [00:00<00:00, 485.43it/s]
SCENARIOS COMPLETE: 100\%|█████████████████████████████████████████████████████████████| 17/17 [00:00<00:00, 202.92it/s]
\end{sphinxVerbatim}



\end{sphinxuseclass}
\end{sphinxuseclass}
}

\end{sphinxuseclass}
\begin{sphinxuseclass}{nboutput}
\begin{sphinxuseclass}{nblast}
{

\kern-\sphinxverbatimsmallskipamount\kern-\baselineskip
\kern+\FrameHeightAdjust\kern-\fboxrule
\vspace{\nbsphinxcodecellspacing}

\sphinxsetup{VerbatimColor={named}{white}}
\sphinxsetup{VerbatimBorderColor={named}{nbsphinx-code-border}}
\begin{sphinxuseclass}{output_area}
\begin{sphinxuseclass}{}


\begin{sphinxVerbatim}[commandchars=\\\{\}]
\llap{\color{nbsphinxout}[16]:\,\hspace{\fboxrule}\hspace{\fboxsep}}                            rate  cost  expected cost
ImportEE no\_v, t=27          0.0   0.0            0.0
ImportEE inf\_v, t=27         0.0   0.0            0.0
ImportWater no\_wat, t=27     0.0   0.0            0.0
ImportSignal no\_sig, t=27    0.0   0.0            0.0
MoveWater mech\_break, t=27   0.0   0.0            0.0
MoveWater short, t=27        0.0   0.0            0.0
ExportWater block, t=27      0.0   0.0            0.0
ImportWater no\_wat, t=2      0.0   0.0            0.0
ImportSignal no\_sig, t=2     0.0   0.0            0.0
MoveWater mech\_break, t=2    0.0   0.0            0.0
MoveWater short, t=2         0.0   0.0            0.0
ExportWater block, t=2       0.0   0.0            0.0
ImportWater no\_wat, t=52     0.0   0.0            0.0
ImportSignal no\_sig, t=52    0.0   0.0            0.0
MoveWater mech\_break, t=52   0.0   0.0            0.0
MoveWater short, t=52        0.0   0.0            0.0
ExportWater block, t=52      0.0   0.0            0.0
nominal                      0.0   0.0            0.0
\end{sphinxVerbatim}



\end{sphinxuseclass}
\end{sphinxuseclass}
}

\end{sphinxuseclass}
\end{sphinxuseclass}
\sphinxAtStartPar
While fmdtools built\sphinxhyphen{}in methods are the easiest way to leverage parallelism, it can also be used with custom arguments/methods to meet the needs of simulation. However, (on Windows) these methods need to be defined in an external module with an “if \sphinxstylestrong{name}==‘\sphinxstylestrong{main}’:” statement, otherwise execution will hang from spawning new processes. This has to do with how multiprocessing works in windows.

\sphinxAtStartPar
To show how parellism can be leveraged manually for a desired use\sphinxhyphen{}case, below the model is run over the blockage fault mode at time t=1 with a different model parameter (delayed failure behavior), as defined in the \sphinxcode{\sphinxupquote{parallelism\_methods.py}} module in this folder.

\begin{sphinxuseclass}{nbinput}
\begin{sphinxuseclass}{nblast}
{
\sphinxsetup{VerbatimColor={named}{nbsphinx-code-bg}}
\sphinxsetup{VerbatimBorderColor={named}{nbsphinx-code-border}}
\begin{sphinxVerbatim}[commandchars=\\\{\}]
\llap{\color{nbsphinxin}[17]:\,\hspace{\fboxrule}\hspace{\fboxsep}}\PYG{k+kn}{from} \PYG{n+nn}{parallelism\PYGZus{}methods} \PYG{k+kn}{import} \PYG{n}{delay\PYGZus{}test}
\end{sphinxVerbatim}
}

\end{sphinxuseclass}
\end{sphinxuseclass}
\begin{sphinxuseclass}{nbinput}
\begin{sphinxuseclass}{nblast}
{
\sphinxsetup{VerbatimColor={named}{nbsphinx-code-bg}}
\sphinxsetup{VerbatimBorderColor={named}{nbsphinx-code-border}}
\begin{sphinxVerbatim}[commandchars=\\\{\}]
\llap{\color{nbsphinxin}[18]:\,\hspace{\fboxrule}\hspace{\fboxsep}}\PYG{n}{results} \PYG{o}{=} \PYG{n}{delay\PYGZus{}test}\PYG{p}{(}\PYG{p}{)}
\PYG{n}{results}
\end{sphinxVerbatim}
}

\end{sphinxuseclass}
\end{sphinxuseclass}
\sphinxAtStartPar
In this method, the model is run many times over a given fault with different delay parameters. It should be noted that this approach is not especially efficient, since the nominal scenario is simulated at each call of \sphinxcode{\sphinxupquote{propagate.one\_fault()}}. It is thus preferred to use the appropriate fault/parameter sampling approaches and propagate methods, since these methods only run the nominal simulation once for fault scenarios and can also use staged execution (copying the model at fault time for
fault scenarios) to reduce \textasciitilde{}1/2 of the cost of each fault simulation.


\subsubsection{Performance Comparison}
\label{\detokenize{example_pump/Parallelism_Tutorial:Performance-Comparison}}
\sphinxAtStartPar
Parallelism is often used in computation to speed up up a set of independent simulations. Conventionally, one might say it leads to a reduced computational cost of \(t/n\), where t was the original time of the set of processes, and n is the number of cores.

\sphinxAtStartPar
However, this computational performance increase is dependent on the implementation. In Python, there is some overhead from from communicating data structures in and out of parallel threads which can become a significant consideration when the data structures are large. Additionally, different Pools can execute more or less efficiently. Below these are each compared.

\begin{sphinxuseclass}{nbinput}
\begin{sphinxuseclass}{nblast}
{
\sphinxsetup{VerbatimColor={named}{nbsphinx-code-bg}}
\sphinxsetup{VerbatimBorderColor={named}{nbsphinx-code-border}}
\begin{sphinxVerbatim}[commandchars=\\\{\}]
\llap{\color{nbsphinxin}[20]:\,\hspace{\fboxrule}\hspace{\fboxsep}}\PYG{k+kn}{import} \PYG{n+nn}{matplotlib}\PYG{n+nn}{.}\PYG{n+nn}{pyplot} \PYG{k}{as} \PYG{n+nn}{plt}
\PYG{k+kn}{from} \PYG{n+nn}{parallelism\PYGZus{}methods} \PYG{k+kn}{import} \PYG{n}{compare\PYGZus{}pools}
\end{sphinxVerbatim}
}

\end{sphinxuseclass}
\end{sphinxuseclass}
\begin{sphinxuseclass}{nbinput}
\begin{sphinxuseclass}{nblast}
{
\sphinxsetup{VerbatimColor={named}{nbsphinx-code-bg}}
\sphinxsetup{VerbatimBorderColor={named}{nbsphinx-code-border}}
\begin{sphinxVerbatim}[commandchars=\\\{\}]
\llap{\color{nbsphinxin}[21]:\,\hspace{\fboxrule}\hspace{\fboxsep}}\PYG{n}{cores}\PYG{o}{=}\PYG{l+m+mi}{4}
\PYG{n}{pools} \PYG{o}{=} \PYG{p}{\PYGZob{}}\PYG{l+s+s1}{\PYGZsq{}}\PYG{l+s+s1}{multiprocessing}\PYG{l+s+s1}{\PYGZsq{}}\PYG{p}{:}\PYG{n}{mp}\PYG{o}{.}\PYG{n}{Pool}\PYG{p}{(}\PYG{n}{cores}\PYG{p}{)}\PYG{p}{,} \PYG{l+s+s1}{\PYGZsq{}}\PYG{l+s+s1}{ProcessPool}\PYG{l+s+s1}{\PYGZsq{}}\PYG{p}{:}\PYG{n}{ProcessPool}\PYG{p}{(}\PYG{n}{nodes}\PYG{o}{=}\PYG{n}{cores}\PYG{p}{)}\PYG{p}{,} \PYG{l+s+s1}{\PYGZsq{}}\PYG{l+s+s1}{ParallelPool}\PYG{l+s+s1}{\PYGZsq{}}\PYG{p}{:} \PYG{n}{ParallelPool}\PYG{p}{(}\PYG{n}{nodes}\PYG{o}{=}\PYG{n}{cores}\PYG{p}{)}\PYG{p}{,} \PYG{l+s+s1}{\PYGZsq{}}\PYG{l+s+s1}{ThreadPool}\PYG{l+s+s1}{\PYGZsq{}}\PYG{p}{:}\PYG{n}{ThreadPool}\PYG{p}{(}\PYG{n}{nodes}\PYG{o}{=}\PYG{n}{cores}\PYG{p}{)}\PYG{p}{,} \PYG{l+s+s1}{\PYGZsq{}}\PYG{l+s+s1}{multiprocess}\PYG{l+s+s1}{\PYGZsq{}}\PYG{p}{:}\PYG{n}{ms}\PYG{o}{.}\PYG{n}{Pool}\PYG{p}{(}\PYG{n}{cores}\PYG{p}{)} \PYG{p}{\PYGZcb{}}
\end{sphinxVerbatim}
}

\end{sphinxuseclass}
\end{sphinxuseclass}
\sphinxAtStartPar
Below is the baseline comparison, where the the following parameters characterize the sampling approach: \sphinxhyphen{} single faults: only the single\sphinxhyphen{}fault scenarios are considered \sphinxhyphen{} 3 points per phase: an evenly\sphinxhyphen{}spaced quadrature is sampled at each phase of operation (start, on, end) for the model \sphinxhyphen{} staged: the model is copied at each point in time where faults is injected during the model time to save computation \sphinxhyphen{} track: the entire model history is returned for each simulation

\sphinxAtStartPar
This is typical for a small model like this where the per\sphinxhyphen{}model expense is low.

\begin{sphinxuseclass}{nbinput}
\begin{sphinxuseclass}{nblast}
{
\sphinxsetup{VerbatimColor={named}{nbsphinx-code-bg}}
\sphinxsetup{VerbatimBorderColor={named}{nbsphinx-code-border}}
\begin{sphinxVerbatim}[commandchars=\\\{\}]
\llap{\color{nbsphinxin}[24]:\,\hspace{\fboxrule}\hspace{\fboxsep}}\PYG{n}{app} \PYG{o}{=} \PYG{n}{SampleApproach}\PYG{p}{(}\PYG{n}{mdl}\PYG{p}{,}\PYG{n}{jointfaults}\PYG{o}{=}\PYG{p}{\PYGZob{}}\PYG{l+s+s1}{\PYGZsq{}}\PYG{l+s+s1}{faults}\PYG{l+s+s1}{\PYGZsq{}}\PYG{p}{:}\PYG{l+m+mi}{1}\PYG{p}{\PYGZcb{}}\PYG{p}{,}\PYG{n}{defaultsamp}\PYG{o}{=}\PYG{p}{\PYGZob{}}\PYG{l+s+s1}{\PYGZsq{}}\PYG{l+s+s1}{samp}\PYG{l+s+s1}{\PYGZsq{}}\PYG{p}{:}\PYG{l+s+s1}{\PYGZsq{}}\PYG{l+s+s1}{evenspacing}\PYG{l+s+s1}{\PYGZsq{}}\PYG{p}{,}\PYG{l+s+s1}{\PYGZsq{}}\PYG{l+s+s1}{numpts}\PYG{l+s+s1}{\PYGZsq{}}\PYG{p}{:}\PYG{l+m+mi}{3}\PYG{p}{\PYGZcb{}}\PYG{p}{)}
\PYG{n}{mdl}\PYG{o}{=}\PYG{n}{Pump}\PYG{p}{(}\PYG{n}{params}\PYG{o}{=}\PYG{p}{\PYGZob{}}\PYG{l+s+s1}{\PYGZsq{}}\PYG{l+s+s1}{cost}\PYG{l+s+s1}{\PYGZsq{}}\PYG{p}{:}\PYG{p}{\PYGZob{}}\PYG{l+s+s1}{\PYGZsq{}}\PYG{l+s+s1}{repair}\PYG{l+s+s1}{\PYGZsq{}}\PYG{p}{\PYGZcb{}}\PYG{p}{,} \PYG{l+s+s1}{\PYGZsq{}}\PYG{l+s+s1}{delay}\PYG{l+s+s1}{\PYGZsq{}}\PYG{p}{:}\PYG{l+m+mi}{10}\PYG{p}{\PYGZcb{}}\PYG{p}{,} \PYG{n}{modelparams} \PYG{o}{=} \PYG{p}{\PYGZob{}}\PYG{l+s+s1}{\PYGZsq{}}\PYG{l+s+s1}{phases}\PYG{l+s+s1}{\PYGZsq{}}\PYG{p}{:}\PYG{p}{\PYGZob{}}\PYG{l+s+s1}{\PYGZsq{}}\PYG{l+s+s1}{start}\PYG{l+s+s1}{\PYGZsq{}}\PYG{p}{:}\PYG{p}{[}\PYG{l+m+mi}{0}\PYG{p}{,}\PYG{l+m+mi}{5}\PYG{p}{]}\PYG{p}{,} \PYG{l+s+s1}{\PYGZsq{}}\PYG{l+s+s1}{on}\PYG{l+s+s1}{\PYGZsq{}}\PYG{p}{:}\PYG{p}{[}\PYG{l+m+mi}{5}\PYG{p}{,} \PYG{l+m+mi}{50}\PYG{p}{]}\PYG{p}{,} \PYG{l+s+s1}{\PYGZsq{}}\PYG{l+s+s1}{end}\PYG{l+s+s1}{\PYGZsq{}}\PYG{p}{:}\PYG{p}{[}\PYG{l+m+mi}{50}\PYG{p}{,}\PYG{l+m+mi}{55}\PYG{p}{]}\PYG{p}{\PYGZcb{}}\PYG{p}{,} \PYG{l+s+s1}{\PYGZsq{}}\PYG{l+s+s1}{times}\PYG{l+s+s1}{\PYGZsq{}}\PYG{p}{:}\PYG{p}{[}\PYG{l+m+mi}{0}\PYG{p}{,}\PYG{l+m+mi}{20}\PYG{p}{,} \PYG{l+m+mi}{55}\PYG{p}{]}\PYG{p}{,} \PYG{l+s+s1}{\PYGZsq{}}\PYG{l+s+s1}{tstep}\PYG{l+s+s1}{\PYGZsq{}}\PYG{p}{:}\PYG{l+m+mi}{1}\PYG{p}{\PYGZcb{}}\PYG{p}{)}
\PYG{n}{exectimes} \PYG{o}{=} \PYG{n}{compare\PYGZus{}pools}\PYG{p}{(}\PYG{n}{mdl}\PYG{p}{,}\PYG{n}{app}\PYG{p}{,}\PYG{n}{pools}\PYG{p}{,} \PYG{n}{staged}\PYG{o}{=}\PYG{k+kc}{True}\PYG{p}{,} \PYG{n}{track}\PYG{o}{=}\PYG{l+s+s1}{\PYGZsq{}}\PYG{l+s+s1}{all}\PYG{l+s+s1}{\PYGZsq{}}\PYG{p}{,} \PYG{n}{verbose}\PYG{o}{=}\PYG{k+kc}{False}\PYG{p}{)}
\PYG{n}{exectimes\PYGZus{}baseline} \PYG{o}{=} \PYG{n}{exectimes}
\end{sphinxVerbatim}
}

\end{sphinxuseclass}
\end{sphinxuseclass}
\begin{sphinxuseclass}{nbinput}
{
\sphinxsetup{VerbatimColor={named}{nbsphinx-code-bg}}
\sphinxsetup{VerbatimBorderColor={named}{nbsphinx-code-border}}
\begin{sphinxVerbatim}[commandchars=\\\{\}]
\llap{\color{nbsphinxin}[25]:\,\hspace{\fboxrule}\hspace{\fboxsep}}\PYG{n}{fig} \PYG{o}{=} \PYG{n}{plt}\PYG{o}{.}\PYG{n}{figure}\PYG{p}{(}\PYG{n}{figsize}\PYG{o}{=}\PYG{p}{(}\PYG{l+m+mi}{9}\PYG{p}{,} \PYG{l+m+mi}{3}\PYG{p}{)}\PYG{p}{,}\PYG{p}{)}
\PYG{n}{plt}\PYG{o}{.}\PYG{n}{bar}\PYG{p}{(}\PYG{n+nb}{range}\PYG{p}{(}\PYG{n+nb}{len}\PYG{p}{(}\PYG{n}{exectimes}\PYG{p}{)}\PYG{p}{)}\PYG{p}{,} \PYG{n+nb}{list}\PYG{p}{(}\PYG{n}{exectimes}\PYG{o}{.}\PYG{n}{values}\PYG{p}{(}\PYG{p}{)}\PYG{p}{)}\PYG{p}{,} \PYG{n}{align}\PYG{o}{=}\PYG{l+s+s1}{\PYGZsq{}}\PYG{l+s+s1}{center}\PYG{l+s+s1}{\PYGZsq{}}\PYG{p}{)}
\PYG{n}{plt}\PYG{o}{.}\PYG{n}{xticks}\PYG{p}{(}\PYG{n+nb}{range}\PYG{p}{(}\PYG{n+nb}{len}\PYG{p}{(}\PYG{n}{exectimes}\PYG{p}{)}\PYG{p}{)}\PYG{p}{,} \PYG{n+nb}{list}\PYG{p}{(}\PYG{n}{exectimes}\PYG{o}{.}\PYG{n}{keys}\PYG{p}{(}\PYG{p}{)}\PYG{p}{)}\PYG{p}{)}
\PYG{n}{plt}\PYG{o}{.}\PYG{n}{title}\PYG{p}{(}\PYG{l+s+s2}{\PYGZdq{}}\PYG{l+s+s2}{Baseline Performance \PYGZhy{} Some faults, Staged, Normal Simulation, Full Model History}\PYG{l+s+s2}{\PYGZdq{}}\PYG{p}{)}
\PYG{n}{plt}\PYG{o}{.}\PYG{n}{ylabel}\PYG{p}{(}\PYG{l+s+s2}{\PYGZdq{}}\PYG{l+s+s2}{Computational Time (s)}\PYG{l+s+s2}{\PYGZdq{}}\PYG{p}{)}
\PYG{n}{plt}\PYG{o}{.}\PYG{n}{grid}\PYG{p}{(}\PYG{n}{axis}\PYG{o}{=}\PYG{l+s+s1}{\PYGZsq{}}\PYG{l+s+s1}{y}\PYG{l+s+s1}{\PYGZsq{}}\PYG{p}{)}
\end{sphinxVerbatim}
}

\end{sphinxuseclass}
\begin{sphinxuseclass}{nboutput}
\begin{sphinxuseclass}{nblast}
\hrule height -\fboxrule\relax
\vspace{\nbsphinxcodecellspacing}

\makeatletter\setbox\nbsphinxpromptbox\box\voidb@x\makeatother

\begin{nbsphinxfancyoutput}

\begin{sphinxuseclass}{output_area}
\begin{sphinxuseclass}{}
\noindent\sphinxincludegraphics[width=559\sphinxpxdimen,height=210\sphinxpxdimen]{{example_pump_Parallelism_Tutorial_34_0}.png}

\end{sphinxuseclass}
\end{sphinxuseclass}
\end{nbsphinxfancyoutput}

\end{sphinxuseclass}
\end{sphinxuseclass}
\sphinxAtStartPar
As shown, in this situation, both the multiprocessing and threadpool pools give computational performance increases.


\paragraph{Comparison: No Histories}
\label{\detokenize{example_pump/Parallelism_Tutorial:Comparison:-No-Histories}}
\sphinxAtStartPar
In the below comparison, the same simulation approach is run, except without tracking a history of model states through the simulation.

\begin{sphinxuseclass}{nbinput}
\begin{sphinxuseclass}{nblast}
{
\sphinxsetup{VerbatimColor={named}{nbsphinx-code-bg}}
\sphinxsetup{VerbatimBorderColor={named}{nbsphinx-code-border}}
\begin{sphinxVerbatim}[commandchars=\\\{\}]
\llap{\color{nbsphinxin}[26]:\,\hspace{\fboxrule}\hspace{\fboxsep}}\PYG{n}{app} \PYG{o}{=} \PYG{n}{SampleApproach}\PYG{p}{(}\PYG{n}{mdl}\PYG{p}{,}\PYG{n}{jointfaults}\PYG{o}{=}\PYG{p}{\PYGZob{}}\PYG{l+s+s1}{\PYGZsq{}}\PYG{l+s+s1}{faults}\PYG{l+s+s1}{\PYGZsq{}}\PYG{p}{:}\PYG{l+m+mi}{1}\PYG{p}{\PYGZcb{}}\PYG{p}{,}\PYG{n}{defaultsamp}\PYG{o}{=}\PYG{p}{\PYGZob{}}\PYG{l+s+s1}{\PYGZsq{}}\PYG{l+s+s1}{samp}\PYG{l+s+s1}{\PYGZsq{}}\PYG{p}{:}\PYG{l+s+s1}{\PYGZsq{}}\PYG{l+s+s1}{evenspacing}\PYG{l+s+s1}{\PYGZsq{}}\PYG{p}{,}\PYG{l+s+s1}{\PYGZsq{}}\PYG{l+s+s1}{numpts}\PYG{l+s+s1}{\PYGZsq{}}\PYG{p}{:}\PYG{l+m+mi}{3}\PYG{p}{\PYGZcb{}}\PYG{p}{)}
\PYG{n}{mdl}\PYG{o}{=}\PYG{n}{Pump}\PYG{p}{(}\PYG{n}{params}\PYG{o}{=}\PYG{p}{\PYGZob{}}\PYG{l+s+s1}{\PYGZsq{}}\PYG{l+s+s1}{cost}\PYG{l+s+s1}{\PYGZsq{}}\PYG{p}{:}\PYG{p}{\PYGZob{}}\PYG{l+s+s1}{\PYGZsq{}}\PYG{l+s+s1}{repair}\PYG{l+s+s1}{\PYGZsq{}}\PYG{p}{\PYGZcb{}}\PYG{p}{,} \PYG{l+s+s1}{\PYGZsq{}}\PYG{l+s+s1}{delay}\PYG{l+s+s1}{\PYGZsq{}}\PYG{p}{:}\PYG{l+m+mi}{10}\PYG{p}{\PYGZcb{}}\PYG{p}{,} \PYG{n}{modelparams} \PYG{o}{=} \PYG{p}{\PYGZob{}}\PYG{l+s+s1}{\PYGZsq{}}\PYG{l+s+s1}{phases}\PYG{l+s+s1}{\PYGZsq{}}\PYG{p}{:}\PYG{p}{\PYGZob{}}\PYG{l+s+s1}{\PYGZsq{}}\PYG{l+s+s1}{start}\PYG{l+s+s1}{\PYGZsq{}}\PYG{p}{:}\PYG{p}{[}\PYG{l+m+mi}{0}\PYG{p}{,}\PYG{l+m+mi}{5}\PYG{p}{]}\PYG{p}{,} \PYG{l+s+s1}{\PYGZsq{}}\PYG{l+s+s1}{on}\PYG{l+s+s1}{\PYGZsq{}}\PYG{p}{:}\PYG{p}{[}\PYG{l+m+mi}{5}\PYG{p}{,} \PYG{l+m+mi}{50}\PYG{p}{]}\PYG{p}{,} \PYG{l+s+s1}{\PYGZsq{}}\PYG{l+s+s1}{end}\PYG{l+s+s1}{\PYGZsq{}}\PYG{p}{:}\PYG{p}{[}\PYG{l+m+mi}{50}\PYG{p}{,}\PYG{l+m+mi}{55}\PYG{p}{]}\PYG{p}{\PYGZcb{}}\PYG{p}{,} \PYG{l+s+s1}{\PYGZsq{}}\PYG{l+s+s1}{times}\PYG{l+s+s1}{\PYGZsq{}}\PYG{p}{:}\PYG{p}{[}\PYG{l+m+mi}{0}\PYG{p}{,}\PYG{l+m+mi}{20}\PYG{p}{,} \PYG{l+m+mi}{55}\PYG{p}{]}\PYG{p}{,} \PYG{l+s+s1}{\PYGZsq{}}\PYG{l+s+s1}{tstep}\PYG{l+s+s1}{\PYGZsq{}}\PYG{p}{:}\PYG{l+m+mi}{1}\PYG{p}{\PYGZcb{}}\PYG{p}{)}
\PYG{n}{exectimes} \PYG{o}{=} \PYG{n}{compare\PYGZus{}pools}\PYG{p}{(}\PYG{n}{mdl}\PYG{p}{,}\PYG{n}{app}\PYG{p}{,}\PYG{n}{pools}\PYG{p}{,} \PYG{n}{staged}\PYG{o}{=}\PYG{k+kc}{True}\PYG{p}{,} \PYG{n}{track}\PYG{o}{=}\PYG{l+s+s1}{\PYGZsq{}}\PYG{l+s+s1}{none}\PYG{l+s+s1}{\PYGZsq{}}\PYG{p}{,} \PYG{n}{verbose}\PYG{o}{=}\PYG{k+kc}{False}\PYG{p}{)}
\end{sphinxVerbatim}
}

\end{sphinxuseclass}
\end{sphinxuseclass}
\begin{sphinxuseclass}{nbinput}
{
\sphinxsetup{VerbatimColor={named}{nbsphinx-code-bg}}
\sphinxsetup{VerbatimBorderColor={named}{nbsphinx-code-border}}
\begin{sphinxVerbatim}[commandchars=\\\{\}]
\llap{\color{nbsphinxin}[27]:\,\hspace{\fboxrule}\hspace{\fboxsep}}\PYG{n}{fig} \PYG{o}{=} \PYG{n}{plt}\PYG{o}{.}\PYG{n}{figure}\PYG{p}{(}\PYG{n}{figsize}\PYG{o}{=}\PYG{p}{(}\PYG{l+m+mi}{9}\PYG{p}{,} \PYG{l+m+mi}{3}\PYG{p}{)}\PYG{p}{,}\PYG{p}{)}
\PYG{n}{width} \PYG{o}{=} \PYG{l+m+mf}{0.8}
\PYG{n}{plt}\PYG{o}{.}\PYG{n}{bar}\PYG{p}{(}\PYG{n+nb}{range}\PYG{p}{(}\PYG{n+nb}{len}\PYG{p}{(}\PYG{n}{exectimes}\PYG{p}{)}\PYG{p}{)}\PYG{p}{,} \PYG{n+nb}{list}\PYG{p}{(}\PYG{n}{exectimes}\PYG{o}{.}\PYG{n}{values}\PYG{p}{(}\PYG{p}{)}\PYG{p}{)}\PYG{p}{,} \PYG{n}{align}\PYG{o}{=}\PYG{l+s+s1}{\PYGZsq{}}\PYG{l+s+s1}{center}\PYG{l+s+s1}{\PYGZsq{}}\PYG{p}{,} \PYG{n}{color}\PYG{o}{=}\PYG{l+s+s2}{\PYGZdq{}}\PYG{l+s+s2}{blue}\PYG{l+s+s2}{\PYGZdq{}}\PYG{p}{,} \PYG{n}{label}\PYG{o}{=}\PYG{l+s+s2}{\PYGZdq{}}\PYG{l+s+s2}{comparison}\PYG{l+s+s2}{\PYGZdq{}}\PYG{p}{)}
\PYG{n}{plt}\PYG{o}{.}\PYG{n}{bar}\PYG{p}{(}\PYG{n+nb}{range}\PYG{p}{(}\PYG{n+nb}{len}\PYG{p}{(}\PYG{n}{exectimes\PYGZus{}baseline}\PYG{p}{)}\PYG{p}{)}\PYG{p}{,} \PYG{n+nb}{list}\PYG{p}{(}\PYG{n}{exectimes\PYGZus{}baseline}\PYG{o}{.}\PYG{n}{values}\PYG{p}{(}\PYG{p}{)}\PYG{p}{)}\PYG{p}{,} \PYG{n}{align}\PYG{o}{=}\PYG{l+s+s1}{\PYGZsq{}}\PYG{l+s+s1}{center}\PYG{l+s+s1}{\PYGZsq{}}\PYG{p}{,} \PYG{n}{color}\PYG{o}{=}\PYG{l+s+s2}{\PYGZdq{}}\PYG{l+s+s2}{gray}\PYG{l+s+s2}{\PYGZdq{}}\PYG{p}{,} \PYG{n}{alpha}\PYG{o}{=}\PYG{l+m+mf}{0.5}\PYG{p}{,} \PYG{n}{label}\PYG{o}{=}\PYG{l+s+s2}{\PYGZdq{}}\PYG{l+s+s2}{baseline}\PYG{l+s+s2}{\PYGZdq{}}\PYG{p}{)}
\PYG{n}{plt}\PYG{o}{.}\PYG{n}{xticks}\PYG{p}{(}\PYG{n+nb}{range}\PYG{p}{(}\PYG{n+nb}{len}\PYG{p}{(}\PYG{n}{exectimes}\PYG{p}{)}\PYG{p}{)}\PYG{p}{,} \PYG{n+nb}{list}\PYG{p}{(}\PYG{n}{exectimes}\PYG{o}{.}\PYG{n}{keys}\PYG{p}{(}\PYG{p}{)}\PYG{p}{)}\PYG{p}{)}
\PYG{n}{plt}\PYG{o}{.}\PYG{n}{title}\PYG{p}{(}\PYG{l+s+s2}{\PYGZdq{}}\PYG{l+s+s2}{Computational Performance \PYGZhy{} Many Faults, Staged, Normal Simulation, No Model History}\PYG{l+s+s2}{\PYGZdq{}}\PYG{p}{)}
\PYG{n}{plt}\PYG{o}{.}\PYG{n}{ylabel}\PYG{p}{(}\PYG{l+s+s2}{\PYGZdq{}}\PYG{l+s+s2}{Computational Time (s)}\PYG{l+s+s2}{\PYGZdq{}}\PYG{p}{)}
\PYG{n}{plt}\PYG{o}{.}\PYG{n}{grid}\PYG{p}{(}\PYG{n}{axis}\PYG{o}{=}\PYG{l+s+s1}{\PYGZsq{}}\PYG{l+s+s1}{y}\PYG{l+s+s1}{\PYGZsq{}}\PYG{p}{)}
\PYG{n}{plt}\PYG{o}{.}\PYG{n}{legend}\PYG{p}{(}\PYG{p}{)}
\end{sphinxVerbatim}
}

\end{sphinxuseclass}
\begin{sphinxuseclass}{nboutput}
{

\kern-\sphinxverbatimsmallskipamount\kern-\baselineskip
\kern+\FrameHeightAdjust\kern-\fboxrule
\vspace{\nbsphinxcodecellspacing}

\sphinxsetup{VerbatimColor={named}{white}}
\sphinxsetup{VerbatimBorderColor={named}{nbsphinx-code-border}}
\begin{sphinxuseclass}{output_area}
\begin{sphinxuseclass}{}


\begin{sphinxVerbatim}[commandchars=\\\{\}]
\llap{\color{nbsphinxout}[27]:\,\hspace{\fboxrule}\hspace{\fboxsep}}<matplotlib.legend.Legend at 0x19d34b19130>
\end{sphinxVerbatim}



\end{sphinxuseclass}
\end{sphinxuseclass}
}

\end{sphinxuseclass}
\begin{sphinxuseclass}{nboutput}
\begin{sphinxuseclass}{nblast}
\hrule height -\fboxrule\relax
\vspace{\nbsphinxcodecellspacing}

\makeatletter\setbox\nbsphinxpromptbox\box\voidb@x\makeatother

\begin{nbsphinxfancyoutput}

\begin{sphinxuseclass}{output_area}
\begin{sphinxuseclass}{}
\noindent\sphinxincludegraphics[width=574\sphinxpxdimen,height=210\sphinxpxdimen]{{example_pump_Parallelism_Tutorial_38_1}.png}

\end{sphinxuseclass}
\end{sphinxuseclass}
\end{nbsphinxfancyoutput}

\end{sphinxuseclass}
\end{sphinxuseclass}
\sphinxAtStartPar
As shown, in this situation, the overall simulation expense decreases dramatically (\textasciitilde{}1/2), even in the serial execution case.

\sphinxAtStartPar
Additionally, the case for using a parallel processing pool increases immensely (the multiprocessing pool takes \textasciitilde{}1/4 of the normal simulation time). This is because passing the model history back to the main process is nearly comparable in time to simulation itself.

\sphinxAtStartPar
As a result, removing it saves a large amount of computational time when using parallel processing.


\paragraph{Comparison: Many Faults}
\label{\detokenize{example_pump/Parallelism_Tutorial:Comparison:-Many-Faults}}
\sphinxAtStartPar
In the below comparison, many faults are injected in the system to increase the number of scenarios (ostensibly making the case better for parallelism)

\begin{sphinxuseclass}{nbinput}
{
\sphinxsetup{VerbatimColor={named}{nbsphinx-code-bg}}
\sphinxsetup{VerbatimBorderColor={named}{nbsphinx-code-border}}
\begin{sphinxVerbatim}[commandchars=\\\{\}]
\llap{\color{nbsphinxin}[28]:\,\hspace{\fboxrule}\hspace{\fboxsep}}\PYG{n}{app} \PYG{o}{=} \PYG{n}{SampleApproach}\PYG{p}{(}\PYG{n}{mdl}\PYG{p}{,}\PYG{n}{jointfaults}\PYG{o}{=}\PYG{p}{\PYGZob{}}\PYG{l+s+s1}{\PYGZsq{}}\PYG{l+s+s1}{faults}\PYG{l+s+s1}{\PYGZsq{}}\PYG{p}{:}\PYG{l+m+mi}{5}\PYG{p}{\PYGZcb{}}\PYG{p}{,}\PYG{n}{defaultsamp}\PYG{o}{=}\PYG{p}{\PYGZob{}}\PYG{l+s+s1}{\PYGZsq{}}\PYG{l+s+s1}{samp}\PYG{l+s+s1}{\PYGZsq{}}\PYG{p}{:}\PYG{l+s+s1}{\PYGZsq{}}\PYG{l+s+s1}{evenspacing}\PYG{l+s+s1}{\PYGZsq{}}\PYG{p}{,}\PYG{l+s+s1}{\PYGZsq{}}\PYG{l+s+s1}{numpts}\PYG{l+s+s1}{\PYGZsq{}}\PYG{p}{:}\PYG{l+m+mi}{3}\PYG{p}{\PYGZcb{}}\PYG{p}{)}
\PYG{n}{mdl}\PYG{o}{=}\PYG{n}{Pump}\PYG{p}{(}\PYG{n}{params}\PYG{o}{=}\PYG{p}{\PYGZob{}}\PYG{l+s+s1}{\PYGZsq{}}\PYG{l+s+s1}{cost}\PYG{l+s+s1}{\PYGZsq{}}\PYG{p}{:}\PYG{p}{\PYGZob{}}\PYG{l+s+s1}{\PYGZsq{}}\PYG{l+s+s1}{repair}\PYG{l+s+s1}{\PYGZsq{}}\PYG{p}{\PYGZcb{}}\PYG{p}{,} \PYG{l+s+s1}{\PYGZsq{}}\PYG{l+s+s1}{delay}\PYG{l+s+s1}{\PYGZsq{}}\PYG{p}{:}\PYG{l+m+mi}{10}\PYG{p}{\PYGZcb{}}\PYG{p}{,} \PYG{n}{modelparams} \PYG{o}{=} \PYG{p}{\PYGZob{}}\PYG{l+s+s1}{\PYGZsq{}}\PYG{l+s+s1}{phases}\PYG{l+s+s1}{\PYGZsq{}}\PYG{p}{:}\PYG{p}{\PYGZob{}}\PYG{l+s+s1}{\PYGZsq{}}\PYG{l+s+s1}{start}\PYG{l+s+s1}{\PYGZsq{}}\PYG{p}{:}\PYG{p}{[}\PYG{l+m+mi}{0}\PYG{p}{,}\PYG{l+m+mi}{5}\PYG{p}{]}\PYG{p}{,} \PYG{l+s+s1}{\PYGZsq{}}\PYG{l+s+s1}{on}\PYG{l+s+s1}{\PYGZsq{}}\PYG{p}{:}\PYG{p}{[}\PYG{l+m+mi}{5}\PYG{p}{,} \PYG{l+m+mi}{50}\PYG{p}{]}\PYG{p}{,} \PYG{l+s+s1}{\PYGZsq{}}\PYG{l+s+s1}{end}\PYG{l+s+s1}{\PYGZsq{}}\PYG{p}{:}\PYG{p}{[}\PYG{l+m+mi}{50}\PYG{p}{,}\PYG{l+m+mi}{55}\PYG{p}{]}\PYG{p}{\PYGZcb{}}\PYG{p}{,} \PYG{l+s+s1}{\PYGZsq{}}\PYG{l+s+s1}{times}\PYG{l+s+s1}{\PYGZsq{}}\PYG{p}{:}\PYG{p}{[}\PYG{l+m+mi}{0}\PYG{p}{,}\PYG{l+m+mi}{20}\PYG{p}{,} \PYG{l+m+mi}{55}\PYG{p}{]}\PYG{p}{,} \PYG{l+s+s1}{\PYGZsq{}}\PYG{l+s+s1}{tstep}\PYG{l+s+s1}{\PYGZsq{}}\PYG{p}{:}\PYG{l+m+mi}{1}\PYG{p}{\PYGZcb{}}\PYG{p}{)}
\PYG{n}{exectimes} \PYG{o}{=} \PYG{n}{compare\PYGZus{}pools}\PYG{p}{(}\PYG{n}{mdl}\PYG{p}{,}\PYG{n}{app}\PYG{p}{,}\PYG{n}{pools}\PYG{p}{,} \PYG{n}{staged}\PYG{o}{=}\PYG{k+kc}{True}\PYG{p}{,} \PYG{n}{track}\PYG{o}{=}\PYG{l+s+s1}{\PYGZsq{}}\PYG{l+s+s1}{None}\PYG{l+s+s1}{\PYGZsq{}}\PYG{p}{,} \PYG{n}{verbose}\PYG{o}{=}\PYG{k+kc}{False}\PYG{p}{)}
\end{sphinxVerbatim}
}

\end{sphinxuseclass}
\begin{sphinxuseclass}{nboutput}
\begin{sphinxuseclass}{nblast}
{

\kern-\sphinxverbatimsmallskipamount\kern-\baselineskip
\kern+\FrameHeightAdjust\kern-\fboxrule
\vspace{\nbsphinxcodecellspacing}

\sphinxsetup{VerbatimColor={named}{nbsphinx-stderr}}
\sphinxsetup{VerbatimBorderColor={named}{nbsphinx-code-border}}
\begin{sphinxuseclass}{output_area}
\begin{sphinxuseclass}{stderr}


\begin{sphinxVerbatim}[commandchars=\\\{\}]
C:\textbackslash{}Users\textbackslash{}dhulse\textbackslash{}Documents\textbackslash{}GitHub\textbackslash{}fmdtools\textbackslash{}example\_pump\textbackslash{}..\textbackslash{}fmdtools\textbackslash{}modeldef.py:1879: RuntimeWarning: invalid value encountered in double\_scalars
  if len(overlap)>1:  self.rates\_timeless[jointmode][phaseid] = self.rates[jointmode][phaseid]/(overlap[1]-overlap[0])
\end{sphinxVerbatim}



\end{sphinxuseclass}
\end{sphinxuseclass}
}

\end{sphinxuseclass}
\end{sphinxuseclass}
\begin{sphinxuseclass}{nbinput}
{
\sphinxsetup{VerbatimColor={named}{nbsphinx-code-bg}}
\sphinxsetup{VerbatimBorderColor={named}{nbsphinx-code-border}}
\begin{sphinxVerbatim}[commandchars=\\\{\}]
\llap{\color{nbsphinxin}[29]:\,\hspace{\fboxrule}\hspace{\fboxsep}}\PYG{n}{fig} \PYG{o}{=} \PYG{n}{plt}\PYG{o}{.}\PYG{n}{figure}\PYG{p}{(}\PYG{n}{figsize}\PYG{o}{=}\PYG{p}{(}\PYG{l+m+mi}{9}\PYG{p}{,} \PYG{l+m+mi}{3}\PYG{p}{)}\PYG{p}{,}\PYG{p}{)}
\PYG{n}{width} \PYG{o}{=} \PYG{l+m+mf}{0.8}
\PYG{n}{plt}\PYG{o}{.}\PYG{n}{bar}\PYG{p}{(}\PYG{n+nb}{range}\PYG{p}{(}\PYG{n+nb}{len}\PYG{p}{(}\PYG{n}{exectimes}\PYG{p}{)}\PYG{p}{)}\PYG{p}{,} \PYG{n+nb}{list}\PYG{p}{(}\PYG{n}{exectimes}\PYG{o}{.}\PYG{n}{values}\PYG{p}{(}\PYG{p}{)}\PYG{p}{)}\PYG{p}{,} \PYG{n}{align}\PYG{o}{=}\PYG{l+s+s1}{\PYGZsq{}}\PYG{l+s+s1}{center}\PYG{l+s+s1}{\PYGZsq{}}\PYG{p}{,} \PYG{n}{color}\PYG{o}{=}\PYG{l+s+s2}{\PYGZdq{}}\PYG{l+s+s2}{blue}\PYG{l+s+s2}{\PYGZdq{}}\PYG{p}{,} \PYG{n}{label}\PYG{o}{=}\PYG{l+s+s2}{\PYGZdq{}}\PYG{l+s+s2}{comparison}\PYG{l+s+s2}{\PYGZdq{}}\PYG{p}{)}
\PYG{n}{plt}\PYG{o}{.}\PYG{n}{bar}\PYG{p}{(}\PYG{n+nb}{range}\PYG{p}{(}\PYG{n+nb}{len}\PYG{p}{(}\PYG{n}{exectimes\PYGZus{}baseline}\PYG{p}{)}\PYG{p}{)}\PYG{p}{,} \PYG{n+nb}{list}\PYG{p}{(}\PYG{n}{exectimes\PYGZus{}baseline}\PYG{o}{.}\PYG{n}{values}\PYG{p}{(}\PYG{p}{)}\PYG{p}{)}\PYG{p}{,} \PYG{n}{align}\PYG{o}{=}\PYG{l+s+s1}{\PYGZsq{}}\PYG{l+s+s1}{center}\PYG{l+s+s1}{\PYGZsq{}}\PYG{p}{,} \PYG{n}{color}\PYG{o}{=}\PYG{l+s+s2}{\PYGZdq{}}\PYG{l+s+s2}{gray}\PYG{l+s+s2}{\PYGZdq{}}\PYG{p}{,} \PYG{n}{alpha}\PYG{o}{=}\PYG{l+m+mf}{0.8}\PYG{p}{,} \PYG{n}{label}\PYG{o}{=}\PYG{l+s+s2}{\PYGZdq{}}\PYG{l+s+s2}{baseline}\PYG{l+s+s2}{\PYGZdq{}}\PYG{p}{)}
\PYG{n}{plt}\PYG{o}{.}\PYG{n}{xticks}\PYG{p}{(}\PYG{n+nb}{range}\PYG{p}{(}\PYG{n+nb}{len}\PYG{p}{(}\PYG{n}{exectimes}\PYG{p}{)}\PYG{p}{)}\PYG{p}{,} \PYG{n+nb}{list}\PYG{p}{(}\PYG{n}{exectimes}\PYG{o}{.}\PYG{n}{keys}\PYG{p}{(}\PYG{p}{)}\PYG{p}{)}\PYG{p}{)}
\PYG{n}{plt}\PYG{o}{.}\PYG{n}{title}\PYG{p}{(}\PYG{l+s+s2}{\PYGZdq{}}\PYG{l+s+s2}{Computational Performance \PYGZhy{} Many Faults, Staged, Normal Simulation, Full Model History}\PYG{l+s+s2}{\PYGZdq{}}\PYG{p}{)}
\PYG{n}{plt}\PYG{o}{.}\PYG{n}{ylabel}\PYG{p}{(}\PYG{l+s+s2}{\PYGZdq{}}\PYG{l+s+s2}{Computational Time (s)}\PYG{l+s+s2}{\PYGZdq{}}\PYG{p}{)}
\PYG{n}{plt}\PYG{o}{.}\PYG{n}{grid}\PYG{p}{(}\PYG{n}{axis}\PYG{o}{=}\PYG{l+s+s1}{\PYGZsq{}}\PYG{l+s+s1}{y}\PYG{l+s+s1}{\PYGZsq{}}\PYG{p}{)}
\PYG{n}{plt}\PYG{o}{.}\PYG{n}{legend}\PYG{p}{(}\PYG{p}{)}
\end{sphinxVerbatim}
}

\end{sphinxuseclass}
\begin{sphinxuseclass}{nboutput}
{

\kern-\sphinxverbatimsmallskipamount\kern-\baselineskip
\kern+\FrameHeightAdjust\kern-\fboxrule
\vspace{\nbsphinxcodecellspacing}

\sphinxsetup{VerbatimColor={named}{white}}
\sphinxsetup{VerbatimBorderColor={named}{nbsphinx-code-border}}
\begin{sphinxuseclass}{output_area}
\begin{sphinxuseclass}{}


\begin{sphinxVerbatim}[commandchars=\\\{\}]
\llap{\color{nbsphinxout}[29]:\,\hspace{\fboxrule}\hspace{\fboxsep}}<matplotlib.legend.Legend at 0x19d34b13f70>
\end{sphinxVerbatim}



\end{sphinxuseclass}
\end{sphinxuseclass}
}

\end{sphinxuseclass}
\begin{sphinxuseclass}{nboutput}
\begin{sphinxuseclass}{nblast}
\hrule height -\fboxrule\relax
\vspace{\nbsphinxcodecellspacing}

\makeatletter\setbox\nbsphinxpromptbox\box\voidb@x\makeatother

\begin{nbsphinxfancyoutput}

\begin{sphinxuseclass}{output_area}
\begin{sphinxuseclass}{}
\noindent\sphinxincludegraphics[width=570\sphinxpxdimen,height=210\sphinxpxdimen]{{example_pump_Parallelism_Tutorial_42_1}.png}

\end{sphinxuseclass}
\end{sphinxuseclass}
\end{nbsphinxfancyoutput}

\end{sphinxuseclass}
\end{sphinxuseclass}
\sphinxAtStartPar
As shown, increasing the number of joint\sphinxhyphen{}fault scenarios increases computational costs significantly–as would be expected.

\sphinxAtStartPar
In this situation, multiprocessing performs comparatively better, but only slightly–instead of taking 1/4 the time, it only takes about 1/2 the time.


\paragraph{Comparison: Long simulation}
\label{\detokenize{example_pump/Parallelism_Tutorial:Comparison:-Long-simulation}}
\sphinxAtStartPar
It may be of interest to simulate how the comparative performance changes for longer simulations. In this comparison, the simulation time is extended tenfold.

\begin{sphinxuseclass}{nbinput}
\begin{sphinxuseclass}{nblast}
{
\sphinxsetup{VerbatimColor={named}{nbsphinx-code-bg}}
\sphinxsetup{VerbatimBorderColor={named}{nbsphinx-code-border}}
\begin{sphinxVerbatim}[commandchars=\\\{\}]
\llap{\color{nbsphinxin}[30]:\,\hspace{\fboxrule}\hspace{\fboxsep}}\PYG{n}{app} \PYG{o}{=} \PYG{n}{SampleApproach}\PYG{p}{(}\PYG{n}{mdl}\PYG{p}{,}\PYG{n}{jointfaults}\PYG{o}{=}\PYG{p}{\PYGZob{}}\PYG{l+s+s1}{\PYGZsq{}}\PYG{l+s+s1}{faults}\PYG{l+s+s1}{\PYGZsq{}}\PYG{p}{:}\PYG{l+m+mi}{1}\PYG{p}{\PYGZcb{}}\PYG{p}{,}\PYG{n}{defaultsamp}\PYG{o}{=}\PYG{p}{\PYGZob{}}\PYG{l+s+s1}{\PYGZsq{}}\PYG{l+s+s1}{samp}\PYG{l+s+s1}{\PYGZsq{}}\PYG{p}{:}\PYG{l+s+s1}{\PYGZsq{}}\PYG{l+s+s1}{evenspacing}\PYG{l+s+s1}{\PYGZsq{}}\PYG{p}{,}\PYG{l+s+s1}{\PYGZsq{}}\PYG{l+s+s1}{numpts}\PYG{l+s+s1}{\PYGZsq{}}\PYG{p}{:}\PYG{l+m+mi}{3}\PYG{p}{\PYGZcb{}}\PYG{p}{)}
\PYG{n}{mdl}\PYG{o}{=}\PYG{n}{Pump}\PYG{p}{(}\PYG{n}{params}\PYG{o}{=}\PYG{p}{\PYGZob{}}\PYG{l+s+s1}{\PYGZsq{}}\PYG{l+s+s1}{cost}\PYG{l+s+s1}{\PYGZsq{}}\PYG{p}{:}\PYG{p}{\PYGZob{}}\PYG{l+s+s1}{\PYGZsq{}}\PYG{l+s+s1}{repair}\PYG{l+s+s1}{\PYGZsq{}}\PYG{p}{\PYGZcb{}}\PYG{p}{,} \PYG{l+s+s1}{\PYGZsq{}}\PYG{l+s+s1}{delay}\PYG{l+s+s1}{\PYGZsq{}}\PYG{p}{:}\PYG{l+m+mi}{10}\PYG{p}{\PYGZcb{}}\PYG{p}{,} \PYG{n}{modelparams} \PYG{o}{=} \PYG{p}{\PYGZob{}}\PYG{l+s+s1}{\PYGZsq{}}\PYG{l+s+s1}{phases}\PYG{l+s+s1}{\PYGZsq{}}\PYG{p}{:}\PYG{p}{\PYGZob{}}\PYG{l+s+s1}{\PYGZsq{}}\PYG{l+s+s1}{start}\PYG{l+s+s1}{\PYGZsq{}}\PYG{p}{:}\PYG{p}{[}\PYG{l+m+mi}{0}\PYG{p}{,}\PYG{l+m+mi}{5}\PYG{p}{]}\PYG{p}{,} \PYG{l+s+s1}{\PYGZsq{}}\PYG{l+s+s1}{on}\PYG{l+s+s1}{\PYGZsq{}}\PYG{p}{:}\PYG{p}{[}\PYG{l+m+mi}{5}\PYG{p}{,} \PYG{l+m+mi}{50}\PYG{p}{]}\PYG{p}{,} \PYG{l+s+s1}{\PYGZsq{}}\PYG{l+s+s1}{end}\PYG{l+s+s1}{\PYGZsq{}}\PYG{p}{:}\PYG{p}{[}\PYG{l+m+mi}{50}\PYG{p}{,}\PYG{l+m+mi}{500}\PYG{p}{]}\PYG{p}{\PYGZcb{}}\PYG{p}{,} \PYG{l+s+s1}{\PYGZsq{}}\PYG{l+s+s1}{times}\PYG{l+s+s1}{\PYGZsq{}}\PYG{p}{:}\PYG{p}{[}\PYG{l+m+mi}{0}\PYG{p}{,}\PYG{l+m+mi}{20}\PYG{p}{,} \PYG{l+m+mi}{500}\PYG{p}{]}\PYG{p}{,} \PYG{l+s+s1}{\PYGZsq{}}\PYG{l+s+s1}{tstep}\PYG{l+s+s1}{\PYGZsq{}}\PYG{p}{:}\PYG{l+m+mi}{1}\PYG{p}{\PYGZcb{}}\PYG{p}{)}
\PYG{n}{exectimes} \PYG{o}{=} \PYG{n}{compare\PYGZus{}pools}\PYG{p}{(}\PYG{n}{mdl}\PYG{p}{,}\PYG{n}{app}\PYG{p}{,}\PYG{n}{pools}\PYG{p}{,} \PYG{n}{staged}\PYG{o}{=}\PYG{k+kc}{True}\PYG{p}{,} \PYG{n}{track}\PYG{o}{=}\PYG{l+s+s2}{\PYGZdq{}}\PYG{l+s+s2}{all}\PYG{l+s+s2}{\PYGZdq{}}\PYG{p}{,} \PYG{n}{verbose}\PYG{o}{=}\PYG{k+kc}{False}\PYG{p}{)}
\end{sphinxVerbatim}
}

\end{sphinxuseclass}
\end{sphinxuseclass}
\begin{sphinxuseclass}{nbinput}
{
\sphinxsetup{VerbatimColor={named}{nbsphinx-code-bg}}
\sphinxsetup{VerbatimBorderColor={named}{nbsphinx-code-border}}
\begin{sphinxVerbatim}[commandchars=\\\{\}]
\llap{\color{nbsphinxin}[31]:\,\hspace{\fboxrule}\hspace{\fboxsep}}\PYG{n}{fig} \PYG{o}{=} \PYG{n}{plt}\PYG{o}{.}\PYG{n}{figure}\PYG{p}{(}\PYG{n}{figsize}\PYG{o}{=}\PYG{p}{(}\PYG{l+m+mi}{9}\PYG{p}{,} \PYG{l+m+mi}{3}\PYG{p}{)}\PYG{p}{,}\PYG{p}{)}
\PYG{n}{width} \PYG{o}{=} \PYG{l+m+mf}{0.8}
\PYG{n}{plt}\PYG{o}{.}\PYG{n}{bar}\PYG{p}{(}\PYG{n+nb}{range}\PYG{p}{(}\PYG{n+nb}{len}\PYG{p}{(}\PYG{n}{exectimes}\PYG{p}{)}\PYG{p}{)}\PYG{p}{,} \PYG{n+nb}{list}\PYG{p}{(}\PYG{n}{exectimes}\PYG{o}{.}\PYG{n}{values}\PYG{p}{(}\PYG{p}{)}\PYG{p}{)}\PYG{p}{,} \PYG{n}{align}\PYG{o}{=}\PYG{l+s+s1}{\PYGZsq{}}\PYG{l+s+s1}{center}\PYG{l+s+s1}{\PYGZsq{}}\PYG{p}{,} \PYG{n}{color}\PYG{o}{=}\PYG{l+s+s2}{\PYGZdq{}}\PYG{l+s+s2}{blue}\PYG{l+s+s2}{\PYGZdq{}}\PYG{p}{,} \PYG{n}{label}\PYG{o}{=}\PYG{l+s+s2}{\PYGZdq{}}\PYG{l+s+s2}{comparison}\PYG{l+s+s2}{\PYGZdq{}}\PYG{p}{)}
\PYG{n}{plt}\PYG{o}{.}\PYG{n}{bar}\PYG{p}{(}\PYG{n+nb}{range}\PYG{p}{(}\PYG{n+nb}{len}\PYG{p}{(}\PYG{n}{exectimes\PYGZus{}baseline}\PYG{p}{)}\PYG{p}{)}\PYG{p}{,} \PYG{n+nb}{list}\PYG{p}{(}\PYG{n}{exectimes\PYGZus{}baseline}\PYG{o}{.}\PYG{n}{values}\PYG{p}{(}\PYG{p}{)}\PYG{p}{)}\PYG{p}{,} \PYG{n}{align}\PYG{o}{=}\PYG{l+s+s1}{\PYGZsq{}}\PYG{l+s+s1}{center}\PYG{l+s+s1}{\PYGZsq{}}\PYG{p}{,} \PYG{n}{color}\PYG{o}{=}\PYG{l+s+s2}{\PYGZdq{}}\PYG{l+s+s2}{gray}\PYG{l+s+s2}{\PYGZdq{}}\PYG{p}{,} \PYG{n}{alpha}\PYG{o}{=}\PYG{l+m+mf}{0.8}\PYG{p}{,} \PYG{n}{label}\PYG{o}{=}\PYG{l+s+s2}{\PYGZdq{}}\PYG{l+s+s2}{baseline}\PYG{l+s+s2}{\PYGZdq{}}\PYG{p}{)}
\PYG{n}{plt}\PYG{o}{.}\PYG{n}{xticks}\PYG{p}{(}\PYG{n+nb}{range}\PYG{p}{(}\PYG{n+nb}{len}\PYG{p}{(}\PYG{n}{exectimes}\PYG{p}{)}\PYG{p}{)}\PYG{p}{,} \PYG{n+nb}{list}\PYG{p}{(}\PYG{n}{exectimes}\PYG{o}{.}\PYG{n}{keys}\PYG{p}{(}\PYG{p}{)}\PYG{p}{)}\PYG{p}{)}
\PYG{n}{plt}\PYG{o}{.}\PYG{n}{title}\PYG{p}{(}\PYG{l+s+s2}{\PYGZdq{}}\PYG{l+s+s2}{Computational Performance \PYGZhy{} Normal Faults, Staged, Long Simulation, Full Model History}\PYG{l+s+s2}{\PYGZdq{}}\PYG{p}{)}
\PYG{n}{plt}\PYG{o}{.}\PYG{n}{ylabel}\PYG{p}{(}\PYG{l+s+s2}{\PYGZdq{}}\PYG{l+s+s2}{Computational Time (s)}\PYG{l+s+s2}{\PYGZdq{}}\PYG{p}{)}
\PYG{n}{plt}\PYG{o}{.}\PYG{n}{grid}\PYG{p}{(}\PYG{n}{axis}\PYG{o}{=}\PYG{l+s+s1}{\PYGZsq{}}\PYG{l+s+s1}{y}\PYG{l+s+s1}{\PYGZsq{}}\PYG{p}{)}
\PYG{n}{plt}\PYG{o}{.}\PYG{n}{legend}\PYG{p}{(}\PYG{p}{)}
\end{sphinxVerbatim}
}

\end{sphinxuseclass}
\begin{sphinxuseclass}{nboutput}
{

\kern-\sphinxverbatimsmallskipamount\kern-\baselineskip
\kern+\FrameHeightAdjust\kern-\fboxrule
\vspace{\nbsphinxcodecellspacing}

\sphinxsetup{VerbatimColor={named}{white}}
\sphinxsetup{VerbatimBorderColor={named}{nbsphinx-code-border}}
\begin{sphinxuseclass}{output_area}
\begin{sphinxuseclass}{}


\begin{sphinxVerbatim}[commandchars=\\\{\}]
\llap{\color{nbsphinxout}[31]:\,\hspace{\fboxrule}\hspace{\fboxsep}}<matplotlib.legend.Legend at 0x19d34cd40a0>
\end{sphinxVerbatim}



\end{sphinxuseclass}
\end{sphinxuseclass}
}

\end{sphinxuseclass}
\begin{sphinxuseclass}{nboutput}
\begin{sphinxuseclass}{nblast}
\hrule height -\fboxrule\relax
\vspace{\nbsphinxcodecellspacing}

\makeatletter\setbox\nbsphinxpromptbox\box\voidb@x\makeatother

\begin{nbsphinxfancyoutput}

\begin{sphinxuseclass}{output_area}
\begin{sphinxuseclass}{}
\noindent\sphinxincludegraphics[width=575\sphinxpxdimen,height=210\sphinxpxdimen]{{example_pump_Parallelism_Tutorial_46_1}.png}

\end{sphinxuseclass}
\end{sphinxuseclass}
\end{nbsphinxfancyoutput}

\end{sphinxuseclass}
\end{sphinxuseclass}
\sphinxAtStartPar
As shown, the simulation time does increase significantly–about tenfold. In terms of comparative performance, the same general trends hold, except some pools take more of a performance decrease than others. Ostensibly because they require more computational overhead to copy larger data structures back to the main process.


\paragraph{Comparison: Long Simulation No Tracking}
\label{\detokenize{example_pump/Parallelism_Tutorial:Comparison:-Long-Simulation-No-Tracking}}
\sphinxAtStartPar
Finally, it may be interesting to see how performance is affected in long simulations when there is no tracking. This is because in these simulations, there should be very little overhead from creating the respective data structures, even when there is a long simulation. This comparison is shown below.

\begin{sphinxuseclass}{nbinput}
\begin{sphinxuseclass}{nblast}
{
\sphinxsetup{VerbatimColor={named}{nbsphinx-code-bg}}
\sphinxsetup{VerbatimBorderColor={named}{nbsphinx-code-border}}
\begin{sphinxVerbatim}[commandchars=\\\{\}]
\llap{\color{nbsphinxin}[32]:\,\hspace{\fboxrule}\hspace{\fboxsep}}\PYG{n}{app} \PYG{o}{=} \PYG{n}{SampleApproach}\PYG{p}{(}\PYG{n}{mdl}\PYG{p}{,}\PYG{n}{jointfaults}\PYG{o}{=}\PYG{p}{\PYGZob{}}\PYG{l+s+s1}{\PYGZsq{}}\PYG{l+s+s1}{faults}\PYG{l+s+s1}{\PYGZsq{}}\PYG{p}{:}\PYG{l+m+mi}{1}\PYG{p}{\PYGZcb{}}\PYG{p}{,}\PYG{n}{defaultsamp}\PYG{o}{=}\PYG{p}{\PYGZob{}}\PYG{l+s+s1}{\PYGZsq{}}\PYG{l+s+s1}{samp}\PYG{l+s+s1}{\PYGZsq{}}\PYG{p}{:}\PYG{l+s+s1}{\PYGZsq{}}\PYG{l+s+s1}{evenspacing}\PYG{l+s+s1}{\PYGZsq{}}\PYG{p}{,}\PYG{l+s+s1}{\PYGZsq{}}\PYG{l+s+s1}{numpts}\PYG{l+s+s1}{\PYGZsq{}}\PYG{p}{:}\PYG{l+m+mi}{3}\PYG{p}{\PYGZcb{}}\PYG{p}{)}
\PYG{n}{mdl}\PYG{o}{=}\PYG{n}{Pump}\PYG{p}{(}\PYG{n}{params}\PYG{o}{=}\PYG{p}{\PYGZob{}}\PYG{l+s+s1}{\PYGZsq{}}\PYG{l+s+s1}{cost}\PYG{l+s+s1}{\PYGZsq{}}\PYG{p}{:}\PYG{p}{\PYGZob{}}\PYG{l+s+s1}{\PYGZsq{}}\PYG{l+s+s1}{repair}\PYG{l+s+s1}{\PYGZsq{}}\PYG{p}{\PYGZcb{}}\PYG{p}{,} \PYG{l+s+s1}{\PYGZsq{}}\PYG{l+s+s1}{delay}\PYG{l+s+s1}{\PYGZsq{}}\PYG{p}{:}\PYG{l+m+mi}{10}\PYG{p}{\PYGZcb{}}\PYG{p}{,} \PYG{n}{modelparams} \PYG{o}{=} \PYG{p}{\PYGZob{}}\PYG{l+s+s1}{\PYGZsq{}}\PYG{l+s+s1}{phases}\PYG{l+s+s1}{\PYGZsq{}}\PYG{p}{:}\PYG{p}{\PYGZob{}}\PYG{l+s+s1}{\PYGZsq{}}\PYG{l+s+s1}{start}\PYG{l+s+s1}{\PYGZsq{}}\PYG{p}{:}\PYG{p}{[}\PYG{l+m+mi}{0}\PYG{p}{,}\PYG{l+m+mi}{5}\PYG{p}{]}\PYG{p}{,} \PYG{l+s+s1}{\PYGZsq{}}\PYG{l+s+s1}{on}\PYG{l+s+s1}{\PYGZsq{}}\PYG{p}{:}\PYG{p}{[}\PYG{l+m+mi}{5}\PYG{p}{,} \PYG{l+m+mi}{50}\PYG{p}{]}\PYG{p}{,} \PYG{l+s+s1}{\PYGZsq{}}\PYG{l+s+s1}{end}\PYG{l+s+s1}{\PYGZsq{}}\PYG{p}{:}\PYG{p}{[}\PYG{l+m+mi}{50}\PYG{p}{,}\PYG{l+m+mi}{500}\PYG{p}{]}\PYG{p}{\PYGZcb{}}\PYG{p}{,} \PYG{l+s+s1}{\PYGZsq{}}\PYG{l+s+s1}{times}\PYG{l+s+s1}{\PYGZsq{}}\PYG{p}{:}\PYG{p}{[}\PYG{l+m+mi}{0}\PYG{p}{,}\PYG{l+m+mi}{20}\PYG{p}{,} \PYG{l+m+mi}{500}\PYG{p}{]}\PYG{p}{,} \PYG{l+s+s1}{\PYGZsq{}}\PYG{l+s+s1}{tstep}\PYG{l+s+s1}{\PYGZsq{}}\PYG{p}{:}\PYG{l+m+mi}{1}\PYG{p}{\PYGZcb{}}\PYG{p}{)}
\PYG{n}{exectimes} \PYG{o}{=} \PYG{n}{compare\PYGZus{}pools}\PYG{p}{(}\PYG{n}{mdl}\PYG{p}{,}\PYG{n}{app}\PYG{p}{,}\PYG{n}{pools}\PYG{p}{,} \PYG{n}{staged}\PYG{o}{=}\PYG{k+kc}{True}\PYG{p}{,} \PYG{n}{track}\PYG{o}{=}\PYG{l+s+s1}{\PYGZsq{}}\PYG{l+s+s1}{none}\PYG{l+s+s1}{\PYGZsq{}}\PYG{p}{,} \PYG{n}{verbose}\PYG{o}{=}\PYG{k+kc}{False}\PYG{p}{)}
\end{sphinxVerbatim}
}

\end{sphinxuseclass}
\end{sphinxuseclass}
\begin{sphinxuseclass}{nbinput}
{
\sphinxsetup{VerbatimColor={named}{nbsphinx-code-bg}}
\sphinxsetup{VerbatimBorderColor={named}{nbsphinx-code-border}}
\begin{sphinxVerbatim}[commandchars=\\\{\}]
\llap{\color{nbsphinxin}[33]:\,\hspace{\fboxrule}\hspace{\fboxsep}}\PYG{n}{fig} \PYG{o}{=} \PYG{n}{plt}\PYG{o}{.}\PYG{n}{figure}\PYG{p}{(}\PYG{n}{figsize}\PYG{o}{=}\PYG{p}{(}\PYG{l+m+mi}{9}\PYG{p}{,} \PYG{l+m+mi}{3}\PYG{p}{)}\PYG{p}{,}\PYG{p}{)}
\PYG{n}{width} \PYG{o}{=} \PYG{l+m+mf}{0.8}
\PYG{n}{plt}\PYG{o}{.}\PYG{n}{bar}\PYG{p}{(}\PYG{n+nb}{range}\PYG{p}{(}\PYG{n+nb}{len}\PYG{p}{(}\PYG{n}{exectimes}\PYG{p}{)}\PYG{p}{)}\PYG{p}{,} \PYG{n+nb}{list}\PYG{p}{(}\PYG{n}{exectimes}\PYG{o}{.}\PYG{n}{values}\PYG{p}{(}\PYG{p}{)}\PYG{p}{)}\PYG{p}{,} \PYG{n}{align}\PYG{o}{=}\PYG{l+s+s1}{\PYGZsq{}}\PYG{l+s+s1}{center}\PYG{l+s+s1}{\PYGZsq{}}\PYG{p}{,} \PYG{n}{color}\PYG{o}{=}\PYG{l+s+s2}{\PYGZdq{}}\PYG{l+s+s2}{blue}\PYG{l+s+s2}{\PYGZdq{}}\PYG{p}{,} \PYG{n}{label}\PYG{o}{=}\PYG{l+s+s2}{\PYGZdq{}}\PYG{l+s+s2}{comparison}\PYG{l+s+s2}{\PYGZdq{}}\PYG{p}{)}
\PYG{n}{plt}\PYG{o}{.}\PYG{n}{bar}\PYG{p}{(}\PYG{n+nb}{range}\PYG{p}{(}\PYG{n+nb}{len}\PYG{p}{(}\PYG{n}{exectimes\PYGZus{}baseline}\PYG{p}{)}\PYG{p}{)}\PYG{p}{,} \PYG{n+nb}{list}\PYG{p}{(}\PYG{n}{exectimes\PYGZus{}baseline}\PYG{o}{.}\PYG{n}{values}\PYG{p}{(}\PYG{p}{)}\PYG{p}{)}\PYG{p}{,} \PYG{n}{align}\PYG{o}{=}\PYG{l+s+s1}{\PYGZsq{}}\PYG{l+s+s1}{center}\PYG{l+s+s1}{\PYGZsq{}}\PYG{p}{,} \PYG{n}{color}\PYG{o}{=}\PYG{l+s+s2}{\PYGZdq{}}\PYG{l+s+s2}{gray}\PYG{l+s+s2}{\PYGZdq{}}\PYG{p}{,} \PYG{n}{alpha}\PYG{o}{=}\PYG{l+m+mf}{0.8}\PYG{p}{,} \PYG{n}{label}\PYG{o}{=}\PYG{l+s+s2}{\PYGZdq{}}\PYG{l+s+s2}{baseline}\PYG{l+s+s2}{\PYGZdq{}}\PYG{p}{)}
\PYG{n}{plt}\PYG{o}{.}\PYG{n}{xticks}\PYG{p}{(}\PYG{n+nb}{range}\PYG{p}{(}\PYG{n+nb}{len}\PYG{p}{(}\PYG{n}{exectimes}\PYG{p}{)}\PYG{p}{)}\PYG{p}{,} \PYG{n+nb}{list}\PYG{p}{(}\PYG{n}{exectimes}\PYG{o}{.}\PYG{n}{keys}\PYG{p}{(}\PYG{p}{)}\PYG{p}{)}\PYG{p}{)}
\PYG{n}{plt}\PYG{o}{.}\PYG{n}{title}\PYG{p}{(}\PYG{l+s+s2}{\PYGZdq{}}\PYG{l+s+s2}{Computational Performance \PYGZhy{} Normal Faults, Staged, Long Simulation, No Model History}\PYG{l+s+s2}{\PYGZdq{}}\PYG{p}{)}
\PYG{n}{plt}\PYG{o}{.}\PYG{n}{ylabel}\PYG{p}{(}\PYG{l+s+s2}{\PYGZdq{}}\PYG{l+s+s2}{Computational Time (s)}\PYG{l+s+s2}{\PYGZdq{}}\PYG{p}{)}
\PYG{n}{plt}\PYG{o}{.}\PYG{n}{grid}\PYG{p}{(}\PYG{n}{axis}\PYG{o}{=}\PYG{l+s+s1}{\PYGZsq{}}\PYG{l+s+s1}{y}\PYG{l+s+s1}{\PYGZsq{}}\PYG{p}{)}
\PYG{n}{plt}\PYG{o}{.}\PYG{n}{legend}\PYG{p}{(}\PYG{p}{)}
\end{sphinxVerbatim}
}

\end{sphinxuseclass}
\begin{sphinxuseclass}{nboutput}
{

\kern-\sphinxverbatimsmallskipamount\kern-\baselineskip
\kern+\FrameHeightAdjust\kern-\fboxrule
\vspace{\nbsphinxcodecellspacing}

\sphinxsetup{VerbatimColor={named}{white}}
\sphinxsetup{VerbatimBorderColor={named}{nbsphinx-code-border}}
\begin{sphinxuseclass}{output_area}
\begin{sphinxuseclass}{}


\begin{sphinxVerbatim}[commandchars=\\\{\}]
\llap{\color{nbsphinxout}[33]:\,\hspace{\fboxrule}\hspace{\fboxsep}}<matplotlib.legend.Legend at 0x19d34d0deb0>
\end{sphinxVerbatim}



\end{sphinxuseclass}
\end{sphinxuseclass}
}

\end{sphinxuseclass}
\begin{sphinxuseclass}{nboutput}
\begin{sphinxuseclass}{nblast}
\hrule height -\fboxrule\relax
\vspace{\nbsphinxcodecellspacing}

\makeatletter\setbox\nbsphinxpromptbox\box\voidb@x\makeatother

\begin{nbsphinxfancyoutput}

\begin{sphinxuseclass}{output_area}
\begin{sphinxuseclass}{}
\noindent\sphinxincludegraphics[width=566\sphinxpxdimen,height=210\sphinxpxdimen]{{example_pump_Parallelism_Tutorial_50_1}.png}

\end{sphinxuseclass}
\end{sphinxuseclass}
\end{nbsphinxfancyoutput}

\end{sphinxuseclass}
\end{sphinxuseclass}
\sphinxAtStartPar
As shown, with a long simulation, the case for pools (other than multiprocessing–ProcessPool, ParallelPool, multiprocess) increases somewhat.


\paragraph{Comparison: Long Simulation Only Necessary Tracking}
\label{\detokenize{example_pump/Parallelism_Tutorial:Comparison:-Long-Simulation-Only-Necessary-Tracking}}
\sphinxAtStartPar
In practice, it can be necessary to track some states over time. Here we perform the same comparison using the ‘valstates’ option, which only tracks states which have been defined in the model to be necessary to track (using ‘valparams’)

\begin{sphinxuseclass}{nbinput}
\begin{sphinxuseclass}{nblast}
{
\sphinxsetup{VerbatimColor={named}{nbsphinx-code-bg}}
\sphinxsetup{VerbatimBorderColor={named}{nbsphinx-code-border}}
\begin{sphinxVerbatim}[commandchars=\\\{\}]
\llap{\color{nbsphinxin}[34]:\,\hspace{\fboxrule}\hspace{\fboxsep}}\PYG{n}{app} \PYG{o}{=} \PYG{n}{SampleApproach}\PYG{p}{(}\PYG{n}{mdl}\PYG{p}{,}\PYG{n}{jointfaults}\PYG{o}{=}\PYG{p}{\PYGZob{}}\PYG{l+s+s1}{\PYGZsq{}}\PYG{l+s+s1}{faults}\PYG{l+s+s1}{\PYGZsq{}}\PYG{p}{:}\PYG{l+m+mi}{1}\PYG{p}{\PYGZcb{}}\PYG{p}{,}\PYG{n}{defaultsamp}\PYG{o}{=}\PYG{p}{\PYGZob{}}\PYG{l+s+s1}{\PYGZsq{}}\PYG{l+s+s1}{samp}\PYG{l+s+s1}{\PYGZsq{}}\PYG{p}{:}\PYG{l+s+s1}{\PYGZsq{}}\PYG{l+s+s1}{evenspacing}\PYG{l+s+s1}{\PYGZsq{}}\PYG{p}{,}\PYG{l+s+s1}{\PYGZsq{}}\PYG{l+s+s1}{numpts}\PYG{l+s+s1}{\PYGZsq{}}\PYG{p}{:}\PYG{l+m+mi}{3}\PYG{p}{\PYGZcb{}}\PYG{p}{)}
\PYG{n}{mdl}\PYG{o}{=}\PYG{n}{Pump}\PYG{p}{(}\PYG{n}{params}\PYG{o}{=}\PYG{p}{\PYGZob{}}\PYG{l+s+s1}{\PYGZsq{}}\PYG{l+s+s1}{cost}\PYG{l+s+s1}{\PYGZsq{}}\PYG{p}{:}\PYG{p}{\PYGZob{}}\PYG{l+s+s1}{\PYGZsq{}}\PYG{l+s+s1}{repair}\PYG{l+s+s1}{\PYGZsq{}}\PYG{p}{\PYGZcb{}}\PYG{p}{,} \PYG{l+s+s1}{\PYGZsq{}}\PYG{l+s+s1}{delay}\PYG{l+s+s1}{\PYGZsq{}}\PYG{p}{:}\PYG{l+m+mi}{10}\PYG{p}{\PYGZcb{}}\PYG{p}{,} \PYG{n}{modelparams} \PYG{o}{=} \PYG{p}{\PYGZob{}}\PYG{l+s+s1}{\PYGZsq{}}\PYG{l+s+s1}{phases}\PYG{l+s+s1}{\PYGZsq{}}\PYG{p}{:}\PYG{p}{\PYGZob{}}\PYG{l+s+s1}{\PYGZsq{}}\PYG{l+s+s1}{start}\PYG{l+s+s1}{\PYGZsq{}}\PYG{p}{:}\PYG{p}{[}\PYG{l+m+mi}{0}\PYG{p}{,}\PYG{l+m+mi}{5}\PYG{p}{]}\PYG{p}{,} \PYG{l+s+s1}{\PYGZsq{}}\PYG{l+s+s1}{on}\PYG{l+s+s1}{\PYGZsq{}}\PYG{p}{:}\PYG{p}{[}\PYG{l+m+mi}{5}\PYG{p}{,} \PYG{l+m+mi}{50}\PYG{p}{]}\PYG{p}{,} \PYG{l+s+s1}{\PYGZsq{}}\PYG{l+s+s1}{end}\PYG{l+s+s1}{\PYGZsq{}}\PYG{p}{:}\PYG{p}{[}\PYG{l+m+mi}{50}\PYG{p}{,}\PYG{l+m+mi}{500}\PYG{p}{]}\PYG{p}{\PYGZcb{}}\PYG{p}{,} \PYG{l+s+s1}{\PYGZsq{}}\PYG{l+s+s1}{times}\PYG{l+s+s1}{\PYGZsq{}}\PYG{p}{:}\PYG{p}{[}\PYG{l+m+mi}{0}\PYG{p}{,}\PYG{l+m+mi}{20}\PYG{p}{,} \PYG{l+m+mi}{500}\PYG{p}{]}\PYG{p}{,} \PYG{l+s+s1}{\PYGZsq{}}\PYG{l+s+s1}{tstep}\PYG{l+s+s1}{\PYGZsq{}}\PYG{p}{:}\PYG{l+m+mi}{1}\PYG{p}{\PYGZcb{}}\PYG{p}{)}
\PYG{n}{exectimes} \PYG{o}{=} \PYG{n}{compare\PYGZus{}pools}\PYG{p}{(}\PYG{n}{mdl}\PYG{p}{,}\PYG{n}{app}\PYG{p}{,}\PYG{n}{pools}\PYG{p}{,} \PYG{n}{staged}\PYG{o}{=}\PYG{k+kc}{True}\PYG{p}{,} \PYG{n}{track}\PYG{o}{=}\PYG{l+s+s1}{\PYGZsq{}}\PYG{l+s+s1}{valparams}\PYG{l+s+s1}{\PYGZsq{}}\PYG{p}{,} \PYG{n}{verbose}\PYG{o}{=}\PYG{k+kc}{False}\PYG{p}{)}
\end{sphinxVerbatim}
}

\end{sphinxuseclass}
\end{sphinxuseclass}
\begin{sphinxuseclass}{nbinput}
{
\sphinxsetup{VerbatimColor={named}{nbsphinx-code-bg}}
\sphinxsetup{VerbatimBorderColor={named}{nbsphinx-code-border}}
\begin{sphinxVerbatim}[commandchars=\\\{\}]
\llap{\color{nbsphinxin}[35]:\,\hspace{\fboxrule}\hspace{\fboxsep}}\PYG{n}{fig} \PYG{o}{=} \PYG{n}{plt}\PYG{o}{.}\PYG{n}{figure}\PYG{p}{(}\PYG{n}{figsize}\PYG{o}{=}\PYG{p}{(}\PYG{l+m+mi}{9}\PYG{p}{,} \PYG{l+m+mi}{3}\PYG{p}{)}\PYG{p}{,}\PYG{p}{)}
\PYG{n}{width} \PYG{o}{=} \PYG{l+m+mf}{0.8}
\PYG{n}{plt}\PYG{o}{.}\PYG{n}{bar}\PYG{p}{(}\PYG{n+nb}{range}\PYG{p}{(}\PYG{n+nb}{len}\PYG{p}{(}\PYG{n}{exectimes}\PYG{p}{)}\PYG{p}{)}\PYG{p}{,} \PYG{n+nb}{list}\PYG{p}{(}\PYG{n}{exectimes}\PYG{o}{.}\PYG{n}{values}\PYG{p}{(}\PYG{p}{)}\PYG{p}{)}\PYG{p}{,} \PYG{n}{align}\PYG{o}{=}\PYG{l+s+s1}{\PYGZsq{}}\PYG{l+s+s1}{center}\PYG{l+s+s1}{\PYGZsq{}}\PYG{p}{,} \PYG{n}{color}\PYG{o}{=}\PYG{l+s+s2}{\PYGZdq{}}\PYG{l+s+s2}{blue}\PYG{l+s+s2}{\PYGZdq{}}\PYG{p}{,} \PYG{n}{label}\PYG{o}{=}\PYG{l+s+s2}{\PYGZdq{}}\PYG{l+s+s2}{comparison}\PYG{l+s+s2}{\PYGZdq{}}\PYG{p}{)}
\PYG{n}{plt}\PYG{o}{.}\PYG{n}{bar}\PYG{p}{(}\PYG{n+nb}{range}\PYG{p}{(}\PYG{n+nb}{len}\PYG{p}{(}\PYG{n}{exectimes\PYGZus{}baseline}\PYG{p}{)}\PYG{p}{)}\PYG{p}{,} \PYG{n+nb}{list}\PYG{p}{(}\PYG{n}{exectimes\PYGZus{}baseline}\PYG{o}{.}\PYG{n}{values}\PYG{p}{(}\PYG{p}{)}\PYG{p}{)}\PYG{p}{,} \PYG{n}{align}\PYG{o}{=}\PYG{l+s+s1}{\PYGZsq{}}\PYG{l+s+s1}{center}\PYG{l+s+s1}{\PYGZsq{}}\PYG{p}{,} \PYG{n}{color}\PYG{o}{=}\PYG{l+s+s2}{\PYGZdq{}}\PYG{l+s+s2}{gray}\PYG{l+s+s2}{\PYGZdq{}}\PYG{p}{,} \PYG{n}{alpha}\PYG{o}{=}\PYG{l+m+mf}{0.8}\PYG{p}{,} \PYG{n}{label}\PYG{o}{=}\PYG{l+s+s2}{\PYGZdq{}}\PYG{l+s+s2}{baseline}\PYG{l+s+s2}{\PYGZdq{}}\PYG{p}{)}
\PYG{n}{plt}\PYG{o}{.}\PYG{n}{xticks}\PYG{p}{(}\PYG{n+nb}{range}\PYG{p}{(}\PYG{n+nb}{len}\PYG{p}{(}\PYG{n}{exectimes}\PYG{p}{)}\PYG{p}{)}\PYG{p}{,} \PYG{n+nb}{list}\PYG{p}{(}\PYG{n}{exectimes}\PYG{o}{.}\PYG{n}{keys}\PYG{p}{(}\PYG{p}{)}\PYG{p}{)}\PYG{p}{)}
\PYG{n}{plt}\PYG{o}{.}\PYG{n}{title}\PYG{p}{(}\PYG{l+s+s2}{\PYGZdq{}}\PYG{l+s+s2}{Computational Performance \PYGZhy{} Normal Faults, Staged, Long Simulation, Only Necessary History}\PYG{l+s+s2}{\PYGZdq{}}\PYG{p}{)}
\PYG{n}{plt}\PYG{o}{.}\PYG{n}{ylabel}\PYG{p}{(}\PYG{l+s+s2}{\PYGZdq{}}\PYG{l+s+s2}{Computational Time (s)}\PYG{l+s+s2}{\PYGZdq{}}\PYG{p}{)}
\PYG{n}{plt}\PYG{o}{.}\PYG{n}{grid}\PYG{p}{(}\PYG{n}{axis}\PYG{o}{=}\PYG{l+s+s1}{\PYGZsq{}}\PYG{l+s+s1}{y}\PYG{l+s+s1}{\PYGZsq{}}\PYG{p}{)}
\PYG{n}{plt}\PYG{o}{.}\PYG{n}{legend}\PYG{p}{(}\PYG{p}{)}
\end{sphinxVerbatim}
}

\end{sphinxuseclass}
\begin{sphinxuseclass}{nboutput}
{

\kern-\sphinxverbatimsmallskipamount\kern-\baselineskip
\kern+\FrameHeightAdjust\kern-\fboxrule
\vspace{\nbsphinxcodecellspacing}

\sphinxsetup{VerbatimColor={named}{white}}
\sphinxsetup{VerbatimBorderColor={named}{nbsphinx-code-border}}
\begin{sphinxuseclass}{output_area}
\begin{sphinxuseclass}{}


\begin{sphinxVerbatim}[commandchars=\\\{\}]
\llap{\color{nbsphinxout}[35]:\,\hspace{\fboxrule}\hspace{\fboxsep}}<matplotlib.legend.Legend at 0x19d35fbc340>
\end{sphinxVerbatim}



\end{sphinxuseclass}
\end{sphinxuseclass}
}

\end{sphinxuseclass}
\begin{sphinxuseclass}{nboutput}
\begin{sphinxuseclass}{nblast}
\hrule height -\fboxrule\relax
\vspace{\nbsphinxcodecellspacing}

\makeatletter\setbox\nbsphinxpromptbox\box\voidb@x\makeatother

\begin{nbsphinxfancyoutput}

\begin{sphinxuseclass}{output_area}
\begin{sphinxuseclass}{}
\noindent\sphinxincludegraphics[width=585\sphinxpxdimen,height=210\sphinxpxdimen]{{example_pump_Parallelism_Tutorial_54_1}.png}

\end{sphinxuseclass}
\end{sphinxuseclass}
\end{nbsphinxfancyoutput}

\end{sphinxuseclass}
\end{sphinxuseclass}
\sphinxAtStartPar
As shown, even though this model has a large increase in simulation time, the this time can be reduced by not returning the full history.

\sphinxAtStartPar
Thus, the major computational performance limitation in this model is not necessarily the model, but the generation, update, and passing of the history itself.


\subsubsection{Comparison: Lower Tracking Time Resolution}
\label{\detokenize{example_pump/Parallelism_Tutorial:Comparison:-Lower-Tracking-Time-Resolution}}
\sphinxAtStartPar
Finally, the number of recorded timesteps can be lowered to lower computational costs while still returning all relevant variables.

\begin{sphinxuseclass}{nbinput}
\begin{sphinxuseclass}{nblast}
{
\sphinxsetup{VerbatimColor={named}{nbsphinx-code-bg}}
\sphinxsetup{VerbatimBorderColor={named}{nbsphinx-code-border}}
\begin{sphinxVerbatim}[commandchars=\\\{\}]
\llap{\color{nbsphinxin}[37]:\,\hspace{\fboxrule}\hspace{\fboxsep}}\PYG{n}{app} \PYG{o}{=} \PYG{n}{SampleApproach}\PYG{p}{(}\PYG{n}{mdl}\PYG{p}{,}\PYG{n}{jointfaults}\PYG{o}{=}\PYG{p}{\PYGZob{}}\PYG{l+s+s1}{\PYGZsq{}}\PYG{l+s+s1}{faults}\PYG{l+s+s1}{\PYGZsq{}}\PYG{p}{:}\PYG{l+m+mi}{1}\PYG{p}{\PYGZcb{}}\PYG{p}{,}\PYG{n}{defaultsamp}\PYG{o}{=}\PYG{p}{\PYGZob{}}\PYG{l+s+s1}{\PYGZsq{}}\PYG{l+s+s1}{samp}\PYG{l+s+s1}{\PYGZsq{}}\PYG{p}{:}\PYG{l+s+s1}{\PYGZsq{}}\PYG{l+s+s1}{evenspacing}\PYG{l+s+s1}{\PYGZsq{}}\PYG{p}{,}\PYG{l+s+s1}{\PYGZsq{}}\PYG{l+s+s1}{numpts}\PYG{l+s+s1}{\PYGZsq{}}\PYG{p}{:}\PYG{l+m+mi}{3}\PYG{p}{\PYGZcb{}}\PYG{p}{)}
\PYG{n}{mdl}\PYG{o}{=}\PYG{n}{Pump}\PYG{p}{(}\PYG{n}{params}\PYG{o}{=}\PYG{p}{\PYGZob{}}\PYG{l+s+s1}{\PYGZsq{}}\PYG{l+s+s1}{cost}\PYG{l+s+s1}{\PYGZsq{}}\PYG{p}{:}\PYG{p}{\PYGZob{}}\PYG{l+s+s1}{\PYGZsq{}}\PYG{l+s+s1}{repair}\PYG{l+s+s1}{\PYGZsq{}}\PYG{p}{\PYGZcb{}}\PYG{p}{,} \PYG{l+s+s1}{\PYGZsq{}}\PYG{l+s+s1}{delay}\PYG{l+s+s1}{\PYGZsq{}}\PYG{p}{:}\PYG{l+m+mi}{10}\PYG{p}{\PYGZcb{}}\PYG{p}{,} \PYG{n}{modelparams} \PYG{o}{=} \PYG{p}{\PYGZob{}}\PYG{l+s+s1}{\PYGZsq{}}\PYG{l+s+s1}{phases}\PYG{l+s+s1}{\PYGZsq{}}\PYG{p}{:}\PYG{p}{\PYGZob{}}\PYG{l+s+s1}{\PYGZsq{}}\PYG{l+s+s1}{start}\PYG{l+s+s1}{\PYGZsq{}}\PYG{p}{:}\PYG{p}{[}\PYG{l+m+mi}{0}\PYG{p}{,}\PYG{l+m+mi}{5}\PYG{p}{]}\PYG{p}{,} \PYG{l+s+s1}{\PYGZsq{}}\PYG{l+s+s1}{on}\PYG{l+s+s1}{\PYGZsq{}}\PYG{p}{:}\PYG{p}{[}\PYG{l+m+mi}{5}\PYG{p}{,} \PYG{l+m+mi}{50}\PYG{p}{]}\PYG{p}{,} \PYG{l+s+s1}{\PYGZsq{}}\PYG{l+s+s1}{end}\PYG{l+s+s1}{\PYGZsq{}}\PYG{p}{:}\PYG{p}{[}\PYG{l+m+mi}{50}\PYG{p}{,}\PYG{l+m+mi}{500}\PYG{p}{]}\PYG{p}{\PYGZcb{}}\PYG{p}{,} \PYG{l+s+s1}{\PYGZsq{}}\PYG{l+s+s1}{times}\PYG{l+s+s1}{\PYGZsq{}}\PYG{p}{:}\PYG{p}{[}\PYG{l+m+mi}{0}\PYG{p}{,}\PYG{l+m+mi}{20}\PYG{p}{,} \PYG{l+m+mi}{55}\PYG{p}{]}\PYG{p}{,} \PYG{l+s+s1}{\PYGZsq{}}\PYG{l+s+s1}{tstep}\PYG{l+s+s1}{\PYGZsq{}}\PYG{p}{:}\PYG{l+m+mi}{1}\PYG{p}{\PYGZcb{}}\PYG{p}{)}
\PYG{n}{exectimes} \PYG{o}{=} \PYG{n}{compare\PYGZus{}pools}\PYG{p}{(}\PYG{n}{mdl}\PYG{p}{,}\PYG{n}{app}\PYG{p}{,}\PYG{n}{pools}\PYG{p}{,} \PYG{n}{staged}\PYG{o}{=}\PYG{k+kc}{True}\PYG{p}{,} \PYG{n}{track}\PYG{o}{=}\PYG{l+s+s1}{\PYGZsq{}}\PYG{l+s+s1}{all}\PYG{l+s+s1}{\PYGZsq{}}\PYG{p}{,} \PYG{n}{verbose}\PYG{o}{=}\PYG{k+kc}{False}\PYG{p}{,} \PYG{n}{track\PYGZus{}times}\PYG{o}{=}\PYG{p}{(}\PYG{l+s+s2}{\PYGZdq{}}\PYG{l+s+s2}{interval}\PYG{l+s+s2}{\PYGZdq{}}\PYG{p}{,} \PYG{l+m+mi}{5}\PYG{p}{)}\PYG{p}{)}
\end{sphinxVerbatim}
}

\end{sphinxuseclass}
\end{sphinxuseclass}
\begin{sphinxuseclass}{nbinput}
{
\sphinxsetup{VerbatimColor={named}{nbsphinx-code-bg}}
\sphinxsetup{VerbatimBorderColor={named}{nbsphinx-code-border}}
\begin{sphinxVerbatim}[commandchars=\\\{\}]
\llap{\color{nbsphinxin}[38]:\,\hspace{\fboxrule}\hspace{\fboxsep}}\PYG{n}{fig} \PYG{o}{=} \PYG{n}{plt}\PYG{o}{.}\PYG{n}{figure}\PYG{p}{(}\PYG{n}{figsize}\PYG{o}{=}\PYG{p}{(}\PYG{l+m+mi}{9}\PYG{p}{,} \PYG{l+m+mi}{3}\PYG{p}{)}\PYG{p}{,}\PYG{p}{)}
\PYG{n}{width} \PYG{o}{=} \PYG{l+m+mf}{0.8}
\PYG{n}{plt}\PYG{o}{.}\PYG{n}{bar}\PYG{p}{(}\PYG{n+nb}{range}\PYG{p}{(}\PYG{n+nb}{len}\PYG{p}{(}\PYG{n}{exectimes}\PYG{p}{)}\PYG{p}{)}\PYG{p}{,} \PYG{n+nb}{list}\PYG{p}{(}\PYG{n}{exectimes}\PYG{o}{.}\PYG{n}{values}\PYG{p}{(}\PYG{p}{)}\PYG{p}{)}\PYG{p}{,} \PYG{n}{align}\PYG{o}{=}\PYG{l+s+s1}{\PYGZsq{}}\PYG{l+s+s1}{center}\PYG{l+s+s1}{\PYGZsq{}}\PYG{p}{,} \PYG{n}{color}\PYG{o}{=}\PYG{l+s+s2}{\PYGZdq{}}\PYG{l+s+s2}{blue}\PYG{l+s+s2}{\PYGZdq{}}\PYG{p}{,} \PYG{n}{label}\PYG{o}{=}\PYG{l+s+s2}{\PYGZdq{}}\PYG{l+s+s2}{comparison}\PYG{l+s+s2}{\PYGZdq{}}\PYG{p}{)}
\PYG{n}{plt}\PYG{o}{.}\PYG{n}{bar}\PYG{p}{(}\PYG{n+nb}{range}\PYG{p}{(}\PYG{n+nb}{len}\PYG{p}{(}\PYG{n}{exectimes\PYGZus{}baseline}\PYG{p}{)}\PYG{p}{)}\PYG{p}{,} \PYG{n+nb}{list}\PYG{p}{(}\PYG{n}{exectimes\PYGZus{}baseline}\PYG{o}{.}\PYG{n}{values}\PYG{p}{(}\PYG{p}{)}\PYG{p}{)}\PYG{p}{,} \PYG{n}{align}\PYG{o}{=}\PYG{l+s+s1}{\PYGZsq{}}\PYG{l+s+s1}{center}\PYG{l+s+s1}{\PYGZsq{}}\PYG{p}{,} \PYG{n}{color}\PYG{o}{=}\PYG{l+s+s2}{\PYGZdq{}}\PYG{l+s+s2}{gray}\PYG{l+s+s2}{\PYGZdq{}}\PYG{p}{,} \PYG{n}{alpha}\PYG{o}{=}\PYG{l+m+mf}{0.5}\PYG{p}{,} \PYG{n}{label}\PYG{o}{=}\PYG{l+s+s2}{\PYGZdq{}}\PYG{l+s+s2}{baseline}\PYG{l+s+s2}{\PYGZdq{}}\PYG{p}{)}
\PYG{n}{plt}\PYG{o}{.}\PYG{n}{xticks}\PYG{p}{(}\PYG{n+nb}{range}\PYG{p}{(}\PYG{n+nb}{len}\PYG{p}{(}\PYG{n}{exectimes}\PYG{p}{)}\PYG{p}{)}\PYG{p}{,} \PYG{n+nb}{list}\PYG{p}{(}\PYG{n}{exectimes}\PYG{o}{.}\PYG{n}{keys}\PYG{p}{(}\PYG{p}{)}\PYG{p}{)}\PYG{p}{)}
\PYG{n}{plt}\PYG{o}{.}\PYG{n}{title}\PYG{p}{(}\PYG{l+s+s2}{\PYGZdq{}}\PYG{l+s+s2}{Computational Performance \PYGZhy{} Many Faults, Staged, Normal Simulation, No Model History}\PYG{l+s+s2}{\PYGZdq{}}\PYG{p}{)}
\PYG{n}{plt}\PYG{o}{.}\PYG{n}{ylabel}\PYG{p}{(}\PYG{l+s+s2}{\PYGZdq{}}\PYG{l+s+s2}{Computational Time (s)}\PYG{l+s+s2}{\PYGZdq{}}\PYG{p}{)}
\PYG{n}{plt}\PYG{o}{.}\PYG{n}{grid}\PYG{p}{(}\PYG{n}{axis}\PYG{o}{=}\PYG{l+s+s1}{\PYGZsq{}}\PYG{l+s+s1}{y}\PYG{l+s+s1}{\PYGZsq{}}\PYG{p}{)}
\PYG{n}{plt}\PYG{o}{.}\PYG{n}{legend}\PYG{p}{(}\PYG{p}{)}
\end{sphinxVerbatim}
}

\end{sphinxuseclass}
\begin{sphinxuseclass}{nboutput}
{

\kern-\sphinxverbatimsmallskipamount\kern-\baselineskip
\kern+\FrameHeightAdjust\kern-\fboxrule
\vspace{\nbsphinxcodecellspacing}

\sphinxsetup{VerbatimColor={named}{white}}
\sphinxsetup{VerbatimBorderColor={named}{nbsphinx-code-border}}
\begin{sphinxuseclass}{output_area}
\begin{sphinxuseclass}{}


\begin{sphinxVerbatim}[commandchars=\\\{\}]
\llap{\color{nbsphinxout}[38]:\,\hspace{\fboxrule}\hspace{\fboxsep}}<matplotlib.legend.Legend at 0x19d35fed670>
\end{sphinxVerbatim}



\end{sphinxuseclass}
\end{sphinxuseclass}
}

\end{sphinxuseclass}
\begin{sphinxuseclass}{nboutput}
\begin{sphinxuseclass}{nblast}
\hrule height -\fboxrule\relax
\vspace{\nbsphinxcodecellspacing}

\makeatletter\setbox\nbsphinxpromptbox\box\voidb@x\makeatother

\begin{nbsphinxfancyoutput}

\begin{sphinxuseclass}{output_area}
\begin{sphinxuseclass}{}
\noindent\sphinxincludegraphics[width=574\sphinxpxdimen,height=210\sphinxpxdimen]{{example_pump_Parallelism_Tutorial_58_1}.png}

\end{sphinxuseclass}
\end{sphinxuseclass}
\end{nbsphinxfancyoutput}

\end{sphinxuseclass}
\end{sphinxuseclass}
\sphinxAtStartPar
As shown, while lowering time resolution could theoretically lower computational time, it does not significantly change much in this example.


\subsubsection{Comparison Conclusions:}
\label{\detokenize{example_pump/Parallelism_Tutorial:Comparison-Conclusions:}}
\sphinxAtStartPar
Parallelism can the improve computational performance of a given resilience simulation approach. However, this improvement is dependent on the parameters of the simulation. Generally, the official python \sphinxcode{\sphinxupquote{multiprocessing}} module seems to be the only package that consistently gives a performance improvement over a single\sphinxhyphen{}process execution, although this can change depending on the underlying model and modelling approach. There are additionally reasons you might choose other pools–\sphinxcode{\sphinxupquote{Pathos}} and
\sphinxcode{\sphinxupquote{multiprocess}} pools may enable more data structures in the model because they extend what can be communicated in and out of threads.

\sphinxAtStartPar
In general, one of the major considerations for optimization compuational time is not just the \sphinxstyleemphasis{simulation of the model}, but the \sphinxstyleemphasis{size of the returned data structures}. Minimizing the size of the returned data structures can reduce computational time both by reducing the time of an individual simulation and by reducing the \sphinxstyleemphasis{parallelism overhead} from copying these data structures in and out of parallel threads. However, it is important to recognize that for resilience assessment, one often
needs a history of model states (or, at least, states of interest) to properly quantify the dynamic costs (i.e., \(\int C_f(t) dt\)). Indeed, in this model, only repair costs were able to be used in the comparison of non\sphinxhyphen{}tracked states, because the other dynamic costs required a history of their corresponding flows. Changing the number and size of tracked model states can influence the computational time, but only to a point–while one would expect lowering time\sphinxhyphen{}fidelity to have a significant
effect, it does not because the overhead is less to do with filling the underlying data structures as it has to do with instantiating and returning them–a far more effective method is to only return the functions/flows which are needed by the model.


\subsubsection{Further Computational Cost Reduction via Profiling}
\label{\detokenize{example_pump/Parallelism_Tutorial:Further-Computational-Cost-Reduction-via-Profiling}}
\sphinxAtStartPar
While parallelism and staged execution are helpful and relatively easy\sphinxhyphen{}to\sphinxhyphen{}implement methods of computational cost reduction, it can be helpful (especially for more complex models) to see what aspects of the model are taking the most computational time.

\sphinxAtStartPar
Python’s builtin \sphinxcode{\sphinxupquote{cProfile}} package can ge used to see the relative computational times of different functions/processes.

\begin{sphinxuseclass}{nbinput}
\begin{sphinxuseclass}{nblast}
{
\sphinxsetup{VerbatimColor={named}{nbsphinx-code-bg}}
\sphinxsetup{VerbatimBorderColor={named}{nbsphinx-code-border}}
\begin{sphinxVerbatim}[commandchars=\\\{\}]
\llap{\color{nbsphinxin}[39]:\,\hspace{\fboxrule}\hspace{\fboxsep}}\PYG{k+kn}{import} \PYG{n+nn}{cProfile}
\end{sphinxVerbatim}
}

\end{sphinxuseclass}
\end{sphinxuseclass}
\begin{sphinxuseclass}{nbinput}
{
\sphinxsetup{VerbatimColor={named}{nbsphinx-code-bg}}
\sphinxsetup{VerbatimBorderColor={named}{nbsphinx-code-border}}
\begin{sphinxVerbatim}[commandchars=\\\{\}]
\llap{\color{nbsphinxin}[40]:\,\hspace{\fboxrule}\hspace{\fboxsep}}\PYG{n}{cProfile}\PYG{o}{.}\PYG{n}{run}\PYG{p}{(}\PYG{l+s+s1}{\PYGZsq{}}\PYG{l+s+s1}{propagate.nominal(mdl)}\PYG{l+s+s1}{\PYGZsq{}}\PYG{p}{)}
\end{sphinxVerbatim}
}

\end{sphinxuseclass}
\begin{sphinxuseclass}{nboutput}
\begin{sphinxuseclass}{nblast}
{

\kern-\sphinxverbatimsmallskipamount\kern-\baselineskip
\kern+\FrameHeightAdjust\kern-\fboxrule
\vspace{\nbsphinxcodecellspacing}

\sphinxsetup{VerbatimColor={named}{white}}
\sphinxsetup{VerbatimBorderColor={named}{nbsphinx-code-border}}
\begin{sphinxuseclass}{output_area}
\begin{sphinxuseclass}{}


\begin{sphinxVerbatim}[commandchars=\\\{\}]
         17261 function calls in 0.011 seconds

   Ordered by: standard name

   ncalls  tottime  percall  cumtime  percall filename:lineno(function)
        1    0.000    0.000    0.000    0.000 <\_\_array\_function\_\_ internals>:2(amin)
      672    0.000    0.000    0.001    0.000 <\_\_array\_function\_\_ internals>:2(can\_cast)
       12    0.000    0.000    0.000    0.000 <\_\_array\_function\_\_ internals>:2(concatenate)
       12    0.000    0.000    0.000    0.000 <\_\_array\_function\_\_ internals>:2(copyto)
        1    0.000    0.000    0.011    0.011 <string>:1(<module>)
       12    0.000    0.000    0.000    0.000 \_asarray.py:23(asarray)
      114    0.000    0.000    0.001    0.000 \_collections\_abc.py:706(\_\_ior\_\_)
       48    0.000    0.000    0.000    0.000 \_collections\_abc.py:779(items)
       48    0.000    0.000    0.000    0.000 \_collections\_abc.py:802(\_\_init\_\_)
      108    0.000    0.000    0.000    0.000 \_collections\_abc.py:849(\_\_iter\_\_)
       24    0.000    0.000    0.000    0.000 \_ufunc\_config.py:132(geterr)
       24    0.000    0.000    0.000    0.000 \_ufunc\_config.py:32(seterr)
       12    0.000    0.000    0.000    0.000 \_ufunc\_config.py:433(\_\_enter\_\_)
       12    0.000    0.000    0.000    0.000 \_ufunc\_config.py:438(\_\_exit\_\_)
       12    0.000    0.000    0.000    0.000 abc.py:117(\_\_instancecheck\_\_)
       12    0.000    0.000    0.000    0.000 contextlib.py:63(\_recreate\_cm)
       12    0.000    0.000    0.000    0.000 contextlib.py:76(inner)
       28    0.000    0.000    0.000    0.000 coreviews.py:268(\_\_init\_\_)
       12    0.000    0.000    0.000    0.000 coreviews.py:275(\_\_iter\_\_)
       24    0.000    0.000    0.000    0.000 coreviews.py:282(<genexpr>)
       12    0.000    0.000    0.000    0.000 coreviews.py:284(\_\_getitem\_\_)
       28    0.000    0.000    0.000    0.000 coreviews.py:316(\_\_init\_\_)
       12    0.000    0.000    0.000    0.000 coreviews.py:324(\_\_iter\_\_)
       36    0.000    0.000    0.000    0.000 coreviews.py:330(<genexpr>)
       48    0.000    0.000    0.000    0.000 coreviews.py:383(\_\_iter\_\_)
       48    0.000    0.000    0.000    0.000 coreviews.py:391(<genexpr>)
       24    0.000    0.000    0.000    0.000 coreviews.py:401(\_\_getitem\_\_)
       24    0.000    0.000    0.000    0.000 coreviews.py:404(new\_node\_ok)
       24    0.000    0.000    0.000    0.000 coreviews.py:439(\_\_getitem\_\_)
       92    0.000    0.000    0.000    0.000 coreviews.py:44(\_\_init\_\_)
       48    0.000    0.000    0.000    0.000 coreviews.py:442(edge\_ok)
       42    0.000    0.000    0.000    0.000 coreviews.py:50(\_\_iter\_\_)
        4    0.000    0.000    0.000    0.000 coreviews.py:53(\_\_getitem\_\_)
       46    0.000    0.000    0.000    0.000 coreviews.py:81(\_\_getitem\_\_)
       59    0.000    0.000    0.000    0.000 ex\_pump.py:107(behavior)
        1    0.000    0.000    0.000    0.000 ex\_pump.py:114(\_\_init\_\_)
       59    0.000    0.000    0.000    0.000 ex\_pump.py:119(behavior)
        1    0.000    0.000    0.000    0.000 ex\_pump.py:125(\_\_init\_\_)
       59    0.000    0.000    0.000    0.000 ex\_pump.py:130(behavior)
        1    0.000    0.000    0.000    0.000 ex\_pump.py:144(\_\_init\_\_)
       61    0.000    0.000    0.000    0.000 ex\_pump.py:158(condfaults)
       60    0.000    0.000    0.000    0.000 ex\_pump.py:171(behavior)
        2    0.000    0.000    0.000    0.000 ex\_pump.py:199(\_\_init\_\_)
        1    0.000    0.000    0.002    0.002 ex\_pump.py:214(\_\_init\_\_)
        1    0.000    0.000    0.000    0.000 ex\_pump.py:259(find\_classification)
        1    0.000    0.000    0.000    0.000 ex\_pump.py:49(\_\_init\_\_)
       60    0.000    0.000    0.000    0.000 ex\_pump.py:79(condfaults)
       59    0.000    0.000    0.000    0.000 ex\_pump.py:87(behavior)
        1    0.000    0.000    0.000    0.000 ex\_pump.py:99(\_\_init\_\_)
       48    0.000    0.000    0.000    0.000 filters.py:20(no\_filter)
        4    0.000    0.000    0.000    0.000 filters.py:51(\_\_init\_\_)
      156    0.000    0.000    0.000    0.000 filters.py:54(\_\_call\_\_)
        1    0.000    0.000    0.000    0.000 fromnumeric.py:2737(\_amin\_dispatcher)
        1    0.000    0.000    0.000    0.000 fromnumeric.py:2742(amin)
        1    0.000    0.000    0.000    0.000 fromnumeric.py:70(\_wrapreduction)
        1    0.000    0.000    0.000    0.000 fromnumeric.py:71(<dictcomp>)
        4    0.000    0.000    0.000    0.000 function.py:161(freeze)
        4    0.000    0.000    0.000    0.000 function.py:590(set\_node\_attributes)
        1    0.000    0.000    0.000    0.000 function.py:715(set\_edge\_attributes)
        5    0.000    0.000    0.000    0.000 getlimits.py:514(\_\_init\_\_)
        5    0.000    0.000    0.000    0.000 getlimits.py:538(max)
       16    0.000    0.000    0.000    0.000 graph.py:1214(neighbors)
        1    0.000    0.000    0.000    0.000 graph.py:1257(edges)
        1    0.000    0.000    0.000    0.000 graph.py:1380(degree)
        3    0.000    0.000    0.000    0.000 graph.py:1454(is\_multigraph)
        2    0.000    0.000    0.000    0.000 graph.py:1458(is\_directed)
        1    0.000    0.000    0.000    0.000 graph.py:1462(copy)
       10    0.000    0.000    0.000    0.000 graph.py:1543(<genexpr>)
       17    0.000    0.000    0.000    0.000 graph.py:1544(<genexpr>)
        4    0.000    0.000    0.000    0.000 graph.py:1664(subgraph)
        4    0.000    0.000    0.000    0.000 graph.py:1863(nbunch\_iter)
       12    0.000    0.000    0.000    0.000 graph.py:1909(bunch\_iter)
        8    0.000    0.000    0.000    0.000 graph.py:289(\_\_init\_\_)
       46    0.000    0.000    0.000    0.000 graph.py:338(adj)
        4    0.000    0.000    0.000    0.000 graph.py:416(\_\_contains\_\_)
       46    0.000    0.000    0.000    0.000 graph.py:452(\_\_getitem\_\_)
        5    0.000    0.000    0.000    0.000 graph.py:526(add\_nodes\_from)
       29    0.000    0.000    0.000    0.000 graph.py:661(nodes)
        7    0.000    0.000    0.000    0.000 graph.py:895(add\_edges\_from)
        4    0.000    0.000    0.000    0.000 graphviews.py:75(subgraph\_view)
        1    0.000    0.000    0.000    0.000 isolate.py:40(isolates)
        1    0.000    0.000    0.000    0.000 isolate.py:82(<genexpr>)
        1    0.000    0.000    0.001    0.001 modeldef.py:1008(build\_model)
        1    0.000    0.000    0.000    0.000 modeldef.py:1023(<listcomp>)
        1    0.000    0.000    0.000    0.000 modeldef.py:1024(<listcomp>)
        1    0.000    0.000    0.000    0.000 modeldef.py:1026(<listcomp>)
        1    0.000    0.000    0.001    0.001 modeldef.py:1027(construct\_graph)
        1    0.000    0.000    0.000    0.000 modeldef.py:1041(<listcomp>)
        4    0.000    0.000    0.000    0.000 modeldef.py:1050(<listcomp>)
        1    0.000    0.000    0.000    0.000 modeldef.py:1110(return\_stategraph)
        1    0.000    0.000    0.000    0.000 modeldef.py:1179(calc\_repaircost)
        1    0.000    0.000    0.000    0.000 modeldef.py:1201(<listcomp>)
        2    0.000    0.000    0.000    0.000 modeldef.py:1204(return\_faultmodes)
       10    0.000    0.000    0.000    0.000 modeldef.py:1217(<listcomp>)
        1    0.000    0.000    0.000    0.000 modeldef.py:1252(reset)
        1    0.000    0.000    0.000    0.000 modeldef.py:1276(\_\_init\_\_)
        1    0.000    0.000    0.000    0.000 modeldef.py:1300(reset)
        5    0.000    0.000    0.000    0.000 modeldef.py:155(\_\_init\_\_)
        5    0.000    0.000    0.000    0.000 modeldef.py:342(assoc\_modes)
      415    0.000    0.000    0.000    0.000 modeldef.py:465(has\_fault)
      301    0.000    0.000    0.000    0.000 modeldef.py:497(add\_fault)
        5    0.000    0.000    0.000    0.000 modeldef.py:553(reset)
      882    0.001    0.000    0.001    0.000 modeldef.py:571(return\_states)
        5    0.000    0.000    0.000    0.000 modeldef.py:605(\_\_init\_\_)
        5    0.000    0.000    0.000    0.000 modeldef.py:641(make\_flowdict)
      301    0.001    0.000    0.002    0.000 modeldef.py:721(updatefxn)
        4    0.000    0.000    0.000    0.000 modeldef.py:790(\_\_init\_\_)
        4    0.000    0.000    0.000    0.000 modeldef.py:811(reset)
      712    0.000    0.000    0.001    0.000 modeldef.py:815(status)
        1    0.000    0.000    0.000    0.000 modeldef.py:876(\_\_init\_\_)
        4    0.000    0.000    0.000    0.000 modeldef.py:931(add\_flow)
        5    0.000    0.000    0.001    0.000 modeldef.py:951(add\_fxn)
        5    0.000    0.000    0.000    0.000 modeldef.py:985(get\_flows)
        5    0.000    0.000    0.000    0.000 modeldef.py:987(<listcomp>)
       12    0.000    0.000    0.000    0.000 multiarray.py:1054(copyto)
       12    0.000    0.000    0.000    0.000 multiarray.py:143(concatenate)
      672    0.000    0.000    0.000    0.000 multiarray.py:478(can\_cast)
        5    0.000    0.000    0.000    0.000 multigraph.py:291(\_\_init\_\_)
        4    0.000    0.000    0.000    0.000 multigraph.py:403(add\_edge)
        8    0.000    0.000    0.000    0.000 multigraph.py:686(has\_edge)
        4    0.000    0.000    0.000    0.000 multigraph.py:742(edges)
        4    0.000    0.000    0.000    0.000 multigraph.py:915(is\_multigraph)
        4    0.000    0.000    0.000    0.000 multigraph.py:919(is\_directed)
       12    0.000    0.000    0.000    0.000 numeric.py:288(full)
       56    0.000    0.000    0.001    0.000 ordered\_set.py:122(copy)
       32    0.000    0.000    0.000    0.000 ordered\_set.py:157(\_\_contains\_\_)
      290    0.000    0.000    0.000    0.000 ordered\_set.py:169(add)
        5    0.000    0.000    0.000    0.000 ordered\_set.py:190(update)
      224    0.000    0.000    0.000    0.000 ordered\_set.py:287(\_\_iter\_\_)
       56    0.000    0.000    0.000    0.000 ordered\_set.py:336(union)
      115    0.000    0.000    0.001    0.000 ordered\_set.py:65(\_\_init\_\_)
      280    0.000    0.000    0.000    0.000 ordered\_set.py:71(\_\_len\_\_)
       12    0.000    0.000    0.000    0.000 projection.py:101(<genexpr>)
       10    0.000    0.000    0.000    0.000 projection.py:103(<setcomp>)
       13    0.000    0.000    0.000    0.000 projection.py:114(<genexpr>)
        2    0.000    0.000    0.000    0.000 projection.py:15(projected\_graph)
        1    0.000    0.000    0.000    0.000 propagate.py:1021(init\_fxnhist)
        5    0.000    0.000    0.000    0.000 propagate.py:1044(<listcomp>)
        5    0.000    0.000    0.000    0.000 propagate.py:1047(<dictcomp>)
        7    0.000    0.000    0.000    0.000 propagate.py:1047(<listcomp>)
        1    0.000    0.000    0.000    0.000 propagate.py:111(new\_mdl\_params)
        1    0.000    0.000    0.011    0.011 propagate.py:36(nominal)
        1    0.000    0.000    0.000    0.000 propagate.py:629(construct\_nomscen)
        1    0.000    0.000    0.009    0.009 propagate.py:685(prop\_one\_scen)
       56    0.000    0.000    0.005    0.000 propagate.py:775(propagate)
       56    0.001    0.000    0.004    0.000 propagate.py:803(prop\_time)
       12    0.000    0.000    0.000    0.000 propagate.py:841(<listcomp>)
       56    0.000    0.000    0.003    0.000 propagate.py:856(update\_mdlhist)
       56    0.001    0.000    0.002    0.000 propagate.py:880(update\_flowhist)
       56    0.000    0.000    0.001    0.000 propagate.py:903(update\_fxnhist)
        3    0.000    0.000    0.000    0.000 propagate.py:92(update\_params)
        1    0.000    0.000    0.000    0.000 propagate.py:930(cut\_mdlhist)
        1    0.000    0.000    0.000    0.000 propagate.py:960(init\_mdlhist)
        1    0.000    0.000    0.000    0.000 propagate.py:992(<listcomp>)
        1    0.000    0.000    0.000    0.000 propagate.py:994(init\_flowhist)
        5    0.000    0.000    0.000    0.000 reportviews.py:1003(\_\_init\_\_)
        5    0.000    0.000    0.000    0.000 reportviews.py:1132(\_\_iter\_\_)
        8    0.000    0.000    0.000    0.000 reportviews.py:1247(\_\_len\_\_)
       16    0.000    0.000    0.000    0.000 reportviews.py:1248(<genexpr>)
       24    0.000    0.000    0.000    0.000 reportviews.py:1250(\_\_iter\_\_)
       29    0.000    0.000    0.000    0.000 reportviews.py:177(\_\_init\_\_)
       29    0.000    0.000    0.000    0.000 reportviews.py:187(\_\_getitem\_\_)
        1    0.000    0.000    0.000    0.000 reportviews.py:355(\_\_init\_\_)
        1    0.000    0.000    0.000    0.000 reportviews.py:362(\_\_call\_\_)
       10    0.000    0.000    0.000    0.000 reportviews.py:465(\_\_iter\_\_)
        5    0.000    0.000    0.000    0.000 \{built-in method \_\_new\_\_ of type object at 0x00007FFC5FBB3C60\}
       12    0.000    0.000    0.000    0.000 \{built-in method \_abc.\_abc\_instancecheck\}
      420    0.000    0.000    0.000    0.000 \{built-in method builtins.any\}
        1    0.000    0.000    0.011    0.011 \{built-in method builtins.exec\}
     2203    0.000    0.000    0.000    0.000 \{built-in method builtins.getattr\}
     1281    0.000    0.000    0.000    0.000 \{built-in method builtins.hasattr\}
       92    0.000    0.000    0.000    0.000 \{built-in method builtins.isinstance\}
      282    0.000    0.000    0.000    0.000 \{built-in method builtins.iter\}
     1021    0.000    0.000    0.000    0.000 \{built-in method builtins.len\}
        5    0.000    0.000    0.000    0.000 \{built-in method builtins.max\}
      180    0.000    0.000    0.000    0.000 \{built-in method builtins.min\}
       33    0.000    0.000    0.000    0.000 \{built-in method builtins.setattr\}
        9    0.000    0.000    0.000    0.000 \{built-in method builtins.sum\}
       56    0.000    0.000    0.000    0.000 \{built-in method from\_iterable\}
       14    0.000    0.000    0.000    0.000 \{built-in method fromkeys\}
        1    0.000    0.000    0.000    0.000 \{built-in method numpy.arange\}
       20    0.000    0.000    0.000    0.000 \{built-in method numpy.array\}
      697    0.001    0.000    0.001    0.000 \{built-in method numpy.core.\_multiarray\_umath.implement\_array\_function\}
       12    0.000    0.000    0.000    0.000 \{built-in method numpy.empty\}
       48    0.000    0.000    0.000    0.000 \{built-in method numpy.geterrobj\}
       24    0.000    0.000    0.000    0.000 \{built-in method numpy.seterrobj\}
       12    0.000    0.000    0.000    0.000 \{function SeedSequence.generate\_state at 0x0000019D2E712B80\}
        5    0.000    0.000    0.000    0.000 \{method 'add' of 'set' objects\}
      298    0.000    0.000    0.000    0.000 \{method 'append' of 'list' objects\}
       65    0.000    0.000    0.000    0.000 \{method 'clear' of 'set' objects\}
       54    0.000    0.000    0.000    0.000 \{method 'copy' of 'dict' objects\}
       60    0.000    0.000    0.000    0.000 \{method 'copy' of 'list' objects\}
      952    0.000    0.000    0.000    0.000 \{method 'copy' of 'set' objects\}
      301    0.000    0.000    0.000    0.000 \{method 'difference' of 'set' objects\}
        1    0.000    0.000    0.000    0.000 \{method 'disable' of '\_lsprof.Profiler' objects\}
       48    0.000    0.000    0.000    0.000 \{method 'get' of 'dict' objects\}
        5    0.000    0.000    0.000    0.000 \{method 'integers' of 'numpy.random.\_generator.Generator' objects\}
      415    0.000    0.000    0.000    0.000 \{method 'intersection' of 'set' objects\}
      546    0.000    0.000    0.000    0.000 \{method 'items' of 'dict' objects\}
       85    0.000    0.000    0.000    0.000 \{method 'keys' of 'dict' objects\}
        1    0.000    0.000    0.000    0.000 \{method 'reduce' of 'numpy.ufunc' objects\}
      130    0.000    0.000    0.000    0.000 \{method 'update' of 'dict' objects\}
      313    0.000    0.000    0.000    0.000 \{method 'update' of 'set' objects\}
        2    0.000    0.000    0.000    0.000 \{method 'values' of 'dict' objects\}
       12    0.000    0.000    0.001    0.000 \{numpy.random.\_generator.default\_rng\}


\end{sphinxVerbatim}



\end{sphinxuseclass}
\end{sphinxuseclass}
}

\end{sphinxuseclass}
\end{sphinxuseclass}
\sphinxAtStartPar
This is difficult to vizualize, although there are methods to store and sort through this output if needed. The \sphinxcode{\sphinxupquote{pycallgraph}} package (requires a graphviz installation) can give a more immediate and insightful picture of where the code bottlenecks are.

\begin{sphinxuseclass}{nbinput}
\begin{sphinxuseclass}{nblast}
{
\sphinxsetup{VerbatimColor={named}{nbsphinx-code-bg}}
\sphinxsetup{VerbatimBorderColor={named}{nbsphinx-code-border}}
\begin{sphinxVerbatim}[commandchars=\\\{\}]
\llap{\color{nbsphinxin}[47]:\,\hspace{\fboxrule}\hspace{\fboxsep}}\PYG{k+kn}{from} \PYG{n+nn}{pycallgraph2} \PYG{k+kn}{import} \PYG{n}{PyCallGraph}
\PYG{k+kn}{from} \PYG{n+nn}{pycallgraph2}\PYG{n+nn}{.}\PYG{n+nn}{output} \PYG{k+kn}{import} \PYG{n}{GraphvizOutput}
\PYG{k+kn}{from} \PYG{n+nn}{IPython}\PYG{n+nn}{.}\PYG{n+nn}{display} \PYG{k+kn}{import} \PYG{n}{Image}
\end{sphinxVerbatim}
}

\end{sphinxuseclass}
\end{sphinxuseclass}
\begin{sphinxuseclass}{nbinput}
\begin{sphinxuseclass}{nblast}
{
\sphinxsetup{VerbatimColor={named}{nbsphinx-code-bg}}
\sphinxsetup{VerbatimBorderColor={named}{nbsphinx-code-border}}
\begin{sphinxVerbatim}[commandchars=\\\{\}]
\llap{\color{nbsphinxin}[48]:\,\hspace{\fboxrule}\hspace{\fboxsep}}\PYG{k}{with} \PYG{n}{PyCallGraph}\PYG{p}{(}\PYG{n}{output}\PYG{o}{=}\PYG{n}{GraphvizOutput}\PYG{p}{(}\PYG{p}{)}\PYG{p}{)}\PYG{p}{:}
    \PYG{n}{propagate}\PYG{o}{.}\PYG{n}{nominal}\PYG{p}{(}\PYG{n}{mdl}\PYG{p}{)}
\end{sphinxVerbatim}
}

\end{sphinxuseclass}
\end{sphinxuseclass}
\begin{sphinxuseclass}{nbinput}
{
\sphinxsetup{VerbatimColor={named}{nbsphinx-code-bg}}
\sphinxsetup{VerbatimBorderColor={named}{nbsphinx-code-border}}
\begin{sphinxVerbatim}[commandchars=\\\{\}]
\llap{\color{nbsphinxin}[49]:\,\hspace{\fboxrule}\hspace{\fboxsep}}\PYG{n}{Image}\PYG{p}{(}\PYG{n}{filename}\PYG{o}{=}\PYG{l+s+s1}{\PYGZsq{}}\PYG{l+s+s1}{pycallgraph.png}\PYG{l+s+s1}{\PYGZsq{}}\PYG{p}{)}
\end{sphinxVerbatim}
}

\end{sphinxuseclass}
\begin{sphinxuseclass}{nboutput}
\begin{sphinxuseclass}{nblast}
\hrule height -\fboxrule\relax
\vspace{\nbsphinxcodecellspacing}

\savebox\nbsphinxpromptbox[0pt][r]{\color{nbsphinxout}\Verb|\strut{[49]:}\,|}

\begin{nbsphinxfancyoutput}

\begin{sphinxuseclass}{output_area}
\begin{sphinxuseclass}{}
\noindent\sphinxincludegraphics[width=8479\sphinxpxdimen,height=963\sphinxpxdimen]{{example_pump_Parallelism_Tutorial_67_0}.png}

\end{sphinxuseclass}
\end{sphinxuseclass}
\end{nbsphinxfancyoutput}

\end{sphinxuseclass}
\end{sphinxuseclass}
\sphinxAtStartPar
As shown, running this model is not particularly computationally expensive. As a result, the majority of the computational expense is not actually because of the simulation itself, but because of the way the model is simulated: \sphinxhyphen{} 0.03 of 0.13 s (23\%) is spent re\sphinxhyphen{}initalizing the model at first so that the model object can be re\sphinxhyphen{}used without worrying about it being modified by any previous executions \sphinxhyphen{} 0.026 of 0.13 s (20\%) is spent recording the model history \sphinxhyphen{} 0.062 of 0.13 s (47\%) is spent
simulating the model This is mostly because the model itself is computationally inexpensive. However, it also shows how one might easily speed up simulation for optimization or large\sphinxhyphen{}n simulations–avoiding unnecessary re\sphinxhyphen{}initialization or tracking fewer model states. This can be done using the options for \sphinxcode{\sphinxupquote{track}} (as mentioned above) and \sphinxcode{\sphinxupquote{protect}}, which specifies whether the model used is re\sphinxhyphen{}instantiated for the simulation (True) or used directly (False).

\begin{sphinxuseclass}{nbinput}
\begin{sphinxuseclass}{nblast}
{
\sphinxsetup{VerbatimColor={named}{nbsphinx-code-bg}}
\sphinxsetup{VerbatimBorderColor={named}{nbsphinx-code-border}}
\begin{sphinxVerbatim}[commandchars=\\\{\}]
\llap{\color{nbsphinxin}[ ]:\,\hspace{\fboxrule}\hspace{\fboxsep}}
\end{sphinxVerbatim}
}

\end{sphinxuseclass}
\end{sphinxuseclass}

\subsection{Defining Nominal Approaches in fmdtools}
\label{\detokenize{docs/Nominal_Approach_Use-Cases:Defining-Nominal-Approaches-in-fmdtools}}\label{\detokenize{docs/Nominal_Approach_Use-Cases::doc}}
\sphinxAtStartPar
Nominal simulation approaches are used to evaluate the performance of a model over a set of input parameters. It can then be used to: \sphinxhyphen{} define/understand the operational envelope for different system parameters (i.e., what inputs can the system safely encounter) \sphinxhyphen{} quantify failure probabilities given stochastic inputs (i.e., if the statistical distribution of inputs are known, what is the resulting probability of hazards given the design)

\begin{sphinxuseclass}{nbinput}
\begin{sphinxuseclass}{nblast}
{
\sphinxsetup{VerbatimColor={named}{nbsphinx-code-bg}}
\sphinxsetup{VerbatimBorderColor={named}{nbsphinx-code-border}}
\begin{sphinxVerbatim}[commandchars=\\\{\}]
\llap{\color{nbsphinxin}[1]:\,\hspace{\fboxrule}\hspace{\fboxsep}}\PYG{k+kn}{import} \PYG{n+nn}{sys}\PYG{o}{,} \PYG{n+nn}{os}
\PYG{n}{sys}\PYG{o}{.}\PYG{n}{path}\PYG{o}{.}\PYG{n}{insert}\PYG{p}{(}\PYG{l+m+mi}{1}\PYG{p}{,}\PYG{n}{os}\PYG{o}{.}\PYG{n}{path}\PYG{o}{.}\PYG{n}{join}\PYG{p}{(}\PYG{l+s+s2}{\PYGZdq{}}\PYG{l+s+s2}{..}\PYG{l+s+s2}{\PYGZdq{}}\PYG{p}{)}\PYG{p}{)}

\PYG{k+kn}{from} \PYG{n+nn}{fmdtools}\PYG{n+nn}{.}\PYG{n+nn}{modeldef} \PYG{k+kn}{import} \PYG{n}{Model}\PYG{p}{,} \PYG{n}{FxnBlock}
\PYG{k+kn}{import} \PYG{n+nn}{fmdtools}\PYG{n+nn}{.}\PYG{n+nn}{resultdisp} \PYG{k}{as} \PYG{n+nn}{rd}
\PYG{k+kn}{import} \PYG{n+nn}{fmdtools}\PYG{n+nn}{.}\PYG{n+nn}{faultsim}\PYG{n+nn}{.}\PYG{n+nn}{propagate} \PYG{k}{as} \PYG{n+nn}{prop}
\end{sphinxVerbatim}
}

\end{sphinxuseclass}
\end{sphinxuseclass}
\sphinxAtStartPar
The following rover model will be used to demonstrate this approach. The main task of the rover is to follow a given lane from a starting location to an ending location.

\sphinxAtStartPar
We can view the ability of the rover to track a given line with the following function:

\begin{sphinxuseclass}{nbinput}
\begin{sphinxuseclass}{nblast}
{
\sphinxsetup{VerbatimColor={named}{nbsphinx-code-bg}}
\sphinxsetup{VerbatimBorderColor={named}{nbsphinx-code-border}}
\begin{sphinxVerbatim}[commandchars=\\\{\}]
\llap{\color{nbsphinxin}[2]:\,\hspace{\fboxrule}\hspace{\fboxsep}}\PYG{k+kn}{import} \PYG{n+nn}{matplotlib}\PYG{n+nn}{.}\PYG{n+nn}{pyplot} \PYG{k}{as} \PYG{n+nn}{plt}
\PYG{k}{def} \PYG{n+nf}{plot\PYGZus{}map}\PYG{p}{(}\PYG{n}{mdl}\PYG{p}{,} \PYG{n}{mdlhist}\PYG{p}{)}\PYG{p}{:}
    \PYG{n}{fig} \PYG{o}{=} \PYG{n}{plt}\PYG{o}{.}\PYG{n}{figure}\PYG{p}{(}\PYG{p}{)}
    \PYG{n}{ax} \PYG{o}{=} \PYG{n}{fig}\PYG{o}{.}\PYG{n}{add\PYGZus{}subplot}\PYG{p}{(}\PYG{l+m+mi}{111}\PYG{p}{)}
    \PYG{n}{x\PYGZus{}ground} \PYG{o}{=} \PYG{n}{mdlhist}\PYG{p}{[}\PYG{l+s+s1}{\PYGZsq{}}\PYG{l+s+s1}{flows}\PYG{l+s+s1}{\PYGZsq{}}\PYG{p}{]}\PYG{p}{[}\PYG{l+s+s1}{\PYGZsq{}}\PYG{l+s+s1}{Ground}\PYG{l+s+s1}{\PYGZsq{}}\PYG{p}{]}\PYG{p}{[}\PYG{l+s+s1}{\PYGZsq{}}\PYG{l+s+s1}{x}\PYG{l+s+s1}{\PYGZsq{}}\PYG{p}{]}
    \PYG{n}{y\PYGZus{}ground} \PYG{o}{=} \PYG{n}{mdlhist}\PYG{p}{[}\PYG{l+s+s1}{\PYGZsq{}}\PYG{l+s+s1}{flows}\PYG{l+s+s1}{\PYGZsq{}}\PYG{p}{]}\PYG{p}{[}\PYG{l+s+s1}{\PYGZsq{}}\PYG{l+s+s1}{Ground}\PYG{l+s+s1}{\PYGZsq{}}\PYG{p}{]}\PYG{p}{[}\PYG{l+s+s1}{\PYGZsq{}}\PYG{l+s+s1}{liney}\PYG{l+s+s1}{\PYGZsq{}}\PYG{p}{]}
    \PYG{n}{plt}\PYG{o}{.}\PYG{n}{plot}\PYG{p}{(}\PYG{n}{x\PYGZus{}ground}\PYG{p}{,}\PYG{n}{y\PYGZus{}ground}\PYG{p}{,} \PYG{n}{label}\PYG{o}{=}\PYG{l+s+s2}{\PYGZdq{}}\PYG{l+s+s2}{Centerline}\PYG{l+s+s2}{\PYGZdq{}}\PYG{p}{)}

    \PYG{n}{x\PYGZus{}rover} \PYG{o}{=} \PYG{n}{mdlhist}\PYG{p}{[}\PYG{l+s+s1}{\PYGZsq{}}\PYG{l+s+s1}{flows}\PYG{l+s+s1}{\PYGZsq{}}\PYG{p}{]}\PYG{p}{[}\PYG{l+s+s1}{\PYGZsq{}}\PYG{l+s+s1}{Ground}\PYG{l+s+s1}{\PYGZsq{}}\PYG{p}{]}\PYG{p}{[}\PYG{l+s+s1}{\PYGZsq{}}\PYG{l+s+s1}{x}\PYG{l+s+s1}{\PYGZsq{}}\PYG{p}{]}
    \PYG{n}{y\PYGZus{}rover} \PYG{o}{=} \PYG{n}{mdlhist}\PYG{p}{[}\PYG{l+s+s1}{\PYGZsq{}}\PYG{l+s+s1}{flows}\PYG{l+s+s1}{\PYGZsq{}}\PYG{p}{]}\PYG{p}{[}\PYG{l+s+s1}{\PYGZsq{}}\PYG{l+s+s1}{Ground}\PYG{l+s+s1}{\PYGZsq{}}\PYG{p}{]}\PYG{p}{[}\PYG{l+s+s1}{\PYGZsq{}}\PYG{l+s+s1}{y}\PYG{l+s+s1}{\PYGZsq{}}\PYG{p}{]}
    \PYG{n}{plt}\PYG{o}{.}\PYG{n}{plot}\PYG{p}{(}\PYG{n}{x\PYGZus{}rover}\PYG{p}{,}\PYG{n}{y\PYGZus{}rover}\PYG{p}{,} \PYG{n}{label} \PYG{o}{=} \PYG{l+s+s2}{\PYGZdq{}}\PYG{l+s+s2}{Rover}\PYG{l+s+s2}{\PYGZdq{}}\PYG{p}{)}

    \PYG{n}{plt}\PYG{o}{.}\PYG{n}{scatter}\PYG{p}{(}\PYG{n}{mdl}\PYG{o}{.}\PYG{n}{params}\PYG{p}{[}\PYG{l+s+s1}{\PYGZsq{}}\PYG{l+s+s1}{end}\PYG{l+s+s1}{\PYGZsq{}}\PYG{p}{]}\PYG{p}{[}\PYG{l+m+mi}{0}\PYG{p}{]}\PYG{p}{,}\PYG{n}{mdl}\PYG{o}{.}\PYG{n}{params}\PYG{p}{[}\PYG{l+s+s1}{\PYGZsq{}}\PYG{l+s+s1}{end}\PYG{l+s+s1}{\PYGZsq{}}\PYG{p}{]}\PYG{p}{[}\PYG{l+m+mi}{1}\PYG{p}{]}\PYG{p}{,} \PYG{n}{label}\PYG{o}{=}\PYG{l+s+s2}{\PYGZdq{}}\PYG{l+s+s2}{End Location}\PYG{l+s+s2}{\PYGZdq{}}\PYG{p}{)}
    \PYG{n}{plt}\PYG{o}{.}\PYG{n}{scatter}\PYG{p}{(}\PYG{n}{mdlhist}\PYG{p}{[}\PYG{l+s+s1}{\PYGZsq{}}\PYG{l+s+s1}{flows}\PYG{l+s+s1}{\PYGZsq{}}\PYG{p}{]}\PYG{p}{[}\PYG{l+s+s1}{\PYGZsq{}}\PYG{l+s+s1}{Ground}\PYG{l+s+s1}{\PYGZsq{}}\PYG{p}{]}\PYG{p}{[}\PYG{l+s+s1}{\PYGZsq{}}\PYG{l+s+s1}{x}\PYG{l+s+s1}{\PYGZsq{}}\PYG{p}{]}\PYG{p}{[}\PYG{o}{\PYGZhy{}}\PYG{l+m+mi}{1}\PYG{p}{]}\PYG{p}{,}\PYG{n}{mdlhist}\PYG{p}{[}\PYG{l+s+s1}{\PYGZsq{}}\PYG{l+s+s1}{flows}\PYG{l+s+s1}{\PYGZsq{}}\PYG{p}{]}\PYG{p}{[}\PYG{l+s+s1}{\PYGZsq{}}\PYG{l+s+s1}{Ground}\PYG{l+s+s1}{\PYGZsq{}}\PYG{p}{]}\PYG{p}{[}\PYG{l+s+s1}{\PYGZsq{}}\PYG{l+s+s1}{y}\PYG{l+s+s1}{\PYGZsq{}}\PYG{p}{]}\PYG{p}{[}\PYG{o}{\PYGZhy{}}\PYG{l+m+mi}{1}\PYG{p}{]}\PYG{p}{,} \PYG{n}{label}\PYG{o}{=}\PYG{l+s+s2}{\PYGZdq{}}\PYG{l+s+s2}{Final Position}\PYG{l+s+s2}{\PYGZdq{}}\PYG{p}{)}

    \PYG{n}{plt}\PYG{o}{.}\PYG{n}{scatter}\PYG{p}{(}\PYG{l+m+mi}{0}\PYG{p}{,}\PYG{l+m+mi}{0}\PYG{p}{,} \PYG{n}{label}\PYG{o}{=}\PYG{l+s+s2}{\PYGZdq{}}\PYG{l+s+s2}{Start Location}\PYG{l+s+s2}{\PYGZdq{}}\PYG{p}{)}

    \PYG{n}{plt}\PYG{o}{.}\PYG{n}{xlabel}\PYG{p}{(}\PYG{l+s+s2}{\PYGZdq{}}\PYG{l+s+s2}{x\PYGZhy{}distance (meters)}\PYG{l+s+s2}{\PYGZdq{}}\PYG{p}{)}
    \PYG{n}{plt}\PYG{o}{.}\PYG{n}{ylabel}\PYG{p}{(}\PYG{l+s+s2}{\PYGZdq{}}\PYG{l+s+s2}{y\PYGZhy{}distance (meters)}\PYG{l+s+s2}{\PYGZdq{}}\PYG{p}{)}
    \PYG{n}{plt}\PYG{o}{.}\PYG{n}{title}\PYG{p}{(}\PYG{l+s+s2}{\PYGZdq{}}\PYG{l+s+s2}{Rover Centerline Tracking}\PYG{l+s+s2}{\PYGZdq{}}\PYG{p}{)}
    \PYG{n}{plt}\PYG{o}{.}\PYG{n}{grid}\PYG{p}{(}\PYG{p}{)}

    \PYG{n}{plt}\PYG{o}{.}\PYG{n}{legend}\PYG{p}{(}\PYG{p}{)}

    \PYG{n}{fig} \PYG{o}{=} \PYG{n}{plt}\PYG{o}{.}\PYG{n}{figure}\PYG{p}{(}\PYG{p}{)}
    \PYG{k}{if} \PYG{n}{mdl}\PYG{o}{.}\PYG{n}{params}\PYG{p}{[}\PYG{l+s+s1}{\PYGZsq{}}\PYG{l+s+s1}{linetype}\PYG{l+s+s1}{\PYGZsq{}}\PYG{p}{]}\PYG{o}{==}\PYG{l+s+s1}{\PYGZsq{}}\PYG{l+s+s1}{sine}\PYG{l+s+s1}{\PYGZsq{}}\PYG{p}{:}
        \PYG{n}{y\PYGZus{}line} \PYG{o}{=} \PYG{p}{[}\PYG{n}{sin\PYGZus{}func}\PYG{p}{(}\PYG{n}{x}\PYG{p}{,}\PYG{n}{y\PYGZus{}rover}\PYG{p}{[}\PYG{n}{i}\PYG{p}{]}\PYG{p}{,} \PYG{n}{mdl}\PYG{o}{.}\PYG{n}{params}\PYG{p}{[}\PYG{l+s+s1}{\PYGZsq{}}\PYG{l+s+s1}{amp}\PYG{l+s+s1}{\PYGZsq{}}\PYG{p}{]}\PYG{p}{,} \PYG{n}{mdl}\PYG{o}{.}\PYG{n}{params}\PYG{p}{[}\PYG{l+s+s1}{\PYGZsq{}}\PYG{l+s+s1}{period}\PYG{l+s+s1}{\PYGZsq{}}\PYG{p}{]}\PYG{p}{)}\PYG{p}{[}\PYG{l+m+mi}{1}\PYG{p}{]} \PYG{k}{for} \PYG{n}{i}\PYG{p}{,}\PYG{n}{x} \PYG{o+ow}{in} \PYG{n+nb}{enumerate}\PYG{p}{(}\PYG{n}{x\PYGZus{}rover}\PYG{p}{)}\PYG{p}{]}
    \PYG{k}{elif} \PYG{n}{mdl}\PYG{o}{.}\PYG{n}{params}\PYG{p}{[}\PYG{l+s+s1}{\PYGZsq{}}\PYG{l+s+s1}{linetype}\PYG{l+s+s1}{\PYGZsq{}}\PYG{p}{]}\PYG{o}{==}\PYG{l+s+s1}{\PYGZsq{}}\PYG{l+s+s1}{turn}\PYG{l+s+s1}{\PYGZsq{}}\PYG{p}{:}
        \PYG{n}{y\PYGZus{}line} \PYG{o}{=} \PYG{p}{[}\PYG{n}{turn\PYGZus{}func}\PYG{p}{(}\PYG{n}{x}\PYG{p}{,}\PYG{n}{y\PYGZus{}rover}\PYG{p}{[}\PYG{n}{i}\PYG{p}{]}\PYG{p}{,} \PYG{n}{mdl}\PYG{o}{.}\PYG{n}{params}\PYG{p}{[}\PYG{l+s+s1}{\PYGZsq{}}\PYG{l+s+s1}{radius}\PYG{l+s+s1}{\PYGZsq{}}\PYG{p}{]}\PYG{p}{,} \PYG{n}{mdl}\PYG{o}{.}\PYG{n}{params}\PYG{p}{[}\PYG{l+s+s1}{\PYGZsq{}}\PYG{l+s+s1}{start}\PYG{l+s+s1}{\PYGZsq{}}\PYG{p}{]}\PYG{p}{)}\PYG{p}{[}\PYG{l+m+mi}{1}\PYG{p}{]} \PYG{k}{for} \PYG{n}{i}\PYG{p}{,}\PYG{n}{x} \PYG{o+ow}{in} \PYG{n+nb}{enumerate}\PYG{p}{(}\PYG{n}{x\PYGZus{}rover}\PYG{p}{)}\PYG{p}{]}

    \PYG{n}{plt}\PYG{o}{.}\PYG{n}{plot}\PYG{p}{(}\PYG{n}{x\PYGZus{}rover}\PYG{p}{,} \PYG{n}{y\PYGZus{}rover}\PYG{o}{\PYGZhy{}}\PYG{n}{y\PYGZus{}line}\PYG{p}{)}
    \PYG{n}{plt}\PYG{o}{.}\PYG{n}{xlabel}\PYG{p}{(}\PYG{l+s+s2}{\PYGZdq{}}\PYG{l+s+s2}{x\PYGZhy{}distance (meters)}\PYG{l+s+s2}{\PYGZdq{}}\PYG{p}{)}
    \PYG{n}{plt}\PYG{o}{.}\PYG{n}{ylabel}\PYG{p}{(}\PYG{l+s+s2}{\PYGZdq{}}\PYG{l+s+s2}{y\PYGZhy{}error (meters)}\PYG{l+s+s2}{\PYGZdq{}}\PYG{p}{)}
    \PYG{n}{plt}\PYG{o}{.}\PYG{n}{title}\PYG{p}{(}\PYG{l+s+s2}{\PYGZdq{}}\PYG{l+s+s2}{Rover Centerline Error}\PYG{l+s+s2}{\PYGZdq{}}\PYG{p}{)}
\end{sphinxVerbatim}
}

\end{sphinxuseclass}
\end{sphinxuseclass}
\sphinxAtStartPar
Additionally, this model has a corresponding \sphinxstyleemphasis{parameter generation} function which generates the design parameters (\sphinxcode{\sphinxupquote{params}}) of the model given a reduced space of input parameters. This is the set use\sphinxhyphen{}case because often the \sphinxstyleemphasis{explored} set of variables is smaller than the full set. Parameter generation functions are specified in the form \sphinxcode{\sphinxupquote{method(staticargs, rangearg1=x, rangearg2=y)}}, where staticargs are arguments used to define discrete cases and rangeargs are arguments used to define
corresponding varied parameters.

\sphinxAtStartPar
In this case, the static cases are ‘sine’ or ‘turn’, the type of curve being generated in the function, with the corresponding range parameters of \sphinxcode{\sphinxupquote{amplitude}} and \sphinxcode{\sphinxupquote{period}} (of the wave used to generate the lane), and \sphinxcode{\sphinxupquote{radius}} (of the turn) and \sphinxcode{\sphinxupquote{start}} (distance before/after the curve starts/ends), respectively.

\begin{sphinxuseclass}{nbinput}
\begin{sphinxuseclass}{nblast}
{
\sphinxsetup{VerbatimColor={named}{nbsphinx-code-bg}}
\sphinxsetup{VerbatimBorderColor={named}{nbsphinx-code-border}}
\begin{sphinxVerbatim}[commandchars=\\\{\}]
\llap{\color{nbsphinxin}[3]:\,\hspace{\fboxrule}\hspace{\fboxsep}}\PYG{k}{def} \PYG{n+nf}{gen\PYGZus{}params}\PYG{p}{(}\PYG{n}{linetype}\PYG{p}{,} \PYG{o}{*}\PYG{o}{*}\PYG{n}{kwargs}\PYG{p}{)}\PYG{p}{:}
    \PYG{k}{if} \PYG{n}{linetype} \PYG{o}{==} \PYG{l+s+s1}{\PYGZsq{}}\PYG{l+s+s1}{sine}\PYG{l+s+s1}{\PYGZsq{}}\PYG{p}{:}
        \PYG{n}{amp} \PYG{o}{=} \PYG{n}{kwargs}\PYG{o}{.}\PYG{n}{pop}\PYG{p}{(}\PYG{l+s+s1}{\PYGZsq{}}\PYG{l+s+s1}{amp}\PYG{l+s+s1}{\PYGZsq{}}\PYG{p}{,}\PYG{l+m+mf}{0.2}\PYG{p}{)}
        \PYG{n}{wavelength}\PYG{o}{=}\PYG{n}{kwargs}\PYG{o}{.}\PYG{n}{pop}\PYG{p}{(}\PYG{l+s+s1}{\PYGZsq{}}\PYG{l+s+s1}{wavelength}\PYG{l+s+s1}{\PYGZsq{}}\PYG{p}{,} \PYG{l+m+mf}{50.0}\PYG{p}{)}
        \PYG{n}{period} \PYG{o}{=} \PYG{l+m+mi}{2}\PYG{o}{*}\PYG{n}{np}\PYG{o}{.}\PYG{n}{pi}\PYG{o}{/}\PYG{n}{wavelength}
        \PYG{n}{initangle} \PYG{o}{=} \PYG{n}{sin\PYGZus{}angle\PYGZus{}func}\PYG{p}{(}\PYG{l+m+mf}{0.0}\PYG{p}{,} \PYG{n}{amp}\PYG{p}{,} \PYG{n}{period}\PYG{p}{)}
        \PYG{n}{lineparams} \PYG{o}{=} \PYG{p}{\PYGZob{}}\PYG{l+s+s1}{\PYGZsq{}}\PYG{l+s+s1}{linetype}\PYG{l+s+s1}{\PYGZsq{}}\PYG{p}{:}\PYG{n}{linetype}\PYG{p}{,} \PYG{l+s+s1}{\PYGZsq{}}\PYG{l+s+s1}{amp}\PYG{l+s+s1}{\PYGZsq{}}\PYG{p}{:}\PYG{n}{amp}\PYG{p}{,} \PYG{l+s+s1}{\PYGZsq{}}\PYG{l+s+s1}{period}\PYG{l+s+s1}{\PYGZsq{}}\PYG{p}{:}\PYG{n}{period}\PYG{p}{,} \PYG{l+s+s1}{\PYGZsq{}}\PYG{l+s+s1}{initangle}\PYG{l+s+s1}{\PYGZsq{}}\PYG{p}{:}\PYG{n}{initangle}\PYG{p}{,} \PYG{l+s+s1}{\PYGZsq{}}\PYG{l+s+s1}{end}\PYG{l+s+s1}{\PYGZsq{}}\PYG{p}{:}\PYG{p}{[}\PYG{n}{wavelength}\PYG{p}{,}\PYG{l+m+mf}{0.0}\PYG{p}{]}\PYG{p}{\PYGZcb{}}
    \PYG{k}{elif} \PYG{n}{linetype} \PYG{o}{==} \PYG{l+s+s1}{\PYGZsq{}}\PYG{l+s+s1}{turn}\PYG{l+s+s1}{\PYGZsq{}}\PYG{p}{:}
        \PYG{n}{radius} \PYG{o}{=} \PYG{n}{kwargs}\PYG{o}{.}\PYG{n}{pop}\PYG{p}{(}\PYG{l+s+s1}{\PYGZsq{}}\PYG{l+s+s1}{radius}\PYG{l+s+s1}{\PYGZsq{}}\PYG{p}{,}\PYG{l+m+mf}{20.0}\PYG{p}{)}
        \PYG{n}{start} \PYG{o}{=} \PYG{n}{kwargs}\PYG{o}{.}\PYG{n}{pop}\PYG{p}{(}\PYG{l+s+s1}{\PYGZsq{}}\PYG{l+s+s1}{start}\PYG{l+s+s1}{\PYGZsq{}}\PYG{p}{,} \PYG{l+m+mf}{20.0}\PYG{p}{)}
        \PYG{n}{lineparams} \PYG{o}{=} \PYG{p}{\PYGZob{}}\PYG{l+s+s1}{\PYGZsq{}}\PYG{l+s+s1}{linetype}\PYG{l+s+s1}{\PYGZsq{}}\PYG{p}{:}\PYG{n}{linetype}\PYG{p}{,} \PYG{l+s+s1}{\PYGZsq{}}\PYG{l+s+s1}{radius}\PYG{l+s+s1}{\PYGZsq{}}\PYG{p}{:}\PYG{n}{radius}\PYG{p}{,} \PYG{l+s+s1}{\PYGZsq{}}\PYG{l+s+s1}{start}\PYG{l+s+s1}{\PYGZsq{}}\PYG{p}{:}\PYG{n}{start}\PYG{p}{,} \PYG{l+s+s1}{\PYGZsq{}}\PYG{l+s+s1}{initangle}\PYG{l+s+s1}{\PYGZsq{}}\PYG{p}{:}\PYG{l+m+mf}{0.0}\PYG{p}{,} \PYG{l+s+s1}{\PYGZsq{}}\PYG{l+s+s1}{end}\PYG{l+s+s1}{\PYGZsq{}}\PYG{p}{:}\PYG{p}{[}\PYG{n}{radius}\PYG{o}{+}\PYG{n}{start}\PYG{p}{,}\PYG{n}{radius}\PYG{o}{+}\PYG{n}{start}\PYG{p}{]}\PYG{p}{\PYGZcb{}}
    \PYG{k}{return} \PYG{n}{lineparams}
\end{sphinxVerbatim}
}

\end{sphinxuseclass}
\end{sphinxuseclass}
\sphinxAtStartPar
Below is the rest of the model setup:

\begin{sphinxuseclass}{nbinput}
\begin{sphinxuseclass}{nblast}
{
\sphinxsetup{VerbatimColor={named}{nbsphinx-code-bg}}
\sphinxsetup{VerbatimBorderColor={named}{nbsphinx-code-border}}
\begin{sphinxVerbatim}[commandchars=\\\{\}]
\llap{\color{nbsphinxin}[4]:\,\hspace{\fboxrule}\hspace{\fboxsep}}\PYG{k+kn}{import} \PYG{n+nn}{numpy} \PYG{k}{as} \PYG{n+nn}{np}
\PYG{k}{class} \PYG{n+nc}{Avionics}\PYG{p}{(}\PYG{n}{FxnBlock}\PYG{p}{)}\PYG{p}{:}
    \PYG{k}{def} \PYG{n+nf+fm}{\PYGZus{}\PYGZus{}init\PYGZus{}\PYGZus{}}\PYG{p}{(}\PYG{n+nb+bp}{self}\PYG{p}{,}\PYG{n}{name}\PYG{p}{,} \PYG{n}{flows}\PYG{p}{,} \PYG{n}{params}\PYG{p}{)}\PYG{p}{:}
        \PYG{n+nb+bp}{self}\PYG{o}{.}\PYG{n}{add\PYGZus{}params}\PYG{p}{(}\PYG{n}{params}\PYG{p}{)}
        \PYG{n+nb}{super}\PYG{p}{(}\PYG{p}{)}\PYG{o}{.}\PYG{n+nf+fm}{\PYGZus{}\PYGZus{}init\PYGZus{}\PYGZus{}}\PYG{p}{(}\PYG{n}{name}\PYG{p}{,} \PYG{n}{flows}\PYG{p}{,} \PYG{n}{flownames}\PYG{o}{=}\PYG{p}{\PYGZob{}}\PYG{l+s+s1}{\PYGZsq{}}\PYG{l+s+s1}{AvionicsControl}\PYG{l+s+s1}{\PYGZsq{}}\PYG{p}{:}\PYG{l+s+s1}{\PYGZsq{}}\PYG{l+s+s1}{Control}\PYG{l+s+s1}{\PYGZsq{}}\PYG{p}{\PYGZcb{}}\PYG{p}{)}
        \PYG{n+nb+bp}{self}\PYG{o}{.}\PYG{n}{assoc\PYGZus{}modes}\PYG{p}{(}\PYG{p}{\PYGZob{}}\PYG{l+s+s1}{\PYGZsq{}}\PYG{l+s+s1}{no\PYGZus{}con}\PYG{l+s+s1}{\PYGZsq{}}\PYG{p}{:}\PYG{p}{[}\PYG{l+m+mf}{1e\PYGZhy{}4}\PYG{p}{,} \PYG{l+m+mi}{200}\PYG{p}{]}\PYG{p}{\PYGZcb{}}\PYG{p}{,} \PYG{p}{[}\PYG{l+s+s1}{\PYGZsq{}}\PYG{l+s+s1}{drive}\PYG{l+s+s1}{\PYGZsq{}}\PYG{p}{,}\PYG{l+s+s1}{\PYGZsq{}}\PYG{l+s+s1}{standby}\PYG{l+s+s1}{\PYGZsq{}}\PYG{p}{]}\PYG{p}{,} \PYG{n}{initmode}\PYG{o}{=}\PYG{l+s+s1}{\PYGZsq{}}\PYG{l+s+s1}{standby}\PYG{l+s+s1}{\PYGZsq{}}\PYG{p}{)}
    \PYG{k}{def} \PYG{n+nf}{dynamic\PYGZus{}behavior}\PYG{p}{(}\PYG{n+nb+bp}{self}\PYG{p}{,}\PYG{n}{time}\PYG{p}{)}\PYG{p}{:}
        \PYG{k}{if} \PYG{o+ow}{not} \PYG{n+nb+bp}{self}\PYG{o}{.}\PYG{n}{in\PYGZus{}mode}\PYG{p}{(}\PYG{l+s+s1}{\PYGZsq{}}\PYG{l+s+s1}{no\PYGZus{}con}\PYG{l+s+s1}{\PYGZsq{}}\PYG{p}{)}\PYG{p}{:}
            \PYG{k}{if} \PYG{n}{time} \PYG{o}{==} \PYG{l+m+mi}{5}\PYG{p}{:} \PYG{n+nb+bp}{self}\PYG{o}{.}\PYG{n}{set\PYGZus{}mode}\PYG{p}{(}\PYG{l+s+s1}{\PYGZsq{}}\PYG{l+s+s1}{drive}\PYG{l+s+s1}{\PYGZsq{}}\PYG{p}{)}
            \PYG{k}{if} \PYG{n}{time} \PYG{o}{==} \PYG{l+m+mi}{50} \PYG{o+ow}{or} \PYG{n}{in\PYGZus{}area}\PYG{p}{(}\PYG{n+nb+bp}{self}\PYG{o}{.}\PYG{n}{end}\PYG{p}{[}\PYG{l+m+mi}{0}\PYG{p}{]}\PYG{p}{,}\PYG{n+nb+bp}{self}\PYG{o}{.}\PYG{n}{end}\PYG{p}{[}\PYG{l+m+mi}{1}\PYG{p}{]}\PYG{p}{,}\PYG{l+m+mi}{1}\PYG{p}{,}\PYG{n+nb+bp}{self}\PYG{o}{.}\PYG{n}{Pos\PYGZus{}Signal}\PYG{o}{.}\PYG{n}{x}\PYG{p}{,}\PYG{n+nb+bp}{self}\PYG{o}{.}\PYG{n}{Pos\PYGZus{}Signal}\PYG{o}{.}\PYG{n}{y}\PYG{p}{)}\PYG{p}{:}
                \PYG{n+nb+bp}{self}\PYG{o}{.}\PYG{n}{set\PYGZus{}mode}\PYG{p}{(}\PYG{l+s+s1}{\PYGZsq{}}\PYG{l+s+s1}{standby}\PYG{l+s+s1}{\PYGZsq{}}\PYG{p}{)}
                \PYG{n+nb+bp}{self}\PYG{o}{.}\PYG{n}{Control}\PYG{o}{.}\PYG{n}{completed}\PYG{o}{=}\PYG{l+m+mi}{1}

        \PYG{k}{if} \PYG{n+nb+bp}{self}\PYG{o}{.}\PYG{n}{in\PYGZus{}mode}\PYG{p}{(}\PYG{l+s+s1}{\PYGZsq{}}\PYG{l+s+s1}{drive}\PYG{l+s+s1}{\PYGZsq{}}\PYG{p}{)}\PYG{p}{:}
            \PYG{n+nb+bp}{self}\PYG{o}{.}\PYG{n}{Pos\PYGZus{}Signal}\PYG{o}{.}\PYG{n}{assign}\PYG{p}{(}\PYG{n+nb+bp}{self}\PYG{o}{.}\PYG{n}{Video}\PYG{p}{,} \PYG{l+s+s1}{\PYGZsq{}}\PYG{l+s+s1}{angle}\PYG{l+s+s1}{\PYGZsq{}}\PYG{p}{,} \PYG{l+s+s1}{\PYGZsq{}}\PYG{l+s+s1}{linex}\PYG{l+s+s1}{\PYGZsq{}}\PYG{p}{,} \PYG{l+s+s1}{\PYGZsq{}}\PYG{l+s+s1}{liney}\PYG{l+s+s1}{\PYGZsq{}}\PYG{p}{)}
            \PYG{n+nb+bp}{self}\PYG{o}{.}\PYG{n}{Pos\PYGZus{}Signal}\PYG{o}{.}\PYG{n}{heading} \PYG{o}{=} \PYG{n+nb+bp}{self}\PYG{o}{.}\PYG{n}{Ground}\PYG{o}{.}\PYG{n}{ang}
            \PYG{n+nb+bp}{self}\PYG{o}{.}\PYG{n}{Pos\PYGZus{}Signal}\PYG{o}{.}\PYG{n}{assign}\PYG{p}{(}\PYG{n+nb+bp}{self}\PYG{o}{.}\PYG{n}{Ground}\PYG{p}{,} \PYG{l+s+s1}{\PYGZsq{}}\PYG{l+s+s1}{x}\PYG{l+s+s1}{\PYGZsq{}}\PYG{p}{,} \PYG{l+s+s1}{\PYGZsq{}}\PYG{l+s+s1}{y}\PYG{l+s+s1}{\PYGZsq{}}\PYG{p}{,} \PYG{l+s+s1}{\PYGZsq{}}\PYG{l+s+s1}{vel}\PYG{l+s+s1}{\PYGZsq{}}\PYG{p}{)}

            \PYG{k}{if} \PYG{n}{in\PYGZus{}area}\PYG{p}{(}\PYG{n+nb+bp}{self}\PYG{o}{.}\PYG{n}{end}\PYG{p}{[}\PYG{l+m+mi}{0}\PYG{p}{]}\PYG{p}{,}\PYG{n+nb+bp}{self}\PYG{o}{.}\PYG{n}{end}\PYG{p}{[}\PYG{l+m+mi}{1}\PYG{p}{]}\PYG{p}{,}\PYG{l+m+mi}{1}\PYG{p}{,}\PYG{n+nb+bp}{self}\PYG{o}{.}\PYG{n}{Pos\PYGZus{}Signal}\PYG{o}{.}\PYG{n}{x}\PYG{p}{,}\PYG{n+nb+bp}{self}\PYG{o}{.}\PYG{n}{Pos\PYGZus{}Signal}\PYG{o}{.}\PYG{n}{y}\PYG{p}{)}\PYG{p}{:}
                \PYG{n+nb+bp}{self}\PYG{o}{.}\PYG{n}{set\PYGZus{}mode}\PYG{p}{(}\PYG{l+s+s1}{\PYGZsq{}}\PYG{l+s+s1}{standby}\PYG{l+s+s1}{\PYGZsq{}}\PYG{p}{)}
            \PYG{k}{elif} \PYG{n+nb+bp}{self}\PYG{o}{.}\PYG{n}{Video}\PYG{o}{.}\PYG{n}{quality}\PYG{o}{==}\PYG{l+m+mi}{0}\PYG{p}{:} \PYG{n+nb+bp}{self}\PYG{o}{.}\PYG{n}{set\PYGZus{}mode}\PYG{p}{(}\PYG{l+s+s1}{\PYGZsq{}}\PYG{l+s+s1}{standby}\PYG{l+s+s1}{\PYGZsq{}}\PYG{p}{)}
            \PYG{k}{else}\PYG{p}{:}
                \PYG{n}{ycorrection}\PYG{o}{=} \PYG{n}{np}\PYG{o}{.}\PYG{n}{arctan}\PYG{p}{(}\PYG{p}{(}\PYG{n+nb+bp}{self}\PYG{o}{.}\PYG{n}{Pos\PYGZus{}Signal}\PYG{o}{.}\PYG{n}{y}\PYG{o}{\PYGZhy{}}\PYG{n+nb+bp}{self}\PYG{o}{.}\PYG{n}{Pos\PYGZus{}Signal}\PYG{o}{.}\PYG{n}{liney}\PYG{p}{)}\PYG{o}{/}\PYG{p}{(}\PYG{n+nb+bp}{self}\PYG{o}{.}\PYG{n}{Pos\PYGZus{}Signal}\PYG{o}{.}\PYG{n}{vel}\PYG{o}{*}\PYG{n}{np}\PYG{o}{.}\PYG{n}{cos}\PYG{p}{(}\PYG{n}{np}\PYG{o}{.}\PYG{n}{pi}\PYG{o}{/}\PYG{l+m+mi}{180} \PYG{o}{*} \PYG{n+nb+bp}{self}\PYG{o}{.}\PYG{n}{Pos\PYGZus{}Signal}\PYG{o}{.}\PYG{n}{heading}\PYG{p}{)}\PYG{o}{+}\PYG{l+m+mf}{0.001}\PYG{p}{)}\PYG{p}{)}
                \PYG{n}{xcorrection}\PYG{o}{=} \PYG{n}{np}\PYG{o}{.}\PYG{n}{arctan}\PYG{p}{(}\PYG{p}{(}\PYG{n+nb+bp}{self}\PYG{o}{.}\PYG{n}{Pos\PYGZus{}Signal}\PYG{o}{.}\PYG{n}{x}\PYG{o}{\PYGZhy{}}\PYG{n+nb+bp}{self}\PYG{o}{.}\PYG{n}{Pos\PYGZus{}Signal}\PYG{o}{.}\PYG{n}{linex}\PYG{p}{)}\PYG{o}{/}\PYG{p}{(}\PYG{n+nb+bp}{self}\PYG{o}{.}\PYG{n}{Pos\PYGZus{}Signal}\PYG{o}{.}\PYG{n}{vel}\PYG{o}{*}\PYG{n}{np}\PYG{o}{.}\PYG{n}{sin}\PYG{p}{(}\PYG{n}{np}\PYG{o}{.}\PYG{n}{pi}\PYG{o}{/}\PYG{l+m+mi}{180} \PYG{o}{*} \PYG{n+nb+bp}{self}\PYG{o}{.}\PYG{n}{Pos\PYGZus{}Signal}\PYG{o}{.}\PYG{n}{heading}\PYG{p}{)}\PYG{o}{+}\PYG{l+m+mf}{0.001}\PYG{p}{)}\PYG{p}{)}
                \PYG{n}{rdiff} \PYG{o}{=} \PYG{p}{(}\PYG{n+nb+bp}{self}\PYG{o}{.}\PYG{n}{Pos\PYGZus{}Signal}\PYG{o}{.}\PYG{n}{angle} \PYG{o}{\PYGZhy{}} \PYG{n+nb+bp}{self}\PYG{o}{.}\PYG{n}{Pos\PYGZus{}Signal}\PYG{o}{.}\PYG{n}{heading} \PYG{o}{\PYGZhy{}}\PYG{l+m+mi}{5}\PYG{o}{*}\PYG{p}{(}\PYG{n}{xcorrection}\PYG{o}{+}\PYG{n}{ycorrection}\PYG{p}{)}\PYG{p}{)}\PYG{o}{/}\PYG{l+m+mi}{180}
                \PYG{n+nb+bp}{self}\PYG{o}{.}\PYG{n}{Control}\PYG{o}{.}\PYG{n}{put}\PYG{p}{(}\PYG{n}{rpower} \PYG{o}{=} \PYG{l+m+mi}{1}\PYG{o}{+}\PYG{n}{rdiff}\PYG{p}{,} \PYG{n}{lpower} \PYG{o}{=} \PYG{l+m+mi}{1}\PYG{o}{\PYGZhy{}}\PYG{n}{rdiff}\PYG{p}{)}
                \PYG{n+nb+bp}{self}\PYG{o}{.}\PYG{n}{Control}\PYG{o}{.}\PYG{n}{limit}\PYG{p}{(}\PYG{n}{rpower}\PYG{o}{=}\PYG{p}{(}\PYG{o}{\PYGZhy{}}\PYG{l+m+mi}{1}\PYG{p}{,}\PYG{l+m+mi}{2}\PYG{p}{)}\PYG{p}{,} \PYG{n}{lpower}\PYG{o}{=}\PYG{p}{(}\PYG{o}{\PYGZhy{}}\PYG{l+m+mi}{1}\PYG{p}{,}\PYG{l+m+mi}{2}\PYG{p}{)}\PYG{p}{)}
        \PYG{k}{if} \PYG{n+nb+bp}{self}\PYG{o}{.}\PYG{n}{in\PYGZus{}mode}\PYG{p}{(}\PYG{l+s+s1}{\PYGZsq{}}\PYG{l+s+s1}{standby}\PYG{l+s+s1}{\PYGZsq{}}\PYG{p}{)}\PYG{p}{:}   \PYG{n+nb+bp}{self}\PYG{o}{.}\PYG{n}{Control}\PYG{o}{.}\PYG{n}{put}\PYG{p}{(}\PYG{n}{rpower} \PYG{o}{=} \PYG{l+m+mi}{0}\PYG{p}{,} \PYG{n}{lpower} \PYG{o}{=} \PYG{l+m+mi}{0}\PYG{p}{)}

\PYG{k}{class} \PYG{n+nc}{Drive}\PYG{p}{(}\PYG{n}{FxnBlock}\PYG{p}{)}\PYG{p}{:}
    \PYG{k}{def} \PYG{n+nf+fm}{\PYGZus{}\PYGZus{}init\PYGZus{}\PYGZus{}}\PYG{p}{(}\PYG{n+nb+bp}{self}\PYG{p}{,}\PYG{n}{name}\PYG{p}{,} \PYG{n}{flows}\PYG{p}{)}\PYG{p}{:}
        \PYG{n+nb}{super}\PYG{p}{(}\PYG{p}{)}\PYG{o}{.}\PYG{n+nf+fm}{\PYGZus{}\PYGZus{}init\PYGZus{}\PYGZus{}}\PYG{p}{(}\PYG{n}{name}\PYG{p}{,} \PYG{n}{flows}\PYG{p}{,} \PYG{n}{flownames}\PYG{o}{=}\PYG{p}{\PYGZob{}}\PYG{l+s+s2}{\PYGZdq{}}\PYG{l+s+s2}{EE\PYGZus{}15}\PYG{l+s+s2}{\PYGZdq{}}\PYG{p}{:}\PYG{l+s+s2}{\PYGZdq{}}\PYG{l+s+s2}{EE\PYGZus{}in}\PYG{l+s+s2}{\PYGZdq{}}\PYG{p}{\PYGZcb{}}\PYG{p}{)}
        \PYG{n+nb+bp}{self}\PYG{o}{.}\PYG{n}{assoc\PYGZus{}modes}\PYG{p}{(}\PYG{p}{\PYGZob{}}\PYG{l+s+s2}{\PYGZdq{}}\PYG{l+s+s2}{mech\PYGZus{}loss}\PYG{l+s+s2}{\PYGZdq{}}\PYG{p}{,} \PYG{l+s+s2}{\PYGZdq{}}\PYG{l+s+s2}{elec\PYGZus{}open}\PYG{l+s+s2}{\PYGZdq{}}\PYG{p}{\PYGZcb{}}\PYG{p}{)}
    \PYG{k}{def} \PYG{n+nf}{dynamic\PYGZus{}behavior}\PYG{p}{(}\PYG{n+nb+bp}{self}\PYG{p}{,} \PYG{n}{time}\PYG{p}{)}\PYG{p}{:}
        \PYG{n}{rpower} \PYG{o}{=} \PYG{n+nb+bp}{self}\PYG{o}{.}\PYG{n}{EE\PYGZus{}in}\PYG{o}{.}\PYG{n}{v}\PYG{o}{*}\PYG{n+nb+bp}{self}\PYG{o}{.}\PYG{n}{MotorControl}\PYG{o}{.}\PYG{n}{rpower}\PYG{o}{*}\PYG{n+nb+bp}{self}\PYG{o}{.}\PYG{n}{no\PYGZus{}fault}\PYG{p}{(}\PYG{l+s+s2}{\PYGZdq{}}\PYG{l+s+s2}{elec\PYGZus{}open}\PYG{l+s+s2}{\PYGZdq{}}\PYG{p}{)}\PYG{o}{/}\PYG{l+m+mi}{15}
        \PYG{n}{lpower} \PYG{o}{=} \PYG{n+nb+bp}{self}\PYG{o}{.}\PYG{n}{EE\PYGZus{}in}\PYG{o}{.}\PYG{n}{v}\PYG{o}{*}\PYG{n+nb+bp}{self}\PYG{o}{.}\PYG{n}{MotorControl}\PYG{o}{.}\PYG{n}{lpower}\PYG{o}{*}\PYG{n+nb+bp}{self}\PYG{o}{.}\PYG{n}{no\PYGZus{}fault}\PYG{p}{(}\PYG{l+s+s2}{\PYGZdq{}}\PYG{l+s+s2}{elec\PYGZus{}open}\PYG{l+s+s2}{\PYGZdq{}}\PYG{p}{)}\PYG{o}{/}\PYG{l+m+mi}{15}
        \PYG{n+nb+bp}{self}\PYG{o}{.}\PYG{n}{EE\PYGZus{}in}\PYG{o}{.}\PYG{n}{a} \PYG{o}{=} \PYG{p}{(}\PYG{n}{lpower} \PYG{o}{+} \PYG{n}{rpower}\PYG{p}{)}\PYG{o}{/}\PYG{l+m+mi}{12}
        \PYG{k}{if} \PYG{p}{(}\PYG{n}{lpower} \PYG{o}{+} \PYG{n}{rpower}\PYG{p}{)} \PYG{o}{\PYGZgt{}}\PYG{l+m+mi}{100}\PYG{p}{:} \PYG{n+nb+bp}{self}\PYG{o}{.}\PYG{n}{add\PYGZus{}fault}\PYG{p}{(}\PYG{l+s+s2}{\PYGZdq{}}\PYG{l+s+s2}{elec\PYGZus{}open}\PYG{l+s+s2}{\PYGZdq{}}\PYG{p}{)}
        \PYG{k}{else}\PYG{p}{:}
            \PYG{n+nb+bp}{self}\PYG{o}{.}\PYG{n}{Ground}\PYG{o}{.}\PYG{n}{vel}\PYG{o}{=} \PYG{n}{rpower} \PYG{o}{+} \PYG{n}{lpower}
            \PYG{n+nb+bp}{self}\PYG{o}{.}\PYG{n}{Ground}\PYG{o}{.}\PYG{n}{inc}\PYG{p}{(}\PYG{n}{ang} \PYG{o}{=} \PYG{l+m+mi}{180}\PYG{o}{*}\PYG{p}{(}\PYG{n}{rpower}\PYG{o}{\PYGZhy{}}\PYG{n}{lpower}\PYG{p}{)}\PYG{o}{/}\PYG{p}{(}\PYG{n}{rpower}\PYG{o}{+}\PYG{n}{lpower} \PYG{o}{+}\PYG{l+m+mf}{0.001}\PYG{p}{)}\PYG{p}{)}
            \PYG{n+nb+bp}{self}\PYG{o}{.}\PYG{n}{Ground}\PYG{o}{.}\PYG{n}{inc}\PYG{p}{(}\PYG{n}{x} \PYG{o}{=} \PYG{n}{np}\PYG{o}{.}\PYG{n}{cos}\PYG{p}{(}\PYG{n}{np}\PYG{o}{.}\PYG{n}{pi}\PYG{o}{/}\PYG{l+m+mi}{180} \PYG{o}{*}\PYG{n+nb+bp}{self}\PYG{o}{.}\PYG{n}{Ground}\PYG{o}{.}\PYG{n}{ang}\PYG{p}{)} \PYG{o}{*} \PYG{n+nb+bp}{self}\PYG{o}{.}\PYG{n}{Ground}\PYG{o}{.}\PYG{n}{vel}\PYG{p}{,} \PYGZbs{}
                            \PYG{n}{y} \PYG{o}{=} \PYG{n}{np}\PYG{o}{.}\PYG{n}{sin}\PYG{p}{(}\PYG{n}{np}\PYG{o}{.}\PYG{n}{pi}\PYG{o}{/}\PYG{l+m+mi}{180} \PYG{o}{*}\PYG{n+nb+bp}{self}\PYG{o}{.}\PYG{n}{Ground}\PYG{o}{.}\PYG{n}{ang}\PYG{p}{)} \PYG{o}{*} \PYG{n+nb+bp}{self}\PYG{o}{.}\PYG{n}{Ground}\PYG{o}{.}\PYG{n}{vel}\PYG{p}{)}

\PYG{k}{class} \PYG{n+nc}{Perception}\PYG{p}{(}\PYG{n}{FxnBlock}\PYG{p}{)}\PYG{p}{:}
    \PYG{k}{def} \PYG{n+nf+fm}{\PYGZus{}\PYGZus{}init\PYGZus{}\PYGZus{}}\PYG{p}{(}\PYG{n+nb+bp}{self}\PYG{p}{,} \PYG{n}{name}\PYG{p}{,} \PYG{n}{flows}\PYG{p}{)}\PYG{p}{:}
        \PYG{n+nb+bp}{self}\PYG{o}{.}\PYG{n}{set\PYGZus{}atts}\PYG{p}{(}\PYG{n}{rad}\PYG{o}{=}\PYG{l+m+mi}{1}\PYG{p}{)}
        \PYG{n+nb}{super}\PYG{p}{(}\PYG{p}{)}\PYG{o}{.}\PYG{n+nf+fm}{\PYGZus{}\PYGZus{}init\PYGZus{}\PYGZus{}}\PYG{p}{(}\PYG{n}{name}\PYG{p}{,} \PYG{n}{flows}\PYG{p}{,} \PYG{n}{flownames}\PYG{o}{=}\PYG{p}{\PYGZob{}}\PYG{l+s+s1}{\PYGZsq{}}\PYG{l+s+s1}{EE\PYGZus{}12}\PYG{l+s+s1}{\PYGZsq{}}\PYG{p}{:}\PYG{l+s+s1}{\PYGZsq{}}\PYG{l+s+s1}{EE}\PYG{l+s+s1}{\PYGZsq{}}\PYG{p}{\PYGZcb{}}\PYG{p}{)}
        \PYG{n+nb+bp}{self}\PYG{o}{.}\PYG{n}{assoc\PYGZus{}modes}\PYG{p}{(}\PYG{p}{\PYGZob{}}\PYG{p}{\PYGZcb{}}\PYG{p}{,} \PYG{p}{[}\PYG{l+s+s1}{\PYGZsq{}}\PYG{l+s+s1}{off}\PYG{l+s+s1}{\PYGZsq{}}\PYG{p}{,} \PYG{l+s+s1}{\PYGZsq{}}\PYG{l+s+s1}{feed}\PYG{l+s+s1}{\PYGZsq{}}\PYG{p}{]}\PYG{p}{,} \PYG{n}{initmode}\PYG{o}{=}\PYG{l+s+s1}{\PYGZsq{}}\PYG{l+s+s1}{off}\PYG{l+s+s1}{\PYGZsq{}}\PYG{p}{)}
    \PYG{k}{def} \PYG{n+nf}{dynamic\PYGZus{}behavior}\PYG{p}{(}\PYG{n+nb+bp}{self}\PYG{p}{,}\PYG{n}{time}\PYG{p}{)}\PYG{p}{:}
        \PYG{k}{if} \PYG{n+nb+bp}{self}\PYG{o}{.}\PYG{n}{in\PYGZus{}mode}\PYG{p}{(}\PYG{l+s+s1}{\PYGZsq{}}\PYG{l+s+s1}{off}\PYG{l+s+s1}{\PYGZsq{}}\PYG{p}{)}\PYG{p}{:}
            \PYG{n+nb+bp}{self}\PYG{o}{.}\PYG{n}{EE}\PYG{o}{.}\PYG{n}{a}\PYG{o}{=}\PYG{l+m+mi}{0}
            \PYG{n+nb+bp}{self}\PYG{o}{.}\PYG{n}{Video}\PYG{o}{.}\PYG{n}{put}\PYG{p}{(}\PYG{n}{linex} \PYG{o}{=} \PYG{l+m+mi}{0}\PYG{p}{,}\PYG{n}{liney}\PYG{o}{=}\PYG{l+m+mi}{0}\PYG{p}{,} \PYG{n}{angle} \PYG{o}{=} \PYG{l+m+mi}{0}\PYG{p}{,} \PYG{n}{quality} \PYG{o}{=} \PYG{l+m+mi}{0}\PYG{p}{)}
            \PYG{k}{if} \PYG{n+nb+bp}{self}\PYG{o}{.}\PYG{n}{EE}\PYG{o}{.}\PYG{n}{v} \PYG{o}{==}\PYG{l+m+mi}{12}\PYG{p}{:} \PYG{n+nb+bp}{self}\PYG{o}{.}\PYG{n}{set\PYGZus{}mode}\PYG{p}{(}\PYG{l+s+s2}{\PYGZdq{}}\PYG{l+s+s2}{feed}\PYG{l+s+s2}{\PYGZdq{}}\PYG{p}{)}
        \PYG{k}{elif} \PYG{n+nb+bp}{self}\PYG{o}{.}\PYG{n}{in\PYGZus{}mode}\PYG{p}{(}\PYG{l+s+s2}{\PYGZdq{}}\PYG{l+s+s2}{feed}\PYG{l+s+s2}{\PYGZdq{}}\PYG{p}{)}\PYG{p}{:}
            \PYG{k}{if} \PYG{n+nb+bp}{self}\PYG{o}{.}\PYG{n}{EE}\PYG{o}{.}\PYG{n}{v} \PYG{o}{\PYGZgt{}} \PYG{l+m+mi}{8}\PYG{p}{:}
                \PYG{k}{if} \PYG{n}{in\PYGZus{}area}\PYG{p}{(}\PYG{n+nb+bp}{self}\PYG{o}{.}\PYG{n}{Ground}\PYG{o}{.}\PYG{n}{x}\PYG{p}{,} \PYG{n+nb+bp}{self}\PYG{o}{.}\PYG{n}{Ground}\PYG{o}{.}\PYG{n}{liney}\PYG{p}{,} \PYG{n+nb+bp}{self}\PYG{o}{.}\PYG{n}{rad}\PYG{p}{,} \PYG{n+nb+bp}{self}\PYG{o}{.}\PYG{n}{Ground}\PYG{o}{.}\PYG{n}{x}\PYG{p}{,} \PYG{n+nb+bp}{self}\PYG{o}{.}\PYG{n}{Ground}\PYG{o}{.}\PYG{n}{y}\PYG{p}{)}\PYG{p}{:}
                    \PYG{n+nb+bp}{self}\PYG{o}{.}\PYG{n}{Video}\PYG{o}{.}\PYG{n}{assign}\PYG{p}{(}\PYG{n+nb+bp}{self}\PYG{o}{.}\PYG{n}{Ground}\PYG{p}{,} \PYG{l+s+s1}{\PYGZsq{}}\PYG{l+s+s1}{linex}\PYG{l+s+s1}{\PYGZsq{}}\PYG{p}{,}\PYG{l+s+s1}{\PYGZsq{}}\PYG{l+s+s1}{liney}\PYG{l+s+s1}{\PYGZsq{}}\PYG{p}{,} \PYG{l+s+s1}{\PYGZsq{}}\PYG{l+s+s1}{angle}\PYG{l+s+s1}{\PYGZsq{}}\PYG{p}{)}
                    \PYG{n+nb+bp}{self}\PYG{o}{.}\PYG{n}{Video}\PYG{o}{.}\PYG{n}{quality} \PYG{o}{=} \PYG{l+m+mi}{1}
                \PYG{k}{else}\PYG{p}{:}
                    \PYG{n+nb+bp}{self}\PYG{o}{.}\PYG{n}{Video}\PYG{o}{.}\PYG{n}{quality}\PYG{o}{=}\PYG{l+m+mi}{0}
            \PYG{k}{elif} \PYG{n+nb+bp}{self}\PYG{o}{.}\PYG{n}{EE}\PYG{o}{.}\PYG{n}{v} \PYG{o}{==} \PYG{l+m+mi}{0}\PYG{p}{:} \PYG{n+nb+bp}{self}\PYG{o}{.}\PYG{n}{set\PYGZus{}mode}\PYG{p}{(}\PYG{l+s+s2}{\PYGZdq{}}\PYG{l+s+s2}{off}\PYG{l+s+s2}{\PYGZdq{}}\PYG{p}{)}
            \PYG{k}{else}\PYG{p}{:} \PYG{n+nb+bp}{self}\PYG{o}{.}\PYG{n}{Video}\PYG{o}{.}\PYG{n}{quality} \PYG{o}{=} \PYG{l+m+mi}{0}

\PYG{k}{def} \PYG{n+nf}{in\PYGZus{}area}\PYG{p}{(}\PYG{n}{x}\PYG{p}{,}\PYG{n}{y}\PYG{p}{,}\PYG{n}{rad}\PYG{p}{,}\PYG{n}{xc}\PYG{p}{,}\PYG{n}{yc}\PYG{p}{)}\PYG{p}{:}
    \PYG{n}{dist} \PYG{o}{=} \PYG{n}{np}\PYG{o}{.}\PYG{n}{sqrt}\PYG{p}{(}\PYG{p}{(}\PYG{n}{x}\PYG{o}{\PYGZhy{}}\PYG{n}{xc}\PYG{p}{)}\PYG{o}{*}\PYG{o}{*}\PYG{l+m+mi}{2}\PYG{o}{+}\PYG{p}{(}\PYG{n}{y}\PYG{o}{\PYGZhy{}}\PYG{n}{yc}\PYG{p}{)}\PYG{o}{*}\PYG{o}{*}\PYG{l+m+mi}{2}\PYG{p}{)}
    \PYG{k}{return} \PYG{o+ow}{not} \PYG{n}{dist} \PYG{o}{\PYGZgt{}} \PYG{n}{rad}


\PYG{k}{class} \PYG{n+nc}{Power}\PYG{p}{(}\PYG{n}{FxnBlock}\PYG{p}{)}\PYG{p}{:}
    \PYG{k}{def} \PYG{n+nf+fm}{\PYGZus{}\PYGZus{}init\PYGZus{}\PYGZus{}}\PYG{p}{(}\PYG{n+nb+bp}{self}\PYG{p}{,} \PYG{n}{name}\PYG{p}{,} \PYG{n}{flows}\PYG{p}{)}\PYG{p}{:}
        \PYG{n+nb}{super}\PYG{p}{(}\PYG{p}{)}\PYG{o}{.}\PYG{n+nf+fm}{\PYGZus{}\PYGZus{}init\PYGZus{}\PYGZus{}}\PYG{p}{(}\PYG{n}{name}\PYG{p}{,}\PYG{n}{flows}\PYG{p}{,} \PYG{n}{states}\PYG{o}{=}\PYG{p}{\PYGZob{}}\PYG{l+s+s2}{\PYGZdq{}}\PYG{l+s+s2}{charge}\PYG{l+s+s2}{\PYGZdq{}}\PYG{p}{:} \PYG{l+m+mf}{100.0}\PYG{p}{,} \PYG{l+s+s1}{\PYGZsq{}}\PYG{l+s+s1}{power}\PYG{l+s+s1}{\PYGZsq{}}\PYG{p}{:}\PYG{l+m+mf}{0.0}\PYG{p}{\PYGZcb{}}\PYG{p}{)}
        \PYG{n+nb+bp}{self}\PYG{o}{.}\PYG{n}{assoc\PYGZus{}modes}\PYG{p}{(}\PYG{p}{\PYGZob{}}\PYG{l+s+s2}{\PYGZdq{}}\PYG{l+s+s2}{no\PYGZus{}charge}\PYG{l+s+s2}{\PYGZdq{}}\PYG{p}{:}\PYG{p}{[}\PYG{l+m+mf}{1e\PYGZhy{}5}\PYG{p}{,} \PYG{p}{\PYGZob{}}\PYG{l+s+s1}{\PYGZsq{}}\PYG{l+s+s1}{off}\PYG{l+s+s1}{\PYGZsq{}}\PYG{p}{:}\PYG{l+m+mf}{1.0}\PYG{p}{\PYGZcb{}}\PYG{p}{,} \PYG{l+m+mi}{100}\PYG{p}{]}\PYG{p}{,}\PYG{l+s+s2}{\PYGZdq{}}\PYG{l+s+s2}{open\PYGZus{}circ}\PYG{l+s+s2}{\PYGZdq{}}\PYG{p}{:}\PYG{p}{[}\PYG{l+m+mf}{1e\PYGZhy{}5}\PYG{p}{,} \PYG{p}{\PYGZob{}}\PYG{l+s+s1}{\PYGZsq{}}\PYG{l+s+s1}{supply}\PYG{l+s+s1}{\PYGZsq{}}\PYG{p}{:}\PYG{l+m+mf}{1.0}\PYG{p}{\PYGZcb{}}\PYG{p}{,} \PYG{l+m+mi}{100}\PYG{p}{]}\PYG{p}{\PYGZcb{}}\PYG{p}{,} \PYG{p}{[}\PYG{l+s+s2}{\PYGZdq{}}\PYG{l+s+s2}{supply}\PYG{l+s+s2}{\PYGZdq{}}\PYG{p}{,}\PYG{l+s+s2}{\PYGZdq{}}\PYG{l+s+s2}{charge}\PYG{l+s+s2}{\PYGZdq{}}\PYG{p}{,}\PYG{l+s+s2}{\PYGZdq{}}\PYG{l+s+s2}{standby}\PYG{l+s+s2}{\PYGZdq{}}\PYG{p}{,}\PYG{l+s+s2}{\PYGZdq{}}\PYG{l+s+s2}{off}\PYG{l+s+s2}{\PYGZdq{}}\PYG{p}{]}\PYG{p}{,} \PYG{n}{initmode}\PYG{o}{=}\PYG{l+s+s2}{\PYGZdq{}}\PYG{l+s+s2}{off}\PYG{l+s+s2}{\PYGZdq{}}\PYG{p}{,} \PYG{n}{exclusive} \PYG{o}{=} \PYG{k+kc}{True}\PYG{p}{,} \PYG{n}{key\PYGZus{}phases\PYGZus{}by}\PYG{o}{=}\PYG{l+s+s1}{\PYGZsq{}}\PYG{l+s+s1}{self}\PYG{l+s+s1}{\PYGZsq{}}\PYG{p}{)}
    \PYG{k}{def} \PYG{n+nf}{static\PYGZus{}behavior}\PYG{p}{(}\PYG{n+nb+bp}{self}\PYG{p}{,}\PYG{n}{time}\PYG{p}{)}\PYG{p}{:}
        \PYG{k}{if} \PYG{n+nb+bp}{self}\PYG{o}{.}\PYG{n}{in\PYGZus{}mode}\PYG{p}{(}\PYG{l+s+s2}{\PYGZdq{}}\PYG{l+s+s2}{off}\PYG{l+s+s2}{\PYGZdq{}}\PYG{p}{)}\PYG{p}{:}
            \PYG{n+nb+bp}{self}\PYG{o}{.}\PYG{n}{EE\PYGZus{}5}\PYG{o}{.}\PYG{n}{put}\PYG{p}{(}\PYG{n}{v}\PYG{o}{=}\PYG{l+m+mi}{0}\PYG{p}{,}\PYG{n}{a}\PYG{o}{=}\PYG{l+m+mi}{0}\PYG{p}{)}\PYG{p}{;} \PYG{n+nb+bp}{self}\PYG{o}{.}\PYG{n}{EE\PYGZus{}12}\PYG{o}{.}\PYG{n}{put}\PYG{p}{(}\PYG{n}{v}\PYG{o}{=}\PYG{l+m+mi}{0}\PYG{p}{,}\PYG{n}{a}\PYG{o}{=}\PYG{l+m+mi}{0}\PYG{p}{)}\PYG{p}{;}\PYG{n+nb+bp}{self}\PYG{o}{.}\PYG{n}{EE\PYGZus{}15}\PYG{o}{.}\PYG{n}{put}\PYG{p}{(}\PYG{n}{v}\PYG{o}{=}\PYG{l+m+mi}{0}\PYG{p}{,}\PYG{n}{a}\PYG{o}{=}\PYG{l+m+mi}{0}\PYG{p}{)}
            \PYG{k}{if} \PYG{n+nb+bp}{self}\PYG{o}{.}\PYG{n}{Control}\PYG{o}{.}\PYG{n}{power}\PYG{o}{==}\PYG{l+m+mi}{1} \PYG{o+ow}{and} \PYG{n+nb+bp}{self}\PYG{o}{.}\PYG{n}{AvionicsControl}\PYG{o}{.}\PYG{n}{completed}\PYG{o}{==}\PYG{l+m+mi}{0}\PYG{p}{:}   \PYG{n+nb+bp}{self}\PYG{o}{.}\PYG{n}{set\PYGZus{}mode}\PYG{p}{(}\PYG{l+s+s2}{\PYGZdq{}}\PYG{l+s+s2}{supply}\PYG{l+s+s2}{\PYGZdq{}}\PYG{p}{)}
        \PYG{k}{elif} \PYG{n+nb+bp}{self}\PYG{o}{.}\PYG{n}{in\PYGZus{}mode}\PYG{p}{(}\PYG{l+s+s2}{\PYGZdq{}}\PYG{l+s+s2}{supply}\PYG{l+s+s2}{\PYGZdq{}}\PYG{p}{)}\PYG{p}{:}
            \PYG{k}{if} \PYG{n+nb+bp}{self}\PYG{o}{.}\PYG{n}{charge} \PYG{o}{\PYGZgt{}} \PYG{l+m+mi}{0}\PYG{p}{:}         \PYG{n+nb+bp}{self}\PYG{o}{.}\PYG{n}{EE\PYGZus{}5}\PYG{o}{.}\PYG{n}{v} \PYG{o}{=} \PYG{l+m+mi}{5}\PYG{p}{;} \PYG{n+nb+bp}{self}\PYG{o}{.}\PYG{n}{EE\PYGZus{}12}\PYG{o}{.}\PYG{n}{v} \PYG{o}{=} \PYG{l+m+mi}{12}\PYG{p}{;} \PYG{n+nb+bp}{self}\PYG{o}{.}\PYG{n}{EE\PYGZus{}15}\PYG{o}{.}\PYG{n}{v} \PYG{o}{=} \PYG{l+m+mi}{15}\PYG{p}{;}
            \PYG{k}{else}\PYG{p}{:}                       \PYG{n+nb+bp}{self}\PYG{o}{.}\PYG{n}{set\PYGZus{}mode}\PYG{p}{(}\PYG{l+s+s2}{\PYGZdq{}}\PYG{l+s+s2}{no\PYGZus{}charge}\PYG{l+s+s2}{\PYGZdq{}}\PYG{p}{)}
            \PYG{k}{if} \PYG{n+nb+bp}{self}\PYG{o}{.}\PYG{n}{Control}\PYG{o}{.}\PYG{n}{power}\PYG{o}{==}\PYG{l+m+mi}{0}\PYG{p}{:}   \PYG{n+nb+bp}{self}\PYG{o}{.}\PYG{n}{set\PYGZus{}mode}\PYG{p}{(}\PYG{l+s+s2}{\PYGZdq{}}\PYG{l+s+s2}{off}\PYG{l+s+s2}{\PYGZdq{}}\PYG{p}{)}
            \PYG{k}{elif} \PYG{n+nb+bp}{self}\PYG{o}{.}\PYG{n}{AvionicsControl}\PYG{o}{.}\PYG{n}{completed}\PYG{o}{==}\PYG{l+m+mi}{1}\PYG{p}{:} \PYG{n+nb+bp}{self}\PYG{o}{.}\PYG{n}{set\PYGZus{}mode}\PYG{p}{(}\PYG{l+s+s2}{\PYGZdq{}}\PYG{l+s+s2}{off}\PYG{l+s+s2}{\PYGZdq{}}\PYG{p}{)}
        \PYG{k}{elif} \PYG{n+nb+bp}{self}\PYG{o}{.}\PYG{n}{in\PYGZus{}mode}\PYG{p}{(}\PYG{l+s+s2}{\PYGZdq{}}\PYG{l+s+s2}{no\PYGZus{}charge}\PYG{l+s+s2}{\PYGZdq{}}\PYG{p}{,}\PYG{l+s+s2}{\PYGZdq{}}\PYG{l+s+s2}{open\PYGZus{}circ}\PYG{l+s+s2}{\PYGZdq{}}\PYG{p}{)}\PYG{p}{:} \PYG{n+nb+bp}{self}\PYG{o}{.}\PYG{n}{EE\PYGZus{}5}\PYG{o}{.}\PYG{n}{v} \PYG{o}{=} \PYG{l+m+mi}{0}\PYG{p}{;} \PYG{n+nb+bp}{self}\PYG{o}{.}\PYG{n}{EE\PYGZus{}12}\PYG{o}{.}\PYG{n}{v} \PYG{o}{=} \PYG{l+m+mi}{0}\PYG{p}{;} \PYG{n+nb+bp}{self}\PYG{o}{.}\PYG{n}{EE\PYGZus{}15}\PYG{o}{.}\PYG{n}{v} \PYG{o}{=} \PYG{l+m+mi}{0}\PYG{p}{;}
        \PYG{k}{if} \PYG{n+nb+bp}{self}\PYG{o}{.}\PYG{n}{in\PYGZus{}mode}\PYG{p}{(}\PYG{l+s+s2}{\PYGZdq{}}\PYG{l+s+s2}{charge}\PYG{l+s+s2}{\PYGZdq{}}\PYG{p}{)}\PYG{p}{:}
            \PYG{n+nb+bp}{self}\PYG{o}{.}\PYG{n}{power} \PYG{o}{=} \PYG{o}{\PYGZhy{}} \PYG{l+m+mi}{1}
            \PYG{k}{if} \PYG{n+nb+bp}{self}\PYG{o}{.}\PYG{n}{charge}\PYG{o}{==}\PYG{l+m+mi}{100}\PYG{p}{:}\PYG{n+nb+bp}{self}\PYG{o}{.}\PYG{n}{set\PYGZus{}mode}\PYG{p}{(}\PYG{l+s+s2}{\PYGZdq{}}\PYG{l+s+s2}{off}\PYG{l+s+s2}{\PYGZdq{}}\PYG{p}{)}
        \PYG{k}{else}\PYG{p}{:}
            \PYG{n+nb+bp}{self}\PYG{o}{.}\PYG{n}{power}\PYG{o}{=}\PYG{l+m+mi}{1}\PYG{o}{+}\PYG{n+nb+bp}{self}\PYG{o}{.}\PYG{n}{EE\PYGZus{}12}\PYG{o}{.}\PYG{n}{mul}\PYG{p}{(}\PYG{l+s+s1}{\PYGZsq{}}\PYG{l+s+s1}{v}\PYG{l+s+s1}{\PYGZsq{}}\PYG{p}{,}\PYG{l+s+s1}{\PYGZsq{}}\PYG{l+s+s1}{a}\PYG{l+s+s1}{\PYGZsq{}}\PYG{p}{)}\PYG{o}{+}\PYG{n+nb+bp}{self}\PYG{o}{.}\PYG{n}{EE\PYGZus{}5}\PYG{o}{.}\PYG{n}{mul}\PYG{p}{(}\PYG{l+s+s1}{\PYGZsq{}}\PYG{l+s+s1}{v}\PYG{l+s+s1}{\PYGZsq{}}\PYG{p}{,}\PYG{l+s+s1}{\PYGZsq{}}\PYG{l+s+s1}{a}\PYG{l+s+s1}{\PYGZsq{}}\PYG{p}{)}\PYG{o}{+}\PYG{n+nb+bp}{self}\PYG{o}{.}\PYG{n}{EE\PYGZus{}15}\PYG{o}{.}\PYG{n}{mul}\PYG{p}{(}\PYG{l+s+s1}{\PYGZsq{}}\PYG{l+s+s1}{v}\PYG{l+s+s1}{\PYGZsq{}}\PYG{p}{,}\PYG{l+s+s1}{\PYGZsq{}}\PYG{l+s+s1}{a}\PYG{l+s+s1}{\PYGZsq{}}\PYG{p}{)}
    \PYG{k}{def} \PYG{n+nf}{dynamic\PYGZus{}behavior}\PYG{p}{(}\PYG{n+nb+bp}{self}\PYG{p}{,}\PYG{n}{time}\PYG{p}{)}\PYG{p}{:}
        \PYG{n+nb+bp}{self}\PYG{o}{.}\PYG{n}{inc}\PYG{p}{(}\PYG{n}{charge} \PYG{o}{=} \PYG{o}{\PYGZhy{}} \PYG{n+nb+bp}{self}\PYG{o}{.}\PYG{n}{power}\PYG{o}{/}\PYG{l+m+mi}{100}\PYG{p}{)}
        \PYG{n+nb+bp}{self}\PYG{o}{.}\PYG{n}{limit}\PYG{p}{(}\PYG{n}{charge}\PYG{o}{=}\PYG{p}{(}\PYG{l+m+mi}{0}\PYG{p}{,}\PYG{l+m+mi}{100}\PYG{p}{)}\PYG{p}{)}

\PYG{k}{class} \PYG{n+nc}{Override}\PYG{p}{(}\PYG{n}{FxnBlock}\PYG{p}{)}\PYG{p}{:}
    \PYG{k}{def} \PYG{n+nf+fm}{\PYGZus{}\PYGZus{}init\PYGZus{}\PYGZus{}}\PYG{p}{(}\PYG{n+nb+bp}{self}\PYG{p}{,}\PYG{n}{name}\PYG{p}{,}\PYG{n}{flows}\PYG{p}{)}\PYG{p}{:}
        \PYG{n+nb}{super}\PYG{p}{(}\PYG{p}{)}\PYG{o}{.}\PYG{n+nf+fm}{\PYGZus{}\PYGZus{}init\PYGZus{}\PYGZus{}}\PYG{p}{(}\PYG{n}{name}\PYG{p}{,}\PYG{n}{flows}\PYG{p}{,} \PYG{n}{flownames}\PYG{o}{=}\PYG{p}{\PYGZob{}}\PYG{l+s+s1}{\PYGZsq{}}\PYG{l+s+s1}{EE\PYGZus{}5}\PYG{l+s+s1}{\PYGZsq{}}\PYG{p}{:}\PYG{l+s+s1}{\PYGZsq{}}\PYG{l+s+s1}{EE}\PYG{l+s+s1}{\PYGZsq{}}\PYG{p}{\PYGZcb{}}\PYG{p}{)}
        \PYG{n+nb+bp}{self}\PYG{o}{.}\PYG{n}{assoc\PYGZus{}modes}\PYG{p}{(}\PYG{p}{\PYGZob{}}\PYG{p}{\PYGZcb{}}\PYG{p}{,} \PYG{p}{[}\PYG{l+s+s1}{\PYGZsq{}}\PYG{l+s+s1}{off}\PYG{l+s+s1}{\PYGZsq{}}\PYG{p}{,}\PYG{l+s+s1}{\PYGZsq{}}\PYG{l+s+s1}{standby}\PYG{l+s+s1}{\PYGZsq{}}\PYG{p}{,}\PYG{l+s+s1}{\PYGZsq{}}\PYG{l+s+s1}{override}\PYG{l+s+s1}{\PYGZsq{}}\PYG{p}{]}\PYG{p}{,} \PYG{n}{initmode} \PYG{o}{=} \PYG{l+s+s1}{\PYGZsq{}}\PYG{l+s+s1}{off}\PYG{l+s+s1}{\PYGZsq{}}\PYG{p}{)}
    \PYG{k}{def} \PYG{n+nf}{dynamic\PYGZus{}behavior}\PYG{p}{(}\PYG{n+nb+bp}{self}\PYG{p}{,}\PYG{n}{time}\PYG{p}{)}\PYG{p}{:}
        \PYG{k}{if} \PYG{n+nb+bp}{self}\PYG{o}{.}\PYG{n}{in\PYGZus{}mode}\PYG{p}{(}\PYG{l+s+s1}{\PYGZsq{}}\PYG{l+s+s1}{off}\PYG{l+s+s1}{\PYGZsq{}}\PYG{p}{)}\PYG{p}{:}
            \PYG{n+nb+bp}{self}\PYG{o}{.}\PYG{n}{EE}\PYG{o}{.}\PYG{n}{a}\PYG{o}{=}\PYG{l+m+mi}{0}
            \PYG{k}{if} \PYG{n+nb+bp}{self}\PYG{o}{.}\PYG{n}{EE}\PYG{o}{.}\PYG{n}{v}\PYG{o}{==}\PYG{l+m+mi}{5}\PYG{p}{:} \PYG{n+nb+bp}{self}\PYG{o}{.}\PYG{n}{set\PYGZus{}mode}\PYG{p}{(}\PYG{l+s+s1}{\PYGZsq{}}\PYG{l+s+s1}{standby}\PYG{l+s+s1}{\PYGZsq{}}\PYG{p}{)}
        \PYG{k}{elif} \PYG{n+nb+bp}{self}\PYG{o}{.}\PYG{n}{in\PYGZus{}mode}\PYG{p}{(}\PYG{l+s+s1}{\PYGZsq{}}\PYG{l+s+s1}{standby}\PYG{l+s+s1}{\PYGZsq{}}\PYG{p}{)}\PYG{p}{:}
            \PYG{n+nb+bp}{self}\PYG{o}{.}\PYG{n}{MotorControl}\PYG{o}{.}\PYG{n}{assign}\PYG{p}{(}\PYG{n+nb+bp}{self}\PYG{o}{.}\PYG{n}{AvionicsControl}\PYG{p}{,} \PYG{l+s+s1}{\PYGZsq{}}\PYG{l+s+s1}{rpower}\PYG{l+s+s1}{\PYGZsq{}}\PYG{p}{,}\PYG{l+s+s1}{\PYGZsq{}}\PYG{l+s+s1}{lpower}\PYG{l+s+s1}{\PYGZsq{}}\PYG{p}{)}
            \PYG{k}{if} \PYG{n+nb+bp}{self}\PYG{o}{.}\PYG{n}{OverrideComms} \PYG{o}{==}\PYG{l+s+s1}{\PYGZsq{}}\PYG{l+s+s1}{active}\PYG{l+s+s1}{\PYGZsq{}} \PYG{o+ow}{and} \PYG{n+nb+bp}{self}\PYG{o}{.}\PYG{n}{EE}\PYG{o}{.}\PYG{n}{v}\PYG{o}{\PYGZgt{}}\PYG{l+m+mi}{4}\PYG{p}{:} \PYG{n+nb+bp}{self}\PYG{o}{.}\PYG{n}{set\PYGZus{}mode}\PYG{p}{(}\PYG{l+s+s1}{\PYGZsq{}}\PYG{l+s+s1}{override}\PYG{l+s+s1}{\PYGZsq{}}\PYG{p}{)}
        \PYG{k}{elif} \PYG{n+nb+bp}{self}\PYG{o}{.}\PYG{n}{in\PYGZus{}mode}\PYG{p}{(}\PYG{l+s+s1}{\PYGZsq{}}\PYG{l+s+s1}{override}\PYG{l+s+s1}{\PYGZsq{}}\PYG{p}{)}\PYG{p}{:}
            \PYG{n+nb+bp}{self}\PYG{o}{.}\PYG{n}{MotorControl}\PYG{o}{.}\PYG{n}{assign}\PYG{p}{(}\PYG{n+nb+bp}{self}\PYG{o}{.}\PYG{n}{OverrideComms}\PYG{p}{,} \PYG{l+s+s1}{\PYGZsq{}}\PYG{l+s+s1}{rpower}\PYG{l+s+s1}{\PYGZsq{}}\PYG{p}{,} \PYG{l+s+s1}{\PYGZsq{}}\PYG{l+s+s1}{lpower}\PYG{l+s+s1}{\PYGZsq{}}\PYG{p}{)}

\PYG{k}{class} \PYG{n+nc}{Communications}\PYG{p}{(}\PYG{n}{FxnBlock}\PYG{p}{)}\PYG{p}{:}
    \PYG{k}{def} \PYG{n+nf+fm}{\PYGZus{}\PYGZus{}init\PYGZus{}\PYGZus{}}\PYG{p}{(}\PYG{n+nb+bp}{self}\PYG{p}{,} \PYG{n}{name}\PYG{p}{,} \PYG{n}{flows}\PYG{p}{)}\PYG{p}{:}
        \PYG{n+nb}{super}\PYG{p}{(}\PYG{p}{)}\PYG{o}{.}\PYG{n+nf+fm}{\PYGZus{}\PYGZus{}init\PYGZus{}\PYGZus{}}\PYG{p}{(}\PYG{n}{name}\PYG{p}{,}\PYG{n}{flows}\PYG{p}{)}
    \PYG{k}{def} \PYG{n+nf}{dynamic\PYGZus{}behavior}\PYG{p}{(}\PYG{n+nb+bp}{self}\PYG{p}{,}\PYG{n}{time}\PYG{p}{)}\PYG{p}{:}
        \PYG{k}{if} \PYG{n+nb+bp}{self}\PYG{o}{.}\PYG{n}{EE\PYGZus{}12}\PYG{o}{.}\PYG{n}{v} \PYG{o}{==} \PYG{l+m+mi}{12}\PYG{p}{:}
            \PYG{n+nb+bp}{self}\PYG{o}{.}\PYG{n}{EE\PYGZus{}12}\PYG{o}{.}\PYG{n}{a}\PYG{o}{=}\PYG{l+m+mi}{1}
            \PYG{n+nb+bp}{self}\PYG{o}{.}\PYG{n}{Comms}\PYG{o}{.}\PYG{n}{assign}\PYG{p}{(}\PYG{n+nb+bp}{self}\PYG{o}{.}\PYG{n}{Pos\PYGZus{}Signal}\PYG{p}{,} \PYG{l+s+s1}{\PYGZsq{}}\PYG{l+s+s1}{x}\PYG{l+s+s1}{\PYGZsq{}}\PYG{p}{,} \PYG{l+s+s1}{\PYGZsq{}}\PYG{l+s+s1}{y}\PYG{l+s+s1}{\PYGZsq{}}\PYG{p}{,} \PYG{l+s+s1}{\PYGZsq{}}\PYG{l+s+s1}{vel}\PYG{l+s+s1}{\PYGZsq{}}\PYG{p}{,} \PYG{l+s+s1}{\PYGZsq{}}\PYG{l+s+s1}{heading}\PYG{l+s+s1}{\PYGZsq{}}\PYG{p}{)}
        \PYG{k}{else}\PYG{p}{:}   \PYG{n+nb+bp}{self}\PYG{o}{.}\PYG{n}{Comms}\PYG{o}{.}\PYG{n}{put}\PYG{p}{(}\PYG{n}{x}\PYG{o}{=}\PYG{l+m+mi}{0}\PYG{p}{,} \PYG{n}{y}\PYG{o}{=}\PYG{l+m+mi}{0}\PYG{p}{,} \PYG{n}{vel}\PYG{o}{=}\PYG{l+m+mi}{0}\PYG{p}{,} \PYG{n}{heading}\PYG{o}{=}\PYG{l+m+mi}{0}\PYG{p}{)}

\PYG{k}{class} \PYG{n+nc}{Operator}\PYG{p}{(}\PYG{n}{FxnBlock}\PYG{p}{)}\PYG{p}{:}
    \PYG{k}{def} \PYG{n+nf+fm}{\PYGZus{}\PYGZus{}init\PYGZus{}\PYGZus{}}\PYG{p}{(}\PYG{n+nb+bp}{self}\PYG{p}{,} \PYG{n}{name}\PYG{p}{,} \PYG{n}{flows}\PYG{p}{)}\PYG{p}{:}
        \PYG{n+nb}{super}\PYG{p}{(}\PYG{p}{)}\PYG{o}{.}\PYG{n+nf+fm}{\PYGZus{}\PYGZus{}init\PYGZus{}\PYGZus{}}\PYG{p}{(}\PYG{n}{name}\PYG{p}{,}\PYG{n}{flows}\PYG{p}{)}
    \PYG{k}{def} \PYG{n+nf}{dynamic\PYGZus{}behavior}\PYG{p}{(}\PYG{n+nb+bp}{self}\PYG{p}{,} \PYG{n}{t}\PYG{p}{)}\PYG{p}{:}
        \PYG{k}{if} \PYG{n}{t}\PYG{o}{==}\PYG{l+m+mi}{3}\PYG{p}{:}    \PYG{n+nb+bp}{self}\PYG{o}{.}\PYG{n}{Control}\PYG{o}{.}\PYG{n}{power}\PYG{o}{=}\PYG{l+m+mi}{1}
        \PYG{k}{elif} \PYG{n}{t}\PYG{o}{==}\PYG{l+m+mi}{55}\PYG{p}{:} \PYG{n+nb+bp}{self}\PYG{o}{.}\PYG{n}{Control}\PYG{o}{.}\PYG{n}{power}\PYG{o}{=}\PYG{l+m+mi}{0}

\PYG{k}{class} \PYG{n+nc}{Environment}\PYG{p}{(}\PYG{n}{FxnBlock}\PYG{p}{)}\PYG{p}{:}
    \PYG{k}{def} \PYG{n+nf+fm}{\PYGZus{}\PYGZus{}init\PYGZus{}\PYGZus{}}\PYG{p}{(}\PYG{n+nb+bp}{self}\PYG{p}{,} \PYG{n}{name}\PYG{p}{,} \PYG{n}{flows}\PYG{p}{,}\PYG{n}{params}\PYG{p}{)}\PYG{p}{:}
        \PYG{n+nb+bp}{self}\PYG{o}{.}\PYG{n}{add\PYGZus{}params}\PYG{p}{(}\PYG{n}{params}\PYG{p}{)}
        \PYG{n+nb}{super}\PYG{p}{(}\PYG{p}{)}\PYG{o}{.}\PYG{n+nf+fm}{\PYGZus{}\PYGZus{}init\PYGZus{}\PYGZus{}}\PYG{p}{(}\PYG{n}{name}\PYG{p}{,}\PYG{n}{flows}\PYG{p}{)}
    \PYG{k}{def} \PYG{n+nf}{dynamic\PYGZus{}behavior}\PYG{p}{(}\PYG{n+nb+bp}{self}\PYG{p}{,} \PYG{n}{t}\PYG{p}{)}\PYG{p}{:}
        \PYG{k}{if} \PYG{n+nb+bp}{self}\PYG{o}{.}\PYG{n}{linetype}\PYG{o}{==}\PYG{l+s+s1}{\PYGZsq{}}\PYG{l+s+s1}{sine}\PYG{l+s+s1}{\PYGZsq{}}\PYG{p}{:}
            \PYG{n+nb+bp}{self}\PYG{o}{.}\PYG{n}{Ground}\PYG{o}{.}\PYG{n}{angle} \PYG{o}{=} \PYG{n}{sin\PYGZus{}angle\PYGZus{}func}\PYG{p}{(}\PYG{n+nb+bp}{self}\PYG{o}{.}\PYG{n}{Ground}\PYG{o}{.}\PYG{n}{x}\PYG{p}{,} \PYG{n+nb+bp}{self}\PYG{o}{.}\PYG{n}{amp}\PYG{p}{,} \PYG{n+nb+bp}{self}\PYG{o}{.}\PYG{n}{period}\PYG{p}{)}
            \PYG{n+nb+bp}{self}\PYG{o}{.}\PYG{n}{Ground}\PYG{o}{.}\PYG{n}{linex}\PYG{p}{,}\PYG{n+nb+bp}{self}\PYG{o}{.}\PYG{n}{Ground}\PYG{o}{.}\PYG{n}{liney} \PYG{o}{=} \PYG{n}{sin\PYGZus{}func}\PYG{p}{(}\PYG{n+nb+bp}{self}\PYG{o}{.}\PYG{n}{Ground}\PYG{o}{.}\PYG{n}{x}\PYG{p}{,}\PYG{n+nb+bp}{self}\PYG{o}{.}\PYG{n}{Ground}\PYG{o}{.}\PYG{n}{y}\PYG{p}{,} \PYG{n+nb+bp}{self}\PYG{o}{.}\PYG{n}{amp}\PYG{p}{,} \PYG{n+nb+bp}{self}\PYG{o}{.}\PYG{n}{period}\PYG{p}{)}
        \PYG{k}{elif} \PYG{n+nb+bp}{self}\PYG{o}{.}\PYG{n}{linetype}\PYG{o}{==}\PYG{l+s+s1}{\PYGZsq{}}\PYG{l+s+s1}{turn}\PYG{l+s+s1}{\PYGZsq{}}\PYG{p}{:}
            \PYG{n+nb+bp}{self}\PYG{o}{.}\PYG{n}{Ground}\PYG{o}{.}\PYG{n}{angle} \PYG{o}{=} \PYG{n}{turn\PYGZus{}angle\PYGZus{}func}\PYG{p}{(}\PYG{n+nb+bp}{self}\PYG{o}{.}\PYG{n}{Ground}\PYG{o}{.}\PYG{n}{x}\PYG{p}{,} \PYG{n+nb+bp}{self}\PYG{o}{.}\PYG{n}{radius}\PYG{p}{,} \PYG{n+nb+bp}{self}\PYG{o}{.}\PYG{n}{start}\PYG{p}{)}
            \PYG{n+nb+bp}{self}\PYG{o}{.}\PYG{n}{Ground}\PYG{o}{.}\PYG{n}{linex}\PYG{p}{,} \PYG{n+nb+bp}{self}\PYG{o}{.}\PYG{n}{Ground}\PYG{o}{.}\PYG{n}{liney} \PYG{o}{=} \PYG{n}{turn\PYGZus{}func}\PYG{p}{(}\PYG{n+nb+bp}{self}\PYG{o}{.}\PYG{n}{Ground}\PYG{o}{.}\PYG{n}{x}\PYG{p}{,} \PYG{n+nb+bp}{self}\PYG{o}{.}\PYG{n}{Ground}\PYG{o}{.}\PYG{n}{y}\PYG{p}{,} \PYG{n+nb+bp}{self}\PYG{o}{.}\PYG{n}{radius}\PYG{p}{,} \PYG{n+nb+bp}{self}\PYG{o}{.}\PYG{n}{start}\PYG{p}{)}

\PYG{k}{def} \PYG{n+nf}{sin\PYGZus{}func}\PYG{p}{(}\PYG{n}{x}\PYG{p}{,}\PYG{n}{y}\PYG{p}{,} \PYG{n}{amp}\PYG{p}{,} \PYG{n}{period}\PYG{p}{)}\PYG{p}{:}
    \PYG{k}{return} \PYG{n}{x}\PYG{p}{,} \PYG{n}{amp} \PYG{o}{*} \PYG{n}{np}\PYG{o}{.}\PYG{n}{sin}\PYG{p}{(}\PYG{n}{period}\PYG{o}{*}\PYG{n}{x}\PYG{p}{)}
\PYG{k}{def} \PYG{n+nf}{sin\PYGZus{}angle\PYGZus{}func}\PYG{p}{(}\PYG{n}{x}\PYG{p}{,} \PYG{n}{amp}\PYG{p}{,} \PYG{n}{period}\PYG{p}{)}\PYG{p}{:}
    \PYG{k}{return} \PYG{n}{amp} \PYG{o}{*} \PYG{n}{period} \PYG{o}{*} \PYG{n}{np}\PYG{o}{.}\PYG{n}{cos}\PYG{p}{(}\PYG{n}{period}\PYG{o}{*}\PYG{n}{x}\PYG{p}{)}\PYG{o}{*}\PYG{l+m+mi}{180}\PYG{o}{/}\PYG{n}{np}\PYG{o}{.}\PYG{n}{pi}

\PYG{k}{def} \PYG{n+nf}{turn\PYGZus{}func}\PYG{p}{(}\PYG{n}{x}\PYG{p}{,}\PYG{n}{y}\PYG{p}{,} \PYG{n}{radius}\PYG{p}{,}\PYG{n}{start}\PYG{p}{)}\PYG{p}{:}
    \PYG{k}{if}   \PYG{n}{x} \PYG{o}{\PYGZgt{}}\PYG{o}{=} \PYG{n}{start}\PYG{o}{+}\PYG{n}{radius}\PYG{p}{:} \PYG{k}{return} \PYG{n}{start}\PYG{o}{+}\PYG{n}{radius}\PYG{p}{,} \PYG{n}{y}
    \PYG{k}{elif} \PYG{n}{y} \PYG{o}{\PYGZgt{}}\PYG{o}{=} \PYG{n}{radius}\PYG{p}{:}       \PYG{k}{return} \PYG{n}{start}\PYG{o}{+}\PYG{n}{radius}\PYG{p}{,} \PYG{n}{y}
    \PYG{k}{elif} \PYG{n}{x} \PYG{o}{\PYGZgt{}}\PYG{o}{=} \PYG{n}{start}\PYG{p}{:}        \PYG{k}{return} \PYG{n}{x}\PYG{p}{,} \PYG{n}{radius} \PYG{o}{\PYGZhy{}} \PYG{n}{np}\PYG{o}{.}\PYG{n}{sqrt}\PYG{p}{(}\PYG{n}{radius}\PYG{o}{*}\PYG{o}{*}\PYG{l+m+mi}{2} \PYG{o}{\PYGZhy{}} \PYG{p}{(}\PYG{n}{x}\PYG{o}{\PYGZhy{}}\PYG{n}{start}\PYG{p}{)}\PYG{o}{*}\PYG{o}{*}\PYG{l+m+mi}{2}\PYG{p}{)}
    \PYG{k}{elif} \PYG{n}{x} \PYG{o}{\PYGZlt{}} \PYG{n}{start}\PYG{p}{:}         \PYG{k}{return} \PYG{n}{x}\PYG{p}{,} \PYG{l+m+mf}{0.0}
\PYG{k}{def} \PYG{n+nf}{turn\PYGZus{}angle\PYGZus{}func}\PYG{p}{(}\PYG{n}{x}\PYG{p}{,} \PYG{n}{radius}\PYG{p}{,} \PYG{n}{start}\PYG{p}{)}\PYG{p}{:}
    \PYG{k}{if}   \PYG{n}{x} \PYG{o}{\PYGZgt{}}\PYG{o}{=} \PYG{n}{start}\PYG{o}{+}\PYG{n}{radius}\PYG{p}{:} \PYG{k}{return} \PYG{l+m+mf}{90.0}
    \PYG{k}{elif} \PYG{n}{x} \PYG{o}{\PYGZgt{}}\PYG{o}{=} \PYG{n}{start}\PYG{p}{:}        \PYG{k}{return} \PYG{l+m+mf}{90.0} \PYG{o}{\PYGZhy{}} \PYG{n}{np}\PYG{o}{.}\PYG{n}{arccos}\PYG{p}{(}\PYG{p}{(}\PYG{p}{(}\PYG{n}{x}\PYG{o}{\PYGZhy{}}\PYG{n}{start}\PYG{p}{)}\PYG{o}{/}\PYG{n}{radius}\PYG{p}{)}\PYG{p}{)}\PYG{o}{*}\PYG{l+m+mi}{180}\PYG{o}{/}\PYG{n}{np}\PYG{o}{.}\PYG{n}{pi}  \PYG{c+c1}{\PYGZsh{}np.arctan((x\PYGZhy{}start)/(radius**2\PYGZhy{}(start\PYGZhy{}x)**2))*180/np.pi}
    \PYG{k}{elif} \PYG{n}{x}\PYG{o}{\PYGZlt{}}\PYG{n}{start}\PYG{p}{:}           \PYG{k}{return} \PYG{l+m+mf}{0.0}

\PYG{k}{class} \PYG{n+nc}{Rover}\PYG{p}{(}\PYG{n}{Model}\PYG{p}{)}\PYG{p}{:}
    \PYG{k}{def} \PYG{n+nf+fm}{\PYGZus{}\PYGZus{}init\PYGZus{}\PYGZus{}}\PYG{p}{(}\PYG{n+nb+bp}{self}\PYG{p}{,} \PYG{n}{params}\PYG{o}{=}\PYG{n}{gen\PYGZus{}params}\PYG{p}{(}\PYG{l+s+s1}{\PYGZsq{}}\PYG{l+s+s1}{turn}\PYG{l+s+s1}{\PYGZsq{}}\PYG{p}{)}\PYG{p}{,}\PYGZbs{}
                 \PYG{n}{modelparams}\PYG{o}{=}\PYG{p}{\PYGZob{}}\PYG{l+s+s1}{\PYGZsq{}}\PYG{l+s+s1}{times}\PYG{l+s+s1}{\PYGZsq{}}\PYG{p}{:}\PYG{p}{[}\PYG{l+m+mi}{0}\PYG{p}{,}\PYG{l+m+mi}{60}\PYG{p}{]}\PYG{p}{,} \PYG{l+s+s1}{\PYGZsq{}}\PYG{l+s+s1}{tstep}\PYG{l+s+s1}{\PYGZsq{}}\PYG{p}{:}\PYG{l+m+mi}{1}\PYG{p}{,} \PYG{l+s+s1}{\PYGZsq{}}\PYG{l+s+s1}{phases}\PYG{l+s+s1}{\PYGZsq{}}\PYG{p}{:}\PYG{p}{\PYGZob{}}\PYG{l+s+s1}{\PYGZsq{}}\PYG{l+s+s1}{start}\PYG{l+s+s1}{\PYGZsq{}}\PYG{p}{:}\PYG{p}{[}\PYG{l+m+mi}{1}\PYG{p}{,}\PYG{l+m+mi}{30}\PYG{p}{]}\PYG{p}{,} \PYG{l+s+s1}{\PYGZsq{}}\PYG{l+s+s1}{end}\PYG{l+s+s1}{\PYGZsq{}}\PYG{p}{:}\PYG{p}{[}\PYG{l+m+mi}{31}\PYG{p}{,} \PYG{l+m+mi}{60}\PYG{p}{]}\PYG{p}{\PYGZcb{}}\PYG{p}{\PYGZcb{}}\PYG{p}{,}\PYGZbs{}
                     \PYG{n}{valparams}\PYG{o}{=}\PYG{p}{\PYGZob{}}\PYG{p}{\PYGZcb{}}\PYG{p}{)}\PYG{p}{:}
        \PYG{n+nb}{super}\PYG{p}{(}\PYG{p}{)}\PYG{o}{.}\PYG{n+nf+fm}{\PYGZus{}\PYGZus{}init\PYGZus{}\PYGZus{}}\PYG{p}{(}\PYG{n}{params}\PYG{p}{,} \PYG{n}{modelparams}\PYG{p}{,} \PYG{n}{valparams}\PYG{p}{)}

        \PYG{n+nb+bp}{self}\PYG{o}{.}\PYG{n}{add\PYGZus{}flow}\PYG{p}{(}\PYG{l+s+s1}{\PYGZsq{}}\PYG{l+s+s1}{Ground}\PYG{l+s+s1}{\PYGZsq{}}\PYG{p}{,} \PYG{p}{\PYGZob{}}\PYG{l+s+s1}{\PYGZsq{}}\PYG{l+s+s1}{x}\PYG{l+s+s1}{\PYGZsq{}}\PYG{p}{:}\PYG{l+m+mf}{0.0}\PYG{p}{,}\PYG{l+s+s1}{\PYGZsq{}}\PYG{l+s+s1}{y}\PYG{l+s+s1}{\PYGZsq{}}\PYG{p}{:}\PYG{l+m+mf}{0.0}\PYG{p}{,}\PYG{l+s+s1}{\PYGZsq{}}\PYG{l+s+s1}{liney}\PYG{l+s+s1}{\PYGZsq{}}\PYG{p}{:}\PYG{l+m+mf}{0.0}\PYG{p}{,}\PYG{l+s+s1}{\PYGZsq{}}\PYG{l+s+s1}{linex}\PYG{l+s+s1}{\PYGZsq{}}\PYG{p}{:}\PYG{l+m+mf}{0.0}\PYG{p}{,} \PYG{l+s+s1}{\PYGZsq{}}\PYG{l+s+s1}{vel}\PYG{l+s+s1}{\PYGZsq{}}\PYG{p}{:}\PYG{l+m+mf}{0.0}\PYG{p}{,} \PYG{l+s+s1}{\PYGZsq{}}\PYG{l+s+s1}{line}\PYG{l+s+s1}{\PYGZsq{}}\PYG{p}{:}\PYG{l+m+mf}{0.0}\PYG{p}{,} \PYG{l+s+s1}{\PYGZsq{}}\PYG{l+s+s1}{angle}\PYG{l+s+s1}{\PYGZsq{}}\PYG{p}{:}\PYG{n}{params}\PYG{p}{[}\PYG{l+s+s1}{\PYGZsq{}}\PYG{l+s+s1}{initangle}\PYG{l+s+s1}{\PYGZsq{}}\PYG{p}{]}\PYG{p}{,} \PYG{l+s+s1}{\PYGZsq{}}\PYG{l+s+s1}{ang}\PYG{l+s+s1}{\PYGZsq{}}\PYG{p}{:}\PYG{l+m+mf}{0.0}\PYG{p}{\PYGZcb{}}\PYG{p}{)}
        \PYG{n+nb+bp}{self}\PYG{o}{.}\PYG{n}{add\PYGZus{}flow}\PYG{p}{(}\PYG{l+s+s1}{\PYGZsq{}}\PYG{l+s+s1}{Pos\PYGZus{}Signal}\PYG{l+s+s1}{\PYGZsq{}}\PYG{p}{,} \PYG{p}{\PYGZob{}}\PYG{l+s+s1}{\PYGZsq{}}\PYG{l+s+s1}{x}\PYG{l+s+s1}{\PYGZsq{}}\PYG{p}{:}\PYG{l+m+mf}{0.0}\PYG{p}{,}\PYG{l+s+s1}{\PYGZsq{}}\PYG{l+s+s1}{y}\PYG{l+s+s1}{\PYGZsq{}}\PYG{p}{:}\PYG{l+m+mf}{0.0}\PYG{p}{,}\PYG{l+s+s1}{\PYGZsq{}}\PYG{l+s+s1}{liney}\PYG{l+s+s1}{\PYGZsq{}}\PYG{p}{:}\PYG{l+m+mf}{0.0}\PYG{p}{,}\PYG{l+s+s1}{\PYGZsq{}}\PYG{l+s+s1}{linex}\PYG{l+s+s1}{\PYGZsq{}}\PYG{p}{:}\PYG{l+m+mf}{0.0}\PYG{p}{,} \PYG{l+s+s1}{\PYGZsq{}}\PYG{l+s+s1}{heading}\PYG{l+s+s1}{\PYGZsq{}}\PYG{p}{:}\PYG{l+m+mf}{0.0}\PYG{p}{,} \PYG{l+s+s1}{\PYGZsq{}}\PYG{l+s+s1}{vel}\PYG{l+s+s1}{\PYGZsq{}}\PYG{p}{:}\PYG{l+m+mf}{0.0}\PYG{p}{,} \PYG{l+s+s1}{\PYGZsq{}}\PYG{l+s+s1}{line}\PYG{l+s+s1}{\PYGZsq{}}\PYG{p}{:}\PYG{l+m+mi}{0}\PYG{p}{,} \PYG{l+s+s1}{\PYGZsq{}}\PYG{l+s+s1}{angle}\PYG{l+s+s1}{\PYGZsq{}}\PYG{p}{:}\PYG{l+m+mf}{0.0}\PYG{p}{\PYGZcb{}}\PYG{p}{)}
        \PYG{n+nb+bp}{self}\PYG{o}{.}\PYG{n}{add\PYGZus{}flow}\PYG{p}{(}\PYG{l+s+s1}{\PYGZsq{}}\PYG{l+s+s1}{EE\PYGZus{}12}\PYG{l+s+s1}{\PYGZsq{}}\PYG{p}{,} \PYG{p}{\PYGZob{}}\PYG{l+s+s1}{\PYGZsq{}}\PYG{l+s+s1}{v}\PYG{l+s+s1}{\PYGZsq{}}\PYG{p}{:}\PYG{l+m+mf}{0.0}\PYG{p}{,} \PYG{l+s+s1}{\PYGZsq{}}\PYG{l+s+s1}{a}\PYG{l+s+s1}{\PYGZsq{}}\PYG{p}{:}\PYG{l+m+mf}{0.0}\PYG{p}{\PYGZcb{}}\PYG{p}{)}
        \PYG{n+nb+bp}{self}\PYG{o}{.}\PYG{n}{add\PYGZus{}flow}\PYG{p}{(}\PYG{l+s+s1}{\PYGZsq{}}\PYG{l+s+s1}{EE\PYGZus{}5}\PYG{l+s+s1}{\PYGZsq{}}\PYG{p}{,} \PYG{p}{\PYGZob{}}\PYG{l+s+s1}{\PYGZsq{}}\PYG{l+s+s1}{v}\PYG{l+s+s1}{\PYGZsq{}}\PYG{p}{:}\PYG{l+m+mf}{0.0}\PYG{p}{,} \PYG{l+s+s1}{\PYGZsq{}}\PYG{l+s+s1}{a}\PYG{l+s+s1}{\PYGZsq{}}\PYG{p}{:}\PYG{l+m+mf}{0.0}\PYG{p}{\PYGZcb{}}\PYG{p}{)}
        \PYG{n+nb+bp}{self}\PYG{o}{.}\PYG{n}{add\PYGZus{}flow}\PYG{p}{(}\PYG{l+s+s1}{\PYGZsq{}}\PYG{l+s+s1}{EE\PYGZus{}15}\PYG{l+s+s1}{\PYGZsq{}}\PYG{p}{,} \PYG{p}{\PYGZob{}}\PYG{l+s+s1}{\PYGZsq{}}\PYG{l+s+s1}{v}\PYG{l+s+s1}{\PYGZsq{}}\PYG{p}{:}\PYG{l+m+mf}{0.0}\PYG{p}{,} \PYG{l+s+s1}{\PYGZsq{}}\PYG{l+s+s1}{a}\PYG{l+s+s1}{\PYGZsq{}}\PYG{p}{:}\PYG{l+m+mf}{0.0}\PYG{p}{\PYGZcb{}}\PYG{p}{)}
        \PYG{n+nb+bp}{self}\PYG{o}{.}\PYG{n}{add\PYGZus{}flow}\PYG{p}{(}\PYG{l+s+s1}{\PYGZsq{}}\PYG{l+s+s1}{Video}\PYG{l+s+s1}{\PYGZsq{}}\PYG{p}{,} \PYG{p}{\PYGZob{}}\PYG{l+s+s1}{\PYGZsq{}}\PYG{l+s+s1}{liney}\PYG{l+s+s1}{\PYGZsq{}}\PYG{p}{:}\PYG{l+m+mf}{0.0}\PYG{p}{,}\PYG{l+s+s1}{\PYGZsq{}}\PYG{l+s+s1}{linex}\PYG{l+s+s1}{\PYGZsq{}}\PYG{p}{:}\PYG{l+m+mf}{0.0}\PYG{p}{,} \PYG{l+s+s1}{\PYGZsq{}}\PYG{l+s+s1}{angle}\PYG{l+s+s1}{\PYGZsq{}}\PYG{p}{:}\PYG{n}{params}\PYG{p}{[}\PYG{l+s+s1}{\PYGZsq{}}\PYG{l+s+s1}{initangle}\PYG{l+s+s1}{\PYGZsq{}}\PYG{p}{]}\PYG{p}{,} \PYG{l+s+s1}{\PYGZsq{}}\PYG{l+s+s1}{quality}\PYG{l+s+s1}{\PYGZsq{}}\PYG{p}{:}\PYG{l+m+mi}{1}\PYG{p}{\PYGZcb{}}\PYG{p}{)}
        \PYG{n+nb+bp}{self}\PYG{o}{.}\PYG{n}{add\PYGZus{}flow}\PYG{p}{(}\PYG{l+s+s1}{\PYGZsq{}}\PYG{l+s+s1}{AvionicsControl}\PYG{l+s+s1}{\PYGZsq{}}\PYG{p}{,} \PYG{p}{\PYGZob{}}\PYG{l+s+s1}{\PYGZsq{}}\PYG{l+s+s1}{rpower}\PYG{l+s+s1}{\PYGZsq{}}\PYG{p}{:}\PYG{l+m+mf}{0.0}\PYG{p}{,} \PYG{l+s+s1}{\PYGZsq{}}\PYG{l+s+s1}{lpower}\PYG{l+s+s1}{\PYGZsq{}}\PYG{p}{:}\PYG{l+m+mf}{0.0}\PYG{p}{,} \PYG{l+s+s1}{\PYGZsq{}}\PYG{l+s+s1}{completed}\PYG{l+s+s1}{\PYGZsq{}}\PYG{p}{:}\PYG{l+m+mi}{0}\PYG{p}{\PYGZcb{}}\PYG{p}{)}
        \PYG{n+nb+bp}{self}\PYG{o}{.}\PYG{n}{add\PYGZus{}flow}\PYG{p}{(}\PYG{l+s+s1}{\PYGZsq{}}\PYG{l+s+s1}{MotorControl}\PYG{l+s+s1}{\PYGZsq{}}\PYG{p}{,} \PYG{p}{\PYGZob{}}\PYG{l+s+s1}{\PYGZsq{}}\PYG{l+s+s1}{rpower}\PYG{l+s+s1}{\PYGZsq{}}\PYG{p}{:}\PYG{l+m+mf}{0.0}\PYG{p}{,} \PYG{l+s+s1}{\PYGZsq{}}\PYG{l+s+s1}{lpower}\PYG{l+s+s1}{\PYGZsq{}}\PYG{p}{:}\PYG{l+m+mf}{0.0}\PYG{p}{\PYGZcb{}}\PYG{p}{)}
        \PYG{n+nb+bp}{self}\PYG{o}{.}\PYG{n}{add\PYGZus{}flow}\PYG{p}{(}\PYG{l+s+s1}{\PYGZsq{}}\PYG{l+s+s1}{Control}\PYG{l+s+s1}{\PYGZsq{}}\PYG{p}{,} \PYG{p}{\PYGZob{}}\PYG{l+s+s1}{\PYGZsq{}}\PYG{l+s+s1}{power}\PYG{l+s+s1}{\PYGZsq{}}\PYG{p}{:}\PYG{l+m+mf}{0.0}\PYG{p}{\PYGZcb{}}\PYG{p}{)}
        \PYG{n+nb+bp}{self}\PYG{o}{.}\PYG{n}{add\PYGZus{}flow}\PYG{p}{(}\PYG{l+s+s1}{\PYGZsq{}}\PYG{l+s+s1}{Comms}\PYG{l+s+s1}{\PYGZsq{}}\PYG{p}{,} \PYG{p}{\PYGZob{}}\PYG{l+s+s1}{\PYGZsq{}}\PYG{l+s+s1}{x}\PYG{l+s+s1}{\PYGZsq{}}\PYG{p}{:}\PYG{l+m+mf}{0.0}\PYG{p}{,}\PYG{l+s+s1}{\PYGZsq{}}\PYG{l+s+s1}{y}\PYG{l+s+s1}{\PYGZsq{}}\PYG{p}{:}\PYG{l+m+mf}{0.0}\PYG{p}{,} \PYG{l+s+s1}{\PYGZsq{}}\PYG{l+s+s1}{vel}\PYG{l+s+s1}{\PYGZsq{}}\PYG{p}{:}\PYG{l+m+mf}{0.0}\PYG{p}{,} \PYG{l+s+s1}{\PYGZsq{}}\PYG{l+s+s1}{heading}\PYG{l+s+s1}{\PYGZsq{}}\PYG{p}{:}\PYG{l+m+mf}{0.0}\PYG{p}{\PYGZcb{}}\PYG{p}{)}
        \PYG{n+nb+bp}{self}\PYG{o}{.}\PYG{n}{add\PYGZus{}flow}\PYG{p}{(}\PYG{l+s+s1}{\PYGZsq{}}\PYG{l+s+s1}{OverrideComms}\PYG{l+s+s1}{\PYGZsq{}}\PYG{p}{,} \PYG{p}{\PYGZob{}}\PYG{l+s+s1}{\PYGZsq{}}\PYG{l+s+s1}{rpower}\PYG{l+s+s1}{\PYGZsq{}}\PYG{p}{:}\PYG{l+m+mf}{0.0}\PYG{p}{,} \PYG{l+s+s1}{\PYGZsq{}}\PYG{l+s+s1}{lpower}\PYG{l+s+s1}{\PYGZsq{}}\PYG{p}{:}\PYG{l+m+mf}{0.0}\PYG{p}{,} \PYG{l+s+s1}{\PYGZsq{}}\PYG{l+s+s1}{active}\PYG{l+s+s1}{\PYGZsq{}}\PYG{p}{:}\PYG{l+m+mi}{0}\PYG{p}{\PYGZcb{}} \PYG{p}{)}
        \PYG{c+c1}{\PYGZsh{}self.add\PYGZus{}flow(\PYGZsq{}Example\PYGZus{}Disconnect\PYGZsq{})}

        \PYG{n+nb+bp}{self}\PYG{o}{.}\PYG{n}{add\PYGZus{}fxn}\PYG{p}{(}\PYG{l+s+s2}{\PYGZdq{}}\PYG{l+s+s2}{Power}\PYG{l+s+s2}{\PYGZdq{}}\PYG{p}{,} \PYG{p}{[}\PYG{l+s+s2}{\PYGZdq{}}\PYG{l+s+s2}{EE\PYGZus{}15}\PYG{l+s+s2}{\PYGZdq{}}\PYG{p}{,}\PYG{l+s+s2}{\PYGZdq{}}\PYG{l+s+s2}{EE\PYGZus{}5}\PYG{l+s+s2}{\PYGZdq{}}\PYG{p}{,}\PYG{l+s+s1}{\PYGZsq{}}\PYG{l+s+s1}{EE\PYGZus{}12}\PYG{l+s+s1}{\PYGZsq{}}\PYG{p}{,} \PYG{l+s+s2}{\PYGZdq{}}\PYG{l+s+s2}{Control}\PYG{l+s+s2}{\PYGZdq{}}\PYG{p}{,} \PYG{l+s+s2}{\PYGZdq{}}\PYG{l+s+s2}{AvionicsControl}\PYG{l+s+s2}{\PYGZdq{}}\PYG{p}{]}\PYG{p}{,} \PYG{n}{Power}\PYG{p}{)}
        \PYG{n+nb+bp}{self}\PYG{o}{.}\PYG{n}{add\PYGZus{}fxn}\PYG{p}{(}\PYG{l+s+s2}{\PYGZdq{}}\PYG{l+s+s2}{Operator}\PYG{l+s+s2}{\PYGZdq{}}\PYG{p}{,} \PYG{p}{[}\PYG{l+s+s2}{\PYGZdq{}}\PYG{l+s+s2}{Comms}\PYG{l+s+s2}{\PYGZdq{}}\PYG{p}{,} \PYG{l+s+s2}{\PYGZdq{}}\PYG{l+s+s2}{OverrideComms}\PYG{l+s+s2}{\PYGZdq{}}\PYG{p}{,} \PYG{l+s+s2}{\PYGZdq{}}\PYG{l+s+s2}{Pos\PYGZus{}Signal}\PYG{l+s+s2}{\PYGZdq{}}\PYG{p}{,} \PYG{l+s+s2}{\PYGZdq{}}\PYG{l+s+s2}{Control}\PYG{l+s+s2}{\PYGZdq{}}\PYG{p}{]}\PYG{p}{,} \PYG{n}{Operator}\PYG{p}{)}
        \PYG{n+nb+bp}{self}\PYG{o}{.}\PYG{n}{add\PYGZus{}fxn}\PYG{p}{(}\PYG{l+s+s2}{\PYGZdq{}}\PYG{l+s+s2}{Communications}\PYG{l+s+s2}{\PYGZdq{}}\PYG{p}{,} \PYG{p}{[}\PYG{l+s+s2}{\PYGZdq{}}\PYG{l+s+s2}{Comms}\PYG{l+s+s2}{\PYGZdq{}}\PYG{p}{,} \PYG{l+s+s2}{\PYGZdq{}}\PYG{l+s+s2}{EE\PYGZus{}12}\PYG{l+s+s2}{\PYGZdq{}}\PYG{p}{,} \PYG{l+s+s1}{\PYGZsq{}}\PYG{l+s+s1}{Pos\PYGZus{}Signal}\PYG{l+s+s1}{\PYGZsq{}}\PYG{p}{]}\PYG{p}{,} \PYG{n}{Communications}\PYG{p}{)}
        \PYG{n+nb+bp}{self}\PYG{o}{.}\PYG{n}{add\PYGZus{}fxn}\PYG{p}{(}\PYG{l+s+s2}{\PYGZdq{}}\PYG{l+s+s2}{Perception}\PYG{l+s+s2}{\PYGZdq{}}\PYG{p}{,} \PYG{p}{[}\PYG{l+s+s2}{\PYGZdq{}}\PYG{l+s+s2}{Ground}\PYG{l+s+s2}{\PYGZdq{}}\PYG{p}{,} \PYG{l+s+s2}{\PYGZdq{}}\PYG{l+s+s2}{EE\PYGZus{}12}\PYG{l+s+s2}{\PYGZdq{}}\PYG{p}{,} \PYG{l+s+s2}{\PYGZdq{}}\PYG{l+s+s2}{Video}\PYG{l+s+s2}{\PYGZdq{}}\PYG{p}{]}\PYG{p}{,} \PYG{n}{Perception}\PYG{p}{)}
        \PYG{n+nb+bp}{self}\PYG{o}{.}\PYG{n}{add\PYGZus{}fxn}\PYG{p}{(}\PYG{l+s+s2}{\PYGZdq{}}\PYG{l+s+s2}{Avionics}\PYG{l+s+s2}{\PYGZdq{}}\PYG{p}{,}\PYG{p}{[}\PYG{l+s+s2}{\PYGZdq{}}\PYG{l+s+s2}{Video}\PYG{l+s+s2}{\PYGZdq{}}\PYG{p}{,}\PYG{l+s+s2}{\PYGZdq{}}\PYG{l+s+s2}{Comms}\PYG{l+s+s2}{\PYGZdq{}}\PYG{p}{,} \PYG{l+s+s2}{\PYGZdq{}}\PYG{l+s+s2}{EE\PYGZus{}5}\PYG{l+s+s2}{\PYGZdq{}}\PYG{p}{,}\PYG{l+s+s1}{\PYGZsq{}}\PYG{l+s+s1}{Pos\PYGZus{}Signal}\PYG{l+s+s1}{\PYGZsq{}}\PYG{p}{,}\PYG{l+s+s2}{\PYGZdq{}}\PYG{l+s+s2}{Ground}\PYG{l+s+s2}{\PYGZdq{}}\PYG{p}{,} \PYG{l+s+s2}{\PYGZdq{}}\PYG{l+s+s2}{AvionicsControl}\PYG{l+s+s2}{\PYGZdq{}}\PYG{p}{]}\PYG{p}{,} \PYG{n}{fclass}\PYG{o}{=}\PYG{n}{Avionics}\PYG{p}{,} \PYG{n}{fparams}\PYG{o}{=}\PYG{n}{params}\PYG{p}{)}
        \PYG{n+nb+bp}{self}\PYG{o}{.}\PYG{n}{add\PYGZus{}fxn}\PYG{p}{(}\PYG{l+s+s2}{\PYGZdq{}}\PYG{l+s+s2}{Override}\PYG{l+s+s2}{\PYGZdq{}}\PYG{p}{,} \PYG{p}{[}\PYG{l+s+s2}{\PYGZdq{}}\PYG{l+s+s2}{OverrideComms}\PYG{l+s+s2}{\PYGZdq{}}\PYG{p}{,} \PYG{l+s+s2}{\PYGZdq{}}\PYG{l+s+s2}{EE\PYGZus{}5}\PYG{l+s+s2}{\PYGZdq{}}\PYG{p}{,} \PYG{l+s+s1}{\PYGZsq{}}\PYG{l+s+s1}{MotorControl}\PYG{l+s+s1}{\PYGZsq{}}\PYG{p}{,}\PYG{l+s+s1}{\PYGZsq{}}\PYG{l+s+s1}{AvionicsControl}\PYG{l+s+s1}{\PYGZsq{}}\PYG{p}{]}\PYG{p}{,} \PYG{n}{Override}\PYG{p}{)}
        \PYG{n+nb+bp}{self}\PYG{o}{.}\PYG{n}{add\PYGZus{}fxn}\PYG{p}{(}\PYG{l+s+s2}{\PYGZdq{}}\PYG{l+s+s2}{Drive}\PYG{l+s+s2}{\PYGZdq{}}\PYG{p}{,} \PYG{p}{[}\PYG{l+s+s2}{\PYGZdq{}}\PYG{l+s+s2}{Ground}\PYG{l+s+s2}{\PYGZdq{}}\PYG{p}{,}\PYG{l+s+s2}{\PYGZdq{}}\PYG{l+s+s2}{EE\PYGZus{}15}\PYG{l+s+s2}{\PYGZdq{}}\PYG{p}{,}\PYG{l+s+s2}{\PYGZdq{}}\PYG{l+s+s2}{EE\PYGZus{}5}\PYG{l+s+s2}{\PYGZdq{}}\PYG{p}{,} \PYG{l+s+s2}{\PYGZdq{}}\PYG{l+s+s2}{MotorControl}\PYG{l+s+s2}{\PYGZdq{}}\PYG{p}{]}\PYG{p}{,} \PYG{n}{fclass} \PYG{o}{=} \PYG{n}{Drive}\PYG{p}{)}
        \PYG{n+nb+bp}{self}\PYG{o}{.}\PYG{n}{add\PYGZus{}fxn}\PYG{p}{(}\PYG{l+s+s2}{\PYGZdq{}}\PYG{l+s+s2}{Environment}\PYG{l+s+s2}{\PYGZdq{}}\PYG{p}{,} \PYG{p}{[}\PYG{l+s+s1}{\PYGZsq{}}\PYG{l+s+s1}{Ground}\PYG{l+s+s1}{\PYGZsq{}}\PYG{p}{]}\PYG{p}{,} \PYG{n}{Environment}\PYG{p}{,} \PYG{n}{fparams} \PYG{o}{=} \PYG{n}{params}\PYG{p}{)}

        \PYG{n}{pos\PYGZus{}bip} \PYG{o}{=} \PYG{p}{\PYGZob{}}\PYG{l+s+s1}{\PYGZsq{}}\PYG{l+s+s1}{Power}\PYG{l+s+s1}{\PYGZsq{}}\PYG{p}{:} \PYG{p}{[}\PYG{o}{\PYGZhy{}}\PYG{l+m+mf}{0.684772948203272}\PYG{p}{,} \PYG{o}{\PYGZhy{}}\PYG{l+m+mf}{0.2551613615446115}\PYG{p}{]}\PYG{p}{,}
                 \PYG{l+s+s1}{\PYGZsq{}}\PYG{l+s+s1}{Operator}\PYG{l+s+s1}{\PYGZsq{}}\PYG{p}{:} \PYG{p}{[}\PYG{o}{\PYGZhy{}}\PYG{l+m+mf}{0.798933011500376}\PYG{p}{,} \PYG{l+m+mf}{0.565156755693186}\PYG{p}{]}\PYG{p}{,}
                 \PYG{l+s+s1}{\PYGZsq{}}\PYG{l+s+s1}{Communications}\PYG{l+s+s1}{\PYGZsq{}}\PYG{p}{:} \PYG{p}{[}\PYG{o}{\PYGZhy{}}\PYG{l+m+mf}{0.5566050878414673}\PYG{p}{,} \PYG{l+m+mf}{0.14159180700630447}\PYG{p}{]}\PYG{p}{,}
                 \PYG{l+s+s1}{\PYGZsq{}}\PYG{l+s+s1}{Perception}\PYG{l+s+s1}{\PYGZsq{}}\PYG{p}{:} \PYG{p}{[}\PYG{l+m+mf}{0.996672509613648}\PYG{p}{,} \PYG{l+m+mf}{0.2507215448302319}\PYG{p}{]}\PYG{p}{,}
                 \PYG{l+s+s1}{\PYGZsq{}}\PYG{l+s+s1}{Avionics}\PYG{l+s+s1}{\PYGZsq{}}\PYG{p}{:} \PYG{p}{[}\PYG{l+m+mf}{0.28027473355741117}\PYG{p}{,} \PYG{l+m+mf}{0.47255264233968597}\PYG{p}{]}\PYG{p}{,}
                 \PYG{l+s+s1}{\PYGZsq{}}\PYG{l+s+s1}{Override}\PYG{l+s+s1}{\PYGZsq{}}\PYG{p}{:} \PYG{p}{[}\PYG{l+m+mf}{0.28987624783062627}\PYG{p}{,} \PYG{o}{\PYGZhy{}}\PYG{l+m+mf}{0.17144760874154652}\PYG{p}{]}\PYG{p}{,}
                 \PYG{l+s+s1}{\PYGZsq{}}\PYG{l+s+s1}{Drive}\PYG{l+s+s1}{\PYGZsq{}}\PYG{p}{:} \PYG{p}{[}\PYG{l+m+mf}{0.6671719569482308}\PYG{p}{,} \PYG{o}{\PYGZhy{}}\PYG{l+m+mf}{0.571646956655247}\PYG{p}{]}\PYG{p}{,}
                 \PYG{l+s+s1}{\PYGZsq{}}\PYG{l+s+s1}{Environment}\PYG{l+s+s1}{\PYGZsq{}}\PYG{p}{:} \PYG{p}{[}\PYG{l+m+mf}{1.1329643169383754}\PYG{p}{,} \PYG{o}{\PYGZhy{}}\PYG{l+m+mf}{0.6375225566564033}\PYG{p}{]}\PYG{p}{,}
                 \PYG{l+s+s1}{\PYGZsq{}}\PYG{l+s+s1}{Ground}\PYG{l+s+s1}{\PYGZsq{}}\PYG{p}{:} \PYG{p}{[}\PYG{l+m+mf}{1.108432946123935}\PYG{p}{,} \PYG{o}{\PYGZhy{}}\PYG{l+m+mf}{0.3228541151507237}\PYG{p}{]}\PYG{p}{,}
                 \PYG{l+s+s1}{\PYGZsq{}}\PYG{l+s+s1}{Pos\PYGZus{}Signal}\PYG{l+s+s1}{\PYGZsq{}}\PYG{p}{:} \PYG{p}{[}\PYG{o}{\PYGZhy{}}\PYG{l+m+mf}{0.256557435572734}\PYG{p}{,} \PYG{l+m+mf}{0.5411037985681082}\PYG{p}{]}\PYG{p}{,}
                 \PYG{l+s+s1}{\PYGZsq{}}\PYG{l+s+s1}{EE\PYGZus{}12}\PYG{l+s+s1}{\PYGZsq{}}\PYG{p}{:} \PYG{p}{[}\PYG{o}{\PYGZhy{}}\PYG{l+m+mf}{0.3676879520509888}\PYG{p}{,} \PYG{o}{\PYGZhy{}}\PYG{l+m+mf}{0.04754907961317867}\PYG{p}{]}\PYG{p}{,}
                 \PYG{l+s+s1}{\PYGZsq{}}\PYG{l+s+s1}{EE\PYGZus{}5}\PYG{l+s+s1}{\PYGZsq{}}\PYG{p}{:} \PYG{p}{[}\PYG{o}{\PYGZhy{}}\PYG{l+m+mf}{0.2181352416728437}\PYG{p}{,} \PYG{o}{\PYGZhy{}}\PYG{l+m+mf}{0.2015320865756482}\PYG{p}{]}\PYG{p}{,}
                 \PYG{l+s+s1}{\PYGZsq{}}\PYG{l+s+s1}{EE\PYGZus{}15}\PYG{l+s+s1}{\PYGZsq{}}\PYG{p}{:} \PYG{p}{[}\PYG{o}{\PYGZhy{}}\PYG{l+m+mf}{0.5352906801304353}\PYG{p}{,} \PYG{o}{\PYGZhy{}}\PYG{l+m+mf}{0.5288715575154177}\PYG{p}{]}\PYG{p}{,}
                 \PYG{l+s+s1}{\PYGZsq{}}\PYG{l+s+s1}{Video}\PYG{l+s+s1}{\PYGZsq{}}\PYG{p}{:} \PYG{p}{[}\PYG{l+m+mf}{0.6726175830840695}\PYG{p}{,} \PYG{l+m+mf}{0.396008366729458}\PYG{p}{]}\PYG{p}{,}
                 \PYG{l+s+s1}{\PYGZsq{}}\PYG{l+s+s1}{AvionicsControl}\PYG{l+s+s1}{\PYGZsq{}}\PYG{p}{:} \PYG{p}{[}\PYG{l+m+mf}{0.45997843863482324}\PYG{p}{,} \PYG{l+m+mf}{0.04522869632581994}\PYG{p}{]}\PYG{p}{,}
                 \PYG{l+s+s1}{\PYGZsq{}}\PYG{l+s+s1}{MotorControl}\PYG{l+s+s1}{\PYGZsq{}}\PYG{p}{:} \PYG{p}{[}\PYG{l+m+mf}{0.6350063940085445}\PYG{p}{,} \PYG{o}{\PYGZhy{}}\PYG{l+m+mf}{0.3013633829278297}\PYG{p}{]}\PYG{p}{,}
                 \PYG{l+s+s1}{\PYGZsq{}}\PYG{l+s+s1}{Control}\PYG{l+s+s1}{\PYGZsq{}}\PYG{p}{:} \PYG{p}{[}\PYG{o}{\PYGZhy{}}\PYG{l+m+mf}{0.9857988678463686}\PYG{p}{,} \PYG{l+m+mf}{0.07960895587242012}\PYG{p}{]}\PYG{p}{,}
                 \PYG{l+s+s1}{\PYGZsq{}}\PYG{l+s+s1}{Comms}\PYG{l+s+s1}{\PYGZsq{}}\PYG{p}{:} \PYG{p}{[}\PYG{o}{\PYGZhy{}}\PYG{l+m+mf}{0.642370284813957}\PYG{p}{,} \PYG{l+m+mf}{0.35285736707043763}\PYG{p}{]}\PYG{p}{,}
                 \PYG{l+s+s1}{\PYGZsq{}}\PYG{l+s+s1}{OverrideComms}\PYG{l+s+s1}{\PYGZsq{}}\PYG{p}{:} \PYG{p}{[}\PYG{o}{\PYGZhy{}}\PYG{l+m+mf}{0.14607433032593392}\PYG{p}{,} \PYG{l+m+mf}{0.2981956996230818}\PYG{p}{]}\PYG{p}{\PYGZcb{}}

        \PYG{n+nb+bp}{self}\PYG{o}{.}\PYG{n}{build\PYGZus{}model}\PYG{p}{(}\PYG{n}{bipartite\PYGZus{}pos} \PYG{o}{=} \PYG{n}{pos\PYGZus{}bip}\PYG{p}{)}
    \PYG{k}{def} \PYG{n+nf}{find\PYGZus{}classification}\PYG{p}{(}\PYG{n+nb+bp}{self}\PYG{p}{,}\PYG{n}{scen}\PYG{p}{,}\PYG{n}{mdlhist}\PYG{p}{)}\PYG{p}{:}
        \PYG{n}{modes}\PYG{p}{,} \PYG{n}{modeproperties} \PYG{o}{=} \PYG{n+nb+bp}{self}\PYG{o}{.}\PYG{n}{return\PYGZus{}faultmodes}\PYG{p}{(}\PYG{p}{)}
        \PYG{n}{classification} \PYG{o}{=} \PYG{n+nb}{str}\PYG{p}{(}\PYG{p}{)}
        \PYG{k}{if} \PYG{o+ow}{not} \PYG{n}{in\PYGZus{}area}\PYG{p}{(}\PYG{n+nb+bp}{self}\PYG{o}{.}\PYG{n}{flows}\PYG{p}{[}\PYG{l+s+s1}{\PYGZsq{}}\PYG{l+s+s1}{Ground}\PYG{l+s+s1}{\PYGZsq{}}\PYG{p}{]}\PYG{o}{.}\PYG{n}{x}\PYG{p}{,}\PYG{n+nb+bp}{self}\PYG{o}{.}\PYG{n}{flows}\PYG{p}{[}\PYG{l+s+s1}{\PYGZsq{}}\PYG{l+s+s1}{Ground}\PYG{l+s+s1}{\PYGZsq{}}\PYG{p}{]}\PYG{o}{.}\PYG{n}{y}\PYG{p}{,}\PYG{l+m+mi}{1}\PYG{p}{,}\PYG{n+nb+bp}{self}\PYG{o}{.}\PYG{n}{params}\PYG{p}{[}\PYG{l+s+s1}{\PYGZsq{}}\PYG{l+s+s1}{end}\PYG{l+s+s1}{\PYGZsq{}}\PYG{p}{]}\PYG{p}{[}\PYG{l+m+mi}{0}\PYG{p}{]}\PYG{p}{,}\PYG{n+nb+bp}{self}\PYG{o}{.}\PYG{n}{params}\PYG{p}{[}\PYG{l+s+s1}{\PYGZsq{}}\PYG{l+s+s1}{end}\PYG{l+s+s1}{\PYGZsq{}}\PYG{p}{]}\PYG{p}{[}\PYG{l+m+mi}{1}\PYG{p}{]}\PYG{p}{)}\PYG{p}{:}
                                \PYG{n}{classification} \PYG{o}{=} \PYG{l+s+s2}{\PYGZdq{}}\PYG{l+s+s2}{incomplete mission}\PYG{l+s+s2}{\PYGZdq{}}\PYG{p}{;} \PYG{n}{cost}\PYG{o}{=}\PYG{l+m+mi}{500}
        \PYG{k}{if} \PYG{o+ow}{not} \PYG{n}{classification}\PYG{p}{:}  \PYG{n}{classification} \PYG{o}{=} \PYG{l+s+s1}{\PYGZsq{}}\PYG{l+s+s1}{nominal mission}\PYG{l+s+s1}{\PYGZsq{}}\PYG{p}{;} \PYG{n}{cost}\PYG{o}{=}\PYG{l+m+mi}{0}
        \PYG{k}{if} \PYG{n+nb}{any}\PYG{p}{(}\PYG{n}{modes}\PYG{p}{)}\PYG{p}{:}          \PYG{n}{classification} \PYG{o}{=} \PYG{n}{classification} \PYG{o}{+}\PYG{l+s+s1}{\PYGZsq{}}\PYG{l+s+s1}{ faulty}\PYG{l+s+s1}{\PYGZsq{}}\PYG{p}{;} \PYG{n}{cost}\PYG{o}{+}\PYG{o}{=}\PYG{l+m+mi}{100}
        \PYG{k}{return} \PYG{p}{\PYGZob{}}\PYG{l+s+s1}{\PYGZsq{}}\PYG{l+s+s1}{rate}\PYG{l+s+s1}{\PYGZsq{}}\PYG{p}{:}\PYG{l+m+mi}{0}\PYG{p}{,}\PYG{l+s+s1}{\PYGZsq{}}\PYG{l+s+s1}{cost}\PYG{l+s+s1}{\PYGZsq{}}\PYG{p}{:}\PYG{n}{cost}\PYG{p}{,} \PYG{l+s+s1}{\PYGZsq{}}\PYG{l+s+s1}{prob}\PYG{l+s+s1}{\PYGZsq{}}\PYG{p}{:}\PYG{n}{scen}\PYG{p}{[}\PYG{l+s+s1}{\PYGZsq{}}\PYG{l+s+s1}{properties}\PYG{l+s+s1}{\PYGZsq{}}\PYG{p}{]}\PYG{o}{.}\PYG{n}{get}\PYG{p}{(}\PYG{l+s+s1}{\PYGZsq{}}\PYG{l+s+s1}{prob}\PYG{l+s+s1}{\PYGZsq{}}\PYG{p}{,}\PYG{l+m+mi}{1}\PYG{p}{)}\PYG{p}{,} \PYG{l+s+s1}{\PYGZsq{}}\PYG{l+s+s1}{expected cost}\PYG{l+s+s1}{\PYGZsq{}}\PYG{p}{:}\PYG{n}{scen}\PYG{p}{[}\PYG{l+s+s1}{\PYGZsq{}}\PYG{l+s+s1}{properties}\PYG{l+s+s1}{\PYGZsq{}}\PYG{p}{]}\PYG{o}{.}\PYG{n}{get}\PYG{p}{(}\PYG{l+s+s1}{\PYGZsq{}}\PYG{l+s+s1}{prob}\PYG{l+s+s1}{\PYGZsq{}}\PYG{p}{,}\PYG{l+m+mi}{1}\PYG{p}{)}\PYG{o}{*}\PYG{n}{cost}\PYG{p}{,} \PYG{l+s+s1}{\PYGZsq{}}\PYG{l+s+s1}{faults}\PYG{l+s+s1}{\PYGZsq{}}\PYG{p}{:}\PYG{n}{modes}\PYG{p}{,} \PYG{l+s+s1}{\PYGZsq{}}\PYG{l+s+s1}{classification}\PYG{l+s+s1}{\PYGZsq{}}\PYG{p}{:}\PYG{n}{classification}\PYG{p}{\PYGZcb{}}
\end{sphinxVerbatim}
}

\end{sphinxuseclass}
\end{sphinxuseclass}
\sphinxAtStartPar
Below shows the performance of the rover during a (default) turn with a radius of 20 meters that begins at 20 meters. As shown, there is a significant amount of drift, but not enough for the rover to get lost (that would take 1 meter of drift).

\begin{sphinxuseclass}{nbinput}
{
\sphinxsetup{VerbatimColor={named}{nbsphinx-code-bg}}
\sphinxsetup{VerbatimBorderColor={named}{nbsphinx-code-border}}
\begin{sphinxVerbatim}[commandchars=\\\{\}]
\llap{\color{nbsphinxin}[5]:\,\hspace{\fboxrule}\hspace{\fboxsep}}\PYG{n}{mdl} \PYG{o}{=} \PYG{n}{Rover}\PYG{p}{(}\PYG{n}{params}\PYG{o}{=}\PYG{n}{gen\PYGZus{}params}\PYG{p}{(}\PYG{l+s+s1}{\PYGZsq{}}\PYG{l+s+s1}{turn}\PYG{l+s+s1}{\PYGZsq{}}\PYG{p}{)}\PYG{p}{)}
\PYG{n}{endresults}\PYG{p}{,} \PYG{n}{resgraph}\PYG{p}{,} \PYG{n}{mdlhist} \PYG{o}{=} \PYG{n}{prop}\PYG{o}{.}\PYG{n}{nominal}\PYG{p}{(}\PYG{n}{mdl}\PYG{p}{)}
\PYG{n}{plot\PYGZus{}map}\PYG{p}{(}\PYG{n}{mdl}\PYG{p}{,} \PYG{n}{mdlhist}\PYG{p}{)}
\end{sphinxVerbatim}
}

\end{sphinxuseclass}
\begin{sphinxuseclass}{nboutput}
\hrule height -\fboxrule\relax
\vspace{\nbsphinxcodecellspacing}

\makeatletter\setbox\nbsphinxpromptbox\box\voidb@x\makeatother

\begin{nbsphinxfancyoutput}

\begin{sphinxuseclass}{output_area}
\begin{sphinxuseclass}{}
\noindent\sphinxincludegraphics[width=382\sphinxpxdimen,height=278\sphinxpxdimen]{{docs_Nominal_Approach_Use-Cases_10_0}.png}

\end{sphinxuseclass}
\end{sphinxuseclass}
\end{nbsphinxfancyoutput}

\end{sphinxuseclass}
\begin{sphinxuseclass}{nboutput}
\begin{sphinxuseclass}{nblast}
\hrule height -\fboxrule\relax
\vspace{\nbsphinxcodecellspacing}

\makeatletter\setbox\nbsphinxpromptbox\box\voidb@x\makeatother

\begin{nbsphinxfancyoutput}

\begin{sphinxuseclass}{output_area}
\begin{sphinxuseclass}{}
\noindent\sphinxincludegraphics[width=394\sphinxpxdimen,height=278\sphinxpxdimen]{{docs_Nominal_Approach_Use-Cases_10_1}.png}

\end{sphinxuseclass}
\end{sphinxuseclass}
\end{nbsphinxfancyoutput}

\end{sphinxuseclass}
\end{sphinxuseclass}
\sphinxAtStartPar
Below shows the performance of the model over a sine wave. As shown, the drift is much smaller because the turns are much less pronounced (a movement of 0.2 meters over 25 meters).

\begin{sphinxuseclass}{nbinput}
{
\sphinxsetup{VerbatimColor={named}{nbsphinx-code-bg}}
\sphinxsetup{VerbatimBorderColor={named}{nbsphinx-code-border}}
\begin{sphinxVerbatim}[commandchars=\\\{\}]
\llap{\color{nbsphinxin}[6]:\,\hspace{\fboxrule}\hspace{\fboxsep}}\PYG{n}{mdl} \PYG{o}{=} \PYG{n}{Rover}\PYG{p}{(}\PYG{n}{params}\PYG{o}{=}\PYG{n}{gen\PYGZus{}params}\PYG{p}{(}\PYG{l+s+s1}{\PYGZsq{}}\PYG{l+s+s1}{sine}\PYG{l+s+s1}{\PYGZsq{}}\PYG{p}{)}\PYG{p}{)}
\PYG{n}{endresults}\PYG{p}{,} \PYG{n}{resgraph}\PYG{p}{,} \PYG{n}{mdlhist} \PYG{o}{=} \PYG{n}{prop}\PYG{o}{.}\PYG{n}{nominal}\PYG{p}{(}\PYG{n}{mdl}\PYG{p}{)}
\PYG{n}{plot\PYGZus{}map}\PYG{p}{(}\PYG{n}{mdl}\PYG{p}{,} \PYG{n}{mdlhist}\PYG{p}{)}
\end{sphinxVerbatim}
}

\end{sphinxuseclass}
\begin{sphinxuseclass}{nboutput}
\hrule height -\fboxrule\relax
\vspace{\nbsphinxcodecellspacing}

\makeatletter\setbox\nbsphinxpromptbox\box\voidb@x\makeatother

\begin{nbsphinxfancyoutput}

\begin{sphinxuseclass}{output_area}
\begin{sphinxuseclass}{}
\noindent\sphinxincludegraphics[width=394\sphinxpxdimen,height=278\sphinxpxdimen]{{docs_Nominal_Approach_Use-Cases_12_0}.png}

\end{sphinxuseclass}
\end{sphinxuseclass}
\end{nbsphinxfancyoutput}

\end{sphinxuseclass}
\begin{sphinxuseclass}{nboutput}
\begin{sphinxuseclass}{nblast}
\hrule height -\fboxrule\relax
\vspace{\nbsphinxcodecellspacing}

\makeatletter\setbox\nbsphinxpromptbox\box\voidb@x\makeatother

\begin{nbsphinxfancyoutput}

\begin{sphinxuseclass}{output_area}
\begin{sphinxuseclass}{}
\noindent\sphinxincludegraphics[width=400\sphinxpxdimen,height=278\sphinxpxdimen]{{docs_Nominal_Approach_Use-Cases_12_1}.png}

\end{sphinxuseclass}
\end{sphinxuseclass}
\end{nbsphinxfancyoutput}

\end{sphinxuseclass}
\end{sphinxuseclass}
\sphinxAtStartPar
The performance of the rover in these situations is dependent on the parameters of the situation (e.g., the radius of the curve and the amplitude of the sine wave). Thus, it is important to define the operational envelope for the system. This can be done using a \sphinxcode{\sphinxupquote{NominalApproach}}, which can be used to define ranges of variables to simulate the system under.

\begin{sphinxuseclass}{nbinput}
\begin{sphinxuseclass}{nblast}
{
\sphinxsetup{VerbatimColor={named}{nbsphinx-code-bg}}
\sphinxsetup{VerbatimBorderColor={named}{nbsphinx-code-border}}
\begin{sphinxVerbatim}[commandchars=\\\{\}]
\llap{\color{nbsphinxin}[7]:\,\hspace{\fboxrule}\hspace{\fboxsep}}\PYG{k+kn}{from} \PYG{n+nn}{fmdtools}\PYG{n+nn}{.}\PYG{n+nn}{modeldef} \PYG{k+kn}{import} \PYG{n}{NominalApproach}
\end{sphinxVerbatim}
}

\end{sphinxuseclass}
\end{sphinxuseclass}
\begin{sphinxuseclass}{nbinput}
{
\sphinxsetup{VerbatimColor={named}{nbsphinx-code-bg}}
\sphinxsetup{VerbatimBorderColor={named}{nbsphinx-code-border}}
\begin{sphinxVerbatim}[commandchars=\\\{\}]
\llap{\color{nbsphinxin}[8]:\,\hspace{\fboxrule}\hspace{\fboxsep}}\PYG{n}{help}\PYG{p}{(}\PYG{n}{NominalApproach}\PYG{p}{)}
\end{sphinxVerbatim}
}

\end{sphinxuseclass}
\begin{sphinxuseclass}{nboutput}
\begin{sphinxuseclass}{nblast}
{

\kern-\sphinxverbatimsmallskipamount\kern-\baselineskip
\kern+\FrameHeightAdjust\kern-\fboxrule
\vspace{\nbsphinxcodecellspacing}

\sphinxsetup{VerbatimColor={named}{white}}
\sphinxsetup{VerbatimBorderColor={named}{nbsphinx-code-border}}
\begin{sphinxuseclass}{output_area}
\begin{sphinxuseclass}{}


\begin{sphinxVerbatim}[commandchars=\\\{\}]
Help on class NominalApproach in module fmdtools.modeldef:

class NominalApproach(builtins.object)
 |  Class for defining sets of nominal simulations. To explain, a given system
 |  may have a number of input situations (missions, terrain, etc) which the
 |  user may want to simulate to ensure the system operates as desired. This
 |  class (in conjunction with propagate.nominal\_approach()) can be used to
 |  perform these simulations.
 |
 |  Attributes
 |  ----------
 |  scenarios : dict
 |      scenarios to inject based on the approach
 |  num\_scenarios : int
 |      number of scenarios in the approach
 |  ranges : dict
 |      dict of the parameters defined in each method for the approach
 |
 |  Methods defined here:
 |
 |  \_\_init\_\_(self)
 |      Instantiates NominalApproach (simulation params are defined using methods)
 |
 |  add\_param\_ranges(self, paramfunc, rangeid, *args, replicates=1, seeds='shared', **kwargs)
 |      Adds a set of scenarios to the approach.
 |
 |      Parameters
 |      ----------
 |      paramfunc : method
 |          Python method which generates a set of model parameters given the input arguments.
 |          method should have form: method(fixedarg, fixedarg{\ldots}, inputarg=X, inputarg=X)
 |      rangeid : str
 |          Name for the range being used. Default is 'nominal'
 |      *args: specifies values for positional args of paramfunc.
 |          May be given as a fixed float/int/dict/str defining a set value for positional arguments
 |      replicates : int
 |          Number of points to take over each range (for random parameters). Default is 1.
 |      seeds : str/list
 |          Options for seeding models/replicates: (Default is 'shared')
 |              - 'shared' creates random seeds and shares them between parameters and models
 |              - 'independent' creates separate random seeds for models and parameter generation
 |              - 'keep\_model' uses the seed provided in the model for all of the model
 |          When a list is provided, these seeds are are used (and shared). Must be of length replicates.
 |      **kwargs : specifies range for keyword args of paramfunc
 |          May be given as a fixed float/int/dict/str (k=value) defining a set value for the range (if not the default) or
 |          as a tuple k=(start, end, step)
 |
 |  add\_param\_replicates(self, paramfunc, rangeid, replicates, *args, ind\_seeds=True, **kwargs)
 |      Adds a set of repeated scenarios to the approach. For use in (external) random scenario generation.
 |
 |      Parameters
 |      ----------
 |      paramfunc : method
 |          Python method which generates a set of model parameters given the input arguments.
 |          method should have form: method(fixedarg, fixedarg{\ldots}, inputarg=X, inputarg=X)
 |      rangeid : str
 |          Name for the set of replicates
 |      replicates : int
 |          Number of replicates to use
 |      *args : any
 |          arguments to send to paramfunc
 |      ind\_seeds : Bool/list
 |          Whether the models should be run with different seeds (rather than the same seed). Default is True
 |          When a list is provided, these seeds are are used. Must be of length replicates.
 |      *kwargs : any
 |          keyword arguments to send to paramfunc
 |
 |  add\_rand\_params(self, paramfunc, rangeid, *fixedargs, prob\_weight=1.0, replicates=1000, seeds='shared', **randvars)
 |      Adds a set of random scenarios to the approach.
 |
 |      Parameters
 |      ----------
 |      paramfunc : method
 |          Python method which generates a set of model parameters given the input arguments.
 |          method should have form: method(fixedarg, fixedarg{\ldots}, inputarg=X, inputarg=X)
 |      rangeid : str
 |          Name for the range being used. Default is 'nominal'
 |      prob\_weight : float (0-1)
 |          Overall probability for the set of scenarios (to use if adding more ranges). Default is 1.0
 |      *fixedargs : any
 |          Fixed positional arguments in the parameter generator function.
 |          Useful for discrete modes with different parameters.
 |      seeds : str/list
 |          Options for seeding models/replicates: (Default is 'shared')
 |              - 'shared' creates random seeds and shares them between parameters and models
 |              - 'independent' creates separate random seeds for models and parameter generation
 |              - 'keep\_model' uses the seed provided in the model for all of the model
 |          When a list is provided, these seeds are are used (and shared). Must be of length replicates.
 |      **randvars : key=tuple
 |          Specification for each random input parameter, specified as
 |          input = (randfunc, param1, param2{\ldots})
 |          where randfunc is the method producing random outputs (e.g. numpy.random.rand)
 |          and the successive parameters param1, param2, etc are inputs to the method
 |
 |  add\_seed\_replicates(self, rangeid, seeds)
 |      Generates an approach with different seeds to use for the model's internal stochastic behaviors
 |
 |      Parameters
 |      ----------
 |      rangeid : str
 |          Name for the set of replicates
 |      seeds : int/list
 |          Number of seeds (if an int) or a list of seeds to use.
 |
 |  assoc\_probs(self, rangeid, prob\_weight=1.0, **inputpdfs)
 |      Associates a probability model (assuming variable independence) with a
 |      given previously-defined range of scenarios using given pdfs
 |
 |      Parameters
 |      ----------
 |      rangeid : str
 |          Name of the range to apply the probability model to.
 |      prob\_weight : float, optional
 |          Overall probability for the set of scenarios (to use if adding more ranges
 |          or if the range does not cover the space of probability). The default is 1.0.
 |      **inputpdfs : key=(pdf, params)
 |          pdf to associate with the different variables of the model.
 |          Where the pdf has form pdf(x, **kwargs) where x is the location and **kwargs is parameters
 |          (for example, scipy.stats.norm.pdf)
 |          and params is a dictionary of parameters (e.g., \{'mu':1,'std':1\}) to use '
 |          as the key/parameter inputs to the pdf
 |
 |  change\_params(self, rangeid='all', **kwargs)
 |      Changes a given parameter accross all scenarios. Modifies 'params' (rather than regenerating params from the paramfunc).
 |
 |      Parameters
 |      ----------
 |      rangeid : str
 |          Name of the range to modify. Optional. Defaults to "all"
 |      **kwargs : any
 |          Parameters to change stated as paramname=value or
 |          as a dict paramname=\{'sub\_param':value\}, where 'sub\_param' is the parameter of the dictionary with name paramname to update
 |
 |  copy(self)
 |      Copies the given sampleapproach. Used in nested scenario sampling.
 |
 |  get\_param\_scens(self, rangeid, *level\_params)
 |      Returns the scenarios of a range associated with given parameter ranges
 |
 |      Parameters
 |      ----------
 |      rangeid : str
 |          Range id to check
 |      level\_params : str (multiple)
 |          Level parameters iterate over
 |
 |      Returns
 |      -------
 |      param\_scens : dict
 |          The scenarios associated with each level of parameter (or joint parameters)
 |
 |  ----------------------------------------------------------------------
 |  Data descriptors defined here:
 |
 |  \_\_dict\_\_
 |      dictionary for instance variables (if defined)
 |
 |  \_\_weakref\_\_
 |      list of weak references to the object (if defined)

\end{sphinxVerbatim}



\end{sphinxuseclass}
\end{sphinxuseclass}
}

\end{sphinxuseclass}
\end{sphinxuseclass}
\sphinxAtStartPar
In this approach we define parameter ranges for the two major situations–a wavelength and amplitude for the sine wave, and a radius and start location for the turn.

\sphinxAtStartPar
Defining an approach in terms of ranges is performed with \sphinxcode{\sphinxupquote{.add\_param\_ranges()}}

\begin{sphinxuseclass}{nbinput}
{
\sphinxsetup{VerbatimColor={named}{nbsphinx-code-bg}}
\sphinxsetup{VerbatimBorderColor={named}{nbsphinx-code-border}}
\begin{sphinxVerbatim}[commandchars=\\\{\}]
\llap{\color{nbsphinxin}[9]:\,\hspace{\fboxrule}\hspace{\fboxsep}}\PYG{n}{nomapp} \PYG{o}{=} \PYG{n}{NominalApproach}\PYG{p}{(}\PYG{p}{)}
\PYG{n}{help}\PYG{p}{(}\PYG{n}{nomapp}\PYG{o}{.}\PYG{n}{add\PYGZus{}param\PYGZus{}ranges}\PYG{p}{)}
\end{sphinxVerbatim}
}

\end{sphinxuseclass}
\begin{sphinxuseclass}{nboutput}
\begin{sphinxuseclass}{nblast}
{

\kern-\sphinxverbatimsmallskipamount\kern-\baselineskip
\kern+\FrameHeightAdjust\kern-\fboxrule
\vspace{\nbsphinxcodecellspacing}

\sphinxsetup{VerbatimColor={named}{white}}
\sphinxsetup{VerbatimBorderColor={named}{nbsphinx-code-border}}
\begin{sphinxuseclass}{output_area}
\begin{sphinxuseclass}{}


\begin{sphinxVerbatim}[commandchars=\\\{\}]
Help on method add\_param\_ranges in module fmdtools.modeldef:

add\_param\_ranges(paramfunc, rangeid, *args, replicates=1, seeds='shared', **kwargs) method of fmdtools.modeldef.NominalApproach instance
    Adds a set of scenarios to the approach.

    Parameters
    ----------
    paramfunc : method
        Python method which generates a set of model parameters given the input arguments.
        method should have form: method(fixedarg, fixedarg{\ldots}, inputarg=X, inputarg=X)
    rangeid : str
        Name for the range being used. Default is 'nominal'
    *args: specifies values for positional args of paramfunc.
        May be given as a fixed float/int/dict/str defining a set value for positional arguments
    replicates : int
        Number of points to take over each range (for random parameters). Default is 1.
    seeds : str/list
        Options for seeding models/replicates: (Default is 'shared')
            - 'shared' creates random seeds and shares them between parameters and models
            - 'independent' creates separate random seeds for models and parameter generation
            - 'keep\_model' uses the seed provided in the model for all of the model
        When a list is provided, these seeds are are used (and shared). Must be of length replicates.
    **kwargs : specifies range for keyword args of paramfunc
        May be given as a fixed float/int/dict/str (k=value) defining a set value for the range (if not the default) or
        as a tuple k=(start, end, step)

\end{sphinxVerbatim}



\end{sphinxuseclass}
\end{sphinxuseclass}
}

\end{sphinxuseclass}
\end{sphinxuseclass}
\begin{sphinxuseclass}{nbinput}
\begin{sphinxuseclass}{nblast}
{
\sphinxsetup{VerbatimColor={named}{nbsphinx-code-bg}}
\sphinxsetup{VerbatimBorderColor={named}{nbsphinx-code-border}}
\begin{sphinxVerbatim}[commandchars=\\\{\}]
\llap{\color{nbsphinxin}[10]:\,\hspace{\fboxrule}\hspace{\fboxsep}}\PYG{n}{nomapp}\PYG{o}{.}\PYG{n}{add\PYGZus{}param\PYGZus{}ranges}\PYG{p}{(}\PYG{n}{gen\PYGZus{}params}\PYG{p}{,}\PYG{l+s+s1}{\PYGZsq{}}\PYG{l+s+s1}{sine}\PYG{l+s+s1}{\PYGZsq{}}\PYG{p}{,}\PYG{l+s+s1}{\PYGZsq{}}\PYG{l+s+s1}{sine}\PYG{l+s+s1}{\PYGZsq{}}\PYG{p}{,} \PYG{n}{amp}\PYG{o}{=}\PYG{p}{(}\PYG{l+m+mi}{0}\PYG{p}{,} \PYG{l+m+mi}{10}\PYG{p}{,} \PYG{l+m+mf}{0.2}\PYG{p}{)}\PYG{p}{,} \PYG{n}{wavelength}\PYG{o}{=}\PYG{p}{(}\PYG{l+m+mi}{10}\PYG{p}{,}\PYG{l+m+mi}{50}\PYG{p}{,}\PYG{l+m+mi}{10}\PYG{p}{)}\PYG{p}{)}
\PYG{n}{nomapp}\PYG{o}{.}\PYG{n}{add\PYGZus{}param\PYGZus{}ranges}\PYG{p}{(}\PYG{n}{gen\PYGZus{}params}\PYG{p}{,}\PYG{l+s+s1}{\PYGZsq{}}\PYG{l+s+s1}{turn}\PYG{l+s+s1}{\PYGZsq{}}\PYG{p}{,}\PYG{l+s+s1}{\PYGZsq{}}\PYG{l+s+s1}{turn}\PYG{l+s+s1}{\PYGZsq{}}\PYG{p}{,} \PYG{n}{radius}\PYG{o}{=}\PYG{p}{(}\PYG{l+m+mi}{5}\PYG{p}{,}\PYG{l+m+mi}{40}\PYG{p}{,}\PYG{l+m+mi}{5}\PYG{p}{)}\PYG{p}{,} \PYG{n}{start}\PYG{o}{=}\PYG{p}{(}\PYG{l+m+mi}{0}\PYG{p}{,} \PYG{l+m+mi}{20}\PYG{p}{,}\PYG{l+m+mi}{5}\PYG{p}{)}\PYG{p}{)}
\end{sphinxVerbatim}
}

\end{sphinxuseclass}
\end{sphinxuseclass}
\sphinxAtStartPar
Notice that \sphinxcode{\sphinxupquote{gen\_params}} is the handle of the method defined earlier to generate the nominal parameters of the model, \sphinxcode{\sphinxupquote{sine}} is a fixed parameter defining a discrete case of scenarios, and \sphinxcode{\sphinxupquote{amp}} and \sphinxcode{\sphinxupquote{wavelenth}} are parameters that are varied (the given tuples define the respective ranges).

\sphinxAtStartPar
The result is a defined set of scenarios which can be run in the model:

\begin{sphinxuseclass}{nbinput}
{
\sphinxsetup{VerbatimColor={named}{nbsphinx-code-bg}}
\sphinxsetup{VerbatimBorderColor={named}{nbsphinx-code-border}}
\begin{sphinxVerbatim}[commandchars=\\\{\}]
\llap{\color{nbsphinxin}[11]:\,\hspace{\fboxrule}\hspace{\fboxsep}}\PYG{n}{nomapp}\PYG{o}{.}\PYG{n}{scenarios}
\end{sphinxVerbatim}
}

\end{sphinxuseclass}
\begin{sphinxuseclass}{nboutput}
\begin{sphinxuseclass}{nblast}
{

\kern-\sphinxverbatimsmallskipamount\kern-\baselineskip
\kern+\FrameHeightAdjust\kern-\fboxrule
\vspace{\nbsphinxcodecellspacing}

\sphinxsetup{VerbatimColor={named}{white}}
\sphinxsetup{VerbatimBorderColor={named}{nbsphinx-code-border}}
\begin{sphinxuseclass}{output_area}
\begin{sphinxuseclass}{}


\begin{sphinxVerbatim}[commandchars=\\\{\}]
\llap{\color{nbsphinxout}[11]:\,\hspace{\fboxrule}\hspace{\fboxsep}}\{'sine\_1': \{'faults': \{\},
  'properties': \{'type': 'nominal',
   'time': 0.0,
   'name': 'sine\_1',
   'rangeid': 'sine',
   'params': \{'linetype': 'sine',
    'amp': 0.0,
    'period': 0.6283185307179586,
    'initangle': 0.0,
    'end': [10, 0.0]\},
   'inputparams': \{'amp': 0.0, 'wavelength': 10\},
   'modelparams': \{'seed': 899083440\},
   'paramfunc': <function \_\_main\_\_.gen\_params(linetype, **kwargs)>,
   'fixedargs': ('sine',),
   'fixedkwargs': \{\},
   'prob': 0.005\}\},
 'sine\_2': \{'faults': \{\},
  'properties': \{'type': 'nominal',
   'time': 0.0,
   'name': 'sine\_2',
   'rangeid': 'sine',
   'params': \{'linetype': 'sine',
    'amp': 0.0,
    'period': 0.3141592653589793,
    'initangle': 0.0,
    'end': [20, 0.0]\},
   'inputparams': \{'amp': 0.0, 'wavelength': 20\},
   'modelparams': \{'seed': 899083440\},
   'paramfunc': <function \_\_main\_\_.gen\_params(linetype, **kwargs)>,
   'fixedargs': ('sine',),
   'fixedkwargs': \{\},
   'prob': 0.005\}\},
 'sine\_3': \{'faults': \{\},
  'properties': \{'type': 'nominal',
   'time': 0.0,
   'name': 'sine\_3',
   'rangeid': 'sine',
   'params': \{'linetype': 'sine',
    'amp': 0.0,
    'period': 0.20943951023931953,
    'initangle': 0.0,
    'end': [30, 0.0]\},
   'inputparams': \{'amp': 0.0, 'wavelength': 30\},
   'modelparams': \{'seed': 899083440\},
   'paramfunc': <function \_\_main\_\_.gen\_params(linetype, **kwargs)>,
   'fixedargs': ('sine',),
   'fixedkwargs': \{\},
   'prob': 0.005\}\},
 'sine\_4': \{'faults': \{\},
  'properties': \{'type': 'nominal',
   'time': 0.0,
   'name': 'sine\_4',
   'rangeid': 'sine',
   'params': \{'linetype': 'sine',
    'amp': 0.0,
    'period': 0.15707963267948966,
    'initangle': 0.0,
    'end': [40, 0.0]\},
   'inputparams': \{'amp': 0.0, 'wavelength': 40\},
   'modelparams': \{'seed': 899083440\},
   'paramfunc': <function \_\_main\_\_.gen\_params(linetype, **kwargs)>,
   'fixedargs': ('sine',),
   'fixedkwargs': \{\},
   'prob': 0.005\}\},
 'sine\_5': \{'faults': \{\},
  'properties': \{'type': 'nominal',
   'time': 0.0,
   'name': 'sine\_5',
   'rangeid': 'sine',
   'params': \{'linetype': 'sine',
    'amp': 0.2,
    'period': 0.6283185307179586,
    'initangle': 7.200000000000001,
    'end': [10, 0.0]\},
   'inputparams': \{'amp': 0.2, 'wavelength': 10\},
   'modelparams': \{'seed': 899083440\},
   'paramfunc': <function \_\_main\_\_.gen\_params(linetype, **kwargs)>,
   'fixedargs': ('sine',),
   'fixedkwargs': \{\},
   'prob': 0.005\}\},
 'sine\_6': \{'faults': \{\},
  'properties': \{'type': 'nominal',
   'time': 0.0,
   'name': 'sine\_6',
   'rangeid': 'sine',
   'params': \{'linetype': 'sine',
    'amp': 0.2,
    'period': 0.3141592653589793,
    'initangle': 3.6000000000000005,
    'end': [20, 0.0]\},
   'inputparams': \{'amp': 0.2, 'wavelength': 20\},
   'modelparams': \{'seed': 899083440\},
   'paramfunc': <function \_\_main\_\_.gen\_params(linetype, **kwargs)>,
   'fixedargs': ('sine',),
   'fixedkwargs': \{\},
   'prob': 0.005\}\},
 'sine\_7': \{'faults': \{\},
  'properties': \{'type': 'nominal',
   'time': 0.0,
   'name': 'sine\_7',
   'rangeid': 'sine',
   'params': \{'linetype': 'sine',
    'amp': 0.2,
    'period': 0.20943951023931953,
    'initangle': 2.4000000000000004,
    'end': [30, 0.0]\},
   'inputparams': \{'amp': 0.2, 'wavelength': 30\},
   'modelparams': \{'seed': 899083440\},
   'paramfunc': <function \_\_main\_\_.gen\_params(linetype, **kwargs)>,
   'fixedargs': ('sine',),
   'fixedkwargs': \{\},
   'prob': 0.005\}\},
 'sine\_8': \{'faults': \{\},
  'properties': \{'type': 'nominal',
   'time': 0.0,
   'name': 'sine\_8',
   'rangeid': 'sine',
   'params': \{'linetype': 'sine',
    'amp': 0.2,
    'period': 0.15707963267948966,
    'initangle': 1.8000000000000003,
    'end': [40, 0.0]\},
   'inputparams': \{'amp': 0.2, 'wavelength': 40\},
   'modelparams': \{'seed': 899083440\},
   'paramfunc': <function \_\_main\_\_.gen\_params(linetype, **kwargs)>,
   'fixedargs': ('sine',),
   'fixedkwargs': \{\},
   'prob': 0.005\}\},
 'sine\_9': \{'faults': \{\},
  'properties': \{'type': 'nominal',
   'time': 0.0,
   'name': 'sine\_9',
   'rangeid': 'sine',
   'params': \{'linetype': 'sine',
    'amp': 0.4,
    'period': 0.6283185307179586,
    'initangle': 14.400000000000002,
    'end': [10, 0.0]\},
   'inputparams': \{'amp': 0.4, 'wavelength': 10\},
   'modelparams': \{'seed': 899083440\},
   'paramfunc': <function \_\_main\_\_.gen\_params(linetype, **kwargs)>,
   'fixedargs': ('sine',),
   'fixedkwargs': \{\},
   'prob': 0.005\}\},
 'sine\_10': \{'faults': \{\},
  'properties': \{'type': 'nominal',
   'time': 0.0,
   'name': 'sine\_10',
   'rangeid': 'sine',
   'params': \{'linetype': 'sine',
    'amp': 0.4,
    'period': 0.3141592653589793,
    'initangle': 7.200000000000001,
    'end': [20, 0.0]\},
   'inputparams': \{'amp': 0.4, 'wavelength': 20\},
   'modelparams': \{'seed': 899083440\},
   'paramfunc': <function \_\_main\_\_.gen\_params(linetype, **kwargs)>,
   'fixedargs': ('sine',),
   'fixedkwargs': \{\},
   'prob': 0.005\}\},
 'sine\_11': \{'faults': \{\},
  'properties': \{'type': 'nominal',
   'time': 0.0,
   'name': 'sine\_11',
   'rangeid': 'sine',
   'params': \{'linetype': 'sine',
    'amp': 0.4,
    'period': 0.20943951023931953,
    'initangle': 4.800000000000001,
    'end': [30, 0.0]\},
   'inputparams': \{'amp': 0.4, 'wavelength': 30\},
   'modelparams': \{'seed': 899083440\},
   'paramfunc': <function \_\_main\_\_.gen\_params(linetype, **kwargs)>,
   'fixedargs': ('sine',),
   'fixedkwargs': \{\},
   'prob': 0.005\}\},
 'sine\_12': \{'faults': \{\},
  'properties': \{'type': 'nominal',
   'time': 0.0,
   'name': 'sine\_12',
   'rangeid': 'sine',
   'params': \{'linetype': 'sine',
    'amp': 0.4,
    'period': 0.15707963267948966,
    'initangle': 3.6000000000000005,
    'end': [40, 0.0]\},
   'inputparams': \{'amp': 0.4, 'wavelength': 40\},
   'modelparams': \{'seed': 899083440\},
   'paramfunc': <function \_\_main\_\_.gen\_params(linetype, **kwargs)>,
   'fixedargs': ('sine',),
   'fixedkwargs': \{\},
   'prob': 0.005\}\},
 'sine\_13': \{'faults': \{\},
  'properties': \{'type': 'nominal',
   'time': 0.0,
   'name': 'sine\_13',
   'rangeid': 'sine',
   'params': \{'linetype': 'sine',
    'amp': 0.6000000000000001,
    'period': 0.6283185307179586,
    'initangle': 21.6,
    'end': [10, 0.0]\},
   'inputparams': \{'amp': 0.6000000000000001, 'wavelength': 10\},
   'modelparams': \{'seed': 899083440\},
   'paramfunc': <function \_\_main\_\_.gen\_params(linetype, **kwargs)>,
   'fixedargs': ('sine',),
   'fixedkwargs': \{\},
   'prob': 0.005\}\},
 'sine\_14': \{'faults': \{\},
  'properties': \{'type': 'nominal',
   'time': 0.0,
   'name': 'sine\_14',
   'rangeid': 'sine',
   'params': \{'linetype': 'sine',
    'amp': 0.6000000000000001,
    'period': 0.3141592653589793,
    'initangle': 10.8,
    'end': [20, 0.0]\},
   'inputparams': \{'amp': 0.6000000000000001, 'wavelength': 20\},
   'modelparams': \{'seed': 899083440\},
   'paramfunc': <function \_\_main\_\_.gen\_params(linetype, **kwargs)>,
   'fixedargs': ('sine',),
   'fixedkwargs': \{\},
   'prob': 0.005\}\},
 'sine\_15': \{'faults': \{\},
  'properties': \{'type': 'nominal',
   'time': 0.0,
   'name': 'sine\_15',
   'rangeid': 'sine',
   'params': \{'linetype': 'sine',
    'amp': 0.6000000000000001,
    'period': 0.20943951023931953,
    'initangle': 7.200000000000001,
    'end': [30, 0.0]\},
   'inputparams': \{'amp': 0.6000000000000001, 'wavelength': 30\},
   'modelparams': \{'seed': 899083440\},
   'paramfunc': <function \_\_main\_\_.gen\_params(linetype, **kwargs)>,
   'fixedargs': ('sine',),
   'fixedkwargs': \{\},
   'prob': 0.005\}\},
 'sine\_16': \{'faults': \{\},
  'properties': \{'type': 'nominal',
   'time': 0.0,
   'name': 'sine\_16',
   'rangeid': 'sine',
   'params': \{'linetype': 'sine',
    'amp': 0.6000000000000001,
    'period': 0.15707963267948966,
    'initangle': 5.4,
    'end': [40, 0.0]\},
   'inputparams': \{'amp': 0.6000000000000001, 'wavelength': 40\},
   'modelparams': \{'seed': 899083440\},
   'paramfunc': <function \_\_main\_\_.gen\_params(linetype, **kwargs)>,
   'fixedargs': ('sine',),
   'fixedkwargs': \{\},
   'prob': 0.005\}\},
 'sine\_17': \{'faults': \{\},
  'properties': \{'type': 'nominal',
   'time': 0.0,
   'name': 'sine\_17',
   'rangeid': 'sine',
   'params': \{'linetype': 'sine',
    'amp': 0.8,
    'period': 0.6283185307179586,
    'initangle': 28.800000000000004,
    'end': [10, 0.0]\},
   'inputparams': \{'amp': 0.8, 'wavelength': 10\},
   'modelparams': \{'seed': 899083440\},
   'paramfunc': <function \_\_main\_\_.gen\_params(linetype, **kwargs)>,
   'fixedargs': ('sine',),
   'fixedkwargs': \{\},
   'prob': 0.005\}\},
 'sine\_18': \{'faults': \{\},
  'properties': \{'type': 'nominal',
   'time': 0.0,
   'name': 'sine\_18',
   'rangeid': 'sine',
   'params': \{'linetype': 'sine',
    'amp': 0.8,
    'period': 0.3141592653589793,
    'initangle': 14.400000000000002,
    'end': [20, 0.0]\},
   'inputparams': \{'amp': 0.8, 'wavelength': 20\},
   'modelparams': \{'seed': 899083440\},
   'paramfunc': <function \_\_main\_\_.gen\_params(linetype, **kwargs)>,
   'fixedargs': ('sine',),
   'fixedkwargs': \{\},
   'prob': 0.005\}\},
 'sine\_19': \{'faults': \{\},
  'properties': \{'type': 'nominal',
   'time': 0.0,
   'name': 'sine\_19',
   'rangeid': 'sine',
   'params': \{'linetype': 'sine',
    'amp': 0.8,
    'period': 0.20943951023931953,
    'initangle': 9.600000000000001,
    'end': [30, 0.0]\},
   'inputparams': \{'amp': 0.8, 'wavelength': 30\},
   'modelparams': \{'seed': 899083440\},
   'paramfunc': <function \_\_main\_\_.gen\_params(linetype, **kwargs)>,
   'fixedargs': ('sine',),
   'fixedkwargs': \{\},
   'prob': 0.005\}\},
 'sine\_20': \{'faults': \{\},
  'properties': \{'type': 'nominal',
   'time': 0.0,
   'name': 'sine\_20',
   'rangeid': 'sine',
   'params': \{'linetype': 'sine',
    'amp': 0.8,
    'period': 0.15707963267948966,
    'initangle': 7.200000000000001,
    'end': [40, 0.0]\},
   'inputparams': \{'amp': 0.8, 'wavelength': 40\},
   'modelparams': \{'seed': 899083440\},
   'paramfunc': <function \_\_main\_\_.gen\_params(linetype, **kwargs)>,
   'fixedargs': ('sine',),
   'fixedkwargs': \{\},
   'prob': 0.005\}\},
 'sine\_21': \{'faults': \{\},
  'properties': \{'type': 'nominal',
   'time': 0.0,
   'name': 'sine\_21',
   'rangeid': 'sine',
   'params': \{'linetype': 'sine',
    'amp': 1.0,
    'period': 0.6283185307179586,
    'initangle': 36.0,
    'end': [10, 0.0]\},
   'inputparams': \{'amp': 1.0, 'wavelength': 10\},
   'modelparams': \{'seed': 899083440\},
   'paramfunc': <function \_\_main\_\_.gen\_params(linetype, **kwargs)>,
   'fixedargs': ('sine',),
   'fixedkwargs': \{\},
   'prob': 0.005\}\},
 'sine\_22': \{'faults': \{\},
  'properties': \{'type': 'nominal',
   'time': 0.0,
   'name': 'sine\_22',
   'rangeid': 'sine',
   'params': \{'linetype': 'sine',
    'amp': 1.0,
    'period': 0.3141592653589793,
    'initangle': 18.0,
    'end': [20, 0.0]\},
   'inputparams': \{'amp': 1.0, 'wavelength': 20\},
   'modelparams': \{'seed': 899083440\},
   'paramfunc': <function \_\_main\_\_.gen\_params(linetype, **kwargs)>,
   'fixedargs': ('sine',),
   'fixedkwargs': \{\},
   'prob': 0.005\}\},
 'sine\_23': \{'faults': \{\},
  'properties': \{'type': 'nominal',
   'time': 0.0,
   'name': 'sine\_23',
   'rangeid': 'sine',
   'params': \{'linetype': 'sine',
    'amp': 1.0,
    'period': 0.20943951023931953,
    'initangle': 12.0,
    'end': [30, 0.0]\},
   'inputparams': \{'amp': 1.0, 'wavelength': 30\},
   'modelparams': \{'seed': 899083440\},
   'paramfunc': <function \_\_main\_\_.gen\_params(linetype, **kwargs)>,
   'fixedargs': ('sine',),
   'fixedkwargs': \{\},
   'prob': 0.005\}\},
 'sine\_24': \{'faults': \{\},
  'properties': \{'type': 'nominal',
   'time': 0.0,
   'name': 'sine\_24',
   'rangeid': 'sine',
   'params': \{'linetype': 'sine',
    'amp': 1.0,
    'period': 0.15707963267948966,
    'initangle': 9.0,
    'end': [40, 0.0]\},
   'inputparams': \{'amp': 1.0, 'wavelength': 40\},
   'modelparams': \{'seed': 899083440\},
   'paramfunc': <function \_\_main\_\_.gen\_params(linetype, **kwargs)>,
   'fixedargs': ('sine',),
   'fixedkwargs': \{\},
   'prob': 0.005\}\},
 'sine\_25': \{'faults': \{\},
  'properties': \{'type': 'nominal',
   'time': 0.0,
   'name': 'sine\_25',
   'rangeid': 'sine',
   'params': \{'linetype': 'sine',
    'amp': 1.2000000000000002,
    'period': 0.6283185307179586,
    'initangle': 43.2,
    'end': [10, 0.0]\},
   'inputparams': \{'amp': 1.2000000000000002, 'wavelength': 10\},
   'modelparams': \{'seed': 899083440\},
   'paramfunc': <function \_\_main\_\_.gen\_params(linetype, **kwargs)>,
   'fixedargs': ('sine',),
   'fixedkwargs': \{\},
   'prob': 0.005\}\},
 'sine\_26': \{'faults': \{\},
  'properties': \{'type': 'nominal',
   'time': 0.0,
   'name': 'sine\_26',
   'rangeid': 'sine',
   'params': \{'linetype': 'sine',
    'amp': 1.2000000000000002,
    'period': 0.3141592653589793,
    'initangle': 21.6,
    'end': [20, 0.0]\},
   'inputparams': \{'amp': 1.2000000000000002, 'wavelength': 20\},
   'modelparams': \{'seed': 899083440\},
   'paramfunc': <function \_\_main\_\_.gen\_params(linetype, **kwargs)>,
   'fixedargs': ('sine',),
   'fixedkwargs': \{\},
   'prob': 0.005\}\},
 'sine\_27': \{'faults': \{\},
  'properties': \{'type': 'nominal',
   'time': 0.0,
   'name': 'sine\_27',
   'rangeid': 'sine',
   'params': \{'linetype': 'sine',
    'amp': 1.2000000000000002,
    'period': 0.20943951023931953,
    'initangle': 14.400000000000002,
    'end': [30, 0.0]\},
   'inputparams': \{'amp': 1.2000000000000002, 'wavelength': 30\},
   'modelparams': \{'seed': 899083440\},
   'paramfunc': <function \_\_main\_\_.gen\_params(linetype, **kwargs)>,
   'fixedargs': ('sine',),
   'fixedkwargs': \{\},
   'prob': 0.005\}\},
 'sine\_28': \{'faults': \{\},
  'properties': \{'type': 'nominal',
   'time': 0.0,
   'name': 'sine\_28',
   'rangeid': 'sine',
   'params': \{'linetype': 'sine',
    'amp': 1.2000000000000002,
    'period': 0.15707963267948966,
    'initangle': 10.8,
    'end': [40, 0.0]\},
   'inputparams': \{'amp': 1.2000000000000002, 'wavelength': 40\},
   'modelparams': \{'seed': 899083440\},
   'paramfunc': <function \_\_main\_\_.gen\_params(linetype, **kwargs)>,
   'fixedargs': ('sine',),
   'fixedkwargs': \{\},
   'prob': 0.005\}\},
 'sine\_29': \{'faults': \{\},
  'properties': \{'type': 'nominal',
   'time': 0.0,
   'name': 'sine\_29',
   'rangeid': 'sine',
   'params': \{'linetype': 'sine',
    'amp': 1.4000000000000001,
    'period': 0.6283185307179586,
    'initangle': 50.400000000000006,
    'end': [10, 0.0]\},
   'inputparams': \{'amp': 1.4000000000000001, 'wavelength': 10\},
   'modelparams': \{'seed': 899083440\},
   'paramfunc': <function \_\_main\_\_.gen\_params(linetype, **kwargs)>,
   'fixedargs': ('sine',),
   'fixedkwargs': \{\},
   'prob': 0.005\}\},
 'sine\_30': \{'faults': \{\},
  'properties': \{'type': 'nominal',
   'time': 0.0,
   'name': 'sine\_30',
   'rangeid': 'sine',
   'params': \{'linetype': 'sine',
    'amp': 1.4000000000000001,
    'period': 0.3141592653589793,
    'initangle': 25.200000000000003,
    'end': [20, 0.0]\},
   'inputparams': \{'amp': 1.4000000000000001, 'wavelength': 20\},
   'modelparams': \{'seed': 899083440\},
   'paramfunc': <function \_\_main\_\_.gen\_params(linetype, **kwargs)>,
   'fixedargs': ('sine',),
   'fixedkwargs': \{\},
   'prob': 0.005\}\},
 'sine\_31': \{'faults': \{\},
  'properties': \{'type': 'nominal',
   'time': 0.0,
   'name': 'sine\_31',
   'rangeid': 'sine',
   'params': \{'linetype': 'sine',
    'amp': 1.4000000000000001,
    'period': 0.20943951023931953,
    'initangle': 16.8,
    'end': [30, 0.0]\},
   'inputparams': \{'amp': 1.4000000000000001, 'wavelength': 30\},
   'modelparams': \{'seed': 899083440\},
   'paramfunc': <function \_\_main\_\_.gen\_params(linetype, **kwargs)>,
   'fixedargs': ('sine',),
   'fixedkwargs': \{\},
   'prob': 0.005\}\},
 'sine\_32': \{'faults': \{\},
  'properties': \{'type': 'nominal',
   'time': 0.0,
   'name': 'sine\_32',
   'rangeid': 'sine',
   'params': \{'linetype': 'sine',
    'amp': 1.4000000000000001,
    'period': 0.15707963267948966,
    'initangle': 12.600000000000001,
    'end': [40, 0.0]\},
   'inputparams': \{'amp': 1.4000000000000001, 'wavelength': 40\},
   'modelparams': \{'seed': 899083440\},
   'paramfunc': <function \_\_main\_\_.gen\_params(linetype, **kwargs)>,
   'fixedargs': ('sine',),
   'fixedkwargs': \{\},
   'prob': 0.005\}\},
 'sine\_33': \{'faults': \{\},
  'properties': \{'type': 'nominal',
   'time': 0.0,
   'name': 'sine\_33',
   'rangeid': 'sine',
   'params': \{'linetype': 'sine',
    'amp': 1.6,
    'period': 0.6283185307179586,
    'initangle': 57.60000000000001,
    'end': [10, 0.0]\},
   'inputparams': \{'amp': 1.6, 'wavelength': 10\},
   'modelparams': \{'seed': 899083440\},
   'paramfunc': <function \_\_main\_\_.gen\_params(linetype, **kwargs)>,
   'fixedargs': ('sine',),
   'fixedkwargs': \{\},
   'prob': 0.005\}\},
 'sine\_34': \{'faults': \{\},
  'properties': \{'type': 'nominal',
   'time': 0.0,
   'name': 'sine\_34',
   'rangeid': 'sine',
   'params': \{'linetype': 'sine',
    'amp': 1.6,
    'period': 0.3141592653589793,
    'initangle': 28.800000000000004,
    'end': [20, 0.0]\},
   'inputparams': \{'amp': 1.6, 'wavelength': 20\},
   'modelparams': \{'seed': 899083440\},
   'paramfunc': <function \_\_main\_\_.gen\_params(linetype, **kwargs)>,
   'fixedargs': ('sine',),
   'fixedkwargs': \{\},
   'prob': 0.005\}\},
 'sine\_35': \{'faults': \{\},
  'properties': \{'type': 'nominal',
   'time': 0.0,
   'name': 'sine\_35',
   'rangeid': 'sine',
   'params': \{'linetype': 'sine',
    'amp': 1.6,
    'period': 0.20943951023931953,
    'initangle': 19.200000000000003,
    'end': [30, 0.0]\},
   'inputparams': \{'amp': 1.6, 'wavelength': 30\},
   'modelparams': \{'seed': 899083440\},
   'paramfunc': <function \_\_main\_\_.gen\_params(linetype, **kwargs)>,
   'fixedargs': ('sine',),
   'fixedkwargs': \{\},
   'prob': 0.005\}\},
 'sine\_36': \{'faults': \{\},
  'properties': \{'type': 'nominal',
   'time': 0.0,
   'name': 'sine\_36',
   'rangeid': 'sine',
   'params': \{'linetype': 'sine',
    'amp': 1.6,
    'period': 0.15707963267948966,
    'initangle': 14.400000000000002,
    'end': [40, 0.0]\},
   'inputparams': \{'amp': 1.6, 'wavelength': 40\},
   'modelparams': \{'seed': 899083440\},
   'paramfunc': <function \_\_main\_\_.gen\_params(linetype, **kwargs)>,
   'fixedargs': ('sine',),
   'fixedkwargs': \{\},
   'prob': 0.005\}\},
 'sine\_37': \{'faults': \{\},
  'properties': \{'type': 'nominal',
   'time': 0.0,
   'name': 'sine\_37',
   'rangeid': 'sine',
   'params': \{'linetype': 'sine',
    'amp': 1.8,
    'period': 0.6283185307179586,
    'initangle': 64.80000000000001,
    'end': [10, 0.0]\},
   'inputparams': \{'amp': 1.8, 'wavelength': 10\},
   'modelparams': \{'seed': 899083440\},
   'paramfunc': <function \_\_main\_\_.gen\_params(linetype, **kwargs)>,
   'fixedargs': ('sine',),
   'fixedkwargs': \{\},
   'prob': 0.005\}\},
 'sine\_38': \{'faults': \{\},
  'properties': \{'type': 'nominal',
   'time': 0.0,
   'name': 'sine\_38',
   'rangeid': 'sine',
   'params': \{'linetype': 'sine',
    'amp': 1.8,
    'period': 0.3141592653589793,
    'initangle': 32.400000000000006,
    'end': [20, 0.0]\},
   'inputparams': \{'amp': 1.8, 'wavelength': 20\},
   'modelparams': \{'seed': 899083440\},
   'paramfunc': <function \_\_main\_\_.gen\_params(linetype, **kwargs)>,
   'fixedargs': ('sine',),
   'fixedkwargs': \{\},
   'prob': 0.005\}\},
 'sine\_39': \{'faults': \{\},
  'properties': \{'type': 'nominal',
   'time': 0.0,
   'name': 'sine\_39',
   'rangeid': 'sine',
   'params': \{'linetype': 'sine',
    'amp': 1.8,
    'period': 0.20943951023931953,
    'initangle': 21.6,
    'end': [30, 0.0]\},
   'inputparams': \{'amp': 1.8, 'wavelength': 30\},
   'modelparams': \{'seed': 899083440\},
   'paramfunc': <function \_\_main\_\_.gen\_params(linetype, **kwargs)>,
   'fixedargs': ('sine',),
   'fixedkwargs': \{\},
   'prob': 0.005\}\},
 'sine\_40': \{'faults': \{\},
  'properties': \{'type': 'nominal',
   'time': 0.0,
   'name': 'sine\_40',
   'rangeid': 'sine',
   'params': \{'linetype': 'sine',
    'amp': 1.8,
    'period': 0.15707963267948966,
    'initangle': 16.200000000000003,
    'end': [40, 0.0]\},
   'inputparams': \{'amp': 1.8, 'wavelength': 40\},
   'modelparams': \{'seed': 899083440\},
   'paramfunc': <function \_\_main\_\_.gen\_params(linetype, **kwargs)>,
   'fixedargs': ('sine',),
   'fixedkwargs': \{\},
   'prob': 0.005\}\},
 'sine\_41': \{'faults': \{\},
  'properties': \{'type': 'nominal',
   'time': 0.0,
   'name': 'sine\_41',
   'rangeid': 'sine',
   'params': \{'linetype': 'sine',
    'amp': 2.0,
    'period': 0.6283185307179586,
    'initangle': 72.0,
    'end': [10, 0.0]\},
   'inputparams': \{'amp': 2.0, 'wavelength': 10\},
   'modelparams': \{'seed': 899083440\},
   'paramfunc': <function \_\_main\_\_.gen\_params(linetype, **kwargs)>,
   'fixedargs': ('sine',),
   'fixedkwargs': \{\},
   'prob': 0.005\}\},
 'sine\_42': \{'faults': \{\},
  'properties': \{'type': 'nominal',
   'time': 0.0,
   'name': 'sine\_42',
   'rangeid': 'sine',
   'params': \{'linetype': 'sine',
    'amp': 2.0,
    'period': 0.3141592653589793,
    'initangle': 36.0,
    'end': [20, 0.0]\},
   'inputparams': \{'amp': 2.0, 'wavelength': 20\},
   'modelparams': \{'seed': 899083440\},
   'paramfunc': <function \_\_main\_\_.gen\_params(linetype, **kwargs)>,
   'fixedargs': ('sine',),
   'fixedkwargs': \{\},
   'prob': 0.005\}\},
 'sine\_43': \{'faults': \{\},
  'properties': \{'type': 'nominal',
   'time': 0.0,
   'name': 'sine\_43',
   'rangeid': 'sine',
   'params': \{'linetype': 'sine',
    'amp': 2.0,
    'period': 0.20943951023931953,
    'initangle': 24.0,
    'end': [30, 0.0]\},
   'inputparams': \{'amp': 2.0, 'wavelength': 30\},
   'modelparams': \{'seed': 899083440\},
   'paramfunc': <function \_\_main\_\_.gen\_params(linetype, **kwargs)>,
   'fixedargs': ('sine',),
   'fixedkwargs': \{\},
   'prob': 0.005\}\},
 'sine\_44': \{'faults': \{\},
  'properties': \{'type': 'nominal',
   'time': 0.0,
   'name': 'sine\_44',
   'rangeid': 'sine',
   'params': \{'linetype': 'sine',
    'amp': 2.0,
    'period': 0.15707963267948966,
    'initangle': 18.0,
    'end': [40, 0.0]\},
   'inputparams': \{'amp': 2.0, 'wavelength': 40\},
   'modelparams': \{'seed': 899083440\},
   'paramfunc': <function \_\_main\_\_.gen\_params(linetype, **kwargs)>,
   'fixedargs': ('sine',),
   'fixedkwargs': \{\},
   'prob': 0.005\}\},
 'sine\_45': \{'faults': \{\},
  'properties': \{'type': 'nominal',
   'time': 0.0,
   'name': 'sine\_45',
   'rangeid': 'sine',
   'params': \{'linetype': 'sine',
    'amp': 2.2,
    'period': 0.6283185307179586,
    'initangle': 79.20000000000002,
    'end': [10, 0.0]\},
   'inputparams': \{'amp': 2.2, 'wavelength': 10\},
   'modelparams': \{'seed': 899083440\},
   'paramfunc': <function \_\_main\_\_.gen\_params(linetype, **kwargs)>,
   'fixedargs': ('sine',),
   'fixedkwargs': \{\},
   'prob': 0.005\}\},
 'sine\_46': \{'faults': \{\},
  'properties': \{'type': 'nominal',
   'time': 0.0,
   'name': 'sine\_46',
   'rangeid': 'sine',
   'params': \{'linetype': 'sine',
    'amp': 2.2,
    'period': 0.3141592653589793,
    'initangle': 39.60000000000001,
    'end': [20, 0.0]\},
   'inputparams': \{'amp': 2.2, 'wavelength': 20\},
   'modelparams': \{'seed': 899083440\},
   'paramfunc': <function \_\_main\_\_.gen\_params(linetype, **kwargs)>,
   'fixedargs': ('sine',),
   'fixedkwargs': \{\},
   'prob': 0.005\}\},
 'sine\_47': \{'faults': \{\},
  'properties': \{'type': 'nominal',
   'time': 0.0,
   'name': 'sine\_47',
   'rangeid': 'sine',
   'params': \{'linetype': 'sine',
    'amp': 2.2,
    'period': 0.20943951023931953,
    'initangle': 26.400000000000006,
    'end': [30, 0.0]\},
   'inputparams': \{'amp': 2.2, 'wavelength': 30\},
   'modelparams': \{'seed': 899083440\},
   'paramfunc': <function \_\_main\_\_.gen\_params(linetype, **kwargs)>,
   'fixedargs': ('sine',),
   'fixedkwargs': \{\},
   'prob': 0.005\}\},
 'sine\_48': \{'faults': \{\},
  'properties': \{'type': 'nominal',
   'time': 0.0,
   'name': 'sine\_48',
   'rangeid': 'sine',
   'params': \{'linetype': 'sine',
    'amp': 2.2,
    'period': 0.15707963267948966,
    'initangle': 19.800000000000004,
    'end': [40, 0.0]\},
   'inputparams': \{'amp': 2.2, 'wavelength': 40\},
   'modelparams': \{'seed': 899083440\},
   'paramfunc': <function \_\_main\_\_.gen\_params(linetype, **kwargs)>,
   'fixedargs': ('sine',),
   'fixedkwargs': \{\},
   'prob': 0.005\}\},
 'sine\_49': \{'faults': \{\},
  'properties': \{'type': 'nominal',
   'time': 0.0,
   'name': 'sine\_49',
   'rangeid': 'sine',
   'params': \{'linetype': 'sine',
    'amp': 2.4000000000000004,
    'period': 0.6283185307179586,
    'initangle': 86.4,
    'end': [10, 0.0]\},
   'inputparams': \{'amp': 2.4000000000000004, 'wavelength': 10\},
   'modelparams': \{'seed': 899083440\},
   'paramfunc': <function \_\_main\_\_.gen\_params(linetype, **kwargs)>,
   'fixedargs': ('sine',),
   'fixedkwargs': \{\},
   'prob': 0.005\}\},
 'sine\_50': \{'faults': \{\},
  'properties': \{'type': 'nominal',
   'time': 0.0,
   'name': 'sine\_50',
   'rangeid': 'sine',
   'params': \{'linetype': 'sine',
    'amp': 2.4000000000000004,
    'period': 0.3141592653589793,
    'initangle': 43.2,
    'end': [20, 0.0]\},
   'inputparams': \{'amp': 2.4000000000000004, 'wavelength': 20\},
   'modelparams': \{'seed': 899083440\},
   'paramfunc': <function \_\_main\_\_.gen\_params(linetype, **kwargs)>,
   'fixedargs': ('sine',),
   'fixedkwargs': \{\},
   'prob': 0.005\}\},
 'sine\_51': \{'faults': \{\},
  'properties': \{'type': 'nominal',
   'time': 0.0,
   'name': 'sine\_51',
   'rangeid': 'sine',
   'params': \{'linetype': 'sine',
    'amp': 2.4000000000000004,
    'period': 0.20943951023931953,
    'initangle': 28.800000000000004,
    'end': [30, 0.0]\},
   'inputparams': \{'amp': 2.4000000000000004, 'wavelength': 30\},
   'modelparams': \{'seed': 899083440\},
   'paramfunc': <function \_\_main\_\_.gen\_params(linetype, **kwargs)>,
   'fixedargs': ('sine',),
   'fixedkwargs': \{\},
   'prob': 0.005\}\},
 'sine\_52': \{'faults': \{\},
  'properties': \{'type': 'nominal',
   'time': 0.0,
   'name': 'sine\_52',
   'rangeid': 'sine',
   'params': \{'linetype': 'sine',
    'amp': 2.4000000000000004,
    'period': 0.15707963267948966,
    'initangle': 21.6,
    'end': [40, 0.0]\},
   'inputparams': \{'amp': 2.4000000000000004, 'wavelength': 40\},
   'modelparams': \{'seed': 899083440\},
   'paramfunc': <function \_\_main\_\_.gen\_params(linetype, **kwargs)>,
   'fixedargs': ('sine',),
   'fixedkwargs': \{\},
   'prob': 0.005\}\},
 'sine\_53': \{'faults': \{\},
  'properties': \{'type': 'nominal',
   'time': 0.0,
   'name': 'sine\_53',
   'rangeid': 'sine',
   'params': \{'linetype': 'sine',
    'amp': 2.6,
    'period': 0.6283185307179586,
    'initangle': 93.60000000000001,
    'end': [10, 0.0]\},
   'inputparams': \{'amp': 2.6, 'wavelength': 10\},
   'modelparams': \{'seed': 899083440\},
   'paramfunc': <function \_\_main\_\_.gen\_params(linetype, **kwargs)>,
   'fixedargs': ('sine',),
   'fixedkwargs': \{\},
   'prob': 0.005\}\},
 'sine\_54': \{'faults': \{\},
  'properties': \{'type': 'nominal',
   'time': 0.0,
   'name': 'sine\_54',
   'rangeid': 'sine',
   'params': \{'linetype': 'sine',
    'amp': 2.6,
    'period': 0.3141592653589793,
    'initangle': 46.800000000000004,
    'end': [20, 0.0]\},
   'inputparams': \{'amp': 2.6, 'wavelength': 20\},
   'modelparams': \{'seed': 899083440\},
   'paramfunc': <function \_\_main\_\_.gen\_params(linetype, **kwargs)>,
   'fixedargs': ('sine',),
   'fixedkwargs': \{\},
   'prob': 0.005\}\},
 'sine\_55': \{'faults': \{\},
  'properties': \{'type': 'nominal',
   'time': 0.0,
   'name': 'sine\_55',
   'rangeid': 'sine',
   'params': \{'linetype': 'sine',
    'amp': 2.6,
    'period': 0.20943951023931953,
    'initangle': 31.2,
    'end': [30, 0.0]\},
   'inputparams': \{'amp': 2.6, 'wavelength': 30\},
   'modelparams': \{'seed': 899083440\},
   'paramfunc': <function \_\_main\_\_.gen\_params(linetype, **kwargs)>,
   'fixedargs': ('sine',),
   'fixedkwargs': \{\},
   'prob': 0.005\}\},
 'sine\_56': \{'faults': \{\},
  'properties': \{'type': 'nominal',
   'time': 0.0,
   'name': 'sine\_56',
   'rangeid': 'sine',
   'params': \{'linetype': 'sine',
    'amp': 2.6,
    'period': 0.15707963267948966,
    'initangle': 23.400000000000002,
    'end': [40, 0.0]\},
   'inputparams': \{'amp': 2.6, 'wavelength': 40\},
   'modelparams': \{'seed': 899083440\},
   'paramfunc': <function \_\_main\_\_.gen\_params(linetype, **kwargs)>,
   'fixedargs': ('sine',),
   'fixedkwargs': \{\},
   'prob': 0.005\}\},
 'sine\_57': \{'faults': \{\},
  'properties': \{'type': 'nominal',
   'time': 0.0,
   'name': 'sine\_57',
   'rangeid': 'sine',
   'params': \{'linetype': 'sine',
    'amp': 2.8000000000000003,
    'period': 0.6283185307179586,
    'initangle': 100.80000000000001,
    'end': [10, 0.0]\},
   'inputparams': \{'amp': 2.8000000000000003, 'wavelength': 10\},
   'modelparams': \{'seed': 899083440\},
   'paramfunc': <function \_\_main\_\_.gen\_params(linetype, **kwargs)>,
   'fixedargs': ('sine',),
   'fixedkwargs': \{\},
   'prob': 0.005\}\},
 'sine\_58': \{'faults': \{\},
  'properties': \{'type': 'nominal',
   'time': 0.0,
   'name': 'sine\_58',
   'rangeid': 'sine',
   'params': \{'linetype': 'sine',
    'amp': 2.8000000000000003,
    'period': 0.3141592653589793,
    'initangle': 50.400000000000006,
    'end': [20, 0.0]\},
   'inputparams': \{'amp': 2.8000000000000003, 'wavelength': 20\},
   'modelparams': \{'seed': 899083440\},
   'paramfunc': <function \_\_main\_\_.gen\_params(linetype, **kwargs)>,
   'fixedargs': ('sine',),
   'fixedkwargs': \{\},
   'prob': 0.005\}\},
 'sine\_59': \{'faults': \{\},
  'properties': \{'type': 'nominal',
   'time': 0.0,
   'name': 'sine\_59',
   'rangeid': 'sine',
   'params': \{'linetype': 'sine',
    'amp': 2.8000000000000003,
    'period': 0.20943951023931953,
    'initangle': 33.6,
    'end': [30, 0.0]\},
   'inputparams': \{'amp': 2.8000000000000003, 'wavelength': 30\},
   'modelparams': \{'seed': 899083440\},
   'paramfunc': <function \_\_main\_\_.gen\_params(linetype, **kwargs)>,
   'fixedargs': ('sine',),
   'fixedkwargs': \{\},
   'prob': 0.005\}\},
 'sine\_60': \{'faults': \{\},
  'properties': \{'type': 'nominal',
   'time': 0.0,
   'name': 'sine\_60',
   'rangeid': 'sine',
   'params': \{'linetype': 'sine',
    'amp': 2.8000000000000003,
    'period': 0.15707963267948966,
    'initangle': 25.200000000000003,
    'end': [40, 0.0]\},
   'inputparams': \{'amp': 2.8000000000000003, 'wavelength': 40\},
   'modelparams': \{'seed': 899083440\},
   'paramfunc': <function \_\_main\_\_.gen\_params(linetype, **kwargs)>,
   'fixedargs': ('sine',),
   'fixedkwargs': \{\},
   'prob': 0.005\}\},
 'sine\_61': \{'faults': \{\},
  'properties': \{'type': 'nominal',
   'time': 0.0,
   'name': 'sine\_61',
   'rangeid': 'sine',
   'params': \{'linetype': 'sine',
    'amp': 3.0,
    'period': 0.6283185307179586,
    'initangle': 108.0,
    'end': [10, 0.0]\},
   'inputparams': \{'amp': 3.0, 'wavelength': 10\},
   'modelparams': \{'seed': 899083440\},
   'paramfunc': <function \_\_main\_\_.gen\_params(linetype, **kwargs)>,
   'fixedargs': ('sine',),
   'fixedkwargs': \{\},
   'prob': 0.005\}\},
 'sine\_62': \{'faults': \{\},
  'properties': \{'type': 'nominal',
   'time': 0.0,
   'name': 'sine\_62',
   'rangeid': 'sine',
   'params': \{'linetype': 'sine',
    'amp': 3.0,
    'period': 0.3141592653589793,
    'initangle': 54.0,
    'end': [20, 0.0]\},
   'inputparams': \{'amp': 3.0, 'wavelength': 20\},
   'modelparams': \{'seed': 899083440\},
   'paramfunc': <function \_\_main\_\_.gen\_params(linetype, **kwargs)>,
   'fixedargs': ('sine',),
   'fixedkwargs': \{\},
   'prob': 0.005\}\},
 'sine\_63': \{'faults': \{\},
  'properties': \{'type': 'nominal',
   'time': 0.0,
   'name': 'sine\_63',
   'rangeid': 'sine',
   'params': \{'linetype': 'sine',
    'amp': 3.0,
    'period': 0.20943951023931953,
    'initangle': 36.0,
    'end': [30, 0.0]\},
   'inputparams': \{'amp': 3.0, 'wavelength': 30\},
   'modelparams': \{'seed': 899083440\},
   'paramfunc': <function \_\_main\_\_.gen\_params(linetype, **kwargs)>,
   'fixedargs': ('sine',),
   'fixedkwargs': \{\},
   'prob': 0.005\}\},
 'sine\_64': \{'faults': \{\},
  'properties': \{'type': 'nominal',
   'time': 0.0,
   'name': 'sine\_64',
   'rangeid': 'sine',
   'params': \{'linetype': 'sine',
    'amp': 3.0,
    'period': 0.15707963267948966,
    'initangle': 27.0,
    'end': [40, 0.0]\},
   'inputparams': \{'amp': 3.0, 'wavelength': 40\},
   'modelparams': \{'seed': 899083440\},
   'paramfunc': <function \_\_main\_\_.gen\_params(linetype, **kwargs)>,
   'fixedargs': ('sine',),
   'fixedkwargs': \{\},
   'prob': 0.005\}\},
 'sine\_65': \{'faults': \{\},
  'properties': \{'type': 'nominal',
   'time': 0.0,
   'name': 'sine\_65',
   'rangeid': 'sine',
   'params': \{'linetype': 'sine',
    'amp': 3.2,
    'period': 0.6283185307179586,
    'initangle': 115.20000000000002,
    'end': [10, 0.0]\},
   'inputparams': \{'amp': 3.2, 'wavelength': 10\},
   'modelparams': \{'seed': 899083440\},
   'paramfunc': <function \_\_main\_\_.gen\_params(linetype, **kwargs)>,
   'fixedargs': ('sine',),
   'fixedkwargs': \{\},
   'prob': 0.005\}\},
 'sine\_66': \{'faults': \{\},
  'properties': \{'type': 'nominal',
   'time': 0.0,
   'name': 'sine\_66',
   'rangeid': 'sine',
   'params': \{'linetype': 'sine',
    'amp': 3.2,
    'period': 0.3141592653589793,
    'initangle': 57.60000000000001,
    'end': [20, 0.0]\},
   'inputparams': \{'amp': 3.2, 'wavelength': 20\},
   'modelparams': \{'seed': 899083440\},
   'paramfunc': <function \_\_main\_\_.gen\_params(linetype, **kwargs)>,
   'fixedargs': ('sine',),
   'fixedkwargs': \{\},
   'prob': 0.005\}\},
 'sine\_67': \{'faults': \{\},
  'properties': \{'type': 'nominal',
   'time': 0.0,
   'name': 'sine\_67',
   'rangeid': 'sine',
   'params': \{'linetype': 'sine',
    'amp': 3.2,
    'period': 0.20943951023931953,
    'initangle': 38.400000000000006,
    'end': [30, 0.0]\},
   'inputparams': \{'amp': 3.2, 'wavelength': 30\},
   'modelparams': \{'seed': 899083440\},
   'paramfunc': <function \_\_main\_\_.gen\_params(linetype, **kwargs)>,
   'fixedargs': ('sine',),
   'fixedkwargs': \{\},
   'prob': 0.005\}\},
 'sine\_68': \{'faults': \{\},
  'properties': \{'type': 'nominal',
   'time': 0.0,
   'name': 'sine\_68',
   'rangeid': 'sine',
   'params': \{'linetype': 'sine',
    'amp': 3.2,
    'period': 0.15707963267948966,
    'initangle': 28.800000000000004,
    'end': [40, 0.0]\},
   'inputparams': \{'amp': 3.2, 'wavelength': 40\},
   'modelparams': \{'seed': 899083440\},
   'paramfunc': <function \_\_main\_\_.gen\_params(linetype, **kwargs)>,
   'fixedargs': ('sine',),
   'fixedkwargs': \{\},
   'prob': 0.005\}\},
 'sine\_69': \{'faults': \{\},
  'properties': \{'type': 'nominal',
   'time': 0.0,
   'name': 'sine\_69',
   'rangeid': 'sine',
   'params': \{'linetype': 'sine',
    'amp': 3.4000000000000004,
    'period': 0.6283185307179586,
    'initangle': 122.40000000000002,
    'end': [10, 0.0]\},
   'inputparams': \{'amp': 3.4000000000000004, 'wavelength': 10\},
   'modelparams': \{'seed': 899083440\},
   'paramfunc': <function \_\_main\_\_.gen\_params(linetype, **kwargs)>,
   'fixedargs': ('sine',),
   'fixedkwargs': \{\},
   'prob': 0.005\}\},
 'sine\_70': \{'faults': \{\},
  'properties': \{'type': 'nominal',
   'time': 0.0,
   'name': 'sine\_70',
   'rangeid': 'sine',
   'params': \{'linetype': 'sine',
    'amp': 3.4000000000000004,
    'period': 0.3141592653589793,
    'initangle': 61.20000000000001,
    'end': [20, 0.0]\},
   'inputparams': \{'amp': 3.4000000000000004, 'wavelength': 20\},
   'modelparams': \{'seed': 899083440\},
   'paramfunc': <function \_\_main\_\_.gen\_params(linetype, **kwargs)>,
   'fixedargs': ('sine',),
   'fixedkwargs': \{\},
   'prob': 0.005\}\},
 'sine\_71': \{'faults': \{\},
  'properties': \{'type': 'nominal',
   'time': 0.0,
   'name': 'sine\_71',
   'rangeid': 'sine',
   'params': \{'linetype': 'sine',
    'amp': 3.4000000000000004,
    'period': 0.20943951023931953,
    'initangle': 40.800000000000004,
    'end': [30, 0.0]\},
   'inputparams': \{'amp': 3.4000000000000004, 'wavelength': 30\},
   'modelparams': \{'seed': 899083440\},
   'paramfunc': <function \_\_main\_\_.gen\_params(linetype, **kwargs)>,
   'fixedargs': ('sine',),
   'fixedkwargs': \{\},
   'prob': 0.005\}\},
 'sine\_72': \{'faults': \{\},
  'properties': \{'type': 'nominal',
   'time': 0.0,
   'name': 'sine\_72',
   'rangeid': 'sine',
   'params': \{'linetype': 'sine',
    'amp': 3.4000000000000004,
    'period': 0.15707963267948966,
    'initangle': 30.600000000000005,
    'end': [40, 0.0]\},
   'inputparams': \{'amp': 3.4000000000000004, 'wavelength': 40\},
   'modelparams': \{'seed': 899083440\},
   'paramfunc': <function \_\_main\_\_.gen\_params(linetype, **kwargs)>,
   'fixedargs': ('sine',),
   'fixedkwargs': \{\},
   'prob': 0.005\}\},
 'sine\_73': \{'faults': \{\},
  'properties': \{'type': 'nominal',
   'time': 0.0,
   'name': 'sine\_73',
   'rangeid': 'sine',
   'params': \{'linetype': 'sine',
    'amp': 3.6,
    'period': 0.6283185307179586,
    'initangle': 129.60000000000002,
    'end': [10, 0.0]\},
   'inputparams': \{'amp': 3.6, 'wavelength': 10\},
   'modelparams': \{'seed': 899083440\},
   'paramfunc': <function \_\_main\_\_.gen\_params(linetype, **kwargs)>,
   'fixedargs': ('sine',),
   'fixedkwargs': \{\},
   'prob': 0.005\}\},
 'sine\_74': \{'faults': \{\},
  'properties': \{'type': 'nominal',
   'time': 0.0,
   'name': 'sine\_74',
   'rangeid': 'sine',
   'params': \{'linetype': 'sine',
    'amp': 3.6,
    'period': 0.3141592653589793,
    'initangle': 64.80000000000001,
    'end': [20, 0.0]\},
   'inputparams': \{'amp': 3.6, 'wavelength': 20\},
   'modelparams': \{'seed': 899083440\},
   'paramfunc': <function \_\_main\_\_.gen\_params(linetype, **kwargs)>,
   'fixedargs': ('sine',),
   'fixedkwargs': \{\},
   'prob': 0.005\}\},
 'sine\_75': \{'faults': \{\},
  'properties': \{'type': 'nominal',
   'time': 0.0,
   'name': 'sine\_75',
   'rangeid': 'sine',
   'params': \{'linetype': 'sine',
    'amp': 3.6,
    'period': 0.20943951023931953,
    'initangle': 43.2,
    'end': [30, 0.0]\},
   'inputparams': \{'amp': 3.6, 'wavelength': 30\},
   'modelparams': \{'seed': 899083440\},
   'paramfunc': <function \_\_main\_\_.gen\_params(linetype, **kwargs)>,
   'fixedargs': ('sine',),
   'fixedkwargs': \{\},
   'prob': 0.005\}\},
 'sine\_76': \{'faults': \{\},
  'properties': \{'type': 'nominal',
   'time': 0.0,
   'name': 'sine\_76',
   'rangeid': 'sine',
   'params': \{'linetype': 'sine',
    'amp': 3.6,
    'period': 0.15707963267948966,
    'initangle': 32.400000000000006,
    'end': [40, 0.0]\},
   'inputparams': \{'amp': 3.6, 'wavelength': 40\},
   'modelparams': \{'seed': 899083440\},
   'paramfunc': <function \_\_main\_\_.gen\_params(linetype, **kwargs)>,
   'fixedargs': ('sine',),
   'fixedkwargs': \{\},
   'prob': 0.005\}\},
 'sine\_77': \{'faults': \{\},
  'properties': \{'type': 'nominal',
   'time': 0.0,
   'name': 'sine\_77',
   'rangeid': 'sine',
   'params': \{'linetype': 'sine',
    'amp': 3.8000000000000003,
    'period': 0.6283185307179586,
    'initangle': 136.8,
    'end': [10, 0.0]\},
   'inputparams': \{'amp': 3.8000000000000003, 'wavelength': 10\},
   'modelparams': \{'seed': 899083440\},
   'paramfunc': <function \_\_main\_\_.gen\_params(linetype, **kwargs)>,
   'fixedargs': ('sine',),
   'fixedkwargs': \{\},
   'prob': 0.005\}\},
 'sine\_78': \{'faults': \{\},
  'properties': \{'type': 'nominal',
   'time': 0.0,
   'name': 'sine\_78',
   'rangeid': 'sine',
   'params': \{'linetype': 'sine',
    'amp': 3.8000000000000003,
    'period': 0.3141592653589793,
    'initangle': 68.4,
    'end': [20, 0.0]\},
   'inputparams': \{'amp': 3.8000000000000003, 'wavelength': 20\},
   'modelparams': \{'seed': 899083440\},
   'paramfunc': <function \_\_main\_\_.gen\_params(linetype, **kwargs)>,
   'fixedargs': ('sine',),
   'fixedkwargs': \{\},
   'prob': 0.005\}\},
 'sine\_79': \{'faults': \{\},
  'properties': \{'type': 'nominal',
   'time': 0.0,
   'name': 'sine\_79',
   'rangeid': 'sine',
   'params': \{'linetype': 'sine',
    'amp': 3.8000000000000003,
    'period': 0.20943951023931953,
    'initangle': 45.6,
    'end': [30, 0.0]\},
   'inputparams': \{'amp': 3.8000000000000003, 'wavelength': 30\},
   'modelparams': \{'seed': 899083440\},
   'paramfunc': <function \_\_main\_\_.gen\_params(linetype, **kwargs)>,
   'fixedargs': ('sine',),
   'fixedkwargs': \{\},
   'prob': 0.005\}\},
 'sine\_80': \{'faults': \{\},
  'properties': \{'type': 'nominal',
   'time': 0.0,
   'name': 'sine\_80',
   'rangeid': 'sine',
   'params': \{'linetype': 'sine',
    'amp': 3.8000000000000003,
    'period': 0.15707963267948966,
    'initangle': 34.2,
    'end': [40, 0.0]\},
   'inputparams': \{'amp': 3.8000000000000003, 'wavelength': 40\},
   'modelparams': \{'seed': 899083440\},
   'paramfunc': <function \_\_main\_\_.gen\_params(linetype, **kwargs)>,
   'fixedargs': ('sine',),
   'fixedkwargs': \{\},
   'prob': 0.005\}\},
 'sine\_81': \{'faults': \{\},
  'properties': \{'type': 'nominal',
   'time': 0.0,
   'name': 'sine\_81',
   'rangeid': 'sine',
   'params': \{'linetype': 'sine',
    'amp': 4.0,
    'period': 0.6283185307179586,
    'initangle': 144.0,
    'end': [10, 0.0]\},
   'inputparams': \{'amp': 4.0, 'wavelength': 10\},
   'modelparams': \{'seed': 899083440\},
   'paramfunc': <function \_\_main\_\_.gen\_params(linetype, **kwargs)>,
   'fixedargs': ('sine',),
   'fixedkwargs': \{\},
   'prob': 0.005\}\},
 'sine\_82': \{'faults': \{\},
  'properties': \{'type': 'nominal',
   'time': 0.0,
   'name': 'sine\_82',
   'rangeid': 'sine',
   'params': \{'linetype': 'sine',
    'amp': 4.0,
    'period': 0.3141592653589793,
    'initangle': 72.0,
    'end': [20, 0.0]\},
   'inputparams': \{'amp': 4.0, 'wavelength': 20\},
   'modelparams': \{'seed': 899083440\},
   'paramfunc': <function \_\_main\_\_.gen\_params(linetype, **kwargs)>,
   'fixedargs': ('sine',),
   'fixedkwargs': \{\},
   'prob': 0.005\}\},
 'sine\_83': \{'faults': \{\},
  'properties': \{'type': 'nominal',
   'time': 0.0,
   'name': 'sine\_83',
   'rangeid': 'sine',
   'params': \{'linetype': 'sine',
    'amp': 4.0,
    'period': 0.20943951023931953,
    'initangle': 48.0,
    'end': [30, 0.0]\},
   'inputparams': \{'amp': 4.0, 'wavelength': 30\},
   'modelparams': \{'seed': 899083440\},
   'paramfunc': <function \_\_main\_\_.gen\_params(linetype, **kwargs)>,
   'fixedargs': ('sine',),
   'fixedkwargs': \{\},
   'prob': 0.005\}\},
 'sine\_84': \{'faults': \{\},
  'properties': \{'type': 'nominal',
   'time': 0.0,
   'name': 'sine\_84',
   'rangeid': 'sine',
   'params': \{'linetype': 'sine',
    'amp': 4.0,
    'period': 0.15707963267948966,
    'initangle': 36.0,
    'end': [40, 0.0]\},
   'inputparams': \{'amp': 4.0, 'wavelength': 40\},
   'modelparams': \{'seed': 899083440\},
   'paramfunc': <function \_\_main\_\_.gen\_params(linetype, **kwargs)>,
   'fixedargs': ('sine',),
   'fixedkwargs': \{\},
   'prob': 0.005\}\},
 'sine\_85': \{'faults': \{\},
  'properties': \{'type': 'nominal',
   'time': 0.0,
   'name': 'sine\_85',
   'rangeid': 'sine',
   'params': \{'linetype': 'sine',
    'amp': 4.2,
    'period': 0.6283185307179586,
    'initangle': 151.20000000000002,
    'end': [10, 0.0]\},
   'inputparams': \{'amp': 4.2, 'wavelength': 10\},
   'modelparams': \{'seed': 899083440\},
   'paramfunc': <function \_\_main\_\_.gen\_params(linetype, **kwargs)>,
   'fixedargs': ('sine',),
   'fixedkwargs': \{\},
   'prob': 0.005\}\},
 'sine\_86': \{'faults': \{\},
  'properties': \{'type': 'nominal',
   'time': 0.0,
   'name': 'sine\_86',
   'rangeid': 'sine',
   'params': \{'linetype': 'sine',
    'amp': 4.2,
    'period': 0.3141592653589793,
    'initangle': 75.60000000000001,
    'end': [20, 0.0]\},
   'inputparams': \{'amp': 4.2, 'wavelength': 20\},
   'modelparams': \{'seed': 899083440\},
   'paramfunc': <function \_\_main\_\_.gen\_params(linetype, **kwargs)>,
   'fixedargs': ('sine',),
   'fixedkwargs': \{\},
   'prob': 0.005\}\},
 'sine\_87': \{'faults': \{\},
  'properties': \{'type': 'nominal',
   'time': 0.0,
   'name': 'sine\_87',
   'rangeid': 'sine',
   'params': \{'linetype': 'sine',
    'amp': 4.2,
    'period': 0.20943951023931953,
    'initangle': 50.400000000000006,
    'end': [30, 0.0]\},
   'inputparams': \{'amp': 4.2, 'wavelength': 30\},
   'modelparams': \{'seed': 899083440\},
   'paramfunc': <function \_\_main\_\_.gen\_params(linetype, **kwargs)>,
   'fixedargs': ('sine',),
   'fixedkwargs': \{\},
   'prob': 0.005\}\},
 'sine\_88': \{'faults': \{\},
  'properties': \{'type': 'nominal',
   'time': 0.0,
   'name': 'sine\_88',
   'rangeid': 'sine',
   'params': \{'linetype': 'sine',
    'amp': 4.2,
    'period': 0.15707963267948966,
    'initangle': 37.800000000000004,
    'end': [40, 0.0]\},
   'inputparams': \{'amp': 4.2, 'wavelength': 40\},
   'modelparams': \{'seed': 899083440\},
   'paramfunc': <function \_\_main\_\_.gen\_params(linetype, **kwargs)>,
   'fixedargs': ('sine',),
   'fixedkwargs': \{\},
   'prob': 0.005\}\},
 'sine\_89': \{'faults': \{\},
  'properties': \{'type': 'nominal',
   'time': 0.0,
   'name': 'sine\_89',
   'rangeid': 'sine',
   'params': \{'linetype': 'sine',
    'amp': 4.4,
    'period': 0.6283185307179586,
    'initangle': 158.40000000000003,
    'end': [10, 0.0]\},
   'inputparams': \{'amp': 4.4, 'wavelength': 10\},
   'modelparams': \{'seed': 899083440\},
   'paramfunc': <function \_\_main\_\_.gen\_params(linetype, **kwargs)>,
   'fixedargs': ('sine',),
   'fixedkwargs': \{\},
   'prob': 0.005\}\},
 'sine\_90': \{'faults': \{\},
  'properties': \{'type': 'nominal',
   'time': 0.0,
   'name': 'sine\_90',
   'rangeid': 'sine',
   'params': \{'linetype': 'sine',
    'amp': 4.4,
    'period': 0.3141592653589793,
    'initangle': 79.20000000000002,
    'end': [20, 0.0]\},
   'inputparams': \{'amp': 4.4, 'wavelength': 20\},
   'modelparams': \{'seed': 899083440\},
   'paramfunc': <function \_\_main\_\_.gen\_params(linetype, **kwargs)>,
   'fixedargs': ('sine',),
   'fixedkwargs': \{\},
   'prob': 0.005\}\},
 'sine\_91': \{'faults': \{\},
  'properties': \{'type': 'nominal',
   'time': 0.0,
   'name': 'sine\_91',
   'rangeid': 'sine',
   'params': \{'linetype': 'sine',
    'amp': 4.4,
    'period': 0.20943951023931953,
    'initangle': 52.80000000000001,
    'end': [30, 0.0]\},
   'inputparams': \{'amp': 4.4, 'wavelength': 30\},
   'modelparams': \{'seed': 899083440\},
   'paramfunc': <function \_\_main\_\_.gen\_params(linetype, **kwargs)>,
   'fixedargs': ('sine',),
   'fixedkwargs': \{\},
   'prob': 0.005\}\},
 'sine\_92': \{'faults': \{\},
  'properties': \{'type': 'nominal',
   'time': 0.0,
   'name': 'sine\_92',
   'rangeid': 'sine',
   'params': \{'linetype': 'sine',
    'amp': 4.4,
    'period': 0.15707963267948966,
    'initangle': 39.60000000000001,
    'end': [40, 0.0]\},
   'inputparams': \{'amp': 4.4, 'wavelength': 40\},
   'modelparams': \{'seed': 899083440\},
   'paramfunc': <function \_\_main\_\_.gen\_params(linetype, **kwargs)>,
   'fixedargs': ('sine',),
   'fixedkwargs': \{\},
   'prob': 0.005\}\},
 'sine\_93': \{'faults': \{\},
  'properties': \{'type': 'nominal',
   'time': 0.0,
   'name': 'sine\_93',
   'rangeid': 'sine',
   'params': \{'linetype': 'sine',
    'amp': 4.6000000000000005,
    'period': 0.6283185307179586,
    'initangle': 165.60000000000002,
    'end': [10, 0.0]\},
   'inputparams': \{'amp': 4.6000000000000005, 'wavelength': 10\},
   'modelparams': \{'seed': 899083440\},
   'paramfunc': <function \_\_main\_\_.gen\_params(linetype, **kwargs)>,
   'fixedargs': ('sine',),
   'fixedkwargs': \{\},
   'prob': 0.005\}\},
 'sine\_94': \{'faults': \{\},
  'properties': \{'type': 'nominal',
   'time': 0.0,
   'name': 'sine\_94',
   'rangeid': 'sine',
   'params': \{'linetype': 'sine',
    'amp': 4.6000000000000005,
    'period': 0.3141592653589793,
    'initangle': 82.80000000000001,
    'end': [20, 0.0]\},
   'inputparams': \{'amp': 4.6000000000000005, 'wavelength': 20\},
   'modelparams': \{'seed': 899083440\},
   'paramfunc': <function \_\_main\_\_.gen\_params(linetype, **kwargs)>,
   'fixedargs': ('sine',),
   'fixedkwargs': \{\},
   'prob': 0.005\}\},
 'sine\_95': \{'faults': \{\},
  'properties': \{'type': 'nominal',
   'time': 0.0,
   'name': 'sine\_95',
   'rangeid': 'sine',
   'params': \{'linetype': 'sine',
    'amp': 4.6000000000000005,
    'period': 0.20943951023931953,
    'initangle': 55.2,
    'end': [30, 0.0]\},
   'inputparams': \{'amp': 4.6000000000000005, 'wavelength': 30\},
   'modelparams': \{'seed': 899083440\},
   'paramfunc': <function \_\_main\_\_.gen\_params(linetype, **kwargs)>,
   'fixedargs': ('sine',),
   'fixedkwargs': \{\},
   'prob': 0.005\}\},
 'sine\_96': \{'faults': \{\},
  'properties': \{'type': 'nominal',
   'time': 0.0,
   'name': 'sine\_96',
   'rangeid': 'sine',
   'params': \{'linetype': 'sine',
    'amp': 4.6000000000000005,
    'period': 0.15707963267948966,
    'initangle': 41.400000000000006,
    'end': [40, 0.0]\},
   'inputparams': \{'amp': 4.6000000000000005, 'wavelength': 40\},
   'modelparams': \{'seed': 899083440\},
   'paramfunc': <function \_\_main\_\_.gen\_params(linetype, **kwargs)>,
   'fixedargs': ('sine',),
   'fixedkwargs': \{\},
   'prob': 0.005\}\},
 'sine\_97': \{'faults': \{\},
  'properties': \{'type': 'nominal',
   'time': 0.0,
   'name': 'sine\_97',
   'rangeid': 'sine',
   'params': \{'linetype': 'sine',
    'amp': 4.800000000000001,
    'period': 0.6283185307179586,
    'initangle': 172.8,
    'end': [10, 0.0]\},
   'inputparams': \{'amp': 4.800000000000001, 'wavelength': 10\},
   'modelparams': \{'seed': 899083440\},
   'paramfunc': <function \_\_main\_\_.gen\_params(linetype, **kwargs)>,
   'fixedargs': ('sine',),
   'fixedkwargs': \{\},
   'prob': 0.005\}\},
 'sine\_98': \{'faults': \{\},
  'properties': \{'type': 'nominal',
   'time': 0.0,
   'name': 'sine\_98',
   'rangeid': 'sine',
   'params': \{'linetype': 'sine',
    'amp': 4.800000000000001,
    'period': 0.3141592653589793,
    'initangle': 86.4,
    'end': [20, 0.0]\},
   'inputparams': \{'amp': 4.800000000000001, 'wavelength': 20\},
   'modelparams': \{'seed': 899083440\},
   'paramfunc': <function \_\_main\_\_.gen\_params(linetype, **kwargs)>,
   'fixedargs': ('sine',),
   'fixedkwargs': \{\},
   'prob': 0.005\}\},
 'sine\_99': \{'faults': \{\},
  'properties': \{'type': 'nominal',
   'time': 0.0,
   'name': 'sine\_99',
   'rangeid': 'sine',
   'params': \{'linetype': 'sine',
    'amp': 4.800000000000001,
    'period': 0.20943951023931953,
    'initangle': 57.60000000000001,
    'end': [30, 0.0]\},
   'inputparams': \{'amp': 4.800000000000001, 'wavelength': 30\},
   'modelparams': \{'seed': 899083440\},
   'paramfunc': <function \_\_main\_\_.gen\_params(linetype, **kwargs)>,
   'fixedargs': ('sine',),
   'fixedkwargs': \{\},
   'prob': 0.005\}\},
 'sine\_100': \{'faults': \{\},
  'properties': \{'type': 'nominal',
   'time': 0.0,
   'name': 'sine\_100',
   'rangeid': 'sine',
   'params': \{'linetype': 'sine',
    'amp': 4.800000000000001,
    'period': 0.15707963267948966,
    'initangle': 43.2,
    'end': [40, 0.0]\},
   'inputparams': \{'amp': 4.800000000000001, 'wavelength': 40\},
   'modelparams': \{'seed': 899083440\},
   'paramfunc': <function \_\_main\_\_.gen\_params(linetype, **kwargs)>,
   'fixedargs': ('sine',),
   'fixedkwargs': \{\},
   'prob': 0.005\}\},
 'sine\_101': \{'faults': \{\},
  'properties': \{'type': 'nominal',
   'time': 0.0,
   'name': 'sine\_101',
   'rangeid': 'sine',
   'params': \{'linetype': 'sine',
    'amp': 5.0,
    'period': 0.6283185307179586,
    'initangle': 180.0,
    'end': [10, 0.0]\},
   'inputparams': \{'amp': 5.0, 'wavelength': 10\},
   'modelparams': \{'seed': 899083440\},
   'paramfunc': <function \_\_main\_\_.gen\_params(linetype, **kwargs)>,
   'fixedargs': ('sine',),
   'fixedkwargs': \{\},
   'prob': 0.005\}\},
 'sine\_102': \{'faults': \{\},
  'properties': \{'type': 'nominal',
   'time': 0.0,
   'name': 'sine\_102',
   'rangeid': 'sine',
   'params': \{'linetype': 'sine',
    'amp': 5.0,
    'period': 0.3141592653589793,
    'initangle': 90.0,
    'end': [20, 0.0]\},
   'inputparams': \{'amp': 5.0, 'wavelength': 20\},
   'modelparams': \{'seed': 899083440\},
   'paramfunc': <function \_\_main\_\_.gen\_params(linetype, **kwargs)>,
   'fixedargs': ('sine',),
   'fixedkwargs': \{\},
   'prob': 0.005\}\},
 'sine\_103': \{'faults': \{\},
  'properties': \{'type': 'nominal',
   'time': 0.0,
   'name': 'sine\_103',
   'rangeid': 'sine',
   'params': \{'linetype': 'sine',
    'amp': 5.0,
    'period': 0.20943951023931953,
    'initangle': 59.99999999999999,
    'end': [30, 0.0]\},
   'inputparams': \{'amp': 5.0, 'wavelength': 30\},
   'modelparams': \{'seed': 899083440\},
   'paramfunc': <function \_\_main\_\_.gen\_params(linetype, **kwargs)>,
   'fixedargs': ('sine',),
   'fixedkwargs': \{\},
   'prob': 0.005\}\},
 'sine\_104': \{'faults': \{\},
  'properties': \{'type': 'nominal',
   'time': 0.0,
   'name': 'sine\_104',
   'rangeid': 'sine',
   'params': \{'linetype': 'sine',
    'amp': 5.0,
    'period': 0.15707963267948966,
    'initangle': 45.0,
    'end': [40, 0.0]\},
   'inputparams': \{'amp': 5.0, 'wavelength': 40\},
   'modelparams': \{'seed': 899083440\},
   'paramfunc': <function \_\_main\_\_.gen\_params(linetype, **kwargs)>,
   'fixedargs': ('sine',),
   'fixedkwargs': \{\},
   'prob': 0.005\}\},
 'sine\_105': \{'faults': \{\},
  'properties': \{'type': 'nominal',
   'time': 0.0,
   'name': 'sine\_105',
   'rangeid': 'sine',
   'params': \{'linetype': 'sine',
    'amp': 5.2,
    'period': 0.6283185307179586,
    'initangle': 187.20000000000002,
    'end': [10, 0.0]\},
   'inputparams': \{'amp': 5.2, 'wavelength': 10\},
   'modelparams': \{'seed': 899083440\},
   'paramfunc': <function \_\_main\_\_.gen\_params(linetype, **kwargs)>,
   'fixedargs': ('sine',),
   'fixedkwargs': \{\},
   'prob': 0.005\}\},
 'sine\_106': \{'faults': \{\},
  'properties': \{'type': 'nominal',
   'time': 0.0,
   'name': 'sine\_106',
   'rangeid': 'sine',
   'params': \{'linetype': 'sine',
    'amp': 5.2,
    'period': 0.3141592653589793,
    'initangle': 93.60000000000001,
    'end': [20, 0.0]\},
   'inputparams': \{'amp': 5.2, 'wavelength': 20\},
   'modelparams': \{'seed': 899083440\},
   'paramfunc': <function \_\_main\_\_.gen\_params(linetype, **kwargs)>,
   'fixedargs': ('sine',),
   'fixedkwargs': \{\},
   'prob': 0.005\}\},
 'sine\_107': \{'faults': \{\},
  'properties': \{'type': 'nominal',
   'time': 0.0,
   'name': 'sine\_107',
   'rangeid': 'sine',
   'params': \{'linetype': 'sine',
    'amp': 5.2,
    'period': 0.20943951023931953,
    'initangle': 62.4,
    'end': [30, 0.0]\},
   'inputparams': \{'amp': 5.2, 'wavelength': 30\},
   'modelparams': \{'seed': 899083440\},
   'paramfunc': <function \_\_main\_\_.gen\_params(linetype, **kwargs)>,
   'fixedargs': ('sine',),
   'fixedkwargs': \{\},
   'prob': 0.005\}\},
 'sine\_108': \{'faults': \{\},
  'properties': \{'type': 'nominal',
   'time': 0.0,
   'name': 'sine\_108',
   'rangeid': 'sine',
   'params': \{'linetype': 'sine',
    'amp': 5.2,
    'period': 0.15707963267948966,
    'initangle': 46.800000000000004,
    'end': [40, 0.0]\},
   'inputparams': \{'amp': 5.2, 'wavelength': 40\},
   'modelparams': \{'seed': 899083440\},
   'paramfunc': <function \_\_main\_\_.gen\_params(linetype, **kwargs)>,
   'fixedargs': ('sine',),
   'fixedkwargs': \{\},
   'prob': 0.005\}\},
 'sine\_109': \{'faults': \{\},
  'properties': \{'type': 'nominal',
   'time': 0.0,
   'name': 'sine\_109',
   'rangeid': 'sine',
   'params': \{'linetype': 'sine',
    'amp': 5.4,
    'period': 0.6283185307179586,
    'initangle': 194.4,
    'end': [10, 0.0]\},
   'inputparams': \{'amp': 5.4, 'wavelength': 10\},
   'modelparams': \{'seed': 899083440\},
   'paramfunc': <function \_\_main\_\_.gen\_params(linetype, **kwargs)>,
   'fixedargs': ('sine',),
   'fixedkwargs': \{\},
   'prob': 0.005\}\},
 'sine\_110': \{'faults': \{\},
  'properties': \{'type': 'nominal',
   'time': 0.0,
   'name': 'sine\_110',
   'rangeid': 'sine',
   'params': \{'linetype': 'sine',
    'amp': 5.4,
    'period': 0.3141592653589793,
    'initangle': 97.2,
    'end': [20, 0.0]\},
   'inputparams': \{'amp': 5.4, 'wavelength': 20\},
   'modelparams': \{'seed': 899083440\},
   'paramfunc': <function \_\_main\_\_.gen\_params(linetype, **kwargs)>,
   'fixedargs': ('sine',),
   'fixedkwargs': \{\},
   'prob': 0.005\}\},
 'sine\_111': \{'faults': \{\},
  'properties': \{'type': 'nominal',
   'time': 0.0,
   'name': 'sine\_111',
   'rangeid': 'sine',
   'params': \{'linetype': 'sine',
    'amp': 5.4,
    'period': 0.20943951023931953,
    'initangle': 64.80000000000001,
    'end': [30, 0.0]\},
   'inputparams': \{'amp': 5.4, 'wavelength': 30\},
   'modelparams': \{'seed': 899083440\},
   'paramfunc': <function \_\_main\_\_.gen\_params(linetype, **kwargs)>,
   'fixedargs': ('sine',),
   'fixedkwargs': \{\},
   'prob': 0.005\}\},
 'sine\_112': \{'faults': \{\},
  'properties': \{'type': 'nominal',
   'time': 0.0,
   'name': 'sine\_112',
   'rangeid': 'sine',
   'params': \{'linetype': 'sine',
    'amp': 5.4,
    'period': 0.15707963267948966,
    'initangle': 48.6,
    'end': [40, 0.0]\},
   'inputparams': \{'amp': 5.4, 'wavelength': 40\},
   'modelparams': \{'seed': 899083440\},
   'paramfunc': <function \_\_main\_\_.gen\_params(linetype, **kwargs)>,
   'fixedargs': ('sine',),
   'fixedkwargs': \{\},
   'prob': 0.005\}\},
 'sine\_113': \{'faults': \{\},
  'properties': \{'type': 'nominal',
   'time': 0.0,
   'name': 'sine\_113',
   'rangeid': 'sine',
   'params': \{'linetype': 'sine',
    'amp': 5.6000000000000005,
    'period': 0.6283185307179586,
    'initangle': 201.60000000000002,
    'end': [10, 0.0]\},
   'inputparams': \{'amp': 5.6000000000000005, 'wavelength': 10\},
   'modelparams': \{'seed': 899083440\},
   'paramfunc': <function \_\_main\_\_.gen\_params(linetype, **kwargs)>,
   'fixedargs': ('sine',),
   'fixedkwargs': \{\},
   'prob': 0.005\}\},
 'sine\_114': \{'faults': \{\},
  'properties': \{'type': 'nominal',
   'time': 0.0,
   'name': 'sine\_114',
   'rangeid': 'sine',
   'params': \{'linetype': 'sine',
    'amp': 5.6000000000000005,
    'period': 0.3141592653589793,
    'initangle': 100.80000000000001,
    'end': [20, 0.0]\},
   'inputparams': \{'amp': 5.6000000000000005, 'wavelength': 20\},
   'modelparams': \{'seed': 899083440\},
   'paramfunc': <function \_\_main\_\_.gen\_params(linetype, **kwargs)>,
   'fixedargs': ('sine',),
   'fixedkwargs': \{\},
   'prob': 0.005\}\},
 'sine\_115': \{'faults': \{\},
  'properties': \{'type': 'nominal',
   'time': 0.0,
   'name': 'sine\_115',
   'rangeid': 'sine',
   'params': \{'linetype': 'sine',
    'amp': 5.6000000000000005,
    'period': 0.20943951023931953,
    'initangle': 67.2,
    'end': [30, 0.0]\},
   'inputparams': \{'amp': 5.6000000000000005, 'wavelength': 30\},
   'modelparams': \{'seed': 899083440\},
   'paramfunc': <function \_\_main\_\_.gen\_params(linetype, **kwargs)>,
   'fixedargs': ('sine',),
   'fixedkwargs': \{\},
   'prob': 0.005\}\},
 'sine\_116': \{'faults': \{\},
  'properties': \{'type': 'nominal',
   'time': 0.0,
   'name': 'sine\_116',
   'rangeid': 'sine',
   'params': \{'linetype': 'sine',
    'amp': 5.6000000000000005,
    'period': 0.15707963267948966,
    'initangle': 50.400000000000006,
    'end': [40, 0.0]\},
   'inputparams': \{'amp': 5.6000000000000005, 'wavelength': 40\},
   'modelparams': \{'seed': 899083440\},
   'paramfunc': <function \_\_main\_\_.gen\_params(linetype, **kwargs)>,
   'fixedargs': ('sine',),
   'fixedkwargs': \{\},
   'prob': 0.005\}\},
 'sine\_117': \{'faults': \{\},
  'properties': \{'type': 'nominal',
   'time': 0.0,
   'name': 'sine\_117',
   'rangeid': 'sine',
   'params': \{'linetype': 'sine',
    'amp': 5.800000000000001,
    'period': 0.6283185307179586,
    'initangle': 208.8,
    'end': [10, 0.0]\},
   'inputparams': \{'amp': 5.800000000000001, 'wavelength': 10\},
   'modelparams': \{'seed': 899083440\},
   'paramfunc': <function \_\_main\_\_.gen\_params(linetype, **kwargs)>,
   'fixedargs': ('sine',),
   'fixedkwargs': \{\},
   'prob': 0.005\}\},
 'sine\_118': \{'faults': \{\},
  'properties': \{'type': 'nominal',
   'time': 0.0,
   'name': 'sine\_118',
   'rangeid': 'sine',
   'params': \{'linetype': 'sine',
    'amp': 5.800000000000001,
    'period': 0.3141592653589793,
    'initangle': 104.4,
    'end': [20, 0.0]\},
   'inputparams': \{'amp': 5.800000000000001, 'wavelength': 20\},
   'modelparams': \{'seed': 899083440\},
   'paramfunc': <function \_\_main\_\_.gen\_params(linetype, **kwargs)>,
   'fixedargs': ('sine',),
   'fixedkwargs': \{\},
   'prob': 0.005\}\},
 'sine\_119': \{'faults': \{\},
  'properties': \{'type': 'nominal',
   'time': 0.0,
   'name': 'sine\_119',
   'rangeid': 'sine',
   'params': \{'linetype': 'sine',
    'amp': 5.800000000000001,
    'period': 0.20943951023931953,
    'initangle': 69.60000000000001,
    'end': [30, 0.0]\},
   'inputparams': \{'amp': 5.800000000000001, 'wavelength': 30\},
   'modelparams': \{'seed': 899083440\},
   'paramfunc': <function \_\_main\_\_.gen\_params(linetype, **kwargs)>,
   'fixedargs': ('sine',),
   'fixedkwargs': \{\},
   'prob': 0.005\}\},
 'sine\_120': \{'faults': \{\},
  'properties': \{'type': 'nominal',
   'time': 0.0,
   'name': 'sine\_120',
   'rangeid': 'sine',
   'params': \{'linetype': 'sine',
    'amp': 5.800000000000001,
    'period': 0.15707963267948966,
    'initangle': 52.2,
    'end': [40, 0.0]\},
   'inputparams': \{'amp': 5.800000000000001, 'wavelength': 40\},
   'modelparams': \{'seed': 899083440\},
   'paramfunc': <function \_\_main\_\_.gen\_params(linetype, **kwargs)>,
   'fixedargs': ('sine',),
   'fixedkwargs': \{\},
   'prob': 0.005\}\},
 'sine\_121': \{'faults': \{\},
  'properties': \{'type': 'nominal',
   'time': 0.0,
   'name': 'sine\_121',
   'rangeid': 'sine',
   'params': \{'linetype': 'sine',
    'amp': 6.0,
    'period': 0.6283185307179586,
    'initangle': 216.0,
    'end': [10, 0.0]\},
   'inputparams': \{'amp': 6.0, 'wavelength': 10\},
   'modelparams': \{'seed': 899083440\},
   'paramfunc': <function \_\_main\_\_.gen\_params(linetype, **kwargs)>,
   'fixedargs': ('sine',),
   'fixedkwargs': \{\},
   'prob': 0.005\}\},
 'sine\_122': \{'faults': \{\},
  'properties': \{'type': 'nominal',
   'time': 0.0,
   'name': 'sine\_122',
   'rangeid': 'sine',
   'params': \{'linetype': 'sine',
    'amp': 6.0,
    'period': 0.3141592653589793,
    'initangle': 108.0,
    'end': [20, 0.0]\},
   'inputparams': \{'amp': 6.0, 'wavelength': 20\},
   'modelparams': \{'seed': 899083440\},
   'paramfunc': <function \_\_main\_\_.gen\_params(linetype, **kwargs)>,
   'fixedargs': ('sine',),
   'fixedkwargs': \{\},
   'prob': 0.005\}\},
 'sine\_123': \{'faults': \{\},
  'properties': \{'type': 'nominal',
   'time': 0.0,
   'name': 'sine\_123',
   'rangeid': 'sine',
   'params': \{'linetype': 'sine',
    'amp': 6.0,
    'period': 0.20943951023931953,
    'initangle': 72.0,
    'end': [30, 0.0]\},
   'inputparams': \{'amp': 6.0, 'wavelength': 30\},
   'modelparams': \{'seed': 899083440\},
   'paramfunc': <function \_\_main\_\_.gen\_params(linetype, **kwargs)>,
   'fixedargs': ('sine',),
   'fixedkwargs': \{\},
   'prob': 0.005\}\},
 'sine\_124': \{'faults': \{\},
  'properties': \{'type': 'nominal',
   'time': 0.0,
   'name': 'sine\_124',
   'rangeid': 'sine',
   'params': \{'linetype': 'sine',
    'amp': 6.0,
    'period': 0.15707963267948966,
    'initangle': 54.0,
    'end': [40, 0.0]\},
   'inputparams': \{'amp': 6.0, 'wavelength': 40\},
   'modelparams': \{'seed': 899083440\},
   'paramfunc': <function \_\_main\_\_.gen\_params(linetype, **kwargs)>,
   'fixedargs': ('sine',),
   'fixedkwargs': \{\},
   'prob': 0.005\}\},
 'sine\_125': \{'faults': \{\},
  'properties': \{'type': 'nominal',
   'time': 0.0,
   'name': 'sine\_125',
   'rangeid': 'sine',
   'params': \{'linetype': 'sine',
    'amp': 6.2,
    'period': 0.6283185307179586,
    'initangle': 223.2,
    'end': [10, 0.0]\},
   'inputparams': \{'amp': 6.2, 'wavelength': 10\},
   'modelparams': \{'seed': 899083440\},
   'paramfunc': <function \_\_main\_\_.gen\_params(linetype, **kwargs)>,
   'fixedargs': ('sine',),
   'fixedkwargs': \{\},
   'prob': 0.005\}\},
 'sine\_126': \{'faults': \{\},
  'properties': \{'type': 'nominal',
   'time': 0.0,
   'name': 'sine\_126',
   'rangeid': 'sine',
   'params': \{'linetype': 'sine',
    'amp': 6.2,
    'period': 0.3141592653589793,
    'initangle': 111.6,
    'end': [20, 0.0]\},
   'inputparams': \{'amp': 6.2, 'wavelength': 20\},
   'modelparams': \{'seed': 899083440\},
   'paramfunc': <function \_\_main\_\_.gen\_params(linetype, **kwargs)>,
   'fixedargs': ('sine',),
   'fixedkwargs': \{\},
   'prob': 0.005\}\},
 'sine\_127': \{'faults': \{\},
  'properties': \{'type': 'nominal',
   'time': 0.0,
   'name': 'sine\_127',
   'rangeid': 'sine',
   'params': \{'linetype': 'sine',
    'amp': 6.2,
    'period': 0.20943951023931953,
    'initangle': 74.4,
    'end': [30, 0.0]\},
   'inputparams': \{'amp': 6.2, 'wavelength': 30\},
   'modelparams': \{'seed': 899083440\},
   'paramfunc': <function \_\_main\_\_.gen\_params(linetype, **kwargs)>,
   'fixedargs': ('sine',),
   'fixedkwargs': \{\},
   'prob': 0.005\}\},
 'sine\_128': \{'faults': \{\},
  'properties': \{'type': 'nominal',
   'time': 0.0,
   'name': 'sine\_128',
   'rangeid': 'sine',
   'params': \{'linetype': 'sine',
    'amp': 6.2,
    'period': 0.15707963267948966,
    'initangle': 55.8,
    'end': [40, 0.0]\},
   'inputparams': \{'amp': 6.2, 'wavelength': 40\},
   'modelparams': \{'seed': 899083440\},
   'paramfunc': <function \_\_main\_\_.gen\_params(linetype, **kwargs)>,
   'fixedargs': ('sine',),
   'fixedkwargs': \{\},
   'prob': 0.005\}\},
 'sine\_129': \{'faults': \{\},
  'properties': \{'type': 'nominal',
   'time': 0.0,
   'name': 'sine\_129',
   'rangeid': 'sine',
   'params': \{'linetype': 'sine',
    'amp': 6.4,
    'period': 0.6283185307179586,
    'initangle': 230.40000000000003,
    'end': [10, 0.0]\},
   'inputparams': \{'amp': 6.4, 'wavelength': 10\},
   'modelparams': \{'seed': 899083440\},
   'paramfunc': <function \_\_main\_\_.gen\_params(linetype, **kwargs)>,
   'fixedargs': ('sine',),
   'fixedkwargs': \{\},
   'prob': 0.005\}\},
 'sine\_130': \{'faults': \{\},
  'properties': \{'type': 'nominal',
   'time': 0.0,
   'name': 'sine\_130',
   'rangeid': 'sine',
   'params': \{'linetype': 'sine',
    'amp': 6.4,
    'period': 0.3141592653589793,
    'initangle': 115.20000000000002,
    'end': [20, 0.0]\},
   'inputparams': \{'amp': 6.4, 'wavelength': 20\},
   'modelparams': \{'seed': 899083440\},
   'paramfunc': <function \_\_main\_\_.gen\_params(linetype, **kwargs)>,
   'fixedargs': ('sine',),
   'fixedkwargs': \{\},
   'prob': 0.005\}\},
 'sine\_131': \{'faults': \{\},
  'properties': \{'type': 'nominal',
   'time': 0.0,
   'name': 'sine\_131',
   'rangeid': 'sine',
   'params': \{'linetype': 'sine',
    'amp': 6.4,
    'period': 0.20943951023931953,
    'initangle': 76.80000000000001,
    'end': [30, 0.0]\},
   'inputparams': \{'amp': 6.4, 'wavelength': 30\},
   'modelparams': \{'seed': 899083440\},
   'paramfunc': <function \_\_main\_\_.gen\_params(linetype, **kwargs)>,
   'fixedargs': ('sine',),
   'fixedkwargs': \{\},
   'prob': 0.005\}\},
 'sine\_132': \{'faults': \{\},
  'properties': \{'type': 'nominal',
   'time': 0.0,
   'name': 'sine\_132',
   'rangeid': 'sine',
   'params': \{'linetype': 'sine',
    'amp': 6.4,
    'period': 0.15707963267948966,
    'initangle': 57.60000000000001,
    'end': [40, 0.0]\},
   'inputparams': \{'amp': 6.4, 'wavelength': 40\},
   'modelparams': \{'seed': 899083440\},
   'paramfunc': <function \_\_main\_\_.gen\_params(linetype, **kwargs)>,
   'fixedargs': ('sine',),
   'fixedkwargs': \{\},
   'prob': 0.005\}\},
 'sine\_133': \{'faults': \{\},
  'properties': \{'type': 'nominal',
   'time': 0.0,
   'name': 'sine\_133',
   'rangeid': 'sine',
   'params': \{'linetype': 'sine',
    'amp': 6.6000000000000005,
    'period': 0.6283185307179586,
    'initangle': 237.60000000000005,
    'end': [10, 0.0]\},
   'inputparams': \{'amp': 6.6000000000000005, 'wavelength': 10\},
   'modelparams': \{'seed': 899083440\},
   'paramfunc': <function \_\_main\_\_.gen\_params(linetype, **kwargs)>,
   'fixedargs': ('sine',),
   'fixedkwargs': \{\},
   'prob': 0.005\}\},
 'sine\_134': \{'faults': \{\},
  'properties': \{'type': 'nominal',
   'time': 0.0,
   'name': 'sine\_134',
   'rangeid': 'sine',
   'params': \{'linetype': 'sine',
    'amp': 6.6000000000000005,
    'period': 0.3141592653589793,
    'initangle': 118.80000000000003,
    'end': [20, 0.0]\},
   'inputparams': \{'amp': 6.6000000000000005, 'wavelength': 20\},
   'modelparams': \{'seed': 899083440\},
   'paramfunc': <function \_\_main\_\_.gen\_params(linetype, **kwargs)>,
   'fixedargs': ('sine',),
   'fixedkwargs': \{\},
   'prob': 0.005\}\},
 'sine\_135': \{'faults': \{\},
  'properties': \{'type': 'nominal',
   'time': 0.0,
   'name': 'sine\_135',
   'rangeid': 'sine',
   'params': \{'linetype': 'sine',
    'amp': 6.6000000000000005,
    'period': 0.20943951023931953,
    'initangle': 79.19999999999999,
    'end': [30, 0.0]\},
   'inputparams': \{'amp': 6.6000000000000005, 'wavelength': 30\},
   'modelparams': \{'seed': 899083440\},
   'paramfunc': <function \_\_main\_\_.gen\_params(linetype, **kwargs)>,
   'fixedargs': ('sine',),
   'fixedkwargs': \{\},
   'prob': 0.005\}\},
 'sine\_136': \{'faults': \{\},
  'properties': \{'type': 'nominal',
   'time': 0.0,
   'name': 'sine\_136',
   'rangeid': 'sine',
   'params': \{'linetype': 'sine',
    'amp': 6.6000000000000005,
    'period': 0.15707963267948966,
    'initangle': 59.40000000000001,
    'end': [40, 0.0]\},
   'inputparams': \{'amp': 6.6000000000000005, 'wavelength': 40\},
   'modelparams': \{'seed': 899083440\},
   'paramfunc': <function \_\_main\_\_.gen\_params(linetype, **kwargs)>,
   'fixedargs': ('sine',),
   'fixedkwargs': \{\},
   'prob': 0.005\}\},
 'sine\_137': \{'faults': \{\},
  'properties': \{'type': 'nominal',
   'time': 0.0,
   'name': 'sine\_137',
   'rangeid': 'sine',
   'params': \{'linetype': 'sine',
    'amp': 6.800000000000001,
    'period': 0.6283185307179586,
    'initangle': 244.80000000000004,
    'end': [10, 0.0]\},
   'inputparams': \{'amp': 6.800000000000001, 'wavelength': 10\},
   'modelparams': \{'seed': 899083440\},
   'paramfunc': <function \_\_main\_\_.gen\_params(linetype, **kwargs)>,
   'fixedargs': ('sine',),
   'fixedkwargs': \{\},
   'prob': 0.005\}\},
 'sine\_138': \{'faults': \{\},
  'properties': \{'type': 'nominal',
   'time': 0.0,
   'name': 'sine\_138',
   'rangeid': 'sine',
   'params': \{'linetype': 'sine',
    'amp': 6.800000000000001,
    'period': 0.3141592653589793,
    'initangle': 122.40000000000002,
    'end': [20, 0.0]\},
   'inputparams': \{'amp': 6.800000000000001, 'wavelength': 20\},
   'modelparams': \{'seed': 899083440\},
   'paramfunc': <function \_\_main\_\_.gen\_params(linetype, **kwargs)>,
   'fixedargs': ('sine',),
   'fixedkwargs': \{\},
   'prob': 0.005\}\},
 'sine\_139': \{'faults': \{\},
  'properties': \{'type': 'nominal',
   'time': 0.0,
   'name': 'sine\_139',
   'rangeid': 'sine',
   'params': \{'linetype': 'sine',
    'amp': 6.800000000000001,
    'period': 0.20943951023931953,
    'initangle': 81.60000000000001,
    'end': [30, 0.0]\},
   'inputparams': \{'amp': 6.800000000000001, 'wavelength': 30\},
   'modelparams': \{'seed': 899083440\},
   'paramfunc': <function \_\_main\_\_.gen\_params(linetype, **kwargs)>,
   'fixedargs': ('sine',),
   'fixedkwargs': \{\},
   'prob': 0.005\}\},
 'sine\_140': \{'faults': \{\},
  'properties': \{'type': 'nominal',
   'time': 0.0,
   'name': 'sine\_140',
   'rangeid': 'sine',
   'params': \{'linetype': 'sine',
    'amp': 6.800000000000001,
    'period': 0.15707963267948966,
    'initangle': 61.20000000000001,
    'end': [40, 0.0]\},
   'inputparams': \{'amp': 6.800000000000001, 'wavelength': 40\},
   'modelparams': \{'seed': 899083440\},
   'paramfunc': <function \_\_main\_\_.gen\_params(linetype, **kwargs)>,
   'fixedargs': ('sine',),
   'fixedkwargs': \{\},
   'prob': 0.005\}\},
 'sine\_141': \{'faults': \{\},
  'properties': \{'type': 'nominal',
   'time': 0.0,
   'name': 'sine\_141',
   'rangeid': 'sine',
   'params': \{'linetype': 'sine',
    'amp': 7.0,
    'period': 0.6283185307179586,
    'initangle': 252.0,
    'end': [10, 0.0]\},
   'inputparams': \{'amp': 7.0, 'wavelength': 10\},
   'modelparams': \{'seed': 899083440\},
   'paramfunc': <function \_\_main\_\_.gen\_params(linetype, **kwargs)>,
   'fixedargs': ('sine',),
   'fixedkwargs': \{\},
   'prob': 0.005\}\},
 'sine\_142': \{'faults': \{\},
  'properties': \{'type': 'nominal',
   'time': 0.0,
   'name': 'sine\_142',
   'rangeid': 'sine',
   'params': \{'linetype': 'sine',
    'amp': 7.0,
    'period': 0.3141592653589793,
    'initangle': 126.0,
    'end': [20, 0.0]\},
   'inputparams': \{'amp': 7.0, 'wavelength': 20\},
   'modelparams': \{'seed': 899083440\},
   'paramfunc': <function \_\_main\_\_.gen\_params(linetype, **kwargs)>,
   'fixedargs': ('sine',),
   'fixedkwargs': \{\},
   'prob': 0.005\}\},
 'sine\_143': \{'faults': \{\},
  'properties': \{'type': 'nominal',
   'time': 0.0,
   'name': 'sine\_143',
   'rangeid': 'sine',
   'params': \{'linetype': 'sine',
    'amp': 7.0,
    'period': 0.20943951023931953,
    'initangle': 83.99999999999999,
    'end': [30, 0.0]\},
   'inputparams': \{'amp': 7.0, 'wavelength': 30\},
   'modelparams': \{'seed': 899083440\},
   'paramfunc': <function \_\_main\_\_.gen\_params(linetype, **kwargs)>,
   'fixedargs': ('sine',),
   'fixedkwargs': \{\},
   'prob': 0.005\}\},
 'sine\_144': \{'faults': \{\},
  'properties': \{'type': 'nominal',
   'time': 0.0,
   'name': 'sine\_144',
   'rangeid': 'sine',
   'params': \{'linetype': 'sine',
    'amp': 7.0,
    'period': 0.15707963267948966,
    'initangle': 63.0,
    'end': [40, 0.0]\},
   'inputparams': \{'amp': 7.0, 'wavelength': 40\},
   'modelparams': \{'seed': 899083440\},
   'paramfunc': <function \_\_main\_\_.gen\_params(linetype, **kwargs)>,
   'fixedargs': ('sine',),
   'fixedkwargs': \{\},
   'prob': 0.005\}\},
 'sine\_145': \{'faults': \{\},
  'properties': \{'type': 'nominal',
   'time': 0.0,
   'name': 'sine\_145',
   'rangeid': 'sine',
   'params': \{'linetype': 'sine',
    'amp': 7.2,
    'period': 0.6283185307179586,
    'initangle': 259.20000000000005,
    'end': [10, 0.0]\},
   'inputparams': \{'amp': 7.2, 'wavelength': 10\},
   'modelparams': \{'seed': 899083440\},
   'paramfunc': <function \_\_main\_\_.gen\_params(linetype, **kwargs)>,
   'fixedargs': ('sine',),
   'fixedkwargs': \{\},
   'prob': 0.005\}\},
 'sine\_146': \{'faults': \{\},
  'properties': \{'type': 'nominal',
   'time': 0.0,
   'name': 'sine\_146',
   'rangeid': 'sine',
   'params': \{'linetype': 'sine',
    'amp': 7.2,
    'period': 0.3141592653589793,
    'initangle': 129.60000000000002,
    'end': [20, 0.0]\},
   'inputparams': \{'amp': 7.2, 'wavelength': 20\},
   'modelparams': \{'seed': 899083440\},
   'paramfunc': <function \_\_main\_\_.gen\_params(linetype, **kwargs)>,
   'fixedargs': ('sine',),
   'fixedkwargs': \{\},
   'prob': 0.005\}\},
 'sine\_147': \{'faults': \{\},
  'properties': \{'type': 'nominal',
   'time': 0.0,
   'name': 'sine\_147',
   'rangeid': 'sine',
   'params': \{'linetype': 'sine',
    'amp': 7.2,
    'period': 0.20943951023931953,
    'initangle': 86.4,
    'end': [30, 0.0]\},
   'inputparams': \{'amp': 7.2, 'wavelength': 30\},
   'modelparams': \{'seed': 899083440\},
   'paramfunc': <function \_\_main\_\_.gen\_params(linetype, **kwargs)>,
   'fixedargs': ('sine',),
   'fixedkwargs': \{\},
   'prob': 0.005\}\},
 'sine\_148': \{'faults': \{\},
  'properties': \{'type': 'nominal',
   'time': 0.0,
   'name': 'sine\_148',
   'rangeid': 'sine',
   'params': \{'linetype': 'sine',
    'amp': 7.2,
    'period': 0.15707963267948966,
    'initangle': 64.80000000000001,
    'end': [40, 0.0]\},
   'inputparams': \{'amp': 7.2, 'wavelength': 40\},
   'modelparams': \{'seed': 899083440\},
   'paramfunc': <function \_\_main\_\_.gen\_params(linetype, **kwargs)>,
   'fixedargs': ('sine',),
   'fixedkwargs': \{\},
   'prob': 0.005\}\},
 'sine\_149': \{'faults': \{\},
  'properties': \{'type': 'nominal',
   'time': 0.0,
   'name': 'sine\_149',
   'rangeid': 'sine',
   'params': \{'linetype': 'sine',
    'amp': 7.4,
    'period': 0.6283185307179586,
    'initangle': 266.40000000000003,
    'end': [10, 0.0]\},
   'inputparams': \{'amp': 7.4, 'wavelength': 10\},
   'modelparams': \{'seed': 899083440\},
   'paramfunc': <function \_\_main\_\_.gen\_params(linetype, **kwargs)>,
   'fixedargs': ('sine',),
   'fixedkwargs': \{\},
   'prob': 0.005\}\},
 'sine\_150': \{'faults': \{\},
  'properties': \{'type': 'nominal',
   'time': 0.0,
   'name': 'sine\_150',
   'rangeid': 'sine',
   'params': \{'linetype': 'sine',
    'amp': 7.4,
    'period': 0.3141592653589793,
    'initangle': 133.20000000000002,
    'end': [20, 0.0]\},
   'inputparams': \{'amp': 7.4, 'wavelength': 20\},
   'modelparams': \{'seed': 899083440\},
   'paramfunc': <function \_\_main\_\_.gen\_params(linetype, **kwargs)>,
   'fixedargs': ('sine',),
   'fixedkwargs': \{\},
   'prob': 0.005\}\},
 'sine\_151': \{'faults': \{\},
  'properties': \{'type': 'nominal',
   'time': 0.0,
   'name': 'sine\_151',
   'rangeid': 'sine',
   'params': \{'linetype': 'sine',
    'amp': 7.4,
    'period': 0.20943951023931953,
    'initangle': 88.8,
    'end': [30, 0.0]\},
   'inputparams': \{'amp': 7.4, 'wavelength': 30\},
   'modelparams': \{'seed': 899083440\},
   'paramfunc': <function \_\_main\_\_.gen\_params(linetype, **kwargs)>,
   'fixedargs': ('sine',),
   'fixedkwargs': \{\},
   'prob': 0.005\}\},
 'sine\_152': \{'faults': \{\},
  'properties': \{'type': 'nominal',
   'time': 0.0,
   'name': 'sine\_152',
   'rangeid': 'sine',
   'params': \{'linetype': 'sine',
    'amp': 7.4,
    'period': 0.15707963267948966,
    'initangle': 66.60000000000001,
    'end': [40, 0.0]\},
   'inputparams': \{'amp': 7.4, 'wavelength': 40\},
   'modelparams': \{'seed': 899083440\},
   'paramfunc': <function \_\_main\_\_.gen\_params(linetype, **kwargs)>,
   'fixedargs': ('sine',),
   'fixedkwargs': \{\},
   'prob': 0.005\}\},
 'sine\_153': \{'faults': \{\},
  'properties': \{'type': 'nominal',
   'time': 0.0,
   'name': 'sine\_153',
   'rangeid': 'sine',
   'params': \{'linetype': 'sine',
    'amp': 7.6000000000000005,
    'period': 0.6283185307179586,
    'initangle': 273.6,
    'end': [10, 0.0]\},
   'inputparams': \{'amp': 7.6000000000000005, 'wavelength': 10\},
   'modelparams': \{'seed': 899083440\},
   'paramfunc': <function \_\_main\_\_.gen\_params(linetype, **kwargs)>,
   'fixedargs': ('sine',),
   'fixedkwargs': \{\},
   'prob': 0.005\}\},
 'sine\_154': \{'faults': \{\},
  'properties': \{'type': 'nominal',
   'time': 0.0,
   'name': 'sine\_154',
   'rangeid': 'sine',
   'params': \{'linetype': 'sine',
    'amp': 7.6000000000000005,
    'period': 0.3141592653589793,
    'initangle': 136.8,
    'end': [20, 0.0]\},
   'inputparams': \{'amp': 7.6000000000000005, 'wavelength': 20\},
   'modelparams': \{'seed': 899083440\},
   'paramfunc': <function \_\_main\_\_.gen\_params(linetype, **kwargs)>,
   'fixedargs': ('sine',),
   'fixedkwargs': \{\},
   'prob': 0.005\}\},
 'sine\_155': \{'faults': \{\},
  'properties': \{'type': 'nominal',
   'time': 0.0,
   'name': 'sine\_155',
   'rangeid': 'sine',
   'params': \{'linetype': 'sine',
    'amp': 7.6000000000000005,
    'period': 0.20943951023931953,
    'initangle': 91.2,
    'end': [30, 0.0]\},
   'inputparams': \{'amp': 7.6000000000000005, 'wavelength': 30\},
   'modelparams': \{'seed': 899083440\},
   'paramfunc': <function \_\_main\_\_.gen\_params(linetype, **kwargs)>,
   'fixedargs': ('sine',),
   'fixedkwargs': \{\},
   'prob': 0.005\}\},
 'sine\_156': \{'faults': \{\},
  'properties': \{'type': 'nominal',
   'time': 0.0,
   'name': 'sine\_156',
   'rangeid': 'sine',
   'params': \{'linetype': 'sine',
    'amp': 7.6000000000000005,
    'period': 0.15707963267948966,
    'initangle': 68.4,
    'end': [40, 0.0]\},
   'inputparams': \{'amp': 7.6000000000000005, 'wavelength': 40\},
   'modelparams': \{'seed': 899083440\},
   'paramfunc': <function \_\_main\_\_.gen\_params(linetype, **kwargs)>,
   'fixedargs': ('sine',),
   'fixedkwargs': \{\},
   'prob': 0.005\}\},
 'sine\_157': \{'faults': \{\},
  'properties': \{'type': 'nominal',
   'time': 0.0,
   'name': 'sine\_157',
   'rangeid': 'sine',
   'params': \{'linetype': 'sine',
    'amp': 7.800000000000001,
    'period': 0.6283185307179586,
    'initangle': 280.8,
    'end': [10, 0.0]\},
   'inputparams': \{'amp': 7.800000000000001, 'wavelength': 10\},
   'modelparams': \{'seed': 899083440\},
   'paramfunc': <function \_\_main\_\_.gen\_params(linetype, **kwargs)>,
   'fixedargs': ('sine',),
   'fixedkwargs': \{\},
   'prob': 0.005\}\},
 'sine\_158': \{'faults': \{\},
  'properties': \{'type': 'nominal',
   'time': 0.0,
   'name': 'sine\_158',
   'rangeid': 'sine',
   'params': \{'linetype': 'sine',
    'amp': 7.800000000000001,
    'period': 0.3141592653589793,
    'initangle': 140.4,
    'end': [20, 0.0]\},
   'inputparams': \{'amp': 7.800000000000001, 'wavelength': 20\},
   'modelparams': \{'seed': 899083440\},
   'paramfunc': <function \_\_main\_\_.gen\_params(linetype, **kwargs)>,
   'fixedargs': ('sine',),
   'fixedkwargs': \{\},
   'prob': 0.005\}\},
 'sine\_159': \{'faults': \{\},
  'properties': \{'type': 'nominal',
   'time': 0.0,
   'name': 'sine\_159',
   'rangeid': 'sine',
   'params': \{'linetype': 'sine',
    'amp': 7.800000000000001,
    'period': 0.20943951023931953,
    'initangle': 93.60000000000001,
    'end': [30, 0.0]\},
   'inputparams': \{'amp': 7.800000000000001, 'wavelength': 30\},
   'modelparams': \{'seed': 899083440\},
   'paramfunc': <function \_\_main\_\_.gen\_params(linetype, **kwargs)>,
   'fixedargs': ('sine',),
   'fixedkwargs': \{\},
   'prob': 0.005\}\},
 'sine\_160': \{'faults': \{\},
  'properties': \{'type': 'nominal',
   'time': 0.0,
   'name': 'sine\_160',
   'rangeid': 'sine',
   'params': \{'linetype': 'sine',
    'amp': 7.800000000000001,
    'period': 0.15707963267948966,
    'initangle': 70.2,
    'end': [40, 0.0]\},
   'inputparams': \{'amp': 7.800000000000001, 'wavelength': 40\},
   'modelparams': \{'seed': 899083440\},
   'paramfunc': <function \_\_main\_\_.gen\_params(linetype, **kwargs)>,
   'fixedargs': ('sine',),
   'fixedkwargs': \{\},
   'prob': 0.005\}\},
 'sine\_161': \{'faults': \{\},
  'properties': \{'type': 'nominal',
   'time': 0.0,
   'name': 'sine\_161',
   'rangeid': 'sine',
   'params': \{'linetype': 'sine',
    'amp': 8.0,
    'period': 0.6283185307179586,
    'initangle': 288.0,
    'end': [10, 0.0]\},
   'inputparams': \{'amp': 8.0, 'wavelength': 10\},
   'modelparams': \{'seed': 899083440\},
   'paramfunc': <function \_\_main\_\_.gen\_params(linetype, **kwargs)>,
   'fixedargs': ('sine',),
   'fixedkwargs': \{\},
   'prob': 0.005\}\},
 'sine\_162': \{'faults': \{\},
  'properties': \{'type': 'nominal',
   'time': 0.0,
   'name': 'sine\_162',
   'rangeid': 'sine',
   'params': \{'linetype': 'sine',
    'amp': 8.0,
    'period': 0.3141592653589793,
    'initangle': 144.0,
    'end': [20, 0.0]\},
   'inputparams': \{'amp': 8.0, 'wavelength': 20\},
   'modelparams': \{'seed': 899083440\},
   'paramfunc': <function \_\_main\_\_.gen\_params(linetype, **kwargs)>,
   'fixedargs': ('sine',),
   'fixedkwargs': \{\},
   'prob': 0.005\}\},
 'sine\_163': \{'faults': \{\},
  'properties': \{'type': 'nominal',
   'time': 0.0,
   'name': 'sine\_163',
   'rangeid': 'sine',
   'params': \{'linetype': 'sine',
    'amp': 8.0,
    'period': 0.20943951023931953,
    'initangle': 96.0,
    'end': [30, 0.0]\},
   'inputparams': \{'amp': 8.0, 'wavelength': 30\},
   'modelparams': \{'seed': 899083440\},
   'paramfunc': <function \_\_main\_\_.gen\_params(linetype, **kwargs)>,
   'fixedargs': ('sine',),
   'fixedkwargs': \{\},
   'prob': 0.005\}\},
 'sine\_164': \{'faults': \{\},
  'properties': \{'type': 'nominal',
   'time': 0.0,
   'name': 'sine\_164',
   'rangeid': 'sine',
   'params': \{'linetype': 'sine',
    'amp': 8.0,
    'period': 0.15707963267948966,
    'initangle': 72.0,
    'end': [40, 0.0]\},
   'inputparams': \{'amp': 8.0, 'wavelength': 40\},
   'modelparams': \{'seed': 899083440\},
   'paramfunc': <function \_\_main\_\_.gen\_params(linetype, **kwargs)>,
   'fixedargs': ('sine',),
   'fixedkwargs': \{\},
   'prob': 0.005\}\},
 'sine\_165': \{'faults': \{\},
  'properties': \{'type': 'nominal',
   'time': 0.0,
   'name': 'sine\_165',
   'rangeid': 'sine',
   'params': \{'linetype': 'sine',
    'amp': 8.200000000000001,
    'period': 0.6283185307179586,
    'initangle': 295.2000000000001,
    'end': [10, 0.0]\},
   'inputparams': \{'amp': 8.200000000000001, 'wavelength': 10\},
   'modelparams': \{'seed': 899083440\},
   'paramfunc': <function \_\_main\_\_.gen\_params(linetype, **kwargs)>,
   'fixedargs': ('sine',),
   'fixedkwargs': \{\},
   'prob': 0.005\}\},
 'sine\_166': \{'faults': \{\},
  'properties': \{'type': 'nominal',
   'time': 0.0,
   'name': 'sine\_166',
   'rangeid': 'sine',
   'params': \{'linetype': 'sine',
    'amp': 8.200000000000001,
    'period': 0.3141592653589793,
    'initangle': 147.60000000000005,
    'end': [20, 0.0]\},
   'inputparams': \{'amp': 8.200000000000001, 'wavelength': 20\},
   'modelparams': \{'seed': 899083440\},
   'paramfunc': <function \_\_main\_\_.gen\_params(linetype, **kwargs)>,
   'fixedargs': ('sine',),
   'fixedkwargs': \{\},
   'prob': 0.005\}\},
 'sine\_167': \{'faults': \{\},
  'properties': \{'type': 'nominal',
   'time': 0.0,
   'name': 'sine\_167',
   'rangeid': 'sine',
   'params': \{'linetype': 'sine',
    'amp': 8.200000000000001,
    'period': 0.20943951023931953,
    'initangle': 98.40000000000002,
    'end': [30, 0.0]\},
   'inputparams': \{'amp': 8.200000000000001, 'wavelength': 30\},
   'modelparams': \{'seed': 899083440\},
   'paramfunc': <function \_\_main\_\_.gen\_params(linetype, **kwargs)>,
   'fixedargs': ('sine',),
   'fixedkwargs': \{\},
   'prob': 0.005\}\},
 'sine\_168': \{'faults': \{\},
  'properties': \{'type': 'nominal',
   'time': 0.0,
   'name': 'sine\_168',
   'rangeid': 'sine',
   'params': \{'linetype': 'sine',
    'amp': 8.200000000000001,
    'period': 0.15707963267948966,
    'initangle': 73.80000000000003,
    'end': [40, 0.0]\},
   'inputparams': \{'amp': 8.200000000000001, 'wavelength': 40\},
   'modelparams': \{'seed': 899083440\},
   'paramfunc': <function \_\_main\_\_.gen\_params(linetype, **kwargs)>,
   'fixedargs': ('sine',),
   'fixedkwargs': \{\},
   'prob': 0.005\}\},
 'sine\_169': \{'faults': \{\},
  'properties': \{'type': 'nominal',
   'time': 0.0,
   'name': 'sine\_169',
   'rangeid': 'sine',
   'params': \{'linetype': 'sine',
    'amp': 8.4,
    'period': 0.6283185307179586,
    'initangle': 302.40000000000003,
    'end': [10, 0.0]\},
   'inputparams': \{'amp': 8.4, 'wavelength': 10\},
   'modelparams': \{'seed': 899083440\},
   'paramfunc': <function \_\_main\_\_.gen\_params(linetype, **kwargs)>,
   'fixedargs': ('sine',),
   'fixedkwargs': \{\},
   'prob': 0.005\}\},
 'sine\_170': \{'faults': \{\},
  'properties': \{'type': 'nominal',
   'time': 0.0,
   'name': 'sine\_170',
   'rangeid': 'sine',
   'params': \{'linetype': 'sine',
    'amp': 8.4,
    'period': 0.3141592653589793,
    'initangle': 151.20000000000002,
    'end': [20, 0.0]\},
   'inputparams': \{'amp': 8.4, 'wavelength': 20\},
   'modelparams': \{'seed': 899083440\},
   'paramfunc': <function \_\_main\_\_.gen\_params(linetype, **kwargs)>,
   'fixedargs': ('sine',),
   'fixedkwargs': \{\},
   'prob': 0.005\}\},
 'sine\_171': \{'faults': \{\},
  'properties': \{'type': 'nominal',
   'time': 0.0,
   'name': 'sine\_171',
   'rangeid': 'sine',
   'params': \{'linetype': 'sine',
    'amp': 8.4,
    'period': 0.20943951023931953,
    'initangle': 100.80000000000001,
    'end': [30, 0.0]\},
   'inputparams': \{'amp': 8.4, 'wavelength': 30\},
   'modelparams': \{'seed': 899083440\},
   'paramfunc': <function \_\_main\_\_.gen\_params(linetype, **kwargs)>,
   'fixedargs': ('sine',),
   'fixedkwargs': \{\},
   'prob': 0.005\}\},
 'sine\_172': \{'faults': \{\},
  'properties': \{'type': 'nominal',
   'time': 0.0,
   'name': 'sine\_172',
   'rangeid': 'sine',
   'params': \{'linetype': 'sine',
    'amp': 8.4,
    'period': 0.15707963267948966,
    'initangle': 75.60000000000001,
    'end': [40, 0.0]\},
   'inputparams': \{'amp': 8.4, 'wavelength': 40\},
   'modelparams': \{'seed': 899083440\},
   'paramfunc': <function \_\_main\_\_.gen\_params(linetype, **kwargs)>,
   'fixedargs': ('sine',),
   'fixedkwargs': \{\},
   'prob': 0.005\}\},
 'sine\_173': \{'faults': \{\},
  'properties': \{'type': 'nominal',
   'time': 0.0,
   'name': 'sine\_173',
   'rangeid': 'sine',
   'params': \{'linetype': 'sine',
    'amp': 8.6,
    'period': 0.6283185307179586,
    'initangle': 309.59999999999997,
    'end': [10, 0.0]\},
   'inputparams': \{'amp': 8.6, 'wavelength': 10\},
   'modelparams': \{'seed': 899083440\},
   'paramfunc': <function \_\_main\_\_.gen\_params(linetype, **kwargs)>,
   'fixedargs': ('sine',),
   'fixedkwargs': \{\},
   'prob': 0.005\}\},
 'sine\_174': \{'faults': \{\},
  'properties': \{'type': 'nominal',
   'time': 0.0,
   'name': 'sine\_174',
   'rangeid': 'sine',
   'params': \{'linetype': 'sine',
    'amp': 8.6,
    'period': 0.3141592653589793,
    'initangle': 154.79999999999998,
    'end': [20, 0.0]\},
   'inputparams': \{'amp': 8.6, 'wavelength': 20\},
   'modelparams': \{'seed': 899083440\},
   'paramfunc': <function \_\_main\_\_.gen\_params(linetype, **kwargs)>,
   'fixedargs': ('sine',),
   'fixedkwargs': \{\},
   'prob': 0.005\}\},
 'sine\_175': \{'faults': \{\},
  'properties': \{'type': 'nominal',
   'time': 0.0,
   'name': 'sine\_175',
   'rangeid': 'sine',
   'params': \{'linetype': 'sine',
    'amp': 8.6,
    'period': 0.20943951023931953,
    'initangle': 103.19999999999999,
    'end': [30, 0.0]\},
   'inputparams': \{'amp': 8.6, 'wavelength': 30\},
   'modelparams': \{'seed': 899083440\},
   'paramfunc': <function \_\_main\_\_.gen\_params(linetype, **kwargs)>,
   'fixedargs': ('sine',),
   'fixedkwargs': \{\},
   'prob': 0.005\}\},
 'sine\_176': \{'faults': \{\},
  'properties': \{'type': 'nominal',
   'time': 0.0,
   'name': 'sine\_176',
   'rangeid': 'sine',
   'params': \{'linetype': 'sine',
    'amp': 8.6,
    'period': 0.15707963267948966,
    'initangle': 77.39999999999999,
    'end': [40, 0.0]\},
   'inputparams': \{'amp': 8.6, 'wavelength': 40\},
   'modelparams': \{'seed': 899083440\},
   'paramfunc': <function \_\_main\_\_.gen\_params(linetype, **kwargs)>,
   'fixedargs': ('sine',),
   'fixedkwargs': \{\},
   'prob': 0.005\}\},
 'sine\_177': \{'faults': \{\},
  'properties': \{'type': 'nominal',
   'time': 0.0,
   'name': 'sine\_177',
   'rangeid': 'sine',
   'params': \{'linetype': 'sine',
    'amp': 8.8,
    'period': 0.6283185307179586,
    'initangle': 316.80000000000007,
    'end': [10, 0.0]\},
   'inputparams': \{'amp': 8.8, 'wavelength': 10\},
   'modelparams': \{'seed': 899083440\},
   'paramfunc': <function \_\_main\_\_.gen\_params(linetype, **kwargs)>,
   'fixedargs': ('sine',),
   'fixedkwargs': \{\},
   'prob': 0.005\}\},
 'sine\_178': \{'faults': \{\},
  'properties': \{'type': 'nominal',
   'time': 0.0,
   'name': 'sine\_178',
   'rangeid': 'sine',
   'params': \{'linetype': 'sine',
    'amp': 8.8,
    'period': 0.3141592653589793,
    'initangle': 158.40000000000003,
    'end': [20, 0.0]\},
   'inputparams': \{'amp': 8.8, 'wavelength': 20\},
   'modelparams': \{'seed': 899083440\},
   'paramfunc': <function \_\_main\_\_.gen\_params(linetype, **kwargs)>,
   'fixedargs': ('sine',),
   'fixedkwargs': \{\},
   'prob': 0.005\}\},
 'sine\_179': \{'faults': \{\},
  'properties': \{'type': 'nominal',
   'time': 0.0,
   'name': 'sine\_179',
   'rangeid': 'sine',
   'params': \{'linetype': 'sine',
    'amp': 8.8,
    'period': 0.20943951023931953,
    'initangle': 105.60000000000002,
    'end': [30, 0.0]\},
   'inputparams': \{'amp': 8.8, 'wavelength': 30\},
   'modelparams': \{'seed': 899083440\},
   'paramfunc': <function \_\_main\_\_.gen\_params(linetype, **kwargs)>,
   'fixedargs': ('sine',),
   'fixedkwargs': \{\},
   'prob': 0.005\}\},
 'sine\_180': \{'faults': \{\},
  'properties': \{'type': 'nominal',
   'time': 0.0,
   'name': 'sine\_180',
   'rangeid': 'sine',
   'params': \{'linetype': 'sine',
    'amp': 8.8,
    'period': 0.15707963267948966,
    'initangle': 79.20000000000002,
    'end': [40, 0.0]\},
   'inputparams': \{'amp': 8.8, 'wavelength': 40\},
   'modelparams': \{'seed': 899083440\},
   'paramfunc': <function \_\_main\_\_.gen\_params(linetype, **kwargs)>,
   'fixedargs': ('sine',),
   'fixedkwargs': \{\},
   'prob': 0.005\}\},
 'sine\_181': \{'faults': \{\},
  'properties': \{'type': 'nominal',
   'time': 0.0,
   'name': 'sine\_181',
   'rangeid': 'sine',
   'params': \{'linetype': 'sine',
    'amp': 9.0,
    'period': 0.6283185307179586,
    'initangle': 324.0,
    'end': [10, 0.0]\},
   'inputparams': \{'amp': 9.0, 'wavelength': 10\},
   'modelparams': \{'seed': 899083440\},
   'paramfunc': <function \_\_main\_\_.gen\_params(linetype, **kwargs)>,
   'fixedargs': ('sine',),
   'fixedkwargs': \{\},
   'prob': 0.005\}\},
 'sine\_182': \{'faults': \{\},
  'properties': \{'type': 'nominal',
   'time': 0.0,
   'name': 'sine\_182',
   'rangeid': 'sine',
   'params': \{'linetype': 'sine',
    'amp': 9.0,
    'period': 0.3141592653589793,
    'initangle': 162.0,
    'end': [20, 0.0]\},
   'inputparams': \{'amp': 9.0, 'wavelength': 20\},
   'modelparams': \{'seed': 899083440\},
   'paramfunc': <function \_\_main\_\_.gen\_params(linetype, **kwargs)>,
   'fixedargs': ('sine',),
   'fixedkwargs': \{\},
   'prob': 0.005\}\},
 'sine\_183': \{'faults': \{\},
  'properties': \{'type': 'nominal',
   'time': 0.0,
   'name': 'sine\_183',
   'rangeid': 'sine',
   'params': \{'linetype': 'sine',
    'amp': 9.0,
    'period': 0.20943951023931953,
    'initangle': 108.0,
    'end': [30, 0.0]\},
   'inputparams': \{'amp': 9.0, 'wavelength': 30\},
   'modelparams': \{'seed': 899083440\},
   'paramfunc': <function \_\_main\_\_.gen\_params(linetype, **kwargs)>,
   'fixedargs': ('sine',),
   'fixedkwargs': \{\},
   'prob': 0.005\}\},
 'sine\_184': \{'faults': \{\},
  'properties': \{'type': 'nominal',
   'time': 0.0,
   'name': 'sine\_184',
   'rangeid': 'sine',
   'params': \{'linetype': 'sine',
    'amp': 9.0,
    'period': 0.15707963267948966,
    'initangle': 81.0,
    'end': [40, 0.0]\},
   'inputparams': \{'amp': 9.0, 'wavelength': 40\},
   'modelparams': \{'seed': 899083440\},
   'paramfunc': <function \_\_main\_\_.gen\_params(linetype, **kwargs)>,
   'fixedargs': ('sine',),
   'fixedkwargs': \{\},
   'prob': 0.005\}\},
 'sine\_185': \{'faults': \{\},
  'properties': \{'type': 'nominal',
   'time': 0.0,
   'name': 'sine\_185',
   'rangeid': 'sine',
   'params': \{'linetype': 'sine',
    'amp': 9.200000000000001,
    'period': 0.6283185307179586,
    'initangle': 331.20000000000005,
    'end': [10, 0.0]\},
   'inputparams': \{'amp': 9.200000000000001, 'wavelength': 10\},
   'modelparams': \{'seed': 899083440\},
   'paramfunc': <function \_\_main\_\_.gen\_params(linetype, **kwargs)>,
   'fixedargs': ('sine',),
   'fixedkwargs': \{\},
   'prob': 0.005\}\},
 'sine\_186': \{'faults': \{\},
  'properties': \{'type': 'nominal',
   'time': 0.0,
   'name': 'sine\_186',
   'rangeid': 'sine',
   'params': \{'linetype': 'sine',
    'amp': 9.200000000000001,
    'period': 0.3141592653589793,
    'initangle': 165.60000000000002,
    'end': [20, 0.0]\},
   'inputparams': \{'amp': 9.200000000000001, 'wavelength': 20\},
   'modelparams': \{'seed': 899083440\},
   'paramfunc': <function \_\_main\_\_.gen\_params(linetype, **kwargs)>,
   'fixedargs': ('sine',),
   'fixedkwargs': \{\},
   'prob': 0.005\}\},
 'sine\_187': \{'faults': \{\},
  'properties': \{'type': 'nominal',
   'time': 0.0,
   'name': 'sine\_187',
   'rangeid': 'sine',
   'params': \{'linetype': 'sine',
    'amp': 9.200000000000001,
    'period': 0.20943951023931953,
    'initangle': 110.4,
    'end': [30, 0.0]\},
   'inputparams': \{'amp': 9.200000000000001, 'wavelength': 30\},
   'modelparams': \{'seed': 899083440\},
   'paramfunc': <function \_\_main\_\_.gen\_params(linetype, **kwargs)>,
   'fixedargs': ('sine',),
   'fixedkwargs': \{\},
   'prob': 0.005\}\},
 'sine\_188': \{'faults': \{\},
  'properties': \{'type': 'nominal',
   'time': 0.0,
   'name': 'sine\_188',
   'rangeid': 'sine',
   'params': \{'linetype': 'sine',
    'amp': 9.200000000000001,
    'period': 0.15707963267948966,
    'initangle': 82.80000000000001,
    'end': [40, 0.0]\},
   'inputparams': \{'amp': 9.200000000000001, 'wavelength': 40\},
   'modelparams': \{'seed': 899083440\},
   'paramfunc': <function \_\_main\_\_.gen\_params(linetype, **kwargs)>,
   'fixedargs': ('sine',),
   'fixedkwargs': \{\},
   'prob': 0.005\}\},
 'sine\_189': \{'faults': \{\},
  'properties': \{'type': 'nominal',
   'time': 0.0,
   'name': 'sine\_189',
   'rangeid': 'sine',
   'params': \{'linetype': 'sine',
    'amp': 9.4,
    'period': 0.6283185307179586,
    'initangle': 338.4,
    'end': [10, 0.0]\},
   'inputparams': \{'amp': 9.4, 'wavelength': 10\},
   'modelparams': \{'seed': 899083440\},
   'paramfunc': <function \_\_main\_\_.gen\_params(linetype, **kwargs)>,
   'fixedargs': ('sine',),
   'fixedkwargs': \{\},
   'prob': 0.005\}\},
 'sine\_190': \{'faults': \{\},
  'properties': \{'type': 'nominal',
   'time': 0.0,
   'name': 'sine\_190',
   'rangeid': 'sine',
   'params': \{'linetype': 'sine',
    'amp': 9.4,
    'period': 0.3141592653589793,
    'initangle': 169.2,
    'end': [20, 0.0]\},
   'inputparams': \{'amp': 9.4, 'wavelength': 20\},
   'modelparams': \{'seed': 899083440\},
   'paramfunc': <function \_\_main\_\_.gen\_params(linetype, **kwargs)>,
   'fixedargs': ('sine',),
   'fixedkwargs': \{\},
   'prob': 0.005\}\},
 'sine\_191': \{'faults': \{\},
  'properties': \{'type': 'nominal',
   'time': 0.0,
   'name': 'sine\_191',
   'rangeid': 'sine',
   'params': \{'linetype': 'sine',
    'amp': 9.4,
    'period': 0.20943951023931953,
    'initangle': 112.79999999999998,
    'end': [30, 0.0]\},
   'inputparams': \{'amp': 9.4, 'wavelength': 30\},
   'modelparams': \{'seed': 899083440\},
   'paramfunc': <function \_\_main\_\_.gen\_params(linetype, **kwargs)>,
   'fixedargs': ('sine',),
   'fixedkwargs': \{\},
   'prob': 0.005\}\},
 'sine\_192': \{'faults': \{\},
  'properties': \{'type': 'nominal',
   'time': 0.0,
   'name': 'sine\_192',
   'rangeid': 'sine',
   'params': \{'linetype': 'sine',
    'amp': 9.4,
    'period': 0.15707963267948966,
    'initangle': 84.6,
    'end': [40, 0.0]\},
   'inputparams': \{'amp': 9.4, 'wavelength': 40\},
   'modelparams': \{'seed': 899083440\},
   'paramfunc': <function \_\_main\_\_.gen\_params(linetype, **kwargs)>,
   'fixedargs': ('sine',),
   'fixedkwargs': \{\},
   'prob': 0.005\}\},
 'sine\_193': \{'faults': \{\},
  'properties': \{'type': 'nominal',
   'time': 0.0,
   'name': 'sine\_193',
   'rangeid': 'sine',
   'params': \{'linetype': 'sine',
    'amp': 9.600000000000001,
    'period': 0.6283185307179586,
    'initangle': 345.6,
    'end': [10, 0.0]\},
   'inputparams': \{'amp': 9.600000000000001, 'wavelength': 10\},
   'modelparams': \{'seed': 899083440\},
   'paramfunc': <function \_\_main\_\_.gen\_params(linetype, **kwargs)>,
   'fixedargs': ('sine',),
   'fixedkwargs': \{\},
   'prob': 0.005\}\},
 'sine\_194': \{'faults': \{\},
  'properties': \{'type': 'nominal',
   'time': 0.0,
   'name': 'sine\_194',
   'rangeid': 'sine',
   'params': \{'linetype': 'sine',
    'amp': 9.600000000000001,
    'period': 0.3141592653589793,
    'initangle': 172.8,
    'end': [20, 0.0]\},
   'inputparams': \{'amp': 9.600000000000001, 'wavelength': 20\},
   'modelparams': \{'seed': 899083440\},
   'paramfunc': <function \_\_main\_\_.gen\_params(linetype, **kwargs)>,
   'fixedargs': ('sine',),
   'fixedkwargs': \{\},
   'prob': 0.005\}\},
 'sine\_195': \{'faults': \{\},
  'properties': \{'type': 'nominal',
   'time': 0.0,
   'name': 'sine\_195',
   'rangeid': 'sine',
   'params': \{'linetype': 'sine',
    'amp': 9.600000000000001,
    'period': 0.20943951023931953,
    'initangle': 115.20000000000002,
    'end': [30, 0.0]\},
   'inputparams': \{'amp': 9.600000000000001, 'wavelength': 30\},
   'modelparams': \{'seed': 899083440\},
   'paramfunc': <function \_\_main\_\_.gen\_params(linetype, **kwargs)>,
   'fixedargs': ('sine',),
   'fixedkwargs': \{\},
   'prob': 0.005\}\},
 'sine\_196': \{'faults': \{\},
  'properties': \{'type': 'nominal',
   'time': 0.0,
   'name': 'sine\_196',
   'rangeid': 'sine',
   'params': \{'linetype': 'sine',
    'amp': 9.600000000000001,
    'period': 0.15707963267948966,
    'initangle': 86.4,
    'end': [40, 0.0]\},
   'inputparams': \{'amp': 9.600000000000001, 'wavelength': 40\},
   'modelparams': \{'seed': 899083440\},
   'paramfunc': <function \_\_main\_\_.gen\_params(linetype, **kwargs)>,
   'fixedargs': ('sine',),
   'fixedkwargs': \{\},
   'prob': 0.005\}\},
 'sine\_197': \{'faults': \{\},
  'properties': \{'type': 'nominal',
   'time': 0.0,
   'name': 'sine\_197',
   'rangeid': 'sine',
   'params': \{'linetype': 'sine',
    'amp': 9.8,
    'period': 0.6283185307179586,
    'initangle': 352.80000000000007,
    'end': [10, 0.0]\},
   'inputparams': \{'amp': 9.8, 'wavelength': 10\},
   'modelparams': \{'seed': 899083440\},
   'paramfunc': <function \_\_main\_\_.gen\_params(linetype, **kwargs)>,
   'fixedargs': ('sine',),
   'fixedkwargs': \{\},
   'prob': 0.005\}\},
 'sine\_198': \{'faults': \{\},
  'properties': \{'type': 'nominal',
   'time': 0.0,
   'name': 'sine\_198',
   'rangeid': 'sine',
   'params': \{'linetype': 'sine',
    'amp': 9.8,
    'period': 0.3141592653589793,
    'initangle': 176.40000000000003,
    'end': [20, 0.0]\},
   'inputparams': \{'amp': 9.8, 'wavelength': 20\},
   'modelparams': \{'seed': 899083440\},
   'paramfunc': <function \_\_main\_\_.gen\_params(linetype, **kwargs)>,
   'fixedargs': ('sine',),
   'fixedkwargs': \{\},
   'prob': 0.005\}\},
 'sine\_199': \{'faults': \{\},
  'properties': \{'type': 'nominal',
   'time': 0.0,
   'name': 'sine\_199',
   'rangeid': 'sine',
   'params': \{'linetype': 'sine',
    'amp': 9.8,
    'period': 0.20943951023931953,
    'initangle': 117.60000000000001,
    'end': [30, 0.0]\},
   'inputparams': \{'amp': 9.8, 'wavelength': 30\},
   'modelparams': \{'seed': 899083440\},
   'paramfunc': <function \_\_main\_\_.gen\_params(linetype, **kwargs)>,
   'fixedargs': ('sine',),
   'fixedkwargs': \{\},
   'prob': 0.005\}\},
 'sine\_200': \{'faults': \{\},
  'properties': \{'type': 'nominal',
   'time': 0.0,
   'name': 'sine\_200',
   'rangeid': 'sine',
   'params': \{'linetype': 'sine',
    'amp': 9.8,
    'period': 0.15707963267948966,
    'initangle': 88.20000000000002,
    'end': [40, 0.0]\},
   'inputparams': \{'amp': 9.8, 'wavelength': 40\},
   'modelparams': \{'seed': 899083440\},
   'paramfunc': <function \_\_main\_\_.gen\_params(linetype, **kwargs)>,
   'fixedargs': ('sine',),
   'fixedkwargs': \{\},
   'prob': 0.005\}\},
 'turn\_201': \{'faults': \{\},
  'properties': \{'type': 'nominal',
   'time': 0.0,
   'name': 'turn\_201',
   'rangeid': 'turn',
   'params': \{'linetype': 'turn',
    'radius': 5,
    'start': 0,
    'initangle': 0.0,
    'end': [5, 5]\},
   'inputparams': \{'radius': 5, 'start': 0\},
   'modelparams': \{'seed': 3408150472\},
   'paramfunc': <function \_\_main\_\_.gen\_params(linetype, **kwargs)>,
   'fixedargs': ('turn',),
   'fixedkwargs': \{\},
   'prob': 0.03571428571428571\}\},
 'turn\_202': \{'faults': \{\},
  'properties': \{'type': 'nominal',
   'time': 0.0,
   'name': 'turn\_202',
   'rangeid': 'turn',
   'params': \{'linetype': 'turn',
    'radius': 5,
    'start': 5,
    'initangle': 0.0,
    'end': [10, 10]\},
   'inputparams': \{'radius': 5, 'start': 5\},
   'modelparams': \{'seed': 3408150472\},
   'paramfunc': <function \_\_main\_\_.gen\_params(linetype, **kwargs)>,
   'fixedargs': ('turn',),
   'fixedkwargs': \{\},
   'prob': 0.03571428571428571\}\},
 'turn\_203': \{'faults': \{\},
  'properties': \{'type': 'nominal',
   'time': 0.0,
   'name': 'turn\_203',
   'rangeid': 'turn',
   'params': \{'linetype': 'turn',
    'radius': 5,
    'start': 10,
    'initangle': 0.0,
    'end': [15, 15]\},
   'inputparams': \{'radius': 5, 'start': 10\},
   'modelparams': \{'seed': 3408150472\},
   'paramfunc': <function \_\_main\_\_.gen\_params(linetype, **kwargs)>,
   'fixedargs': ('turn',),
   'fixedkwargs': \{\},
   'prob': 0.03571428571428571\}\},
 'turn\_204': \{'faults': \{\},
  'properties': \{'type': 'nominal',
   'time': 0.0,
   'name': 'turn\_204',
   'rangeid': 'turn',
   'params': \{'linetype': 'turn',
    'radius': 5,
    'start': 15,
    'initangle': 0.0,
    'end': [20, 20]\},
   'inputparams': \{'radius': 5, 'start': 15\},
   'modelparams': \{'seed': 3408150472\},
   'paramfunc': <function \_\_main\_\_.gen\_params(linetype, **kwargs)>,
   'fixedargs': ('turn',),
   'fixedkwargs': \{\},
   'prob': 0.03571428571428571\}\},
 'turn\_205': \{'faults': \{\},
  'properties': \{'type': 'nominal',
   'time': 0.0,
   'name': 'turn\_205',
   'rangeid': 'turn',
   'params': \{'linetype': 'turn',
    'radius': 10,
    'start': 0,
    'initangle': 0.0,
    'end': [10, 10]\},
   'inputparams': \{'radius': 10, 'start': 0\},
   'modelparams': \{'seed': 3408150472\},
   'paramfunc': <function \_\_main\_\_.gen\_params(linetype, **kwargs)>,
   'fixedargs': ('turn',),
   'fixedkwargs': \{\},
   'prob': 0.03571428571428571\}\},
 'turn\_206': \{'faults': \{\},
  'properties': \{'type': 'nominal',
   'time': 0.0,
   'name': 'turn\_206',
   'rangeid': 'turn',
   'params': \{'linetype': 'turn',
    'radius': 10,
    'start': 5,
    'initangle': 0.0,
    'end': [15, 15]\},
   'inputparams': \{'radius': 10, 'start': 5\},
   'modelparams': \{'seed': 3408150472\},
   'paramfunc': <function \_\_main\_\_.gen\_params(linetype, **kwargs)>,
   'fixedargs': ('turn',),
   'fixedkwargs': \{\},
   'prob': 0.03571428571428571\}\},
 'turn\_207': \{'faults': \{\},
  'properties': \{'type': 'nominal',
   'time': 0.0,
   'name': 'turn\_207',
   'rangeid': 'turn',
   'params': \{'linetype': 'turn',
    'radius': 10,
    'start': 10,
    'initangle': 0.0,
    'end': [20, 20]\},
   'inputparams': \{'radius': 10, 'start': 10\},
   'modelparams': \{'seed': 3408150472\},
   'paramfunc': <function \_\_main\_\_.gen\_params(linetype, **kwargs)>,
   'fixedargs': ('turn',),
   'fixedkwargs': \{\},
   'prob': 0.03571428571428571\}\},
 'turn\_208': \{'faults': \{\},
  'properties': \{'type': 'nominal',
   'time': 0.0,
   'name': 'turn\_208',
   'rangeid': 'turn',
   'params': \{'linetype': 'turn',
    'radius': 10,
    'start': 15,
    'initangle': 0.0,
    'end': [25, 25]\},
   'inputparams': \{'radius': 10, 'start': 15\},
   'modelparams': \{'seed': 3408150472\},
   'paramfunc': <function \_\_main\_\_.gen\_params(linetype, **kwargs)>,
   'fixedargs': ('turn',),
   'fixedkwargs': \{\},
   'prob': 0.03571428571428571\}\},
 'turn\_209': \{'faults': \{\},
  'properties': \{'type': 'nominal',
   'time': 0.0,
   'name': 'turn\_209',
   'rangeid': 'turn',
   'params': \{'linetype': 'turn',
    'radius': 15,
    'start': 0,
    'initangle': 0.0,
    'end': [15, 15]\},
   'inputparams': \{'radius': 15, 'start': 0\},
   'modelparams': \{'seed': 3408150472\},
   'paramfunc': <function \_\_main\_\_.gen\_params(linetype, **kwargs)>,
   'fixedargs': ('turn',),
   'fixedkwargs': \{\},
   'prob': 0.03571428571428571\}\},
 'turn\_210': \{'faults': \{\},
  'properties': \{'type': 'nominal',
   'time': 0.0,
   'name': 'turn\_210',
   'rangeid': 'turn',
   'params': \{'linetype': 'turn',
    'radius': 15,
    'start': 5,
    'initangle': 0.0,
    'end': [20, 20]\},
   'inputparams': \{'radius': 15, 'start': 5\},
   'modelparams': \{'seed': 3408150472\},
   'paramfunc': <function \_\_main\_\_.gen\_params(linetype, **kwargs)>,
   'fixedargs': ('turn',),
   'fixedkwargs': \{\},
   'prob': 0.03571428571428571\}\},
 'turn\_211': \{'faults': \{\},
  'properties': \{'type': 'nominal',
   'time': 0.0,
   'name': 'turn\_211',
   'rangeid': 'turn',
   'params': \{'linetype': 'turn',
    'radius': 15,
    'start': 10,
    'initangle': 0.0,
    'end': [25, 25]\},
   'inputparams': \{'radius': 15, 'start': 10\},
   'modelparams': \{'seed': 3408150472\},
   'paramfunc': <function \_\_main\_\_.gen\_params(linetype, **kwargs)>,
   'fixedargs': ('turn',),
   'fixedkwargs': \{\},
   'prob': 0.03571428571428571\}\},
 'turn\_212': \{'faults': \{\},
  'properties': \{'type': 'nominal',
   'time': 0.0,
   'name': 'turn\_212',
   'rangeid': 'turn',
   'params': \{'linetype': 'turn',
    'radius': 15,
    'start': 15,
    'initangle': 0.0,
    'end': [30, 30]\},
   'inputparams': \{'radius': 15, 'start': 15\},
   'modelparams': \{'seed': 3408150472\},
   'paramfunc': <function \_\_main\_\_.gen\_params(linetype, **kwargs)>,
   'fixedargs': ('turn',),
   'fixedkwargs': \{\},
   'prob': 0.03571428571428571\}\},
 'turn\_213': \{'faults': \{\},
  'properties': \{'type': 'nominal',
   'time': 0.0,
   'name': 'turn\_213',
   'rangeid': 'turn',
   'params': \{'linetype': 'turn',
    'radius': 20,
    'start': 0,
    'initangle': 0.0,
    'end': [20, 20]\},
   'inputparams': \{'radius': 20, 'start': 0\},
   'modelparams': \{'seed': 3408150472\},
   'paramfunc': <function \_\_main\_\_.gen\_params(linetype, **kwargs)>,
   'fixedargs': ('turn',),
   'fixedkwargs': \{\},
   'prob': 0.03571428571428571\}\},
 'turn\_214': \{'faults': \{\},
  'properties': \{'type': 'nominal',
   'time': 0.0,
   'name': 'turn\_214',
   'rangeid': 'turn',
   'params': \{'linetype': 'turn',
    'radius': 20,
    'start': 5,
    'initangle': 0.0,
    'end': [25, 25]\},
   'inputparams': \{'radius': 20, 'start': 5\},
   'modelparams': \{'seed': 3408150472\},
   'paramfunc': <function \_\_main\_\_.gen\_params(linetype, **kwargs)>,
   'fixedargs': ('turn',),
   'fixedkwargs': \{\},
   'prob': 0.03571428571428571\}\},
 'turn\_215': \{'faults': \{\},
  'properties': \{'type': 'nominal',
   'time': 0.0,
   'name': 'turn\_215',
   'rangeid': 'turn',
   'params': \{'linetype': 'turn',
    'radius': 20,
    'start': 10,
    'initangle': 0.0,
    'end': [30, 30]\},
   'inputparams': \{'radius': 20, 'start': 10\},
   'modelparams': \{'seed': 3408150472\},
   'paramfunc': <function \_\_main\_\_.gen\_params(linetype, **kwargs)>,
   'fixedargs': ('turn',),
   'fixedkwargs': \{\},
   'prob': 0.03571428571428571\}\},
 'turn\_216': \{'faults': \{\},
  'properties': \{'type': 'nominal',
   'time': 0.0,
   'name': 'turn\_216',
   'rangeid': 'turn',
   'params': \{'linetype': 'turn',
    'radius': 20,
    'start': 15,
    'initangle': 0.0,
    'end': [35, 35]\},
   'inputparams': \{'radius': 20, 'start': 15\},
   'modelparams': \{'seed': 3408150472\},
   'paramfunc': <function \_\_main\_\_.gen\_params(linetype, **kwargs)>,
   'fixedargs': ('turn',),
   'fixedkwargs': \{\},
   'prob': 0.03571428571428571\}\},
 'turn\_217': \{'faults': \{\},
  'properties': \{'type': 'nominal',
   'time': 0.0,
   'name': 'turn\_217',
   'rangeid': 'turn',
   'params': \{'linetype': 'turn',
    'radius': 25,
    'start': 0,
    'initangle': 0.0,
    'end': [25, 25]\},
   'inputparams': \{'radius': 25, 'start': 0\},
   'modelparams': \{'seed': 3408150472\},
   'paramfunc': <function \_\_main\_\_.gen\_params(linetype, **kwargs)>,
   'fixedargs': ('turn',),
   'fixedkwargs': \{\},
   'prob': 0.03571428571428571\}\},
 'turn\_218': \{'faults': \{\},
  'properties': \{'type': 'nominal',
   'time': 0.0,
   'name': 'turn\_218',
   'rangeid': 'turn',
   'params': \{'linetype': 'turn',
    'radius': 25,
    'start': 5,
    'initangle': 0.0,
    'end': [30, 30]\},
   'inputparams': \{'radius': 25, 'start': 5\},
   'modelparams': \{'seed': 3408150472\},
   'paramfunc': <function \_\_main\_\_.gen\_params(linetype, **kwargs)>,
   'fixedargs': ('turn',),
   'fixedkwargs': \{\},
   'prob': 0.03571428571428571\}\},
 'turn\_219': \{'faults': \{\},
  'properties': \{'type': 'nominal',
   'time': 0.0,
   'name': 'turn\_219',
   'rangeid': 'turn',
   'params': \{'linetype': 'turn',
    'radius': 25,
    'start': 10,
    'initangle': 0.0,
    'end': [35, 35]\},
   'inputparams': \{'radius': 25, 'start': 10\},
   'modelparams': \{'seed': 3408150472\},
   'paramfunc': <function \_\_main\_\_.gen\_params(linetype, **kwargs)>,
   'fixedargs': ('turn',),
   'fixedkwargs': \{\},
   'prob': 0.03571428571428571\}\},
 'turn\_220': \{'faults': \{\},
  'properties': \{'type': 'nominal',
   'time': 0.0,
   'name': 'turn\_220',
   'rangeid': 'turn',
   'params': \{'linetype': 'turn',
    'radius': 25,
    'start': 15,
    'initangle': 0.0,
    'end': [40, 40]\},
   'inputparams': \{'radius': 25, 'start': 15\},
   'modelparams': \{'seed': 3408150472\},
   'paramfunc': <function \_\_main\_\_.gen\_params(linetype, **kwargs)>,
   'fixedargs': ('turn',),
   'fixedkwargs': \{\},
   'prob': 0.03571428571428571\}\},
 'turn\_221': \{'faults': \{\},
  'properties': \{'type': 'nominal',
   'time': 0.0,
   'name': 'turn\_221',
   'rangeid': 'turn',
   'params': \{'linetype': 'turn',
    'radius': 30,
    'start': 0,
    'initangle': 0.0,
    'end': [30, 30]\},
   'inputparams': \{'radius': 30, 'start': 0\},
   'modelparams': \{'seed': 3408150472\},
   'paramfunc': <function \_\_main\_\_.gen\_params(linetype, **kwargs)>,
   'fixedargs': ('turn',),
   'fixedkwargs': \{\},
   'prob': 0.03571428571428571\}\},
 'turn\_222': \{'faults': \{\},
  'properties': \{'type': 'nominal',
   'time': 0.0,
   'name': 'turn\_222',
   'rangeid': 'turn',
   'params': \{'linetype': 'turn',
    'radius': 30,
    'start': 5,
    'initangle': 0.0,
    'end': [35, 35]\},
   'inputparams': \{'radius': 30, 'start': 5\},
   'modelparams': \{'seed': 3408150472\},
   'paramfunc': <function \_\_main\_\_.gen\_params(linetype, **kwargs)>,
   'fixedargs': ('turn',),
   'fixedkwargs': \{\},
   'prob': 0.03571428571428571\}\},
 'turn\_223': \{'faults': \{\},
  'properties': \{'type': 'nominal',
   'time': 0.0,
   'name': 'turn\_223',
   'rangeid': 'turn',
   'params': \{'linetype': 'turn',
    'radius': 30,
    'start': 10,
    'initangle': 0.0,
    'end': [40, 40]\},
   'inputparams': \{'radius': 30, 'start': 10\},
   'modelparams': \{'seed': 3408150472\},
   'paramfunc': <function \_\_main\_\_.gen\_params(linetype, **kwargs)>,
   'fixedargs': ('turn',),
   'fixedkwargs': \{\},
   'prob': 0.03571428571428571\}\},
 'turn\_224': \{'faults': \{\},
  'properties': \{'type': 'nominal',
   'time': 0.0,
   'name': 'turn\_224',
   'rangeid': 'turn',
   'params': \{'linetype': 'turn',
    'radius': 30,
    'start': 15,
    'initangle': 0.0,
    'end': [45, 45]\},
   'inputparams': \{'radius': 30, 'start': 15\},
   'modelparams': \{'seed': 3408150472\},
   'paramfunc': <function \_\_main\_\_.gen\_params(linetype, **kwargs)>,
   'fixedargs': ('turn',),
   'fixedkwargs': \{\},
   'prob': 0.03571428571428571\}\},
 'turn\_225': \{'faults': \{\},
  'properties': \{'type': 'nominal',
   'time': 0.0,
   'name': 'turn\_225',
   'rangeid': 'turn',
   'params': \{'linetype': 'turn',
    'radius': 35,
    'start': 0,
    'initangle': 0.0,
    'end': [35, 35]\},
   'inputparams': \{'radius': 35, 'start': 0\},
   'modelparams': \{'seed': 3408150472\},
   'paramfunc': <function \_\_main\_\_.gen\_params(linetype, **kwargs)>,
   'fixedargs': ('turn',),
   'fixedkwargs': \{\},
   'prob': 0.03571428571428571\}\},
 'turn\_226': \{'faults': \{\},
  'properties': \{'type': 'nominal',
   'time': 0.0,
   'name': 'turn\_226',
   'rangeid': 'turn',
   'params': \{'linetype': 'turn',
    'radius': 35,
    'start': 5,
    'initangle': 0.0,
    'end': [40, 40]\},
   'inputparams': \{'radius': 35, 'start': 5\},
   'modelparams': \{'seed': 3408150472\},
   'paramfunc': <function \_\_main\_\_.gen\_params(linetype, **kwargs)>,
   'fixedargs': ('turn',),
   'fixedkwargs': \{\},
   'prob': 0.03571428571428571\}\},
 'turn\_227': \{'faults': \{\},
  'properties': \{'type': 'nominal',
   'time': 0.0,
   'name': 'turn\_227',
   'rangeid': 'turn',
   'params': \{'linetype': 'turn',
    'radius': 35,
    'start': 10,
    'initangle': 0.0,
    'end': [45, 45]\},
   'inputparams': \{'radius': 35, 'start': 10\},
   'modelparams': \{'seed': 3408150472\},
   'paramfunc': <function \_\_main\_\_.gen\_params(linetype, **kwargs)>,
   'fixedargs': ('turn',),
   'fixedkwargs': \{\},
   'prob': 0.03571428571428571\}\},
 'turn\_228': \{'faults': \{\},
  'properties': \{'type': 'nominal',
   'time': 0.0,
   'name': 'turn\_228',
   'rangeid': 'turn',
   'params': \{'linetype': 'turn',
    'radius': 35,
    'start': 15,
    'initangle': 0.0,
    'end': [50, 50]\},
   'inputparams': \{'radius': 35, 'start': 15\},
   'modelparams': \{'seed': 3408150472\},
   'paramfunc': <function \_\_main\_\_.gen\_params(linetype, **kwargs)>,
   'fixedargs': ('turn',),
   'fixedkwargs': \{\},
   'prob': 0.03571428571428571\}\}\}
\end{sphinxVerbatim}



\end{sphinxuseclass}
\end{sphinxuseclass}
}

\end{sphinxuseclass}
\end{sphinxuseclass}
\sphinxAtStartPar
Nominal Approaches are simulated using \sphinxcode{\sphinxupquote{prop.nominal\_approach}}.

\begin{sphinxuseclass}{nbinput}
{
\sphinxsetup{VerbatimColor={named}{nbsphinx-code-bg}}
\sphinxsetup{VerbatimBorderColor={named}{nbsphinx-code-border}}
\begin{sphinxVerbatim}[commandchars=\\\{\}]
\llap{\color{nbsphinxin}[12]:\,\hspace{\fboxrule}\hspace{\fboxsep}}\PYG{n}{help}\PYG{p}{(}\PYG{n}{prop}\PYG{o}{.}\PYG{n}{nominal\PYGZus{}approach}\PYG{p}{)}
\end{sphinxVerbatim}
}

\end{sphinxuseclass}
\begin{sphinxuseclass}{nboutput}
\begin{sphinxuseclass}{nblast}
{

\kern-\sphinxverbatimsmallskipamount\kern-\baselineskip
\kern+\FrameHeightAdjust\kern-\fboxrule
\vspace{\nbsphinxcodecellspacing}

\sphinxsetup{VerbatimColor={named}{white}}
\sphinxsetup{VerbatimBorderColor={named}{nbsphinx-code-border}}
\begin{sphinxuseclass}{output_area}
\begin{sphinxuseclass}{}


\begin{sphinxVerbatim}[commandchars=\\\{\}]
Help on function nominal\_approach in module fmdtools.faultsim.propagate:

nominal\_approach(mdl, nomapp, track='all', showprogress=True, pool=False, track\_times='all', run\_stochastic=False)
    Simulates a set of nominal scenarios through a model. Useful to understand
    the sets of parameters where the system will run nominally and/or lead to
    a fault.

    Parameters
    ----------
    mdl : Model
        Model to simulate
    nomapp : NominalApproach
        Nominal Approach defining the nominal scenarios to run the system over.
    track : str, optional
        States to track during simulation. The default is 'all'.
    showprogress : bool, optional
        Whether to display progress during simulation. The default is True.
    pool : Pool, optional
        Parallel pool (e.g. multiprocessing.Pool) to simulate with
        (if using parallelism). The default is False.
    track\_times : str/tuple
        Defines what times to include in the history. Options are:
            'all'--all simulated times
            ('interval', n)--includes every nth time in the history
            ('times', [t1, {\ldots} tn])--only includes times defined in the vector [t1 {\ldots} tn]
    run\_stochastic : bool
        Whether to run stochastic behaviors or use default values. Default is False.
    Returns
    -------
    nomapp\_endclasses : Dict
        Classifications of the set of scenarios, with structure \{'scenname':classification\}
    nomapp\_mdlhists : Dict
        Dictionary of model histories, with structure \{'scenname':mdlhist\}

\end{sphinxVerbatim}



\end{sphinxuseclass}
\end{sphinxuseclass}
}

\end{sphinxuseclass}
\end{sphinxuseclass}
\begin{sphinxuseclass}{nbinput}
{
\sphinxsetup{VerbatimColor={named}{nbsphinx-code-bg}}
\sphinxsetup{VerbatimBorderColor={named}{nbsphinx-code-border}}
\begin{sphinxVerbatim}[commandchars=\\\{\}]
\llap{\color{nbsphinxin}[13]:\,\hspace{\fboxrule}\hspace{\fboxsep}}\PYG{n}{endclasses}\PYG{p}{,} \PYG{n}{mdlhists}\PYG{o}{=} \PYG{n}{prop}\PYG{o}{.}\PYG{n}{nominal\PYGZus{}approach}\PYG{p}{(}\PYG{n}{mdl}\PYG{p}{,} \PYG{n}{nomapp}\PYG{p}{)}
\end{sphinxVerbatim}
}

\end{sphinxuseclass}
\begin{sphinxuseclass}{nboutput}
\begin{sphinxuseclass}{nblast}
{

\kern-\sphinxverbatimsmallskipamount\kern-\baselineskip
\kern+\FrameHeightAdjust\kern-\fboxrule
\vspace{\nbsphinxcodecellspacing}

\sphinxsetup{VerbatimColor={named}{nbsphinx-stderr}}
\sphinxsetup{VerbatimBorderColor={named}{nbsphinx-code-border}}
\begin{sphinxuseclass}{output_area}
\begin{sphinxuseclass}{stderr}


\begin{sphinxVerbatim}[commandchars=\\\{\}]
SCENARIOS COMPLETE: 100\%|████████████████████████████████████████████████████████████| 228/228 [00:03<00:00, 68.69it/s]
\end{sphinxVerbatim}



\end{sphinxuseclass}
\end{sphinxuseclass}
}

\end{sphinxuseclass}
\end{sphinxuseclass}
\sphinxAtStartPar
To speed up execution over large numbers of scenarios, multiprocessing can also be used to run the scenarios in parallel by passing an execution pool. This is not done here because it would require the model to be in a different file, and because the gains on a light\sphinxhyphen{}weight model like this are not significant.

\sphinxAtStartPar
Now that the approach has been simulated, the operational envelope can be visualized. There are three methods to perform this visualization \sphinxcode{\sphinxupquote{rd.plot.nominal\_vals\_1d}}, \sphinxcode{\sphinxupquote{rd.plot.nominal\_vals\_2d}}, and \sphinxcode{\sphinxupquote{rd.plot.nominal\_vals\_3d}}, which each plot the \sphinxstyleemphasis{classification} of the model in the 1/2/3 dimensions over the set of given parameters as nominal or incomplete.

\sphinxAtStartPar
Note that this classification must be in the dictionary returned from the Model’s \sphinxcode{\sphinxupquote{find\_classification}} function at the end of the model run under the key \sphinxcode{\sphinxupquote{classification}} as is done in the rover model. This classification must also be encoded as a string.

\begin{sphinxuseclass}{nbinput}
{
\sphinxsetup{VerbatimColor={named}{nbsphinx-code-bg}}
\sphinxsetup{VerbatimBorderColor={named}{nbsphinx-code-border}}
\begin{sphinxVerbatim}[commandchars=\\\{\}]
\llap{\color{nbsphinxin}[14]:\,\hspace{\fboxrule}\hspace{\fboxsep}}\PYG{n}{help}\PYG{p}{(}\PYG{n}{rd}\PYG{o}{.}\PYG{n}{plot}\PYG{o}{.}\PYG{n}{nominal\PYGZus{}vals\PYGZus{}2d}\PYG{p}{)}
\end{sphinxVerbatim}
}

\end{sphinxuseclass}
\begin{sphinxuseclass}{nboutput}
\begin{sphinxuseclass}{nblast}
{

\kern-\sphinxverbatimsmallskipamount\kern-\baselineskip
\kern+\FrameHeightAdjust\kern-\fboxrule
\vspace{\nbsphinxcodecellspacing}

\sphinxsetup{VerbatimColor={named}{white}}
\sphinxsetup{VerbatimBorderColor={named}{nbsphinx-code-border}}
\begin{sphinxuseclass}{output_area}
\begin{sphinxuseclass}{}


\begin{sphinxVerbatim}[commandchars=\\\{\}]
Help on function nominal\_vals\_2d in module fmdtools.resultdisp.plot:

nominal\_vals\_2d(nomapp, nomapp\_endclasses, param1, param2, title='Nominal Operational Envelope', nomlabel='nominal', metric='classification')
    Visualizes the nominal operational envelope along two given parameters

    Parameters
    ----------
    nomapp : NominalApproach
        Nominal sample approach simulated in the model.
    nomapp\_endclasses : dict
        End-classifications for the set of simulations in the model.
    param1 : str
        First parameter (x) desired to visualize in the operational envelope
    param2 : str
        Second arameter (y) desired to visualize in the operational envelope
    title : str, optional
        Plot title. The default is "Nominal Operational Envelope".
    nomlabel : str, optional
        Flag for nominal end-states. The default is 'nominal'.

    Returns
    -------
    fig : matplotlib figure
        Figure for the plot.

\end{sphinxVerbatim}



\end{sphinxuseclass}
\end{sphinxuseclass}
}

\end{sphinxuseclass}
\end{sphinxuseclass}
\sphinxAtStartPar
We can then use these results to visualize the operational envelope for the system over each case. In this case, the parameter ranges of the sine wave are plotted, showing that the rover can only a low ration of amplitude to wavelenght.

\begin{sphinxuseclass}{nbinput}
{
\sphinxsetup{VerbatimColor={named}{nbsphinx-code-bg}}
\sphinxsetup{VerbatimBorderColor={named}{nbsphinx-code-border}}
\begin{sphinxVerbatim}[commandchars=\\\{\}]
\llap{\color{nbsphinxin}[15]:\,\hspace{\fboxrule}\hspace{\fboxsep}}\PYG{n}{fig} \PYG{o}{=} \PYG{n}{rd}\PYG{o}{.}\PYG{n}{plot}\PYG{o}{.}\PYG{n}{nominal\PYGZus{}vals\PYGZus{}2d}\PYG{p}{(}\PYG{n}{nomapp}\PYG{p}{,} \PYG{n}{endclasses}\PYG{p}{,} \PYG{l+s+s1}{\PYGZsq{}}\PYG{l+s+s1}{amp}\PYG{l+s+s1}{\PYGZsq{}}\PYG{p}{,} \PYG{l+s+s1}{\PYGZsq{}}\PYG{l+s+s1}{wavelength}\PYG{l+s+s1}{\PYGZsq{}}\PYG{p}{)}
\end{sphinxVerbatim}
}

\end{sphinxuseclass}
\begin{sphinxuseclass}{nboutput}
\begin{sphinxuseclass}{nblast}
\hrule height -\fboxrule\relax
\vspace{\nbsphinxcodecellspacing}

\makeatletter\setbox\nbsphinxpromptbox\box\voidb@x\makeatother

\begin{nbsphinxfancyoutput}

\begin{sphinxuseclass}{output_area}
\begin{sphinxuseclass}{}
\noindent\sphinxincludegraphics[width=382\sphinxpxdimen,height=278\sphinxpxdimen]{{docs_Nominal_Approach_Use-Cases_28_0}.png}

\end{sphinxuseclass}
\end{sphinxuseclass}
\end{nbsphinxfancyoutput}

\end{sphinxuseclass}
\end{sphinxuseclass}
\sphinxAtStartPar
The plot below shows the same results for the turn parameters.

\begin{sphinxuseclass}{nbinput}
{
\sphinxsetup{VerbatimColor={named}{nbsphinx-code-bg}}
\sphinxsetup{VerbatimBorderColor={named}{nbsphinx-code-border}}
\begin{sphinxVerbatim}[commandchars=\\\{\}]
\llap{\color{nbsphinxin}[16]:\,\hspace{\fboxrule}\hspace{\fboxsep}}\PYG{n}{fig} \PYG{o}{=} \PYG{n}{rd}\PYG{o}{.}\PYG{n}{plot}\PYG{o}{.}\PYG{n}{nominal\PYGZus{}vals\PYGZus{}2d}\PYG{p}{(}\PYG{n}{nomapp}\PYG{p}{,} \PYG{n}{endclasses}\PYG{p}{,} \PYG{l+s+s1}{\PYGZsq{}}\PYG{l+s+s1}{radius}\PYG{l+s+s1}{\PYGZsq{}}\PYG{p}{,} \PYG{l+s+s1}{\PYGZsq{}}\PYG{l+s+s1}{start}\PYG{l+s+s1}{\PYGZsq{}}\PYG{p}{)}
\end{sphinxVerbatim}
}

\end{sphinxuseclass}
\begin{sphinxuseclass}{nboutput}
\begin{sphinxuseclass}{nblast}
\hrule height -\fboxrule\relax
\vspace{\nbsphinxcodecellspacing}

\makeatletter\setbox\nbsphinxpromptbox\box\voidb@x\makeatother

\begin{nbsphinxfancyoutput}

\begin{sphinxuseclass}{output_area}
\begin{sphinxuseclass}{}
\noindent\sphinxincludegraphics[width=382\sphinxpxdimen,height=278\sphinxpxdimen]{{docs_Nominal_Approach_Use-Cases_30_0}.png}

\end{sphinxuseclass}
\end{sphinxuseclass}
\end{nbsphinxfancyoutput}

\end{sphinxuseclass}
\end{sphinxuseclass}
\sphinxAtStartPar
Because the primary effect is one in terms of radius, we might visualize this trend in one dimension instead:

\begin{sphinxuseclass}{nbinput}
{
\sphinxsetup{VerbatimColor={named}{nbsphinx-code-bg}}
\sphinxsetup{VerbatimBorderColor={named}{nbsphinx-code-border}}
\begin{sphinxVerbatim}[commandchars=\\\{\}]
\llap{\color{nbsphinxin}[17]:\,\hspace{\fboxrule}\hspace{\fboxsep}}\PYG{n}{fig} \PYG{o}{=} \PYG{n}{rd}\PYG{o}{.}\PYG{n}{plot}\PYG{o}{.}\PYG{n}{nominal\PYGZus{}vals\PYGZus{}1d}\PYG{p}{(}\PYG{n}{nomapp}\PYG{p}{,} \PYG{n}{endclasses}\PYG{p}{,} \PYG{l+s+s1}{\PYGZsq{}}\PYG{l+s+s1}{radius}\PYG{l+s+s1}{\PYGZsq{}}\PYG{p}{)}
\end{sphinxVerbatim}
}

\end{sphinxuseclass}
\begin{sphinxuseclass}{nboutput}
\begin{sphinxuseclass}{nblast}
\hrule height -\fboxrule\relax
\vspace{\nbsphinxcodecellspacing}

\makeatletter\setbox\nbsphinxpromptbox\box\voidb@x\makeatother

\begin{nbsphinxfancyoutput}

\begin{sphinxuseclass}{output_area}
\begin{sphinxuseclass}{}
\noindent\sphinxincludegraphics[width=352\sphinxpxdimen,height=278\sphinxpxdimen]{{docs_Nominal_Approach_Use-Cases_32_0}.png}

\end{sphinxuseclass}
\end{sphinxuseclass}
\end{nbsphinxfancyoutput}

\end{sphinxuseclass}
\end{sphinxuseclass}
\sphinxAtStartPar
While this is helpful for plotting string classifications, we also might want to compare numeric quantities (e.g., costs, hazard probabilities, etc) over the set of factors. For this, \sphinxcode{\sphinxupquote{rd.tabulate.nominal\_factor\_comparison}} is used, which creates a table of metrics over a given set of parameters.

\begin{sphinxuseclass}{nbinput}
{
\sphinxsetup{VerbatimColor={named}{nbsphinx-code-bg}}
\sphinxsetup{VerbatimBorderColor={named}{nbsphinx-code-border}}
\begin{sphinxVerbatim}[commandchars=\\\{\}]
\llap{\color{nbsphinxin}[18]:\,\hspace{\fboxrule}\hspace{\fboxsep}}\PYG{n}{help}\PYG{p}{(}\PYG{n}{rd}\PYG{o}{.}\PYG{n}{tabulate}\PYG{o}{.}\PYG{n}{nominal\PYGZus{}factor\PYGZus{}comparison}\PYG{p}{)}
\end{sphinxVerbatim}
}

\end{sphinxuseclass}
\begin{sphinxuseclass}{nboutput}
\begin{sphinxuseclass}{nblast}
{

\kern-\sphinxverbatimsmallskipamount\kern-\baselineskip
\kern+\FrameHeightAdjust\kern-\fboxrule
\vspace{\nbsphinxcodecellspacing}

\sphinxsetup{VerbatimColor={named}{white}}
\sphinxsetup{VerbatimBorderColor={named}{nbsphinx-code-border}}
\begin{sphinxuseclass}{output_area}
\begin{sphinxuseclass}{}


\begin{sphinxVerbatim}[commandchars=\\\{\}]
Help on function nominal\_factor\_comparison in module fmdtools.resultdisp.tabulate:

nominal\_factor\_comparison(nomapp, endclasses, params, metrics='all', rangeid='default', nan\_as=nan, percent=True, difference=True, give\_ci=False, **kwargs)
    Compares a metric for a given set of model parameters/factors over set of nominal scenarios.

    Parameters
    ----------
    nomapp : NominalApproach
        Nominal Approach used to generate the simulations
    endclasses : dict
        dict of endclasses from propagate.nominal\_approach or nested\_approach with structure:
            \{scen\_x:\{metric1:x, metric2:x{\ldots}\}\} or \{scen\_x:\{fault:\{metric1:x, metric2:x{\ldots}\}\}\}
    params : list/str
        List of parameters (or parameter) to use for the factor levels in the comparison
    metrics : 'all'/list, optional
        Metrics to show in the table. The default is 'all'.
    rangeid : str, optional
        Nominal Approach range to use for the test (must be run over a single range).
        The default is 'default', which picks the only range (if there is only one).
    nan\_as : float, optional
        Number to parse NaNs as (if present). The default is np.nan.
    percent : bool, optional
        Whether to compare metrics as bools (True - results in a comparison of percentages of indicator variables)
        or as averages (False - results in a comparison of average values of real valued variables). The default is True.
    difference : bool, optional
        Whether to tabulate the difference of the metric from the nominal over each scenario (True),
        or the value of the metric over all (False). The default is True.
    give\_ci = bool:
        gives the bootstrap confidence interval for the given statistic using the given kwargs
        'combined' combines the values as a strings in the table (for display)
    kwargs : keyword arguments for bootstrap\_confidence\_interval (sample\_size, num\_samples, interval, seed)
    Returns
    -------
    table : pandas table
        Table with the metric statistic (percent or average) over the nominal scenario and each listed function/mode (as differences or averages)

\end{sphinxVerbatim}



\end{sphinxuseclass}
\end{sphinxuseclass}
}

\end{sphinxuseclass}
\end{sphinxuseclass}
\begin{sphinxuseclass}{nbinput}
\begin{sphinxuseclass}{nblast}
{
\sphinxsetup{VerbatimColor={named}{nbsphinx-code-bg}}
\sphinxsetup{VerbatimBorderColor={named}{nbsphinx-code-border}}
\begin{sphinxVerbatim}[commandchars=\\\{\}]
\llap{\color{nbsphinxin}[19]:\,\hspace{\fboxrule}\hspace{\fboxsep}}\PYG{n}{nomtab} \PYG{o}{=} \PYG{n}{rd}\PYG{o}{.}\PYG{n}{tabulate}\PYG{o}{.}\PYG{n}{nominal\PYGZus{}factor\PYGZus{}comparison}\PYG{p}{(}\PYG{n}{nomapp}\PYG{p}{,} \PYG{n}{endclasses}\PYG{p}{,} \PYG{p}{[}\PYG{l+s+s1}{\PYGZsq{}}\PYG{l+s+s1}{radius}\PYG{l+s+s1}{\PYGZsq{}}\PYG{p}{,} \PYG{l+s+s1}{\PYGZsq{}}\PYG{l+s+s1}{start}\PYG{l+s+s1}{\PYGZsq{}}\PYG{p}{]}\PYG{p}{,} \PYG{n}{rangeid}\PYG{o}{=}\PYG{l+s+s1}{\PYGZsq{}}\PYG{l+s+s1}{turn}\PYG{l+s+s1}{\PYGZsq{}}\PYG{p}{,} \PYG{n}{percent}\PYG{o}{=}\PYG{k+kc}{False}\PYG{p}{)}
\end{sphinxVerbatim}
}

\end{sphinxuseclass}
\end{sphinxuseclass}
\begin{sphinxuseclass}{nbinput}
{
\sphinxsetup{VerbatimColor={named}{nbsphinx-code-bg}}
\sphinxsetup{VerbatimBorderColor={named}{nbsphinx-code-border}}
\begin{sphinxVerbatim}[commandchars=\\\{\}]
\llap{\color{nbsphinxin}[20]:\,\hspace{\fboxrule}\hspace{\fboxsep}}\PYG{n}{nomtab}
\end{sphinxVerbatim}
}

\end{sphinxuseclass}
\begin{sphinxuseclass}{nboutput}
\begin{sphinxuseclass}{nblast}
{

\kern-\sphinxverbatimsmallskipamount\kern-\baselineskip
\kern+\FrameHeightAdjust\kern-\fboxrule
\vspace{\nbsphinxcodecellspacing}

\sphinxsetup{VerbatimColor={named}{white}}
\sphinxsetup{VerbatimBorderColor={named}{nbsphinx-code-border}}
\begin{sphinxuseclass}{output_area}
\begin{sphinxuseclass}{}


\begin{sphinxVerbatim}[commandchars=\\\{\}]
\llap{\color{nbsphinxout}[20]:\,\hspace{\fboxrule}\hspace{\fboxsep}}                     5                                             10  \textbackslash{}
                     0           5         10          15          0
rate           0.000000    0.000000  0.000000    0.000000    0.000000
cost           0.000000  500.000000  0.000000  500.000000  500.000000
prob           0.035714    0.035714  0.035714    0.035714    0.035714
expected cost  0.000000   17.857143  0.000000   17.857143   17.857143

                                                         15            {\ldots}  \textbackslash{}
                       5           10          15        0         5   {\ldots}
rate             0.000000    0.000000    0.000000  0.000000  0.000000  {\ldots}
cost           500.000000  500.000000  500.000000  0.000000  0.000000  {\ldots}
prob             0.035714    0.035714    0.035714  0.035714  0.035714  {\ldots}
expected cost   17.857143   17.857143   17.857143  0.000000  0.000000  {\ldots}

                     25                  30                                \textbackslash{}
                     10        15        0         5         10        15
rate           0.000000  0.000000  0.000000  0.000000  0.000000  0.000000
cost           0.000000  0.000000  0.000000  0.000000  0.000000  0.000000
prob           0.035714  0.035714  0.035714  0.035714  0.035714  0.035714
expected cost  0.000000  0.000000  0.000000  0.000000  0.000000  0.000000

                     35
                     0         5         10        15
rate           0.000000  0.000000  0.000000  0.000000
cost           0.000000  0.000000  0.000000  0.000000
prob           0.035714  0.035714  0.035714  0.035714
expected cost  0.000000  0.000000  0.000000  0.000000

[4 rows x 28 columns]
\end{sphinxVerbatim}



\end{sphinxuseclass}
\end{sphinxuseclass}
}

\end{sphinxuseclass}
\end{sphinxuseclass}
\sphinxAtStartPar
This table can also be summarized on individual factors:

\begin{sphinxuseclass}{nbinput}
{
\sphinxsetup{VerbatimColor={named}{nbsphinx-code-bg}}
\sphinxsetup{VerbatimBorderColor={named}{nbsphinx-code-border}}
\begin{sphinxVerbatim}[commandchars=\\\{\}]
\llap{\color{nbsphinxin}[21]:\,\hspace{\fboxrule}\hspace{\fboxsep}}\PYG{n}{nomtab\PYGZus{}summ} \PYG{o}{=} \PYG{n}{rd}\PYG{o}{.}\PYG{n}{tabulate}\PYG{o}{.}\PYG{n}{nominal\PYGZus{}factor\PYGZus{}comparison}\PYG{p}{(}\PYG{n}{nomapp}\PYG{p}{,} \PYG{n}{endclasses}\PYG{p}{,} \PYG{p}{[}\PYG{l+s+s1}{\PYGZsq{}}\PYG{l+s+s1}{start}\PYG{l+s+s1}{\PYGZsq{}}\PYG{p}{]}\PYG{p}{,} \PYG{n}{rangeid}\PYG{o}{=}\PYG{l+s+s1}{\PYGZsq{}}\PYG{l+s+s1}{turn}\PYG{l+s+s1}{\PYGZsq{}}\PYG{p}{,} \PYG{n}{percent}\PYG{o}{=}\PYG{k+kc}{False}\PYG{p}{)}
\PYG{n}{nomtab\PYGZus{}summ}
\end{sphinxVerbatim}
}

\end{sphinxuseclass}
\begin{sphinxuseclass}{nboutput}
\begin{sphinxuseclass}{nblast}
{

\kern-\sphinxverbatimsmallskipamount\kern-\baselineskip
\kern+\FrameHeightAdjust\kern-\fboxrule
\vspace{\nbsphinxcodecellspacing}

\sphinxsetup{VerbatimColor={named}{white}}
\sphinxsetup{VerbatimBorderColor={named}{nbsphinx-code-border}}
\begin{sphinxuseclass}{output_area}
\begin{sphinxuseclass}{}


\begin{sphinxVerbatim}[commandchars=\\\{\}]
\llap{\color{nbsphinxout}[21]:\,\hspace{\fboxrule}\hspace{\fboxsep}}('start',)            0           5           10          15
rate            0.000000    0.000000    0.000000    0.000000
cost           71.428571  142.857143  142.857143  142.857143
prob            0.035714    0.035714    0.035714    0.035714
expected cost   2.551020    5.102041    5.102041    5.102041
\end{sphinxVerbatim}



\end{sphinxuseclass}
\end{sphinxuseclass}
}

\end{sphinxuseclass}
\end{sphinxuseclass}
\sphinxAtStartPar
\sphinxcode{\sphinxupquote{rd.plot.nominal\_factor\_comparison}} can then be used to visualize one metric from this table as a bar plot.

\begin{sphinxuseclass}{nbinput}
{
\sphinxsetup{VerbatimColor={named}{nbsphinx-code-bg}}
\sphinxsetup{VerbatimBorderColor={named}{nbsphinx-code-border}}
\begin{sphinxVerbatim}[commandchars=\\\{\}]
\llap{\color{nbsphinxin}[22]:\,\hspace{\fboxrule}\hspace{\fboxsep}}\PYG{n}{help}\PYG{p}{(}\PYG{n}{rd}\PYG{o}{.}\PYG{n}{plot}\PYG{o}{.}\PYG{n}{nominal\PYGZus{}factor\PYGZus{}comparison}\PYG{p}{)}
\end{sphinxVerbatim}
}

\end{sphinxuseclass}
\begin{sphinxuseclass}{nboutput}
\begin{sphinxuseclass}{nblast}
{

\kern-\sphinxverbatimsmallskipamount\kern-\baselineskip
\kern+\FrameHeightAdjust\kern-\fboxrule
\vspace{\nbsphinxcodecellspacing}

\sphinxsetup{VerbatimColor={named}{white}}
\sphinxsetup{VerbatimBorderColor={named}{nbsphinx-code-border}}
\begin{sphinxuseclass}{output_area}
\begin{sphinxuseclass}{}


\begin{sphinxVerbatim}[commandchars=\\\{\}]
Help on function nominal\_factor\_comparison in module fmdtools.resultdisp.plot:

nominal\_factor\_comparison(comparison\_table, metric, ylabel='proportion', figsize=(6, 4), title='', maxy='max', xlabel=True, error\_bars=False)
    Compares/plots a comparison table from tabulate.nominal\_factor\_comparison as a bar plot for a given metric.

    Parameters
    ----------
    comparison\_table : pandas table
        Table from tabulate.nominal\_factor\_comparison
    metrics : string
        Metric to use in the plot
    ylabel : string, optional
        label for the y-axis. The default is 'proportion'.
    figsize : tuple, optional
        Size for the plot. The default is (12,8).
    title : str, optional
        Plot title. The default is ''.
    maxy : float
        Cutoff for the y-axis (to use if the default is bad). The default is 'max'
    xlabel : TYPE, optional
        DESCRIPTION. The default is True.
    error\_bars : TYPE, optional
        DESCRIPTION. The default is False.

    Returns
    -------
    figure: matplotlib figure

\end{sphinxVerbatim}



\end{sphinxuseclass}
\end{sphinxuseclass}
}

\end{sphinxuseclass}
\end{sphinxuseclass}
\begin{sphinxuseclass}{nbinput}
{
\sphinxsetup{VerbatimColor={named}{nbsphinx-code-bg}}
\sphinxsetup{VerbatimBorderColor={named}{nbsphinx-code-border}}
\begin{sphinxVerbatim}[commandchars=\\\{\}]
\llap{\color{nbsphinxin}[23]:\,\hspace{\fboxrule}\hspace{\fboxsep}}\PYG{n}{fig} \PYG{o}{=} \PYG{n}{rd}\PYG{o}{.}\PYG{n}{plot}\PYG{o}{.}\PYG{n}{nominal\PYGZus{}factor\PYGZus{}comparison}\PYG{p}{(}\PYG{n}{nomtab\PYGZus{}summ}\PYG{p}{,} \PYG{l+s+s1}{\PYGZsq{}}\PYG{l+s+s1}{cost}\PYG{l+s+s1}{\PYGZsq{}}\PYG{p}{,} \PYG{n}{ylabel}\PYG{o}{=}\PYG{l+s+s1}{\PYGZsq{}}\PYG{l+s+s1}{cost}\PYG{l+s+s1}{\PYGZsq{}}\PYG{p}{,} \PYG{n}{title}\PYG{o}{=}\PYG{l+s+s1}{\PYGZsq{}}\PYG{l+s+s1}{average cost at different start locations}\PYG{l+s+s1}{\PYGZsq{}}\PYG{p}{,} \PYG{n}{maxy}\PYG{o}{=}\PYG{l+m+mi}{150}\PYG{p}{)}
\end{sphinxVerbatim}
}

\end{sphinxuseclass}
\begin{sphinxuseclass}{nboutput}
\begin{sphinxuseclass}{nblast}
\hrule height -\fboxrule\relax
\vspace{\nbsphinxcodecellspacing}

\makeatletter\setbox\nbsphinxpromptbox\box\voidb@x\makeatother

\begin{nbsphinxfancyoutput}

\begin{sphinxuseclass}{output_area}
\begin{sphinxuseclass}{}
\noindent\sphinxincludegraphics[width=389\sphinxpxdimen,height=264\sphinxpxdimen]{{docs_Nominal_Approach_Use-Cases_41_0}.png}

\end{sphinxuseclass}
\end{sphinxuseclass}
\end{nbsphinxfancyoutput}

\end{sphinxuseclass}
\end{sphinxuseclass}

\subsubsection{Quantifying probabilities}
\label{\detokenize{docs/Nominal_Approach_Use-Cases:Quantifying-probabilities}}
\sphinxAtStartPar
Given the ability to simulate over ranges, it can additionally be used to quantify probabilities of the different end\sphinxhyphen{}state classifications. \sphinxcode{\sphinxupquote{rd.process.state\_probabilities(endclasses)}} can be used to quantify the probability these classifications.

\sphinxAtStartPar
The default probability model over ranges is to assume a uniform distribution and only assume one range has been added. This can lead to fallacious results:

\begin{sphinxuseclass}{nbinput}
{
\sphinxsetup{VerbatimColor={named}{nbsphinx-code-bg}}
\sphinxsetup{VerbatimBorderColor={named}{nbsphinx-code-border}}
\begin{sphinxVerbatim}[commandchars=\\\{\}]
\llap{\color{nbsphinxin}[24]:\,\hspace{\fboxrule}\hspace{\fboxsep}}\PYG{n}{state\PYGZus{}probabilities} \PYG{o}{=} \PYG{n}{rd}\PYG{o}{.}\PYG{n}{process}\PYG{o}{.}\PYG{n}{state\PYGZus{}probabilities}\PYG{p}{(}\PYG{n}{endclasses}\PYG{p}{)}
\PYG{n}{state\PYGZus{}probabilities}
\end{sphinxVerbatim}
}

\end{sphinxuseclass}
\begin{sphinxuseclass}{nboutput}
\begin{sphinxuseclass}{nblast}
{

\kern-\sphinxverbatimsmallskipamount\kern-\baselineskip
\kern+\FrameHeightAdjust\kern-\fboxrule
\vspace{\nbsphinxcodecellspacing}

\sphinxsetup{VerbatimColor={named}{white}}
\sphinxsetup{VerbatimBorderColor={named}{nbsphinx-code-border}}
\begin{sphinxuseclass}{output_area}
\begin{sphinxuseclass}{}


\begin{sphinxVerbatim}[commandchars=\\\{\}]
\llap{\color{nbsphinxout}[24]:\,\hspace{\fboxrule}\hspace{\fboxsep}}\{'incomplete mission': 1.0050000000000006,
 'nominal mission': 0.9949999999999999\}
\end{sphinxVerbatim}



\end{sphinxuseclass}
\end{sphinxuseclass}
}

\end{sphinxuseclass}
\end{sphinxuseclass}
\sphinxAtStartPar
Thus, \sphinxcode{\sphinxupquote{.assoc\_probs}} to: \sphinxhyphen{} associate the probabilities with their corresponding distributions (which may be non\sphinxhyphen{}uniform), and \sphinxhyphen{} rebalance the overall probability of discrete cases when running the approach over ranges.

\begin{sphinxuseclass}{nbinput}
{
\sphinxsetup{VerbatimColor={named}{nbsphinx-code-bg}}
\sphinxsetup{VerbatimBorderColor={named}{nbsphinx-code-border}}
\begin{sphinxVerbatim}[commandchars=\\\{\}]
\llap{\color{nbsphinxin}[25]:\,\hspace{\fboxrule}\hspace{\fboxsep}}\PYG{n}{help}\PYG{p}{(}\PYG{n}{nomapp}\PYG{o}{.}\PYG{n}{assoc\PYGZus{}probs}\PYG{p}{)}
\end{sphinxVerbatim}
}

\end{sphinxuseclass}
\begin{sphinxuseclass}{nboutput}
\begin{sphinxuseclass}{nblast}
{

\kern-\sphinxverbatimsmallskipamount\kern-\baselineskip
\kern+\FrameHeightAdjust\kern-\fboxrule
\vspace{\nbsphinxcodecellspacing}

\sphinxsetup{VerbatimColor={named}{white}}
\sphinxsetup{VerbatimBorderColor={named}{nbsphinx-code-border}}
\begin{sphinxuseclass}{output_area}
\begin{sphinxuseclass}{}


\begin{sphinxVerbatim}[commandchars=\\\{\}]
Help on method assoc\_probs in module fmdtools.modeldef:

assoc\_probs(rangeid, prob\_weight=1.0, **inputpdfs) method of fmdtools.modeldef.NominalApproach instance
    Associates a probability model (assuming variable independence) with a
    given previously-defined range of scenarios using given pdfs

    Parameters
    ----------
    rangeid : str
        Name of the range to apply the probability model to.
    prob\_weight : float, optional
        Overall probability for the set of scenarios (to use if adding more ranges
        or if the range does not cover the space of probability). The default is 1.0.
    **inputpdfs : key=(pdf, params)
        pdf to associate with the different variables of the model.
        Where the pdf has form pdf(x, **kwargs) where x is the location and **kwargs is parameters
        (for example, scipy.stats.norm.pdf)
        and params is a dictionary of parameters (e.g., \{'mu':1,'std':1\}) to use '
        as the key/parameter inputs to the pdf

\end{sphinxVerbatim}



\end{sphinxuseclass}
\end{sphinxuseclass}
}

\end{sphinxuseclass}
\end{sphinxuseclass}
\sphinxAtStartPar
Here, each case is given a weight defining the probability of the discrete case, while the corresponding parameters are given corresponding pdf functions (in this case uniform distributions from the scipy stats package).

\begin{sphinxuseclass}{nbinput}
\begin{sphinxuseclass}{nblast}
{
\sphinxsetup{VerbatimColor={named}{nbsphinx-code-bg}}
\sphinxsetup{VerbatimBorderColor={named}{nbsphinx-code-border}}
\begin{sphinxVerbatim}[commandchars=\\\{\}]
\llap{\color{nbsphinxin}[26]:\,\hspace{\fboxrule}\hspace{\fboxsep}}\PYG{k+kn}{from} \PYG{n+nn}{scipy} \PYG{k+kn}{import} \PYG{n}{stats}
\end{sphinxVerbatim}
}

\end{sphinxuseclass}
\end{sphinxuseclass}
\begin{sphinxuseclass}{nbinput}
\begin{sphinxuseclass}{nblast}
{
\sphinxsetup{VerbatimColor={named}{nbsphinx-code-bg}}
\sphinxsetup{VerbatimBorderColor={named}{nbsphinx-code-border}}
\begin{sphinxVerbatim}[commandchars=\\\{\}]
\llap{\color{nbsphinxin}[27]:\,\hspace{\fboxrule}\hspace{\fboxsep}}\PYG{n}{nomapp}\PYG{o}{.}\PYG{n}{assoc\PYGZus{}probs}\PYG{p}{(}\PYG{l+s+s1}{\PYGZsq{}}\PYG{l+s+s1}{sine}\PYG{l+s+s1}{\PYGZsq{}}\PYG{p}{,} \PYG{n}{prob\PYGZus{}weight}\PYG{o}{=}\PYG{l+m+mf}{0.5}\PYG{p}{,} \PYG{n}{amp}\PYG{o}{=}\PYG{p}{(}\PYG{n}{stats}\PYG{o}{.}\PYG{n}{uniform}\PYG{o}{.}\PYG{n}{pdf}\PYG{p}{,} \PYG{p}{\PYGZob{}}\PYG{l+s+s1}{\PYGZsq{}}\PYG{l+s+s1}{loc}\PYG{l+s+s1}{\PYGZsq{}}\PYG{p}{:}\PYG{l+m+mi}{0}\PYG{p}{,}\PYG{l+s+s1}{\PYGZsq{}}\PYG{l+s+s1}{scale}\PYG{l+s+s1}{\PYGZsq{}}\PYG{p}{:}\PYG{l+m+mi}{10}\PYG{p}{\PYGZcb{}}\PYG{p}{)}\PYG{p}{,} \PYG{n}{wavelength}\PYG{o}{=}\PYG{p}{(}\PYG{n}{stats}\PYG{o}{.}\PYG{n}{uniform}\PYG{o}{.}\PYG{n}{pdf}\PYG{p}{,}\PYG{p}{\PYGZob{}}\PYG{l+s+s1}{\PYGZsq{}}\PYG{l+s+s1}{loc}\PYG{l+s+s1}{\PYGZsq{}}\PYG{p}{:}\PYG{l+m+mi}{10}\PYG{p}{,} \PYG{l+s+s1}{\PYGZsq{}}\PYG{l+s+s1}{scale}\PYG{l+s+s1}{\PYGZsq{}}\PYG{p}{:}\PYG{l+m+mi}{40}\PYG{p}{\PYGZcb{}}\PYG{p}{)}\PYG{p}{)}
\PYG{n}{nomapp}\PYG{o}{.}\PYG{n}{assoc\PYGZus{}probs}\PYG{p}{(}\PYG{l+s+s1}{\PYGZsq{}}\PYG{l+s+s1}{turn}\PYG{l+s+s1}{\PYGZsq{}}\PYG{p}{,} \PYG{n}{prob\PYGZus{}weight}\PYG{o}{=}\PYG{l+m+mf}{0.5}\PYG{p}{,} \PYG{n}{start}\PYG{o}{=}\PYG{p}{(}\PYG{n}{stats}\PYG{o}{.}\PYG{n}{uniform}\PYG{o}{.}\PYG{n}{pdf}\PYG{p}{,} \PYG{p}{\PYGZob{}}\PYG{l+s+s1}{\PYGZsq{}}\PYG{l+s+s1}{loc}\PYG{l+s+s1}{\PYGZsq{}}\PYG{p}{:}\PYG{l+m+mi}{5}\PYG{p}{,}\PYG{l+s+s1}{\PYGZsq{}}\PYG{l+s+s1}{scale}\PYG{l+s+s1}{\PYGZsq{}}\PYG{p}{:}\PYG{l+m+mi}{10}\PYG{p}{\PYGZcb{}}\PYG{p}{)}\PYG{p}{,} \PYG{n}{radius}\PYG{o}{=}\PYG{p}{(}\PYG{n}{stats}\PYG{o}{.}\PYG{n}{uniform}\PYG{o}{.}\PYG{n}{pdf}\PYG{p}{,}\PYG{p}{\PYGZob{}}\PYG{l+s+s1}{\PYGZsq{}}\PYG{l+s+s1}{loc}\PYG{l+s+s1}{\PYGZsq{}}\PYG{p}{:}\PYG{l+m+mi}{5}\PYG{p}{,} \PYG{l+s+s1}{\PYGZsq{}}\PYG{l+s+s1}{scale}\PYG{l+s+s1}{\PYGZsq{}}\PYG{p}{:}\PYG{l+m+mi}{30}\PYG{p}{\PYGZcb{}}\PYG{p}{)}\PYG{p}{)}
\end{sphinxVerbatim}
}

\end{sphinxuseclass}
\end{sphinxuseclass}
\sphinxAtStartPar
Now, when the scenarios are run, they should have the correct corresponding probabilities:

\begin{sphinxuseclass}{nbinput}
{
\sphinxsetup{VerbatimColor={named}{nbsphinx-code-bg}}
\sphinxsetup{VerbatimBorderColor={named}{nbsphinx-code-border}}
\begin{sphinxVerbatim}[commandchars=\\\{\}]
\llap{\color{nbsphinxin}[28]:\,\hspace{\fboxrule}\hspace{\fboxsep}}\PYG{n}{endclasses}\PYG{p}{,} \PYG{n}{mdlhists}\PYG{o}{=} \PYG{n}{prop}\PYG{o}{.}\PYG{n}{nominal\PYGZus{}approach}\PYG{p}{(}\PYG{n}{mdl}\PYG{p}{,} \PYG{n}{nomapp}\PYG{p}{)}
\end{sphinxVerbatim}
}

\end{sphinxuseclass}
\begin{sphinxuseclass}{nboutput}
\begin{sphinxuseclass}{nblast}
{

\kern-\sphinxverbatimsmallskipamount\kern-\baselineskip
\kern+\FrameHeightAdjust\kern-\fboxrule
\vspace{\nbsphinxcodecellspacing}

\sphinxsetup{VerbatimColor={named}{nbsphinx-stderr}}
\sphinxsetup{VerbatimBorderColor={named}{nbsphinx-code-border}}
\begin{sphinxuseclass}{output_area}
\begin{sphinxuseclass}{stderr}


\begin{sphinxVerbatim}[commandchars=\\\{\}]
SCENARIOS COMPLETE: 100\%|████████████████████████████████████████████████████████████| 228/228 [00:03<00:00, 72.64it/s]
\end{sphinxVerbatim}



\end{sphinxuseclass}
\end{sphinxuseclass}
}

\end{sphinxuseclass}
\end{sphinxuseclass}
\begin{sphinxuseclass}{nbinput}
{
\sphinxsetup{VerbatimColor={named}{nbsphinx-code-bg}}
\sphinxsetup{VerbatimBorderColor={named}{nbsphinx-code-border}}
\begin{sphinxVerbatim}[commandchars=\\\{\}]
\llap{\color{nbsphinxin}[29]:\,\hspace{\fboxrule}\hspace{\fboxsep}}\PYG{n}{state\PYGZus{}probabilities} \PYG{o}{=} \PYG{n}{rd}\PYG{o}{.}\PYG{n}{process}\PYG{o}{.}\PYG{n}{state\PYGZus{}probabilities}\PYG{p}{(}\PYG{n}{endclasses}\PYG{p}{)}
\PYG{n}{state\PYGZus{}probabilities}
\end{sphinxVerbatim}
}

\end{sphinxuseclass}
\begin{sphinxuseclass}{nboutput}
\begin{sphinxuseclass}{nblast}
{

\kern-\sphinxverbatimsmallskipamount\kern-\baselineskip
\kern+\FrameHeightAdjust\kern-\fboxrule
\vspace{\nbsphinxcodecellspacing}

\sphinxsetup{VerbatimColor={named}{white}}
\sphinxsetup{VerbatimBorderColor={named}{nbsphinx-code-border}}
\begin{sphinxuseclass}{output_area}
\begin{sphinxuseclass}{}


\begin{sphinxVerbatim}[commandchars=\\\{\}]
\llap{\color{nbsphinxout}[29]:\,\hspace{\fboxrule}\hspace{\fboxsep}}\{'incomplete mission': 0.5203571428571433,
 'nominal mission': 0.4796428571428575\}
\end{sphinxVerbatim}



\end{sphinxuseclass}
\end{sphinxuseclass}
}

\end{sphinxuseclass}
\end{sphinxuseclass}
\sphinxAtStartPar
This result (while seemingly extreme due to the large ranges explored) is consistent with the operational envelopes presented earlier.


\subsubsection{Random input generation}
\label{\detokenize{docs/Nominal_Approach_Use-Cases:Random-input-generation}}
\sphinxAtStartPar
It may additionally be helpful to perform simulations over randomly generated inputs, when there are many parameters and it is difficult to easily quantify the stochastic process. This approach is called Monte Carlo sampling, and enables one to quantify probabilities given stochasticly\sphinxhyphen{}generated inputs. Random inputs are assigned using \sphinxcode{\sphinxupquote{.add\_rand\_params}}

\begin{sphinxuseclass}{nbinput}
{
\sphinxsetup{VerbatimColor={named}{nbsphinx-code-bg}}
\sphinxsetup{VerbatimBorderColor={named}{nbsphinx-code-border}}
\begin{sphinxVerbatim}[commandchars=\\\{\}]
\llap{\color{nbsphinxin}[30]:\,\hspace{\fboxrule}\hspace{\fboxsep}}\PYG{n}{nomapp\PYGZus{}rand} \PYG{o}{=} \PYG{n}{NominalApproach}\PYG{p}{(}\PYG{p}{)}
\PYG{n}{help}\PYG{p}{(}\PYG{n}{nomapp\PYGZus{}rand}\PYG{o}{.}\PYG{n}{add\PYGZus{}rand\PYGZus{}params}\PYG{p}{)}
\end{sphinxVerbatim}
}

\end{sphinxuseclass}
\begin{sphinxuseclass}{nboutput}
\begin{sphinxuseclass}{nblast}
{

\kern-\sphinxverbatimsmallskipamount\kern-\baselineskip
\kern+\FrameHeightAdjust\kern-\fboxrule
\vspace{\nbsphinxcodecellspacing}

\sphinxsetup{VerbatimColor={named}{white}}
\sphinxsetup{VerbatimBorderColor={named}{nbsphinx-code-border}}
\begin{sphinxuseclass}{output_area}
\begin{sphinxuseclass}{}


\begin{sphinxVerbatim}[commandchars=\\\{\}]
Help on method add\_rand\_params in module fmdtools.modeldef:

add\_rand\_params(paramfunc, rangeid, *fixedargs, prob\_weight=1.0, replicates=1000, seeds='shared', **randvars) method of fmdtools.modeldef.NominalApproach instance
    Adds a set of random scenarios to the approach.

    Parameters
    ----------
    paramfunc : method
        Python method which generates a set of model parameters given the input arguments.
        method should have form: method(fixedarg, fixedarg{\ldots}, inputarg=X, inputarg=X)
    rangeid : str
        Name for the range being used. Default is 'nominal'
    prob\_weight : float (0-1)
        Overall probability for the set of scenarios (to use if adding more ranges). Default is 1.0
    *fixedargs : any
        Fixed positional arguments in the parameter generator function.
        Useful for discrete modes with different parameters.
    seeds : str/list
        Options for seeding models/replicates: (Default is 'shared')
            - 'shared' creates random seeds and shares them between parameters and models
            - 'independent' creates separate random seeds for models and parameter generation
            - 'keep\_model' uses the seed provided in the model for all of the model
        When a list is provided, these seeds are are used (and shared). Must be of length replicates.
    **randvars : key=tuple
        Specification for each random input parameter, specified as
        input = (randfunc, param1, param2{\ldots})
        where randfunc is the method producing random outputs (e.g. numpy.random.rand)
        and the successive parameters param1, param2, etc are inputs to the method

\end{sphinxVerbatim}



\end{sphinxuseclass}
\end{sphinxuseclass}
}

\end{sphinxuseclass}
\end{sphinxuseclass}
\sphinxAtStartPar
Below, the same probability model/scenarios used above will be generated using stochastic inputs (rather than a range).

\begin{sphinxuseclass}{nbinput}
\begin{sphinxuseclass}{nblast}
{
\sphinxsetup{VerbatimColor={named}{nbsphinx-code-bg}}
\sphinxsetup{VerbatimBorderColor={named}{nbsphinx-code-border}}
\begin{sphinxVerbatim}[commandchars=\\\{\}]
\llap{\color{nbsphinxin}[31]:\,\hspace{\fboxrule}\hspace{\fboxsep}}\PYG{k+kn}{import} \PYG{n+nn}{numpy} \PYG{k}{as} \PYG{n+nn}{np}
\end{sphinxVerbatim}
}

\end{sphinxuseclass}
\end{sphinxuseclass}
\begin{sphinxuseclass}{nbinput}
\begin{sphinxuseclass}{nblast}
{
\sphinxsetup{VerbatimColor={named}{nbsphinx-code-bg}}
\sphinxsetup{VerbatimBorderColor={named}{nbsphinx-code-border}}
\begin{sphinxVerbatim}[commandchars=\\\{\}]
\llap{\color{nbsphinxin}[32]:\,\hspace{\fboxrule}\hspace{\fboxsep}}\PYG{n}{nomapp\PYGZus{}rand}\PYG{o}{.}\PYG{n}{add\PYGZus{}rand\PYGZus{}params}\PYG{p}{(}\PYG{n}{gen\PYGZus{}params}\PYG{p}{,}\PYG{l+s+s1}{\PYGZsq{}}\PYG{l+s+s1}{sine}\PYG{l+s+s1}{\PYGZsq{}}\PYG{p}{,}\PYG{l+s+s1}{\PYGZsq{}}\PYG{l+s+s1}{sine}\PYG{l+s+s1}{\PYGZsq{}}\PYG{p}{,} \PYG{n}{prob\PYGZus{}weight}\PYG{o}{=}\PYG{l+m+mf}{0.5}\PYG{p}{,} \PYG{n}{replicates}\PYG{o}{=}\PYG{l+m+mi}{100}\PYG{p}{,} \PYG{n}{amp}\PYG{o}{=}\PYG{p}{(}\PYG{n}{np}\PYG{o}{.}\PYG{n}{random}\PYG{o}{.}\PYG{n}{uniform}\PYG{p}{,} \PYG{l+m+mi}{0}\PYG{p}{,} \PYG{l+m+mi}{10}\PYG{p}{)}\PYG{p}{,} \PYG{n}{wavelength}\PYG{o}{=}\PYG{p}{(}\PYG{n}{np}\PYG{o}{.}\PYG{n}{random}\PYG{o}{.}\PYG{n}{uniform}\PYG{p}{,}\PYG{l+m+mi}{10}\PYG{p}{,}\PYG{l+m+mi}{40}\PYG{p}{)}\PYG{p}{)}
\PYG{n}{nomapp\PYGZus{}rand}\PYG{o}{.}\PYG{n}{add\PYGZus{}rand\PYGZus{}params}\PYG{p}{(}\PYG{n}{gen\PYGZus{}params}\PYG{p}{,}\PYG{l+s+s1}{\PYGZsq{}}\PYG{l+s+s1}{turn}\PYG{l+s+s1}{\PYGZsq{}}\PYG{p}{,}\PYG{l+s+s1}{\PYGZsq{}}\PYG{l+s+s1}{turn}\PYG{l+s+s1}{\PYGZsq{}}\PYG{p}{,} \PYG{n}{prob\PYGZus{}weight}\PYG{o}{=}\PYG{l+m+mf}{0.5}\PYG{p}{,} \PYG{n}{replicates}\PYG{o}{=}\PYG{l+m+mi}{100}\PYG{p}{,} \PYG{n}{radius}\PYG{o}{=}\PYG{p}{(}\PYG{n}{np}\PYG{o}{.}\PYG{n}{random}\PYG{o}{.}\PYG{n}{uniform}\PYG{p}{,}\PYG{l+m+mi}{5}\PYG{p}{,}\PYG{l+m+mi}{40}\PYG{p}{)}\PYG{p}{,} \PYG{n}{start}\PYG{o}{=}\PYG{p}{(}\PYG{n}{np}\PYG{o}{.}\PYG{n}{random}\PYG{o}{.}\PYG{n}{uniform}\PYG{p}{,}\PYG{l+m+mi}{0}\PYG{p}{,} \PYG{l+m+mi}{20}\PYG{p}{)}\PYG{p}{)}
\end{sphinxVerbatim}
}

\end{sphinxuseclass}
\end{sphinxuseclass}
\sphinxAtStartPar
We can go through the same process to verify that it tracks the range/pdf method.

\begin{sphinxuseclass}{nbinput}
{
\sphinxsetup{VerbatimColor={named}{nbsphinx-code-bg}}
\sphinxsetup{VerbatimBorderColor={named}{nbsphinx-code-border}}
\begin{sphinxVerbatim}[commandchars=\\\{\}]
\llap{\color{nbsphinxin}[33]:\,\hspace{\fboxrule}\hspace{\fboxsep}}\PYG{n}{endclasses}\PYG{p}{,} \PYG{n}{mdlhists}\PYG{o}{=} \PYG{n}{prop}\PYG{o}{.}\PYG{n}{nominal\PYGZus{}approach}\PYG{p}{(}\PYG{n}{mdl}\PYG{p}{,} \PYG{n}{nomapp\PYGZus{}rand}\PYG{p}{)}
\end{sphinxVerbatim}
}

\end{sphinxuseclass}
\begin{sphinxuseclass}{nboutput}
\begin{sphinxuseclass}{nblast}
{

\kern-\sphinxverbatimsmallskipamount\kern-\baselineskip
\kern+\FrameHeightAdjust\kern-\fboxrule
\vspace{\nbsphinxcodecellspacing}

\sphinxsetup{VerbatimColor={named}{nbsphinx-stderr}}
\sphinxsetup{VerbatimBorderColor={named}{nbsphinx-code-border}}
\begin{sphinxuseclass}{output_area}
\begin{sphinxuseclass}{stderr}


\begin{sphinxVerbatim}[commandchars=\\\{\}]
SCENARIOS COMPLETE: 100\%|████████████████████████████████████████████████████████████| 200/200 [00:02<00:00, 71.57it/s]
\end{sphinxVerbatim}



\end{sphinxuseclass}
\end{sphinxuseclass}
}

\end{sphinxuseclass}
\end{sphinxuseclass}
\begin{sphinxuseclass}{nbinput}
{
\sphinxsetup{VerbatimColor={named}{nbsphinx-code-bg}}
\sphinxsetup{VerbatimBorderColor={named}{nbsphinx-code-border}}
\begin{sphinxVerbatim}[commandchars=\\\{\}]
\llap{\color{nbsphinxin}[34]:\,\hspace{\fboxrule}\hspace{\fboxsep}}\PYG{n}{fig} \PYG{o}{=} \PYG{n}{rd}\PYG{o}{.}\PYG{n}{plot}\PYG{o}{.}\PYG{n}{nominal\PYGZus{}vals\PYGZus{}2d}\PYG{p}{(}\PYG{n}{nomapp\PYGZus{}rand}\PYG{p}{,} \PYG{n}{endclasses}\PYG{p}{,} \PYG{l+s+s1}{\PYGZsq{}}\PYG{l+s+s1}{amp}\PYG{l+s+s1}{\PYGZsq{}}\PYG{p}{,} \PYG{l+s+s1}{\PYGZsq{}}\PYG{l+s+s1}{wavelength}\PYG{l+s+s1}{\PYGZsq{}}\PYG{p}{)}
\end{sphinxVerbatim}
}

\end{sphinxuseclass}
\begin{sphinxuseclass}{nboutput}
\begin{sphinxuseclass}{nblast}
\hrule height -\fboxrule\relax
\vspace{\nbsphinxcodecellspacing}

\makeatletter\setbox\nbsphinxpromptbox\box\voidb@x\makeatother

\begin{nbsphinxfancyoutput}

\begin{sphinxuseclass}{output_area}
\begin{sphinxuseclass}{}
\noindent\sphinxincludegraphics[width=382\sphinxpxdimen,height=278\sphinxpxdimen]{{docs_Nominal_Approach_Use-Cases_61_0}.png}

\end{sphinxuseclass}
\end{sphinxuseclass}
\end{nbsphinxfancyoutput}

\end{sphinxuseclass}
\end{sphinxuseclass}
\begin{sphinxuseclass}{nbinput}
{
\sphinxsetup{VerbatimColor={named}{nbsphinx-code-bg}}
\sphinxsetup{VerbatimBorderColor={named}{nbsphinx-code-border}}
\begin{sphinxVerbatim}[commandchars=\\\{\}]
\llap{\color{nbsphinxin}[35]:\,\hspace{\fboxrule}\hspace{\fboxsep}}\PYG{n}{fig} \PYG{o}{=} \PYG{n}{rd}\PYG{o}{.}\PYG{n}{plot}\PYG{o}{.}\PYG{n}{nominal\PYGZus{}vals\PYGZus{}2d}\PYG{p}{(}\PYG{n}{nomapp\PYGZus{}rand}\PYG{p}{,} \PYG{n}{endclasses}\PYG{p}{,} \PYG{l+s+s1}{\PYGZsq{}}\PYG{l+s+s1}{radius}\PYG{l+s+s1}{\PYGZsq{}}\PYG{p}{,} \PYG{l+s+s1}{\PYGZsq{}}\PYG{l+s+s1}{start}\PYG{l+s+s1}{\PYGZsq{}}\PYG{p}{)}
\end{sphinxVerbatim}
}

\end{sphinxuseclass}
\begin{sphinxuseclass}{nboutput}
\begin{sphinxuseclass}{nblast}
\hrule height -\fboxrule\relax
\vspace{\nbsphinxcodecellspacing}

\makeatletter\setbox\nbsphinxpromptbox\box\voidb@x\makeatother

\begin{nbsphinxfancyoutput}

\begin{sphinxuseclass}{output_area}
\begin{sphinxuseclass}{}
\noindent\sphinxincludegraphics[width=392\sphinxpxdimen,height=278\sphinxpxdimen]{{docs_Nominal_Approach_Use-Cases_62_0}.png}

\end{sphinxuseclass}
\end{sphinxuseclass}
\end{nbsphinxfancyoutput}

\end{sphinxuseclass}
\end{sphinxuseclass}
\begin{sphinxuseclass}{nbinput}
{
\sphinxsetup{VerbatimColor={named}{nbsphinx-code-bg}}
\sphinxsetup{VerbatimBorderColor={named}{nbsphinx-code-border}}
\begin{sphinxVerbatim}[commandchars=\\\{\}]
\llap{\color{nbsphinxin}[36]:\,\hspace{\fboxrule}\hspace{\fboxsep}}\PYG{n}{fig} \PYG{o}{=} \PYG{n}{rd}\PYG{o}{.}\PYG{n}{plot}\PYG{o}{.}\PYG{n}{nominal\PYGZus{}vals\PYGZus{}1d}\PYG{p}{(}\PYG{n}{nomapp\PYGZus{}rand}\PYG{p}{,} \PYG{n}{endclasses}\PYG{p}{,} \PYG{l+s+s1}{\PYGZsq{}}\PYG{l+s+s1}{radius}\PYG{l+s+s1}{\PYGZsq{}}\PYG{p}{)}
\end{sphinxVerbatim}
}

\end{sphinxuseclass}
\begin{sphinxuseclass}{nboutput}
\begin{sphinxuseclass}{nblast}
\hrule height -\fboxrule\relax
\vspace{\nbsphinxcodecellspacing}

\makeatletter\setbox\nbsphinxpromptbox\box\voidb@x\makeatother

\begin{nbsphinxfancyoutput}

\begin{sphinxuseclass}{output_area}
\begin{sphinxuseclass}{}
\noindent\sphinxincludegraphics[width=353\sphinxpxdimen,height=278\sphinxpxdimen]{{docs_Nominal_Approach_Use-Cases_63_0}.png}

\end{sphinxuseclass}
\end{sphinxuseclass}
\end{nbsphinxfancyoutput}

\end{sphinxuseclass}
\end{sphinxuseclass}
\sphinxAtStartPar
As shown, these results track the uniform approach, although the spread catches some non\sphinxhyphen{}nominal scenarios that were not caught using uniform sampling. The resulting probabilities are:

\begin{sphinxuseclass}{nbinput}
{
\sphinxsetup{VerbatimColor={named}{nbsphinx-code-bg}}
\sphinxsetup{VerbatimBorderColor={named}{nbsphinx-code-border}}
\begin{sphinxVerbatim}[commandchars=\\\{\}]
\llap{\color{nbsphinxin}[37]:\,\hspace{\fboxrule}\hspace{\fboxsep}}\PYG{n}{state\PYGZus{}probabilities} \PYG{o}{=} \PYG{n}{rd}\PYG{o}{.}\PYG{n}{process}\PYG{o}{.}\PYG{n}{state\PYGZus{}probabilities}\PYG{p}{(}\PYG{n}{endclasses}\PYG{p}{)}
\PYG{n}{state\PYGZus{}probabilities}
\end{sphinxVerbatim}
}

\end{sphinxuseclass}
\begin{sphinxuseclass}{nboutput}
\begin{sphinxuseclass}{nblast}
{

\kern-\sphinxverbatimsmallskipamount\kern-\baselineskip
\kern+\FrameHeightAdjust\kern-\fboxrule
\vspace{\nbsphinxcodecellspacing}

\sphinxsetup{VerbatimColor={named}{white}}
\sphinxsetup{VerbatimBorderColor={named}{nbsphinx-code-border}}
\begin{sphinxuseclass}{output_area}
\begin{sphinxuseclass}{}


\begin{sphinxVerbatim}[commandchars=\\\{\}]
\llap{\color{nbsphinxout}[37]:\,\hspace{\fboxrule}\hspace{\fboxsep}}\{'incomplete mission': 0.5500000000000004,
 'nominal mission': 0.4500000000000003\}
\end{sphinxVerbatim}



\end{sphinxuseclass}
\end{sphinxuseclass}
}

\end{sphinxuseclass}
\end{sphinxuseclass}
\sphinxAtStartPar
Which is similar to (but not exactly the same as) the uniform approach. The error of Monte Carlo sampling approaches can be reduced by increasing the number of points, but these points increase computational costs.


\subsubsection{Nested Scenario Sampling}
\label{\detokenize{docs/Nominal_Approach_Use-Cases:Nested-Scenario-Sampling}}
\sphinxAtStartPar
Thus far, we have introduced two types of approaches: \sphinxhyphen{} SampleApproach, which is used to evaluate the system resilience to a set of faults \sphinxhyphen{} NominalApproach, which is used to evaluate system performance over a set of parameters

\sphinxAtStartPar
These both have their limitations when used alone. Simulating a \sphinxcode{\sphinxupquote{SampleApproach}} using \sphinxcode{\sphinxupquote{propagate.approach}} solely evaluates evaluates fault\sphinxhyphen{}driven hazards in a single nominal set of parameters (which may not generalize) while simulating a \sphinxcode{\sphinxupquote{NominalApproach}} using \sphinxcode{\sphinxupquote{propagate.nominal\_approach}} evaluates the systerm performance/resilience to external parameters (But not faults).

\sphinxAtStartPar
To resolve these limitations, one can use a \sphinxstyleemphasis{nested} scenario sampling approach where a \sphinxcode{\sphinxupquote{SampleApproach}} is simulated at each parameter level of a \sphinxcode{\sphinxupquote{NominalApproach}}, giving the resilience of the system to faults over a set of operational parameters. This is called using the \sphinxcode{\sphinxupquote{propagate.nested\_approach}} method.

\begin{sphinxuseclass}{nbinput}
{
\sphinxsetup{VerbatimColor={named}{nbsphinx-code-bg}}
\sphinxsetup{VerbatimBorderColor={named}{nbsphinx-code-border}}
\begin{sphinxVerbatim}[commandchars=\\\{\}]
\llap{\color{nbsphinxin}[38]:\,\hspace{\fboxrule}\hspace{\fboxsep}}\PYG{n}{help}\PYG{p}{(}\PYG{n}{prop}\PYG{o}{.}\PYG{n}{nested\PYGZus{}approach}\PYG{p}{)}
\end{sphinxVerbatim}
}

\end{sphinxuseclass}
\begin{sphinxuseclass}{nboutput}
\begin{sphinxuseclass}{nblast}
{

\kern-\sphinxverbatimsmallskipamount\kern-\baselineskip
\kern+\FrameHeightAdjust\kern-\fboxrule
\vspace{\nbsphinxcodecellspacing}

\sphinxsetup{VerbatimColor={named}{white}}
\sphinxsetup{VerbatimBorderColor={named}{nbsphinx-code-border}}
\begin{sphinxuseclass}{output_area}
\begin{sphinxuseclass}{}


\begin{sphinxVerbatim}[commandchars=\\\{\}]
Help on function nested\_approach in module fmdtools.faultsim.propagate:

nested\_approach(mdl, nomapp, staged=False, track='all', get\_phases=False, showprogress=True, pool=False, track\_times='all', run\_stochastic=False, **app\_args)
    Simulates a set of fault modes within a set of nominal scenarios defined by a nominal approach.

    Parameters
    ----------
    mdl : Model
        Model Object to use in the simulation.
    nomapp : NominalApproach
        NominalApproach defining the nominal situations the model will be run over
    staged : bool, optional
        Whether to inject the fault in a copy of the nominal model at the fault time (True) or instantiate a new model for the fault (False). Setting to True roughly halves execution time. The default is False.
    track : str ('all', 'functions', 'flows', 'valparams', dict, 'none'), optional
        Which model states to track over time, which can be given as 'functions', 'flows',
        'all', 'none', 'valparams' (model states specified in mdl.valparams),
        or a dict of form \{'functions':\{'fxn1':'att1'\}, 'flows':\{'flow1':'att1'\}\}
        The default is 'all'.
    get\_phases : Bool/List/Dict, optional
        Whether and how to use nominal simulation phases to set up the SampleApproach. The default is False.
        - If True, all phases from the nominal simulation are passed to SampleApproach()
        - If a list ['Fxn1', 'Fxn2' etc.], only the phases from the listed functions will be passed.
        - If a dict \{'Fxn1':'phase1'\}, only the phase 'phase1' in the function 'Fxn1' will be passed.
    pool : process pool, optional
        Process Pool Object from multiprocessing or pathos packages. Pathos is recommended.
        e.g. parallelpool = mp.pool(n) for n cores (multiprocessing)
        or parallelpool = ProcessPool(nodes=n) for n cores (pathos)
        If False, the set of scenarios is run serially. The default is False
    showprogress: bool, optional
        whether to show a progress bar during execution. default is true
    track\_times : str/tuple
        Defines what times to include in the history. Options are:
            'all'--all simulated times
            ('interval', n)--includes every nth time in the history
            ('times', [t1, {\ldots} tn])--only includes times defined in the vector [t1 {\ldots} tn]
    run\_stochastic : bool
        Whether to run stochastic behaviors or use default values for stochastic variables. Default is False.
    **app\_args : kwargs
        Keyword arguments for the SampleApproach. See modeldef.SampleApproach documentation.

    Returns
    -------
    nested\_endclasses : dict
        A nested dictionary with the rate, cost, and expected cost of each scenario run with structure \{'nomscen1':endclasses, 'nomscen2':mdlhists\}
    nested\_mdlhists : dict
        A nested dictionary with the history of all model states for each scenario with structure \{'nomscen1':mdlhists, 'nomscen2':mdlhists\}

\end{sphinxVerbatim}



\end{sphinxuseclass}
\end{sphinxuseclass}
}

\end{sphinxuseclass}
\end{sphinxuseclass}
\sphinxAtStartPar
Here we use the nominal approach generated earlier with a default sampling approach to quantify resilience.

\begin{sphinxuseclass}{nbinput}
{
\sphinxsetup{VerbatimColor={named}{nbsphinx-code-bg}}
\sphinxsetup{VerbatimBorderColor={named}{nbsphinx-code-border}}
\begin{sphinxVerbatim}[commandchars=\\\{\}]
\llap{\color{nbsphinxin}[39]:\,\hspace{\fboxrule}\hspace{\fboxsep}}\PYG{n}{nested\PYGZus{}endclasses}\PYG{p}{,} \PYG{n}{nested\PYGZus{}mdlhists} \PYG{o}{=} \PYG{n}{prop}\PYG{o}{.}\PYG{n}{nested\PYGZus{}approach}\PYG{p}{(}\PYG{n}{mdl}\PYG{p}{,} \PYG{n}{nomapp}\PYG{p}{)}
\end{sphinxVerbatim}
}

\end{sphinxuseclass}
\begin{sphinxuseclass}{nboutput}
\begin{sphinxuseclass}{nblast}
{

\kern-\sphinxverbatimsmallskipamount\kern-\baselineskip
\kern+\FrameHeightAdjust\kern-\fboxrule
\vspace{\nbsphinxcodecellspacing}

\sphinxsetup{VerbatimColor={named}{nbsphinx-stderr}}
\sphinxsetup{VerbatimBorderColor={named}{nbsphinx-code-border}}
\begin{sphinxuseclass}{output_area}
\begin{sphinxuseclass}{stderr}


\begin{sphinxVerbatim}[commandchars=\\\{\}]
NESTED SCENARIOS COMPLETE: 100\%|█████████████████████████████████████████████████████| 228/228 [00:21<00:00, 10.77it/s]
\end{sphinxVerbatim}



\end{sphinxuseclass}
\end{sphinxuseclass}
}

\end{sphinxuseclass}
\end{sphinxuseclass}
\sphinxAtStartPar
The resulting endclass/mdulhist dictionary is in turn nested within operational scenarios.

\begin{sphinxuseclass}{nbinput}
{
\sphinxsetup{VerbatimColor={named}{nbsphinx-code-bg}}
\sphinxsetup{VerbatimBorderColor={named}{nbsphinx-code-border}}
\begin{sphinxVerbatim}[commandchars=\\\{\}]
\llap{\color{nbsphinxin}[40]:\,\hspace{\fboxrule}\hspace{\fboxsep}}\PYG{n}{nested\PYGZus{}endclasses}\PYG{o}{.}\PYG{n}{keys}\PYG{p}{(}\PYG{p}{)}
\end{sphinxVerbatim}
}

\end{sphinxuseclass}
\begin{sphinxuseclass}{nboutput}
\begin{sphinxuseclass}{nblast}
{

\kern-\sphinxverbatimsmallskipamount\kern-\baselineskip
\kern+\FrameHeightAdjust\kern-\fboxrule
\vspace{\nbsphinxcodecellspacing}

\sphinxsetup{VerbatimColor={named}{white}}
\sphinxsetup{VerbatimBorderColor={named}{nbsphinx-code-border}}
\begin{sphinxuseclass}{output_area}
\begin{sphinxuseclass}{}


\begin{sphinxVerbatim}[commandchars=\\\{\}]
\llap{\color{nbsphinxout}[40]:\,\hspace{\fboxrule}\hspace{\fboxsep}}dict\_keys(['sine\_1', 'sine\_2', 'sine\_3', 'sine\_4', 'sine\_5', 'sine\_6', 'sine\_7', 'sine\_8', 'sine\_9', 'sine\_10', 'sine\_11', 'sine\_12', 'sine\_13', 'sine\_14', 'sine\_15', 'sine\_16', 'sine\_17', 'sine\_18', 'sine\_19', 'sine\_20', 'sine\_21', 'sine\_22', 'sine\_23', 'sine\_24', 'sine\_25', 'sine\_26', 'sine\_27', 'sine\_28', 'sine\_29', 'sine\_30', 'sine\_31', 'sine\_32', 'sine\_33', 'sine\_34', 'sine\_35', 'sine\_36', 'sine\_37', 'sine\_38', 'sine\_39', 'sine\_40', 'sine\_41', 'sine\_42', 'sine\_43', 'sine\_44', 'sine\_45', 'sine\_46', 'sine\_47', 'sine\_48', 'sine\_49', 'sine\_50', 'sine\_51', 'sine\_52', 'sine\_53', 'sine\_54', 'sine\_55', 'sine\_56', 'sine\_57', 'sine\_58', 'sine\_59', 'sine\_60', 'sine\_61', 'sine\_62', 'sine\_63', 'sine\_64', 'sine\_65', 'sine\_66', 'sine\_67', 'sine\_68', 'sine\_69', 'sine\_70', 'sine\_71', 'sine\_72', 'sine\_73', 'sine\_74', 'sine\_75', 'sine\_76', 'sine\_77', 'sine\_78', 'sine\_79', 'sine\_80', 'sine\_81', 'sine\_82', 'sine\_83', 'sine\_84', 'sine\_85', 'sine\_86', 'sine\_87', 'sine\_88', 'sine\_89', 'sine\_90', 'sine\_91', 'sine\_92', 'sine\_93', 'sine\_94', 'sine\_95', 'sine\_96', 'sine\_97', 'sine\_98', 'sine\_99', 'sine\_100', 'sine\_101', 'sine\_102', 'sine\_103', 'sine\_104', 'sine\_105', 'sine\_106', 'sine\_107', 'sine\_108', 'sine\_109', 'sine\_110', 'sine\_111', 'sine\_112', 'sine\_113', 'sine\_114', 'sine\_115', 'sine\_116', 'sine\_117', 'sine\_118', 'sine\_119', 'sine\_120', 'sine\_121', 'sine\_122', 'sine\_123', 'sine\_124', 'sine\_125', 'sine\_126', 'sine\_127', 'sine\_128', 'sine\_129', 'sine\_130', 'sine\_131', 'sine\_132', 'sine\_133', 'sine\_134', 'sine\_135', 'sine\_136', 'sine\_137', 'sine\_138', 'sine\_139', 'sine\_140', 'sine\_141', 'sine\_142', 'sine\_143', 'sine\_144', 'sine\_145', 'sine\_146', 'sine\_147', 'sine\_148', 'sine\_149', 'sine\_150', 'sine\_151', 'sine\_152', 'sine\_153', 'sine\_154', 'sine\_155', 'sine\_156', 'sine\_157', 'sine\_158', 'sine\_159', 'sine\_160', 'sine\_161', 'sine\_162', 'sine\_163', 'sine\_164', 'sine\_165', 'sine\_166', 'sine\_167', 'sine\_168', 'sine\_169', 'sine\_170', 'sine\_171', 'sine\_172', 'sine\_173', 'sine\_174', 'sine\_175', 'sine\_176', 'sine\_177', 'sine\_178', 'sine\_179', 'sine\_180', 'sine\_181', 'sine\_182', 'sine\_183', 'sine\_184', 'sine\_185', 'sine\_186', 'sine\_187', 'sine\_188', 'sine\_189', 'sine\_190', 'sine\_191', 'sine\_192', 'sine\_193', 'sine\_194', 'sine\_195', 'sine\_196', 'sine\_197', 'sine\_198', 'sine\_199', 'sine\_200', 'turn\_201', 'turn\_202', 'turn\_203', 'turn\_204', 'turn\_205', 'turn\_206', 'turn\_207', 'turn\_208', 'turn\_209', 'turn\_210', 'turn\_211', 'turn\_212', 'turn\_213', 'turn\_214', 'turn\_215', 'turn\_216', 'turn\_217', 'turn\_218', 'turn\_219', 'turn\_220', 'turn\_221', 'turn\_222', 'turn\_223', 'turn\_224', 'turn\_225', 'turn\_226', 'turn\_227', 'turn\_228'])
\end{sphinxVerbatim}



\end{sphinxuseclass}
\end{sphinxuseclass}
}

\end{sphinxuseclass}
\end{sphinxuseclass}
\begin{sphinxuseclass}{nbinput}
{
\sphinxsetup{VerbatimColor={named}{nbsphinx-code-bg}}
\sphinxsetup{VerbatimBorderColor={named}{nbsphinx-code-border}}
\begin{sphinxVerbatim}[commandchars=\\\{\}]
\llap{\color{nbsphinxin}[41]:\,\hspace{\fboxrule}\hspace{\fboxsep}}\PYG{n}{nested\PYGZus{}endclasses}\PYG{p}{[}\PYG{l+s+s1}{\PYGZsq{}}\PYG{l+s+s1}{sine\PYGZus{}1}\PYG{l+s+s1}{\PYGZsq{}}\PYG{p}{]}\PYG{p}{[}\PYG{l+s+s1}{\PYGZsq{}}\PYG{l+s+s1}{nominal}\PYG{l+s+s1}{\PYGZsq{}}\PYG{p}{]}
\end{sphinxVerbatim}
}

\end{sphinxuseclass}
\begin{sphinxuseclass}{nboutput}
\begin{sphinxuseclass}{nblast}
{

\kern-\sphinxverbatimsmallskipamount\kern-\baselineskip
\kern+\FrameHeightAdjust\kern-\fboxrule
\vspace{\nbsphinxcodecellspacing}

\sphinxsetup{VerbatimColor={named}{white}}
\sphinxsetup{VerbatimBorderColor={named}{nbsphinx-code-border}}
\begin{sphinxuseclass}{output_area}
\begin{sphinxuseclass}{}


\begin{sphinxVerbatim}[commandchars=\\\{\}]
\llap{\color{nbsphinxout}[41]:\,\hspace{\fboxrule}\hspace{\fboxsep}}\{'rate': 0,
 'cost': 0,
 'prob': 1,
 'expected cost': 0,
 'faults': \{\},
 'classification': 'nominal mission'\}
\end{sphinxVerbatim}



\end{sphinxuseclass}
\end{sphinxuseclass}
}

\end{sphinxuseclass}
\end{sphinxuseclass}
\sphinxAtStartPar
We can compare resilience to faults over the operational envelope using \sphinxcode{\sphinxupquote{rd.tabulate.resilience\_factor\_comparison}}, which generates at table similar to \sphinxcode{\sphinxupquote{rd.tabulate.nominal\_factor\_comparison}} except for a single metric with rows for each fault scenario.

\begin{sphinxuseclass}{nbinput}
\begin{sphinxuseclass}{nblast}
{
\sphinxsetup{VerbatimColor={named}{nbsphinx-code-bg}}
\sphinxsetup{VerbatimBorderColor={named}{nbsphinx-code-border}}
\begin{sphinxVerbatim}[commandchars=\\\{\}]
\llap{\color{nbsphinxin}[42]:\,\hspace{\fboxrule}\hspace{\fboxsep}}\PYG{n}{restab} \PYG{o}{=} \PYG{n}{rd}\PYG{o}{.}\PYG{n}{tabulate}\PYG{o}{.}\PYG{n}{resilience\PYGZus{}factor\PYGZus{}comparison}\PYG{p}{(}\PYG{n}{nomapp}\PYG{p}{,} \PYG{n}{nested\PYGZus{}endclasses}\PYG{p}{,}\PYG{p}{[}\PYG{l+s+s1}{\PYGZsq{}}\PYG{l+s+s1}{start}\PYG{l+s+s1}{\PYGZsq{}}\PYG{p}{]}\PYG{p}{,} \PYG{l+s+s1}{\PYGZsq{}}\PYG{l+s+s1}{cost}\PYG{l+s+s1}{\PYGZsq{}}\PYG{p}{,} \PYG{n}{rangeid}\PYG{o}{=}\PYG{l+s+s1}{\PYGZsq{}}\PYG{l+s+s1}{turn}\PYG{l+s+s1}{\PYGZsq{}}\PYG{p}{,} \PYG{n}{percent}\PYG{o}{=}\PYG{k+kc}{False}\PYG{p}{,} \PYG{n}{difference}\PYG{o}{=}\PYG{k+kc}{False}\PYG{p}{)}
\end{sphinxVerbatim}
}

\end{sphinxuseclass}
\end{sphinxuseclass}
\begin{sphinxuseclass}{nbinput}
{
\sphinxsetup{VerbatimColor={named}{nbsphinx-code-bg}}
\sphinxsetup{VerbatimBorderColor={named}{nbsphinx-code-border}}
\begin{sphinxVerbatim}[commandchars=\\\{\}]
\llap{\color{nbsphinxin}[43]:\,\hspace{\fboxrule}\hspace{\fboxsep}}\PYG{n}{restab}
\end{sphinxVerbatim}
}

\end{sphinxuseclass}
\begin{sphinxuseclass}{nboutput}
\begin{sphinxuseclass}{nblast}
{

\kern-\sphinxverbatimsmallskipamount\kern-\baselineskip
\kern+\FrameHeightAdjust\kern-\fboxrule
\vspace{\nbsphinxcodecellspacing}

\sphinxsetup{VerbatimColor={named}{white}}
\sphinxsetup{VerbatimBorderColor={named}{nbsphinx-code-border}}
\begin{sphinxuseclass}{output_area}
\begin{sphinxuseclass}{}


\begin{sphinxVerbatim}[commandchars=\\\{\}]
\llap{\color{nbsphinxout}[43]:\,\hspace{\fboxrule}\hspace{\fboxsep}}('start',)     nominal       Drive    Avionics
0            71.428571  260.714286  171.428571
5           142.857143  332.142857  242.857143
10          142.857143  332.142857  242.857143
15          142.857143  350.000000  242.857143
\end{sphinxVerbatim}



\end{sphinxuseclass}
\end{sphinxuseclass}
}

\end{sphinxuseclass}
\end{sphinxuseclass}
\sphinxAtStartPar
These factors can further be visualized using \sphinxcode{\sphinxupquote{rd.plot.resilience\_factor\_comparison}}, which generates a set of bar graphs similar to \sphinxcode{\sphinxupquote{rd.plot.nominal\_factor\_comparison}}.

\begin{sphinxuseclass}{nbinput}
{
\sphinxsetup{VerbatimColor={named}{nbsphinx-code-bg}}
\sphinxsetup{VerbatimBorderColor={named}{nbsphinx-code-border}}
\begin{sphinxVerbatim}[commandchars=\\\{\}]
\llap{\color{nbsphinxin}[44]:\,\hspace{\fboxrule}\hspace{\fboxsep}}\PYG{n}{help}\PYG{p}{(}\PYG{n}{rd}\PYG{o}{.}\PYG{n}{plot}\PYG{o}{.}\PYG{n}{resilience\PYGZus{}factor\PYGZus{}comparison}\PYG{p}{)}
\end{sphinxVerbatim}
}

\end{sphinxuseclass}
\begin{sphinxuseclass}{nboutput}
\begin{sphinxuseclass}{nblast}
{

\kern-\sphinxverbatimsmallskipamount\kern-\baselineskip
\kern+\FrameHeightAdjust\kern-\fboxrule
\vspace{\nbsphinxcodecellspacing}

\sphinxsetup{VerbatimColor={named}{white}}
\sphinxsetup{VerbatimBorderColor={named}{nbsphinx-code-border}}
\begin{sphinxuseclass}{output_area}
\begin{sphinxuseclass}{}


\begin{sphinxVerbatim}[commandchars=\\\{\}]
Help on function resilience\_factor\_comparison in module fmdtools.resultdisp.plot:

resilience\_factor\_comparison(comparison\_table, faults='all', rows=1, stat='proportion', figsize=(12, 8), title='', maxy='max', legend='single', stack=False, xlabel=True, error\_bars=False)
    Plots a comparison\_table from tabulate.resilience\_factor\_comparison as a bar plot for each fault scenario/set of fault scenarios.

    Parameters
    ----------
    comparison\_table : pandas table
        Table from tabulate.resilience\_factor\_test with factors as rows and fault scenarios as columns
    faults : list, optional
        iterable of faults/fault types to include in the bar plot (the columns of the table). The default is 'all'.
        a dictionary \{'fault':'title'\} will associate the given fault with a title (otherwise 'fault' is used)
    rows : int, optional
        Number of rows in the multplot. The default is 1.
    stat : str, optional
        Metric being presented in the table (for the y-axis). The default is 'proportion'.
    figsize : tuple(int, int), optional
        Size of the figure in (width, height). The default is (12,8).
    title : string, optional
        Overall title for the plots. The default is ''.
    maxy : float, optional
        Maximum y-value (to ensure same scale). The default is 'max' (finds max value of table).
    legend : str, optional
        'all'/'single'/'none'. The default is "single".
    stack : bool, optional
        Whether or not to stack the nominal and resilience plots. The default is False.
    xlabel : bool/str
        The x-label descriptor for the design factors. Defaults to the column values.
    error\_bars : bool
        Whether to include error bars for the factor. Requires comparison\_table to have lower and upper bound information

    Returns
    -------
    figure: matplotlib figure
        Plot handle of the figure.

\end{sphinxVerbatim}



\end{sphinxuseclass}
\end{sphinxuseclass}
}

\end{sphinxuseclass}
\end{sphinxuseclass}
\begin{sphinxuseclass}{nbinput}
{
\sphinxsetup{VerbatimColor={named}{nbsphinx-code-bg}}
\sphinxsetup{VerbatimBorderColor={named}{nbsphinx-code-border}}
\begin{sphinxVerbatim}[commandchars=\\\{\}]
\llap{\color{nbsphinxin}[45]:\,\hspace{\fboxrule}\hspace{\fboxsep}}\PYG{n}{fig} \PYG{o}{=} \PYG{n}{rd}\PYG{o}{.}\PYG{n}{plot}\PYG{o}{.}\PYG{n}{resilience\PYGZus{}factor\PYGZus{}comparison}\PYG{p}{(}\PYG{n}{restab}\PYG{p}{,} \PYG{n}{stat}\PYG{o}{=}\PYG{l+s+s1}{\PYGZsq{}}\PYG{l+s+s1}{cost}\PYG{l+s+s1}{\PYGZsq{}}\PYG{p}{,} \PYG{n}{figsize} \PYG{o}{=} \PYG{p}{(}\PYG{l+m+mi}{6}\PYG{p}{,}\PYG{l+m+mi}{4}\PYG{p}{)}\PYG{p}{,} \PYG{n}{xlabel}\PYG{o}{=}\PYG{l+s+s1}{\PYGZsq{}}\PYG{l+s+s1}{curve start point}\PYG{l+s+s1}{\PYGZsq{}}\PYG{p}{)}
\end{sphinxVerbatim}
}

\end{sphinxuseclass}
\begin{sphinxuseclass}{nboutput}
\begin{sphinxuseclass}{nblast}
\hrule height -\fboxrule\relax
\vspace{\nbsphinxcodecellspacing}

\makeatletter\setbox\nbsphinxpromptbox\box\voidb@x\makeatother

\begin{nbsphinxfancyoutput}

\begin{sphinxuseclass}{output_area}
\begin{sphinxuseclass}{}
\noindent\sphinxincludegraphics[width=440\sphinxpxdimen,height=296\sphinxpxdimen]{{docs_Nominal_Approach_Use-Cases_79_0}.png}

\end{sphinxuseclass}
\end{sphinxuseclass}
\end{nbsphinxfancyoutput}

\end{sphinxuseclass}
\end{sphinxuseclass}
\sphinxAtStartPar
As shown, on average the costs are higher over fault scenarios than in their nominal counterparts. While this difference appears to be uniform for Avionics faults (\sphinxcode{\sphinxupquote{no\_con}}, which merely adds a fault at the times instantiated (since the rover is already in \sphinxcode{\sphinxupquote{drive}} mode)), resulting in a uniform cost of 100), it changes for the Drive fault depending on the location of the curve. This is because in some cases this fault stops the rover at the finish line (when the line is short), and sometimes
during its mission (when the line is long), leading to a cost from the incomplete mission.

\sphinxAtStartPar
Thus, the assessed consequences of faults are somewhat prone to parameters leading to a different injection time. We might prefer, for example, for avionics faults to be injected when they would make a difference (i.e. at the start) and for drive faults to be injected multiple. This can be fixed by: \sphinxhyphen{} encoding phase information in with modes to ensure they are injected at the right intervals \sphinxhyphen{} using the \sphinxcode{\sphinxupquote{get\_phases}} option in \sphinxcode{\sphinxupquote{propagate.nested\_approach}} to get the phases and inject the
faults in the corresponding times \sphinxhyphen{} passing appropriate Approach arguments to \sphinxcode{\sphinxupquote{propagate.nested\_approach}} (e.g., defaultsamp, etc)

\sphinxAtStartPar
This has already been done in the Power faults, which we evaluate below. (note that they were not included above because they were listed to be injected in specific phases using the \sphinxcode{\sphinxupquote{key\_phases\_by}} option in \sphinxcode{\sphinxupquote{assoc\_modes()}}).

\begin{sphinxuseclass}{nbinput}
{
\sphinxsetup{VerbatimColor={named}{nbsphinx-code-bg}}
\sphinxsetup{VerbatimBorderColor={named}{nbsphinx-code-border}}
\begin{sphinxVerbatim}[commandchars=\\\{\}]
\llap{\color{nbsphinxin}[46]:\,\hspace{\fboxrule}\hspace{\fboxsep}}\PYG{n}{nested\PYGZus{}endclasses\PYGZus{}power}\PYG{p}{,} \PYG{n}{nested\PYGZus{}mdlhists\PYGZus{}power} \PYG{o}{=} \PYG{n}{prop}\PYG{o}{.}\PYG{n}{nested\PYGZus{}approach}\PYG{p}{(}\PYG{n}{mdl}\PYG{p}{,} \PYG{n}{nomapp}\PYG{p}{,} \PYG{n}{get\PYGZus{}phases}\PYG{o}{=}\PYG{k+kc}{True}\PYG{p}{,} \PYG{n}{faults}\PYG{o}{=}\PYG{l+s+s1}{\PYGZsq{}}\PYG{l+s+s1}{Power}\PYG{l+s+s1}{\PYGZsq{}}\PYG{p}{)}
\end{sphinxVerbatim}
}

\end{sphinxuseclass}
\begin{sphinxuseclass}{nboutput}
\begin{sphinxuseclass}{nblast}
{

\kern-\sphinxverbatimsmallskipamount\kern-\baselineskip
\kern+\FrameHeightAdjust\kern-\fboxrule
\vspace{\nbsphinxcodecellspacing}

\sphinxsetup{VerbatimColor={named}{nbsphinx-stderr}}
\sphinxsetup{VerbatimBorderColor={named}{nbsphinx-code-border}}
\begin{sphinxuseclass}{output_area}
\begin{sphinxuseclass}{stderr}


\begin{sphinxVerbatim}[commandchars=\\\{\}]
NESTED SCENARIOS COMPLETE: 100\%|█████████████████████████████████████████████████████| 228/228 [00:14<00:00, 15.80it/s]
\end{sphinxVerbatim}



\end{sphinxuseclass}
\end{sphinxuseclass}
}

\end{sphinxuseclass}
\end{sphinxuseclass}
\begin{sphinxuseclass}{nbinput}
{
\sphinxsetup{VerbatimColor={named}{nbsphinx-code-bg}}
\sphinxsetup{VerbatimBorderColor={named}{nbsphinx-code-border}}
\begin{sphinxVerbatim}[commandchars=\\\{\}]
\llap{\color{nbsphinxin}[47]:\,\hspace{\fboxrule}\hspace{\fboxsep}}\PYG{n}{restab\PYGZus{}power} \PYG{o}{=} \PYG{n}{rd}\PYG{o}{.}\PYG{n}{tabulate}\PYG{o}{.}\PYG{n}{resilience\PYGZus{}factor\PYGZus{}comparison}\PYG{p}{(}\PYG{n}{nomapp}\PYG{p}{,} \PYG{n}{nested\PYGZus{}endclasses\PYGZus{}power}\PYG{p}{,}\PYG{p}{[}\PYG{l+s+s1}{\PYGZsq{}}\PYG{l+s+s1}{start}\PYG{l+s+s1}{\PYGZsq{}}\PYG{p}{]}\PYG{p}{,} \PYG{l+s+s1}{\PYGZsq{}}\PYG{l+s+s1}{cost}\PYG{l+s+s1}{\PYGZsq{}}\PYG{p}{,} \PYG{n}{rangeid}\PYG{o}{=}\PYG{l+s+s1}{\PYGZsq{}}\PYG{l+s+s1}{turn}\PYG{l+s+s1}{\PYGZsq{}}\PYG{p}{,} \PYG{n}{difference}\PYG{o}{=}\PYG{k+kc}{True}\PYG{p}{,} \PYG{n}{percent}\PYG{o}{=}\PYG{k+kc}{False}\PYG{p}{,} \PYG{n}{faults}\PYG{o}{=}\PYG{l+s+s1}{\PYGZsq{}}\PYG{l+s+s1}{modes}\PYG{l+s+s1}{\PYGZsq{}}\PYG{p}{)}
\PYG{n}{restab\PYGZus{}power}
\end{sphinxVerbatim}
}

\end{sphinxuseclass}
\begin{sphinxuseclass}{nboutput}
\begin{sphinxuseclass}{nblast}
{

\kern-\sphinxverbatimsmallskipamount\kern-\baselineskip
\kern+\FrameHeightAdjust\kern-\fboxrule
\vspace{\nbsphinxcodecellspacing}

\sphinxsetup{VerbatimColor={named}{white}}
\sphinxsetup{VerbatimBorderColor={named}{nbsphinx-code-border}}
\begin{sphinxuseclass}{output_area}
\begin{sphinxuseclass}{}


\begin{sphinxVerbatim}[commandchars=\\\{\}]
\llap{\color{nbsphinxout}[47]:\,\hspace{\fboxrule}\hspace{\fboxsep}}('start',)     nominal  Power no\_charge  Power open\_circ
0            71.428571       528.571429       528.571429
5           142.857143       457.142857       457.142857
10          142.857143       457.142857       457.142857
15          142.857143       457.142857       457.142857
\end{sphinxVerbatim}



\end{sphinxuseclass}
\end{sphinxuseclass}
}

\end{sphinxuseclass}
\end{sphinxuseclass}
\begin{sphinxuseclass}{nbinput}
{
\sphinxsetup{VerbatimColor={named}{nbsphinx-code-bg}}
\sphinxsetup{VerbatimBorderColor={named}{nbsphinx-code-border}}
\begin{sphinxVerbatim}[commandchars=\\\{\}]
\llap{\color{nbsphinxin}[48]:\,\hspace{\fboxrule}\hspace{\fboxsep}}\PYG{n}{fig} \PYG{o}{=} \PYG{n}{rd}\PYG{o}{.}\PYG{n}{plot}\PYG{o}{.}\PYG{n}{resilience\PYGZus{}factor\PYGZus{}comparison}\PYG{p}{(}\PYG{n}{restab\PYGZus{}power}\PYG{p}{,} \PYG{n}{stat}\PYG{o}{=}\PYG{l+s+s1}{\PYGZsq{}}\PYG{l+s+s1}{cost}\PYG{l+s+s1}{\PYGZsq{}}\PYG{p}{,} \PYG{n}{figsize} \PYG{o}{=} \PYG{p}{(}\PYG{l+m+mi}{6}\PYG{p}{,}\PYG{l+m+mi}{4}\PYG{p}{)}\PYG{p}{,} \PYG{n}{xlabel}\PYG{o}{=}\PYG{l+s+s1}{\PYGZsq{}}\PYG{l+s+s1}{curve start point}\PYG{l+s+s1}{\PYGZsq{}}\PYG{p}{,} \PYG{n}{stack}\PYG{o}{=}\PYG{k+kc}{True}\PYG{p}{,} \PYG{n}{maxy}\PYG{o}{=}\PYG{l+m+mi}{800}\PYG{p}{)}
\end{sphinxVerbatim}
}

\end{sphinxuseclass}
\begin{sphinxuseclass}{nboutput}
\begin{sphinxuseclass}{nblast}
\hrule height -\fboxrule\relax
\vspace{\nbsphinxcodecellspacing}

\makeatletter\setbox\nbsphinxpromptbox\box\voidb@x\makeatother

\begin{nbsphinxfancyoutput}

\begin{sphinxuseclass}{output_area}
\begin{sphinxuseclass}{}
\noindent\sphinxincludegraphics[width=440\sphinxpxdimen,height=296\sphinxpxdimen]{{docs_Nominal_Approach_Use-Cases_83_0}.png}

\end{sphinxuseclass}
\end{sphinxuseclass}
\end{nbsphinxfancyoutput}

\end{sphinxuseclass}
\end{sphinxuseclass}
\sphinxAtStartPar
As shown, all power faults result in the same cost, since they all result in both a fault and an incomplete mission, in part because none of them are injected after the rover completes its mission (which would happen if we were using the global phases as in the drive faults).

\begin{sphinxuseclass}{nbinput}
\begin{sphinxuseclass}{nblast}
{
\sphinxsetup{VerbatimColor={named}{nbsphinx-code-bg}}
\sphinxsetup{VerbatimBorderColor={named}{nbsphinx-code-border}}
\begin{sphinxVerbatim}[commandchars=\\\{\}]
\llap{\color{nbsphinxin}[ ]:\,\hspace{\fboxrule}\hspace{\fboxsep}}
\end{sphinxVerbatim}
}

\end{sphinxuseclass}
\end{sphinxuseclass}

\subsection{Stochastic Modelling in fmdtools}
\label{\detokenize{example_pump/Stochastic_Modelling:Stochastic-Modelling-in-fmdtools}}\label{\detokenize{example_pump/Stochastic_Modelling::doc}}
\sphinxAtStartPar
This notebook covers the basics of stochastic modelling in fmdtools. Stochastic models are models which run \sphinxstyleemphasis{non\sphinxhyphen{}determinsitically}, meaning that simulating the model at the same inputs results in different outputs because of randomness or uncertainty in the model behavior. Thus, to determine the \sphinxstyleemphasis{distribution of outcomes} which may result from a simulation, a stochastic model must be run a number of times. This notebook will cover: \sphinxhyphen{} How to construct a stochastic model by adding \sphinxstyleemphasis{stochastic
states} to function blocks and incorporating them in function behavior \sphinxhyphen{} How to simulate single scenarios and distributions of scenarios in \sphinxcode{\sphinxupquote{propagate}} methods \sphinxhyphen{} How to visualize and analyze the results of stochastic model simulations

\begin{sphinxuseclass}{nbinput}
\begin{sphinxuseclass}{nblast}
{
\sphinxsetup{VerbatimColor={named}{nbsphinx-code-bg}}
\sphinxsetup{VerbatimBorderColor={named}{nbsphinx-code-border}}
\begin{sphinxVerbatim}[commandchars=\\\{\}]
\llap{\color{nbsphinxin}[1]:\,\hspace{\fboxrule}\hspace{\fboxsep}}\PYG{k+kn}{import} \PYG{n+nn}{sys}\PYG{o}{,} \PYG{n+nn}{os}
\PYG{n}{sys}\PYG{o}{.}\PYG{n}{path}\PYG{o}{.}\PYG{n}{insert}\PYG{p}{(}\PYG{l+m+mi}{1}\PYG{p}{,}\PYG{n}{os}\PYG{o}{.}\PYG{n}{path}\PYG{o}{.}\PYG{n}{join}\PYG{p}{(}\PYG{l+s+s2}{\PYGZdq{}}\PYG{l+s+s2}{..}\PYG{l+s+s2}{\PYGZdq{}}\PYG{p}{)}\PYG{p}{)}

\PYG{k+kn}{from} \PYG{n+nn}{fmdtools}\PYG{n+nn}{.}\PYG{n+nn}{modeldef} \PYG{k+kn}{import} \PYG{n}{SampleApproach}\PYG{p}{,} \PYG{n}{NominalApproach}
\PYG{k+kn}{import} \PYG{n+nn}{fmdtools}\PYG{n+nn}{.}\PYG{n+nn}{faultsim}\PYG{n+nn}{.}\PYG{n+nn}{propagate} \PYG{k}{as} \PYG{n+nn}{propagate}
\PYG{k+kn}{import} \PYG{n+nn}{fmdtools}\PYG{n+nn}{.}\PYG{n+nn}{resultdisp} \PYG{k}{as} \PYG{n+nn}{rd}

\PYG{k+kn}{import} \PYG{n+nn}{inspect} \PYG{c+c1}{\PYGZsh{} we will be using inspect to look at}
\end{sphinxVerbatim}
}

\end{sphinxuseclass}
\end{sphinxuseclass}
\sphinxAtStartPar
This notebook uses Pump model in \sphinxcode{\sphinxupquote{pump\_stochastic.py}}, which is an adaptation of the original \sphinxcode{\sphinxupquote{ex\_pump.py}} model with stochastic states added.

\begin{sphinxuseclass}{nbinput}
\begin{sphinxuseclass}{nblast}
{
\sphinxsetup{VerbatimColor={named}{nbsphinx-code-bg}}
\sphinxsetup{VerbatimBorderColor={named}{nbsphinx-code-border}}
\begin{sphinxVerbatim}[commandchars=\\\{\}]
\llap{\color{nbsphinxin}[2]:\,\hspace{\fboxrule}\hspace{\fboxsep}}\PYG{k+kn}{from} \PYG{n+nn}{pump\PYGZus{}stochastic} \PYG{k+kn}{import} \PYG{o}{*}
\end{sphinxVerbatim}
}

\end{sphinxuseclass}
\end{sphinxuseclass}
\sphinxAtStartPar
Below is the structure, with the same functions/flows as before.

\begin{sphinxuseclass}{nbinput}
{
\sphinxsetup{VerbatimColor={named}{nbsphinx-code-bg}}
\sphinxsetup{VerbatimBorderColor={named}{nbsphinx-code-border}}
\begin{sphinxVerbatim}[commandchars=\\\{\}]
\llap{\color{nbsphinxin}[3]:\,\hspace{\fboxrule}\hspace{\fboxsep}}\PYG{n}{rd}\PYG{o}{.}\PYG{n}{graph}\PYG{o}{.}\PYG{n}{show}\PYG{p}{(}\PYG{n}{Pump}\PYG{p}{(}\PYG{p}{)}\PYG{p}{)}
\end{sphinxVerbatim}
}

\end{sphinxuseclass}
\begin{sphinxuseclass}{nboutput}
{

\kern-\sphinxverbatimsmallskipamount\kern-\baselineskip
\kern+\FrameHeightAdjust\kern-\fboxrule
\vspace{\nbsphinxcodecellspacing}

\sphinxsetup{VerbatimColor={named}{white}}
\sphinxsetup{VerbatimBorderColor={named}{nbsphinx-code-border}}
\begin{sphinxuseclass}{output_area}
\begin{sphinxuseclass}{}


\begin{sphinxVerbatim}[commandchars=\\\{\}]
\llap{\color{nbsphinxout}[3]:\,\hspace{\fboxrule}\hspace{\fboxsep}}(<Figure size 432x288 with 1 Axes>, <AxesSubplot:>)
\end{sphinxVerbatim}



\end{sphinxuseclass}
\end{sphinxuseclass}
}

\end{sphinxuseclass}
\begin{sphinxuseclass}{nboutput}
\begin{sphinxuseclass}{nblast}
\hrule height -\fboxrule\relax
\vspace{\nbsphinxcodecellspacing}

\makeatletter\setbox\nbsphinxpromptbox\box\voidb@x\makeatother

\begin{nbsphinxfancyoutput}

\begin{sphinxuseclass}{output_area}
\begin{sphinxuseclass}{}
\noindent\sphinxincludegraphics[width=349\sphinxpxdimen,height=231\sphinxpxdimen]{{example_pump_Stochastic_Modelling_5_1}.png}

\end{sphinxuseclass}
\end{sphinxuseclass}
\end{nbsphinxfancyoutput}

\end{sphinxuseclass}
\end{sphinxuseclass}

\subsubsection{Model Setup}
\label{\detokenize{example_pump/Stochastic_Modelling:Model-Setup}}
\sphinxAtStartPar
This model has been augmented stochastic states and behaviors to enable stochastic simulation. The main method for incorporating stochastic states is \sphinxcode{\sphinxupquote{assoc\_rand\_state}}, which is called at function initialization.

\begin{sphinxuseclass}{nbinput}
{
\sphinxsetup{VerbatimColor={named}{nbsphinx-code-bg}}
\sphinxsetup{VerbatimBorderColor={named}{nbsphinx-code-border}}
\begin{sphinxVerbatim}[commandchars=\\\{\}]
\llap{\color{nbsphinxin}[4]:\,\hspace{\fboxrule}\hspace{\fboxsep}}\PYG{n}{help}\PYG{p}{(}\PYG{n}{FxnBlock}\PYG{o}{.}\PYG{n}{assoc\PYGZus{}rand\PYGZus{}state}\PYG{p}{)}
\end{sphinxVerbatim}
}

\end{sphinxuseclass}
\begin{sphinxuseclass}{nboutput}
\begin{sphinxuseclass}{nblast}
{

\kern-\sphinxverbatimsmallskipamount\kern-\baselineskip
\kern+\FrameHeightAdjust\kern-\fboxrule
\vspace{\nbsphinxcodecellspacing}

\sphinxsetup{VerbatimColor={named}{white}}
\sphinxsetup{VerbatimBorderColor={named}{nbsphinx-code-border}}
\begin{sphinxuseclass}{output_area}
\begin{sphinxuseclass}{}


\begin{sphinxVerbatim}[commandchars=\\\{\}]
Help on function assoc\_rand\_state in module fmdtools.modeldef:

assoc\_rand\_state(self, name, default, seed=None, auto\_update=[])
    Associate a stochastic state with the Block. Enables the simulation of stochastic behavior over time.

    Parameters
    ----------
    name : str
        name for the parameter to use in the model behavior.
    default : int/float/str/etc
        Default value for the parameter for the parameter
    seed : int
        seed for the random state generator to use. Defaults to None.
    auto\_update : list, optional
        If given, updates the state with the given numpy method at each time-step.
        List is made up of two arguments:
        - generator\_method : str
            Name of the numpy random method to use.
            see: https://numpy.org/doc/stable/reference/random/generator.html
        - generator\_params : tuple
            Parameter inputs for the numpy generator function

\end{sphinxVerbatim}



\end{sphinxuseclass}
\end{sphinxuseclass}
}

\end{sphinxuseclass}
\end{sphinxuseclass}
\sphinxAtStartPar
The main parameters for \sphinxcode{\sphinxupquote{assoc\_rand\_state}} are a name, which will be the name of the variable in the behavior, and default a value, which is both the initial value of the parameter \sphinxstyleemphasis{and} the value of the parameter when the model is run deterministically. That is, fmdtools enables one to run stochastic models \sphinxstyleemphasis{both} using stochastic behaviors (where this number will be updated stochastically) and deterministically (where it will take its default value).

\sphinxAtStartPar
\sphinxcode{\sphinxupquote{assoc\_rand\_states}} is the same as \sphinxcode{\sphinxupquote{assoc\_rand\_state}}, except one can use it to instantiate multiple states by providing tuples with the form: (name, default).

\begin{sphinxuseclass}{nbinput}
{
\sphinxsetup{VerbatimColor={named}{nbsphinx-code-bg}}
\sphinxsetup{VerbatimBorderColor={named}{nbsphinx-code-border}}
\begin{sphinxVerbatim}[commandchars=\\\{\}]
\llap{\color{nbsphinxin}[5]:\,\hspace{\fboxrule}\hspace{\fboxsep}}\PYG{n}{help}\PYG{p}{(}\PYG{n}{FxnBlock}\PYG{o}{.}\PYG{n}{assoc\PYGZus{}rand\PYGZus{}states}\PYG{p}{)}
\end{sphinxVerbatim}
}

\end{sphinxuseclass}
\begin{sphinxuseclass}{nboutput}
\begin{sphinxuseclass}{nblast}
{

\kern-\sphinxverbatimsmallskipamount\kern-\baselineskip
\kern+\FrameHeightAdjust\kern-\fboxrule
\vspace{\nbsphinxcodecellspacing}

\sphinxsetup{VerbatimColor={named}{white}}
\sphinxsetup{VerbatimBorderColor={named}{nbsphinx-code-border}}
\begin{sphinxuseclass}{output_area}
\begin{sphinxuseclass}{}


\begin{sphinxVerbatim}[commandchars=\\\{\}]
Help on function assoc\_rand\_states in module fmdtools.modeldef:

assoc\_rand\_states(self, *states)
    Associates multiple random states with the model

    Parameters
    ----------
    *states : tuple
        can give any number of tuples for each of the states.
        The tuple is of the form (name, default), where:
            name : str
                name for the parameter to use in the model behavior.
            default : int/float/str/etc
                Default value for the parameter

\end{sphinxVerbatim}



\end{sphinxuseclass}
\end{sphinxuseclass}
}

\end{sphinxuseclass}
\end{sphinxuseclass}
\sphinxAtStartPar
Below, \sphinxcode{\sphinxupquote{assoc\_rand\_states}} is used in the ImportEE function to add two random states: \sphinxhyphen{} effstate, the quality of the input voltage (i.e. large fluctuations from a step change in power) \sphinxhyphen{} grid\_noise, which is meant to be fluctuations in voltage from the power source (i.e., more ordinary noise)

\begin{sphinxuseclass}{nbinput}
{
\sphinxsetup{VerbatimColor={named}{nbsphinx-code-bg}}
\sphinxsetup{VerbatimBorderColor={named}{nbsphinx-code-border}}
\begin{sphinxVerbatim}[commandchars=\\\{\}]
\llap{\color{nbsphinxin}[6]:\,\hspace{\fboxrule}\hspace{\fboxsep}}\PYG{n+nb}{print}\PYG{p}{(}\PYG{n}{inspect}\PYG{o}{.}\PYG{n}{getsource}\PYG{p}{(}\PYG{n}{ImportEE}\PYG{o}{.}\PYG{n+nf+fm}{\PYGZus{}\PYGZus{}init\PYGZus{}\PYGZus{}}\PYG{p}{)}\PYG{p}{)}
\end{sphinxVerbatim}
}

\end{sphinxuseclass}
\begin{sphinxuseclass}{nboutput}
\begin{sphinxuseclass}{nblast}
{

\kern-\sphinxverbatimsmallskipamount\kern-\baselineskip
\kern+\FrameHeightAdjust\kern-\fboxrule
\vspace{\nbsphinxcodecellspacing}

\sphinxsetup{VerbatimColor={named}{white}}
\sphinxsetup{VerbatimBorderColor={named}{nbsphinx-code-border}}
\begin{sphinxuseclass}{output_area}
\begin{sphinxuseclass}{}


\begin{sphinxVerbatim}[commandchars=\\\{\}]
    def \_\_init\_\_(self,name, flows):
        super().\_\_init\_\_(name,flows, flownames = ['EEout'])
        self.failrate=1e-5
        self.assoc\_modes(\{'no\_v':[0.80,[0,1,0], 10000], 'inf\_v':[0.20, [0,1,0], 5000]\}, key\_phases\_by='global')
        self.assoc\_rand\_states(('effstate', 1.0), ('grid\_noise',1.0))

\end{sphinxVerbatim}



\end{sphinxuseclass}
\end{sphinxuseclass}
}

\end{sphinxuseclass}
\end{sphinxuseclass}
\sphinxAtStartPar
This is then reflected in the behavior, which is defined in the \sphinxcode{\sphinxupquote{Block.set\_rand}} method.

\begin{sphinxuseclass}{nbinput}
{
\sphinxsetup{VerbatimColor={named}{nbsphinx-code-bg}}
\sphinxsetup{VerbatimBorderColor={named}{nbsphinx-code-border}}
\begin{sphinxVerbatim}[commandchars=\\\{\}]
\llap{\color{nbsphinxin}[7]:\,\hspace{\fboxrule}\hspace{\fboxsep}}\PYG{n}{help}\PYG{p}{(}\PYG{n}{Block}\PYG{o}{.}\PYG{n}{set\PYGZus{}rand}\PYG{p}{)}
\end{sphinxVerbatim}
}

\end{sphinxuseclass}
\begin{sphinxuseclass}{nboutput}
\begin{sphinxuseclass}{nblast}
{

\kern-\sphinxverbatimsmallskipamount\kern-\baselineskip
\kern+\FrameHeightAdjust\kern-\fboxrule
\vspace{\nbsphinxcodecellspacing}

\sphinxsetup{VerbatimColor={named}{white}}
\sphinxsetup{VerbatimBorderColor={named}{nbsphinx-code-border}}
\begin{sphinxuseclass}{output_area}
\begin{sphinxuseclass}{}


\begin{sphinxVerbatim}[commandchars=\\\{\}]
Help on function set\_rand in module fmdtools.modeldef:

set\_rand(self, statename, methodname, *args)
    Update the given random state with a given method and arguments

    Parameters
    ----------
    statename : str
        name of the random state defined in assoc\_rand\_state(s)
    methodname :
        str name of the numpy method to call in the rng
    *args : args
        arguments for the numpy method

\end{sphinxVerbatim}



\end{sphinxuseclass}
\end{sphinxuseclass}
}

\end{sphinxuseclass}
\end{sphinxuseclass}
\sphinxAtStartPar
In this method, \sphinxcode{\sphinxupquote{statename}} is the name of the random state, while \sphinxcode{\sphinxupquote{methodname}} corresponds to the name of a random distribution to update the state from. These distributions are pulled from a numpy random number generator and thus the options for distributions are provided in the numpy documentation: \sphinxurl{https://numpy.org/doc/stable/reference/random/generator.html\#distributions} . These distributions include most one might need including normal, beta, uniform, etc.

\sphinxAtStartPar
\sphinxcode{\sphinxupquote{*args}} in \sphinxcode{\sphinxupquote{set\_rand}} refers to the corresponding arguments to call the given method with, which should be pulled from the corresponding documentation. Below, we have points where this is called. \sphinxhyphen{} First, to pull \sphinxcode{\sphinxupquote{effstate}} from a triangular distribution with minimum 0.9, mode 1.0, and max 1.1 \sphinxhyphen{} Parameters are provided here: \sphinxurl{https://numpy.org/doc/stable/reference/random/generated/numpy.random.Generator.triangular.html\#numpy.random.Generator.triangular} \sphinxhyphen{} Second, to pull \sphinxcode{\sphinxupquote{grid\_noise}} from
a normal distribution centered on 1 with a standard deviation that varies over time according to a sine wave. \sphinxhyphen{} Parameters are provided here: \sphinxurl{https://numpy.org/doc/stable/reference/random/generated/numpy.random.Generator.normal.html\#numpy.random.Generator.normal}

\begin{sphinxuseclass}{nbinput}
{
\sphinxsetup{VerbatimColor={named}{nbsphinx-code-bg}}
\sphinxsetup{VerbatimBorderColor={named}{nbsphinx-code-border}}
\begin{sphinxVerbatim}[commandchars=\\\{\}]
\llap{\color{nbsphinxin}[8]:\,\hspace{\fboxrule}\hspace{\fboxsep}}\PYG{n+nb}{print}\PYG{p}{(}\PYG{n}{inspect}\PYG{o}{.}\PYG{n}{getsource}\PYG{p}{(}\PYG{n}{ImportEE}\PYG{o}{.}\PYG{n}{behavior}\PYG{p}{)}\PYG{p}{)}
\end{sphinxVerbatim}
}

\end{sphinxuseclass}
\begin{sphinxuseclass}{nboutput}
\begin{sphinxuseclass}{nblast}
{

\kern-\sphinxverbatimsmallskipamount\kern-\baselineskip
\kern+\FrameHeightAdjust\kern-\fboxrule
\vspace{\nbsphinxcodecellspacing}

\sphinxsetup{VerbatimColor={named}{white}}
\sphinxsetup{VerbatimBorderColor={named}{nbsphinx-code-border}}
\begin{sphinxuseclass}{output_area}
\begin{sphinxuseclass}{}


\begin{sphinxVerbatim}[commandchars=\\\{\}]
    def behavior(self,time):
        if self.has\_fault('no\_v'):      self.effstate=0.0 \#an open circuit means no voltage is exported
        elif self.has\_fault('inf\_v'):   self.effstate=100.0 \#a voltage spike means voltage is much higher
        else:
            if time>self.time: self.set\_rand('effstate','triangular',0.9,1,1.1)
        if time>self.time:
            self.set\_rand('grid\_noise','normal',1, 0.1*(2+np.sin(np.pi/2*time)))

        self.EEout.voltage= self.grid\_noise*self.effstate * 500

\end{sphinxVerbatim}



\end{sphinxuseclass}
\end{sphinxuseclass}
}

\end{sphinxuseclass}
\end{sphinxuseclass}
\sphinxAtStartPar
The Import Signal function has a similar setup, where \sphinxcode{\sphinxupquote{sig\_noise}} is associated as a random state and then pulled as a random choice from a given set of options. To vary the behavior, here the noise is only pulled every five seconds.

\sphinxAtStartPar
This function also uses the \sphinxcode{\sphinxupquote{to\_default}} method, whic resets a given stochastic state to its default value. Here this is used to zero out any signal noise which would otherwise occur when there is no power.

\begin{sphinxuseclass}{nbinput}
{
\sphinxsetup{VerbatimColor={named}{nbsphinx-code-bg}}
\sphinxsetup{VerbatimBorderColor={named}{nbsphinx-code-border}}
\begin{sphinxVerbatim}[commandchars=\\\{\}]
\llap{\color{nbsphinxin}[9]:\,\hspace{\fboxrule}\hspace{\fboxsep}}\PYG{n}{help}\PYG{p}{(}\PYG{n}{Block}\PYG{o}{.}\PYG{n}{to\PYGZus{}default}\PYG{p}{)}
\end{sphinxVerbatim}
}

\end{sphinxuseclass}
\begin{sphinxuseclass}{nboutput}
\begin{sphinxuseclass}{nblast}
{

\kern-\sphinxverbatimsmallskipamount\kern-\baselineskip
\kern+\FrameHeightAdjust\kern-\fboxrule
\vspace{\nbsphinxcodecellspacing}

\sphinxsetup{VerbatimColor={named}{white}}
\sphinxsetup{VerbatimBorderColor={named}{nbsphinx-code-border}}
\begin{sphinxuseclass}{output_area}
\begin{sphinxuseclass}{}


\begin{sphinxVerbatim}[commandchars=\\\{\}]
Help on function to\_default in module fmdtools.modeldef:

to\_default(self, *statenames)
    Resets (given or all by default) random states to their default values

    Parameters
    ----------
    *statenames : str, str, str{\ldots}
        names of the random state defined in assoc\_rand\_state(s)

\end{sphinxVerbatim}



\end{sphinxuseclass}
\end{sphinxuseclass}
}

\end{sphinxuseclass}
\end{sphinxuseclass}
\begin{sphinxuseclass}{nbinput}
{
\sphinxsetup{VerbatimColor={named}{nbsphinx-code-bg}}
\sphinxsetup{VerbatimBorderColor={named}{nbsphinx-code-border}}
\begin{sphinxVerbatim}[commandchars=\\\{\}]
\llap{\color{nbsphinxin}[10]:\,\hspace{\fboxrule}\hspace{\fboxsep}}\PYG{n+nb}{print}\PYG{p}{(}\PYG{n}{inspect}\PYG{o}{.}\PYG{n}{getsource}\PYG{p}{(}\PYG{n}{ImportSig}\PYG{p}{)}\PYG{p}{)}
\end{sphinxVerbatim}
}

\end{sphinxuseclass}
\begin{sphinxuseclass}{nboutput}
\begin{sphinxuseclass}{nblast}
{

\kern-\sphinxverbatimsmallskipamount\kern-\baselineskip
\kern+\FrameHeightAdjust\kern-\fboxrule
\vspace{\nbsphinxcodecellspacing}

\sphinxsetup{VerbatimColor={named}{white}}
\sphinxsetup{VerbatimBorderColor={named}{nbsphinx-code-border}}
\begin{sphinxuseclass}{output_area}
\begin{sphinxuseclass}{}


\begin{sphinxVerbatim}[commandchars=\\\{\}]
class ImportSig(FxnBlock):
    """ Import Signal is the on/off switch """
    def \_\_init\_\_(self,name,flows):
        """ Here the main flow is the signal"""
        super().\_\_init\_\_(name,flows, flownames=['Sigout'])
        self.failrate=1e-6
        self.assoc\_modes(\{'no\_sig':[1.0, [1.5, 1.0, 1.0], 10000]\}, key\_phases\_by='global')
        self.assoc\_rand\_state('sig\_noise',1.0)
    def behavior(self, time):
        if self.has\_fault('no\_sig'): self.Sigout.power=0.0 \#an open circuit means no voltage is exported
        else:
            if time<5:      self.Sigout.power=0.0;self.to\_default('sig\_noise')
            elif time<50:
                if not time\%5:  self.set\_rand('sig\_noise', 'choice', [1.0, 0.9, 1.1])
                self.Sigout.power=1.0*self.sig\_noise
            else:           self.Sigout.power=0.0; self.to\_default()

\end{sphinxVerbatim}



\end{sphinxuseclass}
\end{sphinxuseclass}
}

\end{sphinxuseclass}
\end{sphinxuseclass}
\sphinxAtStartPar
Finally, below the same is done for the \sphinxcode{\sphinxupquote{eff}} state in \sphinxcode{\sphinxupquote{MoveWat}}. This function takes advantage of the auto\_update option in \sphinxcode{\sphinxupquote{assoc\_rand\_state}}, which makes it so that the state automatically updates at each time\sphinxhyphen{}step.

\sphinxAtStartPar
This auto\sphinxhyphen{}update option uses similar syntax to \sphinxcode{\sphinxupquote{set\_rand}}, where the first argument is the name of the numpy method and the second argument is its parameters.

\begin{sphinxuseclass}{nbinput}
{
\sphinxsetup{VerbatimColor={named}{nbsphinx-code-bg}}
\sphinxsetup{VerbatimBorderColor={named}{nbsphinx-code-border}}
\begin{sphinxVerbatim}[commandchars=\\\{\}]
\llap{\color{nbsphinxin}[11]:\,\hspace{\fboxrule}\hspace{\fboxsep}}\PYG{n+nb}{print}\PYG{p}{(}\PYG{n}{inspect}\PYG{o}{.}\PYG{n}{getsource}\PYG{p}{(}\PYG{n}{MoveWat}\PYG{o}{.}\PYG{n+nf+fm}{\PYGZus{}\PYGZus{}init\PYGZus{}\PYGZus{}}\PYG{p}{)}\PYG{p}{)}
\end{sphinxVerbatim}
}

\end{sphinxuseclass}
\begin{sphinxuseclass}{nboutput}
\begin{sphinxuseclass}{nblast}
{

\kern-\sphinxverbatimsmallskipamount\kern-\baselineskip
\kern+\FrameHeightAdjust\kern-\fboxrule
\vspace{\nbsphinxcodecellspacing}

\sphinxsetup{VerbatimColor={named}{white}}
\sphinxsetup{VerbatimBorderColor={named}{nbsphinx-code-border}}
\begin{sphinxuseclass}{output_area}
\begin{sphinxuseclass}{}


\begin{sphinxVerbatim}[commandchars=\\\{\}]
    def \_\_init\_\_(self,name, flows, delay):
        flownames=['EEin', 'Sigin', 'Watin', 'Watout']
        states=\{'total\_flow':0.0\} \#effectiveness state
        self.delay=delay \#delay parameter
        super().\_\_init\_\_(name,flows,flownames=flownames,states=states, timers=\{'timer'\})
        self.failrate=1e-5
        self.assoc\_modes(\{'mech\_break':[0.6, [0.1, 1.2, 0.1], 5000], 'short':[1.0, [1.5, 1.0, 1.0], 10000]\}, key\_phases\_by='global')
        self.assoc\_rand\_state("eff",1.0,auto\_update=['normal', (1.0, 0.2)])

\end{sphinxVerbatim}



\end{sphinxuseclass}
\end{sphinxuseclass}
}

\end{sphinxuseclass}
\end{sphinxuseclass}
\sphinxAtStartPar
The advantage of this is that simple stochastic behaviors do not need to be defined in the behavior method, however, it is less flexible than the previous approach, since one is limited to always drawing from the same distribution at each time\sphinxhyphen{}step. The resulting behavior method (below) thus has no \sphinxcode{\sphinxupquote{set\_rand}} call, since the state \sphinxcode{\sphinxupquote{eff}} is automaticall updated outside the behavior definition.

\begin{sphinxuseclass}{nbinput}
{
\sphinxsetup{VerbatimColor={named}{nbsphinx-code-bg}}
\sphinxsetup{VerbatimBorderColor={named}{nbsphinx-code-border}}
\begin{sphinxVerbatim}[commandchars=\\\{\}]
\llap{\color{nbsphinxin}[12]:\,\hspace{\fboxrule}\hspace{\fboxsep}}\PYG{n+nb}{print}\PYG{p}{(}\PYG{n}{inspect}\PYG{o}{.}\PYG{n}{getsource}\PYG{p}{(}\PYG{n}{MoveWat}\PYG{o}{.}\PYG{n}{behavior}\PYG{p}{)}\PYG{p}{)}
\end{sphinxVerbatim}
}

\end{sphinxuseclass}
\begin{sphinxuseclass}{nboutput}
\begin{sphinxuseclass}{nblast}
{

\kern-\sphinxverbatimsmallskipamount\kern-\baselineskip
\kern+\FrameHeightAdjust\kern-\fboxrule
\vspace{\nbsphinxcodecellspacing}

\sphinxsetup{VerbatimColor={named}{white}}
\sphinxsetup{VerbatimBorderColor={named}{nbsphinx-code-border}}
\begin{sphinxuseclass}{output_area}
\begin{sphinxuseclass}{}


\begin{sphinxVerbatim}[commandchars=\\\{\}]
    def behavior(self, time):
        """ here we can define how the function will behave with different faults """
        if self.has\_fault('mech\_break'):
            self.Watout.pressure = 0.0
            self.Watout.flowrate = 0.0
        else:
            self.Watout.pressure = 10/500 * self.Sigin.power*self.eff*min(1000, self.EEin.voltage)*self.Watin.level/self.Watout.area
            self.Watout.flowrate = 0.3/500 * self.Sigin.power*self.eff*min(1000, self.EEin.voltage)*self.Watin.level*self.Watout.area

        self.Watin.pressure=self.Watout.pressure
        self.Watin.flowrate=self.Watout.flowrate
        if time>self.time: self.total\_flow+=self.Watout.flowrate

\end{sphinxVerbatim}



\end{sphinxuseclass}
\end{sphinxuseclass}
}

\end{sphinxuseclass}
\end{sphinxuseclass}
\sphinxAtStartPar
Finally, it can be helpful for stochastic simulation to include a default seed in the model’s modelparams, as shown.

\begin{sphinxuseclass}{nbinput}
{
\sphinxsetup{VerbatimColor={named}{nbsphinx-code-bg}}
\sphinxsetup{VerbatimBorderColor={named}{nbsphinx-code-border}}
\begin{sphinxVerbatim}[commandchars=\\\{\}]
\llap{\color{nbsphinxin}[13]:\,\hspace{\fboxrule}\hspace{\fboxsep}}\PYG{n+nb}{print}\PYG{p}{(}\PYG{n}{inspect}\PYG{o}{.}\PYG{n}{getsource}\PYG{p}{(}\PYG{n}{Pump}\PYG{o}{.}\PYG{n+nf+fm}{\PYGZus{}\PYGZus{}init\PYGZus{}\PYGZus{}}\PYG{p}{)}\PYG{p}{)}
\end{sphinxVerbatim}
}

\end{sphinxuseclass}
\begin{sphinxuseclass}{nboutput}
\begin{sphinxuseclass}{nblast}
{

\kern-\sphinxverbatimsmallskipamount\kern-\baselineskip
\kern+\FrameHeightAdjust\kern-\fboxrule
\vspace{\nbsphinxcodecellspacing}

\sphinxsetup{VerbatimColor={named}{white}}
\sphinxsetup{VerbatimBorderColor={named}{nbsphinx-code-border}}
\begin{sphinxuseclass}{output_area}
\begin{sphinxuseclass}{}


\begin{sphinxVerbatim}[commandchars=\\\{\}]
    def \_\_init\_\_(self, params=\{'cost':\{'repair', 'water'\}, 'delay':10, 'units':'hrs'\}, \textbackslash{}
                 modelparams = \{'phases':\{'start':[0,5], 'on':[5, 50], 'end':[50,55]\}, 'times':[0,20, 55], 'tstep':1,'seed':1\}, \textbackslash{}
                     valparams=\{'flows':\{'Wat\_2':'flowrate', 'EE\_1':'current'\}\}):
        super().\_\_init\_\_(params=params, modelparams=modelparams, valparams=valparams)
        self.add\_flow('EE\_1', \{'current':1.0, 'voltage':1.0\})
        self.add\_flow('Sig\_1',  \{'power':1.0\})
        \# custom flows which we defined earlier can be added also:
        self.add\_flow('Wat\_1', Water())
        self.add\_flow('Wat\_2', Water())

        self.add\_fxn('ImportEE',['EE\_1'],fclass=ImportEE)
        self.add\_fxn('ImportWater',['Wat\_1'],fclass=ImportWater)
        self.add\_fxn('ImportSignal',['Sig\_1'],fclass=ImportSig)
        self.add\_fxn('MoveWater', ['EE\_1', 'Sig\_1', 'Wat\_1', 'Wat\_2'],fclass=MoveWat, fparams = params['delay'])
        self.add\_fxn('ExportWater', ['Wat\_2'], fclass=ExportWater)

        self.build\_model()

\end{sphinxVerbatim}



\end{sphinxuseclass}
\end{sphinxuseclass}
}

\end{sphinxuseclass}
\end{sphinxuseclass}
\sphinxAtStartPar
This \sphinxcode{\sphinxupquote{seed}} parameter ensures that each simulation in the same random thread will produce the same results, and that results can be replicated from stochastic simulation by running the model with the same seed.


\subsubsection{Simulation and Analysis}
\label{\detokenize{example_pump/Stochastic_Modelling:Simulation-and-Analysis}}
\sphinxAtStartPar
With this model set up, we can now simulate it using the methods in \sphinxcode{\sphinxupquote{propagate}}.


\subsection{Single\sphinxhyphen{}Scenario Simulation}
\label{\detokenize{example_pump/Stochastic_Modelling:Single-Scenario-Simulation}}
\sphinxAtStartPar
First, let’s simulate it in the nominal scenario:

\begin{sphinxuseclass}{nbinput}
{
\sphinxsetup{VerbatimColor={named}{nbsphinx-code-bg}}
\sphinxsetup{VerbatimBorderColor={named}{nbsphinx-code-border}}
\begin{sphinxVerbatim}[commandchars=\\\{\}]
\llap{\color{nbsphinxin}[14]:\,\hspace{\fboxrule}\hspace{\fboxsep}}\PYG{n}{mdl} \PYG{o}{=} \PYG{n}{Pump}\PYG{p}{(}\PYG{p}{)}
\PYG{n}{endresults}\PYG{p}{,} \PYG{n}{resgraph}\PYG{p}{,} \PYG{n}{mdlhist} \PYG{o}{=} \PYG{n}{propagate}\PYG{o}{.}\PYG{n}{nominal}\PYG{p}{(}\PYG{n}{mdl}\PYG{p}{)}
\PYG{n}{fig} \PYG{o}{=} \PYG{n}{rd}\PYG{o}{.}\PYG{n}{plot}\PYG{o}{.}\PYG{n}{mdlhistvals}\PYG{p}{(}\PYG{n}{mdlhist}\PYG{p}{,} \PYG{n}{fxnflowvals} \PYG{o}{=} \PYG{p}{\PYGZob{}}\PYG{l+s+s1}{\PYGZsq{}}\PYG{l+s+s1}{ImportEE}\PYG{l+s+s1}{\PYGZsq{}}\PYG{p}{:}\PYG{p}{[}\PYG{l+s+s2}{\PYGZdq{}}\PYG{l+s+s2}{effstate}\PYG{l+s+s2}{\PYGZdq{}}\PYG{p}{,}\PYG{l+s+s2}{\PYGZdq{}}\PYG{l+s+s2}{grid\PYGZus{}noise}\PYG{l+s+s2}{\PYGZdq{}}\PYG{p}{]}\PYG{p}{,} \PYG{l+s+s1}{\PYGZsq{}}\PYG{l+s+s1}{ImportSignal}\PYG{l+s+s1}{\PYGZsq{}}\PYG{p}{:}\PYG{l+s+s1}{\PYGZsq{}}\PYG{l+s+s1}{sig\PYGZus{}noise}\PYG{l+s+s1}{\PYGZsq{}}\PYG{p}{,} \PYG{l+s+s1}{\PYGZsq{}}\PYG{l+s+s1}{MoveWater}\PYG{l+s+s1}{\PYGZsq{}}\PYG{p}{:}\PYG{l+s+s1}{\PYGZsq{}}\PYG{l+s+s1}{eff}\PYG{l+s+s1}{\PYGZsq{}} \PYG{p}{\PYGZcb{}}\PYG{p}{)}
\end{sphinxVerbatim}
}

\end{sphinxuseclass}
\begin{sphinxuseclass}{nboutput}
\begin{sphinxuseclass}{nblast}
\hrule height -\fboxrule\relax
\vspace{\nbsphinxcodecellspacing}

\makeatletter\setbox\nbsphinxpromptbox\box\voidb@x\makeatother

\begin{nbsphinxfancyoutput}

\begin{sphinxuseclass}{output_area}
\begin{sphinxuseclass}{}
\noindent\sphinxincludegraphics[width=426\sphinxpxdimen,height=269\sphinxpxdimen]{{example_pump_Stochastic_Modelling_28_0}.png}

\end{sphinxuseclass}
\end{sphinxuseclass}
\end{nbsphinxfancyoutput}

\end{sphinxuseclass}
\end{sphinxuseclass}
\sphinxAtStartPar
As shown, even with all of these random states, it doesn’t appear that the behavior actually changes stochastically over time. This is because, by default, \sphinxcode{\sphinxupquote{propagate}} methods run \sphinxstyleemphasis{deterministically}, meaning the stochastic states take their default values. To run stochastically, we use the option \sphinxcode{\sphinxupquote{run\_stochastic=True}}.

\begin{sphinxuseclass}{nbinput}
\begin{sphinxuseclass}{nblast}
{
\sphinxsetup{VerbatimColor={named}{nbsphinx-code-bg}}
\sphinxsetup{VerbatimBorderColor={named}{nbsphinx-code-border}}
\begin{sphinxVerbatim}[commandchars=\\\{\}]
\llap{\color{nbsphinxin}[15]:\,\hspace{\fboxrule}\hspace{\fboxsep}}\PYG{n}{mdl} \PYG{o}{=} \PYG{n}{Pump}\PYG{p}{(}\PYG{p}{)}
\PYG{n}{endresults}\PYG{p}{,} \PYG{n}{resgraph}\PYG{p}{,} \PYG{n}{mdlhist} \PYG{o}{=} \PYG{n}{propagate}\PYG{o}{.}\PYG{n}{nominal}\PYG{p}{(}\PYG{n}{mdl}\PYG{p}{,} \PYG{n}{run\PYGZus{}stochastic}\PYG{o}{=}\PYG{k+kc}{True}\PYG{p}{)}
\end{sphinxVerbatim}
}

\end{sphinxuseclass}
\end{sphinxuseclass}
\begin{sphinxuseclass}{nbinput}
{
\sphinxsetup{VerbatimColor={named}{nbsphinx-code-bg}}
\sphinxsetup{VerbatimBorderColor={named}{nbsphinx-code-border}}
\begin{sphinxVerbatim}[commandchars=\\\{\}]
\llap{\color{nbsphinxin}[16]:\,\hspace{\fboxrule}\hspace{\fboxsep}}\PYG{n}{fig} \PYG{o}{=} \PYG{n}{rd}\PYG{o}{.}\PYG{n}{plot}\PYG{o}{.}\PYG{n}{mdlhistvals}\PYG{p}{(}\PYG{n}{mdlhist}\PYG{p}{,} \PYG{n}{fxnflowvals} \PYG{o}{=} \PYG{p}{\PYGZob{}}\PYG{l+s+s1}{\PYGZsq{}}\PYG{l+s+s1}{ImportEE}\PYG{l+s+s1}{\PYGZsq{}}\PYG{p}{:}\PYG{p}{[}\PYG{l+s+s2}{\PYGZdq{}}\PYG{l+s+s2}{effstate}\PYG{l+s+s2}{\PYGZdq{}}\PYG{p}{,}\PYG{l+s+s2}{\PYGZdq{}}\PYG{l+s+s2}{grid\PYGZus{}noise}\PYG{l+s+s2}{\PYGZdq{}}\PYG{p}{]}\PYG{p}{,} \PYG{l+s+s1}{\PYGZsq{}}\PYG{l+s+s1}{ImportSignal}\PYG{l+s+s1}{\PYGZsq{}}\PYG{p}{:}\PYG{l+s+s1}{\PYGZsq{}}\PYG{l+s+s1}{sig\PYGZus{}noise}\PYG{l+s+s1}{\PYGZsq{}}\PYG{p}{,} \PYG{l+s+s1}{\PYGZsq{}}\PYG{l+s+s1}{MoveWater}\PYG{l+s+s1}{\PYGZsq{}}\PYG{p}{:}\PYG{l+s+s1}{\PYGZsq{}}\PYG{l+s+s1}{eff}\PYG{l+s+s1}{\PYGZsq{}} \PYG{p}{\PYGZcb{}}\PYG{p}{)}
\end{sphinxVerbatim}
}

\end{sphinxuseclass}
\begin{sphinxuseclass}{nboutput}
\begin{sphinxuseclass}{nblast}
\hrule height -\fboxrule\relax
\vspace{\nbsphinxcodecellspacing}

\makeatletter\setbox\nbsphinxpromptbox\box\voidb@x\makeatother

\begin{nbsphinxfancyoutput}

\begin{sphinxuseclass}{output_area}
\begin{sphinxuseclass}{}
\noindent\sphinxincludegraphics[width=420\sphinxpxdimen,height=269\sphinxpxdimen]{{example_pump_Stochastic_Modelling_31_0}.png}

\end{sphinxuseclass}
\end{sphinxuseclass}
\end{nbsphinxfancyoutput}

\end{sphinxuseclass}
\end{sphinxuseclass}
\sphinxAtStartPar
As shown, this is more what we would expect from a random . Note that this simulation comes from the default model seed, and will this will always be the same. To get a different seed, we can pass new modelparams as an argument to \sphinxcode{\sphinxupquote{propagate.nominal}}

\begin{sphinxuseclass}{nbinput}
{
\sphinxsetup{VerbatimColor={named}{nbsphinx-code-bg}}
\sphinxsetup{VerbatimBorderColor={named}{nbsphinx-code-border}}
\begin{sphinxVerbatim}[commandchars=\\\{\}]
\llap{\color{nbsphinxin}[17]:\,\hspace{\fboxrule}\hspace{\fboxsep}}\PYG{n}{mdl} \PYG{o}{=} \PYG{n}{Pump}\PYG{p}{(}\PYG{p}{)}
\PYG{n}{endresults}\PYG{p}{,} \PYG{n}{resgraph}\PYG{p}{,} \PYG{n}{mdlhist} \PYG{o}{=} \PYG{n}{propagate}\PYG{o}{.}\PYG{n}{nominal}\PYG{p}{(}\PYG{n}{mdl}\PYG{p}{,} \PYG{n}{run\PYGZus{}stochastic}\PYG{o}{=}\PYG{k+kc}{True}\PYG{p}{,} \PYG{n}{modelparams}\PYG{o}{=}\PYG{p}{\PYGZob{}}\PYG{l+s+s1}{\PYGZsq{}}\PYG{l+s+s1}{seed}\PYG{l+s+s1}{\PYGZsq{}}\PYG{p}{:}\PYG{l+m+mi}{10}\PYG{p}{\PYGZcb{}}\PYG{p}{)}
\PYG{n}{fig} \PYG{o}{=} \PYG{n}{rd}\PYG{o}{.}\PYG{n}{plot}\PYG{o}{.}\PYG{n}{mdlhistvals}\PYG{p}{(}\PYG{n}{mdlhist}\PYG{p}{,} \PYG{n}{fxnflowvals} \PYG{o}{=} \PYG{p}{\PYGZob{}}\PYG{l+s+s1}{\PYGZsq{}}\PYG{l+s+s1}{ImportEE}\PYG{l+s+s1}{\PYGZsq{}}\PYG{p}{:}\PYG{p}{[}\PYG{l+s+s2}{\PYGZdq{}}\PYG{l+s+s2}{effstate}\PYG{l+s+s2}{\PYGZdq{}}\PYG{p}{,}\PYG{l+s+s2}{\PYGZdq{}}\PYG{l+s+s2}{grid\PYGZus{}noise}\PYG{l+s+s2}{\PYGZdq{}}\PYG{p}{]}\PYG{p}{,} \PYG{l+s+s1}{\PYGZsq{}}\PYG{l+s+s1}{ImportSignal}\PYG{l+s+s1}{\PYGZsq{}}\PYG{p}{:}\PYG{l+s+s1}{\PYGZsq{}}\PYG{l+s+s1}{sig\PYGZus{}noise}\PYG{l+s+s1}{\PYGZsq{}}\PYG{p}{,} \PYG{l+s+s1}{\PYGZsq{}}\PYG{l+s+s1}{MoveWater}\PYG{l+s+s1}{\PYGZsq{}}\PYG{p}{:}\PYG{l+s+s1}{\PYGZsq{}}\PYG{l+s+s1}{eff}\PYG{l+s+s1}{\PYGZsq{}} \PYG{p}{\PYGZcb{}}\PYG{p}{)}
\end{sphinxVerbatim}
}

\end{sphinxuseclass}
\begin{sphinxuseclass}{nboutput}
\begin{sphinxuseclass}{nblast}
\hrule height -\fboxrule\relax
\vspace{\nbsphinxcodecellspacing}

\makeatletter\setbox\nbsphinxpromptbox\box\voidb@x\makeatother

\begin{nbsphinxfancyoutput}

\begin{sphinxuseclass}{output_area}
\begin{sphinxuseclass}{}
\noindent\sphinxincludegraphics[width=426\sphinxpxdimen,height=269\sphinxpxdimen]{{example_pump_Stochastic_Modelling_33_0}.png}

\end{sphinxuseclass}
\end{sphinxuseclass}
\end{nbsphinxfancyoutput}

\end{sphinxuseclass}
\end{sphinxuseclass}
\sphinxAtStartPar
We can further simulate faults as we would before.

\begin{sphinxuseclass}{nbinput}
{
\sphinxsetup{VerbatimColor={named}{nbsphinx-code-bg}}
\sphinxsetup{VerbatimBorderColor={named}{nbsphinx-code-border}}
\begin{sphinxVerbatim}[commandchars=\\\{\}]
\llap{\color{nbsphinxin}[18]:\,\hspace{\fboxrule}\hspace{\fboxsep}}\PYG{n}{mdl} \PYG{o}{=} \PYG{n}{Pump}\PYG{p}{(}\PYG{p}{)}
\PYG{n}{endresults}\PYG{p}{,} \PYG{n}{resgraph}\PYG{p}{,} \PYG{n}{mdlhist} \PYG{o}{=} \PYG{n}{propagate}\PYG{o}{.}\PYG{n}{one\PYGZus{}fault}\PYG{p}{(}\PYG{n}{mdl}\PYG{p}{,} \PYG{l+s+s1}{\PYGZsq{}}\PYG{l+s+s1}{ExportWater}\PYG{l+s+s1}{\PYGZsq{}}\PYG{p}{,}\PYG{l+s+s1}{\PYGZsq{}}\PYG{l+s+s1}{block}\PYG{l+s+s1}{\PYGZsq{}}\PYG{p}{,} \PYG{n}{time}\PYG{o}{=}\PYG{l+m+mi}{20}\PYG{p}{,} \PYG{n}{run\PYGZus{}stochastic}\PYG{o}{=}\PYG{k+kc}{True}\PYG{p}{,} \PYG{n}{modelparams}\PYG{o}{=}\PYG{p}{\PYGZob{}}\PYG{l+s+s1}{\PYGZsq{}}\PYG{l+s+s1}{seed}\PYG{l+s+s1}{\PYGZsq{}}\PYG{p}{:}\PYG{l+m+mi}{10}\PYG{p}{\PYGZcb{}}\PYG{p}{)}
\PYG{n}{fig} \PYG{o}{=} \PYG{n}{rd}\PYG{o}{.}\PYG{n}{plot}\PYG{o}{.}\PYG{n}{mdlhistvals}\PYG{p}{(}\PYG{n}{mdlhist}\PYG{p}{,} \PYG{n}{time}\PYG{o}{=}\PYG{l+m+mi}{20}\PYG{p}{,} \PYG{n}{fxnflowvals}\PYG{o}{=}\PYG{p}{\PYGZob{}}\PYG{l+s+s1}{\PYGZsq{}}\PYG{l+s+s1}{MoveWater}\PYG{l+s+s1}{\PYGZsq{}}\PYG{p}{:}\PYG{p}{[}\PYG{l+s+s1}{\PYGZsq{}}\PYG{l+s+s1}{eff}\PYG{l+s+s1}{\PYGZsq{}}\PYG{p}{,}\PYG{l+s+s1}{\PYGZsq{}}\PYG{l+s+s1}{total\PYGZus{}flow}\PYG{l+s+s1}{\PYGZsq{}}\PYG{p}{]}\PYG{p}{,} \PYG{l+s+s1}{\PYGZsq{}}\PYG{l+s+s1}{Wat\PYGZus{}2}\PYG{l+s+s1}{\PYGZsq{}}\PYG{p}{:}\PYG{p}{[}\PYG{l+s+s1}{\PYGZsq{}}\PYG{l+s+s1}{flowrate}\PYG{l+s+s1}{\PYGZsq{}}\PYG{p}{,}\PYG{l+s+s1}{\PYGZsq{}}\PYG{l+s+s1}{pressure}\PYG{l+s+s1}{\PYGZsq{}}\PYG{p}{]}\PYG{p}{\PYGZcb{}}\PYG{p}{,} \PYG{n}{legend}\PYG{o}{=}\PYG{k+kc}{False}\PYG{p}{)}
\end{sphinxVerbatim}
}

\end{sphinxuseclass}
\begin{sphinxuseclass}{nboutput}
\begin{sphinxuseclass}{nblast}
\hrule height -\fboxrule\relax
\vspace{\nbsphinxcodecellspacing}

\makeatletter\setbox\nbsphinxpromptbox\box\voidb@x\makeatother

\begin{nbsphinxfancyoutput}

\begin{sphinxuseclass}{output_area}
\begin{sphinxuseclass}{}
\noindent\sphinxincludegraphics[width=426\sphinxpxdimen,height=286\sphinxpxdimen]{{example_pump_Stochastic_Modelling_35_0}.png}

\end{sphinxuseclass}
\end{sphinxuseclass}
\end{nbsphinxfancyoutput}

\end{sphinxuseclass}
\end{sphinxuseclass}
\sphinxAtStartPar
As shown the stochastic states still simulated over time after the fault. One thing to watch is thus that the stochastic update in the faulty scenario does or does not change the variables compared to the nominal state–since many of the visualizations (e.g., in \sphinxcode{\sphinxupquote{graph}}) are based on finding \sphinxstyleemphasis{differences} between faulty and nominal models to visualize degradation, these methods may be less useful/reliable.

\begin{sphinxuseclass}{nbinput}
{
\sphinxsetup{VerbatimColor={named}{nbsphinx-code-bg}}
\sphinxsetup{VerbatimBorderColor={named}{nbsphinx-code-border}}
\begin{sphinxVerbatim}[commandchars=\\\{\}]
\llap{\color{nbsphinxin}[19]:\,\hspace{\fboxrule}\hspace{\fboxsep}}\PYG{n}{fig} \PYG{o}{=} \PYG{n}{rd}\PYG{o}{.}\PYG{n}{plot}\PYG{o}{.}\PYG{n}{mdlhistvals}\PYG{p}{(}\PYG{n}{mdlhist}\PYG{p}{,} \PYG{n}{fxnflowvals} \PYG{o}{=} \PYG{p}{\PYGZob{}}\PYG{l+s+s1}{\PYGZsq{}}\PYG{l+s+s1}{ImportEE}\PYG{l+s+s1}{\PYGZsq{}}\PYG{p}{:}\PYG{p}{[}\PYG{l+s+s2}{\PYGZdq{}}\PYG{l+s+s2}{effstate}\PYG{l+s+s2}{\PYGZdq{}}\PYG{p}{,}\PYG{l+s+s2}{\PYGZdq{}}\PYG{l+s+s2}{grid\PYGZus{}noise}\PYG{l+s+s2}{\PYGZdq{}}\PYG{p}{]}\PYG{p}{,} \PYG{l+s+s1}{\PYGZsq{}}\PYG{l+s+s1}{ImportSignal}\PYG{l+s+s1}{\PYGZsq{}}\PYG{p}{:}\PYG{l+s+s1}{\PYGZsq{}}\PYG{l+s+s1}{sig\PYGZus{}noise}\PYG{l+s+s1}{\PYGZsq{}}\PYG{p}{,} \PYG{l+s+s1}{\PYGZsq{}}\PYG{l+s+s1}{MoveWater}\PYG{l+s+s1}{\PYGZsq{}}\PYG{p}{:}\PYG{l+s+s1}{\PYGZsq{}}\PYG{l+s+s1}{eff}\PYG{l+s+s1}{\PYGZsq{}} \PYG{p}{\PYGZcb{}}\PYG{p}{,} \PYG{n}{legend}\PYG{o}{=}\PYG{k+kc}{False}\PYG{p}{)}
\end{sphinxVerbatim}
}

\end{sphinxuseclass}
\begin{sphinxuseclass}{nboutput}
\begin{sphinxuseclass}{nblast}
\hrule height -\fboxrule\relax
\vspace{\nbsphinxcodecellspacing}

\makeatletter\setbox\nbsphinxpromptbox\box\voidb@x\makeatother

\begin{nbsphinxfancyoutput}

\begin{sphinxuseclass}{output_area}
\begin{sphinxuseclass}{}
\noindent\sphinxincludegraphics[width=443\sphinxpxdimen,height=322\sphinxpxdimen]{{example_pump_Stochastic_Modelling_37_0}.png}

\end{sphinxuseclass}
\end{sphinxuseclass}
\end{nbsphinxfancyoutput}

\end{sphinxuseclass}
\end{sphinxuseclass}
\sphinxAtStartPar
In the blockage scenario, there is no effect on the underlying stochastic states, since nothing was set up for this in the behavior. In the \sphinxcode{\sphinxupquote{no\_sig}} fault, on the other hand, we defined the signal noise to go to zero, as shown:

\begin{sphinxuseclass}{nbinput}
{
\sphinxsetup{VerbatimColor={named}{nbsphinx-code-bg}}
\sphinxsetup{VerbatimBorderColor={named}{nbsphinx-code-border}}
\begin{sphinxVerbatim}[commandchars=\\\{\}]
\llap{\color{nbsphinxin}[20]:\,\hspace{\fboxrule}\hspace{\fboxsep}}\PYG{n}{mdl} \PYG{o}{=} \PYG{n}{Pump}\PYG{p}{(}\PYG{p}{)}
\PYG{n}{endresults}\PYG{p}{,} \PYG{n}{resgraph}\PYG{p}{,} \PYG{n}{mdlhist} \PYG{o}{=} \PYG{n}{propagate}\PYG{o}{.}\PYG{n}{one\PYGZus{}fault}\PYG{p}{(}\PYG{n}{mdl}\PYG{p}{,} \PYG{l+s+s1}{\PYGZsq{}}\PYG{l+s+s1}{ImportSignal}\PYG{l+s+s1}{\PYGZsq{}}\PYG{p}{,} \PYG{l+s+s1}{\PYGZsq{}}\PYG{l+s+s1}{no\PYGZus{}sig}\PYG{l+s+s1}{\PYGZsq{}}\PYG{p}{,} \PYG{n}{time}\PYG{o}{=}\PYG{l+m+mi}{20}\PYG{p}{,} \PYG{n}{run\PYGZus{}stochastic}\PYG{o}{=}\PYG{k+kc}{True}\PYG{p}{,} \PYG{n}{modelparams}\PYG{o}{=}\PYG{p}{\PYGZob{}}\PYG{l+s+s1}{\PYGZsq{}}\PYG{l+s+s1}{seed}\PYG{l+s+s1}{\PYGZsq{}}\PYG{p}{:}\PYG{l+m+mi}{10}\PYG{p}{\PYGZcb{}}\PYG{p}{)}
\PYG{n}{fig} \PYG{o}{=} \PYG{n}{rd}\PYG{o}{.}\PYG{n}{plot}\PYG{o}{.}\PYG{n}{mdlhistvals}\PYG{p}{(}\PYG{n}{mdlhist}\PYG{p}{,} \PYG{n}{time}\PYG{o}{=}\PYG{l+m+mi}{20}\PYG{p}{,} \PYG{n}{fxnflowvals}\PYG{o}{=}\PYG{p}{\PYGZob{}}\PYG{l+s+s1}{\PYGZsq{}}\PYG{l+s+s1}{ImportEE}\PYG{l+s+s1}{\PYGZsq{}}\PYG{p}{:}\PYG{p}{[}\PYG{l+s+s2}{\PYGZdq{}}\PYG{l+s+s2}{effstate}\PYG{l+s+s2}{\PYGZdq{}}\PYG{p}{,}\PYG{l+s+s2}{\PYGZdq{}}\PYG{l+s+s2}{grid\PYGZus{}noise}\PYG{l+s+s2}{\PYGZdq{}}\PYG{p}{]}\PYG{p}{,} \PYG{l+s+s1}{\PYGZsq{}}\PYG{l+s+s1}{ImportSignal}\PYG{l+s+s1}{\PYGZsq{}}\PYG{p}{:}\PYG{l+s+s1}{\PYGZsq{}}\PYG{l+s+s1}{sig\PYGZus{}noise}\PYG{l+s+s1}{\PYGZsq{}}\PYG{p}{,} \PYG{l+s+s1}{\PYGZsq{}}\PYG{l+s+s1}{MoveWater}\PYG{l+s+s1}{\PYGZsq{}}\PYG{p}{:}\PYG{l+s+s1}{\PYGZsq{}}\PYG{l+s+s1}{eff}\PYG{l+s+s1}{\PYGZsq{}} \PYG{p}{\PYGZcb{}}\PYG{p}{,} \PYG{n}{legend}\PYG{o}{=}\PYG{k+kc}{False}\PYG{p}{)}
\end{sphinxVerbatim}
}

\end{sphinxuseclass}
\begin{sphinxuseclass}{nboutput}
\begin{sphinxuseclass}{nblast}
\hrule height -\fboxrule\relax
\vspace{\nbsphinxcodecellspacing}

\makeatletter\setbox\nbsphinxpromptbox\box\voidb@x\makeatother

\begin{nbsphinxfancyoutput}

\begin{sphinxuseclass}{output_area}
\begin{sphinxuseclass}{}
\noindent\sphinxincludegraphics[width=443\sphinxpxdimen,height=322\sphinxpxdimen]{{example_pump_Stochastic_Modelling_39_0}.png}

\end{sphinxuseclass}
\end{sphinxuseclass}
\end{nbsphinxfancyoutput}

\end{sphinxuseclass}
\end{sphinxuseclass}

\subsection{Multi\sphinxhyphen{}Scenario Simulation}
\label{\detokenize{example_pump/Stochastic_Modelling:Multi-Scenario-Simulation}}
\sphinxAtStartPar
Because stochastic models are non\sphinxhyphen{}deterministic, we are often interested not in the results of a single thread, but of the distribution of outcomes that might occur. To perform this kind of assessment, we can use a \sphinxcode{\sphinxupquote{NominalApproach}} to instantiate the model with a number of different seeds.

\begin{sphinxuseclass}{nbinput}
{
\sphinxsetup{VerbatimColor={named}{nbsphinx-code-bg}}
\sphinxsetup{VerbatimBorderColor={named}{nbsphinx-code-border}}
\begin{sphinxVerbatim}[commandchars=\\\{\}]
\llap{\color{nbsphinxin}[21]:\,\hspace{\fboxrule}\hspace{\fboxsep}}\PYG{n}{help}\PYG{p}{(}\PYG{n}{NominalApproach}\PYG{o}{.}\PYG{n}{add\PYGZus{}seed\PYGZus{}replicates}\PYG{p}{)}
\end{sphinxVerbatim}
}

\end{sphinxuseclass}
\begin{sphinxuseclass}{nboutput}
\begin{sphinxuseclass}{nblast}
{

\kern-\sphinxverbatimsmallskipamount\kern-\baselineskip
\kern+\FrameHeightAdjust\kern-\fboxrule
\vspace{\nbsphinxcodecellspacing}

\sphinxsetup{VerbatimColor={named}{white}}
\sphinxsetup{VerbatimBorderColor={named}{nbsphinx-code-border}}
\begin{sphinxuseclass}{output_area}
\begin{sphinxuseclass}{}


\begin{sphinxVerbatim}[commandchars=\\\{\}]
Help on function add\_seed\_replicates in module fmdtools.modeldef:

add\_seed\_replicates(self, rangeid, seeds)
    Generates an approach with different seeds to use for the model's internal stochastic behaviors

    Parameters
    ----------
    rangeid : str
        Name for the set of replicates
    seeds : int/list
        Number of seeds (if an int) or a list of seeds to use.

\end{sphinxVerbatim}



\end{sphinxuseclass}
\end{sphinxuseclass}
}

\end{sphinxuseclass}
\end{sphinxuseclass}
\sphinxAtStartPar
A \sphinxcode{\sphinxupquote{seeds}} option can also be used in several other NominalApproach methods to simultaneously change the input parameters and seeds of the model. Below, we create an approach to simulate the model 100 times.

\begin{sphinxuseclass}{nbinput}
\begin{sphinxuseclass}{nblast}
{
\sphinxsetup{VerbatimColor={named}{nbsphinx-code-bg}}
\sphinxsetup{VerbatimBorderColor={named}{nbsphinx-code-border}}
\begin{sphinxVerbatim}[commandchars=\\\{\}]
\llap{\color{nbsphinxin}[22]:\,\hspace{\fboxrule}\hspace{\fboxsep}}\PYG{n}{app} \PYG{o}{=} \PYG{n}{NominalApproach}\PYG{p}{(}\PYG{p}{)}
\PYG{n}{app}\PYG{o}{.}\PYG{n}{add\PYGZus{}seed\PYGZus{}replicates}\PYG{p}{(}\PYG{l+s+s1}{\PYGZsq{}}\PYG{l+s+s1}{test\PYGZus{}seeds}\PYG{l+s+s1}{\PYGZsq{}}\PYG{p}{,} \PYG{l+m+mi}{100}\PYG{p}{)}
\end{sphinxVerbatim}
}

\end{sphinxuseclass}
\end{sphinxuseclass}
\begin{sphinxuseclass}{nbinput}
{
\sphinxsetup{VerbatimColor={named}{nbsphinx-code-bg}}
\sphinxsetup{VerbatimBorderColor={named}{nbsphinx-code-border}}
\begin{sphinxVerbatim}[commandchars=\\\{\}]
\llap{\color{nbsphinxin}[23]:\,\hspace{\fboxrule}\hspace{\fboxsep}}\PYG{n}{endclasses}\PYG{p}{,} \PYG{n}{mdlhists}\PYG{o}{=}\PYG{n}{propagate}\PYG{o}{.}\PYG{n}{nominal\PYGZus{}approach}\PYG{p}{(}\PYG{n}{mdl}\PYG{p}{,}\PYG{n}{app}\PYG{p}{,} \PYG{n}{run\PYGZus{}stochastic}\PYG{o}{=}\PYG{k+kc}{True}\PYG{p}{)}
\end{sphinxVerbatim}
}

\end{sphinxuseclass}
\begin{sphinxuseclass}{nboutput}
\begin{sphinxuseclass}{nblast}
{

\kern-\sphinxverbatimsmallskipamount\kern-\baselineskip
\kern+\FrameHeightAdjust\kern-\fboxrule
\vspace{\nbsphinxcodecellspacing}

\sphinxsetup{VerbatimColor={named}{nbsphinx-stderr}}
\sphinxsetup{VerbatimBorderColor={named}{nbsphinx-code-border}}
\begin{sphinxuseclass}{output_area}
\begin{sphinxuseclass}{stderr}


\begin{sphinxVerbatim}[commandchars=\\\{\}]
SCENARIOS COMPLETE: 100\%|███████████████████████████████████████████████████████████| 100/100 [00:00<00:00, 110.00it/s]
\end{sphinxVerbatim}



\end{sphinxuseclass}
\end{sphinxuseclass}
}

\end{sphinxuseclass}
\end{sphinxuseclass}
\sphinxAtStartPar
To evaluate this behavior over time, we can then use \sphinxcode{\sphinxupquote{rd.plot.mdlhists}}, which will plot the \sphinxstyleemphasis{distribution} of behaviors over time.

\begin{sphinxuseclass}{nbinput}
{
\sphinxsetup{VerbatimColor={named}{nbsphinx-code-bg}}
\sphinxsetup{VerbatimBorderColor={named}{nbsphinx-code-border}}
\begin{sphinxVerbatim}[commandchars=\\\{\}]
\llap{\color{nbsphinxin}[24]:\,\hspace{\fboxrule}\hspace{\fboxsep}}\PYG{n}{help}\PYG{p}{(}\PYG{n}{rd}\PYG{o}{.}\PYG{n}{plot}\PYG{o}{.}\PYG{n}{mdlhists}\PYG{p}{)}
\end{sphinxVerbatim}
}

\end{sphinxuseclass}
\begin{sphinxuseclass}{nboutput}
\begin{sphinxuseclass}{nblast}
{

\kern-\sphinxverbatimsmallskipamount\kern-\baselineskip
\kern+\FrameHeightAdjust\kern-\fboxrule
\vspace{\nbsphinxcodecellspacing}

\sphinxsetup{VerbatimColor={named}{white}}
\sphinxsetup{VerbatimBorderColor={named}{nbsphinx-code-border}}
\begin{sphinxuseclass}{output_area}
\begin{sphinxuseclass}{}


\begin{sphinxVerbatim}[commandchars=\\\{\}]
Help on function mdlhists in module fmdtools.resultdisp.plot:

mdlhists(mdlhists, fxnflowvals, cols=2, aggregation='individual', comp\_groups=\{\}, legend\_loc=-1, xlabel='time', ylabels=\{\}, max\_ind='max', boundtype='fill', fillalpha=0.3, boundcolor='gray', boundlinestyle='--', ci=0.95, title='', indiv\_kwargs=\{\}, time\_slice=[], figsize='default', **kwargs)
    Plot the behavior over time of the given function/flow values
    over a set of scenarios, with ability to aggregate behaviors as needed.

    Parameters
    ----------
    mdlhists : dict
        Aggregate model history with structure \{'scen':mdlhist\}
    fxnflowsvals : dict, optional
        dict of flow values to plot with structure \{fxnflow:[vals]\}. The default is \{\}, which returns all.
    cols : int, optional
        columns to use in the figure. The default is 2.
    aggregation : str
        Way of aggregating the plot values. The default is 'individual'
        Note that only the `individual` option can be used for histories of non-numeric quantities
        (e.g., modes, which are recorded as strings)
        - 'individual' plots each run individually.
        - 'mean\_std' plots the mean values over the sim with standard deviation error bars
        - 'mean\_ci'  plots the mean values over the sim with mean confidence interval error bars
            - optional argument ci (float 0.0-1.0) specifies the confidence interval (Default:0.95)
        - 'mean\_bound' plots the mean values over the sim with variable bound error bars
        - 'percentile' plots the percentile distribution of the sim over time (does not reject outliers)
            - optional argument 'perc\_range' (int 0-100) specifies the percentile range of the inner bars (Default: 50)
    comp\_groups : dict
        Dictionary for comparison groups (if more than one) with structure:
            \{'group1':('scen1', 'scen2'), 'group2':('scen3', 'scen4')\} Default is \{\}
            If a legend is shown, group names are used as labels.
    legend\_loc : int
        Specifies the plot to place the legend on, if runs are bine compared. Default is -1 (the last plot)
        To remove the legend, give a value of False
    `indiv\_kwargs` dict
        dict of kwargs with structure \{comp1:kwargs1, comp2:kwargs2\}, where
        where kwargs is an individual dict of keyword arguments for the
        comparison group comp (or scenario, if not aggregated) which overrides
        the global kwargs (or default behavior).
    xlabel : str
        Label for the x-axes. Default is 'time'
    ylabel : dict
        Label for the y-axes with structure \{(fxnflowname, value):'label'\}
    max\_ind : int
        index (usually correlates to time) cutoff for the simulation. Default is 'max' which uses the first simulation termination time.
    boundtype : 'fill' or 'line'
        -'fill' plots the error bounds as a filled area
            - optional fillalpha (float) changes the alpha of this area.
        -'line' plots the error bounds as lines
            - optional boundcolor (str) changes the color of the bounds (default 'gray')
            - optional boundlinestyle (str) changes the style of the bound lines (default '--')
    title : str
        overall title for the plot. Default is ''
    time\_slice : int/list
        overlays a bar or bars at the given index when the fault was injected (if any). Default is []
    figsize : tuple (float,float)
        x-y size for the figure. The default is 'default', which dymanically gives 3 for each column and 2 for each row
    **kwargs : kwargs
        keyword arguments to mpl.plot e.g. linestyle, color, etc. See 'aggregation' for specification.

\end{sphinxVerbatim}



\end{sphinxuseclass}
\end{sphinxuseclass}
}

\end{sphinxuseclass}
\end{sphinxuseclass}
\sphinxAtStartPar
\sphinxcode{\sphinxupquote{rd.plot.mdlhists}} has a number of different options for visualization. For example, below we plot model states as individual lines:

\begin{sphinxuseclass}{nbinput}
{
\sphinxsetup{VerbatimColor={named}{nbsphinx-code-bg}}
\sphinxsetup{VerbatimBorderColor={named}{nbsphinx-code-border}}
\begin{sphinxVerbatim}[commandchars=\\\{\}]
\llap{\color{nbsphinxin}[25]:\,\hspace{\fboxrule}\hspace{\fboxsep}}\PYG{n}{fig} \PYG{o}{=} \PYG{n}{rd}\PYG{o}{.}\PYG{n}{plot}\PYG{o}{.}\PYG{n}{mdlhists}\PYG{p}{(}\PYG{n}{mdlhists}\PYG{p}{,} \PYG{p}{\PYGZob{}}\PYG{l+s+s1}{\PYGZsq{}}\PYG{l+s+s1}{MoveWater}\PYG{l+s+s1}{\PYGZsq{}}\PYG{p}{:}\PYG{p}{[}\PYG{l+s+s1}{\PYGZsq{}}\PYG{l+s+s1}{eff}\PYG{l+s+s1}{\PYGZsq{}}\PYG{p}{,}\PYG{l+s+s1}{\PYGZsq{}}\PYG{l+s+s1}{total\PYGZus{}flow}\PYG{l+s+s1}{\PYGZsq{}}\PYG{p}{]}\PYG{p}{,} \PYG{l+s+s1}{\PYGZsq{}}\PYG{l+s+s1}{Wat\PYGZus{}2}\PYG{l+s+s1}{\PYGZsq{}}\PYG{p}{:}\PYG{p}{[}\PYG{l+s+s1}{\PYGZsq{}}\PYG{l+s+s1}{flowrate}\PYG{l+s+s1}{\PYGZsq{}}\PYG{p}{,}\PYG{l+s+s1}{\PYGZsq{}}\PYG{l+s+s1}{pressure}\PYG{l+s+s1}{\PYGZsq{}}\PYG{p}{]}\PYG{p}{,} \PYG{l+s+s1}{\PYGZsq{}}\PYG{l+s+s1}{ImportEE}\PYG{l+s+s1}{\PYGZsq{}}\PYG{p}{:}\PYG{p}{[}\PYG{l+s+s1}{\PYGZsq{}}\PYG{l+s+s1}{effstate}\PYG{l+s+s1}{\PYGZsq{}}\PYG{p}{,} \PYG{l+s+s1}{\PYGZsq{}}\PYG{l+s+s1}{grid\PYGZus{}noise}\PYG{l+s+s1}{\PYGZsq{}}\PYG{p}{]}\PYG{p}{,} \PYG{l+s+s1}{\PYGZsq{}}\PYG{l+s+s1}{EE\PYGZus{}1}\PYG{l+s+s1}{\PYGZsq{}}\PYG{p}{:}\PYG{p}{[}\PYG{l+s+s1}{\PYGZsq{}}\PYG{l+s+s1}{voltage}\PYG{l+s+s1}{\PYGZsq{}}\PYG{p}{]}\PYG{p}{,} \PYG{l+s+s1}{\PYGZsq{}}\PYG{l+s+s1}{Sig\PYGZus{}1}\PYG{l+s+s1}{\PYGZsq{}}\PYG{p}{:}\PYG{p}{[}\PYG{l+s+s1}{\PYGZsq{}}\PYG{l+s+s1}{power}\PYG{l+s+s1}{\PYGZsq{}}\PYG{p}{]}\PYG{p}{\PYGZcb{}}\PYG{p}{,} \PYG{n}{color}\PYG{o}{=}\PYG{l+s+s1}{\PYGZsq{}}\PYG{l+s+s1}{blue}\PYG{l+s+s1}{\PYGZsq{}}\PYG{p}{,} \PYG{n}{alpha}\PYG{o}{=}\PYG{l+m+mf}{0.1}\PYG{p}{,} \PYG{n}{legend\PYGZus{}loc}\PYG{o}{=}\PYG{k+kc}{False}\PYG{p}{)}
\end{sphinxVerbatim}
}

\end{sphinxuseclass}
\begin{sphinxuseclass}{nboutput}
\begin{sphinxuseclass}{nblast}
\hrule height -\fboxrule\relax
\vspace{\nbsphinxcodecellspacing}

\makeatletter\setbox\nbsphinxpromptbox\box\voidb@x\makeatother

\begin{nbsphinxfancyoutput}

\begin{sphinxuseclass}{output_area}
\begin{sphinxuseclass}{}
\noindent\sphinxincludegraphics[width=381\sphinxpxdimen,height=496\sphinxpxdimen]{{example_pump_Stochastic_Modelling_48_0}.png}

\end{sphinxuseclass}
\end{sphinxuseclass}
\end{nbsphinxfancyoutput}

\end{sphinxuseclass}
\end{sphinxuseclass}
\sphinxAtStartPar
Or as percentiles:

\begin{sphinxuseclass}{nbinput}
{
\sphinxsetup{VerbatimColor={named}{nbsphinx-code-bg}}
\sphinxsetup{VerbatimBorderColor={named}{nbsphinx-code-border}}
\begin{sphinxVerbatim}[commandchars=\\\{\}]
\llap{\color{nbsphinxin}[26]:\,\hspace{\fboxrule}\hspace{\fboxsep}}\PYG{n}{fig} \PYG{o}{=} \PYG{n}{rd}\PYG{o}{.}\PYG{n}{plot}\PYG{o}{.}\PYG{n}{mdlhists}\PYG{p}{(}\PYG{n}{mdlhists}\PYG{p}{,} \PYG{p}{\PYGZob{}}\PYG{l+s+s1}{\PYGZsq{}}\PYG{l+s+s1}{MoveWater}\PYG{l+s+s1}{\PYGZsq{}}\PYG{p}{:}\PYG{p}{[}\PYG{l+s+s1}{\PYGZsq{}}\PYG{l+s+s1}{eff}\PYG{l+s+s1}{\PYGZsq{}}\PYG{p}{,}\PYG{l+s+s1}{\PYGZsq{}}\PYG{l+s+s1}{total\PYGZus{}flow}\PYG{l+s+s1}{\PYGZsq{}}\PYG{p}{]}\PYG{p}{,} \PYG{l+s+s1}{\PYGZsq{}}\PYG{l+s+s1}{Wat\PYGZus{}2}\PYG{l+s+s1}{\PYGZsq{}}\PYG{p}{:}\PYG{p}{[}\PYG{l+s+s1}{\PYGZsq{}}\PYG{l+s+s1}{flowrate}\PYG{l+s+s1}{\PYGZsq{}}\PYG{p}{,}\PYG{l+s+s1}{\PYGZsq{}}\PYG{l+s+s1}{pressure}\PYG{l+s+s1}{\PYGZsq{}}\PYG{p}{]}\PYG{p}{,} \PYG{l+s+s1}{\PYGZsq{}}\PYG{l+s+s1}{ImportEE}\PYG{l+s+s1}{\PYGZsq{}}\PYG{p}{:}\PYG{p}{[}\PYG{l+s+s1}{\PYGZsq{}}\PYG{l+s+s1}{effstate}\PYG{l+s+s1}{\PYGZsq{}}\PYG{p}{,} \PYG{l+s+s1}{\PYGZsq{}}\PYG{l+s+s1}{grid\PYGZus{}noise}\PYG{l+s+s1}{\PYGZsq{}}\PYG{p}{]}\PYG{p}{,} \PYG{l+s+s1}{\PYGZsq{}}\PYG{l+s+s1}{EE\PYGZus{}1}\PYG{l+s+s1}{\PYGZsq{}}\PYG{p}{:}\PYG{p}{[}\PYG{l+s+s1}{\PYGZsq{}}\PYG{l+s+s1}{voltage}\PYG{l+s+s1}{\PYGZsq{}}\PYG{p}{]}\PYG{p}{,} \PYG{l+s+s1}{\PYGZsq{}}\PYG{l+s+s1}{Sig\PYGZus{}1}\PYG{l+s+s1}{\PYGZsq{}}\PYG{p}{:}\PYG{p}{[}\PYG{l+s+s1}{\PYGZsq{}}\PYG{l+s+s1}{power}\PYG{l+s+s1}{\PYGZsq{}}\PYG{p}{]}\PYG{p}{\PYGZcb{}}\PYG{p}{,} \PYG{n}{aggregation}\PYG{o}{=}\PYG{l+s+s1}{\PYGZsq{}}\PYG{l+s+s1}{percentile}\PYG{l+s+s1}{\PYGZsq{}}\PYG{p}{)}
\end{sphinxVerbatim}
}

\end{sphinxuseclass}
\begin{sphinxuseclass}{nboutput}
\begin{sphinxuseclass}{nblast}
\hrule height -\fboxrule\relax
\vspace{\nbsphinxcodecellspacing}

\makeatletter\setbox\nbsphinxpromptbox\box\voidb@x\makeatother

\begin{nbsphinxfancyoutput}

\begin{sphinxuseclass}{output_area}
\begin{sphinxuseclass}{}
\noindent\sphinxincludegraphics[width=381\sphinxpxdimen,height=496\sphinxpxdimen]{{example_pump_Stochastic_Modelling_50_0}.png}

\end{sphinxuseclass}
\end{sphinxuseclass}
\end{nbsphinxfancyoutput}

\end{sphinxuseclass}
\end{sphinxuseclass}
\sphinxAtStartPar
Or as a mean with confidence interval:

\begin{sphinxuseclass}{nbinput}
{
\sphinxsetup{VerbatimColor={named}{nbsphinx-code-bg}}
\sphinxsetup{VerbatimBorderColor={named}{nbsphinx-code-border}}
\begin{sphinxVerbatim}[commandchars=\\\{\}]
\llap{\color{nbsphinxin}[27]:\,\hspace{\fboxrule}\hspace{\fboxsep}}\PYG{n}{fig} \PYG{o}{=} \PYG{n}{rd}\PYG{o}{.}\PYG{n}{plot}\PYG{o}{.}\PYG{n}{mdlhists}\PYG{p}{(}\PYG{n}{mdlhists}\PYG{p}{,} \PYG{p}{\PYGZob{}}\PYG{l+s+s1}{\PYGZsq{}}\PYG{l+s+s1}{MoveWater}\PYG{l+s+s1}{\PYGZsq{}}\PYG{p}{:}\PYG{p}{[}\PYG{l+s+s1}{\PYGZsq{}}\PYG{l+s+s1}{eff}\PYG{l+s+s1}{\PYGZsq{}}\PYG{p}{,}\PYG{l+s+s1}{\PYGZsq{}}\PYG{l+s+s1}{total\PYGZus{}flow}\PYG{l+s+s1}{\PYGZsq{}}\PYG{p}{]}\PYG{p}{,} \PYG{l+s+s1}{\PYGZsq{}}\PYG{l+s+s1}{Wat\PYGZus{}2}\PYG{l+s+s1}{\PYGZsq{}}\PYG{p}{:}\PYG{p}{[}\PYG{l+s+s1}{\PYGZsq{}}\PYG{l+s+s1}{flowrate}\PYG{l+s+s1}{\PYGZsq{}}\PYG{p}{,}\PYG{l+s+s1}{\PYGZsq{}}\PYG{l+s+s1}{pressure}\PYG{l+s+s1}{\PYGZsq{}}\PYG{p}{]}\PYG{p}{,} \PYG{l+s+s1}{\PYGZsq{}}\PYG{l+s+s1}{ImportEE}\PYG{l+s+s1}{\PYGZsq{}}\PYG{p}{:}\PYG{p}{[}\PYG{l+s+s1}{\PYGZsq{}}\PYG{l+s+s1}{effstate}\PYG{l+s+s1}{\PYGZsq{}}\PYG{p}{,} \PYG{l+s+s1}{\PYGZsq{}}\PYG{l+s+s1}{grid\PYGZus{}noise}\PYG{l+s+s1}{\PYGZsq{}}\PYG{p}{]}\PYG{p}{,} \PYG{l+s+s1}{\PYGZsq{}}\PYG{l+s+s1}{EE\PYGZus{}1}\PYG{l+s+s1}{\PYGZsq{}}\PYG{p}{:}\PYG{p}{[}\PYG{l+s+s1}{\PYGZsq{}}\PYG{l+s+s1}{voltage}\PYG{l+s+s1}{\PYGZsq{}}\PYG{p}{]}\PYG{p}{,} \PYG{l+s+s1}{\PYGZsq{}}\PYG{l+s+s1}{Sig\PYGZus{}1}\PYG{l+s+s1}{\PYGZsq{}}\PYG{p}{:}\PYG{p}{[}\PYG{l+s+s1}{\PYGZsq{}}\PYG{l+s+s1}{power}\PYG{l+s+s1}{\PYGZsq{}}\PYG{p}{]}\PYG{p}{\PYGZcb{}}\PYG{p}{,} \PYG{n}{aggregation}\PYG{o}{=}\PYG{l+s+s1}{\PYGZsq{}}\PYG{l+s+s1}{mean\PYGZus{}ci}\PYG{l+s+s1}{\PYGZsq{}}\PYG{p}{)}
\end{sphinxVerbatim}
}

\end{sphinxuseclass}
\begin{sphinxuseclass}{nboutput}
\begin{sphinxuseclass}{nblast}
\hrule height -\fboxrule\relax
\vspace{\nbsphinxcodecellspacing}

\makeatletter\setbox\nbsphinxpromptbox\box\voidb@x\makeatother

\begin{nbsphinxfancyoutput}

\begin{sphinxuseclass}{output_area}
\begin{sphinxuseclass}{}
\noindent\sphinxincludegraphics[width=378\sphinxpxdimen,height=496\sphinxpxdimen]{{example_pump_Stochastic_Modelling_52_0}.png}

\end{sphinxuseclass}
\end{sphinxuseclass}
\end{nbsphinxfancyoutput}

\end{sphinxuseclass}
\end{sphinxuseclass}

\subsection{Nested Fault Simulation}
\label{\detokenize{example_pump/Stochastic_Modelling:Nested-Fault-Simulation}}
\sphinxAtStartPar
We can also compare stochastic output over a set of scenarios using \sphinxcode{\sphinxupquote{plot.mdlhistvals}} with the \sphinxcode{\sphinxupquote{comp\_groups}} parameter, which places the scenarios in different groups.

\begin{sphinxuseclass}{nbinput}
\begin{sphinxuseclass}{nblast}
{
\sphinxsetup{VerbatimColor={named}{nbsphinx-code-bg}}
\sphinxsetup{VerbatimBorderColor={named}{nbsphinx-code-border}}
\begin{sphinxVerbatim}[commandchars=\\\{\}]
\llap{\color{nbsphinxin}[28]:\,\hspace{\fboxrule}\hspace{\fboxsep}}\PYG{k}{def} \PYG{n+nf}{paramfunc}\PYG{p}{(}\PYG{n}{delay}\PYG{o}{=}\PYG{l+m+mi}{1}\PYG{p}{)}\PYG{p}{:}
    \PYG{k}{return} \PYG{p}{\PYGZob{}}\PYG{l+s+s1}{\PYGZsq{}}\PYG{l+s+s1}{delay}\PYG{l+s+s1}{\PYGZsq{}}\PYG{p}{:}\PYG{n}{delay}\PYG{p}{\PYGZcb{}}
\end{sphinxVerbatim}
}

\end{sphinxuseclass}
\end{sphinxuseclass}
\sphinxAtStartPar
This can be done by first creating an approach with two parameters we wish to compare. The pump still has one real parameter–the fault delay.

\begin{sphinxuseclass}{nbinput}
\begin{sphinxuseclass}{nblast}
{
\sphinxsetup{VerbatimColor={named}{nbsphinx-code-bg}}
\sphinxsetup{VerbatimBorderColor={named}{nbsphinx-code-border}}
\begin{sphinxVerbatim}[commandchars=\\\{\}]
\llap{\color{nbsphinxin}[29]:\,\hspace{\fboxrule}\hspace{\fboxsep}}\PYG{n}{app\PYGZus{}comp} \PYG{o}{=} \PYG{n}{NominalApproach}\PYG{p}{(}\PYG{p}{)}
\PYG{n}{app\PYGZus{}comp}\PYG{o}{.}\PYG{n}{add\PYGZus{}param\PYGZus{}replicates}\PYG{p}{(}\PYG{n}{paramfunc}\PYG{p}{,} \PYG{l+s+s1}{\PYGZsq{}}\PYG{l+s+s1}{delay\PYGZus{}1}\PYG{l+s+s1}{\PYGZsq{}}\PYG{p}{,} \PYG{l+m+mi}{100}\PYG{p}{,} \PYG{n}{delay}\PYG{o}{=}\PYG{l+m+mi}{1}\PYG{p}{)}
\PYG{n}{app\PYGZus{}comp}\PYG{o}{.}\PYG{n}{add\PYGZus{}param\PYGZus{}replicates}\PYG{p}{(}\PYG{n}{paramfunc}\PYG{p}{,} \PYG{l+s+s1}{\PYGZsq{}}\PYG{l+s+s1}{delay\PYGZus{}10}\PYG{l+s+s1}{\PYGZsq{}}\PYG{p}{,} \PYG{l+m+mi}{100}\PYG{p}{,} \PYG{n}{delay}\PYG{o}{=}\PYG{l+m+mi}{10}\PYG{p}{)}
\end{sphinxVerbatim}
}

\end{sphinxuseclass}
\end{sphinxuseclass}
\sphinxAtStartPar
Since this delay only shows up in blockage fault modes, to compare the behaviors, we will first simulate it in a nested approach with only the blockage fault added.

\begin{sphinxuseclass}{nbinput}
\begin{sphinxuseclass}{nblast}
{
\sphinxsetup{VerbatimColor={named}{nbsphinx-code-bg}}
\sphinxsetup{VerbatimBorderColor={named}{nbsphinx-code-border}}
\begin{sphinxVerbatim}[commandchars=\\\{\}]
\llap{\color{nbsphinxin}[30]:\,\hspace{\fboxrule}\hspace{\fboxsep}}\PYG{k+kn}{import} \PYG{n+nn}{multiprocessing} \PYG{k}{as} \PYG{n+nn}{mp}
\end{sphinxVerbatim}
}

\end{sphinxuseclass}
\end{sphinxuseclass}
\begin{sphinxuseclass}{nbinput}
{
\sphinxsetup{VerbatimColor={named}{nbsphinx-code-bg}}
\sphinxsetup{VerbatimBorderColor={named}{nbsphinx-code-border}}
\begin{sphinxVerbatim}[commandchars=\\\{\}]
\llap{\color{nbsphinxin}[31]:\,\hspace{\fboxrule}\hspace{\fboxsep}}\PYG{n}{endclasses}\PYG{p}{,} \PYG{n}{mdlhists}\PYG{o}{=}\PYG{n}{propagate}\PYG{o}{.}\PYG{n}{nested\PYGZus{}approach}\PYG{p}{(}\PYG{n}{mdl}\PYG{p}{,}\PYG{n}{app\PYGZus{}comp}\PYG{p}{,} \PYG{n}{run\PYGZus{}stochastic}\PYG{o}{=}\PYG{k+kc}{True}\PYG{p}{,} \PYG{n}{faults}\PYG{o}{=}\PYG{p}{[}\PYG{p}{(}\PYG{l+s+s1}{\PYGZsq{}}\PYG{l+s+s1}{ExportWater}\PYG{l+s+s1}{\PYGZsq{}}\PYG{p}{,}\PYG{l+s+s1}{\PYGZsq{}}\PYG{l+s+s1}{block}\PYG{l+s+s1}{\PYGZsq{}}\PYG{p}{)}\PYG{p}{]}\PYG{p}{,} \PYG{n}{pool}\PYG{o}{=}\PYG{n}{mp}\PYG{o}{.}\PYG{n}{Pool}\PYG{p}{(}\PYG{l+m+mi}{4}\PYG{p}{)}\PYG{p}{)}
\end{sphinxVerbatim}
}

\end{sphinxuseclass}
\begin{sphinxuseclass}{nboutput}
{

\kern-\sphinxverbatimsmallskipamount\kern-\baselineskip
\kern+\FrameHeightAdjust\kern-\fboxrule
\vspace{\nbsphinxcodecellspacing}

\sphinxsetup{VerbatimColor={named}{nbsphinx-stderr}}
\sphinxsetup{VerbatimBorderColor={named}{nbsphinx-code-border}}
\begin{sphinxuseclass}{output_area}
\begin{sphinxuseclass}{stderr}


\begin{sphinxVerbatim}[commandchars=\\\{\}]
NESTED SCENARIOS COMPLETE:   0\%|                                                               | 0/200 [00:00<?, ?it/s]
\end{sphinxVerbatim}



\end{sphinxuseclass}
\end{sphinxuseclass}
}

\end{sphinxuseclass}
\begin{sphinxuseclass}{nboutput}
{

\kern-\sphinxverbatimsmallskipamount\kern-\baselineskip
\kern+\FrameHeightAdjust\kern-\fboxrule
\vspace{\nbsphinxcodecellspacing}

\sphinxsetup{VerbatimColor={named}{white}}
\sphinxsetup{VerbatimBorderColor={named}{nbsphinx-code-border}}
\begin{sphinxuseclass}{output_area}
\begin{sphinxuseclass}{}


\begin{sphinxVerbatim}[commandchars=\\\{\}]
Faults found during the nominal run \{'MoveWater': ['mech\_break']\}
\end{sphinxVerbatim}



\end{sphinxuseclass}
\end{sphinxuseclass}
}

\end{sphinxuseclass}
\begin{sphinxuseclass}{nboutput}
{

\kern-\sphinxverbatimsmallskipamount\kern-\baselineskip
\kern+\FrameHeightAdjust\kern-\fboxrule
\vspace{\nbsphinxcodecellspacing}

\sphinxsetup{VerbatimColor={named}{nbsphinx-stderr}}
\sphinxsetup{VerbatimBorderColor={named}{nbsphinx-code-border}}
\begin{sphinxuseclass}{output_area}
\begin{sphinxuseclass}{stderr}


\begin{sphinxVerbatim}[commandchars=\\\{\}]
NESTED SCENARIOS COMPLETE:   0\%|▎                                                      | 1/200 [00:01<05:29,  1.65s/it]
\end{sphinxVerbatim}



\end{sphinxuseclass}
\end{sphinxuseclass}
}

\end{sphinxuseclass}
\begin{sphinxuseclass}{nboutput}
{

\kern-\sphinxverbatimsmallskipamount\kern-\baselineskip
\kern+\FrameHeightAdjust\kern-\fboxrule
\vspace{\nbsphinxcodecellspacing}

\sphinxsetup{VerbatimColor={named}{white}}
\sphinxsetup{VerbatimBorderColor={named}{nbsphinx-code-border}}
\begin{sphinxuseclass}{output_area}
\begin{sphinxuseclass}{}


\begin{sphinxVerbatim}[commandchars=\\\{\}]
Faults found during the nominal run \{'MoveWater': ['mech\_break']\}
\end{sphinxVerbatim}



\end{sphinxuseclass}
\end{sphinxuseclass}
}

\end{sphinxuseclass}
\begin{sphinxuseclass}{nboutput}
{

\kern-\sphinxverbatimsmallskipamount\kern-\baselineskip
\kern+\FrameHeightAdjust\kern-\fboxrule
\vspace{\nbsphinxcodecellspacing}

\sphinxsetup{VerbatimColor={named}{nbsphinx-stderr}}
\sphinxsetup{VerbatimBorderColor={named}{nbsphinx-code-border}}
\begin{sphinxuseclass}{output_area}
\begin{sphinxuseclass}{stderr}


\begin{sphinxVerbatim}[commandchars=\\\{\}]
NESTED SCENARIOS COMPLETE:   6\%|███▏                                                  | 12/200 [00:03<00:28,  6.50it/s]
\end{sphinxVerbatim}



\end{sphinxuseclass}
\end{sphinxuseclass}
}

\end{sphinxuseclass}
\begin{sphinxuseclass}{nboutput}
{

\kern-\sphinxverbatimsmallskipamount\kern-\baselineskip
\kern+\FrameHeightAdjust\kern-\fboxrule
\vspace{\nbsphinxcodecellspacing}

\sphinxsetup{VerbatimColor={named}{white}}
\sphinxsetup{VerbatimBorderColor={named}{nbsphinx-code-border}}
\begin{sphinxuseclass}{output_area}
\begin{sphinxuseclass}{}


\begin{sphinxVerbatim}[commandchars=\\\{\}]
Faults found during the nominal run \{'MoveWater': ['mech\_break']\}
Faults found during the nominal run \{'MoveWater': ['mech\_break']\}
Faults found during the nominal run \{'MoveWater': ['mech\_break']\}
Faults found during the nominal run \{'MoveWater': ['mech\_break']\}
Faults found during the nominal run \{'MoveWater': ['mech\_break']\}
Faults found during the nominal run \{'MoveWater': ['mech\_break']\}
Faults found during the nominal run \{'MoveWater': ['mech\_break']\}
\end{sphinxVerbatim}



\end{sphinxuseclass}
\end{sphinxuseclass}
}

\end{sphinxuseclass}
\begin{sphinxuseclass}{nboutput}
{

\kern-\sphinxverbatimsmallskipamount\kern-\baselineskip
\kern+\FrameHeightAdjust\kern-\fboxrule
\vspace{\nbsphinxcodecellspacing}

\sphinxsetup{VerbatimColor={named}{nbsphinx-stderr}}
\sphinxsetup{VerbatimBorderColor={named}{nbsphinx-code-border}}
\begin{sphinxuseclass}{output_area}
\begin{sphinxuseclass}{stderr}


\begin{sphinxVerbatim}[commandchars=\\\{\}]
NESTED SCENARIOS COMPLETE:  11\%|█████▉                                                | 22/200 [00:03<00:12, 14.36it/s]
\end{sphinxVerbatim}



\end{sphinxuseclass}
\end{sphinxuseclass}
}

\end{sphinxuseclass}
\begin{sphinxuseclass}{nboutput}
{

\kern-\sphinxverbatimsmallskipamount\kern-\baselineskip
\kern+\FrameHeightAdjust\kern-\fboxrule
\vspace{\nbsphinxcodecellspacing}

\sphinxsetup{VerbatimColor={named}{white}}
\sphinxsetup{VerbatimBorderColor={named}{nbsphinx-code-border}}
\begin{sphinxuseclass}{output_area}
\begin{sphinxuseclass}{}


\begin{sphinxVerbatim}[commandchars=\\\{\}]
Faults found during the nominal run \{'MoveWater': ['mech\_break']\}
Faults found during the nominal run \{'MoveWater': ['mech\_break']\}
Faults found during the nominal run \{'MoveWater': ['mech\_break']\}
Faults found during the nominal run \{'MoveWater': ['mech\_break']\}
Faults found during the nominal run \{'MoveWater': ['mech\_break']\}
Faults found during the nominal run \{'MoveWater': ['mech\_break']\}
\end{sphinxVerbatim}



\end{sphinxuseclass}
\end{sphinxuseclass}
}

\end{sphinxuseclass}
\begin{sphinxuseclass}{nboutput}
{

\kern-\sphinxverbatimsmallskipamount\kern-\baselineskip
\kern+\FrameHeightAdjust\kern-\fboxrule
\vspace{\nbsphinxcodecellspacing}

\sphinxsetup{VerbatimColor={named}{nbsphinx-stderr}}
\sphinxsetup{VerbatimBorderColor={named}{nbsphinx-code-border}}
\begin{sphinxuseclass}{output_area}
\begin{sphinxuseclass}{stderr}


\begin{sphinxVerbatim}[commandchars=\\\{\}]
NESTED SCENARIOS COMPLETE:  14\%|███████▎                                              | 27/200 [00:03<00:09, 18.79it/s]
\end{sphinxVerbatim}



\end{sphinxuseclass}
\end{sphinxuseclass}
}

\end{sphinxuseclass}
\begin{sphinxuseclass}{nboutput}
{

\kern-\sphinxverbatimsmallskipamount\kern-\baselineskip
\kern+\FrameHeightAdjust\kern-\fboxrule
\vspace{\nbsphinxcodecellspacing}

\sphinxsetup{VerbatimColor={named}{white}}
\sphinxsetup{VerbatimBorderColor={named}{nbsphinx-code-border}}
\begin{sphinxuseclass}{output_area}
\begin{sphinxuseclass}{}


\begin{sphinxVerbatim}[commandchars=\\\{\}]
Faults found during the nominal run \{'MoveWater': ['mech\_break']\}
Faults found during the nominal run \{'MoveWater': ['mech\_break']\}
Faults found during the nominal run \{'MoveWater': ['mech\_break']\}
Faults found during the nominal run \{'MoveWater': ['mech\_break']\}
Faults found during the nominal run \{'MoveWater': ['mech\_break']\}
Faults found during the nominal run \{'MoveWater': ['mech\_break']\}
Faults found during the nominal run \{'MoveWater': ['mech\_break']\}
Faults found during the nominal run \{'MoveWater': ['mech\_break']\}
\end{sphinxVerbatim}



\end{sphinxuseclass}
\end{sphinxuseclass}
}

\end{sphinxuseclass}
\begin{sphinxuseclass}{nboutput}
{

\kern-\sphinxverbatimsmallskipamount\kern-\baselineskip
\kern+\FrameHeightAdjust\kern-\fboxrule
\vspace{\nbsphinxcodecellspacing}

\sphinxsetup{VerbatimColor={named}{nbsphinx-stderr}}
\sphinxsetup{VerbatimBorderColor={named}{nbsphinx-code-border}}
\begin{sphinxuseclass}{output_area}
\begin{sphinxuseclass}{stderr}


\begin{sphinxVerbatim}[commandchars=\\\{\}]
NESTED SCENARIOS COMPLETE:  21\%|███████████▎                                          | 42/200 [00:03<00:05, 30.60it/s]
\end{sphinxVerbatim}



\end{sphinxuseclass}
\end{sphinxuseclass}
}

\end{sphinxuseclass}
\begin{sphinxuseclass}{nboutput}
{

\kern-\sphinxverbatimsmallskipamount\kern-\baselineskip
\kern+\FrameHeightAdjust\kern-\fboxrule
\vspace{\nbsphinxcodecellspacing}

\sphinxsetup{VerbatimColor={named}{white}}
\sphinxsetup{VerbatimBorderColor={named}{nbsphinx-code-border}}
\begin{sphinxuseclass}{output_area}
\begin{sphinxuseclass}{}


\begin{sphinxVerbatim}[commandchars=\\\{\}]
Faults found during the nominal run \{'MoveWater': ['mech\_break']\}
Faults found during the nominal run \{'MoveWater': ['mech\_break']\}
Faults found during the nominal run \{'MoveWater': ['mech\_break']\}
Faults found during the nominal run \{'MoveWater': ['mech\_break']\}
Faults found during the nominal run \{'MoveWater': ['mech\_break']\}
Faults found during the nominal run \{'MoveWater': ['mech\_break']\}
\end{sphinxVerbatim}



\end{sphinxuseclass}
\end{sphinxuseclass}
}

\end{sphinxuseclass}
\begin{sphinxuseclass}{nboutput}
{

\kern-\sphinxverbatimsmallskipamount\kern-\baselineskip
\kern+\FrameHeightAdjust\kern-\fboxrule
\vspace{\nbsphinxcodecellspacing}

\sphinxsetup{VerbatimColor={named}{nbsphinx-stderr}}
\sphinxsetup{VerbatimBorderColor={named}{nbsphinx-code-border}}
\begin{sphinxuseclass}{output_area}
\begin{sphinxuseclass}{stderr}


\begin{sphinxVerbatim}[commandchars=\\\{\}]
NESTED SCENARIOS COMPLETE:  24\%|████████████▋                                         | 47/200 [00:04<00:04, 33.18it/s]
\end{sphinxVerbatim}



\end{sphinxuseclass}
\end{sphinxuseclass}
}

\end{sphinxuseclass}
\begin{sphinxuseclass}{nboutput}
{

\kern-\sphinxverbatimsmallskipamount\kern-\baselineskip
\kern+\FrameHeightAdjust\kern-\fboxrule
\vspace{\nbsphinxcodecellspacing}

\sphinxsetup{VerbatimColor={named}{white}}
\sphinxsetup{VerbatimBorderColor={named}{nbsphinx-code-border}}
\begin{sphinxuseclass}{output_area}
\begin{sphinxuseclass}{}


\begin{sphinxVerbatim}[commandchars=\\\{\}]
Faults found during the nominal run \{'MoveWater': ['mech\_break']\}
Faults found during the nominal run \{'MoveWater': ['mech\_break']\}
Faults found during the nominal run \{'MoveWater': ['mech\_break']\}
Faults found during the nominal run \{'MoveWater': ['mech\_break']\}
Faults found during the nominal run \{'MoveWater': ['mech\_break']\}
Faults found during the nominal run \{'MoveWater': ['mech\_break']\}
Faults found during the nominal run \{'MoveWater': ['mech\_break']\}
Faults found during the nominal run \{'MoveWater': ['mech\_break']\}
Faults found during the nominal run \{'MoveWater': ['mech\_break']\}
\end{sphinxVerbatim}



\end{sphinxuseclass}
\end{sphinxuseclass}
}

\end{sphinxuseclass}
\begin{sphinxuseclass}{nboutput}
{

\kern-\sphinxverbatimsmallskipamount\kern-\baselineskip
\kern+\FrameHeightAdjust\kern-\fboxrule
\vspace{\nbsphinxcodecellspacing}

\sphinxsetup{VerbatimColor={named}{nbsphinx-stderr}}
\sphinxsetup{VerbatimBorderColor={named}{nbsphinx-code-border}}
\begin{sphinxuseclass}{output_area}
\begin{sphinxuseclass}{stderr}


\begin{sphinxVerbatim}[commandchars=\\\{\}]
NESTED SCENARIOS COMPLETE:  28\%|███████████████▍                                      | 57/200 [00:04<00:03, 36.89it/s]
\end{sphinxVerbatim}



\end{sphinxuseclass}
\end{sphinxuseclass}
}

\end{sphinxuseclass}
\begin{sphinxuseclass}{nboutput}
{

\kern-\sphinxverbatimsmallskipamount\kern-\baselineskip
\kern+\FrameHeightAdjust\kern-\fboxrule
\vspace{\nbsphinxcodecellspacing}

\sphinxsetup{VerbatimColor={named}{white}}
\sphinxsetup{VerbatimBorderColor={named}{nbsphinx-code-border}}
\begin{sphinxuseclass}{output_area}
\begin{sphinxuseclass}{}


\begin{sphinxVerbatim}[commandchars=\\\{\}]
Faults found during the nominal run \{'MoveWater': ['mech\_break']\}
Faults found during the nominal run \{'MoveWater': ['mech\_break']\}
Faults found during the nominal run \{'MoveWater': ['mech\_break']\}
Faults found during the nominal run \{'MoveWater': ['mech\_break']\}
Faults found during the nominal run \{'MoveWater': ['mech\_break']\}
\end{sphinxVerbatim}



\end{sphinxuseclass}
\end{sphinxuseclass}
}

\end{sphinxuseclass}
\begin{sphinxuseclass}{nboutput}
{

\kern-\sphinxverbatimsmallskipamount\kern-\baselineskip
\kern+\FrameHeightAdjust\kern-\fboxrule
\vspace{\nbsphinxcodecellspacing}

\sphinxsetup{VerbatimColor={named}{nbsphinx-stderr}}
\sphinxsetup{VerbatimBorderColor={named}{nbsphinx-code-border}}
\begin{sphinxuseclass}{output_area}
\begin{sphinxuseclass}{stderr}


\begin{sphinxVerbatim}[commandchars=\\\{\}]
NESTED SCENARIOS COMPLETE:  34\%|██████████████████                                    | 67/200 [00:04<00:03, 38.89it/s]
\end{sphinxVerbatim}



\end{sphinxuseclass}
\end{sphinxuseclass}
}

\end{sphinxuseclass}
\begin{sphinxuseclass}{nboutput}
{

\kern-\sphinxverbatimsmallskipamount\kern-\baselineskip
\kern+\FrameHeightAdjust\kern-\fboxrule
\vspace{\nbsphinxcodecellspacing}

\sphinxsetup{VerbatimColor={named}{white}}
\sphinxsetup{VerbatimBorderColor={named}{nbsphinx-code-border}}
\begin{sphinxuseclass}{output_area}
\begin{sphinxuseclass}{}


\begin{sphinxVerbatim}[commandchars=\\\{\}]
Faults found during the nominal run \{'MoveWater': ['mech\_break']\}
Faults found during the nominal run \{'MoveWater': ['mech\_break']\}
Faults found during the nominal run \{'MoveWater': ['mech\_break']\}
Faults found during the nominal run \{'MoveWater': ['mech\_break']\}
Faults found during the nominal run \{'MoveWater': ['mech\_break']\}
Faults found during the nominal run \{'MoveWater': ['mech\_break']\}
Faults found during the nominal run \{'MoveWater': ['mech\_break']\}
Faults found during the nominal run \{'MoveWater': ['mech\_break']\}
\end{sphinxVerbatim}



\end{sphinxuseclass}
\end{sphinxuseclass}
}

\end{sphinxuseclass}
\begin{sphinxuseclass}{nboutput}
{

\kern-\sphinxverbatimsmallskipamount\kern-\baselineskip
\kern+\FrameHeightAdjust\kern-\fboxrule
\vspace{\nbsphinxcodecellspacing}

\sphinxsetup{VerbatimColor={named}{nbsphinx-stderr}}
\sphinxsetup{VerbatimBorderColor={named}{nbsphinx-code-border}}
\begin{sphinxuseclass}{output_area}
\begin{sphinxuseclass}{stderr}


\begin{sphinxVerbatim}[commandchars=\\\{\}]
NESTED SCENARIOS COMPLETE:  38\%|████████████████████▊                                 | 77/200 [00:04<00:03, 39.99it/s]
\end{sphinxVerbatim}



\end{sphinxuseclass}
\end{sphinxuseclass}
}

\end{sphinxuseclass}
\begin{sphinxuseclass}{nboutput}
{

\kern-\sphinxverbatimsmallskipamount\kern-\baselineskip
\kern+\FrameHeightAdjust\kern-\fboxrule
\vspace{\nbsphinxcodecellspacing}

\sphinxsetup{VerbatimColor={named}{white}}
\sphinxsetup{VerbatimBorderColor={named}{nbsphinx-code-border}}
\begin{sphinxuseclass}{output_area}
\begin{sphinxuseclass}{}


\begin{sphinxVerbatim}[commandchars=\\\{\}]
Faults found during the nominal run \{'MoveWater': ['mech\_break']\}
Faults found during the nominal run \{'MoveWater': ['mech\_break']\}
Faults found during the nominal run \{'MoveWater': ['mech\_break']\}
Faults found during the nominal run \{'MoveWater': ['mech\_break']\}
Faults found during the nominal run \{'MoveWater': ['mech\_break']\}
Faults found during the nominal run \{'MoveWater': ['mech\_break']\}
Faults found during the nominal run \{'MoveWater': ['mech\_break']\}
Faults found during the nominal run \{'MoveWater': ['mech\_break']\}
\end{sphinxVerbatim}



\end{sphinxuseclass}
\end{sphinxuseclass}
}

\end{sphinxuseclass}
\begin{sphinxuseclass}{nboutput}
{

\kern-\sphinxverbatimsmallskipamount\kern-\baselineskip
\kern+\FrameHeightAdjust\kern-\fboxrule
\vspace{\nbsphinxcodecellspacing}

\sphinxsetup{VerbatimColor={named}{nbsphinx-stderr}}
\sphinxsetup{VerbatimBorderColor={named}{nbsphinx-code-border}}
\begin{sphinxuseclass}{output_area}
\begin{sphinxuseclass}{stderr}


\begin{sphinxVerbatim}[commandchars=\\\{\}]
NESTED SCENARIOS COMPLETE:  44\%|███████████████████████▍                              | 87/200 [00:05<00:02, 40.85it/s]
\end{sphinxVerbatim}



\end{sphinxuseclass}
\end{sphinxuseclass}
}

\end{sphinxuseclass}
\begin{sphinxuseclass}{nboutput}
{

\kern-\sphinxverbatimsmallskipamount\kern-\baselineskip
\kern+\FrameHeightAdjust\kern-\fboxrule
\vspace{\nbsphinxcodecellspacing}

\sphinxsetup{VerbatimColor={named}{white}}
\sphinxsetup{VerbatimBorderColor={named}{nbsphinx-code-border}}
\begin{sphinxuseclass}{output_area}
\begin{sphinxuseclass}{}


\begin{sphinxVerbatim}[commandchars=\\\{\}]
Faults found during the nominal run \{'MoveWater': ['mech\_break']\}
Faults found during the nominal run \{'MoveWater': ['mech\_break']\}
Faults found during the nominal run \{'MoveWater': ['mech\_break']\}
Faults found during the nominal run \{'MoveWater': ['mech\_break']\}
Faults found during the nominal run \{'MoveWater': ['mech\_break']\}
Faults found during the nominal run \{'MoveWater': ['mech\_break']\}
\end{sphinxVerbatim}



\end{sphinxuseclass}
\end{sphinxuseclass}
}

\end{sphinxuseclass}
\begin{sphinxuseclass}{nboutput}
{

\kern-\sphinxverbatimsmallskipamount\kern-\baselineskip
\kern+\FrameHeightAdjust\kern-\fboxrule
\vspace{\nbsphinxcodecellspacing}

\sphinxsetup{VerbatimColor={named}{nbsphinx-stderr}}
\sphinxsetup{VerbatimBorderColor={named}{nbsphinx-code-border}}
\begin{sphinxuseclass}{output_area}
\begin{sphinxuseclass}{stderr}


\begin{sphinxVerbatim}[commandchars=\\\{\}]
NESTED SCENARIOS COMPLETE:  48\%|██████████████████████████▏                           | 97/200 [00:05<00:02, 41.51it/s]
\end{sphinxVerbatim}



\end{sphinxuseclass}
\end{sphinxuseclass}
}

\end{sphinxuseclass}
\begin{sphinxuseclass}{nboutput}
{

\kern-\sphinxverbatimsmallskipamount\kern-\baselineskip
\kern+\FrameHeightAdjust\kern-\fboxrule
\vspace{\nbsphinxcodecellspacing}

\sphinxsetup{VerbatimColor={named}{white}}
\sphinxsetup{VerbatimBorderColor={named}{nbsphinx-code-border}}
\begin{sphinxuseclass}{output_area}
\begin{sphinxuseclass}{}


\begin{sphinxVerbatim}[commandchars=\\\{\}]
Faults found during the nominal run \{'MoveWater': ['mech\_break']\}
Faults found during the nominal run \{'MoveWater': ['mech\_break']\}
Faults found during the nominal run \{'MoveWater': ['mech\_break']\}
Faults found during the nominal run \{'MoveWater': ['mech\_break']\}
Faults found during the nominal run \{'MoveWater': ['mech\_break']\}
Faults found during the nominal run \{'MoveWater': ['mech\_break']\}
Faults found during the nominal run \{'MoveWater': ['mech\_break']\}
\end{sphinxVerbatim}



\end{sphinxuseclass}
\end{sphinxuseclass}
}

\end{sphinxuseclass}
\begin{sphinxuseclass}{nboutput}
\begin{sphinxuseclass}{nblast}
{

\kern-\sphinxverbatimsmallskipamount\kern-\baselineskip
\kern+\FrameHeightAdjust\kern-\fboxrule
\vspace{\nbsphinxcodecellspacing}

\sphinxsetup{VerbatimColor={named}{nbsphinx-stderr}}
\sphinxsetup{VerbatimBorderColor={named}{nbsphinx-code-border}}
\begin{sphinxuseclass}{output_area}
\begin{sphinxuseclass}{stderr}


\begin{sphinxVerbatim}[commandchars=\\\{\}]
NESTED SCENARIOS COMPLETE: 100\%|█████████████████████████████████████████████████████| 200/200 [00:07<00:00, 25.70it/s]
\end{sphinxVerbatim}



\end{sphinxuseclass}
\end{sphinxuseclass}
}

\end{sphinxuseclass}
\end{sphinxuseclass}
\sphinxAtStartPar
Next, we get just the scenarios of interest:

\begin{sphinxuseclass}{nbinput}
\begin{sphinxuseclass}{nblast}
{
\sphinxsetup{VerbatimColor={named}{nbsphinx-code-bg}}
\sphinxsetup{VerbatimBorderColor={named}{nbsphinx-code-border}}
\begin{sphinxVerbatim}[commandchars=\\\{\}]
\llap{\color{nbsphinxin}[32]:\,\hspace{\fboxrule}\hspace{\fboxsep}}\PYG{n}{comp\PYGZus{}mdlhists} \PYG{o}{=} \PYG{p}{\PYGZob{}}\PYG{n}{scen}\PYG{p}{:}\PYG{n}{mdlhist}\PYG{p}{[}\PYG{l+s+s1}{\PYGZsq{}}\PYG{l+s+s1}{ExportWater block, t=27}\PYG{l+s+s1}{\PYGZsq{}}\PYG{p}{]} \PYG{k}{for} \PYG{n}{scen}\PYG{p}{,}\PYG{n}{mdlhist} \PYG{o+ow}{in} \PYG{n}{mdlhists}\PYG{o}{.}\PYG{n}{items}\PYG{p}{(}\PYG{p}{)}\PYG{p}{\PYGZcb{}}
\end{sphinxVerbatim}
}

\end{sphinxuseclass}
\end{sphinxuseclass}
\sphinxAtStartPar
Finally, we create some comparison groups to group the results by:

\begin{sphinxuseclass}{nbinput}
\begin{sphinxuseclass}{nblast}
{
\sphinxsetup{VerbatimColor={named}{nbsphinx-code-bg}}
\sphinxsetup{VerbatimBorderColor={named}{nbsphinx-code-border}}
\begin{sphinxVerbatim}[commandchars=\\\{\}]
\llap{\color{nbsphinxin}[33]:\,\hspace{\fboxrule}\hspace{\fboxsep}}\PYG{n}{comp\PYGZus{}groups} \PYG{o}{=} \PYG{p}{\PYGZob{}}\PYG{l+s+s1}{\PYGZsq{}}\PYG{l+s+s1}{delay\PYGZus{}1}\PYG{l+s+s1}{\PYGZsq{}}\PYG{p}{:} \PYG{n}{app\PYGZus{}comp}\PYG{o}{.}\PYG{n}{ranges}\PYG{p}{[}\PYG{l+s+s1}{\PYGZsq{}}\PYG{l+s+s1}{delay\PYGZus{}1}\PYG{l+s+s1}{\PYGZsq{}}\PYG{p}{]}\PYG{p}{[}\PYG{l+s+s1}{\PYGZsq{}}\PYG{l+s+s1}{scenarios}\PYG{l+s+s1}{\PYGZsq{}}\PYG{p}{]}\PYG{p}{,} \PYG{l+s+s1}{\PYGZsq{}}\PYG{l+s+s1}{delay\PYGZus{}10}\PYG{l+s+s1}{\PYGZsq{}}\PYG{p}{:}\PYG{n}{app\PYGZus{}comp}\PYG{o}{.}\PYG{n}{ranges}\PYG{p}{[}\PYG{l+s+s1}{\PYGZsq{}}\PYG{l+s+s1}{delay\PYGZus{}10}\PYG{l+s+s1}{\PYGZsq{}}\PYG{p}{]}\PYG{p}{[}\PYG{l+s+s1}{\PYGZsq{}}\PYG{l+s+s1}{scenarios}\PYG{l+s+s1}{\PYGZsq{}}\PYG{p}{]}\PYG{p}{\PYGZcb{}}
\end{sphinxVerbatim}
}

\end{sphinxuseclass}
\end{sphinxuseclass}
\sphinxAtStartPar
These are the resulting behaviors:

\begin{sphinxuseclass}{nbinput}
{
\sphinxsetup{VerbatimColor={named}{nbsphinx-code-bg}}
\sphinxsetup{VerbatimBorderColor={named}{nbsphinx-code-border}}
\begin{sphinxVerbatim}[commandchars=\\\{\}]
\llap{\color{nbsphinxin}[34]:\,\hspace{\fboxrule}\hspace{\fboxsep}}\PYG{n}{fig} \PYG{o}{=} \PYG{n}{rd}\PYG{o}{.}\PYG{n}{plot}\PYG{o}{.}\PYG{n}{mdlhists}\PYG{p}{(}\PYG{n}{comp\PYGZus{}mdlhists}\PYG{p}{,} \PYG{p}{\PYGZob{}}\PYG{l+s+s1}{\PYGZsq{}}\PYG{l+s+s1}{MoveWater}\PYG{l+s+s1}{\PYGZsq{}}\PYG{p}{:}\PYG{p}{[}\PYG{l+s+s1}{\PYGZsq{}}\PYG{l+s+s1}{eff}\PYG{l+s+s1}{\PYGZsq{}}\PYG{p}{,}\PYG{l+s+s1}{\PYGZsq{}}\PYG{l+s+s1}{total\PYGZus{}flow}\PYG{l+s+s1}{\PYGZsq{}}\PYG{p}{]}\PYG{p}{,} \PYG{l+s+s1}{\PYGZsq{}}\PYG{l+s+s1}{Wat\PYGZus{}2}\PYG{l+s+s1}{\PYGZsq{}}\PYG{p}{:}\PYG{p}{[}\PYG{l+s+s1}{\PYGZsq{}}\PYG{l+s+s1}{flowrate}\PYG{l+s+s1}{\PYGZsq{}}\PYG{p}{,}\PYG{l+s+s1}{\PYGZsq{}}\PYG{l+s+s1}{pressure}\PYG{l+s+s1}{\PYGZsq{}}\PYG{p}{]}\PYG{p}{\PYGZcb{}}\PYG{p}{,} \PYG{n}{comp\PYGZus{}groups}\PYG{o}{=}\PYG{n}{comp\PYGZus{}groups}\PYG{p}{,} \PYG{n}{aggregation}\PYG{o}{=}\PYG{l+s+s1}{\PYGZsq{}}\PYG{l+s+s1}{percentile}\PYG{l+s+s1}{\PYGZsq{}}\PYG{p}{,} \PYG{n}{time\PYGZus{}slice}\PYG{o}{=}\PYG{l+m+mi}{27}\PYG{p}{)}
\end{sphinxVerbatim}
}

\end{sphinxuseclass}
\begin{sphinxuseclass}{nboutput}
\begin{sphinxuseclass}{nblast}
\hrule height -\fboxrule\relax
\vspace{\nbsphinxcodecellspacing}

\makeatletter\setbox\nbsphinxpromptbox\box\voidb@x\makeatother

\begin{nbsphinxfancyoutput}

\begin{sphinxuseclass}{output_area}
\begin{sphinxuseclass}{}
\noindent\sphinxincludegraphics[width=378\sphinxpxdimen,height=278\sphinxpxdimen]{{example_pump_Stochastic_Modelling_66_0}.png}

\end{sphinxuseclass}
\end{sphinxuseclass}
\end{nbsphinxfancyoutput}

\end{sphinxuseclass}
\end{sphinxuseclass}
\sphinxAtStartPar
There a few interesting aspects of this simulation: \sphinxhyphen{} First, some of the \sphinxcode{\sphinxupquote{delay\_1}} design simulations have a zero flowrate before the fault is injected. This is because the reduced delay means that nominal behavior can easily cause the fault–all that nees to happen is for the pressure to go over 15 for two time\sphinxhyphen{}steps, which is entirely forseeable with all the variability in the simulation. This can also be seen in the “total\_flow” graph. \sphinxhyphen{} Second, the \sphinxcode{\sphinxupquote{delay\_10}} gives the nominal behavior
until the fault time, when pressure increases dramatically. This is also observed in the \sphinxcode{\sphinxupquote{delay\_1}} design, but with a much smaller delay (1 instead of 1).

\sphinxAtStartPar
Finally, running multiple simulations relies on the NominalApproach, it can be combined fairly easily with existing methods used in conjunction with NominalApproach. (e.g., \sphinxcode{\sphinxupquote{tabulate.nested\_stats}}).

\begin{sphinxuseclass}{nbinput}
{
\sphinxsetup{VerbatimColor={named}{nbsphinx-code-bg}}
\sphinxsetup{VerbatimBorderColor={named}{nbsphinx-code-border}}
\begin{sphinxVerbatim}[commandchars=\\\{\}]
\llap{\color{nbsphinxin}[35]:\,\hspace{\fboxrule}\hspace{\fboxsep}}\PYG{n}{rd}\PYG{o}{.}\PYG{n}{tabulate}\PYG{o}{.}\PYG{n}{nested\PYGZus{}stats}\PYG{p}{(}\PYG{n}{app\PYGZus{}comp}\PYG{p}{,} \PYG{n}{endclasses}\PYG{p}{,} \PYG{n}{average\PYGZus{}metrics}\PYG{o}{=}\PYG{p}{[}\PYG{l+s+s1}{\PYGZsq{}}\PYG{l+s+s1}{cost}\PYG{l+s+s1}{\PYGZsq{}}\PYG{p}{]}\PYG{p}{,} \PYG{n}{inputparams}\PYG{o}{=}\PYG{p}{[}\PYG{l+s+s1}{\PYGZsq{}}\PYG{l+s+s1}{delay}\PYG{l+s+s1}{\PYGZsq{}}\PYG{p}{]}\PYG{p}{)}
\end{sphinxVerbatim}
}

\end{sphinxuseclass}
\begin{sphinxuseclass}{nboutput}
\begin{sphinxuseclass}{nblast}
{

\kern-\sphinxverbatimsmallskipamount\kern-\baselineskip
\kern+\FrameHeightAdjust\kern-\fboxrule
\vspace{\nbsphinxcodecellspacing}

\sphinxsetup{VerbatimColor={named}{white}}
\sphinxsetup{VerbatimBorderColor={named}{nbsphinx-code-border}}
\begin{sphinxuseclass}{output_area}
\begin{sphinxuseclass}{}


\begin{sphinxVerbatim}[commandchars=\\\{\}]
\llap{\color{nbsphinxout}[35]:\,\hspace{\fboxrule}\hspace{\fboxsep}}          delay\_1\_1    delay\_1\_2    delay\_1\_3    delay\_1\_4   delay\_1\_5  \textbackslash{}
delay         1.000     1.000000     1.000000     1.000000     1.00000
ave\_cost  12047.481  9704.926554  9902.772044  9970.041994  8991.76441

            delay\_1\_6    delay\_1\_7     delay\_1\_8     delay\_1\_9    delay\_1\_10  \textbackslash{}
delay        1.000000      1.00000      1.000000      1.000000      1.000000
ave\_cost  9951.531744  12584.24885  12128.869103  10475.316805  11150.976817

          {\ldots}  delay\_10\_191  delay\_10\_192  delay\_10\_193  delay\_10\_194  \textbackslash{}
delay     {\ldots}     10.000000     10.000000      10.00000     10.000000
ave\_cost  {\ldots}  10389.226105  10170.589217   10300.64816   9963.804855

          delay\_10\_195  delay\_10\_196  delay\_10\_197  delay\_10\_198  \textbackslash{}
delay        10.000000     10.000000     10.000000     10.000000
ave\_cost  10100.371529  10307.337124  10022.831267   9755.299692

          delay\_10\_199  delay\_10\_200
delay        10.000000     10.000000
ave\_cost  10074.886941  10231.796737

[2 rows x 200 columns]
\end{sphinxVerbatim}



\end{sphinxuseclass}
\end{sphinxuseclass}
}

\end{sphinxuseclass}
\end{sphinxuseclass}
\begin{sphinxuseclass}{nbinput}
{
\sphinxsetup{VerbatimColor={named}{nbsphinx-code-bg}}
\sphinxsetup{VerbatimBorderColor={named}{nbsphinx-code-border}}
\begin{sphinxVerbatim}[commandchars=\\\{\}]
\llap{\color{nbsphinxin}[37]:\,\hspace{\fboxrule}\hspace{\fboxsep}}\PYG{n}{rd}\PYG{o}{.}\PYG{n}{tabulate}\PYG{o}{.}\PYG{n}{resilience\PYGZus{}factor\PYGZus{}comparison}\PYG{p}{(}\PYG{n}{app\PYGZus{}comp}\PYG{p}{,} \PYG{n}{endclasses}\PYG{p}{,} \PYG{p}{[}\PYG{l+s+s1}{\PYGZsq{}}\PYG{l+s+s1}{delay}\PYG{l+s+s1}{\PYGZsq{}}\PYG{p}{]}\PYG{p}{,} \PYG{l+s+s1}{\PYGZsq{}}\PYG{l+s+s1}{cost}\PYG{l+s+s1}{\PYGZsq{}}\PYG{p}{,} \PYG{n}{difference}\PYG{o}{=}\PYG{k+kc}{False}\PYG{p}{,} \PYG{n}{percent}\PYG{o}{=}\PYG{k+kc}{False}\PYG{p}{)}
\end{sphinxVerbatim}
}

\end{sphinxuseclass}
\begin{sphinxuseclass}{nboutput}
\begin{sphinxuseclass}{nblast}
{

\kern-\sphinxverbatimsmallskipamount\kern-\baselineskip
\kern+\FrameHeightAdjust\kern-\fboxrule
\vspace{\nbsphinxcodecellspacing}

\sphinxsetup{VerbatimColor={named}{white}}
\sphinxsetup{VerbatimBorderColor={named}{nbsphinx-code-border}}
\begin{sphinxuseclass}{output_area}
\begin{sphinxuseclass}{}


\begin{sphinxVerbatim}[commandchars=\\\{\}]
\llap{\color{nbsphinxout}[37]:\,\hspace{\fboxrule}\hspace{\fboxsep}}('delay',)  nominal   ExportWater
delay\_1      3600.0  12515.568381
delay\_10        0.0  13400.999963
\end{sphinxVerbatim}



\end{sphinxuseclass}
\end{sphinxuseclass}
}

\end{sphinxuseclass}
\end{sphinxuseclass}
\begin{sphinxuseclass}{nbinput}
\begin{sphinxuseclass}{nblast}
{
\sphinxsetup{VerbatimColor={named}{nbsphinx-code-bg}}
\sphinxsetup{VerbatimBorderColor={named}{nbsphinx-code-border}}
\begin{sphinxVerbatim}[commandchars=\\\{\}]
\llap{\color{nbsphinxin}[ ]:\,\hspace{\fboxrule}\hspace{\fboxsep}}
\end{sphinxVerbatim}
}

\end{sphinxuseclass}
\end{sphinxuseclass}

\subsection{Hold\sphinxhyphen{}up Tank Model}
\label{\detokenize{example_tank/Tank_Analysis:Hold-up-Tank-Model}}\label{\detokenize{example_tank/Tank_Analysis::doc}}
\sphinxAtStartPar
Using fmdtools to simulate hazards in a system with human\sphinxhyphen{}component interactions, including:
\begin{itemize}
\item {} 
\sphinxAtStartPar
human\sphinxhyphen{}induced failure modes

\item {} 
\sphinxAtStartPar
human responses to component failure modes

\item {} 
\sphinxAtStartPar
joint human\sphinxhyphen{}component failure modes

\end{itemize}

\sphinxAtStartPar
The system to model is in \sphinxcode{\sphinxupquote{tank\_model.py}}

\begin{sphinxuseclass}{nbinput}
\begin{sphinxuseclass}{nblast}
{
\sphinxsetup{VerbatimColor={named}{nbsphinx-code-bg}}
\sphinxsetup{VerbatimBorderColor={named}{nbsphinx-code-border}}
\begin{sphinxVerbatim}[commandchars=\\\{\}]
\llap{\color{nbsphinxin}[1]:\,\hspace{\fboxrule}\hspace{\fboxsep}}\PYG{k+kn}{import} \PYG{n+nn}{sys}\PYG{o}{,} \PYG{n+nn}{os}
\PYG{n}{sys}\PYG{o}{.}\PYG{n}{path}\PYG{o}{.}\PYG{n}{insert}\PYG{p}{(}\PYG{l+m+mi}{1}\PYG{p}{,}\PYG{n}{os}\PYG{o}{.}\PYG{n}{path}\PYG{o}{.}\PYG{n}{join}\PYG{p}{(}\PYG{l+s+s2}{\PYGZdq{}}\PYG{l+s+s2}{..}\PYG{l+s+s2}{\PYGZdq{}}\PYG{p}{)}\PYG{p}{)}

\PYG{k+kn}{import} \PYG{n+nn}{fmdtools}\PYG{n+nn}{.}\PYG{n+nn}{faultsim}\PYG{n+nn}{.}\PYG{n+nn}{propagate} \PYG{k}{as} \PYG{n+nn}{propagate}
\PYG{k+kn}{import} \PYG{n+nn}{fmdtools}\PYG{n+nn}{.}\PYG{n+nn}{resultdisp} \PYG{k}{as} \PYG{n+nn}{rd}
\PYG{k+kn}{from} \PYG{n+nn}{tank\PYGZus{}model} \PYG{k+kn}{import} \PYG{n}{Tank}
\PYG{k+kn}{from} \PYG{n+nn}{fmdtools}\PYG{n+nn}{.}\PYG{n+nn}{modeldef} \PYG{k+kn}{import} \PYG{n}{SampleApproach}
\end{sphinxVerbatim}
}

\end{sphinxuseclass}
\end{sphinxuseclass}

\subsubsection{Verifying the nominal state:}
\label{\detokenize{example_tank/Tank_Analysis:Verifying-the-nominal-state:}}
\sphinxAtStartPar
In the nominal state, no change in system state should occur and the tank level should remain at 5.

\begin{sphinxuseclass}{nbinput}
\begin{sphinxuseclass}{nblast}
{
\sphinxsetup{VerbatimColor={named}{nbsphinx-code-bg}}
\sphinxsetup{VerbatimBorderColor={named}{nbsphinx-code-border}}
\begin{sphinxVerbatim}[commandchars=\\\{\}]
\llap{\color{nbsphinxin}[2]:\,\hspace{\fboxrule}\hspace{\fboxsep}}\PYG{n}{mdl} \PYG{o}{=} \PYG{n}{Tank}\PYG{p}{(}\PYG{p}{)}
\PYG{n}{endresults}\PYG{p}{,} \PYG{n}{resgraph}\PYG{p}{,} \PYG{n}{mdlhist} \PYG{o}{=} \PYG{n}{propagate}\PYG{o}{.}\PYG{n}{nominal}\PYG{p}{(}\PYG{n}{mdl}\PYG{p}{)}
\end{sphinxVerbatim}
}

\end{sphinxuseclass}
\end{sphinxuseclass}
\begin{sphinxuseclass}{nbinput}
{
\sphinxsetup{VerbatimColor={named}{nbsphinx-code-bg}}
\sphinxsetup{VerbatimBorderColor={named}{nbsphinx-code-border}}
\begin{sphinxVerbatim}[commandchars=\\\{\}]
\llap{\color{nbsphinxin}[3]:\,\hspace{\fboxrule}\hspace{\fboxsep}}\PYG{n}{mdlhist}\PYG{o}{.}\PYG{n}{keys}\PYG{p}{(}\PYG{p}{)}
\end{sphinxVerbatim}
}

\end{sphinxuseclass}
\begin{sphinxuseclass}{nboutput}
\begin{sphinxuseclass}{nblast}
{

\kern-\sphinxverbatimsmallskipamount\kern-\baselineskip
\kern+\FrameHeightAdjust\kern-\fboxrule
\vspace{\nbsphinxcodecellspacing}

\sphinxsetup{VerbatimColor={named}{white}}
\sphinxsetup{VerbatimBorderColor={named}{nbsphinx-code-border}}
\begin{sphinxuseclass}{output_area}
\begin{sphinxuseclass}{}


\begin{sphinxVerbatim}[commandchars=\\\{\}]
\llap{\color{nbsphinxout}[3]:\,\hspace{\fboxrule}\hspace{\fboxsep}}dict\_keys(['flows', 'functions', 'time'])
\end{sphinxVerbatim}



\end{sphinxuseclass}
\end{sphinxuseclass}
}

\end{sphinxuseclass}
\end{sphinxuseclass}
\begin{sphinxuseclass}{nbinput}
{
\sphinxsetup{VerbatimColor={named}{nbsphinx-code-bg}}
\sphinxsetup{VerbatimBorderColor={named}{nbsphinx-code-border}}
\begin{sphinxVerbatim}[commandchars=\\\{\}]
\llap{\color{nbsphinxin}[4]:\,\hspace{\fboxrule}\hspace{\fboxsep}}\PYG{n}{rd}\PYG{o}{.}\PYG{n}{graph}\PYG{o}{.}\PYG{n}{show}\PYG{p}{(}\PYG{n}{resgraph}\PYG{p}{)}
\end{sphinxVerbatim}
}

\end{sphinxuseclass}
\begin{sphinxuseclass}{nboutput}
{

\kern-\sphinxverbatimsmallskipamount\kern-\baselineskip
\kern+\FrameHeightAdjust\kern-\fboxrule
\vspace{\nbsphinxcodecellspacing}

\sphinxsetup{VerbatimColor={named}{white}}
\sphinxsetup{VerbatimBorderColor={named}{nbsphinx-code-border}}
\begin{sphinxuseclass}{output_area}
\begin{sphinxuseclass}{}


\begin{sphinxVerbatim}[commandchars=\\\{\}]
\llap{\color{nbsphinxout}[4]:\,\hspace{\fboxrule}\hspace{\fboxsep}}(<Figure size 432x288 with 1 Axes>, <AxesSubplot:>)
\end{sphinxVerbatim}



\end{sphinxuseclass}
\end{sphinxuseclass}
}

\end{sphinxuseclass}
\begin{sphinxuseclass}{nboutput}
\begin{sphinxuseclass}{nblast}
\hrule height -\fboxrule\relax
\vspace{\nbsphinxcodecellspacing}

\makeatletter\setbox\nbsphinxpromptbox\box\voidb@x\makeatother

\begin{nbsphinxfancyoutput}

\begin{sphinxuseclass}{output_area}
\begin{sphinxuseclass}{}
\noindent\sphinxincludegraphics[width=349\sphinxpxdimen,height=231\sphinxpxdimen]{{example_tank_Tank_Analysis_5_1}.png}

\end{sphinxuseclass}
\end{sphinxuseclass}
\end{nbsphinxfancyoutput}

\end{sphinxuseclass}
\end{sphinxuseclass}
\begin{sphinxuseclass}{nbinput}
{
\sphinxsetup{VerbatimColor={named}{nbsphinx-code-bg}}
\sphinxsetup{VerbatimBorderColor={named}{nbsphinx-code-border}}
\begin{sphinxVerbatim}[commandchars=\\\{\}]
\llap{\color{nbsphinxin}[5]:\,\hspace{\fboxrule}\hspace{\fboxsep}}\PYG{n}{rd}\PYG{o}{.}\PYG{n}{plot}\PYG{o}{.}\PYG{n}{mdlhistvals}\PYG{p}{(}\PYG{n}{mdlhist}\PYG{p}{)}
\end{sphinxVerbatim}
}

\end{sphinxuseclass}
\begin{sphinxuseclass}{nboutput}
\hrule height -\fboxrule\relax
\vspace{\nbsphinxcodecellspacing}

\savebox\nbsphinxpromptbox[0pt][r]{\color{nbsphinxout}\Verb|\strut{[5]:}\,|}

\begin{nbsphinxfancyoutput}

\begin{sphinxuseclass}{output_area}
\begin{sphinxuseclass}{}
\noindent\sphinxincludegraphics[width=426\sphinxpxdimen,height=1197\sphinxpxdimen]{{example_tank_Tank_Analysis_6_0}.png}

\end{sphinxuseclass}
\end{sphinxuseclass}
\end{nbsphinxfancyoutput}

\end{sphinxuseclass}
\begin{sphinxuseclass}{nboutput}
\begin{sphinxuseclass}{nblast}
\hrule height -\fboxrule\relax
\vspace{\nbsphinxcodecellspacing}

\makeatletter\setbox\nbsphinxpromptbox\box\voidb@x\makeatother

\begin{nbsphinxfancyoutput}

\begin{sphinxuseclass}{output_area}
\begin{sphinxuseclass}{}
\noindent\sphinxincludegraphics[width=426\sphinxpxdimen,height=1197\sphinxpxdimen]{{example_tank_Tank_Analysis_6_1}.png}

\end{sphinxuseclass}
\end{sphinxuseclass}
\end{nbsphinxfancyoutput}

\end{sphinxuseclass}
\end{sphinxuseclass}

\subsubsection{What happens under component faults?}
\label{\detokenize{example_tank/Tank_Analysis:What-happens-under-component-faults?}}
\sphinxAtStartPar
Here we model a leak of the tank. To compensate for this leak, the operator opens the first valve to a higher setting, maintaining the level of the tank

\begin{sphinxuseclass}{nbinput}
{
\sphinxsetup{VerbatimColor={named}{nbsphinx-code-bg}}
\sphinxsetup{VerbatimBorderColor={named}{nbsphinx-code-border}}
\begin{sphinxVerbatim}[commandchars=\\\{\}]
\llap{\color{nbsphinxin}[6]:\,\hspace{\fboxrule}\hspace{\fboxsep}}\PYG{n}{endresults}\PYG{p}{,} \PYG{n}{resgraph}\PYG{p}{,} \PYG{n}{mdlhist} \PYG{o}{=} \PYG{n}{propagate}\PYG{o}{.}\PYG{n}{one\PYGZus{}fault}\PYG{p}{(}\PYG{n}{mdl}\PYG{p}{,}\PYG{l+s+s1}{\PYGZsq{}}\PYG{l+s+s1}{Store\PYGZus{}Water}\PYG{l+s+s1}{\PYGZsq{}}\PYG{p}{,}\PYG{l+s+s1}{\PYGZsq{}}\PYG{l+s+s1}{Leak}\PYG{l+s+s1}{\PYGZsq{}}\PYG{p}{,} \PYG{n}{time}\PYG{o}{=}\PYG{l+m+mi}{3}\PYG{p}{)}

\PYG{n}{rd}\PYG{o}{.}\PYG{n}{plot}\PYG{o}{.}\PYG{n}{mdlhistvals}\PYG{p}{(}\PYG{n}{mdlhist}\PYG{p}{,} \PYG{n}{fault}\PYG{o}{=}\PYG{l+s+s1}{\PYGZsq{}}\PYG{l+s+s1}{Leak}\PYG{l+s+s1}{\PYGZsq{}}\PYG{p}{,} \PYG{n}{time}\PYG{o}{=}\PYG{l+m+mi}{3}\PYG{p}{,} \PYG{n}{fxnflowvals}\PYG{o}{=}\PYG{p}{\PYGZob{}}\PYG{l+s+s1}{\PYGZsq{}}\PYG{l+s+s1}{Store\PYGZus{}Water}\PYG{l+s+s1}{\PYGZsq{}}\PYG{p}{:}\PYG{p}{[}\PYG{l+s+s1}{\PYGZsq{}}\PYG{l+s+s1}{level}\PYG{l+s+s1}{\PYGZsq{}}\PYG{p}{,} \PYG{l+s+s1}{\PYGZsq{}}\PYG{l+s+s1}{net\PYGZus{}flow}\PYG{l+s+s1}{\PYGZsq{}}\PYG{p}{]}\PYG{p}{,} \PYG{l+s+s1}{\PYGZsq{}}\PYG{l+s+s1}{Tank\PYGZus{}Sig}\PYG{l+s+s1}{\PYGZsq{}}\PYG{p}{:}\PYG{p}{[}\PYG{l+s+s1}{\PYGZsq{}}\PYG{l+s+s1}{indicator}\PYG{l+s+s1}{\PYGZsq{}}\PYG{p}{]}\PYG{p}{,} \PYG{l+s+s1}{\PYGZsq{}}\PYG{l+s+s1}{Valve1\PYGZus{}Sig}\PYG{l+s+s1}{\PYGZsq{}}\PYG{p}{:}\PYG{p}{[}\PYG{l+s+s1}{\PYGZsq{}}\PYG{l+s+s1}{action}\PYG{l+s+s1}{\PYGZsq{}}\PYG{p}{]}\PYG{p}{\PYGZcb{}}\PYG{p}{,} \PYG{n}{legend}\PYG{o}{=}\PYG{k+kc}{False}\PYG{p}{)}
\PYG{n}{rd}\PYG{o}{.}\PYG{n}{graph}\PYG{o}{.}\PYG{n}{show}\PYG{p}{(}\PYG{n}{resgraph}\PYG{p}{,}\PYG{n}{faultscen}\PYG{o}{=}\PYG{l+s+s1}{\PYGZsq{}}\PYG{l+s+s1}{Leak}\PYG{l+s+s1}{\PYGZsq{}}\PYG{p}{,} \PYG{n}{time}\PYG{o}{=}\PYG{l+m+mi}{3}\PYG{p}{)}

\end{sphinxVerbatim}
}

\end{sphinxuseclass}
\begin{sphinxuseclass}{nboutput}
{

\kern-\sphinxverbatimsmallskipamount\kern-\baselineskip
\kern+\FrameHeightAdjust\kern-\fboxrule
\vspace{\nbsphinxcodecellspacing}

\sphinxsetup{VerbatimColor={named}{white}}
\sphinxsetup{VerbatimBorderColor={named}{nbsphinx-code-border}}
\begin{sphinxuseclass}{output_area}
\begin{sphinxuseclass}{}


\begin{sphinxVerbatim}[commandchars=\\\{\}]
\llap{\color{nbsphinxout}[6]:\,\hspace{\fboxrule}\hspace{\fboxsep}}(<Figure size 432x288 with 1 Axes>,
 <AxesSubplot:title=\{'center':'Propagation of faults to Leak at t=3'\}>)
\end{sphinxVerbatim}



\end{sphinxuseclass}
\end{sphinxuseclass}
}

\end{sphinxuseclass}
\begin{sphinxuseclass}{nboutput}
\hrule height -\fboxrule\relax
\vspace{\nbsphinxcodecellspacing}

\makeatletter\setbox\nbsphinxpromptbox\box\voidb@x\makeatother

\begin{nbsphinxfancyoutput}

\begin{sphinxuseclass}{output_area}
\begin{sphinxuseclass}{}
\noindent\sphinxincludegraphics[width=462\sphinxpxdimen,height=286\sphinxpxdimen]{{example_tank_Tank_Analysis_8_1}.png}

\end{sphinxuseclass}
\end{sphinxuseclass}
\end{nbsphinxfancyoutput}

\end{sphinxuseclass}
\begin{sphinxuseclass}{nboutput}
\begin{sphinxuseclass}{nblast}
\hrule height -\fboxrule\relax
\vspace{\nbsphinxcodecellspacing}

\makeatletter\setbox\nbsphinxpromptbox\box\voidb@x\makeatother

\begin{nbsphinxfancyoutput}

\begin{sphinxuseclass}{output_area}
\begin{sphinxuseclass}{}
\noindent\sphinxincludegraphics[width=349\sphinxpxdimen,height=247\sphinxpxdimen]{{example_tank_Tank_Analysis_8_2}.png}

\end{sphinxuseclass}
\end{sphinxuseclass}
\end{nbsphinxfancyoutput}

\end{sphinxuseclass}
\end{sphinxuseclass}

\subsubsection{What about human\sphinxhyphen{}induced faults?}
\label{\detokenize{example_tank/Tank_Analysis:What-about-human-induced-faults?}}
\sphinxAtStartPar
Here we evaluate what happens if the operator thinks they see a low or high indicator and takes those given actions.

\sphinxAtStartPar
Note that in these cases, because of the indicator/procedures, the operators are able to correct for the fault.

\begin{sphinxuseclass}{nbinput}
{
\sphinxsetup{VerbatimColor={named}{nbsphinx-code-bg}}
\sphinxsetup{VerbatimBorderColor={named}{nbsphinx-code-border}}
\begin{sphinxVerbatim}[commandchars=\\\{\}]
\llap{\color{nbsphinxin}[7]:\,\hspace{\fboxrule}\hspace{\fboxsep}}\PYG{n}{endresults}\PYG{p}{,} \PYG{n}{resgraph}\PYG{p}{,} \PYG{n}{mdlhist} \PYG{o}{=} \PYG{n}{propagate}\PYG{o}{.}\PYG{n}{one\PYGZus{}fault}\PYG{p}{(}\PYG{n}{mdl}\PYG{p}{,}\PYG{l+s+s1}{\PYGZsq{}}\PYG{l+s+s1}{Human}\PYG{l+s+s1}{\PYGZsq{}}\PYG{p}{,}\PYG{l+s+s1}{\PYGZsq{}}\PYG{l+s+s1}{FalseDetection\PYGZus{}low}\PYG{l+s+s1}{\PYGZsq{}}\PYG{p}{,} \PYG{n}{time}\PYG{o}{=}\PYG{l+m+mi}{3}\PYG{p}{)}

\PYG{n}{rd}\PYG{o}{.}\PYG{n}{plot}\PYG{o}{.}\PYG{n}{mdlhistvals}\PYG{p}{(}\PYG{n}{mdlhist}\PYG{p}{,} \PYG{n}{fault}\PYG{o}{=}\PYG{l+s+s1}{\PYGZsq{}}\PYG{l+s+s1}{FalseDetection\PYGZus{}low}\PYG{l+s+s1}{\PYGZsq{}}\PYG{p}{,} \PYG{n}{time}\PYG{o}{=}\PYG{l+m+mi}{3}\PYG{p}{,} \PYG{n}{fxnflowvals}\PYG{o}{=}\PYG{p}{\PYGZob{}}\PYG{l+s+s1}{\PYGZsq{}}\PYG{l+s+s1}{Store\PYGZus{}Water}\PYG{l+s+s1}{\PYGZsq{}}\PYG{p}{:}\PYG{p}{[}\PYG{l+s+s1}{\PYGZsq{}}\PYG{l+s+s1}{level}\PYG{l+s+s1}{\PYGZsq{}}\PYG{p}{,} \PYG{l+s+s1}{\PYGZsq{}}\PYG{l+s+s1}{net\PYGZus{}flow}\PYG{l+s+s1}{\PYGZsq{}}\PYG{p}{]}\PYG{p}{,} \PYG{l+s+s1}{\PYGZsq{}}\PYG{l+s+s1}{Tank\PYGZus{}Sig}\PYG{l+s+s1}{\PYGZsq{}}\PYG{p}{:}\PYG{p}{[}\PYG{l+s+s1}{\PYGZsq{}}\PYG{l+s+s1}{indicator}\PYG{l+s+s1}{\PYGZsq{}}\PYG{p}{]}\PYG{p}{,} \PYG{l+s+s1}{\PYGZsq{}}\PYG{l+s+s1}{Valve1\PYGZus{}Sig}\PYG{l+s+s1}{\PYGZsq{}}\PYG{p}{:}\PYG{p}{[}\PYG{l+s+s1}{\PYGZsq{}}\PYG{l+s+s1}{action}\PYG{l+s+s1}{\PYGZsq{}}\PYG{p}{]}\PYG{p}{\PYGZcb{}}\PYG{p}{,} \PYG{n}{legend}\PYG{o}{=}\PYG{k+kc}{False}\PYG{p}{)}
\PYG{n}{rd}\PYG{o}{.}\PYG{n}{graph}\PYG{o}{.}\PYG{n}{show}\PYG{p}{(}\PYG{n}{resgraph}\PYG{p}{,}\PYG{n}{faultscen}\PYG{o}{=}\PYG{l+s+s1}{\PYGZsq{}}\PYG{l+s+s1}{FalseDetection\PYGZus{}low}\PYG{l+s+s1}{\PYGZsq{}}\PYG{p}{,} \PYG{n}{time}\PYG{o}{=}\PYG{l+m+mi}{3}\PYG{p}{)}
\end{sphinxVerbatim}
}

\end{sphinxuseclass}
\begin{sphinxuseclass}{nboutput}
{

\kern-\sphinxverbatimsmallskipamount\kern-\baselineskip
\kern+\FrameHeightAdjust\kern-\fboxrule
\vspace{\nbsphinxcodecellspacing}

\sphinxsetup{VerbatimColor={named}{white}}
\sphinxsetup{VerbatimBorderColor={named}{nbsphinx-code-border}}
\begin{sphinxuseclass}{output_area}
\begin{sphinxuseclass}{}


\begin{sphinxVerbatim}[commandchars=\\\{\}]
\llap{\color{nbsphinxout}[7]:\,\hspace{\fboxrule}\hspace{\fboxsep}}(<Figure size 432x288 with 1 Axes>,
 <AxesSubplot:title=\{'center':'Propagation of faults to FalseDetection\_low at t=3'\}>)
\end{sphinxVerbatim}



\end{sphinxuseclass}
\end{sphinxuseclass}
}

\end{sphinxuseclass}
\begin{sphinxuseclass}{nboutput}
\hrule height -\fboxrule\relax
\vspace{\nbsphinxcodecellspacing}

\makeatletter\setbox\nbsphinxpromptbox\box\voidb@x\makeatother

\begin{nbsphinxfancyoutput}

\begin{sphinxuseclass}{output_area}
\begin{sphinxuseclass}{}
\noindent\sphinxincludegraphics[width=549\sphinxpxdimen,height=286\sphinxpxdimen]{{example_tank_Tank_Analysis_10_1}.png}

\end{sphinxuseclass}
\end{sphinxuseclass}
\end{nbsphinxfancyoutput}

\end{sphinxuseclass}
\begin{sphinxuseclass}{nboutput}
\begin{sphinxuseclass}{nblast}
\hrule height -\fboxrule\relax
\vspace{\nbsphinxcodecellspacing}

\makeatletter\setbox\nbsphinxpromptbox\box\voidb@x\makeatother

\begin{nbsphinxfancyoutput}

\begin{sphinxuseclass}{output_area}
\begin{sphinxuseclass}{}
\noindent\sphinxincludegraphics[width=349\sphinxpxdimen,height=247\sphinxpxdimen]{{example_tank_Tank_Analysis_10_2}.png}

\end{sphinxuseclass}
\end{sphinxuseclass}
\end{nbsphinxfancyoutput}

\end{sphinxuseclass}
\end{sphinxuseclass}
\begin{sphinxuseclass}{nbinput}
{
\sphinxsetup{VerbatimColor={named}{nbsphinx-code-bg}}
\sphinxsetup{VerbatimBorderColor={named}{nbsphinx-code-border}}
\begin{sphinxVerbatim}[commandchars=\\\{\}]
\llap{\color{nbsphinxin}[8]:\,\hspace{\fboxrule}\hspace{\fboxsep}}\PYG{n}{endresults}\PYG{p}{,} \PYG{n}{resgraph}\PYG{p}{,} \PYG{n}{mdlhist} \PYG{o}{=} \PYG{n}{propagate}\PYG{o}{.}\PYG{n}{one\PYGZus{}fault}\PYG{p}{(}\PYG{n}{mdl}\PYG{p}{,}\PYG{l+s+s1}{\PYGZsq{}}\PYG{l+s+s1}{Human}\PYG{l+s+s1}{\PYGZsq{}}\PYG{p}{,}\PYG{l+s+s1}{\PYGZsq{}}\PYG{l+s+s1}{FalseDetection\PYGZus{}high}\PYG{l+s+s1}{\PYGZsq{}}\PYG{p}{,} \PYG{n}{time}\PYG{o}{=}\PYG{l+m+mi}{3}\PYG{p}{)}

\PYG{n}{rd}\PYG{o}{.}\PYG{n}{plot}\PYG{o}{.}\PYG{n}{mdlhistvals}\PYG{p}{(}\PYG{n}{mdlhist}\PYG{p}{,} \PYG{n}{fault}\PYG{o}{=}\PYG{l+s+s1}{\PYGZsq{}}\PYG{l+s+s1}{FalseDetection\PYGZus{}high}\PYG{l+s+s1}{\PYGZsq{}}\PYG{p}{,} \PYG{n}{time}\PYG{o}{=}\PYG{l+m+mi}{3}\PYG{p}{,} \PYG{n}{fxnflowvals}\PYG{o}{=}\PYG{p}{\PYGZob{}}\PYG{l+s+s1}{\PYGZsq{}}\PYG{l+s+s1}{Store\PYGZus{}Water}\PYG{l+s+s1}{\PYGZsq{}}\PYG{p}{:}\PYG{p}{[}\PYG{l+s+s1}{\PYGZsq{}}\PYG{l+s+s1}{level}\PYG{l+s+s1}{\PYGZsq{}}\PYG{p}{,} \PYG{l+s+s1}{\PYGZsq{}}\PYG{l+s+s1}{net\PYGZus{}flow}\PYG{l+s+s1}{\PYGZsq{}}\PYG{p}{]}\PYG{p}{,} \PYG{l+s+s1}{\PYGZsq{}}\PYG{l+s+s1}{Tank\PYGZus{}Sig}\PYG{l+s+s1}{\PYGZsq{}}\PYG{p}{:}\PYG{p}{[}\PYG{l+s+s1}{\PYGZsq{}}\PYG{l+s+s1}{indicator}\PYG{l+s+s1}{\PYGZsq{}}\PYG{p}{]}\PYG{p}{,} \PYG{l+s+s1}{\PYGZsq{}}\PYG{l+s+s1}{Valve1\PYGZus{}Sig}\PYG{l+s+s1}{\PYGZsq{}}\PYG{p}{:}\PYG{p}{[}\PYG{l+s+s1}{\PYGZsq{}}\PYG{l+s+s1}{action}\PYG{l+s+s1}{\PYGZsq{}}\PYG{p}{]}\PYG{p}{\PYGZcb{}}\PYG{p}{,} \PYG{n}{legend}\PYG{o}{=}\PYG{k+kc}{False}\PYG{p}{)}
\PYG{n}{rd}\PYG{o}{.}\PYG{n}{graph}\PYG{o}{.}\PYG{n}{show}\PYG{p}{(}\PYG{n}{resgraph}\PYG{p}{,}\PYG{n}{faultscen}\PYG{o}{=}\PYG{l+s+s1}{\PYGZsq{}}\PYG{l+s+s1}{FalseDetection\PYGZus{}high}\PYG{l+s+s1}{\PYGZsq{}}\PYG{p}{,} \PYG{n}{time}\PYG{o}{=}\PYG{l+m+mi}{3}\PYG{p}{)}
\end{sphinxVerbatim}
}

\end{sphinxuseclass}
\begin{sphinxuseclass}{nboutput}
{

\kern-\sphinxverbatimsmallskipamount\kern-\baselineskip
\kern+\FrameHeightAdjust\kern-\fboxrule
\vspace{\nbsphinxcodecellspacing}

\sphinxsetup{VerbatimColor={named}{white}}
\sphinxsetup{VerbatimBorderColor={named}{nbsphinx-code-border}}
\begin{sphinxuseclass}{output_area}
\begin{sphinxuseclass}{}


\begin{sphinxVerbatim}[commandchars=\\\{\}]
\llap{\color{nbsphinxout}[8]:\,\hspace{\fboxrule}\hspace{\fboxsep}}(<Figure size 432x288 with 1 Axes>,
 <AxesSubplot:title=\{'center':'Propagation of faults to FalseDetection\_high at t=3'\}>)
\end{sphinxVerbatim}



\end{sphinxuseclass}
\end{sphinxuseclass}
}

\end{sphinxuseclass}
\begin{sphinxuseclass}{nboutput}
\hrule height -\fboxrule\relax
\vspace{\nbsphinxcodecellspacing}

\makeatletter\setbox\nbsphinxpromptbox\box\voidb@x\makeatother

\begin{nbsphinxfancyoutput}

\begin{sphinxuseclass}{output_area}
\begin{sphinxuseclass}{}
\noindent\sphinxincludegraphics[width=554\sphinxpxdimen,height=286\sphinxpxdimen]{{example_tank_Tank_Analysis_11_1}.png}

\end{sphinxuseclass}
\end{sphinxuseclass}
\end{nbsphinxfancyoutput}

\end{sphinxuseclass}
\begin{sphinxuseclass}{nboutput}
\begin{sphinxuseclass}{nblast}
\hrule height -\fboxrule\relax
\vspace{\nbsphinxcodecellspacing}

\makeatletter\setbox\nbsphinxpromptbox\box\voidb@x\makeatother

\begin{nbsphinxfancyoutput}

\begin{sphinxuseclass}{output_area}
\begin{sphinxuseclass}{}
\noindent\sphinxincludegraphics[width=349\sphinxpxdimen,height=247\sphinxpxdimen]{{example_tank_Tank_Analysis_11_2}.png}

\end{sphinxuseclass}
\end{sphinxuseclass}
\end{nbsphinxfancyoutput}

\end{sphinxuseclass}
\end{sphinxuseclass}

\subsubsection{Evaluating Joint Component\sphinxhyphen{}Human fault modes}
\label{\detokenize{example_tank/Tank_Analysis:Evaluating-Joint-Component-Human-fault-modes}}
\sphinxAtStartPar
To understand where the risks of failure are, we need to find the scenarios, that, with the modelled human controls, still lead to failures. To assess this, we develop a sample approach.

\begin{sphinxuseclass}{nbinput}
\begin{sphinxuseclass}{nblast}
{
\sphinxsetup{VerbatimColor={named}{nbsphinx-code-bg}}
\sphinxsetup{VerbatimBorderColor={named}{nbsphinx-code-border}}
\begin{sphinxVerbatim}[commandchars=\\\{\}]
\llap{\color{nbsphinxin}[9]:\,\hspace{\fboxrule}\hspace{\fboxsep}}\PYG{c+c1}{\PYGZsh{}app\PYGZus{}full = SampleApproach(mdl)}
\PYG{c+c1}{\PYGZsh{}endclasses, mdlhists = fp.run\PYGZus{}approach(mdl, app\PYGZus{}full)}
\end{sphinxVerbatim}
}

\end{sphinxuseclass}
\end{sphinxuseclass}
\sphinxAtStartPar
Here we consider all single and joint\sphinxhyphen{}fault scenarios in the set of simulations to see which ones lead to failure:

\begin{sphinxuseclass}{nbinput}
{
\sphinxsetup{VerbatimColor={named}{nbsphinx-code-bg}}
\sphinxsetup{VerbatimBorderColor={named}{nbsphinx-code-border}}
\begin{sphinxVerbatim}[commandchars=\\\{\}]
\llap{\color{nbsphinxin}[10]:\,\hspace{\fboxrule}\hspace{\fboxsep}}\PYG{n}{app\PYGZus{}joint\PYGZus{}faults} \PYG{o}{=} \PYG{n}{SampleApproach}\PYG{p}{(}\PYG{n}{mdl}\PYG{p}{,} \PYG{n}{faults}\PYG{o}{=}\PYG{l+s+s1}{\PYGZsq{}}\PYG{l+s+s1}{all}\PYG{l+s+s1}{\PYGZsq{}}\PYG{p}{,} \PYG{n}{jointfaults}\PYG{o}{=}\PYG{p}{\PYGZob{}}\PYG{l+s+s1}{\PYGZsq{}}\PYG{l+s+s1}{faults}\PYG{l+s+s1}{\PYGZsq{}}\PYG{p}{:} \PYG{l+m+mi}{2}\PYG{p}{\PYGZcb{}}\PYG{p}{)}
\PYG{n}{endclasses}\PYG{p}{,} \PYG{n}{mdlhists} \PYG{o}{=} \PYG{n}{propagate}\PYG{o}{.}\PYG{n}{approach}\PYG{p}{(}\PYG{n}{mdl}\PYG{p}{,} \PYG{n}{app\PYGZus{}joint\PYGZus{}faults}\PYG{p}{)}
\PYG{n}{fmea\PYGZus{}tab} \PYG{o}{=} \PYG{n}{rd}\PYG{o}{.}\PYG{n}{tabulate}\PYG{o}{.}\PYG{n}{simplefmea}\PYG{p}{(}\PYG{n}{endclasses}\PYG{p}{)}
\PYG{n}{fmea\PYGZus{}tab}
\end{sphinxVerbatim}
}

\end{sphinxuseclass}
\begin{sphinxuseclass}{nboutput}
{

\kern-\sphinxverbatimsmallskipamount\kern-\baselineskip
\kern+\FrameHeightAdjust\kern-\fboxrule
\vspace{\nbsphinxcodecellspacing}

\sphinxsetup{VerbatimColor={named}{nbsphinx-stderr}}
\sphinxsetup{VerbatimBorderColor={named}{nbsphinx-code-border}}
\begin{sphinxuseclass}{output_area}
\begin{sphinxuseclass}{stderr}


\begin{sphinxVerbatim}[commandchars=\\\{\}]
C:\textbackslash{}Users\textbackslash{}dhulse\textbackslash{}Documents\textbackslash{}GitHub\textbackslash{}fmdtools\textbackslash{}example\_tank\textbackslash{}..\textbackslash{}fmdtools\textbackslash{}modeldef.py:1879: RuntimeWarning: invalid value encountered in double\_scalars
  if len(overlap)>1:  self.rates\_timeless[jointmode][phaseid] = self.rates[jointmode][phaseid]/(overlap[1]-overlap[0])
SCENARIOS COMPLETE: 100\%|█████████████████████████████████████████████████████████████| 92/92 [00:00<00:00, 266.39it/s]
\end{sphinxVerbatim}



\end{sphinxuseclass}
\end{sphinxuseclass}
}

\end{sphinxuseclass}
\begin{sphinxuseclass}{nboutput}
\begin{sphinxuseclass}{nblast}
{

\kern-\sphinxverbatimsmallskipamount\kern-\baselineskip
\kern+\FrameHeightAdjust\kern-\fboxrule
\vspace{\nbsphinxcodecellspacing}

\sphinxsetup{VerbatimColor={named}{white}}
\sphinxsetup{VerbatimBorderColor={named}{nbsphinx-code-border}}
\begin{sphinxuseclass}{output_area}
\begin{sphinxuseclass}{}


\begin{sphinxVerbatim}[commandchars=\\\{\}]
\llap{\color{nbsphinxout}[10]:\,\hspace{\fboxrule}\hspace{\fboxsep}}                                                            rate       cost  \textbackslash{}
Import\_Water Stuck, t=0                             1.666667e-06        0.0
Guide\_Water\_In Leak, t=0                            1.666667e-06        0.0
Guide\_Water\_In Clogged, t=0                         1.666667e-06  1000000.0
Store\_Water Leak, t=0                               1.666667e-06        0.0
Guide\_Water\_Out Leak, t=0                           1.666667e-06  1000000.0
{\ldots}                                                          {\ldots}        {\ldots}
Export\_Water: Stuck, Human: FalseDetection\_low,{\ldots}  1.388878e-11        0.0
Export\_Water: Stuck, Human: FalseDetection\_high{\ldots}  1.388878e-11        0.0
Human FalseDetection\_low, t=10                      1.583333e-04        0.0
Human FalseDetection\_high, t=10                     1.583333e-04        0.0
nominal                                             1.000000e+00        0.0

                                                    expected cost
Import\_Water Stuck, t=0                                  0.000000
Guide\_Water\_In Leak, t=0                                 0.000000
Guide\_Water\_In Clogged, t=0                         166666.666667
Store\_Water Leak, t=0                                    0.000000
Guide\_Water\_Out Leak, t=0                           166666.666667
{\ldots}                                                           {\ldots}
Export\_Water: Stuck, Human: FalseDetection\_low,{\ldots}       0.000000
Export\_Water: Stuck, Human: FalseDetection\_high{\ldots}       0.000000
Human FalseDetection\_low, t=10                           0.000000
Human FalseDetection\_high, t=10                          0.000000
nominal                                                  0.000000

[93 rows x 3 columns]
\end{sphinxVerbatim}



\end{sphinxuseclass}
\end{sphinxuseclass}
}

\end{sphinxuseclass}
\end{sphinxuseclass}
\sphinxAtStartPar
Next, we can filter out non\sphinxhyphen{}failures and sort by the failures with the highest expected cost (though rate would give the same results here)

\begin{sphinxuseclass}{nbinput}
{
\sphinxsetup{VerbatimColor={named}{nbsphinx-code-bg}}
\sphinxsetup{VerbatimBorderColor={named}{nbsphinx-code-border}}
\begin{sphinxVerbatim}[commandchars=\\\{\}]
\llap{\color{nbsphinxin}[11]:\,\hspace{\fboxrule}\hspace{\fboxsep}}\PYG{n}{failure\PYGZus{}tab} \PYG{o}{=} \PYG{n}{fmea\PYGZus{}tab}\PYG{p}{[}\PYG{n}{fmea\PYGZus{}tab}\PYG{p}{[}\PYG{l+s+s1}{\PYGZsq{}}\PYG{l+s+s1}{cost}\PYG{l+s+s1}{\PYGZsq{}}\PYG{p}{]} \PYG{o}{\PYGZgt{}} \PYG{l+m+mi}{1}\PYG{p}{]}
\PYG{n}{failure\PYGZus{}tab}\PYG{o}{.}\PYG{n}{sort\PYGZus{}values}\PYG{p}{(}\PYG{l+s+s1}{\PYGZsq{}}\PYG{l+s+s1}{expected cost}\PYG{l+s+s1}{\PYGZsq{}}\PYG{p}{,} \PYG{n}{ascending} \PYG{o}{=} \PYG{k+kc}{False}\PYG{p}{)}
\end{sphinxVerbatim}
}

\end{sphinxuseclass}
\begin{sphinxuseclass}{nboutput}
\begin{sphinxuseclass}{nblast}
{

\kern-\sphinxverbatimsmallskipamount\kern-\baselineskip
\kern+\FrameHeightAdjust\kern-\fboxrule
\vspace{\nbsphinxcodecellspacing}

\sphinxsetup{VerbatimColor={named}{white}}
\sphinxsetup{VerbatimBorderColor={named}{nbsphinx-code-border}}
\begin{sphinxuseclass}{output_area}
\begin{sphinxuseclass}{}


\begin{sphinxVerbatim}[commandchars=\\\{\}]
\llap{\color{nbsphinxout}[11]:\,\hspace{\fboxrule}\hspace{\fboxsep}}                                                            rate       cost  \textbackslash{}
Guide\_Water\_In Clogged, t=0                         1.666667e-06  1000000.0
Guide\_Water\_Out Leak, t=0                           1.666667e-06  1000000.0
Guide\_Water\_In: Leak, Human: CannotTurn, t=0        4.034949e-07  1000000.0
Guide\_Water\_Out: Clogged, Human: CannotTurn, t=0    4.034949e-07  1000000.0
Store\_Water: Leak, Human: CannotTurn, t=0           4.034949e-07  1000000.0
Guide\_Water\_In: Clogged, Human: CannotTurn, t=0     4.034949e-07  1000000.0
Guide\_Water\_Out: Leak, Human: CannotTurn, t=0       4.034949e-07  1000000.0
Store\_Water: Leak, Human: NotDetected, t=0          3.444339e-07  1000000.0
Guide\_Water\_In: Clogged, Human: NotDetected, t=0    3.444339e-07  1000000.0
Guide\_Water\_Out: Leak, Human: NotDetected, t=0      3.444339e-07  1000000.0
Guide\_Water\_In: Leak, Human: NotDetected, t=0       3.444339e-07  1000000.0
Guide\_Water\_Out: Clogged, Human: NotDetected, t=0   3.444339e-07  1000000.0
Guide\_Water\_In: Leak, Human: FalseReach, t=0        1.840245e-07  1000000.0
Guide\_Water\_In: Clogged, Human: FalseReach, t=0     1.840245e-07  1000000.0
Guide\_Water\_Out: Clogged, Human: CannotReach, t=0   1.840245e-07  1000000.0
Guide\_Water\_Out: Leak, Human: CannotReach, t=0      1.840245e-07  1000000.0
Guide\_Water\_Out: Clogged, Human: FalseReach, t=0    1.840245e-07  1000000.0
Guide\_Water\_In: Leak, Human: CannotReach, t=0       1.840245e-07  1000000.0
Guide\_Water\_In: Clogged, Human: CannotReach, t=0    1.840245e-07  1000000.0
Store\_Water: Leak, Human: FalseReach, t=0           1.840245e-07  1000000.0
Store\_Water: Leak, Human: CannotReach, t=0          1.840245e-07  1000000.0
Guide\_Water\_Out: Leak, Human: FalseReach, t=0       1.840245e-07  1000000.0
Guide\_Water\_In: Clogged, Human: NotVisible, t=0     1.271420e-07  1000000.0
Guide\_Water\_Out: Leak, Human: NotVisible, t=0       1.271420e-07  1000000.0
Guide\_Water\_In: Leak, Human: NotVisible, t=0        1.271420e-07  1000000.0
Store\_Water: Leak, Human: NotVisible, t=0           1.271420e-07  1000000.0
Guide\_Water\_Out: Clogged, Human: NotVisible, t=0    1.271420e-07  1000000.0
Store\_Water: Leak, Human: CannotGrasp, t=0          3.300218e-08  1000000.0
Guide\_Water\_Out: Clogged, Human: CannotGrasp, t=0   3.300218e-08  1000000.0
Guide\_Water\_In: Clogged, Human: CannotGrasp, t=0    3.300218e-08  1000000.0
Guide\_Water\_In: Leak, Human: CannotGrasp, t=0       3.300218e-08  1000000.0
Guide\_Water\_Out: Leak, Human: CannotGrasp, t=0      3.300218e-08  1000000.0
Guide\_Water\_Out: Leak, Human: FalseDetection\_lo{\ldots}  1.388878e-11  1000000.0
Guide\_Water\_Out: Leak, Human: FalseDetection\_hi{\ldots}  1.388878e-11  1000000.0
Guide\_Water\_In: Clogged, Human: FalseDetection\_{\ldots}  1.388878e-11  1000000.0
Store\_Water: Leak, Human: FalseDetection\_high, t=0  1.388878e-11  1000000.0
Guide\_Water\_In: Clogged, Human: FalseDetection\_{\ldots}  1.388878e-11  1000000.0
Guide\_Water\_In: Leak, Human: FalseDetection\_hig{\ldots}  1.388878e-11  1000000.0
Guide\_Water\_Out: Clogged, Human: FalseDetection{\ldots}  1.388878e-11  1000000.0
Guide\_Water\_Out: Leak, Export\_Water: Stuck, t=0     2.777778e-12  1000000.0
Guide\_Water\_In: Clogged, Export\_Water: Stuck, t=0   2.777778e-12  1000000.0
Guide\_Water\_In: Clogged, Guide\_Water\_Out: Leak,{\ldots}  2.777778e-12  1000000.0
Guide\_Water\_In: Clogged, Store\_Water: Leak, t=0     2.777778e-12  1000000.0
Guide\_Water\_In: Leak, Store\_Water: Leak, t=0        2.777778e-12  1000000.0
Import\_Water: Stuck, Guide\_Water\_Out: Clogged, t=0  2.777778e-12  1000000.0
Import\_Water: Stuck, Guide\_Water\_Out: Leak, t=0     2.777778e-12  1000000.0
Import\_Water: Stuck, Store\_Water: Leak, t=0         2.777778e-12  1000000.0
Import\_Water: Stuck, Guide\_Water\_In: Clogged, t=0   2.777778e-12  1000000.0
Import\_Water: Stuck, Guide\_Water\_In: Leak, t=0      2.777778e-12  1000000.0

                                                    expected cost
Guide\_Water\_In Clogged, t=0                         166666.666667
Guide\_Water\_Out Leak, t=0                           166666.666667
Guide\_Water\_In: Leak, Human: CannotTurn, t=0         40349.490951
Guide\_Water\_Out: Clogged, Human: CannotTurn, t=0     40349.490951
Store\_Water: Leak, Human: CannotTurn, t=0            40349.490951
Guide\_Water\_In: Clogged, Human: CannotTurn, t=0      40349.490951
Guide\_Water\_Out: Leak, Human: CannotTurn, t=0        40349.490951
Store\_Water: Leak, Human: NotDetected, t=0           34443.390210
Guide\_Water\_In: Clogged, Human: NotDetected, t=0     34443.390210
Guide\_Water\_Out: Leak, Human: NotDetected, t=0       34443.390210
Guide\_Water\_In: Leak, Human: NotDetected, t=0        34443.390210
Guide\_Water\_Out: Clogged, Human: NotDetected, t=0    34443.390210
Guide\_Water\_In: Leak, Human: FalseReach, t=0         18402.454166
Guide\_Water\_In: Clogged, Human: FalseReach, t=0      18402.454166
Guide\_Water\_Out: Clogged, Human: CannotReach, t=0    18402.454166
Guide\_Water\_Out: Leak, Human: CannotReach, t=0       18402.454166
Guide\_Water\_Out: Clogged, Human: FalseReach, t=0     18402.454166
Guide\_Water\_In: Leak, Human: CannotReach, t=0        18402.454166
Guide\_Water\_In: Clogged, Human: CannotReach, t=0     18402.454166
Store\_Water: Leak, Human: FalseReach, t=0            18402.454166
Store\_Water: Leak, Human: CannotReach, t=0           18402.454166
Guide\_Water\_Out: Leak, Human: FalseReach, t=0        18402.454166
Guide\_Water\_In: Clogged, Human: NotVisible, t=0      12714.203610
Guide\_Water\_Out: Leak, Human: NotVisible, t=0        12714.203610
Guide\_Water\_In: Leak, Human: NotVisible, t=0         12714.203610
Store\_Water: Leak, Human: NotVisible, t=0            12714.203610
Guide\_Water\_Out: Clogged, Human: NotVisible, t=0     12714.203610
Store\_Water: Leak, Human: CannotGrasp, t=0            3300.218420
Guide\_Water\_Out: Clogged, Human: CannotGrasp, t=0     3300.218420
Guide\_Water\_In: Clogged, Human: CannotGrasp, t=0      3300.218420
Guide\_Water\_In: Leak, Human: CannotGrasp, t=0         3300.218420
Guide\_Water\_Out: Leak, Human: CannotGrasp, t=0        3300.218420
Guide\_Water\_Out: Leak, Human: FalseDetection\_lo{\ldots}       1.388878
Guide\_Water\_Out: Leak, Human: FalseDetection\_hi{\ldots}       1.388878
Guide\_Water\_In: Clogged, Human: FalseDetection\_{\ldots}       1.388878
Store\_Water: Leak, Human: FalseDetection\_high, t=0       1.388878
Guide\_Water\_In: Clogged, Human: FalseDetection\_{\ldots}       1.388878
Guide\_Water\_In: Leak, Human: FalseDetection\_hig{\ldots}       1.388878
Guide\_Water\_Out: Clogged, Human: FalseDetection{\ldots}       1.388878
Guide\_Water\_Out: Leak, Export\_Water: Stuck, t=0          0.277778
Guide\_Water\_In: Clogged, Export\_Water: Stuck, t=0        0.277778
Guide\_Water\_In: Clogged, Guide\_Water\_Out: Leak,{\ldots}       0.277778
Guide\_Water\_In: Clogged, Store\_Water: Leak, t=0          0.277778
Guide\_Water\_In: Leak, Store\_Water: Leak, t=0             0.277778
Import\_Water: Stuck, Guide\_Water\_Out: Clogged, t=0       0.277778
Import\_Water: Stuck, Guide\_Water\_Out: Leak, t=0          0.277778
Import\_Water: Stuck, Store\_Water: Leak, t=0              0.277778
Import\_Water: Stuck, Guide\_Water\_In: Clogged, t=0        0.277778
Import\_Water: Stuck, Guide\_Water\_In: Leak, t=0           0.277778
\end{sphinxVerbatim}



\end{sphinxuseclass}
\end{sphinxuseclass}
}

\end{sphinxuseclass}
\end{sphinxuseclass}
\sphinxAtStartPar
One of the top modes is a joint human\sphinxhyphen{}component failure mode. Let’s see what happens in this case:

\begin{sphinxuseclass}{nbinput}
{
\sphinxsetup{VerbatimColor={named}{nbsphinx-code-bg}}
\sphinxsetup{VerbatimBorderColor={named}{nbsphinx-code-border}}
\begin{sphinxVerbatim}[commandchars=\\\{\}]
\llap{\color{nbsphinxin}[12]:\,\hspace{\fboxrule}\hspace{\fboxsep}}\PYG{n}{scenhists} \PYG{o}{=}\PYG{p}{\PYGZob{}}\PYG{l+s+s1}{\PYGZsq{}}\PYG{l+s+s1}{nominal}\PYG{l+s+s1}{\PYGZsq{}}\PYG{p}{:}\PYG{n}{mdlhists}\PYG{p}{[}\PYG{l+s+s1}{\PYGZsq{}}\PYG{l+s+s1}{nominal}\PYG{l+s+s1}{\PYGZsq{}}\PYG{p}{]}\PYG{p}{,} \PYG{l+s+s1}{\PYGZsq{}}\PYG{l+s+s1}{faulty}\PYG{l+s+s1}{\PYGZsq{}}\PYG{p}{:}\PYG{n}{mdlhists}\PYG{p}{[}\PYG{l+s+s1}{\PYGZsq{}}\PYG{l+s+s1}{Guide\PYGZus{}Water\PYGZus{}Out: Leak, Human: FalseDetection\PYGZus{}high, t=0}\PYG{l+s+s1}{\PYGZsq{}}\PYG{p}{]}\PYG{p}{\PYGZcb{}}
\PYG{n}{rd}\PYG{o}{.}\PYG{n}{plot}\PYG{o}{.}\PYG{n}{mdlhistvals}\PYG{p}{(}\PYG{n}{scenhists}\PYG{p}{)}
\end{sphinxVerbatim}
}

\end{sphinxuseclass}
\begin{sphinxuseclass}{nboutput}
\hrule height -\fboxrule\relax
\vspace{\nbsphinxcodecellspacing}

\savebox\nbsphinxpromptbox[0pt][r]{\color{nbsphinxout}\Verb|\strut{[12]:}\,|}

\begin{nbsphinxfancyoutput}

\begin{sphinxuseclass}{output_area}
\begin{sphinxuseclass}{}
\noindent\sphinxincludegraphics[width=426\sphinxpxdimen,height=1197\sphinxpxdimen]{{example_tank_Tank_Analysis_19_0}.png}

\end{sphinxuseclass}
\end{sphinxuseclass}
\end{nbsphinxfancyoutput}

\end{sphinxuseclass}
\begin{sphinxuseclass}{nboutput}
\begin{sphinxuseclass}{nblast}
\hrule height -\fboxrule\relax
\vspace{\nbsphinxcodecellspacing}

\makeatletter\setbox\nbsphinxpromptbox\box\voidb@x\makeatother

\begin{nbsphinxfancyoutput}

\begin{sphinxuseclass}{output_area}
\begin{sphinxuseclass}{}
\noindent\sphinxincludegraphics[width=426\sphinxpxdimen,height=1197\sphinxpxdimen]{{example_tank_Tank_Analysis_19_1}.png}

\end{sphinxuseclass}
\end{sphinxuseclass}
\end{nbsphinxfancyoutput}

\end{sphinxuseclass}
\end{sphinxuseclass}
\sphinxAtStartPar
In this case, there is a leak, but the operator cannot turn the valve, resulting in the tank filling too high, which is a failure.

\sphinxAtStartPar
To consider the leak again, we can see what happens when the leak is not detected:

\begin{sphinxuseclass}{nbinput}
{
\sphinxsetup{VerbatimColor={named}{nbsphinx-code-bg}}
\sphinxsetup{VerbatimBorderColor={named}{nbsphinx-code-border}}
\begin{sphinxVerbatim}[commandchars=\\\{\}]
\llap{\color{nbsphinxin}[13]:\,\hspace{\fboxrule}\hspace{\fboxsep}}\PYG{n}{scenhists} \PYG{o}{=}\PYG{p}{\PYGZob{}}\PYG{l+s+s1}{\PYGZsq{}}\PYG{l+s+s1}{nominal}\PYG{l+s+s1}{\PYGZsq{}}\PYG{p}{:}\PYG{n}{mdlhists}\PYG{p}{[}\PYG{l+s+s1}{\PYGZsq{}}\PYG{l+s+s1}{nominal}\PYG{l+s+s1}{\PYGZsq{}}\PYG{p}{]}\PYG{p}{,} \PYG{l+s+s1}{\PYGZsq{}}\PYG{l+s+s1}{faulty}\PYG{l+s+s1}{\PYGZsq{}}\PYG{p}{:}\PYG{n}{mdlhists}\PYG{p}{[}\PYG{l+s+s1}{\PYGZsq{}}\PYG{l+s+s1}{Guide\PYGZus{}Water\PYGZus{}Out: Leak, Human: FalseDetection\PYGZus{}low, t=0}\PYG{l+s+s1}{\PYGZsq{}}\PYG{p}{]}\PYG{p}{\PYGZcb{}}
\PYG{n}{rd}\PYG{o}{.}\PYG{n}{plot}\PYG{o}{.}\PYG{n}{mdlhistvals}\PYG{p}{(}\PYG{n}{scenhists}\PYG{p}{)}
\end{sphinxVerbatim}
}

\end{sphinxuseclass}
\begin{sphinxuseclass}{nboutput}
\hrule height -\fboxrule\relax
\vspace{\nbsphinxcodecellspacing}

\savebox\nbsphinxpromptbox[0pt][r]{\color{nbsphinxout}\Verb|\strut{[13]:}\,|}

\begin{nbsphinxfancyoutput}

\begin{sphinxuseclass}{output_area}
\begin{sphinxuseclass}{}
\noindent\sphinxincludegraphics[width=426\sphinxpxdimen,height=1197\sphinxpxdimen]{{example_tank_Tank_Analysis_22_0}.png}

\end{sphinxuseclass}
\end{sphinxuseclass}
\end{nbsphinxfancyoutput}

\end{sphinxuseclass}
\begin{sphinxuseclass}{nboutput}
\begin{sphinxuseclass}{nblast}
\hrule height -\fboxrule\relax
\vspace{\nbsphinxcodecellspacing}

\makeatletter\setbox\nbsphinxpromptbox\box\voidb@x\makeatother

\begin{nbsphinxfancyoutput}

\begin{sphinxuseclass}{output_area}
\begin{sphinxuseclass}{}
\noindent\sphinxincludegraphics[width=426\sphinxpxdimen,height=1197\sphinxpxdimen]{{example_tank_Tank_Analysis_22_1}.png}

\end{sphinxuseclass}
\end{sphinxuseclass}
\end{nbsphinxfancyoutput}

\end{sphinxuseclass}
\end{sphinxuseclass}
\sphinxAtStartPar
In this case, there is a leak, but it is not caugh, resulting in a failure again.


\subsubsection{Testing different reaction times}
\label{\detokenize{example_tank/Tank_Analysis:Testing-different-reaction-times}}
\sphinxAtStartPar
The model set up in \sphinxcode{\sphinxupquote{tank\_model}} is parameterized by the reaction time of the operator. As a result, we can assess how long or short reaction times affect the given scenarios.

\begin{sphinxuseclass}{nbinput}
\begin{sphinxuseclass}{nblast}
{
\sphinxsetup{VerbatimColor={named}{nbsphinx-code-bg}}
\sphinxsetup{VerbatimBorderColor={named}{nbsphinx-code-border}}
\begin{sphinxVerbatim}[commandchars=\\\{\}]
\llap{\color{nbsphinxin}[14]:\,\hspace{\fboxrule}\hspace{\fboxsep}}\PYG{n}{mdl\PYGZus{}long\PYGZus{}reaction\PYGZus{}time} \PYG{o}{=} \PYG{n}{Tank}\PYG{p}{(}\PYG{n}{params}\PYG{o}{=}\PYG{p}{\PYGZob{}}\PYG{l+s+s1}{\PYGZsq{}}\PYG{l+s+s1}{reacttime}\PYG{l+s+s1}{\PYGZsq{}}\PYG{p}{:}\PYG{l+m+mi}{10}\PYG{p}{\PYGZcb{}}\PYG{p}{)}
\end{sphinxVerbatim}
}

\end{sphinxuseclass}
\end{sphinxuseclass}
\sphinxAtStartPar
In this case, we will show the affect of reaction time on the operator’s ability to catch a leak.

\begin{sphinxuseclass}{nbinput}
{
\sphinxsetup{VerbatimColor={named}{nbsphinx-code-bg}}
\sphinxsetup{VerbatimBorderColor={named}{nbsphinx-code-border}}
\begin{sphinxVerbatim}[commandchars=\\\{\}]
\llap{\color{nbsphinxin}[15]:\,\hspace{\fboxrule}\hspace{\fboxsep}}\PYG{n}{endresults}\PYG{p}{,} \PYG{n}{resgraph}\PYG{p}{,} \PYG{n}{mdlhist} \PYG{o}{=} \PYG{n}{propagate}\PYG{o}{.}\PYG{n}{one\PYGZus{}fault}\PYG{p}{(}\PYG{n}{mdl\PYGZus{}long\PYGZus{}reaction\PYGZus{}time}\PYG{p}{,}\PYG{l+s+s1}{\PYGZsq{}}\PYG{l+s+s1}{Store\PYGZus{}Water}\PYG{l+s+s1}{\PYGZsq{}}\PYG{p}{,}\PYG{l+s+s1}{\PYGZsq{}}\PYG{l+s+s1}{Leak}\PYG{l+s+s1}{\PYGZsq{}}\PYG{p}{,} \PYG{n}{time}\PYG{o}{=}\PYG{l+m+mi}{3}\PYG{p}{)}

\PYG{n}{rd}\PYG{o}{.}\PYG{n}{plot}\PYG{o}{.}\PYG{n}{mdlhistvals}\PYG{p}{(}\PYG{n}{mdlhist}\PYG{p}{,} \PYG{n}{fault}\PYG{o}{=}\PYG{l+s+s1}{\PYGZsq{}}\PYG{l+s+s1}{Leak}\PYG{l+s+s1}{\PYGZsq{}}\PYG{p}{,} \PYG{n}{time}\PYG{o}{=}\PYG{l+m+mi}{3}\PYG{p}{,} \PYG{n}{fxnflowvals}\PYG{o}{=}\PYG{p}{\PYGZob{}}\PYG{l+s+s1}{\PYGZsq{}}\PYG{l+s+s1}{Store\PYGZus{}Water}\PYG{l+s+s1}{\PYGZsq{}}\PYG{p}{:}\PYG{p}{[}\PYG{l+s+s1}{\PYGZsq{}}\PYG{l+s+s1}{level}\PYG{l+s+s1}{\PYGZsq{}}\PYG{p}{,} \PYG{l+s+s1}{\PYGZsq{}}\PYG{l+s+s1}{net\PYGZus{}flow}\PYG{l+s+s1}{\PYGZsq{}}\PYG{p}{]}\PYG{p}{,} \PYG{l+s+s1}{\PYGZsq{}}\PYG{l+s+s1}{Tank\PYGZus{}Sig}\PYG{l+s+s1}{\PYGZsq{}}\PYG{p}{:}\PYG{p}{[}\PYG{l+s+s1}{\PYGZsq{}}\PYG{l+s+s1}{indicator}\PYG{l+s+s1}{\PYGZsq{}}\PYG{p}{]}\PYG{p}{,} \PYG{l+s+s1}{\PYGZsq{}}\PYG{l+s+s1}{Valve1\PYGZus{}Sig}\PYG{l+s+s1}{\PYGZsq{}}\PYG{p}{:}\PYG{p}{[}\PYG{l+s+s1}{\PYGZsq{}}\PYG{l+s+s1}{action}\PYG{l+s+s1}{\PYGZsq{}}\PYG{p}{]}\PYG{p}{\PYGZcb{}}\PYG{p}{,} \PYG{n}{legend}\PYG{o}{=}\PYG{k+kc}{False}\PYG{p}{)}
\PYG{n}{rd}\PYG{o}{.}\PYG{n}{graph}\PYG{o}{.}\PYG{n}{show}\PYG{p}{(}\PYG{n}{resgraph}\PYG{p}{,}\PYG{n}{faultscen}\PYG{o}{=}\PYG{l+s+s1}{\PYGZsq{}}\PYG{l+s+s1}{Leak}\PYG{l+s+s1}{\PYGZsq{}}\PYG{p}{,} \PYG{n}{time}\PYG{o}{=}\PYG{l+m+mi}{3}\PYG{p}{)}

\end{sphinxVerbatim}
}

\end{sphinxuseclass}
\begin{sphinxuseclass}{nboutput}
{

\kern-\sphinxverbatimsmallskipamount\kern-\baselineskip
\kern+\FrameHeightAdjust\kern-\fboxrule
\vspace{\nbsphinxcodecellspacing}

\sphinxsetup{VerbatimColor={named}{white}}
\sphinxsetup{VerbatimBorderColor={named}{nbsphinx-code-border}}
\begin{sphinxuseclass}{output_area}
\begin{sphinxuseclass}{}


\begin{sphinxVerbatim}[commandchars=\\\{\}]
\llap{\color{nbsphinxout}[15]:\,\hspace{\fboxrule}\hspace{\fboxsep}}(<Figure size 432x288 with 1 Axes>,
 <AxesSubplot:title=\{'center':'Propagation of faults to Leak at t=3'\}>)
\end{sphinxVerbatim}



\end{sphinxuseclass}
\end{sphinxuseclass}
}

\end{sphinxuseclass}
\begin{sphinxuseclass}{nboutput}
\hrule height -\fboxrule\relax
\vspace{\nbsphinxcodecellspacing}

\makeatletter\setbox\nbsphinxpromptbox\box\voidb@x\makeatother

\begin{nbsphinxfancyoutput}

\begin{sphinxuseclass}{output_area}
\begin{sphinxuseclass}{}
\noindent\sphinxincludegraphics[width=462\sphinxpxdimen,height=286\sphinxpxdimen]{{example_tank_Tank_Analysis_27_1}.png}

\end{sphinxuseclass}
\end{sphinxuseclass}
\end{nbsphinxfancyoutput}

\end{sphinxuseclass}
\begin{sphinxuseclass}{nboutput}
\begin{sphinxuseclass}{nblast}
\hrule height -\fboxrule\relax
\vspace{\nbsphinxcodecellspacing}

\makeatletter\setbox\nbsphinxpromptbox\box\voidb@x\makeatother

\begin{nbsphinxfancyoutput}

\begin{sphinxuseclass}{output_area}
\begin{sphinxuseclass}{}
\noindent\sphinxincludegraphics[width=349\sphinxpxdimen,height=247\sphinxpxdimen]{{example_tank_Tank_Analysis_27_2}.png}

\end{sphinxuseclass}
\end{sphinxuseclass}
\end{nbsphinxfancyoutput}

\end{sphinxuseclass}
\end{sphinxuseclass}
\sphinxAtStartPar
As shown, the operator does not respond in time, resulting in the tank draining all the way, a failure state. We can further use the \sphinxcode{\sphinxupquote{propagate.approach}} function to compare the number of scenarios caught in this system compared with the other.

\begin{sphinxuseclass}{nbinput}
\begin{sphinxuseclass}{nblast}
{
\sphinxsetup{VerbatimColor={named}{nbsphinx-code-bg}}
\sphinxsetup{VerbatimBorderColor={named}{nbsphinx-code-border}}
\begin{sphinxVerbatim}[commandchars=\\\{\}]
\llap{\color{nbsphinxin}[ ]:\,\hspace{\fboxrule}\hspace{\fboxsep}}
\end{sphinxVerbatim}
}

\end{sphinxuseclass}
\end{sphinxuseclass}

\subsection{EPS Example Notebook}
\label{\detokenize{example_eps/EPS_Example_Notebook:EPS-Example-Notebook}}\label{\detokenize{example_eps/EPS_Example_Notebook::doc}}
\sphinxAtStartPar
This notebook shows an example replicating previous the simple electric power system implemented in \sphinxhref{https://github.com/DesignEngrLab/IBFM}{IBFM} in the \sphinxcode{\sphinxupquote{eps example}} directory, with some basic fault propagation and visualization.

\begin{sphinxuseclass}{nbinput}
\begin{sphinxuseclass}{nblast}
{
\sphinxsetup{VerbatimColor={named}{nbsphinx-code-bg}}
\sphinxsetup{VerbatimBorderColor={named}{nbsphinx-code-border}}
\begin{sphinxVerbatim}[commandchars=\\\{\}]
\llap{\color{nbsphinxin}[1]:\,\hspace{\fboxrule}\hspace{\fboxsep}}\PYG{k+kn}{import} \PYG{n+nn}{sys}\PYG{o}{,} \PYG{n+nn}{os}
\PYG{n}{sys}\PYG{o}{.}\PYG{n}{path}\PYG{o}{.}\PYG{n}{insert}\PYG{p}{(}\PYG{l+m+mi}{1}\PYG{p}{,}\PYG{n}{os}\PYG{o}{.}\PYG{n}{path}\PYG{o}{.}\PYG{n}{join}\PYG{p}{(}\PYG{l+s+s2}{\PYGZdq{}}\PYG{l+s+s2}{..}\PYG{l+s+s2}{\PYGZdq{}}\PYG{p}{)}\PYG{p}{)}
\PYG{k+kn}{from} \PYG{n+nn}{eps} \PYG{k+kn}{import} \PYG{n}{EPS}
\PYG{k+kn}{import} \PYG{n+nn}{fmdtools}\PYG{n+nn}{.}\PYG{n+nn}{faultsim}\PYG{n+nn}{.}\PYG{n+nn}{propagate} \PYG{k}{as} \PYG{n+nn}{propagate}
\PYG{k+kn}{import} \PYG{n+nn}{fmdtools}\PYG{n+nn}{.}\PYG{n+nn}{resultdisp} \PYG{k}{as} \PYG{n+nn}{rd}
\end{sphinxVerbatim}
}

\end{sphinxuseclass}
\end{sphinxuseclass}
\sphinxAtStartPar
This script provides some example I/O for using static models, using the EPS system implemented in eps.py as an example.

\sphinxAtStartPar
A graphical representaiton of this system is shown below:

\begin{sphinxuseclass}{nbinput}
{
\sphinxsetup{VerbatimColor={named}{nbsphinx-code-bg}}
\sphinxsetup{VerbatimBorderColor={named}{nbsphinx-code-border}}
\begin{sphinxVerbatim}[commandchars=\\\{\}]
\llap{\color{nbsphinxin}[2]:\,\hspace{\fboxrule}\hspace{\fboxsep}}\PYG{n}{mdl}\PYG{o}{=} \PYG{n}{EPS}\PYG{p}{(}\PYG{p}{)}
\PYG{n}{rd}\PYG{o}{.}\PYG{n}{graph}\PYG{o}{.}\PYG{n}{show}\PYG{p}{(}\PYG{n}{mdl}\PYG{o}{.}\PYG{n}{bipartite}\PYG{p}{,} \PYG{n}{gtype}\PYG{o}{=}\PYG{l+s+s1}{\PYGZsq{}}\PYG{l+s+s1}{bipartite}\PYG{l+s+s1}{\PYGZsq{}}\PYG{p}{)}
\end{sphinxVerbatim}
}

\end{sphinxuseclass}
\begin{sphinxuseclass}{nboutput}
{

\kern-\sphinxverbatimsmallskipamount\kern-\baselineskip
\kern+\FrameHeightAdjust\kern-\fboxrule
\vspace{\nbsphinxcodecellspacing}

\sphinxsetup{VerbatimColor={named}{white}}
\sphinxsetup{VerbatimBorderColor={named}{nbsphinx-code-border}}
\begin{sphinxuseclass}{output_area}
\begin{sphinxuseclass}{}


\begin{sphinxVerbatim}[commandchars=\\\{\}]
\llap{\color{nbsphinxout}[2]:\,\hspace{\fboxrule}\hspace{\fboxsep}}(<Figure size 432x288 with 1 Axes>, <AxesSubplot:>)
\end{sphinxVerbatim}



\end{sphinxuseclass}
\end{sphinxuseclass}
}

\end{sphinxuseclass}
\begin{sphinxuseclass}{nboutput}
\begin{sphinxuseclass}{nblast}
\hrule height -\fboxrule\relax
\vspace{\nbsphinxcodecellspacing}

\makeatletter\setbox\nbsphinxpromptbox\box\voidb@x\makeatother

\begin{nbsphinxfancyoutput}

\begin{sphinxuseclass}{output_area}
\begin{sphinxuseclass}{}
\noindent\sphinxincludegraphics[width=349\sphinxpxdimen,height=231\sphinxpxdimen]{{example_eps_EPS_Example_Notebook_3_1}.png}

\end{sphinxuseclass}
\end{sphinxuseclass}
\end{nbsphinxfancyoutput}

\end{sphinxuseclass}
\end{sphinxuseclass}
\sphinxAtStartPar
As with dynamic models, in static models we use \sphinxcode{\sphinxupquote{fp.run\_one\_fault}} to see the effects of single faults. All setup is performed in the Model class definition

\begin{sphinxuseclass}{nbinput}
\begin{sphinxuseclass}{nblast}
{
\sphinxsetup{VerbatimColor={named}{nbsphinx-code-bg}}
\sphinxsetup{VerbatimBorderColor={named}{nbsphinx-code-border}}
\begin{sphinxVerbatim}[commandchars=\\\{\}]
\llap{\color{nbsphinxin}[3]:\,\hspace{\fboxrule}\hspace{\fboxsep}}\PYG{n}{endresults}\PYG{p}{,}\PYG{n}{resgraph}\PYG{p}{,} \PYG{n}{mdlhist} \PYG{o}{=} \PYG{n}{propagate}\PYG{o}{.}\PYG{n}{one\PYGZus{}fault}\PYG{p}{(}\PYG{n}{mdl}\PYG{p}{,} \PYG{l+s+s1}{\PYGZsq{}}\PYG{l+s+s1}{EE\PYGZus{}to\PYGZus{}ME}\PYG{l+s+s1}{\PYGZsq{}}\PYG{p}{,} \PYG{l+s+s1}{\PYGZsq{}}\PYG{l+s+s1}{toohigh\PYGZus{}torque}\PYG{l+s+s1}{\PYGZsq{}}\PYG{p}{)}
\end{sphinxVerbatim}
}

\end{sphinxuseclass}
\end{sphinxuseclass}
\sphinxAtStartPar
In this case, however, the output in \sphinxcode{\sphinxupquote{mdlhist}} will be a single\sphinxhyphen{}dimensional dictionary (not something we can plot very well)

\begin{sphinxuseclass}{nbinput}
{
\sphinxsetup{VerbatimColor={named}{nbsphinx-code-bg}}
\sphinxsetup{VerbatimBorderColor={named}{nbsphinx-code-border}}
\begin{sphinxVerbatim}[commandchars=\\\{\}]
\llap{\color{nbsphinxin}[4]:\,\hspace{\fboxrule}\hspace{\fboxsep}}\PYG{n}{rd}\PYG{o}{.}\PYG{n}{plot}\PYG{o}{.}\PYG{n}{mdlhist}\PYG{p}{(}\PYG{n}{mdlhist}\PYG{p}{)}
\end{sphinxVerbatim}
}

\end{sphinxuseclass}
\begin{sphinxuseclass}{nboutput}
{

\kern-\sphinxverbatimsmallskipamount\kern-\baselineskip
\kern+\FrameHeightAdjust\kern-\fboxrule
\vspace{\nbsphinxcodecellspacing}

\sphinxsetup{VerbatimColor={named}{nbsphinx-stderr}}
\sphinxsetup{VerbatimBorderColor={named}{nbsphinx-code-border}}
\begin{sphinxuseclass}{output_area}
\begin{sphinxuseclass}{stderr}


\begin{sphinxVerbatim}[commandchars=\\\{\}]
C:\textbackslash{}Users\textbackslash{}dhulse\textbackslash{}Documents\textbackslash{}GitHub\textbackslash{}fmdtools\textbackslash{}example\_eps\textbackslash{}..\textbackslash{}fmdtools\textbackslash{}resultdisp\textbackslash{}plot.py:120: RuntimeWarning: More than 20 figures have been opened. Figures created through the pyplot interface (`matplotlib.pyplot.figure`) are retained until explicitly closed and may consume too much memory. (To control this warning, see the rcParam `figure.max\_open\_warning`).
  fig = plt.figure(figsize=(cols*3, 2*num\_plots/cols))
\end{sphinxVerbatim}



\end{sphinxuseclass}
\end{sphinxuseclass}
}

\end{sphinxuseclass}
\begin{sphinxuseclass}{nboutput}
{

\kern-\sphinxverbatimsmallskipamount\kern-\baselineskip
\kern+\FrameHeightAdjust\kern-\fboxrule
\vspace{\nbsphinxcodecellspacing}

\sphinxsetup{VerbatimColor={named}{white}}
\sphinxsetup{VerbatimBorderColor={named}{nbsphinx-code-border}}
\begin{sphinxuseclass}{output_area}
\begin{sphinxuseclass}{}


\begin{sphinxVerbatim}[commandchars=\\\{\}]
<Figure size 432x144 with 0 Axes>
\end{sphinxVerbatim}



\end{sphinxuseclass}
\end{sphinxuseclass}
}

\end{sphinxuseclass}
\begin{sphinxuseclass}{nboutput}
{

\kern-\sphinxverbatimsmallskipamount\kern-\baselineskip
\kern+\FrameHeightAdjust\kern-\fboxrule
\vspace{\nbsphinxcodecellspacing}

\sphinxsetup{VerbatimColor={named}{white}}
\sphinxsetup{VerbatimBorderColor={named}{nbsphinx-code-border}}
\begin{sphinxuseclass}{output_area}
\begin{sphinxuseclass}{}


\begin{sphinxVerbatim}[commandchars=\\\{\}]
<Figure size 432x144 with 0 Axes>
\end{sphinxVerbatim}



\end{sphinxuseclass}
\end{sphinxuseclass}
}

\end{sphinxuseclass}
\begin{sphinxuseclass}{nboutput}
{

\kern-\sphinxverbatimsmallskipamount\kern-\baselineskip
\kern+\FrameHeightAdjust\kern-\fboxrule
\vspace{\nbsphinxcodecellspacing}

\sphinxsetup{VerbatimColor={named}{white}}
\sphinxsetup{VerbatimBorderColor={named}{nbsphinx-code-border}}
\begin{sphinxuseclass}{output_area}
\begin{sphinxuseclass}{}


\begin{sphinxVerbatim}[commandchars=\\\{\}]
<Figure size 432x144 with 0 Axes>
\end{sphinxVerbatim}



\end{sphinxuseclass}
\end{sphinxuseclass}
}

\end{sphinxuseclass}
\begin{sphinxuseclass}{nboutput}
{

\kern-\sphinxverbatimsmallskipamount\kern-\baselineskip
\kern+\FrameHeightAdjust\kern-\fboxrule
\vspace{\nbsphinxcodecellspacing}

\sphinxsetup{VerbatimColor={named}{white}}
\sphinxsetup{VerbatimBorderColor={named}{nbsphinx-code-border}}
\begin{sphinxuseclass}{output_area}
\begin{sphinxuseclass}{}


\begin{sphinxVerbatim}[commandchars=\\\{\}]
<Figure size 432x144 with 0 Axes>
\end{sphinxVerbatim}



\end{sphinxuseclass}
\end{sphinxuseclass}
}

\end{sphinxuseclass}
\begin{sphinxuseclass}{nboutput}
{

\kern-\sphinxverbatimsmallskipamount\kern-\baselineskip
\kern+\FrameHeightAdjust\kern-\fboxrule
\vspace{\nbsphinxcodecellspacing}

\sphinxsetup{VerbatimColor={named}{white}}
\sphinxsetup{VerbatimBorderColor={named}{nbsphinx-code-border}}
\begin{sphinxuseclass}{output_area}
\begin{sphinxuseclass}{}


\begin{sphinxVerbatim}[commandchars=\\\{\}]
<Figure size 432x144 with 0 Axes>
\end{sphinxVerbatim}



\end{sphinxuseclass}
\end{sphinxuseclass}
}

\end{sphinxuseclass}
\begin{sphinxuseclass}{nboutput}
{

\kern-\sphinxverbatimsmallskipamount\kern-\baselineskip
\kern+\FrameHeightAdjust\kern-\fboxrule
\vspace{\nbsphinxcodecellspacing}

\sphinxsetup{VerbatimColor={named}{white}}
\sphinxsetup{VerbatimBorderColor={named}{nbsphinx-code-border}}
\begin{sphinxuseclass}{output_area}
\begin{sphinxuseclass}{}


\begin{sphinxVerbatim}[commandchars=\\\{\}]
<Figure size 432x144 with 0 Axes>
\end{sphinxVerbatim}



\end{sphinxuseclass}
\end{sphinxuseclass}
}

\end{sphinxuseclass}
\begin{sphinxuseclass}{nboutput}
{

\kern-\sphinxverbatimsmallskipamount\kern-\baselineskip
\kern+\FrameHeightAdjust\kern-\fboxrule
\vspace{\nbsphinxcodecellspacing}

\sphinxsetup{VerbatimColor={named}{white}}
\sphinxsetup{VerbatimBorderColor={named}{nbsphinx-code-border}}
\begin{sphinxuseclass}{output_area}
\begin{sphinxuseclass}{}


\begin{sphinxVerbatim}[commandchars=\\\{\}]
<Figure size 432x144 with 0 Axes>
\end{sphinxVerbatim}



\end{sphinxuseclass}
\end{sphinxuseclass}
}

\end{sphinxuseclass}
\begin{sphinxuseclass}{nboutput}
{

\kern-\sphinxverbatimsmallskipamount\kern-\baselineskip
\kern+\FrameHeightAdjust\kern-\fboxrule
\vspace{\nbsphinxcodecellspacing}

\sphinxsetup{VerbatimColor={named}{white}}
\sphinxsetup{VerbatimBorderColor={named}{nbsphinx-code-border}}
\begin{sphinxuseclass}{output_area}
\begin{sphinxuseclass}{}


\begin{sphinxVerbatim}[commandchars=\\\{\}]
<Figure size 432x144 with 0 Axes>
\end{sphinxVerbatim}



\end{sphinxuseclass}
\end{sphinxuseclass}
}

\end{sphinxuseclass}
\begin{sphinxuseclass}{nboutput}
{

\kern-\sphinxverbatimsmallskipamount\kern-\baselineskip
\kern+\FrameHeightAdjust\kern-\fboxrule
\vspace{\nbsphinxcodecellspacing}

\sphinxsetup{VerbatimColor={named}{white}}
\sphinxsetup{VerbatimBorderColor={named}{nbsphinx-code-border}}
\begin{sphinxuseclass}{output_area}
\begin{sphinxuseclass}{}


\begin{sphinxVerbatim}[commandchars=\\\{\}]
<Figure size 432x144 with 0 Axes>
\end{sphinxVerbatim}



\end{sphinxuseclass}
\end{sphinxuseclass}
}

\end{sphinxuseclass}
\begin{sphinxuseclass}{nboutput}
{

\kern-\sphinxverbatimsmallskipamount\kern-\baselineskip
\kern+\FrameHeightAdjust\kern-\fboxrule
\vspace{\nbsphinxcodecellspacing}

\sphinxsetup{VerbatimColor={named}{white}}
\sphinxsetup{VerbatimBorderColor={named}{nbsphinx-code-border}}
\begin{sphinxuseclass}{output_area}
\begin{sphinxuseclass}{}


\begin{sphinxVerbatim}[commandchars=\\\{\}]
<Figure size 432x144 with 0 Axes>
\end{sphinxVerbatim}



\end{sphinxuseclass}
\end{sphinxuseclass}
}

\end{sphinxuseclass}
\begin{sphinxuseclass}{nboutput}
{

\kern-\sphinxverbatimsmallskipamount\kern-\baselineskip
\kern+\FrameHeightAdjust\kern-\fboxrule
\vspace{\nbsphinxcodecellspacing}

\sphinxsetup{VerbatimColor={named}{white}}
\sphinxsetup{VerbatimBorderColor={named}{nbsphinx-code-border}}
\begin{sphinxuseclass}{output_area}
\begin{sphinxuseclass}{}


\begin{sphinxVerbatim}[commandchars=\\\{\}]
<Figure size 432x144 with 0 Axes>
\end{sphinxVerbatim}



\end{sphinxuseclass}
\end{sphinxuseclass}
}

\end{sphinxuseclass}
\begin{sphinxuseclass}{nboutput}
{

\kern-\sphinxverbatimsmallskipamount\kern-\baselineskip
\kern+\FrameHeightAdjust\kern-\fboxrule
\vspace{\nbsphinxcodecellspacing}

\sphinxsetup{VerbatimColor={named}{white}}
\sphinxsetup{VerbatimBorderColor={named}{nbsphinx-code-border}}
\begin{sphinxuseclass}{output_area}
\begin{sphinxuseclass}{}


\begin{sphinxVerbatim}[commandchars=\\\{\}]
<Figure size 432x144 with 0 Axes>
\end{sphinxVerbatim}



\end{sphinxuseclass}
\end{sphinxuseclass}
}

\end{sphinxuseclass}
\begin{sphinxuseclass}{nboutput}
\hrule height -\fboxrule\relax
\vspace{\nbsphinxcodecellspacing}

\makeatletter\setbox\nbsphinxpromptbox\box\voidb@x\makeatother

\begin{nbsphinxfancyoutput}

\begin{sphinxuseclass}{output_area}
\begin{sphinxuseclass}{}
\noindent\sphinxincludegraphics[width=425\sphinxpxdimen,height=216\sphinxpxdimen]{{example_eps_EPS_Example_Notebook_7_13}.png}

\end{sphinxuseclass}
\end{sphinxuseclass}
\end{nbsphinxfancyoutput}

\end{sphinxuseclass}
\begin{sphinxuseclass}{nboutput}
\hrule height -\fboxrule\relax
\vspace{\nbsphinxcodecellspacing}

\makeatletter\setbox\nbsphinxpromptbox\box\voidb@x\makeatother

\begin{nbsphinxfancyoutput}

\begin{sphinxuseclass}{output_area}
\begin{sphinxuseclass}{}
\noindent\sphinxincludegraphics[width=425\sphinxpxdimen,height=216\sphinxpxdimen]{{example_eps_EPS_Example_Notebook_7_14}.png}

\end{sphinxuseclass}
\end{sphinxuseclass}
\end{nbsphinxfancyoutput}

\end{sphinxuseclass}
\begin{sphinxuseclass}{nboutput}
\hrule height -\fboxrule\relax
\vspace{\nbsphinxcodecellspacing}

\makeatletter\setbox\nbsphinxpromptbox\box\voidb@x\makeatother

\begin{nbsphinxfancyoutput}

\begin{sphinxuseclass}{output_area}
\begin{sphinxuseclass}{}
\noindent\sphinxincludegraphics[width=425\sphinxpxdimen,height=216\sphinxpxdimen]{{example_eps_EPS_Example_Notebook_7_15}.png}

\end{sphinxuseclass}
\end{sphinxuseclass}
\end{nbsphinxfancyoutput}

\end{sphinxuseclass}
\begin{sphinxuseclass}{nboutput}
\hrule height -\fboxrule\relax
\vspace{\nbsphinxcodecellspacing}

\makeatletter\setbox\nbsphinxpromptbox\box\voidb@x\makeatother

\begin{nbsphinxfancyoutput}

\begin{sphinxuseclass}{output_area}
\begin{sphinxuseclass}{}
\noindent\sphinxincludegraphics[width=425\sphinxpxdimen,height=216\sphinxpxdimen]{{example_eps_EPS_Example_Notebook_7_16}.png}

\end{sphinxuseclass}
\end{sphinxuseclass}
\end{nbsphinxfancyoutput}

\end{sphinxuseclass}
\begin{sphinxuseclass}{nboutput}
\hrule height -\fboxrule\relax
\vspace{\nbsphinxcodecellspacing}

\makeatletter\setbox\nbsphinxpromptbox\box\voidb@x\makeatother

\begin{nbsphinxfancyoutput}

\begin{sphinxuseclass}{output_area}
\begin{sphinxuseclass}{}
\noindent\sphinxincludegraphics[width=425\sphinxpxdimen,height=216\sphinxpxdimen]{{example_eps_EPS_Example_Notebook_7_17}.png}

\end{sphinxuseclass}
\end{sphinxuseclass}
\end{nbsphinxfancyoutput}

\end{sphinxuseclass}
\begin{sphinxuseclass}{nboutput}
\hrule height -\fboxrule\relax
\vspace{\nbsphinxcodecellspacing}

\makeatletter\setbox\nbsphinxpromptbox\box\voidb@x\makeatother

\begin{nbsphinxfancyoutput}

\begin{sphinxuseclass}{output_area}
\begin{sphinxuseclass}{}
\noindent\sphinxincludegraphics[width=425\sphinxpxdimen,height=216\sphinxpxdimen]{{example_eps_EPS_Example_Notebook_7_18}.png}

\end{sphinxuseclass}
\end{sphinxuseclass}
\end{nbsphinxfancyoutput}

\end{sphinxuseclass}
\begin{sphinxuseclass}{nboutput}
\hrule height -\fboxrule\relax
\vspace{\nbsphinxcodecellspacing}

\makeatletter\setbox\nbsphinxpromptbox\box\voidb@x\makeatother

\begin{nbsphinxfancyoutput}

\begin{sphinxuseclass}{output_area}
\begin{sphinxuseclass}{}
\noindent\sphinxincludegraphics[width=422\sphinxpxdimen,height=216\sphinxpxdimen]{{example_eps_EPS_Example_Notebook_7_19}.png}

\end{sphinxuseclass}
\end{sphinxuseclass}
\end{nbsphinxfancyoutput}

\end{sphinxuseclass}
\begin{sphinxuseclass}{nboutput}
\hrule height -\fboxrule\relax
\vspace{\nbsphinxcodecellspacing}

\makeatletter\setbox\nbsphinxpromptbox\box\voidb@x\makeatother

\begin{nbsphinxfancyoutput}

\begin{sphinxuseclass}{output_area}
\begin{sphinxuseclass}{}
\noindent\sphinxincludegraphics[width=425\sphinxpxdimen,height=216\sphinxpxdimen]{{example_eps_EPS_Example_Notebook_7_20}.png}

\end{sphinxuseclass}
\end{sphinxuseclass}
\end{nbsphinxfancyoutput}

\end{sphinxuseclass}
\begin{sphinxuseclass}{nboutput}
\hrule height -\fboxrule\relax
\vspace{\nbsphinxcodecellspacing}

\makeatletter\setbox\nbsphinxpromptbox\box\voidb@x\makeatother

\begin{nbsphinxfancyoutput}

\begin{sphinxuseclass}{output_area}
\begin{sphinxuseclass}{}
\noindent\sphinxincludegraphics[width=425\sphinxpxdimen,height=216\sphinxpxdimen]{{example_eps_EPS_Example_Notebook_7_21}.png}

\end{sphinxuseclass}
\end{sphinxuseclass}
\end{nbsphinxfancyoutput}

\end{sphinxuseclass}
\begin{sphinxuseclass}{nboutput}
\hrule height -\fboxrule\relax
\vspace{\nbsphinxcodecellspacing}

\makeatletter\setbox\nbsphinxpromptbox\box\voidb@x\makeatother

\begin{nbsphinxfancyoutput}

\begin{sphinxuseclass}{output_area}
\begin{sphinxuseclass}{}
\noindent\sphinxincludegraphics[width=425\sphinxpxdimen,height=216\sphinxpxdimen]{{example_eps_EPS_Example_Notebook_7_22}.png}

\end{sphinxuseclass}
\end{sphinxuseclass}
\end{nbsphinxfancyoutput}

\end{sphinxuseclass}
\begin{sphinxuseclass}{nboutput}
\hrule height -\fboxrule\relax
\vspace{\nbsphinxcodecellspacing}

\makeatletter\setbox\nbsphinxpromptbox\box\voidb@x\makeatother

\begin{nbsphinxfancyoutput}

\begin{sphinxuseclass}{output_area}
\begin{sphinxuseclass}{}
\noindent\sphinxincludegraphics[width=425\sphinxpxdimen,height=216\sphinxpxdimen]{{example_eps_EPS_Example_Notebook_7_23}.png}

\end{sphinxuseclass}
\end{sphinxuseclass}
\end{nbsphinxfancyoutput}

\end{sphinxuseclass}
\begin{sphinxuseclass}{nboutput}
\hrule height -\fboxrule\relax
\vspace{\nbsphinxcodecellspacing}

\makeatletter\setbox\nbsphinxpromptbox\box\voidb@x\makeatother

\begin{nbsphinxfancyoutput}

\begin{sphinxuseclass}{output_area}
\begin{sphinxuseclass}{}
\noindent\sphinxincludegraphics[width=425\sphinxpxdimen,height=216\sphinxpxdimen]{{example_eps_EPS_Example_Notebook_7_24}.png}

\end{sphinxuseclass}
\end{sphinxuseclass}
\end{nbsphinxfancyoutput}

\end{sphinxuseclass}
\begin{sphinxuseclass}{nboutput}
\begin{sphinxuseclass}{nblast}
\hrule height -\fboxrule\relax
\vspace{\nbsphinxcodecellspacing}

\makeatletter\setbox\nbsphinxpromptbox\box\voidb@x\makeatother

\begin{nbsphinxfancyoutput}

\begin{sphinxuseclass}{output_area}
\begin{sphinxuseclass}{}
\noindent\sphinxincludegraphics[width=420\sphinxpxdimen,height=145\sphinxpxdimen]{{example_eps_EPS_Example_Notebook_7_25}.png}

\end{sphinxuseclass}
\end{sphinxuseclass}
\end{nbsphinxfancyoutput}

\end{sphinxuseclass}
\end{sphinxuseclass}
\sphinxAtStartPar
As a result, it’s better to look at the results graph for a visualization of what went wrong. In this case \sphinxcode{\sphinxupquote{resgraph}} better represents the fault propagation of the system than in a dynamic model, since there is only one time\sphinxhyphen{}step to represent (rather than a set)

\begin{sphinxuseclass}{nbinput}
{
\sphinxsetup{VerbatimColor={named}{nbsphinx-code-bg}}
\sphinxsetup{VerbatimBorderColor={named}{nbsphinx-code-border}}
\begin{sphinxVerbatim}[commandchars=\\\{\}]
\llap{\color{nbsphinxin}[5]:\,\hspace{\fboxrule}\hspace{\fboxsep}}\PYG{n}{rd}\PYG{o}{.}\PYG{n}{graph}\PYG{o}{.}\PYG{n}{show}\PYG{p}{(}\PYG{n}{resgraph}\PYG{p}{)}
\end{sphinxVerbatim}
}

\end{sphinxuseclass}
\begin{sphinxuseclass}{nboutput}
{

\kern-\sphinxverbatimsmallskipamount\kern-\baselineskip
\kern+\FrameHeightAdjust\kern-\fboxrule
\vspace{\nbsphinxcodecellspacing}

\sphinxsetup{VerbatimColor={named}{white}}
\sphinxsetup{VerbatimBorderColor={named}{nbsphinx-code-border}}
\begin{sphinxuseclass}{output_area}
\begin{sphinxuseclass}{}


\begin{sphinxVerbatim}[commandchars=\\\{\}]
\llap{\color{nbsphinxout}[5]:\,\hspace{\fboxrule}\hspace{\fboxsep}}(<Figure size 432x288 with 1 Axes>, <AxesSubplot:>)
\end{sphinxVerbatim}



\end{sphinxuseclass}
\end{sphinxuseclass}
}

\end{sphinxuseclass}
\begin{sphinxuseclass}{nboutput}
\begin{sphinxuseclass}{nblast}
\hrule height -\fboxrule\relax
\vspace{\nbsphinxcodecellspacing}

\makeatletter\setbox\nbsphinxpromptbox\box\voidb@x\makeatother

\begin{nbsphinxfancyoutput}

\begin{sphinxuseclass}{output_area}
\begin{sphinxuseclass}{}
\noindent\sphinxincludegraphics[width=349\sphinxpxdimen,height=231\sphinxpxdimen]{{example_eps_EPS_Example_Notebook_9_1}.png}

\end{sphinxuseclass}
\end{sphinxuseclass}
\end{nbsphinxfancyoutput}

\end{sphinxuseclass}
\end{sphinxuseclass}
\sphinxAtStartPar
We can run the set of single\sphinxhyphen{}fault scenarios on this model using \sphinxcode{\sphinxupquote{fmdtools.faultsim.propagate.single\_faults}}. For single\sphinxhyphen{}fault scenarios, one does not need to use a \sphinxcode{\sphinxupquote{SampleApproach}}, since all faults are injected at a single time\sphinxhyphen{}step.

\begin{sphinxuseclass}{nbinput}
{
\sphinxsetup{VerbatimColor={named}{nbsphinx-code-bg}}
\sphinxsetup{VerbatimBorderColor={named}{nbsphinx-code-border}}
\begin{sphinxVerbatim}[commandchars=\\\{\}]
\llap{\color{nbsphinxin}[6]:\,\hspace{\fboxrule}\hspace{\fboxsep}}\PYG{n}{endclasses}\PYG{p}{,} \PYG{n}{mdlhists} \PYG{o}{=} \PYG{n}{propagate}\PYG{o}{.}\PYG{n}{single\PYGZus{}faults}\PYG{p}{(}\PYG{n}{mdl}\PYG{p}{)}
\end{sphinxVerbatim}
}

\end{sphinxuseclass}
\begin{sphinxuseclass}{nboutput}
\begin{sphinxuseclass}{nblast}
{

\kern-\sphinxverbatimsmallskipamount\kern-\baselineskip
\kern+\FrameHeightAdjust\kern-\fboxrule
\vspace{\nbsphinxcodecellspacing}

\sphinxsetup{VerbatimColor={named}{nbsphinx-stderr}}
\sphinxsetup{VerbatimBorderColor={named}{nbsphinx-code-border}}
\begin{sphinxuseclass}{output_area}
\begin{sphinxuseclass}{stderr}


\begin{sphinxVerbatim}[commandchars=\\\{\}]
SCENARIOS COMPLETE: 100\%|█████████████████████████████████████████████████████████████| 35/35 [00:00<00:00, 385.80it/s]
\end{sphinxVerbatim}



\end{sphinxuseclass}
\end{sphinxuseclass}
}

\end{sphinxuseclass}
\end{sphinxuseclass}
\sphinxAtStartPar
Using \sphinxcode{\sphinxupquote{make\_summarytable}}, one can see the degradation effects of this fault on the flows:

\begin{sphinxuseclass}{nbinput}
{
\sphinxsetup{VerbatimColor={named}{nbsphinx-code-bg}}
\sphinxsetup{VerbatimBorderColor={named}{nbsphinx-code-border}}
\begin{sphinxVerbatim}[commandchars=\\\{\}]
\llap{\color{nbsphinxin}[7]:\,\hspace{\fboxrule}\hspace{\fboxsep}}\PYG{n}{reshists}\PYG{p}{,} \PYG{n}{diffs}\PYG{p}{,} \PYG{n}{summary} \PYG{o}{=} \PYG{n}{rd}\PYG{o}{.}\PYG{n}{process}\PYG{o}{.}\PYG{n}{hists}\PYG{p}{(}\PYG{n}{mdlhists}\PYG{p}{)}
\PYG{n}{sumtable} \PYG{o}{=} \PYG{n}{rd}\PYG{o}{.}\PYG{n}{tabulate}\PYG{o}{.}\PYG{n}{summary}\PYG{p}{(}\PYG{n}{summary}\PYG{p}{)}
\PYG{n}{sumtable}
\end{sphinxVerbatim}
}

\end{sphinxuseclass}
\begin{sphinxuseclass}{nboutput}
\begin{sphinxuseclass}{nblast}
{

\kern-\sphinxverbatimsmallskipamount\kern-\baselineskip
\kern+\FrameHeightAdjust\kern-\fboxrule
\vspace{\nbsphinxcodecellspacing}

\sphinxsetup{VerbatimColor={named}{white}}
\sphinxsetup{VerbatimBorderColor={named}{nbsphinx-code-border}}
\begin{sphinxuseclass}{output_area}
\begin{sphinxuseclass}{}


\begin{sphinxVerbatim}[commandchars=\\\{\}]
\llap{\color{nbsphinxout}[7]:\,\hspace{\fboxrule}\hspace{\fboxsep}}                                                         degraded functions  \textbackslash{}
Import\_EE low\_v, t=1                                            [Import\_EE]
Import\_EE high\_v, t=1                                 [Import\_EE, Store\_EE]
Import\_EE no\_v, t=1                                             [Import\_EE]
Supply\_EE adverse\_resist, t=1                                   [Supply\_EE]
Supply\_EE minor\_overload, t=1          [Supply\_EE, Store\_EE, Distribute\_EE]
Supply\_EE major\_overload, t=1                         [Supply\_EE, Store\_EE]
Supply\_EE short, t=1                   [Supply\_EE, Store\_EE, Distribute\_EE]
Supply\_EE open\_circuit, t=1                                     [Supply\_EE]
Store\_EE low\_storage, t=1                                        [Store\_EE]
Store\_EE no\_storage, t=1                                         [Store\_EE]
Import\_Signal partial\_signal, t=1                           [Import\_Signal]
Import\_Signal no\_signal, t=1                                [Import\_Signal]
Distribute\_EE adverse\_resist, t=1                           [Distribute\_EE]
Distribute\_EE poor\_alloc, t=1                               [Distribute\_EE]
Distribute\_EE short, t=1                          [Store\_EE, Distribute\_EE]
Distribute\_EE open\_circuit, t=1                             [Distribute\_EE]
EE\_to\_ME high\_torque, t=1                                        [EE\_to\_ME]
EE\_to\_ME low\_torque, t=1                [Store\_EE, Distribute\_EE, EE\_to\_ME]
EE\_to\_ME toohigh\_torque, t=1            [Store\_EE, Distribute\_EE, EE\_to\_ME]
EE\_to\_ME open\_circuit, t=1                                       [EE\_to\_ME]
EE\_to\_ME short, t=1                     [Store\_EE, Distribute\_EE, EE\_to\_ME]
EE\_to\_OE optical\_resist, t=1                                     [EE\_to\_OE]
EE\_to\_OE burnt\_out, t=1                                          [EE\_to\_OE]
EE\_to\_HE low\_heat, t=1                                           [EE\_to\_HE]
EE\_to\_HE high\_heat, t=1                     [Supply\_EE, Store\_EE, EE\_to\_HE]
EE\_to\_HE toohigh\_heat, t=1              [Store\_EE, Distribute\_EE, EE\_to\_HE]
EE\_to\_HE open\_circuit, t=1                                       [EE\_to\_HE]
Export\_HE hot\_sink, t=1                                         [Export\_HE]
Export\_HE ineffective\_sink, t=1                                 [Export\_HE]
Export\_waste\_H1 hot\_sink, t=1                             [Export\_waste\_H1]
Export\_waste\_H1 ineffective\_sink, t=1                     [Export\_waste\_H1]
Export\_waste\_HO hot\_sink, t=1                             [Export\_waste\_HO]
Export\_waste\_HO ineffective\_sink, t=1                     [Export\_waste\_HO]
Export\_waste\_HM hot\_sink, t=1                             [Export\_waste\_HM]
Export\_waste\_HM ineffective\_sink, t=1                     [Export\_waste\_HM]

                                                                          degraded flows
Import\_EE low\_v, t=1                   [EE\_1, EE\_2, EE\_3, EE\_M, EE\_O, EE\_H, ME, OE, w{\ldots}
Import\_EE high\_v, t=1                  [EE\_1, EE\_2, EE\_3, EE\_M, EE\_O, EE\_H, ME, OE, w{\ldots}
Import\_EE no\_v, t=1                    [EE\_1, EE\_2, EE\_3, EE\_M, EE\_O, EE\_H, ME, OE, w{\ldots}
Supply\_EE adverse\_resist, t=1          [EE\_2, EE\_3, EE\_M, EE\_O, EE\_H, ME, OE, waste\_H{\ldots}
Supply\_EE minor\_overload, t=1          [EE\_2, EE\_3, EE\_M, EE\_O, EE\_H, ME, OE, waste\_H{\ldots}
Supply\_EE major\_overload, t=1          [EE\_2, EE\_3, EE\_M, EE\_O, EE\_H, ME, OE, waste\_H{\ldots}
Supply\_EE short, t=1                   [EE\_1, EE\_2, EE\_3, EE\_M, EE\_O, EE\_H, ME, OE, w{\ldots}
Supply\_EE open\_circuit, t=1            [EE\_2, EE\_3, EE\_M, EE\_O, EE\_H, ME, OE, waste\_H{\ldots}
Store\_EE low\_storage, t=1                                                             []
Store\_EE no\_storage, t=1                    [EE\_3, EE\_M, EE\_O, EE\_H, ME, OE, waste\_HE\_M]
Import\_Signal partial\_signal, t=1      [EE\_2, EE\_3, EE\_M, EE\_O, EE\_H, ME, OE, waste\_H{\ldots}
Import\_Signal no\_signal, t=1           [EE\_2, EE\_3, EE\_M, EE\_O, EE\_H, ME, OE, waste\_H{\ldots}
Distribute\_EE adverse\_resist, t=1      [EE\_2, EE\_3, EE\_M, EE\_O, EE\_H, ME, OE, waste\_H{\ldots}
Distribute\_EE poor\_alloc, t=1          [EE\_2, EE\_3, EE\_M, EE\_O, EE\_H, ME, OE, waste\_H{\ldots}
Distribute\_EE short, t=1                    [EE\_3, EE\_M, EE\_O, EE\_H, ME, OE, waste\_HE\_M]
Distribute\_EE open\_circuit, t=1        [EE\_2, EE\_3, EE\_M, EE\_O, EE\_H, ME, OE, waste\_H{\ldots}
EE\_to\_ME high\_torque, t=1                                         [EE\_M, ME, waste\_HE\_M]
EE\_to\_ME low\_torque, t=1                    [EE\_3, EE\_M, EE\_O, EE\_H, ME, OE, waste\_HE\_M]
EE\_to\_ME toohigh\_torque, t=1           [EE\_3, EE\_M, EE\_O, EE\_H, ME, OE, waste\_HE\_O, w{\ldots}
EE\_to\_ME open\_circuit, t=1                                        [EE\_M, ME, waste\_HE\_M]
EE\_to\_ME short, t=1                    [EE\_3, EE\_M, EE\_O, EE\_H, ME, OE, waste\_HE\_O, w{\ldots}
EE\_to\_OE optical\_resist, t=1                                                  [EE\_O, OE]
EE\_to\_OE burnt\_out, t=1                                                       [EE\_O, OE]
EE\_to\_HE low\_heat, t=1                                                                []
EE\_to\_HE high\_heat, t=1                                               [EE\_2, EE\_3, EE\_H]
EE\_to\_HE toohigh\_heat, t=1             [EE\_3, EE\_M, EE\_O, EE\_H, ME, OE, waste\_HE\_O, w{\ldots}
EE\_to\_HE open\_circuit, t=1                                                        [EE\_H]
Export\_HE hot\_sink, t=1                                                             [HE]
Export\_HE ineffective\_sink, t=1                                                     [HE]
Export\_waste\_H1 hot\_sink, t=1                                                         []
Export\_waste\_H1 ineffective\_sink, t=1                                                 []
Export\_waste\_HO hot\_sink, t=1                                               [waste\_HE\_O]
Export\_waste\_HO ineffective\_sink, t=1                                       [waste\_HE\_O]
Export\_waste\_HM hot\_sink, t=1                                               [waste\_HE\_M]
Export\_waste\_HM ineffective\_sink, t=1                                       [waste\_HE\_M]
\end{sphinxVerbatim}



\end{sphinxuseclass}
\end{sphinxuseclass}
}

\end{sphinxuseclass}
\end{sphinxuseclass}
\sphinxAtStartPar
Given the value model in find\_classification, we can finaly use this to make a simple fmea table:

\begin{sphinxuseclass}{nbinput}
{
\sphinxsetup{VerbatimColor={named}{nbsphinx-code-bg}}
\sphinxsetup{VerbatimBorderColor={named}{nbsphinx-code-border}}
\begin{sphinxVerbatim}[commandchars=\\\{\}]
\llap{\color{nbsphinxin}[8]:\,\hspace{\fboxrule}\hspace{\fboxsep}}\PYG{n}{rd}\PYG{o}{.}\PYG{n}{tabulate}\PYG{o}{.}\PYG{n}{simplefmea}\PYG{p}{(}\PYG{n}{endclasses}\PYG{p}{)}
\end{sphinxVerbatim}
}

\end{sphinxuseclass}
\begin{sphinxuseclass}{nboutput}
\begin{sphinxuseclass}{nblast}
{

\kern-\sphinxverbatimsmallskipamount\kern-\baselineskip
\kern+\FrameHeightAdjust\kern-\fboxrule
\vspace{\nbsphinxcodecellspacing}

\sphinxsetup{VerbatimColor={named}{white}}
\sphinxsetup{VerbatimBorderColor={named}{nbsphinx-code-border}}
\begin{sphinxuseclass}{output_area}
\begin{sphinxuseclass}{}


\begin{sphinxVerbatim}[commandchars=\\\{\}]
\llap{\color{nbsphinxout}[8]:\,\hspace{\fboxrule}\hspace{\fboxsep}}                                               rate    cost  expected cost
Import\_EE low\_v, t=1                   1.000000e-05   600.0        262.800
Import\_EE high\_v, t=1                  5.000000e-06  3000.0        657.000
Import\_EE no\_v, t=1                    1.000000e-05  1200.0        525.600
Supply\_EE adverse\_resist, t=1          2.000000e-06  1300.0        113.880
Supply\_EE minor\_overload, t=1          1.000000e-05  4800.0       2102.400
Supply\_EE major\_overload, t=1          3.000000e-06  3300.0        433.620
Supply\_EE short, t=1                   1.000000e-07  4800.0         21.024
Supply\_EE open\_circuit, t=1            5.000000e-08  1100.0          2.409
Store\_EE low\_storage, t=1              5.000000e-06  2000.0        438.000
Store\_EE no\_storage, t=1               5.000000e-06  2900.0        635.100
Import\_Signal partial\_signal, t=1      1.000000e-05  1650.0        722.700
Import\_Signal no\_signal, t=1           1.000000e-06  1650.0         72.270
Distribute\_EE adverse\_resist, t=1      1.000000e-05  2400.0       1051.200
Distribute\_EE poor\_alloc, t=1          2.000000e-05  1400.0       1226.400
Distribute\_EE short, t=1               2.000000e-05  4400.0       3854.400
Distribute\_EE open\_circuit, t=1        3.000000e-05  2400.0       3153.600
EE\_to\_ME high\_torque, t=1              1.000000e-04   450.0       1971.000
EE\_to\_ME low\_torque, t=1               1.000000e-04  4600.0      20148.000
EE\_to\_ME toohigh\_torque, t=1           5.000000e-05  4700.0      10293.000
EE\_to\_ME open\_circuit, t=1             5.000000e-05   650.0       1423.500
EE\_to\_ME short, t=1                    5.000000e-05  4600.0      10074.000
EE\_to\_OE optical\_resist, t=1           5.000000e-07   520.0         11.388
EE\_to\_OE burnt\_out, t=1                2.000000e-06   550.0         48.180
EE\_to\_HE low\_heat, t=1                 2.000000e-06   200.0         17.520
EE\_to\_HE high\_heat, t=1                1.000000e-07  2400.0         10.512
EE\_to\_HE toohigh\_heat, t=1             5.000000e-07  4600.0        100.740
EE\_to\_HE open\_circuit, t=1             1.000000e-07   200.0          0.876
Export\_HE hot\_sink, t=1                1.000000e-05   600.0        262.800
Export\_HE ineffective\_sink, t=1        5.000000e-06  1500.0        328.500
Export\_waste\_H1 hot\_sink, t=1          1.000000e-05   500.0        219.000
Export\_waste\_H1 ineffective\_sink, t=1  5.000000e-06  1000.0        219.000
Export\_waste\_HO hot\_sink, t=1          1.000000e-05   500.0        219.000
Export\_waste\_HO ineffective\_sink, t=1  5.000000e-06  1000.0        219.000
Export\_waste\_HM hot\_sink, t=1          1.000000e-05   500.0        219.000
Export\_waste\_HM ineffective\_sink, t=1  5.000000e-06  1000.0        219.000
\end{sphinxVerbatim}



\end{sphinxuseclass}
\end{sphinxuseclass}
}

\end{sphinxuseclass}
\end{sphinxuseclass}
\begin{sphinxuseclass}{nbinput}
\begin{sphinxuseclass}{nblast}
{
\sphinxsetup{VerbatimColor={named}{nbsphinx-code-bg}}
\sphinxsetup{VerbatimBorderColor={named}{nbsphinx-code-border}}
\begin{sphinxVerbatim}[commandchars=\\\{\}]
\llap{\color{nbsphinxin}[ ]:\,\hspace{\fboxrule}\hspace{\fboxsep}}
\end{sphinxVerbatim}
}

\end{sphinxuseclass}
\end{sphinxuseclass}

\section{Module Reference}
\label{\detokenize{docs/fmdtools:module-reference}}\label{\detokenize{docs/fmdtools::doc}}
\noindent\sphinxincludegraphics[width=800\sphinxpxdimen]{{module_organization}.PNG}

\sphinxAtStartPar
The fmdtools package is split into three modules for design, simulation, and analysis, as shown above. The {\hyperref[\detokenize{docs/fmdtools:module-fmdtools.modeldef}]{\sphinxcrossref{\sphinxcode{\sphinxupquote{fmdtools.modeldef}}}}} module provides constructs for model and simulation definition, the \sphinxcode{\sphinxupquote{fmdtools.faultsim}} subpackage provides functions to simulate these models and the  the \sphinxcode{\sphinxupquote{fmdtools.resultdisp}} subpackage provides functions to analyze and visualize the results of these simulations.

\sphinxAtStartPar
Thus, working with fmdtools often means creating a model file which extends classes from {\hyperref[\detokenize{docs/fmdtools:module-fmdtools.modeldef}]{\sphinxcrossref{\sphinxcode{\sphinxupquote{fmdtools.modeldef}}}}}, and then simulating and analyzing that model in a script or notebook using the \sphinxcode{\sphinxupquote{fmdtools.faultsim}} and \sphinxcode{\sphinxupquote{fmdtools.resultdisp}} subpackages. This page provides references for the functions and classes in these modules.

\sphinxAtStartPar
\sphinxstylestrong{Submodule Links}


\subsection{fmdtools.faultsim}
\label{\detokenize{docs/fmdtools.faultsim:fmdtools-faultsim}}\label{\detokenize{docs/fmdtools.faultsim::doc}}
\sphinxAtStartPar
The \sphinxcode{\sphinxupquote{fmdtools.faultsim}} package is used to simulate {\hyperref[\detokenize{docs/fmdtools:fmdtools.modeldef.Model}]{\sphinxcrossref{\sphinxcode{\sphinxupquote{fmdtools.modeldef.Model}}}}} models. It consists of two modules:
\begin{itemize}
\item {} 
\sphinxAtStartPar
{\hyperref[\detokenize{docs/fmdtools.faultsim:module-fmdtools.faultsim.networks}]{\sphinxcrossref{\sphinxcode{\sphinxupquote{fmdtools.faultsim.networks}}}}}, which conducts network assessment on a given model’s network (which does not require classes for function blocks or behaviors to be defined), and

\item {} 
\sphinxAtStartPar
{\hyperref[\detokenize{docs/fmdtools.faultsim:module-fmdtools.faultsim.propagate}]{\sphinxcrossref{\sphinxcode{\sphinxupquote{fmdtools.faultsim.propagate}}}}}, which simulates the user\sphinxhyphen{}defined behaviors of a model over set time(s).

\end{itemize}


\subsubsection{fmdtools.faultsim.networks}
\label{\detokenize{docs/fmdtools.faultsim:module-fmdtools.faultsim.networks}}\label{\detokenize{docs/fmdtools.faultsim:fmdtools-faultsim-networks}}\index{module@\spxentry{module}!fmdtools.faultsim.networks@\spxentry{fmdtools.faultsim.networks}}\index{fmdtools.faultsim.networks@\spxentry{fmdtools.faultsim.networks}!module@\spxentry{module}}
\sphinxAtStartPar
Description: Methods for high\sphinxhyphen{}level network simulation and analysis.
\begin{description}
\item[{Main Methods:}] \leavevmode\begin{itemize}
\item {} 
\sphinxAtStartPar
{\hyperref[\detokenize{docs/fmdtools.faultsim:fmdtools.faultsim.networks.calc_aspl}]{\sphinxcrossref{\sphinxcode{\sphinxupquote{calc\_aspl()}}}}}:              Computes average shortest path length of graph representation of model mdl.

\item {} 
\sphinxAtStartPar
{\hyperref[\detokenize{docs/fmdtools.faultsim:fmdtools.faultsim.networks.calc_modularity}]{\sphinxcrossref{\sphinxcode{\sphinxupquote{calc\_modularity()}}}}}:        Computes graph modularity given a graph representation of model mdl.

\item {} 
\sphinxAtStartPar
{\hyperref[\detokenize{docs/fmdtools.faultsim:fmdtools.faultsim.networks.find_bridging_nodes}]{\sphinxcrossref{\sphinxcode{\sphinxupquote{find\_bridging\_nodes()}}}}}:    Determines bridging nodes in a graph representation of model mdl.

\item {} 
\sphinxAtStartPar
{\hyperref[\detokenize{docs/fmdtools.faultsim:fmdtools.faultsim.networks.find_high_degree_nodes}]{\sphinxcrossref{\sphinxcode{\sphinxupquote{find\_high\_degree\_nodes()}}}}}: Determines highest degree nodes, up to percentile p, in graph representation of model mdl.

\item {} 
\sphinxAtStartPar
{\hyperref[\detokenize{docs/fmdtools.faultsim:fmdtools.faultsim.networks.calc_robustness_coefficient}]{\sphinxcrossref{\sphinxcode{\sphinxupquote{calc\_robustness\_coefficient()}}}}}:    Computes robustness coefficient of graph representation of model mdl.

\item {} 
\sphinxAtStartPar
{\hyperref[\detokenize{docs/fmdtools.faultsim:fmdtools.faultsim.networks.sff_model}]{\sphinxcrossref{\sphinxcode{\sphinxupquote{sff\_model()}}}}}:                      Susceptible\sphinxhyphen{}Fixed\sphinxhyphen{}Failed Model Simulation

\item {} 
\sphinxAtStartPar
{\hyperref[\detokenize{docs/fmdtools.faultsim:fmdtools.faultsim.networks.degree_dist}]{\sphinxcrossref{\sphinxcode{\sphinxupquote{degree\_dist()}}}}}:            Plots degree distribution of graph representation of model mdl.

\end{itemize}

\end{description}
\index{calc\_aspl() (in module fmdtools.faultsim.networks)@\spxentry{calc\_aspl()}\spxextra{in module fmdtools.faultsim.networks}}

\begin{fulllineitems}
\phantomsection\label{\detokenize{docs/fmdtools.faultsim:fmdtools.faultsim.networks.calc_aspl}}\pysiglinewithargsret{\sphinxcode{\sphinxupquote{fmdtools.faultsim.networks.}}\sphinxbfcode{\sphinxupquote{calc\_aspl}}}{\emph{\DUrole{n}{mdl}}, \emph{\DUrole{n}{gtype}\DUrole{o}{=}\DUrole{default_value}{'parameter'}}}{}
\sphinxAtStartPar
Computes average shortest path length of graph representation of model mdl.
\begin{quote}\begin{description}
\item[{Parameters}] \leavevmode\begin{itemize}
\item {} 
\sphinxAtStartPar
\sphinxstyleliteralstrong{\sphinxupquote{mdl}} (\sphinxstyleliteralemphasis{\sphinxupquote{model}}\sphinxstyleliteralemphasis{\sphinxupquote{ or }}\sphinxstyleliteralemphasis{\sphinxupquote{graph}}) – graph to run the analysis for. (will get from model if provided)

\item {} 
\sphinxAtStartPar
\sphinxstyleliteralstrong{\sphinxupquote{gtype}} (\sphinxstyleliteralemphasis{\sphinxupquote{str}}) – type of graph representation of the model to show. default is ‘bipartite’

\end{itemize}

\item[{Returns}] \leavevmode
\sphinxAtStartPar
\sphinxstylestrong{ASPL}

\item[{Return type}] \leavevmode
\sphinxAtStartPar
average shortest path length

\end{description}\end{quote}

\end{fulllineitems}

\index{calc\_modularity() (in module fmdtools.faultsim.networks)@\spxentry{calc\_modularity()}\spxextra{in module fmdtools.faultsim.networks}}

\begin{fulllineitems}
\phantomsection\label{\detokenize{docs/fmdtools.faultsim:fmdtools.faultsim.networks.calc_modularity}}\pysiglinewithargsret{\sphinxcode{\sphinxupquote{fmdtools.faultsim.networks.}}\sphinxbfcode{\sphinxupquote{calc\_modularity}}}{\emph{\DUrole{n}{mdl}}, \emph{\DUrole{n}{gtype}\DUrole{o}{=}\DUrole{default_value}{'parameter'}}}{}
\sphinxAtStartPar
Computes graph modularity given a graph representation of model mdl.
\begin{quote}\begin{description}
\item[{Parameters}] \leavevmode\begin{itemize}
\item {} 
\sphinxAtStartPar
\sphinxstyleliteralstrong{\sphinxupquote{mdl}} (\sphinxstyleliteralemphasis{\sphinxupquote{model}}\sphinxstyleliteralemphasis{\sphinxupquote{ or }}\sphinxstyleliteralemphasis{\sphinxupquote{graph}}) – graph to run the analysis for. (will get from model if provided)

\item {} 
\sphinxAtStartPar
\sphinxstyleliteralstrong{\sphinxupquote{gtype}} (\sphinxstyleliteralemphasis{\sphinxupquote{str}}) – type of graph representation of the model to show. default is ‘bipartite’

\end{itemize}

\item[{Returns}] \leavevmode
\sphinxAtStartPar
\sphinxstylestrong{modularity}

\item[{Return type}] \leavevmode
\sphinxAtStartPar
Modularity

\end{description}\end{quote}

\end{fulllineitems}

\index{calc\_robustness\_coefficient() (in module fmdtools.faultsim.networks)@\spxentry{calc\_robustness\_coefficient()}\spxextra{in module fmdtools.faultsim.networks}}

\begin{fulllineitems}
\phantomsection\label{\detokenize{docs/fmdtools.faultsim:fmdtools.faultsim.networks.calc_robustness_coefficient}}\pysiglinewithargsret{\sphinxcode{\sphinxupquote{fmdtools.faultsim.networks.}}\sphinxbfcode{\sphinxupquote{calc\_robustness\_coefficient}}}{\emph{\DUrole{n}{mdl}}, \emph{\DUrole{n}{trials}\DUrole{o}{=}\DUrole{default_value}{100}}, \emph{\DUrole{n}{gtype}\DUrole{o}{=}\DUrole{default_value}{'bipartite'}}}{}
\sphinxAtStartPar
Computes robustness coefficient of graph representation of model mdl.
\begin{quote}\begin{description}
\item[{Parameters}] \leavevmode\begin{itemize}
\item {} 
\sphinxAtStartPar
\sphinxstyleliteralstrong{\sphinxupquote{mdl}} (\sphinxstyleliteralemphasis{\sphinxupquote{model}}\sphinxstyleliteralemphasis{\sphinxupquote{ or }}\sphinxstyleliteralemphasis{\sphinxupquote{graph}}) – graph to calculate robustness coefficent for. (will get from model if provided)

\item {} 
\sphinxAtStartPar
\sphinxstyleliteralstrong{\sphinxupquote{trials}} (\sphinxstyleliteralemphasis{\sphinxupquote{int}}) – number of times to run robustness coefficient algorithm (result is averaged over all trials)

\item {} 
\sphinxAtStartPar
\sphinxstyleliteralstrong{\sphinxupquote{gtype}} (\sphinxstyleliteralemphasis{\sphinxupquote{str}}) – type of graph representation of the model to show. default is ‘bipartite’

\end{itemize}

\item[{Returns}] \leavevmode
\sphinxAtStartPar
\sphinxstylestrong{RC}

\item[{Return type}] \leavevmode
\sphinxAtStartPar
robustness coefficient

\end{description}\end{quote}

\end{fulllineitems}

\index{data\_average() (in module fmdtools.faultsim.networks)@\spxentry{data\_average()}\spxextra{in module fmdtools.faultsim.networks}}

\begin{fulllineitems}
\phantomsection\label{\detokenize{docs/fmdtools.faultsim:fmdtools.faultsim.networks.data_average}}\pysiglinewithargsret{\sphinxcode{\sphinxupquote{fmdtools.faultsim.networks.}}\sphinxbfcode{\sphinxupquote{data\_average}}}{\emph{\DUrole{n}{data}}}{}
\sphinxAtStartPar
Averages each column in data

\end{fulllineitems}

\index{data\_error() (in module fmdtools.faultsim.networks)@\spxentry{data\_error()}\spxextra{in module fmdtools.faultsim.networks}}

\begin{fulllineitems}
\phantomsection\label{\detokenize{docs/fmdtools.faultsim:fmdtools.faultsim.networks.data_error}}\pysiglinewithargsret{\sphinxcode{\sphinxupquote{fmdtools.faultsim.networks.}}\sphinxbfcode{\sphinxupquote{data\_error}}}{\emph{\DUrole{n}{data}}, \emph{\DUrole{n}{average}}}{}
\sphinxAtStartPar
Calculates error for each column in data

\end{fulllineitems}

\index{degree\_dist() (in module fmdtools.faultsim.networks)@\spxentry{degree\_dist()}\spxextra{in module fmdtools.faultsim.networks}}

\begin{fulllineitems}
\phantomsection\label{\detokenize{docs/fmdtools.faultsim:fmdtools.faultsim.networks.degree_dist}}\pysiglinewithargsret{\sphinxcode{\sphinxupquote{fmdtools.faultsim.networks.}}\sphinxbfcode{\sphinxupquote{degree\_dist}}}{\emph{\DUrole{n}{mdl}}, \emph{\DUrole{n}{gtype}\DUrole{o}{=}\DUrole{default_value}{'bipartite'}}}{}
\sphinxAtStartPar
Plots degree distribution of graph representation of model mdl.
\begin{quote}\begin{description}
\item[{Parameters}] \leavevmode\begin{itemize}
\item {} 
\sphinxAtStartPar
\sphinxstyleliteralstrong{\sphinxupquote{mdl}} (\sphinxstyleliteralemphasis{\sphinxupquote{model}}\sphinxstyleliteralemphasis{\sphinxupquote{ or }}\sphinxstyleliteralemphasis{\sphinxupquote{graph}}) – graph to calculated degree distribution for. (will get from model if provided)

\item {} 
\sphinxAtStartPar
\sphinxstyleliteralstrong{\sphinxupquote{gtype}} (\sphinxstyleliteralemphasis{\sphinxupquote{str}}) – type of graph representation of the model to show. default is ‘bipartite’

\end{itemize}

\item[{Returns}] \leavevmode
\sphinxAtStartPar
\sphinxstylestrong{fig} – plot of distribution

\item[{Return type}] \leavevmode
\sphinxAtStartPar
matplotlib figure

\end{description}\end{quote}

\end{fulllineitems}

\index{find\_bridging\_nodes() (in module fmdtools.faultsim.networks)@\spxentry{find\_bridging\_nodes()}\spxextra{in module fmdtools.faultsim.networks}}

\begin{fulllineitems}
\phantomsection\label{\detokenize{docs/fmdtools.faultsim:fmdtools.faultsim.networks.find_bridging_nodes}}\pysiglinewithargsret{\sphinxcode{\sphinxupquote{fmdtools.faultsim.networks.}}\sphinxbfcode{\sphinxupquote{find\_bridging\_nodes}}}{\emph{\DUrole{n}{mdl}}, \emph{\DUrole{n}{plot}\DUrole{o}{=}\DUrole{default_value}{'off'}}, \emph{\DUrole{n}{gtype}\DUrole{o}{=}\DUrole{default_value}{'parameter'}}, \emph{\DUrole{n}{pos}\DUrole{o}{=}\DUrole{default_value}{\{\}}}, \emph{\DUrole{n}{scale}\DUrole{o}{=}\DUrole{default_value}{1}}}{}
\sphinxAtStartPar
Determines bridging nodes in a graph representation of model mdl.
\begin{quote}\begin{description}
\item[{Parameters}] \leavevmode\begin{itemize}
\item {} 
\sphinxAtStartPar
\sphinxstyleliteralstrong{\sphinxupquote{mdl}} (\sphinxstyleliteralemphasis{\sphinxupquote{model}}\sphinxstyleliteralemphasis{\sphinxupquote{ or }}\sphinxstyleliteralemphasis{\sphinxupquote{graph}}) – graph to run the analysis for. (will get from model if provided)

\item {} 
\sphinxAtStartPar
\sphinxstyleliteralstrong{\sphinxupquote{plot}} (\sphinxstyleliteralemphasis{\sphinxupquote{str}}\sphinxstyleliteralemphasis{\sphinxupquote{ (}}\sphinxstyleliteralemphasis{\sphinxupquote{optional}}\sphinxstyleliteralemphasis{\sphinxupquote{)}}) – plots graph with high degree nodes visualized if set to ‘on’

\item {} 
\sphinxAtStartPar
\sphinxstyleliteralstrong{\sphinxupquote{gtype}} (\sphinxstyleliteralemphasis{\sphinxupquote{str}}) – type of graph representation of the model to show. default is ‘bipartite’

\item {} 
\sphinxAtStartPar
\sphinxstyleliteralstrong{\sphinxupquote{pos}} (\sphinxstyleliteralemphasis{\sphinxupquote{dict}}\sphinxstyleliteralemphasis{\sphinxupquote{  (}}\sphinxstyleliteralemphasis{\sphinxupquote{optional}}\sphinxstyleliteralemphasis{\sphinxupquote{)}}) – dict of node positions in the model (if desired)

\item {} 
\sphinxAtStartPar
\sphinxstyleliteralstrong{\sphinxupquote{scale}} (\sphinxstyleliteralemphasis{\sphinxupquote{int}}\sphinxstyleliteralemphasis{\sphinxupquote{ (}}\sphinxstyleliteralemphasis{\sphinxupquote{optional}}\sphinxstyleliteralemphasis{\sphinxupquote{)}}) – scale for the plot. Default is 1.

\end{itemize}

\item[{Returns}] \leavevmode
\sphinxAtStartPar
\sphinxstylestrong{bridgingNodes}

\item[{Return type}] \leavevmode
\sphinxAtStartPar
list of bridging nodes

\end{description}\end{quote}

\end{fulllineitems}

\index{find\_high\_degree\_nodes() (in module fmdtools.faultsim.networks)@\spxentry{find\_high\_degree\_nodes()}\spxextra{in module fmdtools.faultsim.networks}}

\begin{fulllineitems}
\phantomsection\label{\detokenize{docs/fmdtools.faultsim:fmdtools.faultsim.networks.find_high_degree_nodes}}\pysiglinewithargsret{\sphinxcode{\sphinxupquote{fmdtools.faultsim.networks.}}\sphinxbfcode{\sphinxupquote{find\_high\_degree\_nodes}}}{\emph{\DUrole{n}{mdl}}, \emph{\DUrole{n}{p}\DUrole{o}{=}\DUrole{default_value}{90}}, \emph{\DUrole{n}{plot}\DUrole{o}{=}\DUrole{default_value}{'off'}}, \emph{\DUrole{n}{gtype}\DUrole{o}{=}\DUrole{default_value}{'bipartite'}}, \emph{\DUrole{n}{pos}\DUrole{o}{=}\DUrole{default_value}{\{\}}}, \emph{\DUrole{n}{scale}\DUrole{o}{=}\DUrole{default_value}{1}}}{}
\sphinxAtStartPar
Determines highest degree nodes, up to percentile p, in graph representation of model mdl.
\begin{quote}\begin{description}
\item[{Parameters}] \leavevmode\begin{itemize}
\item {} 
\sphinxAtStartPar
\sphinxstyleliteralstrong{\sphinxupquote{mdl}} (\sphinxstyleliteralemphasis{\sphinxupquote{model}}\sphinxstyleliteralemphasis{\sphinxupquote{ or }}\sphinxstyleliteralemphasis{\sphinxupquote{graph}}) – graph to run the analysis for. (will get from model if provided)

\item {} 
\sphinxAtStartPar
\sphinxstyleliteralstrong{\sphinxupquote{p}} (\sphinxstyleliteralemphasis{\sphinxupquote{int}}\sphinxstyleliteralemphasis{\sphinxupquote{ (}}\sphinxstyleliteralemphasis{\sphinxupquote{optional}}\sphinxstyleliteralemphasis{\sphinxupquote{)}}) – percentile of degrees to return, between 0 and 100

\item {} 
\sphinxAtStartPar
\sphinxstyleliteralstrong{\sphinxupquote{plot}} (\sphinxstyleliteralemphasis{\sphinxupquote{str}}\sphinxstyleliteralemphasis{\sphinxupquote{ (}}\sphinxstyleliteralemphasis{\sphinxupquote{optional}}\sphinxstyleliteralemphasis{\sphinxupquote{)}}) – plots graph with high degree nodes visualized if set to ‘on’

\item {} 
\sphinxAtStartPar
\sphinxstyleliteralstrong{\sphinxupquote{gtype}} (\sphinxstyleliteralemphasis{\sphinxupquote{str}}) – type of graph representation of the model to show. default is ‘bipartite’

\item {} 
\sphinxAtStartPar
\sphinxstyleliteralstrong{\sphinxupquote{pos}} (\sphinxstyleliteralemphasis{\sphinxupquote{dict}}\sphinxstyleliteralemphasis{\sphinxupquote{  (}}\sphinxstyleliteralemphasis{\sphinxupquote{optional}}\sphinxstyleliteralemphasis{\sphinxupquote{)}}) – dict of node positions in the model (if desired)

\item {} 
\sphinxAtStartPar
\sphinxstyleliteralstrong{\sphinxupquote{scale}} (\sphinxstyleliteralemphasis{\sphinxupquote{int}}\sphinxstyleliteralemphasis{\sphinxupquote{ (}}\sphinxstyleliteralemphasis{\sphinxupquote{optional}}\sphinxstyleliteralemphasis{\sphinxupquote{)}}) – scale for the plot. Default is 1.

\end{itemize}

\item[{Returns}] \leavevmode
\sphinxAtStartPar
\sphinxstylestrong{highDegreeNodes}

\item[{Return type}] \leavevmode
\sphinxAtStartPar
list of high degree nodes in format (node,degree)

\end{description}\end{quote}

\end{fulllineitems}

\index{get\_graph() (in module fmdtools.faultsim.networks)@\spxentry{get\_graph()}\spxextra{in module fmdtools.faultsim.networks}}

\begin{fulllineitems}
\phantomsection\label{\detokenize{docs/fmdtools.faultsim:fmdtools.faultsim.networks.get_graph}}\pysiglinewithargsret{\sphinxcode{\sphinxupquote{fmdtools.faultsim.networks.}}\sphinxbfcode{\sphinxupquote{get\_graph}}}{\emph{\DUrole{n}{mdl}}, \emph{\DUrole{n}{gtype}}}{}
\sphinxAtStartPar
gets the appropriate graph of type gtype from mdl

\end{fulllineitems}

\index{sff\_model() (in module fmdtools.faultsim.networks)@\spxentry{sff\_model()}\spxextra{in module fmdtools.faultsim.networks}}

\begin{fulllineitems}
\phantomsection\label{\detokenize{docs/fmdtools.faultsim:fmdtools.faultsim.networks.sff_model}}\pysiglinewithargsret{\sphinxcode{\sphinxupquote{fmdtools.faultsim.networks.}}\sphinxbfcode{\sphinxupquote{sff\_model}}}{\emph{\DUrole{n}{mdl}}, \emph{\DUrole{n}{gtype}\DUrole{o}{=}\DUrole{default_value}{'parameter'}}, \emph{\DUrole{n}{endtime}\DUrole{o}{=}\DUrole{default_value}{5}}, \emph{\DUrole{n}{pi}\DUrole{o}{=}\DUrole{default_value}{0.1}}, \emph{\DUrole{n}{pr}\DUrole{o}{=}\DUrole{default_value}{0.1}}, \emph{\DUrole{n}{num\_trials}\DUrole{o}{=}\DUrole{default_value}{100}}, \emph{\DUrole{n}{start\_node}\DUrole{o}{=}\DUrole{default_value}{'random'}}, \emph{\DUrole{n}{error\_bar\_option}\DUrole{o}{=}\DUrole{default_value}{'off'}}}{}
\sphinxAtStartPar
susc\sphinxhyphen{}fix\sphinxhyphen{}fail model.
\begin{quote}\begin{description}
\item[{Parameters}] \leavevmode\begin{itemize}
\item {} 
\sphinxAtStartPar
\sphinxstyleliteralstrong{\sphinxupquote{mdl}} (\sphinxstyleliteralemphasis{\sphinxupquote{model}}\sphinxstyleliteralemphasis{\sphinxupquote{ or }}\sphinxstyleliteralemphasis{\sphinxupquote{graph}}) – graph to run trials over (will get from model if provided)

\item {} 
\sphinxAtStartPar
\sphinxstyleliteralstrong{\sphinxupquote{endtime}} (\sphinxstyleliteralemphasis{\sphinxupquote{int}}) – simulation end time

\item {} 
\sphinxAtStartPar
\sphinxstyleliteralstrong{\sphinxupquote{pi}} (\sphinxstyleliteralemphasis{\sphinxupquote{float}}) – infection (failure spread) rate

\item {} 
\sphinxAtStartPar
\sphinxstyleliteralstrong{\sphinxupquote{pf}} (\sphinxstyleliteralemphasis{\sphinxupquote{float}}) – recovery (fix) rate

\item {} 
\sphinxAtStartPar
\sphinxstyleliteralstrong{\sphinxupquote{num\_trials}} (\sphinxstyleliteralemphasis{\sphinxupquote{int}}) – number of times to run the epidemic model, default is 100

\item {} 
\sphinxAtStartPar
\sphinxstyleliteralstrong{\sphinxupquote{error\_bar\_option}} (\sphinxstyleliteralemphasis{\sphinxupquote{str}}) – option for plotting error bars (first to third quartile), default is off

\item {} 
\sphinxAtStartPar
\sphinxstyleliteralstrong{\sphinxupquote{start\_node}} (\sphinxstyleliteralemphasis{\sphinxupquote{str}}) – start node to use in the trial. default is ‘random’

\end{itemize}

\item[{Returns}] \leavevmode
\sphinxAtStartPar
\sphinxstylestrong{fig}

\item[{Return type}] \leavevmode
\sphinxAtStartPar
plot of susc, fail, and fix nodes over time

\end{description}\end{quote}

\end{fulllineitems}

\index{sff\_one\_trial() (in module fmdtools.faultsim.networks)@\spxentry{sff\_one\_trial()}\spxextra{in module fmdtools.faultsim.networks}}

\begin{fulllineitems}
\phantomsection\label{\detokenize{docs/fmdtools.faultsim:fmdtools.faultsim.networks.sff_one_trial}}\pysiglinewithargsret{\sphinxcode{\sphinxupquote{fmdtools.faultsim.networks.}}\sphinxbfcode{\sphinxupquote{sff\_one\_trial}}}{\emph{\DUrole{n}{start\_node\_selected}}, \emph{\DUrole{n}{g}}, \emph{\DUrole{n}{endtime}\DUrole{o}{=}\DUrole{default_value}{5}}, \emph{\DUrole{n}{pi}\DUrole{o}{=}\DUrole{default_value}{0.1}}, \emph{\DUrole{n}{pr}\DUrole{o}{=}\DUrole{default_value}{0.1}}}{}
\sphinxAtStartPar
Calculates one trial of the sff model
\begin{quote}\begin{description}
\item[{Parameters}] \leavevmode\begin{itemize}
\item {} 
\sphinxAtStartPar
\sphinxstyleliteralstrong{\sphinxupquote{start\_node\_selected}} (\sphinxstyleliteralemphasis{\sphinxupquote{str}}) – node to start the trial from

\item {} 
\sphinxAtStartPar
\sphinxstyleliteralstrong{\sphinxupquote{g}} (\sphinxstyleliteralemphasis{\sphinxupquote{networkx graph}}) – graph to run the trial over

\item {} 
\sphinxAtStartPar
\sphinxstyleliteralstrong{\sphinxupquote{endtime}} (\sphinxstyleliteralemphasis{\sphinxupquote{int}}) – simulation end time

\item {} 
\sphinxAtStartPar
\sphinxstyleliteralstrong{\sphinxupquote{pi}} (\sphinxstyleliteralemphasis{\sphinxupquote{float}}) – infection (failure spread) rate

\item {} 
\sphinxAtStartPar
\sphinxstyleliteralstrong{\sphinxupquote{pf}} (\sphinxstyleliteralemphasis{\sphinxupquote{float}}) – recovery (fix) rate

\end{itemize}

\end{description}\end{quote}

\end{fulllineitems}



\subsubsection{fmdtools.faultsim.propagate}
\label{\detokenize{docs/fmdtools.faultsim:fmdtools-faultsim-propagate}}
\noindent\sphinxincludegraphics[width=800\sphinxpxdimen]{{simulation_types}.png}

\sphinxAtStartPar
The {\hyperref[\detokenize{docs/fmdtools.faultsim:module-fmdtools.faultsim.propagate}]{\sphinxcrossref{\sphinxcode{\sphinxupquote{fmdtools.faultsim.propagate}}}}} module is used to simulate the behaviors of a {\hyperref[\detokenize{docs/fmdtools:fmdtools.modeldef.Model}]{\sphinxcrossref{\sphinxcode{\sphinxupquote{fmdtools.modeldef.Model}}}}} model with and without faults. As shown above, each of the methods (described below) fit a given simulation use\sphinxhyphen{}case for resilience assessment–single/multiple scenarios, in nominal/faulty scenarios, and at a single set or multiple sets of parameters.

\phantomsection\label{\detokenize{docs/fmdtools.faultsim:module-fmdtools.faultsim.propagate}}\index{module@\spxentry{module}!fmdtools.faultsim.propagate@\spxentry{fmdtools.faultsim.propagate}}\index{fmdtools.faultsim.propagate@\spxentry{fmdtools.faultsim.propagate}!module@\spxentry{module}}
\sphinxAtStartPar
Description: functions to propagate faults through a user\sphinxhyphen{}defined fault model
\begin{description}
\item[{Main Methods:}] \leavevmode\begin{itemize}
\item {} 
\sphinxAtStartPar
{\hyperref[\detokenize{docs/fmdtools.faultsim:fmdtools.faultsim.propagate.nominal}]{\sphinxcrossref{\sphinxcode{\sphinxupquote{nominal()}}}}}:            Runs the model over time in the nominal scenario.

\item {} 
\sphinxAtStartPar
{\hyperref[\detokenize{docs/fmdtools.faultsim:fmdtools.faultsim.propagate.one_fault}]{\sphinxcrossref{\sphinxcode{\sphinxupquote{one\_fault()}}}}}:          Runs one fault in the model at a specified time.

\item {} 
\sphinxAtStartPar
{\hyperref[\detokenize{docs/fmdtools.faultsim:fmdtools.faultsim.propagate.mult_fault}]{\sphinxcrossref{\sphinxcode{\sphinxupquote{mult\_fault()}}}}}:         Runs arbitrary scenario of fault modes at specified times

\item {} 
\sphinxAtStartPar
\sphinxcode{\sphinxupquote{singlefaults()}}:       Creates and propagates a list of failure scenarios in a model over given model times

\item {} 
\sphinxAtStartPar
{\hyperref[\detokenize{docs/fmdtools.faultsim:fmdtools.faultsim.propagate.approach}]{\sphinxcrossref{\sphinxcode{\sphinxupquote{approach()}}}}}:             Injects and propagates faults in the model defined by a given sample approach.

\item {} 
\sphinxAtStartPar
{\hyperref[\detokenize{docs/fmdtools.faultsim:fmdtools.faultsim.propagate.nominal_approach}]{\sphinxcrossref{\sphinxcode{\sphinxupquote{nominal\_approach()}}}}}:     Simulates a model over a range of parameters defined by a nominal approach.

\item {} 
\sphinxAtStartPar
{\hyperref[\detokenize{docs/fmdtools.faultsim:fmdtools.faultsim.propagate.nested_approach}]{\sphinxcrossref{\sphinxcode{\sphinxupquote{nested\_approach()}}}}}:      Injects and propagates faults in the model defined by a given sample approach over a range of parameters defined by a nominal approach.

\end{itemize}

\item[{Private Methods:}] \leavevmode\begin{itemize}
\item {} 
\sphinxAtStartPar
{\hyperref[\detokenize{docs/fmdtools.faultsim:fmdtools.faultsim.propagate.list_init_faults}]{\sphinxcrossref{\sphinxcode{\sphinxupquote{list\_init\_faults()}}}}}:   Creates a list of single\sphinxhyphen{}fault scenarios for the graph, given the modes set up in the fault model

\item {} 
\sphinxAtStartPar
{\hyperref[\detokenize{docs/fmdtools.faultsim:fmdtools.faultsim.propagate.prop_one_scen}]{\sphinxcrossref{\sphinxcode{\sphinxupquote{prop\_one\_scen()}}}}}:      Runs a fault scenario in the model over time

\item {} 
\sphinxAtStartPar
{\hyperref[\detokenize{docs/fmdtools.faultsim:fmdtools.faultsim.propagate.propagate}]{\sphinxcrossref{\sphinxcode{\sphinxupquote{propagate()}}}}}:          Injects and propagates faults through the graph at one time\sphinxhyphen{}step

\item {} 
\sphinxAtStartPar
{\hyperref[\detokenize{docs/fmdtools.faultsim:fmdtools.faultsim.propagate.prop_time}]{\sphinxcrossref{\sphinxcode{\sphinxupquote{prop\_time()}}}}}:          Propagates faults through model graph.

\item {} \begin{description}
\item[{{\hyperref[\detokenize{docs/fmdtools.faultsim:fmdtools.faultsim.propagate.update_mdlhist}]{\sphinxcrossref{\sphinxcode{\sphinxupquote{update\_mdlhist()}}}}}:     Updates the model history at a given time.}] \leavevmode\begin{itemize}
\item {} 
\sphinxAtStartPar
{\hyperref[\detokenize{docs/fmdtools.faultsim:fmdtools.faultsim.propagate.update_flowhist}]{\sphinxcrossref{\sphinxcode{\sphinxupquote{update\_flowhist()}}}}}:Updates the flows in the model history at t\_ind

\item {} 
\sphinxAtStartPar
{\hyperref[\detokenize{docs/fmdtools.faultsim:fmdtools.faultsim.propagate.update_fxnhist}]{\sphinxcrossref{\sphinxcode{\sphinxupquote{update\_fxnhist()}}}}}: Updates the functions (faults and states) in the model history at t\_ind

\end{itemize}

\end{description}

\item {} \begin{description}
\item[{{\hyperref[\detokenize{docs/fmdtools.faultsim:fmdtools.faultsim.propagate.init_mdlhist}]{\sphinxcrossref{\sphinxcode{\sphinxupquote{init\_mdlhist()}}}}}:       Initializes the model history over a given timerange}] \leavevmode\begin{itemize}
\item {} 
\sphinxAtStartPar
{\hyperref[\detokenize{docs/fmdtools.faultsim:fmdtools.faultsim.propagate.init_flowhist}]{\sphinxcrossref{\sphinxcode{\sphinxupquote{init\_flowhist()}}}}}:  Initializes the flow history flowhist of the model mdl over the time range timerange

\item {} 
\sphinxAtStartPar
{\hyperref[\detokenize{docs/fmdtools.faultsim:fmdtools.faultsim.propagate.init_fxnhist}]{\sphinxcrossref{\sphinxcode{\sphinxupquote{init\_fxnhist()}}}}}:   Initializes the function state history fxnhist of the model mdl over the time range timerange

\end{itemize}

\end{description}

\end{itemize}

\end{description}
\index{approach() (in module fmdtools.faultsim.propagate)@\spxentry{approach()}\spxextra{in module fmdtools.faultsim.propagate}}

\begin{fulllineitems}
\phantomsection\label{\detokenize{docs/fmdtools.faultsim:fmdtools.faultsim.propagate.approach}}\pysiglinewithargsret{\sphinxcode{\sphinxupquote{fmdtools.faultsim.propagate.}}\sphinxbfcode{\sphinxupquote{approach}}}{\emph{\DUrole{n}{mdl}}, \emph{\DUrole{n}{app}}, \emph{\DUrole{n}{staged}\DUrole{o}{=}\DUrole{default_value}{False}}, \emph{\DUrole{n}{track}\DUrole{o}{=}\DUrole{default_value}{'all'}}, \emph{\DUrole{n}{pool}\DUrole{o}{=}\DUrole{default_value}{False}}, \emph{\DUrole{n}{showprogress}\DUrole{o}{=}\DUrole{default_value}{True}}, \emph{\DUrole{n}{track\_times}\DUrole{o}{=}\DUrole{default_value}{'all'}}, \emph{\DUrole{n}{protect}\DUrole{o}{=}\DUrole{default_value}{True}}, \emph{\DUrole{n}{run\_stochastic}\DUrole{o}{=}\DUrole{default_value}{False}}, \emph{\DUrole{o}{**}\DUrole{n}{kwargs}}}{}
\sphinxAtStartPar
Injects and propagates faults in the model defined by a given sample approach
\begin{quote}\begin{description}
\item[{Parameters}] \leavevmode\begin{itemize}
\item {} 
\sphinxAtStartPar
\sphinxstyleliteralstrong{\sphinxupquote{mdl}} (\sphinxstyleliteralemphasis{\sphinxupquote{model}}) – The model to inject faults in.

\item {} 
\sphinxAtStartPar
\sphinxstyleliteralstrong{\sphinxupquote{app}} (\sphinxstyleliteralemphasis{\sphinxupquote{sampleapproach}}) – SampleApproach used to define the list of faults and sample time for the model.

\item {} 
\sphinxAtStartPar
\sphinxstyleliteralstrong{\sphinxupquote{staged}} (\sphinxstyleliteralemphasis{\sphinxupquote{bool}}\sphinxstyleliteralemphasis{\sphinxupquote{, }}\sphinxstyleliteralemphasis{\sphinxupquote{optional}}) – Whether to inject the fault in a copy of the nominal model at the fault time (True) or instantiate a new model for the fault (False). Setting to True roughly halves execution time. The default is False.

\item {} 
\sphinxAtStartPar
\sphinxstyleliteralstrong{\sphinxupquote{track}} (\sphinxstyleliteralemphasis{\sphinxupquote{str}}\sphinxstyleliteralemphasis{\sphinxupquote{ (}}\sphinxstyleliteralemphasis{\sphinxupquote{'all'}}\sphinxstyleliteralemphasis{\sphinxupquote{, }}\sphinxstyleliteralemphasis{\sphinxupquote{'functions'}}\sphinxstyleliteralemphasis{\sphinxupquote{, }}\sphinxstyleliteralemphasis{\sphinxupquote{'flows'}}\sphinxstyleliteralemphasis{\sphinxupquote{, }}\sphinxstyleliteralemphasis{\sphinxupquote{'valparams'}}\sphinxstyleliteralemphasis{\sphinxupquote{, }}\sphinxstyleliteralemphasis{\sphinxupquote{dict}}\sphinxstyleliteralemphasis{\sphinxupquote{, }}\sphinxstyleliteralemphasis{\sphinxupquote{'none'}}\sphinxstyleliteralemphasis{\sphinxupquote{)}}\sphinxstyleliteralemphasis{\sphinxupquote{, }}\sphinxstyleliteralemphasis{\sphinxupquote{optional}}) – Which model states to track over time, which can be given as ‘functions’, ‘flows’,
‘all’, ‘none’, ‘valparams’ (model states specified in mdl.valparams),
or a dict of form \{‘functions’:\{‘fxn1’:’att1’\}, ‘flows’:\{‘flow1’:’att1’\}\}
The default is ‘all’.

\item {} 
\sphinxAtStartPar
\sphinxstyleliteralstrong{\sphinxupquote{pool}} (\sphinxstyleliteralemphasis{\sphinxupquote{process pool}}\sphinxstyleliteralemphasis{\sphinxupquote{, }}\sphinxstyleliteralemphasis{\sphinxupquote{optional}}) – Process Pool Object from multiprocessing or pathos packages. Pathos is recommended.
e.g. parallelpool = mp.pool(n) for n cores (multiprocessing)
or parallelpool = ProcessPool(nodes=n) for n cores (pathos)
If False, the set of scenarios is run serially. The default is False

\item {} 
\sphinxAtStartPar
\sphinxstyleliteralstrong{\sphinxupquote{showprogress}} (\sphinxstyleliteralemphasis{\sphinxupquote{bool}}\sphinxstyleliteralemphasis{\sphinxupquote{, }}\sphinxstyleliteralemphasis{\sphinxupquote{optional}}) – whether to show a progress bar during execution. default is true

\item {} 
\sphinxAtStartPar
\sphinxstyleliteralstrong{\sphinxupquote{track\_times}} (\sphinxstyleliteralemphasis{\sphinxupquote{str/tuple}}) – \begin{description}
\item[{Defines what times to include in the history. Options are:}] \leavevmode
\sphinxAtStartPar
’all’–all simulated times
(‘interval’, n)–includes every nth time in the history
(‘times’, {[}t1, … tn{]})–only includes times defined in the vector {[}t1 … tn{]}

\end{description}


\item {} 
\sphinxAtStartPar
\sphinxstyleliteralstrong{\sphinxupquote{protect}} (\sphinxstyleliteralemphasis{\sphinxupquote{bool}}) – \begin{description}
\item[{Whether or not to protect the model object via copying}] \leavevmode
\sphinxAtStartPar
True (default) \sphinxhyphen{} re\sphinxhyphen{}instances the model (safe)
False \sphinxhyphen{} model is not re\sphinxhyphen{}instantiated (unsafe–do not use model afterwards)

\end{description}


\item {} 
\sphinxAtStartPar
\sphinxstyleliteralstrong{\sphinxupquote{run\_stochastic}} (\sphinxstyleliteralemphasis{\sphinxupquote{bool}}) – Whether to run stochastic behaviors or use default values for stochastic variables. Default is False.

\item {} 
\sphinxAtStartPar
\sphinxstyleliteralstrong{\sphinxupquote{**kwargs}} (\sphinxstyleliteralemphasis{\sphinxupquote{kwargs}}\sphinxstyleliteralemphasis{\sphinxupquote{ (}}\sphinxstyleliteralemphasis{\sphinxupquote{params}}\sphinxstyleliteralemphasis{\sphinxupquote{, }}\sphinxstyleliteralemphasis{\sphinxupquote{modelparams}}\sphinxstyleliteralemphasis{\sphinxupquote{, }}\sphinxstyleliteralemphasis{\sphinxupquote{and/or valparams}}\sphinxstyleliteralemphasis{\sphinxupquote{)}}) – passing parameter dictionaries (e.g., params, modelparams, valparams) instantiates the model
to be simulated with the given parameters. Parameter dictionaries do not
need to be complete (if incomplete)

\end{itemize}

\item[{Returns}] \leavevmode
\sphinxAtStartPar
\begin{itemize}
\item {} 
\sphinxAtStartPar
\sphinxstylestrong{endclasses} (\sphinxstyleemphasis{dict}) – A dictionary with the rate, cost, and expected cost of each scenario run with structure \{scenname:\{expected cost, cost, rate\}\}

\item {} 
\sphinxAtStartPar
\sphinxstylestrong{mdlhists} (\sphinxstyleemphasis{dict}) – A dictionary with the history of all model states for each scenario (including the nominal)

\end{itemize}


\end{description}\end{quote}

\end{fulllineitems}

\index{construct\_nomscen() (in module fmdtools.faultsim.propagate)@\spxentry{construct\_nomscen()}\spxextra{in module fmdtools.faultsim.propagate}}

\begin{fulllineitems}
\phantomsection\label{\detokenize{docs/fmdtools.faultsim:fmdtools.faultsim.propagate.construct_nomscen}}\pysiglinewithargsret{\sphinxcode{\sphinxupquote{fmdtools.faultsim.propagate.}}\sphinxbfcode{\sphinxupquote{construct\_nomscen}}}{\emph{\DUrole{n}{mdl}}}{}
\sphinxAtStartPar
Creates a nominal scenario nomscen given a graph object g by setting all function modes to nominal.
\begin{quote}\begin{description}
\item[{Parameters}] \leavevmode
\sphinxAtStartPar
\sphinxstyleliteralstrong{\sphinxupquote{mdl}} ({\hyperref[\detokenize{docs/fmdtools:fmdtools.modeldef.Model}]{\sphinxcrossref{\sphinxstyleliteralemphasis{\sphinxupquote{Model}}}}}) – 

\item[{Returns}] \leavevmode
\sphinxAtStartPar
\sphinxstylestrong{nomscen}

\item[{Return type}] \leavevmode
\sphinxAtStartPar
scen

\end{description}\end{quote}

\end{fulllineitems}

\index{cut\_mdlhist() (in module fmdtools.faultsim.propagate)@\spxentry{cut\_mdlhist()}\spxextra{in module fmdtools.faultsim.propagate}}

\begin{fulllineitems}
\phantomsection\label{\detokenize{docs/fmdtools.faultsim:fmdtools.faultsim.propagate.cut_mdlhist}}\pysiglinewithargsret{\sphinxcode{\sphinxupquote{fmdtools.faultsim.propagate.}}\sphinxbfcode{\sphinxupquote{cut\_mdlhist}}}{\emph{\DUrole{n}{mdlhist}}, \emph{\DUrole{n}{ind}}}{}
\sphinxAtStartPar
Cuts unsimulated values from end of array
\begin{quote}\begin{description}
\item[{Parameters}] \leavevmode\begin{itemize}
\item {} 
\sphinxAtStartPar
\sphinxstyleliteralstrong{\sphinxupquote{mdlhist}} (\sphinxstyleliteralemphasis{\sphinxupquote{dict}}) – dictionary of model histories for functions/flows

\item {} 
\sphinxAtStartPar
\sphinxstyleliteralstrong{\sphinxupquote{ind}} (\sphinxstyleliteralemphasis{\sphinxupquote{int}}) – index to cut the history at

\end{itemize}

\item[{Returns}] \leavevmode
\sphinxAtStartPar
\sphinxstylestrong{mdlhist} – The model history until the given index.

\item[{Return type}] \leavevmode
\sphinxAtStartPar
dict

\end{description}\end{quote}

\end{fulllineitems}

\index{eq\_units() (in module fmdtools.faultsim.propagate)@\spxentry{eq\_units()}\spxextra{in module fmdtools.faultsim.propagate}}

\begin{fulllineitems}
\phantomsection\label{\detokenize{docs/fmdtools.faultsim:fmdtools.faultsim.propagate.eq_units}}\pysiglinewithargsret{\sphinxcode{\sphinxupquote{fmdtools.faultsim.propagate.}}\sphinxbfcode{\sphinxupquote{eq\_units}}}{\emph{\DUrole{n}{rateunit}}, \emph{\DUrole{n}{timeunit}}}{}
\sphinxAtStartPar
Provides conversion factor for from rateunit (str) to timeunit (str)
Options for units are: ‘sec’, ‘min’, ‘hr’, ‘day’, ‘wk’, ‘month’, and ‘year’

\end{fulllineitems}

\index{exec\_nom\_helper() (in module fmdtools.faultsim.propagate)@\spxentry{exec\_nom\_helper()}\spxextra{in module fmdtools.faultsim.propagate}}

\begin{fulllineitems}
\phantomsection\label{\detokenize{docs/fmdtools.faultsim:fmdtools.faultsim.propagate.exec_nom_helper}}\pysiglinewithargsret{\sphinxcode{\sphinxupquote{fmdtools.faultsim.propagate.}}\sphinxbfcode{\sphinxupquote{exec\_nom\_helper}}}{\emph{\DUrole{n}{arg}}}{}
\sphinxAtStartPar
Helper function for executing nominal scenarios

\end{fulllineitems}

\index{exec\_scen() (in module fmdtools.faultsim.propagate)@\spxentry{exec\_scen()}\spxextra{in module fmdtools.faultsim.propagate}}

\begin{fulllineitems}
\phantomsection\label{\detokenize{docs/fmdtools.faultsim:fmdtools.faultsim.propagate.exec_scen}}\pysiglinewithargsret{\sphinxcode{\sphinxupquote{fmdtools.faultsim.propagate.}}\sphinxbfcode{\sphinxupquote{exec\_scen}}}{\emph{\DUrole{n}{mdl}}, \emph{\DUrole{n}{scen}}, \emph{\DUrole{n}{nomresgraph}}, \emph{\DUrole{n}{nomhist}}, \emph{\DUrole{n}{track}\DUrole{o}{=}\DUrole{default_value}{'all'}}, \emph{\DUrole{n}{staged}\DUrole{o}{=}\DUrole{default_value}{True}}, \emph{\DUrole{n}{track\_times}\DUrole{o}{=}\DUrole{default_value}{'all'}}, \emph{\DUrole{n}{run\_stochastic}\DUrole{o}{=}\DUrole{default_value}{False}}}{}
\sphinxAtStartPar
Executes a scenario and generates results and classifications given a model and nominal model history
\begin{quote}

\sphinxAtStartPar
Parameters
\end{quote}
\begin{description}
\item[{mdl}] \leavevmode{[}model{]}
\sphinxAtStartPar
The model to inject faults in.

\item[{scen}] \leavevmode{[}scenario{]}
\sphinxAtStartPar
scenario used to define time and faults where the fault is to be injected

\item[{nomresgraph:}] \leavevmode
\sphinxAtStartPar
results graph of the nominal model run

\item[{nomhist:}] \leavevmode
\sphinxAtStartPar
history of results in the nominal model run

\item[{c\_mdl:}] \leavevmode
\sphinxAtStartPar
the nominal model at the time to be executed in the scenarios (a dict keyed by times)

\item[{staged}] \leavevmode{[}bool, optional{]}
\sphinxAtStartPar
Whether to inject the fault in a copy of the nominal model at the fault time (True) or instantiate a new model for the fault (False). Setting to True roughly halves execution time. The default is False.

\item[{track}] \leavevmode{[}str (‘all’, ‘functions’, ‘flows’, ‘valparams’, dict, ‘none’), optional{]}
\sphinxAtStartPar
Which model states to track over time, which can be given as ‘functions’, ‘flows’,
‘all’, ‘none’, ‘valparams’ (model states specified in mdl.valparams),
or a dict of form \{‘functions’:\{‘fxn1’:’att1’\}, ‘flows’:\{‘flow1’:’att1’\}\}
The default is ‘all’.

\item[{track\_times}] \leavevmode{[}str/tuple{]}\begin{description}
\item[{Defines what times to include in the history. Options are:}] \leavevmode
\sphinxAtStartPar
‘all’–all simulated times
(‘interval’, n)–includes every nth time in the history
(‘times’, {[}t1, … tn{]})–only includes times defined in the vector {[}t1 … tn{]}

\end{description}

\item[{run\_stochastic}] \leavevmode{[}bool{]}
\sphinxAtStartPar
Whether to run stochastic behaviors or use default values for stochastic variables. Default is False.

\end{description}

\end{fulllineitems}

\index{exec\_scen\_par() (in module fmdtools.faultsim.propagate)@\spxentry{exec\_scen\_par()}\spxextra{in module fmdtools.faultsim.propagate}}

\begin{fulllineitems}
\phantomsection\label{\detokenize{docs/fmdtools.faultsim:fmdtools.faultsim.propagate.exec_scen_par}}\pysiglinewithargsret{\sphinxcode{\sphinxupquote{fmdtools.faultsim.propagate.}}\sphinxbfcode{\sphinxupquote{exec\_scen\_par}}}{\emph{\DUrole{n}{args}}}{}
\sphinxAtStartPar
Helper function for executing the scenario in parallel

\end{fulllineitems}

\index{init\_flowhist() (in module fmdtools.faultsim.propagate)@\spxentry{init\_flowhist()}\spxextra{in module fmdtools.faultsim.propagate}}

\begin{fulllineitems}
\phantomsection\label{\detokenize{docs/fmdtools.faultsim:fmdtools.faultsim.propagate.init_flowhist}}\pysiglinewithargsret{\sphinxcode{\sphinxupquote{fmdtools.faultsim.propagate.}}\sphinxbfcode{\sphinxupquote{init\_flowhist}}}{\emph{\DUrole{n}{mdl}}, \emph{\DUrole{n}{timerange}}, \emph{\DUrole{n}{track}\DUrole{o}{=}\DUrole{default_value}{'all'}}}{}
\sphinxAtStartPar
Initializes the flow history flowhist of the model mdl over the time range timerange
\begin{quote}\begin{description}
\item[{Parameters}] \leavevmode\begin{itemize}
\item {} 
\sphinxAtStartPar
\sphinxstyleliteralstrong{\sphinxupquote{mdl}} (\sphinxstyleliteralemphasis{\sphinxupquote{model}}) – the Model object

\item {} 
\sphinxAtStartPar
\sphinxstyleliteralstrong{\sphinxupquote{timerange}} (\sphinxstyleliteralemphasis{\sphinxupquote{array}}) – Numpy array of times to initialize in the dictionary.

\item {} 
\sphinxAtStartPar
\sphinxstyleliteralstrong{\sphinxupquote{track}} (\sphinxstyleliteralemphasis{\sphinxupquote{'all'}}\sphinxstyleliteralemphasis{\sphinxupquote{ or }}\sphinxstyleliteralemphasis{\sphinxupquote{dict}}\sphinxstyleliteralemphasis{\sphinxupquote{, }}\sphinxstyleliteralemphasis{\sphinxupquote{'none'}}\sphinxstyleliteralemphasis{\sphinxupquote{)}}\sphinxstyleliteralemphasis{\sphinxupquote{, }}\sphinxstyleliteralemphasis{\sphinxupquote{optional}}) – Which model states to track over time, which can be given as ‘all’ or a
dict of form \{‘functions’:\{‘fxn1’:’att1’\}, ‘flows’:\{‘flow1’:’att1’\}\}
The default is ‘all’.

\end{itemize}

\item[{Returns}] \leavevmode
\sphinxAtStartPar
\sphinxstylestrong{flowhist} – A dictionary history of each recorded flow state over the given timerange.

\item[{Return type}] \leavevmode
\sphinxAtStartPar
dict

\end{description}\end{quote}

\end{fulllineitems}

\index{init\_fxnhist() (in module fmdtools.faultsim.propagate)@\spxentry{init\_fxnhist()}\spxextra{in module fmdtools.faultsim.propagate}}

\begin{fulllineitems}
\phantomsection\label{\detokenize{docs/fmdtools.faultsim:fmdtools.faultsim.propagate.init_fxnhist}}\pysiglinewithargsret{\sphinxcode{\sphinxupquote{fmdtools.faultsim.propagate.}}\sphinxbfcode{\sphinxupquote{init\_fxnhist}}}{\emph{\DUrole{n}{mdl}}, \emph{\DUrole{n}{timerange}}, \emph{\DUrole{n}{track}\DUrole{o}{=}\DUrole{default_value}{'all'}}}{}
\sphinxAtStartPar
Initializes the function state history fxnhist of the model mdl over the time range timerange
\begin{quote}\begin{description}
\item[{Parameters}] \leavevmode\begin{itemize}
\item {} 
\sphinxAtStartPar
\sphinxstyleliteralstrong{\sphinxupquote{mdl}} (\sphinxstyleliteralemphasis{\sphinxupquote{model}}) – the Model object

\item {} 
\sphinxAtStartPar
\sphinxstyleliteralstrong{\sphinxupquote{timerange}} (\sphinxstyleliteralemphasis{\sphinxupquote{array}}) – Numpy array of times to initialize in the dictionary.

\item {} 
\sphinxAtStartPar
\sphinxstyleliteralstrong{\sphinxupquote{track}} (\sphinxstyleliteralemphasis{\sphinxupquote{'all'}}\sphinxstyleliteralemphasis{\sphinxupquote{ or }}\sphinxstyleliteralemphasis{\sphinxupquote{dict}}\sphinxstyleliteralemphasis{\sphinxupquote{, }}\sphinxstyleliteralemphasis{\sphinxupquote{'none'}}\sphinxstyleliteralemphasis{\sphinxupquote{)}}\sphinxstyleliteralemphasis{\sphinxupquote{, }}\sphinxstyleliteralemphasis{\sphinxupquote{optional}}) – Which model states to track over time, which can be given as ‘all’ or a
dict of form \{‘functions’:\{‘fxn1’:’att1’\}, ‘flows’:\{‘flow1’:’att1’\}\}
The default is ‘all’.

\end{itemize}

\item[{Returns}] \leavevmode
\sphinxAtStartPar
\sphinxstylestrong{fxnhist} – A dictionary history of each recorded function state over the given timerange.

\item[{Return type}] \leavevmode
\sphinxAtStartPar
dict

\end{description}\end{quote}

\end{fulllineitems}

\index{init\_mdlhist() (in module fmdtools.faultsim.propagate)@\spxentry{init\_mdlhist()}\spxextra{in module fmdtools.faultsim.propagate}}

\begin{fulllineitems}
\phantomsection\label{\detokenize{docs/fmdtools.faultsim:fmdtools.faultsim.propagate.init_mdlhist}}\pysiglinewithargsret{\sphinxcode{\sphinxupquote{fmdtools.faultsim.propagate.}}\sphinxbfcode{\sphinxupquote{init\_mdlhist}}}{\emph{\DUrole{n}{mdl}}, \emph{\DUrole{n}{timerange}}, \emph{\DUrole{n}{track}\DUrole{o}{=}\DUrole{default_value}{'all'}}}{}
\sphinxAtStartPar
Initializes the model history over a given timerange
\begin{quote}\begin{description}
\item[{Parameters}] \leavevmode\begin{itemize}
\item {} 
\sphinxAtStartPar
\sphinxstyleliteralstrong{\sphinxupquote{mdl}} (\sphinxstyleliteralemphasis{\sphinxupquote{model}}) – the Model object

\item {} 
\sphinxAtStartPar
\sphinxstyleliteralstrong{\sphinxupquote{timerange}} (\sphinxstyleliteralemphasis{\sphinxupquote{array}}) – Numpy array of times to initialize in the dictionary.

\item {} 
\sphinxAtStartPar
\sphinxstyleliteralstrong{\sphinxupquote{track}} (\sphinxstyleliteralemphasis{\sphinxupquote{str}}\sphinxstyleliteralemphasis{\sphinxupquote{ (}}\sphinxstyleliteralemphasis{\sphinxupquote{'all'}}\sphinxstyleliteralemphasis{\sphinxupquote{, }}\sphinxstyleliteralemphasis{\sphinxupquote{'functions'}}\sphinxstyleliteralemphasis{\sphinxupquote{, }}\sphinxstyleliteralemphasis{\sphinxupquote{'flows'}}\sphinxstyleliteralemphasis{\sphinxupquote{, }}\sphinxstyleliteralemphasis{\sphinxupquote{'valparams'}}\sphinxstyleliteralemphasis{\sphinxupquote{, }}\sphinxstyleliteralemphasis{\sphinxupquote{dict}}\sphinxstyleliteralemphasis{\sphinxupquote{, }}\sphinxstyleliteralemphasis{\sphinxupquote{'none'}}\sphinxstyleliteralemphasis{\sphinxupquote{)}}\sphinxstyleliteralemphasis{\sphinxupquote{, }}\sphinxstyleliteralemphasis{\sphinxupquote{optional}}) – Which model states to track over time, which can be given as ‘functions’, ‘flows’,
‘all’, ‘none’, ‘valparams’ (model states specified in mdl.valparams),
or a dict of form \{‘functions’:\{‘fxn1’:’att1’\}, ‘flows’:\{‘flow1’:’att1’\}\}
The default is ‘all’.

\end{itemize}

\item[{Returns}] \leavevmode
\sphinxAtStartPar
\sphinxstylestrong{mdlhist} – A dictionary history of each model state over the given timerange.

\item[{Return type}] \leavevmode
\sphinxAtStartPar
dict

\end{description}\end{quote}

\end{fulllineitems}

\index{list\_init\_faults() (in module fmdtools.faultsim.propagate)@\spxentry{list\_init\_faults()}\spxextra{in module fmdtools.faultsim.propagate}}

\begin{fulllineitems}
\phantomsection\label{\detokenize{docs/fmdtools.faultsim:fmdtools.faultsim.propagate.list_init_faults}}\pysiglinewithargsret{\sphinxcode{\sphinxupquote{fmdtools.faultsim.propagate.}}\sphinxbfcode{\sphinxupquote{list\_init\_faults}}}{\emph{\DUrole{n}{mdl}}}{}
\sphinxAtStartPar
Creates a list of single\sphinxhyphen{}fault scenarios for the graph, given the modes set up in the fault model
\begin{quote}\begin{description}
\item[{Parameters}] \leavevmode
\sphinxAtStartPar
\sphinxstyleliteralstrong{\sphinxupquote{mdl}} ({\hyperref[\detokenize{docs/fmdtools:fmdtools.modeldef.Model}]{\sphinxcrossref{\sphinxstyleliteralemphasis{\sphinxupquote{Model}}}}}) – Model with list of times in mdl.times

\item[{Returns}] \leavevmode
\sphinxAtStartPar
\sphinxstylestrong{faultlist} – A list of fault scenarios, where a scenario is defined as: \{faults:\{functions:faultmodes\}, properties:\{(changes depending scenario type)\} \}

\item[{Return type}] \leavevmode
\sphinxAtStartPar
list

\end{description}\end{quote}

\end{fulllineitems}

\index{mult\_fault() (in module fmdtools.faultsim.propagate)@\spxentry{mult\_fault()}\spxextra{in module fmdtools.faultsim.propagate}}

\begin{fulllineitems}
\phantomsection\label{\detokenize{docs/fmdtools.faultsim:fmdtools.faultsim.propagate.mult_fault}}\pysiglinewithargsret{\sphinxcode{\sphinxupquote{fmdtools.faultsim.propagate.}}\sphinxbfcode{\sphinxupquote{mult\_fault}}}{\emph{\DUrole{n}{mdl}}, \emph{\DUrole{n}{faultseq}}, \emph{\DUrole{n}{track}\DUrole{o}{=}\DUrole{default_value}{'all'}}, \emph{\DUrole{n}{rate}\DUrole{o}{=}\DUrole{default_value}{nan}}, \emph{\DUrole{n}{gtype}\DUrole{o}{=}\DUrole{default_value}{'bipartite'}}, \emph{\DUrole{n}{track\_times}\DUrole{o}{=}\DUrole{default_value}{'all'}}, \emph{\DUrole{n}{protect}\DUrole{o}{=}\DUrole{default_value}{True}}, \emph{\DUrole{n}{run\_stochastic}\DUrole{o}{=}\DUrole{default_value}{False}}, \emph{\DUrole{o}{**}\DUrole{n}{kwargs}}}{}
\sphinxAtStartPar
Runs one fault in the model at a specified time.
\begin{quote}\begin{description}
\item[{Parameters}] \leavevmode\begin{itemize}
\item {} 
\sphinxAtStartPar
\sphinxstyleliteralstrong{\sphinxupquote{mdl}} ({\hyperref[\detokenize{docs/fmdtools:fmdtools.modeldef.Model}]{\sphinxcrossref{\sphinxstyleliteralemphasis{\sphinxupquote{Model}}}}}) – The model to inject the fault in.

\item {} 
\sphinxAtStartPar
\sphinxstyleliteralstrong{\sphinxupquote{faultseq}} (\sphinxstyleliteralemphasis{\sphinxupquote{dict}}) – Dict of times and modes defining the fault scenario \{time:\{fxns: {[}modes{]}\},\}

\item {} 
\sphinxAtStartPar
\sphinxstyleliteralstrong{\sphinxupquote{track}} (\sphinxstyleliteralemphasis{\sphinxupquote{str}}\sphinxstyleliteralemphasis{\sphinxupquote{ (}}\sphinxstyleliteralemphasis{\sphinxupquote{'all'}}\sphinxstyleliteralemphasis{\sphinxupquote{, }}\sphinxstyleliteralemphasis{\sphinxupquote{'functions'}}\sphinxstyleliteralemphasis{\sphinxupquote{, }}\sphinxstyleliteralemphasis{\sphinxupquote{'flows'}}\sphinxstyleliteralemphasis{\sphinxupquote{, }}\sphinxstyleliteralemphasis{\sphinxupquote{'valparams'}}\sphinxstyleliteralemphasis{\sphinxupquote{, }}\sphinxstyleliteralemphasis{\sphinxupquote{dict}}\sphinxstyleliteralemphasis{\sphinxupquote{, }}\sphinxstyleliteralemphasis{\sphinxupquote{'none'}}\sphinxstyleliteralemphasis{\sphinxupquote{)}}\sphinxstyleliteralemphasis{\sphinxupquote{, }}\sphinxstyleliteralemphasis{\sphinxupquote{optional}}) – Which model states to track over time, which can be given as ‘functions’, ‘flows’,
‘all’, ‘none’, ‘valparams’ (model states specified in mdl.valparams),
or a dict of form \{‘functions’:\{‘fxn1’:’att1’\}, ‘flows’:\{‘flow1’:’att1’\}\}
The default is ‘all’.

\item {} 
\sphinxAtStartPar
\sphinxstyleliteralstrong{\sphinxupquote{rate}} (\sphinxstyleliteralemphasis{\sphinxupquote{float}}\sphinxstyleliteralemphasis{\sphinxupquote{, }}\sphinxstyleliteralemphasis{\sphinxupquote{optional}}) – Input rate for the sequence (must be calculated elsewhere)

\item {} 
\sphinxAtStartPar
\sphinxstyleliteralstrong{\sphinxupquote{gtype}} (\sphinxstyleliteralemphasis{\sphinxupquote{str}}\sphinxstyleliteralemphasis{\sphinxupquote{, }}\sphinxstyleliteralemphasis{\sphinxupquote{optional}}) – The graph type to return (‘bipartite’/’normal’/’typegraph’). The default is ‘bipartite’.

\item {} 
\sphinxAtStartPar
\sphinxstyleliteralstrong{\sphinxupquote{track\_times}} (\sphinxstyleliteralemphasis{\sphinxupquote{str/tuple}}) – \begin{description}
\item[{Defines what times to include in the history. Options are:}] \leavevmode
\sphinxAtStartPar
’all’–all simulated times
(‘interval’, n)–includes every nth time in the history
(‘times’, {[}t1, … tn{]})–only includes times defined in the vector {[}t1 … tn{]}

\end{description}


\item {} 
\sphinxAtStartPar
\sphinxstyleliteralstrong{\sphinxupquote{protect}} (\sphinxstyleliteralemphasis{\sphinxupquote{bool}}) – \begin{description}
\item[{Whether or not to protect the model object via copying}] \leavevmode
\sphinxAtStartPar
True (default) \sphinxhyphen{} re\sphinxhyphen{}instances the model so that multiple simulations can be run successively without causing problems
False \sphinxhyphen{} Thus, the model object that is returned can be modified and analyzed if needed

\end{description}


\item {} 
\sphinxAtStartPar
\sphinxstyleliteralstrong{\sphinxupquote{run\_stochastic}} (\sphinxstyleliteralemphasis{\sphinxupquote{bool}}) – Whether to run stochastic behaviors or use default values for stochastic variables. Default is False.

\item {} 
\sphinxAtStartPar
\sphinxstyleliteralstrong{\sphinxupquote{**kwargs}} (\sphinxstyleliteralemphasis{\sphinxupquote{kwargs}}\sphinxstyleliteralemphasis{\sphinxupquote{ (}}\sphinxstyleliteralemphasis{\sphinxupquote{params}}\sphinxstyleliteralemphasis{\sphinxupquote{, }}\sphinxstyleliteralemphasis{\sphinxupquote{modelparams}}\sphinxstyleliteralemphasis{\sphinxupquote{, }}\sphinxstyleliteralemphasis{\sphinxupquote{and/or valparams}}\sphinxstyleliteralemphasis{\sphinxupquote{)}}) – passing parameter dictionaries (e.g., params, modelparams, valparams) instantiates the model
to be simulated with the given parameters. Parameter dictionaries do not
need to be complete (if incomplete)

\end{itemize}

\item[{Returns}] \leavevmode
\sphinxAtStartPar
\begin{itemize}
\item {} 
\sphinxAtStartPar
\sphinxstylestrong{endresult} (\sphinxstyleemphasis{dict}) – A dictionary summary of results at the end of the simulation with structure \{flows:\{flow:attribute:value\},faults:\{function:\{faults\}\}, classification:\{rate:val, cost:val, expected cost: val\}

\item {} 
\sphinxAtStartPar
\sphinxstylestrong{resgraph} (\sphinxstyleemphasis{networkx.classes.graph.Graph}) – A graph object with function faults and degraded flows noted as attributes

\item {} 
\sphinxAtStartPar
\sphinxstylestrong{mdlhists} (\sphinxstyleemphasis{dict}) – A dictionary of the states of the model of each fault scenario over time.

\end{itemize}


\end{description}\end{quote}

\end{fulllineitems}

\index{nested\_approach() (in module fmdtools.faultsim.propagate)@\spxentry{nested\_approach()}\spxextra{in module fmdtools.faultsim.propagate}}

\begin{fulllineitems}
\phantomsection\label{\detokenize{docs/fmdtools.faultsim:fmdtools.faultsim.propagate.nested_approach}}\pysiglinewithargsret{\sphinxcode{\sphinxupquote{fmdtools.faultsim.propagate.}}\sphinxbfcode{\sphinxupquote{nested\_approach}}}{\emph{\DUrole{n}{mdl}}, \emph{\DUrole{n}{nomapp}}, \emph{\DUrole{n}{staged}\DUrole{o}{=}\DUrole{default_value}{False}}, \emph{\DUrole{n}{track}\DUrole{o}{=}\DUrole{default_value}{'all'}}, \emph{\DUrole{n}{get\_phases}\DUrole{o}{=}\DUrole{default_value}{False}}, \emph{\DUrole{n}{showprogress}\DUrole{o}{=}\DUrole{default_value}{True}}, \emph{\DUrole{n}{pool}\DUrole{o}{=}\DUrole{default_value}{False}}, \emph{\DUrole{n}{track\_times}\DUrole{o}{=}\DUrole{default_value}{'all'}}, \emph{\DUrole{n}{run\_stochastic}\DUrole{o}{=}\DUrole{default_value}{False}}, \emph{\DUrole{o}{**}\DUrole{n}{app\_args}}}{}
\sphinxAtStartPar
Simulates a set of fault modes within a set of nominal scenarios defined by a nominal approach.
\begin{quote}\begin{description}
\item[{Parameters}] \leavevmode\begin{itemize}
\item {} 
\sphinxAtStartPar
\sphinxstyleliteralstrong{\sphinxupquote{mdl}} ({\hyperref[\detokenize{docs/fmdtools:fmdtools.modeldef.Model}]{\sphinxcrossref{\sphinxstyleliteralemphasis{\sphinxupquote{Model}}}}}) – Model Object to use in the simulation.

\item {} 
\sphinxAtStartPar
\sphinxstyleliteralstrong{\sphinxupquote{nomapp}} ({\hyperref[\detokenize{docs/fmdtools:fmdtools.modeldef.NominalApproach}]{\sphinxcrossref{\sphinxstyleliteralemphasis{\sphinxupquote{NominalApproach}}}}}) – NominalApproach defining the nominal situations the model will be run over

\item {} 
\sphinxAtStartPar
\sphinxstyleliteralstrong{\sphinxupquote{staged}} (\sphinxstyleliteralemphasis{\sphinxupquote{bool}}\sphinxstyleliteralemphasis{\sphinxupquote{, }}\sphinxstyleliteralemphasis{\sphinxupquote{optional}}) – Whether to inject the fault in a copy of the nominal model at the fault time (True) or instantiate a new model for the fault (False). Setting to True roughly halves execution time. The default is False.

\item {} 
\sphinxAtStartPar
\sphinxstyleliteralstrong{\sphinxupquote{track}} (\sphinxstyleliteralemphasis{\sphinxupquote{str}}\sphinxstyleliteralemphasis{\sphinxupquote{ (}}\sphinxstyleliteralemphasis{\sphinxupquote{'all'}}\sphinxstyleliteralemphasis{\sphinxupquote{, }}\sphinxstyleliteralemphasis{\sphinxupquote{'functions'}}\sphinxstyleliteralemphasis{\sphinxupquote{, }}\sphinxstyleliteralemphasis{\sphinxupquote{'flows'}}\sphinxstyleliteralemphasis{\sphinxupquote{, }}\sphinxstyleliteralemphasis{\sphinxupquote{'valparams'}}\sphinxstyleliteralemphasis{\sphinxupquote{, }}\sphinxstyleliteralemphasis{\sphinxupquote{dict}}\sphinxstyleliteralemphasis{\sphinxupquote{, }}\sphinxstyleliteralemphasis{\sphinxupquote{'none'}}\sphinxstyleliteralemphasis{\sphinxupquote{)}}\sphinxstyleliteralemphasis{\sphinxupquote{, }}\sphinxstyleliteralemphasis{\sphinxupquote{optional}}) – Which model states to track over time, which can be given as ‘functions’, ‘flows’,
‘all’, ‘none’, ‘valparams’ (model states specified in mdl.valparams),
or a dict of form \{‘functions’:\{‘fxn1’:’att1’\}, ‘flows’:\{‘flow1’:’att1’\}\}
The default is ‘all’.

\item {} 
\sphinxAtStartPar
\sphinxstyleliteralstrong{\sphinxupquote{get\_phases}} (\sphinxstyleliteralemphasis{\sphinxupquote{Bool/List/Dict}}\sphinxstyleliteralemphasis{\sphinxupquote{, }}\sphinxstyleliteralemphasis{\sphinxupquote{optional}}) – Whether and how to use nominal simulation phases to set up the SampleApproach. The default is False.
\sphinxhyphen{} If True, all phases from the nominal simulation are passed to SampleApproach()
\sphinxhyphen{} If a list {[}‘Fxn1’, ‘Fxn2’ etc.{]}, only the phases from the listed functions will be passed.
\sphinxhyphen{} If a dict \{‘Fxn1’:’phase1’\}, only the phase ‘phase1’ in the function ‘Fxn1’ will be passed.

\item {} 
\sphinxAtStartPar
\sphinxstyleliteralstrong{\sphinxupquote{pool}} (\sphinxstyleliteralemphasis{\sphinxupquote{process pool}}\sphinxstyleliteralemphasis{\sphinxupquote{, }}\sphinxstyleliteralemphasis{\sphinxupquote{optional}}) – Process Pool Object from multiprocessing or pathos packages. Pathos is recommended.
e.g. parallelpool = mp.pool(n) for n cores (multiprocessing)
or parallelpool = ProcessPool(nodes=n) for n cores (pathos)
If False, the set of scenarios is run serially. The default is False

\item {} 
\sphinxAtStartPar
\sphinxstyleliteralstrong{\sphinxupquote{showprogress}} (\sphinxstyleliteralemphasis{\sphinxupquote{bool}}\sphinxstyleliteralemphasis{\sphinxupquote{, }}\sphinxstyleliteralemphasis{\sphinxupquote{optional}}) – whether to show a progress bar during execution. default is true

\item {} 
\sphinxAtStartPar
\sphinxstyleliteralstrong{\sphinxupquote{track\_times}} (\sphinxstyleliteralemphasis{\sphinxupquote{str/tuple}}) – \begin{description}
\item[{Defines what times to include in the history. Options are:}] \leavevmode
\sphinxAtStartPar
’all’–all simulated times
(‘interval’, n)–includes every nth time in the history
(‘times’, {[}t1, … tn{]})–only includes times defined in the vector {[}t1 … tn{]}

\end{description}


\item {} 
\sphinxAtStartPar
\sphinxstyleliteralstrong{\sphinxupquote{run\_stochastic}} (\sphinxstyleliteralemphasis{\sphinxupquote{bool}}) – Whether to run stochastic behaviors or use default values for stochastic variables. Default is False.

\item {} 
\sphinxAtStartPar
\sphinxstyleliteralstrong{\sphinxupquote{**app\_args}} (\sphinxstyleliteralemphasis{\sphinxupquote{kwargs}}) – Keyword arguments for the SampleApproach. See modeldef.SampleApproach documentation.

\end{itemize}

\item[{Returns}] \leavevmode
\sphinxAtStartPar
\begin{itemize}
\item {} 
\sphinxAtStartPar
\sphinxstylestrong{nested\_endclasses} (\sphinxstyleemphasis{dict}) – A nested dictionary with the rate, cost, and expected cost of each scenario run with structure \{‘nomscen1’:endclasses, ‘nomscen2’:mdlhists\}

\item {} 
\sphinxAtStartPar
\sphinxstylestrong{nested\_mdlhists} (\sphinxstyleemphasis{dict}) – A nested dictionary with the history of all model states for each scenario with structure \{‘nomscen1’:mdlhists, ‘nomscen2’:mdlhists\}

\end{itemize}


\end{description}\end{quote}

\end{fulllineitems}

\index{new\_mdl\_params() (in module fmdtools.faultsim.propagate)@\spxentry{new\_mdl\_params()}\spxextra{in module fmdtools.faultsim.propagate}}

\begin{fulllineitems}
\phantomsection\label{\detokenize{docs/fmdtools.faultsim:fmdtools.faultsim.propagate.new_mdl_params}}\pysiglinewithargsret{\sphinxcode{\sphinxupquote{fmdtools.faultsim.propagate.}}\sphinxbfcode{\sphinxupquote{new\_mdl\_params}}}{\emph{\DUrole{n}{mdl}}, \emph{\DUrole{n}{paramdict}}}{}
\sphinxAtStartPar
Creates parameter inputs for a new model. Used for exploring parameter ranges and seeding models.
\begin{quote}\begin{description}
\item[{Parameters}] \leavevmode\begin{itemize}
\item {} 
\sphinxAtStartPar
\sphinxstyleliteralstrong{\sphinxupquote{mdl}} ({\hyperref[\detokenize{docs/fmdtools:fmdtools.modeldef.Model}]{\sphinxcrossref{\sphinxstyleliteralemphasis{\sphinxupquote{Model}}}}}) – fmdtools simulation model

\item {} 
\sphinxAtStartPar
\sphinxstyleliteralstrong{\sphinxupquote{paramdict}} (\sphinxstyleliteralemphasis{\sphinxupquote{Dict}}) – Dict of parameters to update with structure params/modelparams/valparams to update
e.g. \{‘params’:\{‘param1’: 1.0\}\}

\end{itemize}

\item[{Returns}] \leavevmode
\sphinxAtStartPar
\begin{itemize}
\item {} 
\sphinxAtStartPar
\sphinxstylestrong{params} (\sphinxstyleemphasis{dict}) – Updated param dictionary

\item {} 
\sphinxAtStartPar
\sphinxstylestrong{modelparams} (\sphinxstyleemphasis{dict}) – Updated modelparam dictionary

\item {} 
\sphinxAtStartPar
\sphinxstylestrong{valparams} (\sphinxstyleemphasis{dict}) – Updated valparam dictionary

\end{itemize}


\end{description}\end{quote}

\end{fulllineitems}

\index{nominal() (in module fmdtools.faultsim.propagate)@\spxentry{nominal()}\spxextra{in module fmdtools.faultsim.propagate}}

\begin{fulllineitems}
\phantomsection\label{\detokenize{docs/fmdtools.faultsim:fmdtools.faultsim.propagate.nominal}}\pysiglinewithargsret{\sphinxcode{\sphinxupquote{fmdtools.faultsim.propagate.}}\sphinxbfcode{\sphinxupquote{nominal}}}{\emph{\DUrole{n}{mdl}}, \emph{\DUrole{n}{track}\DUrole{o}{=}\DUrole{default_value}{'all'}}, \emph{\DUrole{n}{gtype}\DUrole{o}{=}\DUrole{default_value}{'bipartite'}}, \emph{\DUrole{n}{track\_times}\DUrole{o}{=}\DUrole{default_value}{'all'}}, \emph{\DUrole{n}{protect}\DUrole{o}{=}\DUrole{default_value}{True}}, \emph{\DUrole{n}{run\_stochastic}\DUrole{o}{=}\DUrole{default_value}{False}}, \emph{\DUrole{o}{**}\DUrole{n}{kwargs}}}{}
\sphinxAtStartPar
Runs the model over time in the nominal scenario.
\begin{quote}\begin{description}
\item[{Parameters}] \leavevmode\begin{itemize}
\item {} 
\sphinxAtStartPar
\sphinxstyleliteralstrong{\sphinxupquote{mdl}} ({\hyperref[\detokenize{docs/fmdtools:fmdtools.modeldef.Model}]{\sphinxcrossref{\sphinxstyleliteralemphasis{\sphinxupquote{Model}}}}}) – Model of the system

\item {} 
\sphinxAtStartPar
\sphinxstyleliteralstrong{\sphinxupquote{track}} (\sphinxstyleliteralemphasis{\sphinxupquote{str}}\sphinxstyleliteralemphasis{\sphinxupquote{ (}}\sphinxstyleliteralemphasis{\sphinxupquote{'all'}}\sphinxstyleliteralemphasis{\sphinxupquote{, }}\sphinxstyleliteralemphasis{\sphinxupquote{'functions'}}\sphinxstyleliteralemphasis{\sphinxupquote{, }}\sphinxstyleliteralemphasis{\sphinxupquote{'flows'}}\sphinxstyleliteralemphasis{\sphinxupquote{, }}\sphinxstyleliteralemphasis{\sphinxupquote{'valparams'}}\sphinxstyleliteralemphasis{\sphinxupquote{, }}\sphinxstyleliteralemphasis{\sphinxupquote{dict}}\sphinxstyleliteralemphasis{\sphinxupquote{, }}\sphinxstyleliteralemphasis{\sphinxupquote{'none'}}\sphinxstyleliteralemphasis{\sphinxupquote{)}}\sphinxstyleliteralemphasis{\sphinxupquote{, }}\sphinxstyleliteralemphasis{\sphinxupquote{optional}}) – Which model states to track over time, which can be given as ‘functions’, ‘flows’,
‘all’, ‘none’, ‘valparams’ (model states specified in mdl.valparams),
or a dict of form \{‘functions’:\{‘fxn1’:’att1’\}, ‘flows’:\{‘flow1’:’att1’\}\}
The default is ‘all’.

\item {} 
\sphinxAtStartPar
\sphinxstyleliteralstrong{\sphinxupquote{gtype}} (\sphinxstyleliteralemphasis{\sphinxupquote{TYPE}}\sphinxstyleliteralemphasis{\sphinxupquote{, }}\sphinxstyleliteralemphasis{\sphinxupquote{optional}}) – The type of graph to return (‘bipartite’/’normal’/’typegraph’). The default is ‘bipartite’.

\item {} 
\sphinxAtStartPar
\sphinxstyleliteralstrong{\sphinxupquote{track\_times}} (\sphinxstyleliteralemphasis{\sphinxupquote{str/tuple}}) – \begin{description}
\item[{Defines what times to include in the history. Options are:}] \leavevmode
\sphinxAtStartPar
’all’–all simulated times
(‘interval’, n)–includes every nth time in the history
(‘times’, {[}t1, … tn{]})–only includes times defined in the vector {[}t1 … tn{]}

\end{description}


\item {} 
\sphinxAtStartPar
\sphinxstyleliteralstrong{\sphinxupquote{protect}} (\sphinxstyleliteralemphasis{\sphinxupquote{bool}}) – \begin{description}
\item[{Whether or not to protect the model object via copying}] \leavevmode
\sphinxAtStartPar
True (default) \sphinxhyphen{} re\sphinxhyphen{}instances the model so that multiple simulations can be run successively without causing problems
False \sphinxhyphen{} Thus, the model object that is returned can be modified and analyzed if needed

\end{description}


\item {} 
\sphinxAtStartPar
\sphinxstyleliteralstrong{\sphinxupquote{run\_stochastic}} (\sphinxstyleliteralemphasis{\sphinxupquote{bool}}) – Whether to run stochastic behaviors or use default values. Default is False.

\item {} 
\sphinxAtStartPar
\sphinxstyleliteralstrong{\sphinxupquote{**kwargs}} (\sphinxstyleliteralemphasis{\sphinxupquote{kwargs}}\sphinxstyleliteralemphasis{\sphinxupquote{ (}}\sphinxstyleliteralemphasis{\sphinxupquote{params}}\sphinxstyleliteralemphasis{\sphinxupquote{, }}\sphinxstyleliteralemphasis{\sphinxupquote{modelparams}}\sphinxstyleliteralemphasis{\sphinxupquote{, }}\sphinxstyleliteralemphasis{\sphinxupquote{and/or valparams}}\sphinxstyleliteralemphasis{\sphinxupquote{)}}) – passing parameter dictionaries (e.g., params, modelparams, valparams) instantiates the model
to be simulated with the given parameters. Parameter dictionaries do not
need to be complete (if incomplete)

\end{itemize}

\item[{Returns}] \leavevmode
\sphinxAtStartPar
\begin{itemize}
\item {} 
\sphinxAtStartPar
\sphinxstylestrong{endresult} (\sphinxstyleemphasis{Dict}) – A dictionary summary of results at the end of the simulation with structure \{faults:\{function:\{faults\}\}, classification:\{rate:val, cost:val, expected cost: val\} \}

\item {} 
\sphinxAtStartPar
\sphinxstylestrong{resgraph} (\sphinxstyleemphasis{MultiGraph}) – A networkx graph object with function faults and degraded flows as graph attributes

\item {} 
\sphinxAtStartPar
\sphinxstylestrong{mdlhist} (\sphinxstyleemphasis{Dict}) – A dictionary with a history of modelstates

\end{itemize}


\end{description}\end{quote}

\end{fulllineitems}

\index{nominal\_approach() (in module fmdtools.faultsim.propagate)@\spxentry{nominal\_approach()}\spxextra{in module fmdtools.faultsim.propagate}}

\begin{fulllineitems}
\phantomsection\label{\detokenize{docs/fmdtools.faultsim:fmdtools.faultsim.propagate.nominal_approach}}\pysiglinewithargsret{\sphinxcode{\sphinxupquote{fmdtools.faultsim.propagate.}}\sphinxbfcode{\sphinxupquote{nominal\_approach}}}{\emph{\DUrole{n}{mdl}}, \emph{\DUrole{n}{nomapp}}, \emph{\DUrole{n}{track}\DUrole{o}{=}\DUrole{default_value}{'all'}}, \emph{\DUrole{n}{showprogress}\DUrole{o}{=}\DUrole{default_value}{True}}, \emph{\DUrole{n}{pool}\DUrole{o}{=}\DUrole{default_value}{False}}, \emph{\DUrole{n}{track\_times}\DUrole{o}{=}\DUrole{default_value}{'all'}}, \emph{\DUrole{n}{run\_stochastic}\DUrole{o}{=}\DUrole{default_value}{False}}}{}
\sphinxAtStartPar
Simulates a set of nominal scenarios through a model. Useful to understand
the sets of parameters where the system will run nominally and/or lead to
a fault.
\begin{quote}\begin{description}
\item[{Parameters}] \leavevmode\begin{itemize}
\item {} 
\sphinxAtStartPar
\sphinxstyleliteralstrong{\sphinxupquote{mdl}} ({\hyperref[\detokenize{docs/fmdtools:fmdtools.modeldef.Model}]{\sphinxcrossref{\sphinxstyleliteralemphasis{\sphinxupquote{Model}}}}}) – Model to simulate

\item {} 
\sphinxAtStartPar
\sphinxstyleliteralstrong{\sphinxupquote{nomapp}} ({\hyperref[\detokenize{docs/fmdtools:fmdtools.modeldef.NominalApproach}]{\sphinxcrossref{\sphinxstyleliteralemphasis{\sphinxupquote{NominalApproach}}}}}) – Nominal Approach defining the nominal scenarios to run the system over.

\item {} 
\sphinxAtStartPar
\sphinxstyleliteralstrong{\sphinxupquote{track}} (\sphinxstyleliteralemphasis{\sphinxupquote{str}}\sphinxstyleliteralemphasis{\sphinxupquote{, }}\sphinxstyleliteralemphasis{\sphinxupquote{optional}}) – States to track during simulation. The default is ‘all’.

\item {} 
\sphinxAtStartPar
\sphinxstyleliteralstrong{\sphinxupquote{showprogress}} (\sphinxstyleliteralemphasis{\sphinxupquote{bool}}\sphinxstyleliteralemphasis{\sphinxupquote{, }}\sphinxstyleliteralemphasis{\sphinxupquote{optional}}) – Whether to display progress during simulation. The default is True.

\item {} 
\sphinxAtStartPar
\sphinxstyleliteralstrong{\sphinxupquote{pool}} (\sphinxstyleliteralemphasis{\sphinxupquote{Pool}}\sphinxstyleliteralemphasis{\sphinxupquote{, }}\sphinxstyleliteralemphasis{\sphinxupquote{optional}}) – Parallel pool (e.g. multiprocessing.Pool) to simulate with
(if using parallelism). The default is False.

\item {} 
\sphinxAtStartPar
\sphinxstyleliteralstrong{\sphinxupquote{track\_times}} (\sphinxstyleliteralemphasis{\sphinxupquote{str/tuple}}) – \begin{description}
\item[{Defines what times to include in the history. Options are:}] \leavevmode
\sphinxAtStartPar
’all’–all simulated times
(‘interval’, n)–includes every nth time in the history
(‘times’, {[}t1, … tn{]})–only includes times defined in the vector {[}t1 … tn{]}

\end{description}


\item {} 
\sphinxAtStartPar
\sphinxstyleliteralstrong{\sphinxupquote{run\_stochastic}} (\sphinxstyleliteralemphasis{\sphinxupquote{bool}}) – Whether to run stochastic behaviors or use default values. Default is False.

\end{itemize}

\item[{Returns}] \leavevmode
\sphinxAtStartPar
\begin{itemize}
\item {} 
\sphinxAtStartPar
\sphinxstylestrong{nomapp\_endclasses} (\sphinxstyleemphasis{Dict}) – Classifications of the set of scenarios, with structure \{‘scenname’:classification\}

\item {} 
\sphinxAtStartPar
\sphinxstylestrong{nomapp\_mdlhists} (\sphinxstyleemphasis{Dict}) – Dictionary of model histories, with structure \{‘scenname’:mdlhist\}

\end{itemize}


\end{description}\end{quote}

\end{fulllineitems}

\index{one\_fault() (in module fmdtools.faultsim.propagate)@\spxentry{one\_fault()}\spxextra{in module fmdtools.faultsim.propagate}}

\begin{fulllineitems}
\phantomsection\label{\detokenize{docs/fmdtools.faultsim:fmdtools.faultsim.propagate.one_fault}}\pysiglinewithargsret{\sphinxcode{\sphinxupquote{fmdtools.faultsim.propagate.}}\sphinxbfcode{\sphinxupquote{one\_fault}}}{\emph{\DUrole{n}{mdl}}, \emph{\DUrole{n}{fxnname}}, \emph{\DUrole{n}{faultmode}}, \emph{\DUrole{n}{time}\DUrole{o}{=}\DUrole{default_value}{1}}, \emph{\DUrole{n}{track}\DUrole{o}{=}\DUrole{default_value}{'all'}}, \emph{\DUrole{n}{staged}\DUrole{o}{=}\DUrole{default_value}{False}}, \emph{\DUrole{n}{gtype}\DUrole{o}{=}\DUrole{default_value}{'bipartite'}}, \emph{\DUrole{n}{track\_times}\DUrole{o}{=}\DUrole{default_value}{'all'}}, \emph{\DUrole{n}{protect}\DUrole{o}{=}\DUrole{default_value}{True}}, \emph{\DUrole{n}{run\_stochastic}\DUrole{o}{=}\DUrole{default_value}{False}}, \emph{\DUrole{o}{**}\DUrole{n}{kwargs}}}{}
\sphinxAtStartPar
Runs one fault in the model at a specified time.
\begin{quote}\begin{description}
\item[{Parameters}] \leavevmode\begin{itemize}
\item {} 
\sphinxAtStartPar
\sphinxstyleliteralstrong{\sphinxupquote{mdl}} ({\hyperref[\detokenize{docs/fmdtools:fmdtools.modeldef.Model}]{\sphinxcrossref{\sphinxstyleliteralemphasis{\sphinxupquote{Model}}}}}) – The model to inject the fault in.

\item {} 
\sphinxAtStartPar
\sphinxstyleliteralstrong{\sphinxupquote{fxnname}} (\sphinxstyleliteralemphasis{\sphinxupquote{str}}) – Name of the function with the faultmode

\item {} 
\sphinxAtStartPar
\sphinxstyleliteralstrong{\sphinxupquote{faultmode}} (\sphinxstyleliteralemphasis{\sphinxupquote{str}}) – Name of the faultmode

\item {} 
\sphinxAtStartPar
\sphinxstyleliteralstrong{\sphinxupquote{time}} (\sphinxstyleliteralemphasis{\sphinxupquote{float}}\sphinxstyleliteralemphasis{\sphinxupquote{, }}\sphinxstyleliteralemphasis{\sphinxupquote{optional}}) – Time to inject fault. Must be in the range of model times (i.e. in range(0, end, mdl.tstep)). The default is 0.

\item {} 
\sphinxAtStartPar
\sphinxstyleliteralstrong{\sphinxupquote{track}} (\sphinxstyleliteralemphasis{\sphinxupquote{str}}\sphinxstyleliteralemphasis{\sphinxupquote{ (}}\sphinxstyleliteralemphasis{\sphinxupquote{'all'}}\sphinxstyleliteralemphasis{\sphinxupquote{, }}\sphinxstyleliteralemphasis{\sphinxupquote{'functions'}}\sphinxstyleliteralemphasis{\sphinxupquote{, }}\sphinxstyleliteralemphasis{\sphinxupquote{'flows'}}\sphinxstyleliteralemphasis{\sphinxupquote{, }}\sphinxstyleliteralemphasis{\sphinxupquote{'valparams'}}\sphinxstyleliteralemphasis{\sphinxupquote{, }}\sphinxstyleliteralemphasis{\sphinxupquote{dict}}\sphinxstyleliteralemphasis{\sphinxupquote{, }}\sphinxstyleliteralemphasis{\sphinxupquote{'none'}}\sphinxstyleliteralemphasis{\sphinxupquote{)}}\sphinxstyleliteralemphasis{\sphinxupquote{, }}\sphinxstyleliteralemphasis{\sphinxupquote{optional}}) – Which model states to track over time, which can be given as ‘functions’, ‘flows’,
‘all’, ‘none’, ‘valparams’ (model states specified in mdl.valparams),
or a dict of form \{‘functions’:\{‘fxn1’:’att1’\}, ‘flows’:\{‘flow1’:’att1’\}\}
The default is ‘all’.

\item {} 
\sphinxAtStartPar
\sphinxstyleliteralstrong{\sphinxupquote{staged}} (\sphinxstyleliteralemphasis{\sphinxupquote{bool}}\sphinxstyleliteralemphasis{\sphinxupquote{, }}\sphinxstyleliteralemphasis{\sphinxupquote{optional}}) – Whether to inject the fault in a copy of the nominal model at the fault time (True) or instantiate a new model for the fault (False). The default is False.

\item {} 
\sphinxAtStartPar
\sphinxstyleliteralstrong{\sphinxupquote{gtype}} (\sphinxstyleliteralemphasis{\sphinxupquote{str}}\sphinxstyleliteralemphasis{\sphinxupquote{, }}\sphinxstyleliteralemphasis{\sphinxupquote{optional}}) – The graph type to return (‘bipartite’/’normal’/’typegraph’). The default is ‘bipartite’.

\item {} 
\sphinxAtStartPar
\sphinxstyleliteralstrong{\sphinxupquote{track\_times}} (\sphinxstyleliteralemphasis{\sphinxupquote{str/tuple}}) – \begin{description}
\item[{Defines what times to include in the history. Options are:}] \leavevmode
\sphinxAtStartPar
’all’–all simulated times
(‘interval’, n)–includes every nth time in the history
(‘times’, {[}t1, … tn{]})–only includes times defined in the vector {[}t1 … tn{]}

\end{description}


\item {} 
\sphinxAtStartPar
\sphinxstyleliteralstrong{\sphinxupquote{protect}} (\sphinxstyleliteralemphasis{\sphinxupquote{bool}}) – \begin{description}
\item[{Whether or not to protect the model object via copying}] \leavevmode
\sphinxAtStartPar
True (default) \sphinxhyphen{} re\sphinxhyphen{}instances the model so that multiple simulations can be run successively without causing problems
False \sphinxhyphen{} Thus, the model object that is returned can be modified and analyzed if needed

\end{description}


\item {} 
\sphinxAtStartPar
\sphinxstyleliteralstrong{\sphinxupquote{run\_stochastic}} (\sphinxstyleliteralemphasis{\sphinxupquote{bool}}) – Whether to run stochastic behaviors or use default values. Default is False.

\item {} 
\sphinxAtStartPar
\sphinxstyleliteralstrong{\sphinxupquote{**kwargs}} (\sphinxstyleliteralemphasis{\sphinxupquote{kwargs}}\sphinxstyleliteralemphasis{\sphinxupquote{ (}}\sphinxstyleliteralemphasis{\sphinxupquote{params}}\sphinxstyleliteralemphasis{\sphinxupquote{, }}\sphinxstyleliteralemphasis{\sphinxupquote{modelparams}}\sphinxstyleliteralemphasis{\sphinxupquote{, }}\sphinxstyleliteralemphasis{\sphinxupquote{and/or valparams}}\sphinxstyleliteralemphasis{\sphinxupquote{)}}) – passing parameter dictionaries (e.g., params, modelparams, valparams) instantiates the model
to be simulated with the given parameters. Parameter dictionaries do not
need to be complete (if incomplete)

\end{itemize}

\item[{Returns}] \leavevmode
\sphinxAtStartPar
\begin{itemize}
\item {} 
\sphinxAtStartPar
\sphinxstylestrong{endresult} (\sphinxstyleemphasis{dict}) – A dictionary summary of results at the end of the simulation with structure \{flows:\{flow:attribute:value\},faults:\{function:\{faults\}\}, classification:\{rate:val, cost:val, expected cost: val\}

\item {} 
\sphinxAtStartPar
\sphinxstylestrong{resgraph} (\sphinxstyleemphasis{networkx.classes.graph.Graph}) – A graph object with function faults and degraded flows noted as attributes

\item {} 
\sphinxAtStartPar
\sphinxstylestrong{mdlhists} (\sphinxstyleemphasis{dict}) – A dictionary of the states of the model of each fault scenario over time.

\end{itemize}


\end{description}\end{quote}

\end{fulllineitems}

\index{prop\_one\_scen() (in module fmdtools.faultsim.propagate)@\spxentry{prop\_one\_scen()}\spxextra{in module fmdtools.faultsim.propagate}}

\begin{fulllineitems}
\phantomsection\label{\detokenize{docs/fmdtools.faultsim:fmdtools.faultsim.propagate.prop_one_scen}}\pysiglinewithargsret{\sphinxcode{\sphinxupquote{fmdtools.faultsim.propagate.}}\sphinxbfcode{\sphinxupquote{prop\_one\_scen}}}{\emph{\DUrole{n}{mdl}}, \emph{\DUrole{n}{scen}}, \emph{\DUrole{n}{track}\DUrole{o}{=}\DUrole{default_value}{'all'}}, \emph{\DUrole{n}{staged}\DUrole{o}{=}\DUrole{default_value}{False}}, \emph{\DUrole{n}{ctimes}\DUrole{o}{=}\DUrole{default_value}{{[}{]}}}, \emph{\DUrole{n}{prevhist}\DUrole{o}{=}\DUrole{default_value}{\{\}}}, \emph{\DUrole{n}{track\_times}\DUrole{o}{=}\DUrole{default_value}{'all'}}, \emph{\DUrole{n}{run\_stochastic}\DUrole{o}{=}\DUrole{default_value}{False}}}{}
\sphinxAtStartPar
Runs a fault scenario in the model over time
\begin{quote}\begin{description}
\item[{Parameters}] \leavevmode\begin{itemize}
\item {} 
\sphinxAtStartPar
\sphinxstyleliteralstrong{\sphinxupquote{mdl}} (\sphinxstyleliteralemphasis{\sphinxupquote{model}}) – The model to inject faults in.

\item {} 
\sphinxAtStartPar
\sphinxstyleliteralstrong{\sphinxupquote{scen}} (\sphinxstyleliteralemphasis{\sphinxupquote{Dict}}) – The fault scenario to run. Has structure: \{‘faults’:\{fxn:fault\}, ‘properties’:\{rate, time, name, etc\}\}

\item {} 
\sphinxAtStartPar
\sphinxstyleliteralstrong{\sphinxupquote{track}} (\sphinxstyleliteralemphasis{\sphinxupquote{str}}\sphinxstyleliteralemphasis{\sphinxupquote{ (}}\sphinxstyleliteralemphasis{\sphinxupquote{'all'}}\sphinxstyleliteralemphasis{\sphinxupquote{, }}\sphinxstyleliteralemphasis{\sphinxupquote{'functions'}}\sphinxstyleliteralemphasis{\sphinxupquote{, }}\sphinxstyleliteralemphasis{\sphinxupquote{'flows'}}\sphinxstyleliteralemphasis{\sphinxupquote{, }}\sphinxstyleliteralemphasis{\sphinxupquote{'valparams'}}\sphinxstyleliteralemphasis{\sphinxupquote{, }}\sphinxstyleliteralemphasis{\sphinxupquote{dict}}\sphinxstyleliteralemphasis{\sphinxupquote{, }}\sphinxstyleliteralemphasis{\sphinxupquote{'none'}}\sphinxstyleliteralemphasis{\sphinxupquote{)}}\sphinxstyleliteralemphasis{\sphinxupquote{, }}\sphinxstyleliteralemphasis{\sphinxupquote{optional}}) – Which model states to track over time, which can be given as ‘functions’, ‘flows’,
‘all’, ‘none’, ‘valparams’ (model states specified in mdl.valparams),
or a dict of form \{‘functions’:\{‘fxn1’:’att1’\}, ‘flows’:\{‘flow1’:’att1’\}\}
The default is ‘all’.

\item {} 
\sphinxAtStartPar
\sphinxstyleliteralstrong{\sphinxupquote{staged}} (\sphinxstyleliteralemphasis{\sphinxupquote{bool}}\sphinxstyleliteralemphasis{\sphinxupquote{, }}\sphinxstyleliteralemphasis{\sphinxupquote{optional}}) – Whether to inject the fault in a copy of the nominal model at the fault time (True) or instantiate a new model for the fault (False). Setting to True roughly halves execution time. The default is False.

\item {} 
\sphinxAtStartPar
\sphinxstyleliteralstrong{\sphinxupquote{ctimes}} (\sphinxstyleliteralemphasis{\sphinxupquote{list}}\sphinxstyleliteralemphasis{\sphinxupquote{, }}\sphinxstyleliteralemphasis{\sphinxupquote{optional}}) – List of times to copy the model (for use in staged execution). The default is {[}{]}.

\item {} 
\sphinxAtStartPar
\sphinxstyleliteralstrong{\sphinxupquote{prevhist}} (\sphinxstyleliteralemphasis{\sphinxupquote{dict}}\sphinxstyleliteralemphasis{\sphinxupquote{, }}\sphinxstyleliteralemphasis{\sphinxupquote{optional}}) – The previous results hist (for used in staged execution). The default is \{\}.

\item {} 
\sphinxAtStartPar
\sphinxstyleliteralstrong{\sphinxupquote{run\_stochastic}} (\sphinxstyleliteralemphasis{\sphinxupquote{bool}}) – Whether to run stochastic behaviors or use default values for stochastic variables. Default is False.

\end{itemize}

\item[{Returns}] \leavevmode
\sphinxAtStartPar
\begin{itemize}
\item {} 
\sphinxAtStartPar
\sphinxstylestrong{mdlhist} (\sphinxstyleemphasis{dict}) – A dictionary with a history of modelstates.

\item {} 
\sphinxAtStartPar
\sphinxstylestrong{c\_mdl} (\sphinxstyleemphasis{dict}) – A dictionary of models at each time given in ctimes with structure \{time:model\}

\item {} 
\sphinxAtStartPar
\sphinxstylestrong{track\_times} (\sphinxstyleemphasis{str/tuple}) –
\begin{description}
\item[{Defines what times to include in the history. Options are:}] \leavevmode
\sphinxAtStartPar
’all’–all simulated times
(‘interval’, n)–includes every nth time in the history
(‘times’, {[}t1, … tn{]})–only includes times defined in the vector {[}t1 … tn{]}

\end{description}

\end{itemize}


\end{description}\end{quote}

\end{fulllineitems}

\index{prop\_time() (in module fmdtools.faultsim.propagate)@\spxentry{prop\_time()}\spxextra{in module fmdtools.faultsim.propagate}}

\begin{fulllineitems}
\phantomsection\label{\detokenize{docs/fmdtools.faultsim:fmdtools.faultsim.propagate.prop_time}}\pysiglinewithargsret{\sphinxcode{\sphinxupquote{fmdtools.faultsim.propagate.}}\sphinxbfcode{\sphinxupquote{prop\_time}}}{\emph{\DUrole{n}{mdl}}, \emph{\DUrole{n}{time}}, \emph{\DUrole{n}{initfaults}}, \emph{\DUrole{n}{flowstates}\DUrole{o}{=}\DUrole{default_value}{\{\}}}, \emph{\DUrole{n}{run\_stochastic}\DUrole{o}{=}\DUrole{default_value}{False}}}{}
\sphinxAtStartPar
Propagates faults through model graph.
\begin{quote}\begin{description}
\item[{Parameters}] \leavevmode\begin{itemize}
\item {} 
\sphinxAtStartPar
\sphinxstyleliteralstrong{\sphinxupquote{mdl}} (\sphinxstyleliteralemphasis{\sphinxupquote{model}}) – Model to propagate faults in

\item {} 
\sphinxAtStartPar
\sphinxstyleliteralstrong{\sphinxupquote{time}} (\sphinxstyleliteralemphasis{\sphinxupquote{float}}) – Current time\sphinxhyphen{}step.

\item {} 
\sphinxAtStartPar
\sphinxstyleliteralstrong{\sphinxupquote{initfaults}} (\sphinxstyleliteralemphasis{\sphinxupquote{dict}}) – Faults to inject during this propagation step.

\item {} 
\sphinxAtStartPar
\sphinxstyleliteralstrong{\sphinxupquote{run\_stochastic}} (\sphinxstyleliteralemphasis{\sphinxupquote{bool}}) – Whether to run stochastic behaviors or use default values for stochastic variables. Default is False.

\end{itemize}

\item[{Returns}] \leavevmode
\sphinxAtStartPar
\sphinxstylestrong{flowstates} – States of each flow in the model after propagation

\item[{Return type}] \leavevmode
\sphinxAtStartPar
dict

\end{description}\end{quote}

\end{fulllineitems}

\index{propagate() (in module fmdtools.faultsim.propagate)@\spxentry{propagate()}\spxextra{in module fmdtools.faultsim.propagate}}

\begin{fulllineitems}
\phantomsection\label{\detokenize{docs/fmdtools.faultsim:fmdtools.faultsim.propagate.propagate}}\pysiglinewithargsret{\sphinxcode{\sphinxupquote{fmdtools.faultsim.propagate.}}\sphinxbfcode{\sphinxupquote{propagate}}}{\emph{\DUrole{n}{mdl}}, \emph{\DUrole{n}{initfaults}}, \emph{\DUrole{n}{time}}, \emph{\DUrole{n}{flowstates}\DUrole{o}{=}\DUrole{default_value}{\{\}}}, \emph{\DUrole{n}{run\_stochastic}\DUrole{o}{=}\DUrole{default_value}{False}}}{}
\sphinxAtStartPar
Injects and propagates faults through the graph at one time\sphinxhyphen{}step
\begin{quote}\begin{description}
\item[{Parameters}] \leavevmode\begin{itemize}
\item {} 
\sphinxAtStartPar
\sphinxstyleliteralstrong{\sphinxupquote{mdl}} (\sphinxstyleliteralemphasis{\sphinxupquote{model}}) – The model to propagate the fault in

\item {} 
\sphinxAtStartPar
\sphinxstyleliteralstrong{\sphinxupquote{initfaults}} (\sphinxstyleliteralemphasis{\sphinxupquote{dict}}) – The faults to inject in the model with structure \{fxn:fault\}

\item {} 
\sphinxAtStartPar
\sphinxstyleliteralstrong{\sphinxupquote{time}} (\sphinxstyleliteralemphasis{\sphinxupquote{float}}) – The current timestep.

\item {} 
\sphinxAtStartPar
\sphinxstyleliteralstrong{\sphinxupquote{run\_stochastic}} (\sphinxstyleliteralemphasis{\sphinxupquote{bool}}) – Whether to run stochastic behaviors or use default values for stochastic variables. Default is False.

\end{itemize}

\item[{Returns}] \leavevmode
\sphinxAtStartPar
\sphinxstylestrong{flowstates} – States of the model at the current time\sphinxhyphen{}step.

\item[{Return type}] \leavevmode
\sphinxAtStartPar
dict

\end{description}\end{quote}

\end{fulllineitems}

\index{single\_faults() (in module fmdtools.faultsim.propagate)@\spxentry{single\_faults()}\spxextra{in module fmdtools.faultsim.propagate}}

\begin{fulllineitems}
\phantomsection\label{\detokenize{docs/fmdtools.faultsim:fmdtools.faultsim.propagate.single_faults}}\pysiglinewithargsret{\sphinxcode{\sphinxupquote{fmdtools.faultsim.propagate.}}\sphinxbfcode{\sphinxupquote{single\_faults}}}{\emph{\DUrole{n}{mdl}}, \emph{\DUrole{n}{staged}\DUrole{o}{=}\DUrole{default_value}{False}}, \emph{\DUrole{n}{track}\DUrole{o}{=}\DUrole{default_value}{'all'}}, \emph{\DUrole{n}{pool}\DUrole{o}{=}\DUrole{default_value}{False}}, \emph{\DUrole{n}{showprogress}\DUrole{o}{=}\DUrole{default_value}{True}}, \emph{\DUrole{n}{track\_times}\DUrole{o}{=}\DUrole{default_value}{'all'}}, \emph{\DUrole{n}{protect}\DUrole{o}{=}\DUrole{default_value}{True}}, \emph{\DUrole{n}{run\_stochastic}\DUrole{o}{=}\DUrole{default_value}{False}}, \emph{\DUrole{o}{**}\DUrole{n}{kwargs}}}{}
\sphinxAtStartPar
Creates and propagates a list of failure scenarios in a model.
\begin{description}
\item[{NOTE: When calling in a script using parallel=True, keep the script in the if statement:}] \leavevmode\begin{description}
\item[{“if \_\_name\_\_==’main’:}] \leavevmode
\sphinxAtStartPar
endclasses, mdlhists = single\_faults(mdl)”

\end{description}

\sphinxAtStartPar
Otherwise, the method will keep spawning parallel processes. See multiprocessing documentation.

\end{description}
\begin{quote}\begin{description}
\item[{Parameters}] \leavevmode\begin{itemize}
\item {} 
\sphinxAtStartPar
\sphinxstyleliteralstrong{\sphinxupquote{mdl}} (\sphinxstyleliteralemphasis{\sphinxupquote{model}}) – The model to inject faults in

\item {} 
\sphinxAtStartPar
\sphinxstyleliteralstrong{\sphinxupquote{staged}} (\sphinxstyleliteralemphasis{\sphinxupquote{bool}}\sphinxstyleliteralemphasis{\sphinxupquote{, }}\sphinxstyleliteralemphasis{\sphinxupquote{optional}}) – Whether to inject the fault in a copy of the nominal model at the fault time (True) or instantiate a new model for the fault (False). Setting to True roughly halves execution time. The default is False.

\item {} 
\sphinxAtStartPar
\sphinxstyleliteralstrong{\sphinxupquote{track}} (\sphinxstyleliteralemphasis{\sphinxupquote{str}}\sphinxstyleliteralemphasis{\sphinxupquote{ (}}\sphinxstyleliteralemphasis{\sphinxupquote{'all'}}\sphinxstyleliteralemphasis{\sphinxupquote{, }}\sphinxstyleliteralemphasis{\sphinxupquote{'functions'}}\sphinxstyleliteralemphasis{\sphinxupquote{, }}\sphinxstyleliteralemphasis{\sphinxupquote{'flows'}}\sphinxstyleliteralemphasis{\sphinxupquote{, }}\sphinxstyleliteralemphasis{\sphinxupquote{'valparams'}}\sphinxstyleliteralemphasis{\sphinxupquote{, }}\sphinxstyleliteralemphasis{\sphinxupquote{dict}}\sphinxstyleliteralemphasis{\sphinxupquote{, }}\sphinxstyleliteralemphasis{\sphinxupquote{'none'}}\sphinxstyleliteralemphasis{\sphinxupquote{)}}\sphinxstyleliteralemphasis{\sphinxupquote{, }}\sphinxstyleliteralemphasis{\sphinxupquote{optional}}) – Which model states to track over time, which can be given as ‘functions’, ‘flows’,
‘all’, ‘none’, ‘valparams’ (model states specified in mdl.valparams),
or a dict of form \{‘functions’:\{‘fxn1’:’att1’\}, ‘flows’:\{‘flow1’:’att1’\}\}
The default is ‘all’.

\item {} 
\sphinxAtStartPar
\sphinxstyleliteralstrong{\sphinxupquote{pool}} (\sphinxstyleliteralemphasis{\sphinxupquote{process pool}}\sphinxstyleliteralemphasis{\sphinxupquote{, }}\sphinxstyleliteralemphasis{\sphinxupquote{optional}}) – Process Pool Object from multiprocessing or pathos packages. multiprocessing is recommended.
e.g. parallelpool = mp.pool(n) for n cores (multiprocessing)
or parallelpool = ProcessPool(nodes=n) for n cores (pathos)
If False, the set of scenarios is run serially. The default is False

\item {} 
\sphinxAtStartPar
\sphinxstyleliteralstrong{\sphinxupquote{showprogress}} (\sphinxstyleliteralemphasis{\sphinxupquote{bool}}\sphinxstyleliteralemphasis{\sphinxupquote{, }}\sphinxstyleliteralemphasis{\sphinxupquote{optional}}) – whether to show a progress bar during execution. default is true

\item {} 
\sphinxAtStartPar
\sphinxstyleliteralstrong{\sphinxupquote{track\_times}} (\sphinxstyleliteralemphasis{\sphinxupquote{str/tuple}}) – \begin{description}
\item[{Defines what times to include in the history. Options are:}] \leavevmode
\sphinxAtStartPar
’all’–all simulated times
(‘interval’, n)–includes every nth time in the history
(‘times’, {[}t1, … tn{]})–only includes times defined in the vector {[}t1 … tn{]}

\end{description}


\item {} 
\sphinxAtStartPar
\sphinxstyleliteralstrong{\sphinxupquote{protect}} (\sphinxstyleliteralemphasis{\sphinxupquote{bool}}) – \begin{description}
\item[{Whether or not to protect the model object via copying}] \leavevmode
\sphinxAtStartPar
True (default) \sphinxhyphen{} re\sphinxhyphen{}instances the model (safe)
False \sphinxhyphen{} model is not re\sphinxhyphen{}instantiated (unsafe–do not use model afterwards)

\end{description}


\item {} 
\sphinxAtStartPar
\sphinxstyleliteralstrong{\sphinxupquote{run\_stochastic}} (\sphinxstyleliteralemphasis{\sphinxupquote{bool}}) – Whether to run stochastic behaviors or use default values for stochastic variables. Default is False.

\item {} 
\sphinxAtStartPar
\sphinxstyleliteralstrong{\sphinxupquote{**kwargs}} (\sphinxstyleliteralemphasis{\sphinxupquote{kwargs}}\sphinxstyleliteralemphasis{\sphinxupquote{ (}}\sphinxstyleliteralemphasis{\sphinxupquote{params}}\sphinxstyleliteralemphasis{\sphinxupquote{, }}\sphinxstyleliteralemphasis{\sphinxupquote{modelparams}}\sphinxstyleliteralemphasis{\sphinxupquote{, }}\sphinxstyleliteralemphasis{\sphinxupquote{and/or valparams}}\sphinxstyleliteralemphasis{\sphinxupquote{)}}) – passing parameter dictionaries (e.g., params, modelparams, valparams) instantiates the model
to be simulated with the given parameters. Parameter dictionaries do not
need to be complete (if incomplete)

\end{itemize}

\item[{Returns}] \leavevmode
\sphinxAtStartPar
\begin{itemize}
\item {} 
\sphinxAtStartPar
\sphinxstylestrong{endclasses} (\sphinxstyleemphasis{dict}) – A dictionary with the rate, cost, and expected cost of each scenario run with structure \{scenname:\{expected cost, cost, rate\}\}

\item {} 
\sphinxAtStartPar
\sphinxstylestrong{mdlhists} (\sphinxstyleemphasis{dict}) – A dictionary with the history of all model states for each scenario (including the nominal)

\end{itemize}


\end{description}\end{quote}

\end{fulllineitems}

\index{update\_flowhist() (in module fmdtools.faultsim.propagate)@\spxentry{update\_flowhist()}\spxextra{in module fmdtools.faultsim.propagate}}

\begin{fulllineitems}
\phantomsection\label{\detokenize{docs/fmdtools.faultsim:fmdtools.faultsim.propagate.update_flowhist}}\pysiglinewithargsret{\sphinxcode{\sphinxupquote{fmdtools.faultsim.propagate.}}\sphinxbfcode{\sphinxupquote{update\_flowhist}}}{\emph{\DUrole{n}{mdl}}, \emph{\DUrole{n}{mdlhist}}, \emph{\DUrole{n}{t\_ind}}}{}
\sphinxAtStartPar
Updates the flows in the model history at t\_ind
\begin{quote}\begin{description}
\item[{Parameters}] \leavevmode\begin{itemize}
\item {} 
\sphinxAtStartPar
\sphinxstyleliteralstrong{\sphinxupquote{mdl}} (\sphinxstyleliteralemphasis{\sphinxupquote{model}}) – the Model object

\item {} 
\sphinxAtStartPar
\sphinxstyleliteralstrong{\sphinxupquote{mdlhist}} (\sphinxstyleliteralemphasis{\sphinxupquote{dict}}) – dictionary of model histories for functions/flows

\item {} 
\sphinxAtStartPar
\sphinxstyleliteralstrong{\sphinxupquote{t\_ind}} (\sphinxstyleliteralemphasis{\sphinxupquote{int}}) – index to update the history at

\end{itemize}

\end{description}\end{quote}

\end{fulllineitems}

\index{update\_fxnhist() (in module fmdtools.faultsim.propagate)@\spxentry{update\_fxnhist()}\spxextra{in module fmdtools.faultsim.propagate}}

\begin{fulllineitems}
\phantomsection\label{\detokenize{docs/fmdtools.faultsim:fmdtools.faultsim.propagate.update_fxnhist}}\pysiglinewithargsret{\sphinxcode{\sphinxupquote{fmdtools.faultsim.propagate.}}\sphinxbfcode{\sphinxupquote{update\_fxnhist}}}{\emph{\DUrole{n}{mdl}}, \emph{\DUrole{n}{mdlhist}}, \emph{\DUrole{n}{t\_ind}}}{}
\sphinxAtStartPar
Updates the functions (faults and states) in the model history at t\_ind
\begin{quote}\begin{description}
\item[{Parameters}] \leavevmode\begin{itemize}
\item {} 
\sphinxAtStartPar
\sphinxstyleliteralstrong{\sphinxupquote{mdl}} (\sphinxstyleliteralemphasis{\sphinxupquote{model}}) – the Model object

\item {} 
\sphinxAtStartPar
\sphinxstyleliteralstrong{\sphinxupquote{mdlhist}} (\sphinxstyleliteralemphasis{\sphinxupquote{dict}}) – dictionary of model histories for functions/flows

\item {} 
\sphinxAtStartPar
\sphinxstyleliteralstrong{\sphinxupquote{t\_ind}} (\sphinxstyleliteralemphasis{\sphinxupquote{int}}) – index to update the history at

\end{itemize}

\end{description}\end{quote}

\end{fulllineitems}

\index{update\_mdlhist() (in module fmdtools.faultsim.propagate)@\spxentry{update\_mdlhist()}\spxextra{in module fmdtools.faultsim.propagate}}

\begin{fulllineitems}
\phantomsection\label{\detokenize{docs/fmdtools.faultsim:fmdtools.faultsim.propagate.update_mdlhist}}\pysiglinewithargsret{\sphinxcode{\sphinxupquote{fmdtools.faultsim.propagate.}}\sphinxbfcode{\sphinxupquote{update\_mdlhist}}}{\emph{\DUrole{n}{mdl}}, \emph{\DUrole{n}{mdlhist}}, \emph{\DUrole{n}{t\_ind}}, \emph{\DUrole{n}{track}\DUrole{o}{=}\DUrole{default_value}{'all'}}}{}
\sphinxAtStartPar
Updates the model history at a given time.
\begin{quote}\begin{description}
\item[{Parameters}] \leavevmode\begin{itemize}
\item {} 
\sphinxAtStartPar
\sphinxstyleliteralstrong{\sphinxupquote{mdl}} (\sphinxstyleliteralemphasis{\sphinxupquote{model}}) – Model at the timestep

\item {} 
\sphinxAtStartPar
\sphinxstyleliteralstrong{\sphinxupquote{mdlhist}} (\sphinxstyleliteralemphasis{\sphinxupquote{dict}}) – History of model states (a dict with a vector of each state)

\item {} 
\sphinxAtStartPar
\sphinxstyleliteralstrong{\sphinxupquote{t\_ind}} (\sphinxstyleliteralemphasis{\sphinxupquote{float}}) – The time to update the model history at.

\item {} 
\sphinxAtStartPar
\sphinxstyleliteralstrong{\sphinxupquote{track}} (\sphinxstyleliteralemphasis{\sphinxupquote{str}}\sphinxstyleliteralemphasis{\sphinxupquote{ (}}\sphinxstyleliteralemphasis{\sphinxupquote{'all'}}\sphinxstyleliteralemphasis{\sphinxupquote{, }}\sphinxstyleliteralemphasis{\sphinxupquote{'functions'}}\sphinxstyleliteralemphasis{\sphinxupquote{, }}\sphinxstyleliteralemphasis{\sphinxupquote{'flows'}}\sphinxstyleliteralemphasis{\sphinxupquote{, }}\sphinxstyleliteralemphasis{\sphinxupquote{'valparams'}}\sphinxstyleliteralemphasis{\sphinxupquote{, }}\sphinxstyleliteralemphasis{\sphinxupquote{dict}}\sphinxstyleliteralemphasis{\sphinxupquote{, }}\sphinxstyleliteralemphasis{\sphinxupquote{'none'}}\sphinxstyleliteralemphasis{\sphinxupquote{)}}\sphinxstyleliteralemphasis{\sphinxupquote{, }}\sphinxstyleliteralemphasis{\sphinxupquote{optional}}) – Which model states to track over time, which can be given as ‘functions’, ‘flows’,
‘all’, ‘none’, ‘valparams’ (model states specified in mdl.valparams),
or a dict of form \{‘functions’:\{‘fxn1’:’att1’\}, ‘flows’:\{‘flow1’:’att1’\}\}
The default is ‘all’.

\end{itemize}

\end{description}\end{quote}

\end{fulllineitems}

\index{update\_params() (in module fmdtools.faultsim.propagate)@\spxentry{update\_params()}\spxextra{in module fmdtools.faultsim.propagate}}

\begin{fulllineitems}
\phantomsection\label{\detokenize{docs/fmdtools.faultsim:fmdtools.faultsim.propagate.update_params}}\pysiglinewithargsret{\sphinxcode{\sphinxupquote{fmdtools.faultsim.propagate.}}\sphinxbfcode{\sphinxupquote{update\_params}}}{\emph{\DUrole{n}{params}}, \emph{\DUrole{o}{**}\DUrole{n}{kwargs}}}{}
\sphinxAtStartPar
Updates a dictionary with the given keyword arguments
\begin{quote}\begin{description}
\item[{Parameters}] \leavevmode\begin{itemize}
\item {} 
\sphinxAtStartPar
\sphinxstyleliteralstrong{\sphinxupquote{params}} (\sphinxstyleliteralemphasis{\sphinxupquote{dict}}) – Parameter dictionary

\item {} 
\sphinxAtStartPar
\sphinxstyleliteralstrong{\sphinxupquote{**kwargs}} (\sphinxstyleliteralemphasis{\sphinxupquote{kwargs}}) – New arguments to add/update in the parameter dictionary

\end{itemize}

\item[{Returns}] \leavevmode
\sphinxAtStartPar
\sphinxstylestrong{params} – Updated parameter dictionary

\item[{Return type}] \leavevmode
\sphinxAtStartPar
dict

\end{description}\end{quote}

\end{fulllineitems}



\subsection{fmdtools.resultdisp package}
\label{\detokenize{docs/fmdtools.resultdisp:fmdtools-resultdisp-package}}\label{\detokenize{docs/fmdtools.resultdisp::doc}}
\noindent\sphinxincludegraphics[width=800\sphinxpxdimen]{{resultdisp}.png}

\sphinxAtStartPar
The resultdisp package is organized into the  {\hyperref[\detokenize{docs/fmdtools.resultdisp:module-fmdtools.resultdisp.process}]{\sphinxcrossref{\sphinxcode{\sphinxupquote{fmdtools.resultdisp.process}}}}}, {\hyperref[\detokenize{docs/fmdtools.resultdisp:module-fmdtools.resultdisp.plot}]{\sphinxcrossref{\sphinxcode{\sphinxupquote{fmdtools.resultdisp.plot}}}}}, {\hyperref[\detokenize{docs/fmdtools.resultdisp:module-fmdtools.resultdisp.graph}]{\sphinxcrossref{\sphinxcode{\sphinxupquote{fmdtools.resultdisp.graph}}}}}, and  {\hyperref[\detokenize{docs/fmdtools.resultdisp:module-fmdtools.resultdisp.tabulate}]{\sphinxcrossref{\sphinxcode{\sphinxupquote{fmdtools.resultdisp.tabulate}}}}} modules, as shown above. {\hyperref[\detokenize{docs/fmdtools.resultdisp:module-fmdtools.resultdisp.process}]{\sphinxcrossref{\sphinxcode{\sphinxupquote{fmdtools.resultdisp.process}}}}} is used to process simulation results into convenient metrics and statistics for an analysis. The rest of the modules can be thought of as \sphinxstyleemphasis{convenience interfaces} for their respective packages, where:
\begin{itemize}
\item {} 
\sphinxAtStartPar
\sphinxcode{\sphinxupquote{plot}} creates plots in \sphinxcode{\sphinxupquote{matplotlib}} for simulation results (e.g., model histories, end\sphinxhyphen{}state classifications, etc).

\item {} 
\sphinxAtStartPar
\sphinxcode{\sphinxupquote{graph}} creates visualizations of the model graph using \sphinxcode{\sphinxupquote{NetworkX}}, \sphinxcode{\sphinxupquote{Netgraph}} and/or \sphinxcode{\sphinxupquote{Graphviz}} packages.

\item {} 
\sphinxAtStartPar
\sphinxcode{\sphinxupquote{tabulate}} creates \sphinxcode{\sphinxupquote{pandas}} tables of desired simulation metrics.

\end{itemize}

\sphinxAtStartPar
The model reference for each of these is provided below:


\subsubsection{fmdtools.resultdisp.graph}
\label{\detokenize{docs/fmdtools.resultdisp:module-fmdtools.resultdisp.graph}}\label{\detokenize{docs/fmdtools.resultdisp:fmdtools-resultdisp-graph}}\index{module@\spxentry{module}!fmdtools.resultdisp.graph@\spxentry{fmdtools.resultdisp.graph}}\index{fmdtools.resultdisp.graph@\spxentry{fmdtools.resultdisp.graph}!module@\spxentry{module}}
\sphinxAtStartPar
Description: Gives graph\sphinxhyphen{}level visualizations of the model using installed renderers.
\begin{description}
\item[{Public user\sphinxhyphen{}facing methods:}] \leavevmode\begin{itemize}
\item {} 
\sphinxAtStartPar
{\hyperref[\detokenize{docs/fmdtools.resultdisp:fmdtools.resultdisp.graph.set_pos}]{\sphinxcrossref{\sphinxcode{\sphinxupquote{set\_pos()}}}}}:              Set graph node positions manually (uses netgraph)

\item {} 
\sphinxAtStartPar
{\hyperref[\detokenize{docs/fmdtools.resultdisp:fmdtools.resultdisp.graph.show}]{\sphinxcrossref{\sphinxcode{\sphinxupquote{show()}}}}}:                         Plots a single graph object g. Has options for heatmaps/overlays and matplotlib/graphviz/netgraph/pyvis renderers.

\item {} 
\sphinxAtStartPar
{\hyperref[\detokenize{docs/fmdtools.resultdisp:fmdtools.resultdisp.graph.exec_order}]{\sphinxcrossref{\sphinxcode{\sphinxupquote{exec\_order()}}}}}:                   Displays the propagation order and type (dynamic/static) in the model. Works with matplotlib/graphviz/netgraph renderers.

\item {} 
\sphinxAtStartPar
{\hyperref[\detokenize{docs/fmdtools.resultdisp:fmdtools.resultdisp.graph.history}]{\sphinxcrossref{\sphinxcode{\sphinxupquote{history()}}}}}:                      Displays plots of the graph over time given a dict history of graph objects.  Works with matplotlib/graphviz/netgraph renderers.

\item {} 
\sphinxAtStartPar
{\hyperref[\detokenize{docs/fmdtools.resultdisp:fmdtools.resultdisp.graph.result_from}]{\sphinxcrossref{\sphinxcode{\sphinxupquote{result\_from()}}}}}:                  Plots a representation of the model graph at a specific time in the results history. Works with matplotlib/graphviz/netgraph renderers.

\item {} 
\sphinxAtStartPar
{\hyperref[\detokenize{docs/fmdtools.resultdisp:fmdtools.resultdisp.graph.results_from}]{\sphinxcrossref{\sphinxcode{\sphinxupquote{results\_from()}}}}}:                 Plots a set of representations of the model graph at given times in the results history. Works with matplotlib/graphviz/netgraph renderers.

\item {} 
\sphinxAtStartPar
{\hyperref[\detokenize{docs/fmdtools.resultdisp:fmdtools.resultdisp.graph.animation_from}]{\sphinxcrossref{\sphinxcode{\sphinxupquote{animation\_from()}}}}}:               Creates an animation of the model graph using results at given times in the results history.  Works with matplotlib/netgraph renderers.

\end{itemize}

\end{description}
\index{animation\_from() (in module fmdtools.resultdisp.graph)@\spxentry{animation\_from()}\spxextra{in module fmdtools.resultdisp.graph}}

\begin{fulllineitems}
\phantomsection\label{\detokenize{docs/fmdtools.resultdisp:fmdtools.resultdisp.graph.animation_from}}\pysiglinewithargsret{\sphinxcode{\sphinxupquote{fmdtools.resultdisp.graph.}}\sphinxbfcode{\sphinxupquote{animation\_from}}}{\emph{\DUrole{n}{mdl}}, \emph{\DUrole{n}{reshist}}, \emph{\DUrole{n}{times}\DUrole{o}{=}\DUrole{default_value}{'all'}}, \emph{\DUrole{n}{faultscen}\DUrole{o}{=}\DUrole{default_value}{{[}{]}}}, \emph{\DUrole{n}{gtype}\DUrole{o}{=}\DUrole{default_value}{'bipartite'}}, \emph{\DUrole{n}{figsize}\DUrole{o}{=}\DUrole{default_value}{(6, 4)}}, \emph{\DUrole{n}{showfaultlabels}\DUrole{o}{=}\DUrole{default_value}{True}}, \emph{\DUrole{n}{scale}\DUrole{o}{=}\DUrole{default_value}{1}}, \emph{\DUrole{n}{show}\DUrole{o}{=}\DUrole{default_value}{False}}, \emph{\DUrole{n}{pos}\DUrole{o}{=}\DUrole{default_value}{{[}{]}}}, \emph{\DUrole{n}{colors}\DUrole{o}{=}\DUrole{default_value}{{[}'lightgray', 'orange', 'red'{]}}}, \emph{\DUrole{n}{renderer}\DUrole{o}{=}\DUrole{default_value}{'matplotlib'}}}{}
\sphinxAtStartPar
Creates an animation of the model graph using results at given times in the results history.
To view, use \%matplotlib qt from spyder or \%matplotlib notebook from jupyter
To save (or do anything useful)h, make sure ffmpeg is installed  \sphinxurl{https://www.wikihow.com/Install-FFmpeg-on-Windows}
\begin{quote}\begin{description}
\item[{Parameters}] \leavevmode\begin{itemize}
\item {} 
\sphinxAtStartPar
\sphinxstyleliteralstrong{\sphinxupquote{mdl}} (\sphinxstyleliteralemphasis{\sphinxupquote{model}}) – The model the faults were run in.

\item {} 
\sphinxAtStartPar
\sphinxstyleliteralstrong{\sphinxupquote{reshist}} (\sphinxstyleliteralemphasis{\sphinxupquote{dict}}) – A dictionary of results (from process.hists() or process.typehist() for the typegraph option)

\item {} 
\sphinxAtStartPar
\sphinxstyleliteralstrong{\sphinxupquote{times}} (\sphinxstyleliteralemphasis{\sphinxupquote{list}}\sphinxstyleliteralemphasis{\sphinxupquote{ or }}\sphinxstyleliteralemphasis{\sphinxupquote{'all'}}) – The times in the history to plot the graph at. If ‘all’, plots them all

\item {} 
\sphinxAtStartPar
\sphinxstyleliteralstrong{\sphinxupquote{faultscen}} (\sphinxstyleliteralemphasis{\sphinxupquote{str}}\sphinxstyleliteralemphasis{\sphinxupquote{, }}\sphinxstyleliteralemphasis{\sphinxupquote{optional}}) – Name of the fault scenario. The default is {[}{]}.

\item {} 
\sphinxAtStartPar
\sphinxstyleliteralstrong{\sphinxupquote{gtype}} (\sphinxstyleliteralemphasis{\sphinxupquote{str}}\sphinxstyleliteralemphasis{\sphinxupquote{, }}\sphinxstyleliteralemphasis{\sphinxupquote{optional}}) – The type of graph to plot (normal or bipartite). The default is ‘bipartite’.

\item {} 
\sphinxAtStartPar
\sphinxstyleliteralstrong{\sphinxupquote{showfaultlabels}} (\sphinxstyleliteralemphasis{\sphinxupquote{bool}}\sphinxstyleliteralemphasis{\sphinxupquote{, }}\sphinxstyleliteralemphasis{\sphinxupquote{optional}}) – Whether or not to list faults on the plot. The default is True.

\item {} 
\sphinxAtStartPar
\sphinxstyleliteralstrong{\sphinxupquote{scale}} (\sphinxstyleliteralemphasis{\sphinxupquote{float}}\sphinxstyleliteralemphasis{\sphinxupquote{, }}\sphinxstyleliteralemphasis{\sphinxupquote{optional}}) – Scale factor for the node/label sizes. The default is 1.

\item {} 
\sphinxAtStartPar
\sphinxstyleliteralstrong{\sphinxupquote{show}} (\sphinxstyleliteralemphasis{\sphinxupquote{bool}}\sphinxstyleliteralemphasis{\sphinxupquote{, }}\sphinxstyleliteralemphasis{\sphinxupquote{optional}}) – Whether to show the plot at the end (may be redundant). The default is True.

\item {} 
\sphinxAtStartPar
\sphinxstyleliteralstrong{\sphinxupquote{pos}} (\sphinxstyleliteralemphasis{\sphinxupquote{dict}}\sphinxstyleliteralemphasis{\sphinxupquote{, }}\sphinxstyleliteralemphasis{\sphinxupquote{optional}}) – dict of node positions (if re\sphinxhyphen{}using positions). The default is {[}{]}.

\end{itemize}

\end{description}\end{quote}

\end{fulllineitems}

\index{exec\_order() (in module fmdtools.resultdisp.graph)@\spxentry{exec\_order()}\spxextra{in module fmdtools.resultdisp.graph}}

\begin{fulllineitems}
\phantomsection\label{\detokenize{docs/fmdtools.resultdisp:fmdtools.resultdisp.graph.exec_order}}\pysiglinewithargsret{\sphinxcode{\sphinxupquote{fmdtools.resultdisp.graph.}}\sphinxbfcode{\sphinxupquote{exec\_order}}}{\emph{\DUrole{n}{mdl}}, \emph{\DUrole{n}{renderer}\DUrole{o}{=}\DUrole{default_value}{'matplotlib'}}, \emph{\DUrole{n}{gtype}\DUrole{o}{=}\DUrole{default_value}{'bipartite'}}, \emph{\DUrole{n}{colors}\DUrole{o}{=}\DUrole{default_value}{{[}'lightgray', 'cyan', 'teal'{]}}}, \emph{\DUrole{n}{show\_dyn\_order}\DUrole{o}{=}\DUrole{default_value}{True}}, \emph{\DUrole{n}{title}\DUrole{o}{=}\DUrole{default_value}{'Function Execution Order'}}, \emph{\DUrole{n}{legend}\DUrole{o}{=}\DUrole{default_value}{True}}, \emph{\DUrole{o}{**}\DUrole{n}{kwargs}}}{}
\sphinxAtStartPar
Displays the execution order/types of the model, where the functions and flows in the
static step are highlighted and the functions in the dynamic step are listed (with corresponding order)
\begin{quote}\begin{description}
\item[{Parameters}] \leavevmode\begin{itemize}
\item {} 
\sphinxAtStartPar
\sphinxstyleliteralstrong{\sphinxupquote{mdl}} (\sphinxstyleliteralemphasis{\sphinxupquote{fmdtools Model}}) – Model of the system to visualize.

\item {} 
\sphinxAtStartPar
\sphinxstyleliteralstrong{\sphinxupquote{renderer}} (\sphinxstyleliteralemphasis{\sphinxupquote{'matplotlib'}}\sphinxstyleliteralemphasis{\sphinxupquote{ or }}\sphinxstyleliteralemphasis{\sphinxupquote{'graphviz'}}) – Renderer to use for the graph

\item {} 
\sphinxAtStartPar
\sphinxstyleliteralstrong{\sphinxupquote{gtype}} (\sphinxstyleliteralemphasis{\sphinxupquote{'normal'/'bipartite'}}\sphinxstyleliteralemphasis{\sphinxupquote{, }}\sphinxstyleliteralemphasis{\sphinxupquote{optional}}) – Representation of the model to use. The default is ‘bipartite’.

\item {} 
\sphinxAtStartPar
\sphinxstyleliteralstrong{\sphinxupquote{colors}} (\sphinxstyleliteralemphasis{\sphinxupquote{list}}\sphinxstyleliteralemphasis{\sphinxupquote{, }}\sphinxstyleliteralemphasis{\sphinxupquote{optional}}) – Colors to use for unexecuted functions, static propagation steps, and dynamic functions.
The default is {[}‘lightgray’, ‘cyan’,’teal’{]}.

\item {} 
\sphinxAtStartPar
\sphinxstyleliteralstrong{\sphinxupquote{show\_dyn\_order}} (\sphinxstyleliteralemphasis{\sphinxupquote{bool}}\sphinxstyleliteralemphasis{\sphinxupquote{, }}\sphinxstyleliteralemphasis{\sphinxupquote{optional}}) – Whether to label the execution order for dynamic functions. The default is True.

\item {} 
\sphinxAtStartPar
\sphinxstyleliteralstrong{\sphinxupquote{title}} (\sphinxstyleliteralemphasis{\sphinxupquote{str}}\sphinxstyleliteralemphasis{\sphinxupquote{, }}\sphinxstyleliteralemphasis{\sphinxupquote{optional}}) – Title for the plot. The default is “Function Execution Order”.

\item {} 
\sphinxAtStartPar
\sphinxstyleliteralstrong{\sphinxupquote{legend}} (\sphinxstyleliteralemphasis{\sphinxupquote{bool}}\sphinxstyleliteralemphasis{\sphinxupquote{, }}\sphinxstyleliteralemphasis{\sphinxupquote{optional}}) – Whether to show a legend. The default is True.

\item {} 
\sphinxAtStartPar
\sphinxstyleliteralstrong{\sphinxupquote{**kwargs}} (\sphinxstyleliteralemphasis{\sphinxupquote{see arguments for the respective renderers}}) – 

\end{itemize}

\item[{Returns}] \leavevmode
\sphinxAtStartPar


\item[{Return type}] \leavevmode
\sphinxAtStartPar
tuple of form (figure, axis)

\end{description}\end{quote}

\end{fulllineitems}

\index{get\_graph\_annotations() (in module fmdtools.resultdisp.graph)@\spxentry{get\_graph\_annotations()}\spxextra{in module fmdtools.resultdisp.graph}}

\begin{fulllineitems}
\phantomsection\label{\detokenize{docs/fmdtools.resultdisp:fmdtools.resultdisp.graph.get_graph_annotations}}\pysiglinewithargsret{\sphinxcode{\sphinxupquote{fmdtools.resultdisp.graph.}}\sphinxbfcode{\sphinxupquote{get\_graph\_annotations}}}{\emph{\DUrole{n}{g}}, \emph{\DUrole{n}{gtype}\DUrole{o}{=}\DUrole{default_value}{'bipartite'}}}{}
\sphinxAtStartPar
Helper method that returns labels/lists degraded nodes for the plot annotations

\end{fulllineitems}

\index{get\_graph\_pos() (in module fmdtools.resultdisp.graph)@\spxentry{get\_graph\_pos()}\spxextra{in module fmdtools.resultdisp.graph}}

\begin{fulllineitems}
\phantomsection\label{\detokenize{docs/fmdtools.resultdisp:fmdtools.resultdisp.graph.get_graph_pos}}\pysiglinewithargsret{\sphinxcode{\sphinxupquote{fmdtools.resultdisp.graph.}}\sphinxbfcode{\sphinxupquote{get\_graph\_pos}}}{\emph{\DUrole{n}{mdl}}, \emph{\DUrole{n}{pos}}, \emph{\DUrole{n}{gtype}}}{}
\sphinxAtStartPar
Helper function for getting the right graph/positions from a model

\end{fulllineitems}

\index{get\_plotlabels() (in module fmdtools.resultdisp.graph)@\spxentry{get\_plotlabels()}\spxextra{in module fmdtools.resultdisp.graph}}

\begin{fulllineitems}
\phantomsection\label{\detokenize{docs/fmdtools.resultdisp:fmdtools.resultdisp.graph.get_plotlabels}}\pysiglinewithargsret{\sphinxcode{\sphinxupquote{fmdtools.resultdisp.graph.}}\sphinxbfcode{\sphinxupquote{get\_plotlabels}}}{\emph{\DUrole{n}{g}}, \emph{\DUrole{n}{reshist}}, \emph{\DUrole{n}{t\_ind}}}{}
\sphinxAtStartPar
Assigns labels to a graph g from reshist at time t so that it can be plotted
\begin{quote}\begin{description}
\item[{Parameters}] \leavevmode\begin{itemize}
\item {} 
\sphinxAtStartPar
\sphinxstyleliteralstrong{\sphinxupquote{g}} (\sphinxstyleliteralemphasis{\sphinxupquote{networkx graph}}) – The graph to get labels for

\item {} 
\sphinxAtStartPar
\sphinxstyleliteralstrong{\sphinxupquote{reshist}} (\sphinxstyleliteralemphasis{\sphinxupquote{dict}}) – The dict of results history over time (from process.hists() or process.typehist() for the typegraph option)

\item {} 
\sphinxAtStartPar
\sphinxstyleliteralstrong{\sphinxupquote{t\_ind}} (\sphinxstyleliteralemphasis{\sphinxupquote{float}}) – The time in reshist to update the graph at

\end{itemize}

\item[{Returns}] \leavevmode
\sphinxAtStartPar
\begin{itemize}
\item {} 
\sphinxAtStartPar
\sphinxstylestrong{labels} (\sphinxstyleemphasis{dict}) – labels for the graph.

\item {} 
\sphinxAtStartPar
\sphinxstylestrong{faultfxns} (\sphinxstyleemphasis{dict}) – functions with faults in them

\item {} 
\sphinxAtStartPar
\sphinxstylestrong{degfxns} (\sphinxstyleemphasis{dict}) – functions that are degraded

\item {} 
\sphinxAtStartPar
\sphinxstylestrong{degflows} (\sphinxstyleemphasis{dict}) – flows that are degraded

\item {} 
\sphinxAtStartPar
\sphinxstylestrong{faultlabels} (\sphinxstyleemphasis{dict}) – names of each fault

\item {} 
\sphinxAtStartPar
\sphinxstylestrong{faultedges} (\sphinxstyleemphasis{dict}) – edges with faults in them

\item {} 
\sphinxAtStartPar
\sphinxstylestrong{faultedgeflows} (\sphinxstyleemphasis{dict}) – names of flows that are degraded on each edge

\item {} 
\sphinxAtStartPar
\sphinxstylestrong{edgelabels} (\sphinxstyleemphasis{dict}) – labels of each edge

\end{itemize}


\end{description}\end{quote}

\end{fulllineitems}

\index{gv\_colors() (in module fmdtools.resultdisp.graph)@\spxentry{gv\_colors()}\spxextra{in module fmdtools.resultdisp.graph}}

\begin{fulllineitems}
\phantomsection\label{\detokenize{docs/fmdtools.resultdisp:fmdtools.resultdisp.graph.gv_colors}}\pysiglinewithargsret{\sphinxcode{\sphinxupquote{fmdtools.resultdisp.graph.}}\sphinxbfcode{\sphinxupquote{gv\_colors}}}{\emph{\DUrole{n}{g}}, \emph{\DUrole{n}{gtype}}, \emph{\DUrole{n}{colors}}, \emph{\DUrole{n}{heatmap}}, \emph{\DUrole{n}{cmap}}, \emph{\DUrole{n}{faultnodes}}, \emph{\DUrole{n}{degradednodes}}, \emph{\DUrole{n}{faultedges}\DUrole{o}{=}\DUrole{default_value}{{[}{]}}}, \emph{\DUrole{n}{edgeflows}\DUrole{o}{=}\DUrole{default_value}{\{\}}}, \emph{\DUrole{n}{functions}\DUrole{o}{=}\DUrole{default_value}{{[}{]}}}, \emph{\DUrole{n}{flows}\DUrole{o}{=}\DUrole{default_value}{{[}{]}}}, \emph{\DUrole{n}{highlight}\DUrole{o}{=}\DUrole{default_value}{{[}{]}}}}{}
\sphinxAtStartPar
creates dictonary of node/edge colors for a graphviz plot
\begin{quote}\begin{description}
\item[{Parameters}] \leavevmode\begin{itemize}
\item {} 
\sphinxAtStartPar
\sphinxstyleliteralstrong{\sphinxupquote{g}} (\sphinxstyleliteralemphasis{\sphinxupquote{nx graph object}}\sphinxstyleliteralemphasis{\sphinxupquote{ or }}\sphinxstyleliteralemphasis{\sphinxupquote{model}}) – The multigraph to plot

\item {} 
\sphinxAtStartPar
\sphinxstyleliteralstrong{\sphinxupquote{gtype}} (\sphinxstyleliteralemphasis{\sphinxupquote{string}}\sphinxstyleliteralemphasis{\sphinxupquote{, }}\sphinxstyleliteralemphasis{\sphinxupquote{optional}}) – Type of graph input to show
values are ‘normal’, ‘bipartite’, or ‘typegraph’.

\item {} 
\sphinxAtStartPar
\sphinxstyleliteralstrong{\sphinxupquote{colors}} (\sphinxstyleliteralemphasis{\sphinxupquote{list}}\sphinxstyleliteralemphasis{\sphinxupquote{, }}\sphinxstyleliteralemphasis{\sphinxupquote{optional}}) – List of colors to use for nominal, degraded, and faulty functions/flows.
Default is: {[}‘lightgray’,’orange’, ‘red’{]}

\item {} 
\sphinxAtStartPar
\sphinxstyleliteralstrong{\sphinxupquote{heatmap}} (\sphinxstyleliteralemphasis{\sphinxupquote{dict}}\sphinxstyleliteralemphasis{\sphinxupquote{, }}\sphinxstyleliteralemphasis{\sphinxupquote{optional}}) – A heatmap dictionary to overlay on the plot. The default is \{\}.

\item {} 
\sphinxAtStartPar
\sphinxstyleliteralstrong{\sphinxupquote{cmap}} (\sphinxstyleliteralemphasis{\sphinxupquote{mpl colormap}}) – Colormap to use for heatmap visualization

\item {} 
\sphinxAtStartPar
\sphinxstyleliteralstrong{\sphinxupquote{faultnodes}} (\sphinxstyleliteralemphasis{\sphinxupquote{list}}) – list of the nodes with faults

\item {} 
\sphinxAtStartPar
\sphinxstyleliteralstrong{\sphinxupquote{degradednodes}} (\sphinxstyleliteralemphasis{\sphinxupquote{list}}) – list of the nodes with degraded functionality

\item {} 
\sphinxAtStartPar
\sphinxstyleliteralstrong{\sphinxupquote{faultedges}} (\sphinxstyleliteralemphasis{\sphinxupquote{list}}) – list of edges(flows) that have faults. Only used for ‘normal’ graph. The default is {[}{]}.

\item {} 
\sphinxAtStartPar
\sphinxstyleliteralstrong{\sphinxupquote{edgeflows}} (\sphinxstyleliteralemphasis{\sphinxupquote{dictionary}}) – dictionary of edges (n1,n2) and edge/flow names. The default is \{\}.

\item {} 
\sphinxAtStartPar
\sphinxstyleliteralstrong{\sphinxupquote{functions}} (\sphinxstyleliteralemphasis{\sphinxupquote{list}}\sphinxstyleliteralemphasis{\sphinxupquote{, }}\sphinxstyleliteralemphasis{\sphinxupquote{optional}}) – list of function nodes. Only used for ‘bipartite’ graph. The default is {[}{]}.

\item {} 
\sphinxAtStartPar
\sphinxstyleliteralstrong{\sphinxupquote{flows}} (\sphinxstyleliteralemphasis{\sphinxupquote{list}}\sphinxstyleliteralemphasis{\sphinxupquote{, }}\sphinxstyleliteralemphasis{\sphinxupquote{optional}}) – list of flow nodes. Only used for ‘bipartite’ graph. The default is {[}{]}.

\end{itemize}

\item[{Returns}] \leavevmode
\sphinxAtStartPar
\sphinxstylestrong{colors\_dict} – dictionary withe keys as nodes/edges and values colors.

\item[{Return type}] \leavevmode
\sphinxAtStartPar
dictionary

\end{description}\end{quote}

\end{fulllineitems}

\index{gv\_execute\_order\_legend() (in module fmdtools.resultdisp.graph)@\spxentry{gv\_execute\_order\_legend()}\spxextra{in module fmdtools.resultdisp.graph}}

\begin{fulllineitems}
\phantomsection\label{\detokenize{docs/fmdtools.resultdisp:fmdtools.resultdisp.graph.gv_execute_order_legend}}\pysiglinewithargsret{\sphinxcode{\sphinxupquote{fmdtools.resultdisp.graph.}}\sphinxbfcode{\sphinxupquote{gv\_execute\_order\_legend}}}{\emph{\DUrole{n}{colors}}}{}
\sphinxAtStartPar
Provides legend for model execution order in the graphviz toolkit

\end{fulllineitems}

\index{gv\_import\_check() (in module fmdtools.resultdisp.graph)@\spxentry{gv\_import\_check()}\spxextra{in module fmdtools.resultdisp.graph}}

\begin{fulllineitems}
\phantomsection\label{\detokenize{docs/fmdtools.resultdisp:fmdtools.resultdisp.graph.gv_import_check}}\pysiglinewithargsret{\sphinxcode{\sphinxupquote{fmdtools.resultdisp.graph.}}\sphinxbfcode{\sphinxupquote{gv\_import\_check}}}{}{}
\sphinxAtStartPar
Checks if graphviz is installed on the system before plotting.

\end{fulllineitems}

\index{history() (in module fmdtools.resultdisp.graph)@\spxentry{history()}\spxextra{in module fmdtools.resultdisp.graph}}

\begin{fulllineitems}
\phantomsection\label{\detokenize{docs/fmdtools.resultdisp:fmdtools.resultdisp.graph.history}}\pysiglinewithargsret{\sphinxcode{\sphinxupquote{fmdtools.resultdisp.graph.}}\sphinxbfcode{\sphinxupquote{history}}}{\emph{\DUrole{n}{ghist}}, \emph{\DUrole{o}{**}\DUrole{n}{kwargs}}}{}
\sphinxAtStartPar
Displays plots of the graph over time given a dict history of graph objects
\begin{quote}\begin{description}
\item[{Parameters}] \leavevmode\begin{itemize}
\item {} 
\sphinxAtStartPar
\sphinxstyleliteralstrong{\sphinxupquote{ghist}} (\sphinxstyleliteralemphasis{\sphinxupquote{dict}}) – 
\sphinxAtStartPar
A dictionary of the history of the graph over time with structure:
\{time: graphobject\}, where
\begin{itemize}
\item {} 
\sphinxAtStartPar
time is the time where the snapshot of the graph was recorded

\item {} 
\sphinxAtStartPar
graphobject is the snapshot of the graph at that time

\end{itemize}


\item {} 
\sphinxAtStartPar
\sphinxstyleliteralstrong{\sphinxupquote{**kwargs}} (\sphinxstyleliteralemphasis{\sphinxupquote{kwargs}}) – keyword arguments for graph.show()

\end{itemize}

\item[{Returns}] \leavevmode
\sphinxAtStartPar
\sphinxstylestrong{figobjs} – Set of graph objects from graph.show() for the given renderer

\item[{Return type}] \leavevmode
\sphinxAtStartPar
dict

\end{description}\end{quote}

\end{fulllineitems}

\index{plot\_bip\_netgraph() (in module fmdtools.resultdisp.graph)@\spxentry{plot\_bip\_netgraph()}\spxextra{in module fmdtools.resultdisp.graph}}

\begin{fulllineitems}
\phantomsection\label{\detokenize{docs/fmdtools.resultdisp:fmdtools.resultdisp.graph.plot_bip_netgraph}}\pysiglinewithargsret{\sphinxcode{\sphinxupquote{fmdtools.resultdisp.graph.}}\sphinxbfcode{\sphinxupquote{plot\_bip\_netgraph}}}{\emph{\DUrole{n}{g}}, \emph{\DUrole{n}{labels}}, \emph{\DUrole{n}{faultfxns}}, \emph{\DUrole{n}{degnodes}}, \emph{\DUrole{n}{faultlabels}}, \emph{\DUrole{n}{faultscen}\DUrole{o}{=}\DUrole{default_value}{{[}{]}}}, \emph{\DUrole{n}{time}\DUrole{o}{=}\DUrole{default_value}{0}}, \emph{\DUrole{n}{showfaultlabels}\DUrole{o}{=}\DUrole{default_value}{True}}, \emph{\DUrole{n}{scale}\DUrole{o}{=}\DUrole{default_value}{1}}, \emph{\DUrole{n}{pos}\DUrole{o}{=}\DUrole{default_value}{{[}{]}}}, \emph{\DUrole{n}{show}\DUrole{o}{=}\DUrole{default_value}{True}}, \emph{\DUrole{n}{colors}\DUrole{o}{=}\DUrole{default_value}{{[}'lightgray', 'orange', 'red'{]}}}, \emph{\DUrole{n}{title}\DUrole{o}{=}\DUrole{default_value}{{[}{]}}}, \emph{\DUrole{n}{functions}\DUrole{o}{=}\DUrole{default_value}{{[}{]}}}, \emph{\DUrole{n}{flows}\DUrole{o}{=}\DUrole{default_value}{{[}{]}}}, \emph{\DUrole{o}{**}\DUrole{n}{kwargs}}}{}
\sphinxAtStartPar
Experimental method for plotting with netgraph instead of networkx

\end{fulllineitems}

\index{plot\_bipgraph() (in module fmdtools.resultdisp.graph)@\spxentry{plot\_bipgraph()}\spxextra{in module fmdtools.resultdisp.graph}}

\begin{fulllineitems}
\phantomsection\label{\detokenize{docs/fmdtools.resultdisp:fmdtools.resultdisp.graph.plot_bipgraph}}\pysiglinewithargsret{\sphinxcode{\sphinxupquote{fmdtools.resultdisp.graph.}}\sphinxbfcode{\sphinxupquote{plot\_bipgraph}}}{\emph{\DUrole{n}{g}}, \emph{\DUrole{n}{labels}}, \emph{\DUrole{n}{faultfxns}}, \emph{\DUrole{n}{degnodes}}, \emph{\DUrole{n}{faultlabels}}, \emph{\DUrole{n}{faultscen}\DUrole{o}{=}\DUrole{default_value}{{[}{]}}}, \emph{\DUrole{n}{time}\DUrole{o}{=}\DUrole{default_value}{0}}, \emph{\DUrole{n}{showfaultlabels}\DUrole{o}{=}\DUrole{default_value}{True}}, \emph{\DUrole{n}{scale}\DUrole{o}{=}\DUrole{default_value}{1}}, \emph{\DUrole{n}{pos}\DUrole{o}{=}\DUrole{default_value}{{[}{]}}}, \emph{\DUrole{n}{show}\DUrole{o}{=}\DUrole{default_value}{True}}, \emph{\DUrole{n}{colors}\DUrole{o}{=}\DUrole{default_value}{{[}'lightgray', 'orange', 'red'{]}}}, \emph{\DUrole{n}{title}\DUrole{o}{=}\DUrole{default_value}{{[}{]}}}, \emph{\DUrole{n}{functions}\DUrole{o}{=}\DUrole{default_value}{{[}{]}}}, \emph{\DUrole{n}{flows}\DUrole{o}{=}\DUrole{default_value}{{[}{]}}}}{}
\sphinxAtStartPar
Plots a bipartite graph. Used in other functions

\end{fulllineitems}

\index{plot\_gv\_bipartite() (in module fmdtools.resultdisp.graph)@\spxentry{plot\_gv\_bipartite()}\spxextra{in module fmdtools.resultdisp.graph}}

\begin{fulllineitems}
\phantomsection\label{\detokenize{docs/fmdtools.resultdisp:fmdtools.resultdisp.graph.plot_gv_bipartite}}\pysiglinewithargsret{\sphinxcode{\sphinxupquote{fmdtools.resultdisp.graph.}}\sphinxbfcode{\sphinxupquote{plot\_gv\_bipartite}}}{\emph{\DUrole{n}{g}}, \emph{\DUrole{n}{faultnodes}}, \emph{\DUrole{n}{degradednodes}}, \emph{\DUrole{n}{faultlabels}}, \emph{\DUrole{n}{faultscen}}, \emph{\DUrole{n}{time}}, \emph{\DUrole{n}{showfaultlabels}}, \emph{\DUrole{n}{colors\_dict}}, \emph{\DUrole{n}{functions}}, \emph{\DUrole{n}{flows}}, \emph{\DUrole{n}{edges}}, \emph{\DUrole{n}{dot}}}{}
\sphinxAtStartPar
Plots a bipartite graph representation using the graphviz toolkit. Used in other functions

\end{fulllineitems}

\index{plot\_gv\_normgraph() (in module fmdtools.resultdisp.graph)@\spxentry{plot\_gv\_normgraph()}\spxextra{in module fmdtools.resultdisp.graph}}

\begin{fulllineitems}
\phantomsection\label{\detokenize{docs/fmdtools.resultdisp:fmdtools.resultdisp.graph.plot_gv_normgraph}}\pysiglinewithargsret{\sphinxcode{\sphinxupquote{fmdtools.resultdisp.graph.}}\sphinxbfcode{\sphinxupquote{plot\_gv\_normgraph}}}{\emph{\DUrole{n}{g}}, \emph{\DUrole{n}{edgeflows}}, \emph{\DUrole{n}{faultnodes}}, \emph{\DUrole{n}{degradednodes}}, \emph{\DUrole{n}{faultflows}}, \emph{\DUrole{n}{faultlabels}}, \emph{\DUrole{n}{faultedges}}, \emph{\DUrole{n}{faultscen}}, \emph{\DUrole{n}{time}}, \emph{\DUrole{n}{showfaultlabels}}, \emph{\DUrole{n}{colors\_dict}}, \emph{\DUrole{n}{dot}}}{}
\sphinxAtStartPar
Plots a normal graph representation using the graphviz toolkit. Used in other functions

\end{fulllineitems}

\index{plot\_norm\_netgraph() (in module fmdtools.resultdisp.graph)@\spxentry{plot\_norm\_netgraph()}\spxextra{in module fmdtools.resultdisp.graph}}

\begin{fulllineitems}
\phantomsection\label{\detokenize{docs/fmdtools.resultdisp:fmdtools.resultdisp.graph.plot_norm_netgraph}}\pysiglinewithargsret{\sphinxcode{\sphinxupquote{fmdtools.resultdisp.graph.}}\sphinxbfcode{\sphinxupquote{plot\_norm\_netgraph}}}{\emph{\DUrole{n}{g}}, \emph{\DUrole{n}{labels}}, \emph{\DUrole{n}{faultfxns}}, \emph{\DUrole{n}{degfxns}}, \emph{\DUrole{n}{degflows}}, \emph{\DUrole{n}{faultlabels}}, \emph{\DUrole{n}{faultedges}}, \emph{\DUrole{n}{faultedgeflows}}, \emph{\DUrole{n}{faultscen}}, \emph{\DUrole{n}{time}}, \emph{\DUrole{n}{showfaultlabels}}, \emph{\DUrole{n}{edgeflows}}, \emph{\DUrole{n}{scale}\DUrole{o}{=}\DUrole{default_value}{1}}, \emph{\DUrole{n}{pos}\DUrole{o}{=}\DUrole{default_value}{{[}{]}}}, \emph{\DUrole{n}{show}\DUrole{o}{=}\DUrole{default_value}{True}}, \emph{\DUrole{n}{colors}\DUrole{o}{=}\DUrole{default_value}{{[}'lightgray', 'orange', 'red'{]}}}, \emph{\DUrole{n}{title}\DUrole{o}{=}\DUrole{default_value}{{[}{]}}}, \emph{\DUrole{n}{show\_edgelabels}\DUrole{o}{=}\DUrole{default_value}{True}}, \emph{\DUrole{o}{**}\DUrole{n}{kwargs}}}{}
\sphinxAtStartPar
Experimental method for plotting with netgraph instead of networkx

\end{fulllineitems}

\index{plot\_normgraph() (in module fmdtools.resultdisp.graph)@\spxentry{plot\_normgraph()}\spxextra{in module fmdtools.resultdisp.graph}}

\begin{fulllineitems}
\phantomsection\label{\detokenize{docs/fmdtools.resultdisp:fmdtools.resultdisp.graph.plot_normgraph}}\pysiglinewithargsret{\sphinxcode{\sphinxupquote{fmdtools.resultdisp.graph.}}\sphinxbfcode{\sphinxupquote{plot\_normgraph}}}{\emph{\DUrole{n}{g}}, \emph{\DUrole{n}{labels}}, \emph{\DUrole{n}{faultfxns}}, \emph{\DUrole{n}{degfxns}}, \emph{\DUrole{n}{degflows}}, \emph{\DUrole{n}{faultlabels}}, \emph{\DUrole{n}{faultedges}}, \emph{\DUrole{n}{faultedgeflows}}, \emph{\DUrole{n}{faultscen}}, \emph{\DUrole{n}{time}}, \emph{\DUrole{n}{showfaultlabels}}, \emph{\DUrole{n}{edgeflows}}, \emph{\DUrole{n}{scale}\DUrole{o}{=}\DUrole{default_value}{1}}, \emph{\DUrole{n}{pos}\DUrole{o}{=}\DUrole{default_value}{{[}{]}}}, \emph{\DUrole{n}{show}\DUrole{o}{=}\DUrole{default_value}{True}}, \emph{\DUrole{n}{colors}\DUrole{o}{=}\DUrole{default_value}{{[}'lightgray', 'orange', 'red'{]}}}, \emph{\DUrole{n}{title}\DUrole{o}{=}\DUrole{default_value}{{[}{]}}}, \emph{\DUrole{n}{show\_edgelabels}\DUrole{o}{=}\DUrole{default_value}{True}}}{}
\sphinxAtStartPar
Plots a standard graph. Used in other functions

\end{fulllineitems}

\index{result\_from() (in module fmdtools.resultdisp.graph)@\spxentry{result\_from()}\spxextra{in module fmdtools.resultdisp.graph}}

\begin{fulllineitems}
\phantomsection\label{\detokenize{docs/fmdtools.resultdisp:fmdtools.resultdisp.graph.result_from}}\pysiglinewithargsret{\sphinxcode{\sphinxupquote{fmdtools.resultdisp.graph.}}\sphinxbfcode{\sphinxupquote{result\_from}}}{\emph{\DUrole{n}{mdl}}, \emph{\DUrole{n}{reshist}}, \emph{\DUrole{n}{time}}, \emph{\DUrole{n}{renderer}\DUrole{o}{=}\DUrole{default_value}{'matplotlib'}}, \emph{\DUrole{n}{gtype}\DUrole{o}{=}\DUrole{default_value}{'bipartite'}}, \emph{\DUrole{o}{**}\DUrole{n}{kwargs}}}{}
\sphinxAtStartPar
Plots a representation of the model graph at a specific time in the results history.
\begin{quote}\begin{description}
\item[{Parameters}] \leavevmode\begin{itemize}
\item {} 
\sphinxAtStartPar
\sphinxstyleliteralstrong{\sphinxupquote{mdl}} (\sphinxstyleliteralemphasis{\sphinxupquote{model}}) – The model the faults were run in.

\item {} 
\sphinxAtStartPar
\sphinxstyleliteralstrong{\sphinxupquote{reshist}} (\sphinxstyleliteralemphasis{\sphinxupquote{dict}}) – A dictionary of results (from process.hists() or process.typehist() for the typegraph option)

\item {} 
\sphinxAtStartPar
\sphinxstyleliteralstrong{\sphinxupquote{time}} (\sphinxstyleliteralemphasis{\sphinxupquote{float}}) – The time in the history to plot the graph at.

\item {} 
\sphinxAtStartPar
\sphinxstyleliteralstrong{\sphinxupquote{renderer}} (\sphinxstyleliteralemphasis{\sphinxupquote{'matplotlib'}}\sphinxstyleliteralemphasis{\sphinxupquote{ or }}\sphinxstyleliteralemphasis{\sphinxupquote{'graphviz'}}\sphinxstyleliteralemphasis{\sphinxupquote{ or }}\sphinxstyleliteralemphasis{\sphinxupquote{'netgraph'}}) – Renderer to use to plot the graph. Default is ‘matplotlib’

\item {} 
\sphinxAtStartPar
\sphinxstyleliteralstrong{\sphinxupquote{gtype}} (\sphinxstyleliteralemphasis{\sphinxupquote{str}}\sphinxstyleliteralemphasis{\sphinxupquote{, }}\sphinxstyleliteralemphasis{\sphinxupquote{optional}}) – The type of graph to plot (normal or bipartite). The default is ‘bipartite’.

\item {} 
\sphinxAtStartPar
\sphinxstyleliteralstrong{\sphinxupquote{OPTIONS}} (\sphinxstyleliteralemphasis{\sphinxupquote{MATPLOTLIB}}) – 

\item {} 
\sphinxAtStartPar
\sphinxstyleliteralstrong{\sphinxupquote{\sphinxhyphen{}\sphinxhyphen{}\sphinxhyphen{}\sphinxhyphen{}\sphinxhyphen{}\sphinxhyphen{}\sphinxhyphen{}\sphinxhyphen{}\sphinxhyphen{}\sphinxhyphen{}}} – 

\item {} 
\sphinxAtStartPar
\sphinxstyleliteralstrong{\sphinxupquote{faultscen}} (\sphinxstyleliteralemphasis{\sphinxupquote{str}}\sphinxstyleliteralemphasis{\sphinxupquote{, }}\sphinxstyleliteralemphasis{\sphinxupquote{optional}}) – Name of the fault scenario. The default is {[}{]}.

\item {} 
\sphinxAtStartPar
\sphinxstyleliteralstrong{\sphinxupquote{showfaultlabels}} (\sphinxstyleliteralemphasis{\sphinxupquote{bool}}\sphinxstyleliteralemphasis{\sphinxupquote{, }}\sphinxstyleliteralemphasis{\sphinxupquote{optional}}) – Whether or not to list faults on the plot. The default is True.

\item {} 
\sphinxAtStartPar
\sphinxstyleliteralstrong{\sphinxupquote{scale}} (\sphinxstyleliteralemphasis{\sphinxupquote{float}}\sphinxstyleliteralemphasis{\sphinxupquote{, }}\sphinxstyleliteralemphasis{\sphinxupquote{optional}}) – Scale factor for the node/label sizes. The default is 1.

\item {} 
\sphinxAtStartPar
\sphinxstyleliteralstrong{\sphinxupquote{pos}} (\sphinxstyleliteralemphasis{\sphinxupquote{dict}}\sphinxstyleliteralemphasis{\sphinxupquote{, }}\sphinxstyleliteralemphasis{\sphinxupquote{optional}}) – dict of node positions (if re\sphinxhyphen{}using positions). The default is {[}{]}.

\end{itemize}

\end{description}\end{quote}

\end{fulllineitems}

\index{results\_from() (in module fmdtools.resultdisp.graph)@\spxentry{results\_from()}\spxextra{in module fmdtools.resultdisp.graph}}

\begin{fulllineitems}
\phantomsection\label{\detokenize{docs/fmdtools.resultdisp:fmdtools.resultdisp.graph.results_from}}\pysiglinewithargsret{\sphinxcode{\sphinxupquote{fmdtools.resultdisp.graph.}}\sphinxbfcode{\sphinxupquote{results\_from}}}{\emph{\DUrole{n}{mdl}}, \emph{\DUrole{n}{reshist}}, \emph{\DUrole{n}{times}}, \emph{\DUrole{n}{renderer}\DUrole{o}{=}\DUrole{default_value}{'matplotlib'}}, \emph{\DUrole{n}{gtype}\DUrole{o}{=}\DUrole{default_value}{'bipartite'}}, \emph{\DUrole{o}{**}\DUrole{n}{kwargs}}}{}
\sphinxAtStartPar
Plots a set of representations of the model graph at given times in the results history.
\begin{quote}\begin{description}
\item[{Parameters}] \leavevmode\begin{itemize}
\item {} 
\sphinxAtStartPar
\sphinxstyleliteralstrong{\sphinxupquote{mdl}} (\sphinxstyleliteralemphasis{\sphinxupquote{model}}) – The model the faults were run in.

\item {} 
\sphinxAtStartPar
\sphinxstyleliteralstrong{\sphinxupquote{reshist}} (\sphinxstyleliteralemphasis{\sphinxupquote{dict}}) – A dictionary of results (from process.hists() or process.typehist() for the typegraph option)

\item {} 
\sphinxAtStartPar
\sphinxstyleliteralstrong{\sphinxupquote{times}} (\sphinxstyleliteralemphasis{\sphinxupquote{list}}\sphinxstyleliteralemphasis{\sphinxupquote{ or }}\sphinxstyleliteralemphasis{\sphinxupquote{'all'}}) – The times in the history to plot the graph at. If ‘all’, plots them all

\item {} 
\sphinxAtStartPar
\sphinxstyleliteralstrong{\sphinxupquote{renderer}} (\sphinxstyleliteralemphasis{\sphinxupquote{'matplotlib'}}\sphinxstyleliteralemphasis{\sphinxupquote{ or }}\sphinxstyleliteralemphasis{\sphinxupquote{'graphviz'}}\sphinxstyleliteralemphasis{\sphinxupquote{ or }}\sphinxstyleliteralemphasis{\sphinxupquote{'netgraph'}}) – Renderer to use to plot the graph. Default is ‘matplotlib’

\item {} 
\sphinxAtStartPar
\sphinxstyleliteralstrong{\sphinxupquote{gtype}} (\sphinxstyleliteralemphasis{\sphinxupquote{str}}\sphinxstyleliteralemphasis{\sphinxupquote{, }}\sphinxstyleliteralemphasis{\sphinxupquote{optional}}) – The type of graph to plot (normal or bipartite). The default is ‘bipartite’.

\item {} 
\sphinxAtStartPar
\sphinxstyleliteralstrong{\sphinxupquote{OPTIONS}} (\sphinxstyleliteralemphasis{\sphinxupquote{MATPLOTLIB}}) – 

\item {} 
\sphinxAtStartPar
\sphinxstyleliteralstrong{\sphinxupquote{\sphinxhyphen{}\sphinxhyphen{}\sphinxhyphen{}\sphinxhyphen{}\sphinxhyphen{}\sphinxhyphen{}\sphinxhyphen{}\sphinxhyphen{}\sphinxhyphen{}\sphinxhyphen{}}} – 

\item {} 
\sphinxAtStartPar
\sphinxstyleliteralstrong{\sphinxupquote{faultscen}} (\sphinxstyleliteralemphasis{\sphinxupquote{str}}\sphinxstyleliteralemphasis{\sphinxupquote{, }}\sphinxstyleliteralemphasis{\sphinxupquote{optional}}) – Name of the fault scenario. The default is {[}{]}.

\item {} 
\sphinxAtStartPar
\sphinxstyleliteralstrong{\sphinxupquote{showfaultlabels}} (\sphinxstyleliteralemphasis{\sphinxupquote{bool}}\sphinxstyleliteralemphasis{\sphinxupquote{, }}\sphinxstyleliteralemphasis{\sphinxupquote{optional}}) – Whether or not to list faults on the plot. The default is True.

\item {} 
\sphinxAtStartPar
\sphinxstyleliteralstrong{\sphinxupquote{scale}} (\sphinxstyleliteralemphasis{\sphinxupquote{float}}\sphinxstyleliteralemphasis{\sphinxupquote{, }}\sphinxstyleliteralemphasis{\sphinxupquote{optional}}) – Scale factor for the node/label sizes. The default is 1.

\item {} 
\sphinxAtStartPar
\sphinxstyleliteralstrong{\sphinxupquote{pos}} (\sphinxstyleliteralemphasis{\sphinxupquote{dict}}\sphinxstyleliteralemphasis{\sphinxupquote{, }}\sphinxstyleliteralemphasis{\sphinxupquote{optional}}) – dict of node positions (if re\sphinxhyphen{}using positions). The default is {[}{]}.

\end{itemize}

\item[{Returns}] \leavevmode
\sphinxAtStartPar
\sphinxstylestrong{frames} – Dictionary of mpl figures keyed at each time \{time:fig\}

\item[{Return type}] \leavevmode
\sphinxAtStartPar
Dict

\end{description}\end{quote}

\end{fulllineitems}

\index{set\_pos() (in module fmdtools.resultdisp.graph)@\spxentry{set\_pos()}\spxextra{in module fmdtools.resultdisp.graph}}

\begin{fulllineitems}
\phantomsection\label{\detokenize{docs/fmdtools.resultdisp:fmdtools.resultdisp.graph.set_pos}}\pysiglinewithargsret{\sphinxcode{\sphinxupquote{fmdtools.resultdisp.graph.}}\sphinxbfcode{\sphinxupquote{set\_pos}}}{\emph{\DUrole{n}{g}}, \emph{\DUrole{n}{gtype}\DUrole{o}{=}\DUrole{default_value}{'bipartite'}}, \emph{\DUrole{n}{scale}\DUrole{o}{=}\DUrole{default_value}{1}}, \emph{\DUrole{n}{node\_color}\DUrole{o}{=}\DUrole{default_value}{'lightgray'}}, \emph{\DUrole{n}{label\_size}\DUrole{o}{=}\DUrole{default_value}{7}}, \emph{\DUrole{n}{initpos}\DUrole{o}{=}\DUrole{default_value}{\{\}}}, \emph{\DUrole{n}{figsize}\DUrole{o}{=}\DUrole{default_value}{(6, 4)}}}{}
\sphinxAtStartPar
Provides graphical interface to set graph node positions. If model is provided, it will also set the positions in the model object.

\sphinxAtStartPar
To work, this method must be opened in an external window, so change the IPython before use usings \%matplotlib qt’ (or ‘\%matplotlib osx’)
\begin{quote}\begin{description}
\item[{Parameters}] \leavevmode\begin{itemize}
\item {} 
\sphinxAtStartPar
\sphinxstyleliteralstrong{\sphinxupquote{g}} (\sphinxstyleliteralemphasis{\sphinxupquote{networkx graph}}\sphinxstyleliteralemphasis{\sphinxupquote{ or }}\sphinxstyleliteralemphasis{\sphinxupquote{model}}) – normal or bipartite graph of the model of interest

\item {} 
\sphinxAtStartPar
\sphinxstyleliteralstrong{\sphinxupquote{gtype}} (\sphinxstyleliteralemphasis{\sphinxupquote{'normal'}}\sphinxstyleliteralemphasis{\sphinxupquote{ or }}\sphinxstyleliteralemphasis{\sphinxupquote{'bipartite'}}\sphinxstyleliteralemphasis{\sphinxupquote{, }}\sphinxstyleliteralemphasis{\sphinxupquote{optional}}) – Type of graph to plot. The default is ‘normal’.

\item {} 
\sphinxAtStartPar
\sphinxstyleliteralstrong{\sphinxupquote{scale}} (\sphinxstyleliteralemphasis{\sphinxupquote{float}}\sphinxstyleliteralemphasis{\sphinxupquote{, }}\sphinxstyleliteralemphasis{\sphinxupquote{optional}}) – scale for the node sizes. The default is 1.

\item {} 
\sphinxAtStartPar
\sphinxstyleliteralstrong{\sphinxupquote{node\_color}} (\sphinxstyleliteralemphasis{\sphinxupquote{str}}\sphinxstyleliteralemphasis{\sphinxupquote{, }}\sphinxstyleliteralemphasis{\sphinxupquote{optional}}) – color to use for the nodes. The default is ‘lightgray’.

\item {} 
\sphinxAtStartPar
\sphinxstyleliteralstrong{\sphinxupquote{label\_size}} (\sphinxstyleliteralemphasis{\sphinxupquote{float}}\sphinxstyleliteralemphasis{\sphinxupquote{, }}\sphinxstyleliteralemphasis{\sphinxupquote{optional}}) – size to use for the labels. The default is 8.

\item {} 
\sphinxAtStartPar
\sphinxstyleliteralstrong{\sphinxupquote{initpos}} (\sphinxstyleliteralemphasis{\sphinxupquote{dict}}\sphinxstyleliteralemphasis{\sphinxupquote{, }}\sphinxstyleliteralemphasis{\sphinxupquote{optional}}) – dict of initial positions for the labels (e.g. from nx.spring\_layout). The default is \{\}.

\item {} 
\sphinxAtStartPar
\sphinxstyleliteralstrong{\sphinxupquote{figsize}} (\sphinxstyleliteralemphasis{\sphinxupquote{tuple}}\sphinxstyleliteralemphasis{\sphinxupquote{, }}\sphinxstyleliteralemphasis{\sphinxupquote{optional}}) – size of matplotlib frame. Default is (6,4)

\end{itemize}

\item[{Returns}] \leavevmode
\sphinxAtStartPar
\sphinxstylestrong{pos} – dict of node positions for use in graph plotting functions

\item[{Return type}] \leavevmode
\sphinxAtStartPar
dict

\end{description}\end{quote}

\end{fulllineitems}

\index{show() (in module fmdtools.resultdisp.graph)@\spxentry{show()}\spxextra{in module fmdtools.resultdisp.graph}}

\begin{fulllineitems}
\phantomsection\label{\detokenize{docs/fmdtools.resultdisp:fmdtools.resultdisp.graph.show}}\pysiglinewithargsret{\sphinxcode{\sphinxupquote{fmdtools.resultdisp.graph.}}\sphinxbfcode{\sphinxupquote{show}}}{\emph{\DUrole{n}{g}}, \emph{\DUrole{n}{gtype}\DUrole{o}{=}\DUrole{default_value}{'bipartite'}}, \emph{\DUrole{n}{renderer}\DUrole{o}{=}\DUrole{default_value}{'matplotlib'}}, \emph{\DUrole{n}{filename}\DUrole{o}{=}\DUrole{default_value}{''}}, \emph{\DUrole{o}{**}\DUrole{n}{kwargs}}}{}
\sphinxAtStartPar
Plots a single graph object g.
\begin{quote}\begin{description}
\item[{Parameters}] \leavevmode\begin{itemize}
\item {} 
\sphinxAtStartPar
\sphinxstyleliteralstrong{\sphinxupquote{g}} (\sphinxstyleliteralemphasis{\sphinxupquote{networkx graph}}\sphinxstyleliteralemphasis{\sphinxupquote{ or }}\sphinxstyleliteralemphasis{\sphinxupquote{model}}) – The multigraph to plot

\item {} 
\sphinxAtStartPar
\sphinxstyleliteralstrong{\sphinxupquote{gtype}} (\sphinxstyleliteralemphasis{\sphinxupquote{'normal'}}\sphinxstyleliteralemphasis{\sphinxupquote{ or }}\sphinxstyleliteralemphasis{\sphinxupquote{'bipartite'}}) – Type of graph input to show–normal (multgraph) or bipartite

\item {} 
\sphinxAtStartPar
\sphinxstyleliteralstrong{\sphinxupquote{renderer}} (\sphinxstyleliteralemphasis{\sphinxupquote{'matplotlib'}}\sphinxstyleliteralemphasis{\sphinxupquote{ or }}\sphinxstyleliteralemphasis{\sphinxupquote{'graphviz'}}\sphinxstyleliteralemphasis{\sphinxupquote{ or }}\sphinxstyleliteralemphasis{\sphinxupquote{'pyvis'}}\sphinxstyleliteralemphasis{\sphinxupquote{ or }}\sphinxstyleliteralemphasis{\sphinxupquote{'netgraph'}}) – Renderer to use with the drawing. Renderer must be installed. Default is ‘matplotlib’

\item {} 
\sphinxAtStartPar
\sphinxstyleliteralstrong{\sphinxupquote{filename}} (\sphinxstyleliteralemphasis{\sphinxupquote{string}}\sphinxstyleliteralemphasis{\sphinxupquote{, }}\sphinxstyleliteralemphasis{\sphinxupquote{optional}}) – the filename for the output. The default is ‘’ (in which a file is not saved except in pyvis).

\item {} 
\sphinxAtStartPar
\sphinxstyleliteralstrong{\sphinxupquote{**kwargs}} (\sphinxstyleliteralemphasis{\sphinxupquote{dictionary}}) – \begin{description}
\item[{keyword arguments for the individual methods. See the documentation for}] \leavevmode
\sphinxAtStartPar
graph.show\_graphviz
graph.show\_maplotlib
graph.show\_pyvis
graph.show\_netgraph

\end{description}

\sphinxAtStartPar
for more information on these arguments


\end{itemize}

\end{description}\end{quote}

\end{fulllineitems}

\index{show\_graphviz() (in module fmdtools.resultdisp.graph)@\spxentry{show\_graphviz()}\spxextra{in module fmdtools.resultdisp.graph}}

\begin{fulllineitems}
\phantomsection\label{\detokenize{docs/fmdtools.resultdisp:fmdtools.resultdisp.graph.show_graphviz}}\pysiglinewithargsret{\sphinxcode{\sphinxupquote{fmdtools.resultdisp.graph.}}\sphinxbfcode{\sphinxupquote{show\_graphviz}}}{\emph{g, gtype='bipartite', faultscen={[}{]}, time={[}{]}, filename='', filetype='png', showfaultlabels=True, highlight={[}{]}, colors={[}'lightgray', 'orange', 'red'{]}, heatmap=\{\}, cmap=<matplotlib.colors.LinearSegmentedColormap object>, **kwargs}}{}
\sphinxAtStartPar
Translates an existing nx graph to a graphviz graph. Saves the graph output and dot file.
Called from show() by passing in graphviz=True and filename
\begin{quote}\begin{description}
\item[{Parameters}] \leavevmode\begin{itemize}
\item {} 
\sphinxAtStartPar
\sphinxstyleliteralstrong{\sphinxupquote{g}} (\sphinxstyleliteralemphasis{\sphinxupquote{nx graph object}}\sphinxstyleliteralemphasis{\sphinxupquote{ or }}\sphinxstyleliteralemphasis{\sphinxupquote{model}}) – The multigraph to plot

\item {} 
\sphinxAtStartPar
\sphinxstyleliteralstrong{\sphinxupquote{gtype}} (\sphinxstyleliteralemphasis{\sphinxupquote{string}}\sphinxstyleliteralemphasis{\sphinxupquote{, }}\sphinxstyleliteralemphasis{\sphinxupquote{optional}}) – Type of graph input to show
values are ‘normal’, ‘bipartite’, or ‘typegraph’. The default is ‘normal’.

\item {} 
\sphinxAtStartPar
\sphinxstyleliteralstrong{\sphinxupquote{filename}} (\sphinxstyleliteralemphasis{\sphinxupquote{string}}\sphinxstyleliteralemphasis{\sphinxupquote{, }}\sphinxstyleliteralemphasis{\sphinxupquote{optional}}) – the filename for the rendered output (if any). The default is ‘’ (in which the file is not saved).

\item {} 
\sphinxAtStartPar
\sphinxstyleliteralstrong{\sphinxupquote{filetype}} (\sphinxstyleliteralemphasis{\sphinxupquote{string}}) – Type of file to save the figure as (if saving)

\item {} 
\sphinxAtStartPar
\sphinxstyleliteralstrong{\sphinxupquote{faultscen}} (\sphinxstyleliteralemphasis{\sphinxupquote{str}}\sphinxstyleliteralemphasis{\sphinxupquote{, }}\sphinxstyleliteralemphasis{\sphinxupquote{optional}}) – Name of the fault scenario (for the title). The default is {[}{]}.

\item {} 
\sphinxAtStartPar
\sphinxstyleliteralstrong{\sphinxupquote{time}} (\sphinxstyleliteralemphasis{\sphinxupquote{float}}\sphinxstyleliteralemphasis{\sphinxupquote{, }}\sphinxstyleliteralemphasis{\sphinxupquote{optional}}) – Time of fault injection. The default is {[}{]}.

\item {} 
\sphinxAtStartPar
\sphinxstyleliteralstrong{\sphinxupquote{showfaultlabels}} (\sphinxstyleliteralemphasis{\sphinxupquote{bool}}\sphinxstyleliteralemphasis{\sphinxupquote{, }}\sphinxstyleliteralemphasis{\sphinxupquote{optional}}) – Whether or not to label the faults on the functions. The default is True.

\item {} 
\sphinxAtStartPar
\sphinxstyleliteralstrong{\sphinxupquote{highlight}} (\sphinxstyleliteralemphasis{\sphinxupquote{list}}\sphinxstyleliteralemphasis{\sphinxupquote{, }}\sphinxstyleliteralemphasis{\sphinxupquote{optional}}) – Functions/flows to highlight using {[}faulty functions, degraded functions, degraded flows{]} labelling scheme.
Used for custom overlays. Default is {[}{]}

\item {} 
\sphinxAtStartPar
\sphinxstyleliteralstrong{\sphinxupquote{colors}} (\sphinxstyleliteralemphasis{\sphinxupquote{list}}\sphinxstyleliteralemphasis{\sphinxupquote{, }}\sphinxstyleliteralemphasis{\sphinxupquote{optional}}) – List of colors to use for nominal, degraded, and faulty functions/flows.
Default is: {[}‘lightgray’,’orange’, ‘red’{]}

\item {} 
\sphinxAtStartPar
\sphinxstyleliteralstrong{\sphinxupquote{heatmap}} (\sphinxstyleliteralemphasis{\sphinxupquote{dict}}\sphinxstyleliteralemphasis{\sphinxupquote{, }}\sphinxstyleliteralemphasis{\sphinxupquote{optional}}) – A heatmap dictionary to overlay on the plot. The default is \{\}.

\item {} 
\sphinxAtStartPar
\sphinxstyleliteralstrong{\sphinxupquote{cmap}} (\sphinxstyleliteralemphasis{\sphinxupquote{mpl colormap}}) – Colormap to use for heatmap visualization

\item {} 
\sphinxAtStartPar
\sphinxstyleliteralstrong{\sphinxupquote{**kwargs}} (\sphinxstyleliteralemphasis{\sphinxupquote{dictionary}}) – dictionary of graphviz attributes used to customize the output.
this includes layout, overlap, node padding, node separation, font, fontsize, etc.
see \sphinxurl{http://www.graphviz.org/doc/info/attrs.html} for all options

\end{itemize}

\item[{Returns}] \leavevmode
\sphinxAtStartPar
\sphinxstylestrong{dot}

\item[{Return type}] \leavevmode
\sphinxAtStartPar
a graphviz object

\end{description}\end{quote}

\end{fulllineitems}

\index{show\_matplotlib() (in module fmdtools.resultdisp.graph)@\spxentry{show\_matplotlib()}\spxextra{in module fmdtools.resultdisp.graph}}

\begin{fulllineitems}
\phantomsection\label{\detokenize{docs/fmdtools.resultdisp:fmdtools.resultdisp.graph.show_matplotlib}}\pysiglinewithargsret{\sphinxcode{\sphinxupquote{fmdtools.resultdisp.graph.}}\sphinxbfcode{\sphinxupquote{show\_matplotlib}}}{\emph{g, gtype='bipartite', filename='', filetype='png', pos={[}{]}, scale=1, faultscen={[}{]}, time={[}{]}, figsize=(6, 4), showfaultlabels=True, highlight={[}{]}, colors={[}'lightgray', 'orange', 'red'{]}, heatmap=\{\}, cmap=<matplotlib.colors.LinearSegmentedColormap object>}}{}
\sphinxAtStartPar
Plots a single graph object g using matplotlib
\begin{quote}\begin{description}
\item[{Parameters}] \leavevmode\begin{itemize}
\item {} 
\sphinxAtStartPar
\sphinxstyleliteralstrong{\sphinxupquote{g}} (\sphinxstyleliteralemphasis{\sphinxupquote{networkx graph}}\sphinxstyleliteralemphasis{\sphinxupquote{ or }}\sphinxstyleliteralemphasis{\sphinxupquote{model}}) – The multigraph to plot

\item {} 
\sphinxAtStartPar
\sphinxstyleliteralstrong{\sphinxupquote{gtype}} (\sphinxstyleliteralemphasis{\sphinxupquote{'normal'}}\sphinxstyleliteralemphasis{\sphinxupquote{ or }}\sphinxstyleliteralemphasis{\sphinxupquote{'bipartite'}}) – Type of graph input to show–normal (multgraph) or bipartite

\item {} 
\sphinxAtStartPar
\sphinxstyleliteralstrong{\sphinxupquote{filename}} (\sphinxstyleliteralemphasis{\sphinxupquote{string}}) – Name to give the saved file, if saved. Default is ‘’ (not saving the file)

\item {} 
\sphinxAtStartPar
\sphinxstyleliteralstrong{\sphinxupquote{filetype}} (\sphinxstyleliteralemphasis{\sphinxupquote{string}}) – Type of file to save the figure as (if saving)

\item {} 
\sphinxAtStartPar
\sphinxstyleliteralstrong{\sphinxupquote{pos}} (\sphinxstyleliteralemphasis{\sphinxupquote{dict}}) – Positions for nodes

\item {} 
\sphinxAtStartPar
\sphinxstyleliteralstrong{\sphinxupquote{scale}} (\sphinxstyleliteralemphasis{\sphinxupquote{float}}) – Changes sizes of nodes in bipartite graph

\item {} 
\sphinxAtStartPar
\sphinxstyleliteralstrong{\sphinxupquote{faultscen}} (\sphinxstyleliteralemphasis{\sphinxupquote{str}}\sphinxstyleliteralemphasis{\sphinxupquote{, }}\sphinxstyleliteralemphasis{\sphinxupquote{optional}}) – Name of the fault scenario (for the title). The default is {[}{]}.

\item {} 
\sphinxAtStartPar
\sphinxstyleliteralstrong{\sphinxupquote{time}} (\sphinxstyleliteralemphasis{\sphinxupquote{float}}\sphinxstyleliteralemphasis{\sphinxupquote{, }}\sphinxstyleliteralemphasis{\sphinxupquote{optional}}) – Time of fault injection. The default is {[}{]}.

\item {} 
\sphinxAtStartPar
\sphinxstyleliteralstrong{\sphinxupquote{showfaultlabels}} (\sphinxstyleliteralemphasis{\sphinxupquote{bool}}\sphinxstyleliteralemphasis{\sphinxupquote{, }}\sphinxstyleliteralemphasis{\sphinxupquote{optional}}) – Whether or not to label the faults on the functions. The default is True.

\item {} 
\sphinxAtStartPar
\sphinxstyleliteralstrong{\sphinxupquote{highlight}} (\sphinxstyleliteralemphasis{\sphinxupquote{list}}\sphinxstyleliteralemphasis{\sphinxupquote{, }}\sphinxstyleliteralemphasis{\sphinxupquote{optional}}) – Functions/flows to highlight using {[}faulty functions, degraded functions, degraded flows{]} labelling scheme.
Used for custom overlays. Default is {[}{]}

\item {} 
\sphinxAtStartPar
\sphinxstyleliteralstrong{\sphinxupquote{colors}} (\sphinxstyleliteralemphasis{\sphinxupquote{list}}\sphinxstyleliteralemphasis{\sphinxupquote{, }}\sphinxstyleliteralemphasis{\sphinxupquote{optional}}) – List of colors to use for nominal, degraded, and faulty functions/flows.
Default is: {[}‘lightgray’,’orange’, ‘red’{]}

\item {} 
\sphinxAtStartPar
\sphinxstyleliteralstrong{\sphinxupquote{heatmap}} (\sphinxstyleliteralemphasis{\sphinxupquote{dict}}\sphinxstyleliteralemphasis{\sphinxupquote{, }}\sphinxstyleliteralemphasis{\sphinxupquote{optional}}) – A heatmap dictionary to overlay on the plot. The default is \{\}.

\item {} 
\sphinxAtStartPar
\sphinxstyleliteralstrong{\sphinxupquote{cmap}} (\sphinxstyleliteralemphasis{\sphinxupquote{mpl colormap}}) – Colormap to use for heatmap visualizations

\end{itemize}

\item[{Returns}] \leavevmode
\sphinxAtStartPar
\sphinxstylestrong{fig, ax} – Matplotlib figure object of the drawn graph

\item[{Return type}] \leavevmode
\sphinxAtStartPar
matplotlib figure/axis

\end{description}\end{quote}

\end{fulllineitems}

\index{show\_netgraph() (in module fmdtools.resultdisp.graph)@\spxentry{show\_netgraph()}\spxextra{in module fmdtools.resultdisp.graph}}

\begin{fulllineitems}
\phantomsection\label{\detokenize{docs/fmdtools.resultdisp:fmdtools.resultdisp.graph.show_netgraph}}\pysiglinewithargsret{\sphinxcode{\sphinxupquote{fmdtools.resultdisp.graph.}}\sphinxbfcode{\sphinxupquote{show\_netgraph}}}{\emph{\DUrole{n}{g}}, \emph{\DUrole{n}{gtype}\DUrole{o}{=}\DUrole{default_value}{'bipartite'}}, \emph{\DUrole{n}{filename}\DUrole{o}{=}\DUrole{default_value}{''}}, \emph{\DUrole{n}{filetype}\DUrole{o}{=}\DUrole{default_value}{'png'}}, \emph{\DUrole{n}{pos}\DUrole{o}{=}\DUrole{default_value}{{[}{]}}}, \emph{\DUrole{n}{scale}\DUrole{o}{=}\DUrole{default_value}{1}}, \emph{\DUrole{n}{faultscen}\DUrole{o}{=}\DUrole{default_value}{{[}{]}}}, \emph{\DUrole{n}{time}\DUrole{o}{=}\DUrole{default_value}{{[}{]}}}, \emph{\DUrole{n}{figsize}\DUrole{o}{=}\DUrole{default_value}{(6, 4)}}, \emph{\DUrole{n}{showfaultlabels}\DUrole{o}{=}\DUrole{default_value}{True}}, \emph{\DUrole{n}{highlight}\DUrole{o}{=}\DUrole{default_value}{{[}{]}}}, \emph{\DUrole{n}{colors}\DUrole{o}{=}\DUrole{default_value}{{[}'lightgray', 'orange', 'red'{]}}}, \emph{\DUrole{o}{**}\DUrole{n}{kwargs}}}{}
\sphinxAtStartPar
Plots a single graph object g using netgraph
\begin{quote}\begin{description}
\item[{Parameters}] \leavevmode\begin{itemize}
\item {} 
\sphinxAtStartPar
\sphinxstyleliteralstrong{\sphinxupquote{g}} (\sphinxstyleliteralemphasis{\sphinxupquote{networkx graph}}\sphinxstyleliteralemphasis{\sphinxupquote{ or }}\sphinxstyleliteralemphasis{\sphinxupquote{model}}) – The multigraph to plot

\item {} 
\sphinxAtStartPar
\sphinxstyleliteralstrong{\sphinxupquote{gtype}} (\sphinxstyleliteralemphasis{\sphinxupquote{'normal'}}\sphinxstyleliteralemphasis{\sphinxupquote{ or }}\sphinxstyleliteralemphasis{\sphinxupquote{'bipartite'}}) – Type of graph input to show–normal (multgraph) or bipartite

\item {} 
\sphinxAtStartPar
\sphinxstyleliteralstrong{\sphinxupquote{filename}} (\sphinxstyleliteralemphasis{\sphinxupquote{string}}) – Name to give the saved file, if saved. Default is ‘’ (not saving the file)

\item {} 
\sphinxAtStartPar
\sphinxstyleliteralstrong{\sphinxupquote{filetype}} (\sphinxstyleliteralemphasis{\sphinxupquote{string}}) – Type of file to save the figure as (if saving)

\item {} 
\sphinxAtStartPar
\sphinxstyleliteralstrong{\sphinxupquote{pos}} (\sphinxstyleliteralemphasis{\sphinxupquote{dict}}) – Positions for nodes

\item {} 
\sphinxAtStartPar
\sphinxstyleliteralstrong{\sphinxupquote{scale}} (\sphinxstyleliteralemphasis{\sphinxupquote{float}}) – Changes sizes of nodes in bipartite graph

\item {} 
\sphinxAtStartPar
\sphinxstyleliteralstrong{\sphinxupquote{faultscen}} (\sphinxstyleliteralemphasis{\sphinxupquote{str}}\sphinxstyleliteralemphasis{\sphinxupquote{, }}\sphinxstyleliteralemphasis{\sphinxupquote{optional}}) – Name of the fault scenario (for the title). The default is {[}{]}.

\item {} 
\sphinxAtStartPar
\sphinxstyleliteralstrong{\sphinxupquote{time}} (\sphinxstyleliteralemphasis{\sphinxupquote{float}}\sphinxstyleliteralemphasis{\sphinxupquote{, }}\sphinxstyleliteralemphasis{\sphinxupquote{optional}}) – Time of fault injection. The default is {[}{]}.

\item {} 
\sphinxAtStartPar
\sphinxstyleliteralstrong{\sphinxupquote{showfaultlabels}} (\sphinxstyleliteralemphasis{\sphinxupquote{bool}}\sphinxstyleliteralemphasis{\sphinxupquote{, }}\sphinxstyleliteralemphasis{\sphinxupquote{optional}}) – Whether or not to label the faults on the functions. The default is True.

\item {} 
\sphinxAtStartPar
\sphinxstyleliteralstrong{\sphinxupquote{highlight}} (\sphinxstyleliteralemphasis{\sphinxupquote{list}}\sphinxstyleliteralemphasis{\sphinxupquote{, }}\sphinxstyleliteralemphasis{\sphinxupquote{optional}}) – Functions/flows to highlight using {[}faulty functions, degraded functions, degraded flows{]} labelling scheme.
Used for custom overlays. Default is {[}{]}

\item {} 
\sphinxAtStartPar
\sphinxstyleliteralstrong{\sphinxupquote{colors}} (\sphinxstyleliteralemphasis{\sphinxupquote{list}}\sphinxstyleliteralemphasis{\sphinxupquote{, }}\sphinxstyleliteralemphasis{\sphinxupquote{optional}}) – List of colors to use for nominal, degraded, and faulty functions/flows.
Default is: {[}‘lightgray’,’orange’, ‘red’{]}

\end{itemize}

\item[{Returns}] \leavevmode
\sphinxAtStartPar
\begin{itemize}
\item {} 
\sphinxAtStartPar
\sphinxstylestrong{fig, ax} (\sphinxstyleemphasis{matplotlib figure/axis}) – Matplotlib figure object of the drawn graph

\item {} 
\sphinxAtStartPar
\sphinxstylestrong{gra} (\sphinxstyleemphasis{netgraph Graph}) – Netgraph object which can be further manipulated

\end{itemize}


\end{description}\end{quote}

\end{fulllineitems}

\index{show\_pyvis() (in module fmdtools.resultdisp.graph)@\spxentry{show\_pyvis()}\spxextra{in module fmdtools.resultdisp.graph}}

\begin{fulllineitems}
\phantomsection\label{\detokenize{docs/fmdtools.resultdisp:fmdtools.resultdisp.graph.show_pyvis}}\pysiglinewithargsret{\sphinxcode{\sphinxupquote{fmdtools.resultdisp.graph.}}\sphinxbfcode{\sphinxupquote{show\_pyvis}}}{\emph{\DUrole{n}{g}}, \emph{\DUrole{n}{gtype}\DUrole{o}{=}\DUrole{default_value}{'typegraph'}}, \emph{\DUrole{n}{filename}\DUrole{o}{=}\DUrole{default_value}{'typegraph'}}, \emph{\DUrole{n}{width}\DUrole{o}{=}\DUrole{default_value}{1000}}, \emph{\DUrole{n}{filt}\DUrole{o}{=}\DUrole{default_value}{True}}, \emph{\DUrole{n}{physics}\DUrole{o}{=}\DUrole{default_value}{False}}, \emph{\DUrole{n}{notebook}\DUrole{o}{=}\DUrole{default_value}{False}}}{}
\sphinxAtStartPar
Method for plotting graphs with pyvis. Produces interactive HTML!
\begin{quote}\begin{description}
\item[{Parameters}] \leavevmode\begin{itemize}
\item {} 
\sphinxAtStartPar
\sphinxstyleliteralstrong{\sphinxupquote{g}} (\sphinxstyleliteralemphasis{\sphinxupquote{networkx graph}}\sphinxstyleliteralemphasis{\sphinxupquote{ or }}\sphinxstyleliteralemphasis{\sphinxupquote{model}}) – Graph to plot or fmdtools model (which will be used to get the graph)

\item {} 
\sphinxAtStartPar
\sphinxstyleliteralstrong{\sphinxupquote{gtype}} (\sphinxstyleliteralemphasis{\sphinxupquote{'hierarchical'/'bipartite'/'component'}}\sphinxstyleliteralemphasis{\sphinxupquote{, }}\sphinxstyleliteralemphasis{\sphinxupquote{optional}}) – Type of model graph to plot The default is ‘hierarchical’.

\item {} 
\sphinxAtStartPar
\sphinxstyleliteralstrong{\sphinxupquote{filename}} (\sphinxstyleliteralemphasis{\sphinxupquote{str}}\sphinxstyleliteralemphasis{\sphinxupquote{, }}\sphinxstyleliteralemphasis{\sphinxupquote{optional}}) – File to save the html to. The default is “typegraph.html”.

\item {} 
\sphinxAtStartPar
\sphinxstyleliteralstrong{\sphinxupquote{width}} (\sphinxstyleliteralemphasis{\sphinxupquote{int}}\sphinxstyleliteralemphasis{\sphinxupquote{, }}\sphinxstyleliteralemphasis{\sphinxupquote{optional}}) – Width of the frame in px. The default is 1000.

\item {} 
\sphinxAtStartPar
\sphinxstyleliteralstrong{\sphinxupquote{filt}} (\sphinxstyleliteralemphasis{\sphinxupquote{Dict/Bool}}\sphinxstyleliteralemphasis{\sphinxupquote{, }}\sphinxstyleliteralemphasis{\sphinxupquote{optional}}) – Whether to display sliders. The default is True.

\item {} 
\sphinxAtStartPar
\sphinxstyleliteralstrong{\sphinxupquote{physics}} (\sphinxstyleliteralemphasis{\sphinxupquote{Bool}}\sphinxstyleliteralemphasis{\sphinxupquote{, }}\sphinxstyleliteralemphasis{\sphinxupquote{optional}}) – Whether to use physics during node placement. The default is False.

\end{itemize}

\item[{Returns}] \leavevmode
\sphinxAtStartPar
\sphinxstylestrong{n} – pyvis object of the drawn graph

\item[{Return type}] \leavevmode
\sphinxAtStartPar
pyvis object

\end{description}\end{quote}

\end{fulllineitems}

\index{update\_bipplot() (in module fmdtools.resultdisp.graph)@\spxentry{update\_bipplot()}\spxextra{in module fmdtools.resultdisp.graph}}

\begin{fulllineitems}
\phantomsection\label{\detokenize{docs/fmdtools.resultdisp:fmdtools.resultdisp.graph.update_bipplot}}\pysiglinewithargsret{\sphinxcode{\sphinxupquote{fmdtools.resultdisp.graph.}}\sphinxbfcode{\sphinxupquote{update\_bipplot}}}{\emph{\DUrole{n}{t\_ind}}, \emph{\DUrole{n}{reshist}}, \emph{\DUrole{n}{g}}, \emph{\DUrole{n}{pos}}, \emph{\DUrole{n}{faultscen}\DUrole{o}{=}\DUrole{default_value}{{[}{]}}}, \emph{\DUrole{n}{showfaultlabels}\DUrole{o}{=}\DUrole{default_value}{True}}, \emph{\DUrole{n}{scale}\DUrole{o}{=}\DUrole{default_value}{1}}, \emph{\DUrole{n}{show}\DUrole{o}{=}\DUrole{default_value}{True}}, \emph{\DUrole{n}{colors}\DUrole{o}{=}\DUrole{default_value}{{[}'lightgray', 'orange', 'red'{]}}}, \emph{\DUrole{o}{**}\DUrole{n}{kwargs}}}{}
\sphinxAtStartPar
Updates a bipartite graph plot at a given timestep t\_ind given the result history reshist

\end{fulllineitems}

\index{update\_graphplot() (in module fmdtools.resultdisp.graph)@\spxentry{update\_graphplot()}\spxextra{in module fmdtools.resultdisp.graph}}

\begin{fulllineitems}
\phantomsection\label{\detokenize{docs/fmdtools.resultdisp:fmdtools.resultdisp.graph.update_graphplot}}\pysiglinewithargsret{\sphinxcode{\sphinxupquote{fmdtools.resultdisp.graph.}}\sphinxbfcode{\sphinxupquote{update\_graphplot}}}{\emph{\DUrole{n}{t\_ind}}, \emph{\DUrole{n}{reshist}}, \emph{\DUrole{n}{g}}, \emph{\DUrole{n}{pos}}, \emph{\DUrole{n}{faultscen}\DUrole{o}{=}\DUrole{default_value}{{[}{]}}}, \emph{\DUrole{n}{showfaultlabels}\DUrole{o}{=}\DUrole{default_value}{True}}, \emph{\DUrole{n}{scale}\DUrole{o}{=}\DUrole{default_value}{1}}, \emph{\DUrole{n}{show}\DUrole{o}{=}\DUrole{default_value}{True}}, \emph{\DUrole{n}{colors}\DUrole{o}{=}\DUrole{default_value}{{[}'lightgray', 'orange', 'red'{]}}}, \emph{\DUrole{o}{**}\DUrole{n}{kwargs}}}{}
\sphinxAtStartPar
Updates a normal graph plot at a given timestep t\_ind given the result history reshist

\end{fulllineitems}

\index{update\_gv\_bipplot() (in module fmdtools.resultdisp.graph)@\spxentry{update\_gv\_bipplot()}\spxextra{in module fmdtools.resultdisp.graph}}

\begin{fulllineitems}
\phantomsection\label{\detokenize{docs/fmdtools.resultdisp:fmdtools.resultdisp.graph.update_gv_bipplot}}\pysiglinewithargsret{\sphinxcode{\sphinxupquote{fmdtools.resultdisp.graph.}}\sphinxbfcode{\sphinxupquote{update\_gv\_bipplot}}}{\emph{t\_ind, reshist, g, faultscen={[}{]}, showfaultlabels=True, colors={[}'lightgray', 'orange', 'red'{]}, heatmap=\{\}, cmap=<matplotlib.colors.LinearSegmentedColormap object>, **kwargs}}{}
\sphinxAtStartPar
graphviz helper: updates a bipartite graph plot at a given timestep t\_ind given the result history reshist

\end{fulllineitems}

\index{update\_gv\_graphplot() (in module fmdtools.resultdisp.graph)@\spxentry{update\_gv\_graphplot()}\spxextra{in module fmdtools.resultdisp.graph}}

\begin{fulllineitems}
\phantomsection\label{\detokenize{docs/fmdtools.resultdisp:fmdtools.resultdisp.graph.update_gv_graphplot}}\pysiglinewithargsret{\sphinxcode{\sphinxupquote{fmdtools.resultdisp.graph.}}\sphinxbfcode{\sphinxupquote{update\_gv\_graphplot}}}{\emph{t\_ind, reshist, g, faultscen={[}{]}, showfaultlabels=True, colors={[}'lightgray', 'orange', 'red'{]}, heatmap=\{\}, cmap=<matplotlib.colors.LinearSegmentedColormap object>, **kwargs}}{}
\sphinxAtStartPar
graphviz helpwer: Updates a normal graph plot at a given timestep t\_ind given the result history reshist

\end{fulllineitems}

\index{update\_net\_bipplot() (in module fmdtools.resultdisp.graph)@\spxentry{update\_net\_bipplot()}\spxextra{in module fmdtools.resultdisp.graph}}

\begin{fulllineitems}
\phantomsection\label{\detokenize{docs/fmdtools.resultdisp:fmdtools.resultdisp.graph.update_net_bipplot}}\pysiglinewithargsret{\sphinxcode{\sphinxupquote{fmdtools.resultdisp.graph.}}\sphinxbfcode{\sphinxupquote{update\_net\_bipplot}}}{\emph{\DUrole{n}{t\_ind}}, \emph{\DUrole{n}{reshist}}, \emph{\DUrole{n}{g}}, \emph{\DUrole{n}{pos}}, \emph{\DUrole{n}{faultscen}\DUrole{o}{=}\DUrole{default_value}{{[}{]}}}, \emph{\DUrole{n}{showfaultlabels}\DUrole{o}{=}\DUrole{default_value}{True}}, \emph{\DUrole{n}{scale}\DUrole{o}{=}\DUrole{default_value}{1}}, \emph{\DUrole{n}{show}\DUrole{o}{=}\DUrole{default_value}{True}}, \emph{\DUrole{n}{colors}\DUrole{o}{=}\DUrole{default_value}{{[}'lightgray', 'orange', 'red'{]}}}, \emph{\DUrole{o}{**}\DUrole{n}{kwargs}}}{}
\sphinxAtStartPar
Updates a bipartite graph plot at a given timestep t\_ind given the result history reshist

\end{fulllineitems}

\index{update\_net\_graphplot() (in module fmdtools.resultdisp.graph)@\spxentry{update\_net\_graphplot()}\spxextra{in module fmdtools.resultdisp.graph}}

\begin{fulllineitems}
\phantomsection\label{\detokenize{docs/fmdtools.resultdisp:fmdtools.resultdisp.graph.update_net_graphplot}}\pysiglinewithargsret{\sphinxcode{\sphinxupquote{fmdtools.resultdisp.graph.}}\sphinxbfcode{\sphinxupquote{update\_net\_graphplot}}}{\emph{\DUrole{n}{t\_ind}}, \emph{\DUrole{n}{reshist}}, \emph{\DUrole{n}{g}}, \emph{\DUrole{n}{pos}}, \emph{\DUrole{n}{faultscen}\DUrole{o}{=}\DUrole{default_value}{{[}{]}}}, \emph{\DUrole{n}{showfaultlabels}\DUrole{o}{=}\DUrole{default_value}{True}}, \emph{\DUrole{n}{scale}\DUrole{o}{=}\DUrole{default_value}{1}}, \emph{\DUrole{n}{show}\DUrole{o}{=}\DUrole{default_value}{True}}, \emph{\DUrole{n}{colors}\DUrole{o}{=}\DUrole{default_value}{{[}'lightgray', 'orange', 'red'{]}}}, \emph{\DUrole{o}{**}\DUrole{n}{kwargs}}}{}
\sphinxAtStartPar
Updates a normal graph plot at a given timestep t\_ind given the result history reshist

\end{fulllineitems}

\index{update\_net\_typegraphplot() (in module fmdtools.resultdisp.graph)@\spxentry{update\_net\_typegraphplot()}\spxextra{in module fmdtools.resultdisp.graph}}

\begin{fulllineitems}
\phantomsection\label{\detokenize{docs/fmdtools.resultdisp:fmdtools.resultdisp.graph.update_net_typegraphplot}}\pysiglinewithargsret{\sphinxcode{\sphinxupquote{fmdtools.resultdisp.graph.}}\sphinxbfcode{\sphinxupquote{update\_net\_typegraphplot}}}{\emph{\DUrole{n}{t\_ind}}, \emph{\DUrole{n}{reshist}}, \emph{\DUrole{n}{g}}, \emph{\DUrole{n}{pos}}, \emph{\DUrole{n}{faultscen}\DUrole{o}{=}\DUrole{default_value}{{[}{]}}}, \emph{\DUrole{n}{showfaultlabels}\DUrole{o}{=}\DUrole{default_value}{True}}, \emph{\DUrole{n}{scale}\DUrole{o}{=}\DUrole{default_value}{1}}, \emph{\DUrole{n}{show}\DUrole{o}{=}\DUrole{default_value}{True}}, \emph{\DUrole{n}{colors}\DUrole{o}{=}\DUrole{default_value}{{[}'lightgray', 'orange', 'red'{]}}}, \emph{\DUrole{o}{**}\DUrole{n}{kwargs}}}{}
\sphinxAtStartPar
Updates a typegraph\sphinxhyphen{}stype plot at a given timestep t\_ind given the result history reshist

\end{fulllineitems}

\index{update\_typegraphplot() (in module fmdtools.resultdisp.graph)@\spxentry{update\_typegraphplot()}\spxextra{in module fmdtools.resultdisp.graph}}

\begin{fulllineitems}
\phantomsection\label{\detokenize{docs/fmdtools.resultdisp:fmdtools.resultdisp.graph.update_typegraphplot}}\pysiglinewithargsret{\sphinxcode{\sphinxupquote{fmdtools.resultdisp.graph.}}\sphinxbfcode{\sphinxupquote{update\_typegraphplot}}}{\emph{\DUrole{n}{t\_ind}}, \emph{\DUrole{n}{reshist}}, \emph{\DUrole{n}{g}}, \emph{\DUrole{n}{pos}}, \emph{\DUrole{n}{faultscen}\DUrole{o}{=}\DUrole{default_value}{{[}{]}}}, \emph{\DUrole{n}{showfaultlabels}\DUrole{o}{=}\DUrole{default_value}{True}}, \emph{\DUrole{n}{scale}\DUrole{o}{=}\DUrole{default_value}{1}}, \emph{\DUrole{n}{show}\DUrole{o}{=}\DUrole{default_value}{True}}, \emph{\DUrole{n}{colors}\DUrole{o}{=}\DUrole{default_value}{{[}'lightgray', 'orange', 'red'{]}}}, \emph{\DUrole{o}{**}\DUrole{n}{kwargs}}}{}
\sphinxAtStartPar
Updates a typegraph\sphinxhyphen{}stype plot at a given timestep t\_ind given the result history reshist

\end{fulllineitems}



\subsubsection{fmdtools.resultdisp.plot}
\label{\detokenize{docs/fmdtools.resultdisp:module-fmdtools.resultdisp.plot}}\label{\detokenize{docs/fmdtools.resultdisp:fmdtools-resultdisp-plot}}\index{module@\spxentry{module}!fmdtools.resultdisp.plot@\spxentry{fmdtools.resultdisp.plot}}\index{fmdtools.resultdisp.plot@\spxentry{fmdtools.resultdisp.plot}!module@\spxentry{module}}
\sphinxAtStartPar
Description: Plots quantities of interest over time using matplotlib.
\begin{description}
\item[{Uses the following methods:}] \leavevmode\begin{itemize}
\item {} 
\sphinxAtStartPar
{\hyperref[\detokenize{docs/fmdtools.resultdisp:fmdtools.resultdisp.plot.mdlhist}]{\sphinxcrossref{\sphinxcode{\sphinxupquote{mdlhist()}}}}}:         plots function and flow histories over time (with different plots for each function/flow)

\item {} 
\sphinxAtStartPar
{\hyperref[\detokenize{docs/fmdtools.resultdisp:fmdtools.resultdisp.plot.mdlhistvals}]{\sphinxcrossref{\sphinxcode{\sphinxupquote{mdlhistvals()}}}}}:     plots function and flow histories over time on a single plot

\item {} 
\sphinxAtStartPar
{\hyperref[\detokenize{docs/fmdtools.resultdisp:fmdtools.resultdisp.plot.mdlhists}]{\sphinxcrossref{\sphinxcode{\sphinxupquote{mdlhists()}}}}}:        plots function and flow histories over time with multiple scenarios on the same plot

\item {} 
\sphinxAtStartPar
{\hyperref[\detokenize{docs/fmdtools.resultdisp:fmdtools.resultdisp.plot.nominal_vals_1d}]{\sphinxcrossref{\sphinxcode{\sphinxupquote{nominal\_vals\_1d()}}}}}: plots the end\sphinxhyphen{}state classification of a system over a (1\sphinxhyphen{}D) range of nominal runs

\item {} 
\sphinxAtStartPar
{\hyperref[\detokenize{docs/fmdtools.resultdisp:fmdtools.resultdisp.plot.nominal_vals_2d}]{\sphinxcrossref{\sphinxcode{\sphinxupquote{nominal\_vals\_2d()}}}}}: plots the end\sphinxhyphen{}state classification of a system over a (2\sphinxhyphen{}D) range of nominal runs

\item {} 
\sphinxAtStartPar
{\hyperref[\detokenize{docs/fmdtools.resultdisp:fmdtools.resultdisp.plot.nominal_vals_3d}]{\sphinxcrossref{\sphinxcode{\sphinxupquote{nominal\_vals\_3d()}}}}}: plots the end\sphinxhyphen{}state classification of a system over a (3\sphinxhyphen{}D) range of nominal runs

\item {} 
\sphinxAtStartPar
{\hyperref[\detokenize{docs/fmdtools.resultdisp:fmdtools.resultdisp.plot.nominal_factor_comparison}]{\sphinxcrossref{\sphinxcode{\sphinxupquote{nominal\_factor\_comparison()}}}}}:    gives a bar plot of nominal simulation statistics over given factors

\item {} 
\sphinxAtStartPar
{\hyperref[\detokenize{docs/fmdtools.resultdisp:fmdtools.resultdisp.plot.resilience_factor_comparison}]{\sphinxcrossref{\sphinxcode{\sphinxupquote{resilience\_factor\_comparison()}}}}}: gives a bar plot of fault simulation statistics over given factors

\item {} 
\sphinxAtStartPar
{\hyperref[\detokenize{docs/fmdtools.resultdisp:fmdtools.resultdisp.plot.phases}]{\sphinxcrossref{\sphinxcode{\sphinxupquote{phases()}}}}}:          plots the phases of operation that the model progresses through.

\item {} 
\sphinxAtStartPar
{\hyperref[\detokenize{docs/fmdtools.resultdisp:fmdtools.resultdisp.plot.samplecost}]{\sphinxcrossref{\sphinxcode{\sphinxupquote{samplecost()}}}}}:      plots the costs for a single fault sampled by a SampleApproach over time with rates

\item {} 
\sphinxAtStartPar
{\hyperref[\detokenize{docs/fmdtools.resultdisp:fmdtools.resultdisp.plot.samplecosts}]{\sphinxcrossref{\sphinxcode{\sphinxupquote{samplecosts()}}}}}:     plots the costs for a set of faults sampled by a SampleApproach over time with rates on separate plots

\item {} 
\sphinxAtStartPar
{\hyperref[\detokenize{docs/fmdtools.resultdisp:fmdtools.resultdisp.plot.costovertime}]{\sphinxcrossref{\sphinxcode{\sphinxupquote{costovertime()}}}}}:    plots the total cost/explected cost of a set of faults sampled by a SampleApproach over time

\end{itemize}

\end{description}
\index{costovertime() (in module fmdtools.resultdisp.plot)@\spxentry{costovertime()}\spxextra{in module fmdtools.resultdisp.plot}}

\begin{fulllineitems}
\phantomsection\label{\detokenize{docs/fmdtools.resultdisp:fmdtools.resultdisp.plot.costovertime}}\pysiglinewithargsret{\sphinxcode{\sphinxupquote{fmdtools.resultdisp.plot.}}\sphinxbfcode{\sphinxupquote{costovertime}}}{\emph{\DUrole{n}{endclasses}}, \emph{\DUrole{n}{app}}, \emph{\DUrole{n}{costtype}\DUrole{o}{=}\DUrole{default_value}{'expected cost'}}}{}
\sphinxAtStartPar
Plots the total cost or total expected cost of faults over time.
\begin{quote}\begin{description}
\item[{Parameters}] \leavevmode\begin{itemize}
\item {} 
\sphinxAtStartPar
\sphinxstyleliteralstrong{\sphinxupquote{endclasses}} (\sphinxstyleliteralemphasis{\sphinxupquote{dict}}) – dict with rate,cost, and expected cost for each injected scenario (e.g. from run\_approach())

\item {} 
\sphinxAtStartPar
\sphinxstyleliteralstrong{\sphinxupquote{app}} (\sphinxstyleliteralemphasis{\sphinxupquote{sampleapproach}}) – sample approach used to generate the list of scenarios

\item {} 
\sphinxAtStartPar
\sphinxstyleliteralstrong{\sphinxupquote{costtype}} (\sphinxstyleliteralemphasis{\sphinxupquote{str}}\sphinxstyleliteralemphasis{\sphinxupquote{, }}\sphinxstyleliteralemphasis{\sphinxupquote{optional}}) – type of cost to plot (‘cost’, ‘expected cost’ or ‘rate’). The default is ‘expected cost’.

\end{itemize}

\end{description}\end{quote}

\end{fulllineitems}

\index{dyn\_order() (in module fmdtools.resultdisp.plot)@\spxentry{dyn\_order()}\spxextra{in module fmdtools.resultdisp.plot}}

\begin{fulllineitems}
\phantomsection\label{\detokenize{docs/fmdtools.resultdisp:fmdtools.resultdisp.plot.dyn_order}}\pysiglinewithargsret{\sphinxcode{\sphinxupquote{fmdtools.resultdisp.plot.}}\sphinxbfcode{\sphinxupquote{dyn\_order}}}{\emph{\DUrole{n}{mdl}}, \emph{\DUrole{n}{rotateticks}\DUrole{o}{=}\DUrole{default_value}{False}}, \emph{\DUrole{n}{title}\DUrole{o}{=}\DUrole{default_value}{'Dynamic Run Order'}}}{}
\sphinxAtStartPar
Plots the run order for the model during the dynamic propagation step used
by dynamic\_behavior() methods, where the x\sphinxhyphen{}direction is the order of each
function executed and the y are the corresponding flows acted on by the
given methods.
\begin{quote}\begin{description}
\item[{Parameters}] \leavevmode\begin{itemize}
\item {} 
\sphinxAtStartPar
\sphinxstyleliteralstrong{\sphinxupquote{mdl}} ({\hyperref[\detokenize{docs/fmdtools:fmdtools.modeldef.Model}]{\sphinxcrossref{\sphinxstyleliteralemphasis{\sphinxupquote{Model}}}}}) – fmdtools model

\item {} 
\sphinxAtStartPar
\sphinxstyleliteralstrong{\sphinxupquote{rotateticks}} (\sphinxstyleliteralemphasis{\sphinxupquote{Bool}}\sphinxstyleliteralemphasis{\sphinxupquote{, }}\sphinxstyleliteralemphasis{\sphinxupquote{optional}}) – Whether to rotate the x\sphinxhyphen{}ticks (for bigger plots). The default is False.

\item {} 
\sphinxAtStartPar
\sphinxstyleliteralstrong{\sphinxupquote{title}} (\sphinxstyleliteralemphasis{\sphinxupquote{str}}\sphinxstyleliteralemphasis{\sphinxupquote{, }}\sphinxstyleliteralemphasis{\sphinxupquote{optional}}) – String to use for the title (if any). The default is “Dynamic Run Order”.

\end{itemize}

\item[{Returns}] \leavevmode
\sphinxAtStartPar
\begin{itemize}
\item {} 
\sphinxAtStartPar
\sphinxstylestrong{fig} (\sphinxstyleemphasis{figure}) – Matplotlib figure object

\item {} 
\sphinxAtStartPar
\sphinxstylestrong{ax} (\sphinxstyleemphasis{axis}) – Corresponding matplotlib axis

\end{itemize}


\end{description}\end{quote}

\end{fulllineitems}

\index{mdlhist() (in module fmdtools.resultdisp.plot)@\spxentry{mdlhist()}\spxextra{in module fmdtools.resultdisp.plot}}

\begin{fulllineitems}
\phantomsection\label{\detokenize{docs/fmdtools.resultdisp:fmdtools.resultdisp.plot.mdlhist}}\pysiglinewithargsret{\sphinxcode{\sphinxupquote{fmdtools.resultdisp.plot.}}\sphinxbfcode{\sphinxupquote{mdlhist}}}{\emph{\DUrole{n}{mdlhist}}, \emph{\DUrole{n}{fault}\DUrole{o}{=}\DUrole{default_value}{''}}, \emph{\DUrole{n}{time}\DUrole{o}{=}\DUrole{default_value}{0}}, \emph{\DUrole{n}{fxnflows}\DUrole{o}{=}\DUrole{default_value}{{[}{]}}}, \emph{\DUrole{n}{cols}\DUrole{o}{=}\DUrole{default_value}{2}}, \emph{\DUrole{n}{returnfigs}\DUrole{o}{=}\DUrole{default_value}{False}}, \emph{\DUrole{n}{legend}\DUrole{o}{=}\DUrole{default_value}{True}}, \emph{\DUrole{n}{timelabel}\DUrole{o}{=}\DUrole{default_value}{'Time'}}, \emph{\DUrole{n}{units}\DUrole{o}{=}\DUrole{default_value}{{[}{]}}}, \emph{\DUrole{n}{phases}\DUrole{o}{=}\DUrole{default_value}{\{\}}}, \emph{\DUrole{n}{modephases}\DUrole{o}{=}\DUrole{default_value}{\{\}}}, \emph{\DUrole{n}{label\_phases}\DUrole{o}{=}\DUrole{default_value}{True}}}{}
\sphinxAtStartPar
Plots all states of the model at a time given a model history on separate plots.
\begin{quote}\begin{description}
\item[{Parameters}] \leavevmode\begin{itemize}
\item {} 
\sphinxAtStartPar
\sphinxstyleliteralstrong{\sphinxupquote{mdlhist}} (\sphinxstyleliteralemphasis{\sphinxupquote{dict}}) – History of states over time. Can be just the scenario states or a dict of scenario states and nominal states per \{‘nominal’:nomhist,’faulty’:mdlhist\}

\item {} 
\sphinxAtStartPar
\sphinxstyleliteralstrong{\sphinxupquote{fault}} (\sphinxstyleliteralemphasis{\sphinxupquote{str}}\sphinxstyleliteralemphasis{\sphinxupquote{, }}\sphinxstyleliteralemphasis{\sphinxupquote{optional}}) – Name of the fault (for the title). The default is ‘’.

\item {} 
\sphinxAtStartPar
\sphinxstyleliteralstrong{\sphinxupquote{time}} (\sphinxstyleliteralemphasis{\sphinxupquote{float}}\sphinxstyleliteralemphasis{\sphinxupquote{, }}\sphinxstyleliteralemphasis{\sphinxupquote{optional}}) – Time of fault injection. The default is 0.

\item {} 
\sphinxAtStartPar
\sphinxstyleliteralstrong{\sphinxupquote{fxnflows}} (\sphinxstyleliteralemphasis{\sphinxupquote{list}}\sphinxstyleliteralemphasis{\sphinxupquote{, }}\sphinxstyleliteralemphasis{\sphinxupquote{optional}}) – List of functions and flows to plot. The default is {[}{]}, which returns all.

\item {} 
\sphinxAtStartPar
\sphinxstyleliteralstrong{\sphinxupquote{cols}} (\sphinxstyleliteralemphasis{\sphinxupquote{int}}\sphinxstyleliteralemphasis{\sphinxupquote{, }}\sphinxstyleliteralemphasis{\sphinxupquote{optional}}) – columns to use in the figure. The default is 2.

\item {} 
\sphinxAtStartPar
\sphinxstyleliteralstrong{\sphinxupquote{returnfigs}} (\sphinxstyleliteralemphasis{\sphinxupquote{bool}}\sphinxstyleliteralemphasis{\sphinxupquote{, }}\sphinxstyleliteralemphasis{\sphinxupquote{optional}}) – Whether to return the figure objects in a list. The default is False.

\item {} 
\sphinxAtStartPar
\sphinxstyleliteralstrong{\sphinxupquote{legend}} (\sphinxstyleliteralemphasis{\sphinxupquote{bool}}\sphinxstyleliteralemphasis{\sphinxupquote{, }}\sphinxstyleliteralemphasis{\sphinxupquote{optional}}) – Whether the plot should have a legend for faulty and nominal states. The default is true.

\item {} 
\sphinxAtStartPar
\sphinxstyleliteralstrong{\sphinxupquote{timelabel}} (\sphinxstyleliteralemphasis{\sphinxupquote{str}}\sphinxstyleliteralemphasis{\sphinxupquote{, }}\sphinxstyleliteralemphasis{\sphinxupquote{optional}}) – Label to use for the x\sphinxhyphen{}axes (e.g., seconds, minutes). Default is “time”.

\item {} 
\sphinxAtStartPar
\sphinxstyleliteralstrong{\sphinxupquote{units}} (\sphinxstyleliteralemphasis{\sphinxupquote{dict}}\sphinxstyleliteralemphasis{\sphinxupquote{, }}\sphinxstyleliteralemphasis{\sphinxupquote{optional}}) – Labels to use for the y\sphinxhyphen{}axes (e.g., power, voltage). Default is ‘’

\item {} 
\sphinxAtStartPar
\sphinxstyleliteralstrong{\sphinxupquote{phases}} (\sphinxstyleliteralemphasis{\sphinxupquote{dict}}\sphinxstyleliteralemphasis{\sphinxupquote{, }}\sphinxstyleliteralemphasis{\sphinxupquote{optional}}) – Phase dictionary from process.modephases. Overlays lines over function values corresponding to the phase progression.

\item {} 
\sphinxAtStartPar
\sphinxstyleliteralstrong{\sphinxupquote{modephases}} (\sphinxstyleliteralemphasis{\sphinxupquote{dict}}\sphinxstyleliteralemphasis{\sphinxupquote{, }}\sphinxstyleliteralemphasis{\sphinxupquote{optional}}) – Modephase dictionary from process.modephases. Makes the phase overlay labels correspond to mode names instead of phases.

\item {} 
\sphinxAtStartPar
\sphinxstyleliteralstrong{\sphinxupquote{label\_phases}} (\sphinxstyleliteralemphasis{\sphinxupquote{book}}\sphinxstyleliteralemphasis{\sphinxupquote{, }}\sphinxstyleliteralemphasis{\sphinxupquote{optional}}) – Whether to overlay labels on phases (or just leave lines)

\end{itemize}

\end{description}\end{quote}

\end{fulllineitems}

\index{mdlhists() (in module fmdtools.resultdisp.plot)@\spxentry{mdlhists()}\spxextra{in module fmdtools.resultdisp.plot}}

\begin{fulllineitems}
\phantomsection\label{\detokenize{docs/fmdtools.resultdisp:fmdtools.resultdisp.plot.mdlhists}}\pysiglinewithargsret{\sphinxcode{\sphinxupquote{fmdtools.resultdisp.plot.}}\sphinxbfcode{\sphinxupquote{mdlhists}}}{\emph{\DUrole{n}{mdlhists}}, \emph{\DUrole{n}{fxnflowvals}}, \emph{\DUrole{n}{cols}\DUrole{o}{=}\DUrole{default_value}{2}}, \emph{\DUrole{n}{aggregation}\DUrole{o}{=}\DUrole{default_value}{'individual'}}, \emph{\DUrole{n}{comp\_groups}\DUrole{o}{=}\DUrole{default_value}{\{\}}}, \emph{\DUrole{n}{legend\_loc}\DUrole{o}{=}\DUrole{default_value}{\sphinxhyphen{} 1}}, \emph{\DUrole{n}{xlabel}\DUrole{o}{=}\DUrole{default_value}{'time'}}, \emph{\DUrole{n}{ylabels}\DUrole{o}{=}\DUrole{default_value}{\{\}}}, \emph{\DUrole{n}{max\_ind}\DUrole{o}{=}\DUrole{default_value}{'max'}}, \emph{\DUrole{n}{boundtype}\DUrole{o}{=}\DUrole{default_value}{'fill'}}, \emph{\DUrole{n}{fillalpha}\DUrole{o}{=}\DUrole{default_value}{0.3}}, \emph{\DUrole{n}{boundcolor}\DUrole{o}{=}\DUrole{default_value}{'gray'}}, \emph{\DUrole{n}{boundlinestyle}\DUrole{o}{=}\DUrole{default_value}{'\sphinxhyphen{}\sphinxhyphen{}'}}, \emph{\DUrole{n}{ci}\DUrole{o}{=}\DUrole{default_value}{0.95}}, \emph{\DUrole{n}{title}\DUrole{o}{=}\DUrole{default_value}{''}}, \emph{\DUrole{n}{indiv\_kwargs}\DUrole{o}{=}\DUrole{default_value}{\{\}}}, \emph{\DUrole{n}{time\_slice}\DUrole{o}{=}\DUrole{default_value}{{[}{]}}}, \emph{\DUrole{n}{figsize}\DUrole{o}{=}\DUrole{default_value}{'default'}}, \emph{\DUrole{o}{**}\DUrole{n}{kwargs}}}{}
\sphinxAtStartPar
Plot the behavior over time of the given function/flow values
over a set of scenarios, with ability to aggregate behaviors as needed.
\begin{quote}\begin{description}
\item[{Parameters}] \leavevmode\begin{itemize}
\item {} 
\sphinxAtStartPar
\sphinxstyleliteralstrong{\sphinxupquote{mdlhists}} (\sphinxstyleliteralemphasis{\sphinxupquote{dict}}) – Aggregate model history with structure \{‘scen’:mdlhist\}

\item {} 
\sphinxAtStartPar
\sphinxstyleliteralstrong{\sphinxupquote{fxnflowsvals}} (\sphinxstyleliteralemphasis{\sphinxupquote{dict}}\sphinxstyleliteralemphasis{\sphinxupquote{, }}\sphinxstyleliteralemphasis{\sphinxupquote{optional}}) – dict of flow values to plot with structure \{fxnflow:{[}vals{]}\}. The default is \{\}, which returns all.

\item {} 
\sphinxAtStartPar
\sphinxstyleliteralstrong{\sphinxupquote{cols}} (\sphinxstyleliteralemphasis{\sphinxupquote{int}}\sphinxstyleliteralemphasis{\sphinxupquote{, }}\sphinxstyleliteralemphasis{\sphinxupquote{optional}}) – columns to use in the figure. The default is 2.

\item {} 
\sphinxAtStartPar
\sphinxstyleliteralstrong{\sphinxupquote{aggregation}} (\sphinxstyleliteralemphasis{\sphinxupquote{str}}) – 
\sphinxAtStartPar
Way of aggregating the plot values. The default is ‘individual’
Note that only the \sphinxtitleref{individual} option can be used for histories of non\sphinxhyphen{}numeric quantities
(e.g., modes, which are recorded as strings)
\sphinxhyphen{} ‘individual’ plots each run individually.
\sphinxhyphen{} ‘mean\_std’ plots the mean values over the sim with standard deviation error bars
\sphinxhyphen{} ‘mean\_ci’  plots the mean values over the sim with mean confidence interval error bars
\begin{itemize}
\item {} 
\sphinxAtStartPar
optional argument ci (float 0.0\sphinxhyphen{}1.0) specifies the confidence interval (Default:0.95)

\end{itemize}
\begin{itemize}
\item {} 
\sphinxAtStartPar
’mean\_bound’ plots the mean values over the sim with variable bound error bars

\item {} \begin{description}
\item[{’percentile’ plots the percentile distribution of the sim over time (does not reject outliers)}] \leavevmode\begin{itemize}
\item {} 
\sphinxAtStartPar
optional argument ‘perc\_range’ (int 0\sphinxhyphen{}100) specifies the percentile range of the inner bars (Default: 50)

\end{itemize}

\end{description}

\end{itemize}


\item {} 
\sphinxAtStartPar
\sphinxstyleliteralstrong{\sphinxupquote{comp\_groups}} (\sphinxstyleliteralemphasis{\sphinxupquote{dict}}) – \begin{description}
\item[{Dictionary for comparison groups (if more than one) with structure:}] \leavevmode
\sphinxAtStartPar
\{‘group1’:(‘scen1’, ‘scen2’), ‘group2’:(‘scen3’, ‘scen4’)\} Default is \{\}
If a legend is shown, group names are used as labels.

\end{description}


\item {} 
\sphinxAtStartPar
\sphinxstyleliteralstrong{\sphinxupquote{legend\_loc}} (\sphinxstyleliteralemphasis{\sphinxupquote{int}}) – Specifies the plot to place the legend on, if runs are bine compared. Default is \sphinxhyphen{}1 (the last plot)
To remove the legend, give a value of False

\item {} 
\sphinxAtStartPar
\sphinxstyleliteralstrong{\sphinxupquote{dict}} (\sphinxstyleliteralemphasis{\sphinxupquote{indiv\_kwargs}}) – dict of kwargs with structure \{comp1:kwargs1, comp2:kwargs2\}, where
where kwargs is an individual dict of keyword arguments for the
comparison group comp (or scenario, if not aggregated) which overrides
the global kwargs (or default behavior).

\item {} 
\sphinxAtStartPar
\sphinxstyleliteralstrong{\sphinxupquote{xlabel}} (\sphinxstyleliteralemphasis{\sphinxupquote{str}}) – Label for the x\sphinxhyphen{}axes. Default is ‘time’

\item {} 
\sphinxAtStartPar
\sphinxstyleliteralstrong{\sphinxupquote{ylabel}} (\sphinxstyleliteralemphasis{\sphinxupquote{dict}}) – Label for the y\sphinxhyphen{}axes with structure \{(fxnflowname, value):’label’\}

\item {} 
\sphinxAtStartPar
\sphinxstyleliteralstrong{\sphinxupquote{max\_ind}} (\sphinxstyleliteralemphasis{\sphinxupquote{int}}) – index (usually correlates to time) cutoff for the simulation. Default is ‘max’ which uses the first simulation termination time.

\item {} 
\sphinxAtStartPar
\sphinxstyleliteralstrong{\sphinxupquote{boundtype}} (\sphinxstyleliteralemphasis{\sphinxupquote{'fill'}}\sphinxstyleliteralemphasis{\sphinxupquote{ or }}\sphinxstyleliteralemphasis{\sphinxupquote{'line'}}) – \begin{description}
\item[{\sphinxhyphen{}‘fill’ plots the error bounds as a filled area}] \leavevmode\begin{itemize}
\item {} 
\sphinxAtStartPar
optional fillalpha (float) changes the alpha of this area.

\end{itemize}

\item[{\sphinxhyphen{}‘line’ plots the error bounds as lines}] \leavevmode\begin{itemize}
\item {} 
\sphinxAtStartPar
optional boundcolor (str) changes the color of the bounds (default ‘gray’)

\item {} 
\sphinxAtStartPar
optional boundlinestyle (str) changes the style of the bound lines (default ‘–‘)

\end{itemize}

\end{description}


\item {} 
\sphinxAtStartPar
\sphinxstyleliteralstrong{\sphinxupquote{title}} (\sphinxstyleliteralemphasis{\sphinxupquote{str}}) – overall title for the plot. Default is ‘’

\item {} 
\sphinxAtStartPar
\sphinxstyleliteralstrong{\sphinxupquote{time\_slice}} (\sphinxstyleliteralemphasis{\sphinxupquote{int/list}}) – overlays a bar or bars at the given index when the fault was injected (if any). Default is {[}{]}

\item {} 
\sphinxAtStartPar
\sphinxstyleliteralstrong{\sphinxupquote{figsize}} (\sphinxstyleliteralemphasis{\sphinxupquote{tuple}}\sphinxstyleliteralemphasis{\sphinxupquote{ (}}\sphinxstyleliteralemphasis{\sphinxupquote{float}}\sphinxstyleliteralemphasis{\sphinxupquote{,}}\sphinxstyleliteralemphasis{\sphinxupquote{float}}\sphinxstyleliteralemphasis{\sphinxupquote{)}}) – x\sphinxhyphen{}y size for the figure. The default is ‘default’, which dymanically gives 3 for each column and 2 for each row

\item {} 
\sphinxAtStartPar
\sphinxstyleliteralstrong{\sphinxupquote{**kwargs}} (\sphinxstyleliteralemphasis{\sphinxupquote{kwargs}}) – keyword arguments to mpl.plot e.g. linestyle, color, etc. See ‘aggregation’ for specification.

\end{itemize}

\end{description}\end{quote}

\end{fulllineitems}

\index{mdlhistvals() (in module fmdtools.resultdisp.plot)@\spxentry{mdlhistvals()}\spxextra{in module fmdtools.resultdisp.plot}}

\begin{fulllineitems}
\phantomsection\label{\detokenize{docs/fmdtools.resultdisp:fmdtools.resultdisp.plot.mdlhistvals}}\pysiglinewithargsret{\sphinxcode{\sphinxupquote{fmdtools.resultdisp.plot.}}\sphinxbfcode{\sphinxupquote{mdlhistvals}}}{\emph{\DUrole{n}{mdlhist}}, \emph{\DUrole{n}{fault}\DUrole{o}{=}\DUrole{default_value}{''}}, \emph{\DUrole{n}{time}\DUrole{o}{=}\DUrole{default_value}{0}}, \emph{\DUrole{n}{fxnflowvals}\DUrole{o}{=}\DUrole{default_value}{\{\}}}, \emph{\DUrole{n}{cols}\DUrole{o}{=}\DUrole{default_value}{2}}, \emph{\DUrole{n}{returnfig}\DUrole{o}{=}\DUrole{default_value}{True}}, \emph{\DUrole{n}{legend}\DUrole{o}{=}\DUrole{default_value}{True}}, \emph{\DUrole{n}{timelabel}\DUrole{o}{=}\DUrole{default_value}{'time'}}, \emph{\DUrole{n}{units}\DUrole{o}{=}\DUrole{default_value}{{[}{]}}}, \emph{\DUrole{n}{phases}\DUrole{o}{=}\DUrole{default_value}{\{\}}}, \emph{\DUrole{n}{modephases}\DUrole{o}{=}\DUrole{default_value}{\{\}}}, \emph{\DUrole{n}{label\_phases}\DUrole{o}{=}\DUrole{default_value}{True}}}{}
\sphinxAtStartPar
Plots the states of a model over time given a history.
\begin{quote}\begin{description}
\item[{Parameters}] \leavevmode\begin{itemize}
\item {} 
\sphinxAtStartPar
\sphinxstyleliteralstrong{\sphinxupquote{mdlhist}} (\sphinxstyleliteralemphasis{\sphinxupquote{dict}}) – History of states over time. Can be just the scenario states or a dict of scenario states and nominal states per \{‘nominal’:nomhist,’faulty’:mdlhist\}

\item {} 
\sphinxAtStartPar
\sphinxstyleliteralstrong{\sphinxupquote{fault}} (\sphinxstyleliteralemphasis{\sphinxupquote{str}}\sphinxstyleliteralemphasis{\sphinxupquote{, }}\sphinxstyleliteralemphasis{\sphinxupquote{optional}}) – Name of the fault (for the title). The default is ‘’.

\item {} 
\sphinxAtStartPar
\sphinxstyleliteralstrong{\sphinxupquote{time}} (\sphinxstyleliteralemphasis{\sphinxupquote{float}}\sphinxstyleliteralemphasis{\sphinxupquote{, }}\sphinxstyleliteralemphasis{\sphinxupquote{optional}}) – Time of fault injection. The default is 0.

\item {} 
\sphinxAtStartPar
\sphinxstyleliteralstrong{\sphinxupquote{fxnflowsvals}} (\sphinxstyleliteralemphasis{\sphinxupquote{dict}}\sphinxstyleliteralemphasis{\sphinxupquote{, }}\sphinxstyleliteralemphasis{\sphinxupquote{optional}}) – dict of flow values to plot with structure \{fxnflow:{[}vals{]}\}. The default is \{\}, which returns all.

\item {} 
\sphinxAtStartPar
\sphinxstyleliteralstrong{\sphinxupquote{cols}} (\sphinxstyleliteralemphasis{\sphinxupquote{int}}\sphinxstyleliteralemphasis{\sphinxupquote{, }}\sphinxstyleliteralemphasis{\sphinxupquote{optional}}) – columns to use in the figure. The default is 2.

\item {} 
\sphinxAtStartPar
\sphinxstyleliteralstrong{\sphinxupquote{returnfig}} (\sphinxstyleliteralemphasis{\sphinxupquote{bool}}\sphinxstyleliteralemphasis{\sphinxupquote{, }}\sphinxstyleliteralemphasis{\sphinxupquote{optional}}) – Whether to return the figure. The default is False.

\item {} 
\sphinxAtStartPar
\sphinxstyleliteralstrong{\sphinxupquote{legend}} (\sphinxstyleliteralemphasis{\sphinxupquote{bool}}\sphinxstyleliteralemphasis{\sphinxupquote{, }}\sphinxstyleliteralemphasis{\sphinxupquote{optional}}) – Whether the plot should have a legend for faulty and nominal states. The default is true

\item {} 
\sphinxAtStartPar
\sphinxstyleliteralstrong{\sphinxupquote{timelabel}} (\sphinxstyleliteralemphasis{\sphinxupquote{str}}\sphinxstyleliteralemphasis{\sphinxupquote{, }}\sphinxstyleliteralemphasis{\sphinxupquote{optional}}) – Label to use for the x\sphinxhyphen{}axes (e.g., seconds, minutes). Default is “time”.

\item {} 
\sphinxAtStartPar
\sphinxstyleliteralstrong{\sphinxupquote{units}} (\sphinxstyleliteralemphasis{\sphinxupquote{dict}}\sphinxstyleliteralemphasis{\sphinxupquote{, }}\sphinxstyleliteralemphasis{\sphinxupquote{optional}}) – Labels to use for the y\sphinxhyphen{}axes (e.g., power, voltage). Default is ‘’

\item {} 
\sphinxAtStartPar
\sphinxstyleliteralstrong{\sphinxupquote{phases}} (\sphinxstyleliteralemphasis{\sphinxupquote{dict}}\sphinxstyleliteralemphasis{\sphinxupquote{, }}\sphinxstyleliteralemphasis{\sphinxupquote{optional}}) – Phase dictionary from process.modephases. Overlays lines over function values corresponding to the phase progression.

\item {} 
\sphinxAtStartPar
\sphinxstyleliteralstrong{\sphinxupquote{modephases}} (\sphinxstyleliteralemphasis{\sphinxupquote{dict}}\sphinxstyleliteralemphasis{\sphinxupquote{, }}\sphinxstyleliteralemphasis{\sphinxupquote{optional}}) – Modephase dictionary from process.modephases. Makes the phase overlay labels correspond to mode names instead of phases.

\item {} 
\sphinxAtStartPar
\sphinxstyleliteralstrong{\sphinxupquote{label\_phases}} (\sphinxstyleliteralemphasis{\sphinxupquote{book}}\sphinxstyleliteralemphasis{\sphinxupquote{, }}\sphinxstyleliteralemphasis{\sphinxupquote{optional}}) – Whether to overlay labels on phases (or just leave lines)

\end{itemize}

\end{description}\end{quote}

\end{fulllineitems}

\index{nominal\_factor\_comparison() (in module fmdtools.resultdisp.plot)@\spxentry{nominal\_factor\_comparison()}\spxextra{in module fmdtools.resultdisp.plot}}

\begin{fulllineitems}
\phantomsection\label{\detokenize{docs/fmdtools.resultdisp:fmdtools.resultdisp.plot.nominal_factor_comparison}}\pysiglinewithargsret{\sphinxcode{\sphinxupquote{fmdtools.resultdisp.plot.}}\sphinxbfcode{\sphinxupquote{nominal\_factor\_comparison}}}{\emph{\DUrole{n}{comparison\_table}}, \emph{\DUrole{n}{metric}}, \emph{\DUrole{n}{ylabel}\DUrole{o}{=}\DUrole{default_value}{'proportion'}}, \emph{\DUrole{n}{figsize}\DUrole{o}{=}\DUrole{default_value}{(6, 4)}}, \emph{\DUrole{n}{title}\DUrole{o}{=}\DUrole{default_value}{''}}, \emph{\DUrole{n}{maxy}\DUrole{o}{=}\DUrole{default_value}{'max'}}, \emph{\DUrole{n}{xlabel}\DUrole{o}{=}\DUrole{default_value}{True}}, \emph{\DUrole{n}{error\_bars}\DUrole{o}{=}\DUrole{default_value}{False}}}{}
\sphinxAtStartPar
Compares/plots a comparison table from tabulate.nominal\_factor\_comparison as a bar plot for a given metric.
\begin{quote}\begin{description}
\item[{Parameters}] \leavevmode\begin{itemize}
\item {} 
\sphinxAtStartPar
\sphinxstyleliteralstrong{\sphinxupquote{comparison\_table}} (\sphinxstyleliteralemphasis{\sphinxupquote{pandas table}}) – Table from tabulate.nominal\_factor\_comparison

\item {} 
\sphinxAtStartPar
\sphinxstyleliteralstrong{\sphinxupquote{metrics}} (\sphinxstyleliteralemphasis{\sphinxupquote{string}}) – Metric to use in the plot

\item {} 
\sphinxAtStartPar
\sphinxstyleliteralstrong{\sphinxupquote{ylabel}} (\sphinxstyleliteralemphasis{\sphinxupquote{string}}\sphinxstyleliteralemphasis{\sphinxupquote{, }}\sphinxstyleliteralemphasis{\sphinxupquote{optional}}) – label for the y\sphinxhyphen{}axis. The default is ‘proportion’.

\item {} 
\sphinxAtStartPar
\sphinxstyleliteralstrong{\sphinxupquote{figsize}} (\sphinxstyleliteralemphasis{\sphinxupquote{tuple}}\sphinxstyleliteralemphasis{\sphinxupquote{, }}\sphinxstyleliteralemphasis{\sphinxupquote{optional}}) – Size for the plot. The default is (12,8).

\item {} 
\sphinxAtStartPar
\sphinxstyleliteralstrong{\sphinxupquote{title}} (\sphinxstyleliteralemphasis{\sphinxupquote{str}}\sphinxstyleliteralemphasis{\sphinxupquote{, }}\sphinxstyleliteralemphasis{\sphinxupquote{optional}}) – Plot title. The default is ‘’.

\item {} 
\sphinxAtStartPar
\sphinxstyleliteralstrong{\sphinxupquote{maxy}} (\sphinxstyleliteralemphasis{\sphinxupquote{float}}) – Cutoff for the y\sphinxhyphen{}axis (to use if the default is bad). The default is ‘max’

\item {} 
\sphinxAtStartPar
\sphinxstyleliteralstrong{\sphinxupquote{xlabel}} (\sphinxstyleliteralemphasis{\sphinxupquote{TYPE}}\sphinxstyleliteralemphasis{\sphinxupquote{, }}\sphinxstyleliteralemphasis{\sphinxupquote{optional}}) – DESCRIPTION. The default is True.

\item {} 
\sphinxAtStartPar
\sphinxstyleliteralstrong{\sphinxupquote{error\_bars}} (\sphinxstyleliteralemphasis{\sphinxupquote{TYPE}}\sphinxstyleliteralemphasis{\sphinxupquote{, }}\sphinxstyleliteralemphasis{\sphinxupquote{optional}}) – DESCRIPTION. The default is False.

\end{itemize}

\item[{Returns}] \leavevmode
\sphinxAtStartPar
\sphinxstylestrong{figure}

\item[{Return type}] \leavevmode
\sphinxAtStartPar
matplotlib figure

\end{description}\end{quote}

\end{fulllineitems}

\index{nominal\_vals\_1d() (in module fmdtools.resultdisp.plot)@\spxentry{nominal\_vals\_1d()}\spxextra{in module fmdtools.resultdisp.plot}}

\begin{fulllineitems}
\phantomsection\label{\detokenize{docs/fmdtools.resultdisp:fmdtools.resultdisp.plot.nominal_vals_1d}}\pysiglinewithargsret{\sphinxcode{\sphinxupquote{fmdtools.resultdisp.plot.}}\sphinxbfcode{\sphinxupquote{nominal\_vals\_1d}}}{\emph{\DUrole{n}{nomapp}}, \emph{\DUrole{n}{nomapp\_endclasses}}, \emph{\DUrole{n}{param1}}, \emph{\DUrole{n}{title}\DUrole{o}{=}\DUrole{default_value}{'Nominal Operational Envelope'}}, \emph{\DUrole{n}{nomlabel}\DUrole{o}{=}\DUrole{default_value}{'nominal'}}, \emph{\DUrole{n}{metric}\DUrole{o}{=}\DUrole{default_value}{'classification'}}}{}
\sphinxAtStartPar
Visualizes the nominal operational envelope along one given parameter
\begin{quote}\begin{description}
\item[{Parameters}] \leavevmode\begin{itemize}
\item {} 
\sphinxAtStartPar
\sphinxstyleliteralstrong{\sphinxupquote{nomapp}} ({\hyperref[\detokenize{docs/fmdtools:fmdtools.modeldef.NominalApproach}]{\sphinxcrossref{\sphinxstyleliteralemphasis{\sphinxupquote{NominalApproach}}}}}) – Nominal sample approach simulated in the model.

\item {} 
\sphinxAtStartPar
\sphinxstyleliteralstrong{\sphinxupquote{nomapp\_endclasses}} (\sphinxstyleliteralemphasis{\sphinxupquote{dict}}) – End\sphinxhyphen{}classifications for the set of simulations in the model.

\item {} 
\sphinxAtStartPar
\sphinxstyleliteralstrong{\sphinxupquote{param1}} (\sphinxstyleliteralemphasis{\sphinxupquote{str}}) – Parameter range desired to visualize in the operational envelope

\item {} 
\sphinxAtStartPar
\sphinxstyleliteralstrong{\sphinxupquote{title}} (\sphinxstyleliteralemphasis{\sphinxupquote{str}}\sphinxstyleliteralemphasis{\sphinxupquote{, }}\sphinxstyleliteralemphasis{\sphinxupquote{optional}}) – Plot title. The default is “Nominal Operational Envelope”.

\item {} 
\sphinxAtStartPar
\sphinxstyleliteralstrong{\sphinxupquote{nomlabel}} (\sphinxstyleliteralemphasis{\sphinxupquote{str}}\sphinxstyleliteralemphasis{\sphinxupquote{, }}\sphinxstyleliteralemphasis{\sphinxupquote{optional}}) – Flag for nominal end\sphinxhyphen{}states. The default is ‘nominal’.

\end{itemize}

\item[{Returns}] \leavevmode
\sphinxAtStartPar
\sphinxstylestrong{fig} – Figure for the plot.

\item[{Return type}] \leavevmode
\sphinxAtStartPar
matplotlib figure

\end{description}\end{quote}

\end{fulllineitems}

\index{nominal\_vals\_2d() (in module fmdtools.resultdisp.plot)@\spxentry{nominal\_vals\_2d()}\spxextra{in module fmdtools.resultdisp.plot}}

\begin{fulllineitems}
\phantomsection\label{\detokenize{docs/fmdtools.resultdisp:fmdtools.resultdisp.plot.nominal_vals_2d}}\pysiglinewithargsret{\sphinxcode{\sphinxupquote{fmdtools.resultdisp.plot.}}\sphinxbfcode{\sphinxupquote{nominal\_vals\_2d}}}{\emph{\DUrole{n}{nomapp}}, \emph{\DUrole{n}{nomapp\_endclasses}}, \emph{\DUrole{n}{param1}}, \emph{\DUrole{n}{param2}}, \emph{\DUrole{n}{title}\DUrole{o}{=}\DUrole{default_value}{'Nominal Operational Envelope'}}, \emph{\DUrole{n}{nomlabel}\DUrole{o}{=}\DUrole{default_value}{'nominal'}}, \emph{\DUrole{n}{metric}\DUrole{o}{=}\DUrole{default_value}{'classification'}}}{}
\sphinxAtStartPar
Visualizes the nominal operational envelope along two given parameters
\begin{quote}\begin{description}
\item[{Parameters}] \leavevmode\begin{itemize}
\item {} 
\sphinxAtStartPar
\sphinxstyleliteralstrong{\sphinxupquote{nomapp}} ({\hyperref[\detokenize{docs/fmdtools:fmdtools.modeldef.NominalApproach}]{\sphinxcrossref{\sphinxstyleliteralemphasis{\sphinxupquote{NominalApproach}}}}}) – Nominal sample approach simulated in the model.

\item {} 
\sphinxAtStartPar
\sphinxstyleliteralstrong{\sphinxupquote{nomapp\_endclasses}} (\sphinxstyleliteralemphasis{\sphinxupquote{dict}}) – End\sphinxhyphen{}classifications for the set of simulations in the model.

\item {} 
\sphinxAtStartPar
\sphinxstyleliteralstrong{\sphinxupquote{param1}} (\sphinxstyleliteralemphasis{\sphinxupquote{str}}) – First parameter (x) desired to visualize in the operational envelope

\item {} 
\sphinxAtStartPar
\sphinxstyleliteralstrong{\sphinxupquote{param2}} (\sphinxstyleliteralemphasis{\sphinxupquote{str}}) – Second arameter (y) desired to visualize in the operational envelope

\item {} 
\sphinxAtStartPar
\sphinxstyleliteralstrong{\sphinxupquote{title}} (\sphinxstyleliteralemphasis{\sphinxupquote{str}}\sphinxstyleliteralemphasis{\sphinxupquote{, }}\sphinxstyleliteralemphasis{\sphinxupquote{optional}}) – Plot title. The default is “Nominal Operational Envelope”.

\item {} 
\sphinxAtStartPar
\sphinxstyleliteralstrong{\sphinxupquote{nomlabel}} (\sphinxstyleliteralemphasis{\sphinxupquote{str}}\sphinxstyleliteralemphasis{\sphinxupquote{, }}\sphinxstyleliteralemphasis{\sphinxupquote{optional}}) – Flag for nominal end\sphinxhyphen{}states. The default is ‘nominal’.

\end{itemize}

\item[{Returns}] \leavevmode
\sphinxAtStartPar
\sphinxstylestrong{fig} – Figure for the plot.

\item[{Return type}] \leavevmode
\sphinxAtStartPar
matplotlib figure

\end{description}\end{quote}

\end{fulllineitems}

\index{nominal\_vals\_3d() (in module fmdtools.resultdisp.plot)@\spxentry{nominal\_vals\_3d()}\spxextra{in module fmdtools.resultdisp.plot}}

\begin{fulllineitems}
\phantomsection\label{\detokenize{docs/fmdtools.resultdisp:fmdtools.resultdisp.plot.nominal_vals_3d}}\pysiglinewithargsret{\sphinxcode{\sphinxupquote{fmdtools.resultdisp.plot.}}\sphinxbfcode{\sphinxupquote{nominal\_vals\_3d}}}{\emph{\DUrole{n}{nomapp}}, \emph{\DUrole{n}{nomapp\_endclasses}}, \emph{\DUrole{n}{param1}}, \emph{\DUrole{n}{param2}}, \emph{\DUrole{n}{param3}}, \emph{\DUrole{n}{title}\DUrole{o}{=}\DUrole{default_value}{'Nominal Operational Envelope'}}, \emph{\DUrole{n}{nomlabel}\DUrole{o}{=}\DUrole{default_value}{'nominal'}}, \emph{\DUrole{n}{metric}\DUrole{o}{=}\DUrole{default_value}{'classification'}}}{}
\sphinxAtStartPar
Visualizes the nominal operational envelope along three given parameters
\begin{quote}\begin{description}
\item[{Parameters}] \leavevmode\begin{itemize}
\item {} 
\sphinxAtStartPar
\sphinxstyleliteralstrong{\sphinxupquote{nomapp}} ({\hyperref[\detokenize{docs/fmdtools:fmdtools.modeldef.NominalApproach}]{\sphinxcrossref{\sphinxstyleliteralemphasis{\sphinxupquote{NominalApproach}}}}}) – Nominal sample approach simulated in the model.

\item {} 
\sphinxAtStartPar
\sphinxstyleliteralstrong{\sphinxupquote{nomapp\_endclasses}} (\sphinxstyleliteralemphasis{\sphinxupquote{dict}}) – End\sphinxhyphen{}classifications for the set of simulations in the model.

\item {} 
\sphinxAtStartPar
\sphinxstyleliteralstrong{\sphinxupquote{param1}} (\sphinxstyleliteralemphasis{\sphinxupquote{str}}) – First parameter (x) desired to visualize in the operational envelope

\item {} 
\sphinxAtStartPar
\sphinxstyleliteralstrong{\sphinxupquote{param2}} (\sphinxstyleliteralemphasis{\sphinxupquote{str}}) – Second parameter (y) desired to visualize in the operational envelope

\item {} 
\sphinxAtStartPar
\sphinxstyleliteralstrong{\sphinxupquote{param3}} (\sphinxstyleliteralemphasis{\sphinxupquote{str}}) – Third parameter (y) desired to visualize in the operational envelope

\item {} 
\sphinxAtStartPar
\sphinxstyleliteralstrong{\sphinxupquote{title}} (\sphinxstyleliteralemphasis{\sphinxupquote{str}}\sphinxstyleliteralemphasis{\sphinxupquote{, }}\sphinxstyleliteralemphasis{\sphinxupquote{optional}}) – Plot title. The default is “Nominal Operational Envelope”.

\item {} 
\sphinxAtStartPar
\sphinxstyleliteralstrong{\sphinxupquote{nomlabel}} (\sphinxstyleliteralemphasis{\sphinxupquote{str}}\sphinxstyleliteralemphasis{\sphinxupquote{, }}\sphinxstyleliteralemphasis{\sphinxupquote{optional}}) – Flag for nominal end\sphinxhyphen{}states. The default is ‘nominal’.

\end{itemize}

\item[{Returns}] \leavevmode
\sphinxAtStartPar
\sphinxstylestrong{fig} – Figure for the plot.

\item[{Return type}] \leavevmode
\sphinxAtStartPar
matplotlib figure

\end{description}\end{quote}

\end{fulllineitems}

\index{phases() (in module fmdtools.resultdisp.plot)@\spxentry{phases()}\spxextra{in module fmdtools.resultdisp.plot}}

\begin{fulllineitems}
\phantomsection\label{\detokenize{docs/fmdtools.resultdisp:fmdtools.resultdisp.plot.phases}}\pysiglinewithargsret{\sphinxcode{\sphinxupquote{fmdtools.resultdisp.plot.}}\sphinxbfcode{\sphinxupquote{phases}}}{\emph{\DUrole{n}{mdlphases}}, \emph{\DUrole{n}{modephases}\DUrole{o}{=}\DUrole{default_value}{{[}{]}}}, \emph{\DUrole{n}{mdl}\DUrole{o}{=}\DUrole{default_value}{{[}{]}}}, \emph{\DUrole{n}{singleplot}\DUrole{o}{=}\DUrole{default_value}{True}}, \emph{\DUrole{n}{phase\_ticks}\DUrole{o}{=}\DUrole{default_value}{'both'}}}{}
\sphinxAtStartPar
Plots the phases of operation that the model progresses through.
\begin{quote}\begin{description}
\item[{Parameters}] \leavevmode\begin{itemize}
\item {} 
\sphinxAtStartPar
\sphinxstyleliteralstrong{\sphinxupquote{mdlphases}} (\sphinxstyleliteralemphasis{\sphinxupquote{dict}}) – phases that the functions of the model progresses through (e.g. from rd.process.mdlhist)
of structure \{‘fxnname’:’phase’:{[}start, end{]}\}

\item {} 
\sphinxAtStartPar
\sphinxstyleliteralstrong{\sphinxupquote{modephases}} (\sphinxstyleliteralemphasis{\sphinxupquote{dict}}\sphinxstyleliteralemphasis{\sphinxupquote{, }}\sphinxstyleliteralemphasis{\sphinxupquote{optional}}) – dictionary that maps the phases to operational modes, if it is desired to track the progression
through modes

\item {} 
\sphinxAtStartPar
\sphinxstyleliteralstrong{\sphinxupquote{mdl}} ({\hyperref[\detokenize{docs/fmdtools:fmdtools.modeldef.Model}]{\sphinxcrossref{\sphinxstyleliteralemphasis{\sphinxupquote{Model}}}}}\sphinxstyleliteralemphasis{\sphinxupquote{, }}\sphinxstyleliteralemphasis{\sphinxupquote{optional}}) – model, if it is desired to additionally plot the phases of the model with the function phases

\item {} 
\sphinxAtStartPar
\sphinxstyleliteralstrong{\sphinxupquote{singleplot}} (\sphinxstyleliteralemphasis{\sphinxupquote{bool}}\sphinxstyleliteralemphasis{\sphinxupquote{, }}\sphinxstyleliteralemphasis{\sphinxupquote{optional}}) – Whether the functions’ progressions through phases are plotted on the same plot or on different plots.
The default is True.

\item {} 
\sphinxAtStartPar
\sphinxstyleliteralstrong{\sphinxupquote{phase\_ticks}} (\sphinxstyleliteralemphasis{\sphinxupquote{'std'/'phases'/'both'}}) – x\sphinxhyphen{}ticks to use (standard, at the edge of phases, or both). Default is ‘both’

\end{itemize}

\item[{Returns}] \leavevmode
\sphinxAtStartPar
\sphinxstylestrong{fig/figs} – Matplotlib figures to edit/use.

\item[{Return type}] \leavevmode
\sphinxAtStartPar
Figure or list of Figures

\end{description}\end{quote}

\end{fulllineitems}

\index{plot\_err\_lines() (in module fmdtools.resultdisp.plot)@\spxentry{plot\_err\_lines()}\spxextra{in module fmdtools.resultdisp.plot}}

\begin{fulllineitems}
\phantomsection\label{\detokenize{docs/fmdtools.resultdisp:fmdtools.resultdisp.plot.plot_err_lines}}\pysiglinewithargsret{\sphinxcode{\sphinxupquote{fmdtools.resultdisp.plot.}}\sphinxbfcode{\sphinxupquote{plot\_err\_lines}}}{\emph{\DUrole{n}{ax}}, \emph{\DUrole{n}{times}}, \emph{\DUrole{n}{lows}}, \emph{\DUrole{n}{highs}}, \emph{\DUrole{o}{**}\DUrole{n}{kwargs}}}{}
\sphinxAtStartPar
Plots error lines on the given plot
\begin{quote}\begin{description}
\item[{Parameters}] \leavevmode\begin{itemize}
\item {} 
\sphinxAtStartPar
\sphinxstyleliteralstrong{\sphinxupquote{ax}} (\sphinxstyleliteralemphasis{\sphinxupquote{mpl axis}}) – axis to plot the line on

\item {} 
\sphinxAtStartPar
\sphinxstyleliteralstrong{\sphinxupquote{times}} (\sphinxstyleliteralemphasis{\sphinxupquote{list/array}}) – x data (time, typically)

\item {} 
\sphinxAtStartPar
\sphinxstyleliteralstrong{\sphinxupquote{line}} (\sphinxstyleliteralemphasis{\sphinxupquote{list/array}}) – y center data to plot

\item {} 
\sphinxAtStartPar
\sphinxstyleliteralstrong{\sphinxupquote{lows}} (\sphinxstyleliteralemphasis{\sphinxupquote{list/array}}) – y lower bound to plot

\item {} 
\sphinxAtStartPar
\sphinxstyleliteralstrong{\sphinxupquote{highs}} (\sphinxstyleliteralemphasis{\sphinxupquote{list/array}}) – y upper bound to plot

\item {} 
\sphinxAtStartPar
\sphinxstyleliteralstrong{\sphinxupquote{**kwargs}} (\sphinxstyleliteralemphasis{\sphinxupquote{kwargs}}) – kwargs for the line

\end{itemize}

\end{description}\end{quote}

\end{fulllineitems}

\index{plot\_line\_and\_err() (in module fmdtools.resultdisp.plot)@\spxentry{plot\_line\_and\_err()}\spxextra{in module fmdtools.resultdisp.plot}}

\begin{fulllineitems}
\phantomsection\label{\detokenize{docs/fmdtools.resultdisp:fmdtools.resultdisp.plot.plot_line_and_err}}\pysiglinewithargsret{\sphinxcode{\sphinxupquote{fmdtools.resultdisp.plot.}}\sphinxbfcode{\sphinxupquote{plot\_line\_and\_err}}}{\emph{\DUrole{n}{ax}}, \emph{\DUrole{n}{times}}, \emph{\DUrole{n}{line}}, \emph{\DUrole{n}{lows}}, \emph{\DUrole{n}{highs}}, \emph{\DUrole{n}{boundtype}}, \emph{\DUrole{n}{boundcolor}\DUrole{o}{=}\DUrole{default_value}{'gray'}}, \emph{\DUrole{n}{boundlinestyle}\DUrole{o}{=}\DUrole{default_value}{'\sphinxhyphen{}\sphinxhyphen{}'}}, \emph{\DUrole{n}{fillalpha}\DUrole{o}{=}\DUrole{default_value}{0.3}}, \emph{\DUrole{o}{**}\DUrole{n}{kwargs}}}{}
\sphinxAtStartPar
Plots a line with a given range of uncertainty around it.
\begin{quote}\begin{description}
\item[{Parameters}] \leavevmode\begin{itemize}
\item {} 
\sphinxAtStartPar
\sphinxstyleliteralstrong{\sphinxupquote{ax}} (\sphinxstyleliteralemphasis{\sphinxupquote{mpl axis}}) – axis to plot the line on

\item {} 
\sphinxAtStartPar
\sphinxstyleliteralstrong{\sphinxupquote{times}} (\sphinxstyleliteralemphasis{\sphinxupquote{list/array}}) – x data (time, typically)

\item {} 
\sphinxAtStartPar
\sphinxstyleliteralstrong{\sphinxupquote{line}} (\sphinxstyleliteralemphasis{\sphinxupquote{list/array}}) – y center data to plot

\item {} 
\sphinxAtStartPar
\sphinxstyleliteralstrong{\sphinxupquote{lows}} (\sphinxstyleliteralemphasis{\sphinxupquote{list/array}}) – y lower bound to plot

\item {} 
\sphinxAtStartPar
\sphinxstyleliteralstrong{\sphinxupquote{highs}} (\sphinxstyleliteralemphasis{\sphinxupquote{list/array}}) – y upper bound to plot

\item {} 
\sphinxAtStartPar
\sphinxstyleliteralstrong{\sphinxupquote{boundtype}} (\sphinxstyleliteralemphasis{\sphinxupquote{'fill'}}\sphinxstyleliteralemphasis{\sphinxupquote{ or }}\sphinxstyleliteralemphasis{\sphinxupquote{'line'}}) – Whether the bounds should be marked with lines or a fill

\item {} 
\sphinxAtStartPar
\sphinxstyleliteralstrong{\sphinxupquote{boundcolor}} (\sphinxstyleliteralemphasis{\sphinxupquote{str}}\sphinxstyleliteralemphasis{\sphinxupquote{, }}\sphinxstyleliteralemphasis{\sphinxupquote{optional}}) – Color for bound fill The default is ‘gray’.

\item {} 
\sphinxAtStartPar
\sphinxstyleliteralstrong{\sphinxupquote{boundlinestyle}} (\sphinxstyleliteralemphasis{\sphinxupquote{str}}\sphinxstyleliteralemphasis{\sphinxupquote{, }}\sphinxstyleliteralemphasis{\sphinxupquote{optional}}) – linestyle for bound lines (if any). The default is ‘–‘.

\item {} 
\sphinxAtStartPar
\sphinxstyleliteralstrong{\sphinxupquote{fillalpha}} (\sphinxstyleliteralemphasis{\sphinxupquote{float}}\sphinxstyleliteralemphasis{\sphinxupquote{, }}\sphinxstyleliteralemphasis{\sphinxupquote{optional}}) – Alpha for fill. The default is 0.3.

\item {} 
\sphinxAtStartPar
\sphinxstyleliteralstrong{\sphinxupquote{**kwargs}} (\sphinxstyleliteralemphasis{\sphinxupquote{kwargs}}) – kwargs for the line

\end{itemize}

\end{description}\end{quote}

\end{fulllineitems}

\index{resilience\_factor\_comparison() (in module fmdtools.resultdisp.plot)@\spxentry{resilience\_factor\_comparison()}\spxextra{in module fmdtools.resultdisp.plot}}

\begin{fulllineitems}
\phantomsection\label{\detokenize{docs/fmdtools.resultdisp:fmdtools.resultdisp.plot.resilience_factor_comparison}}\pysiglinewithargsret{\sphinxcode{\sphinxupquote{fmdtools.resultdisp.plot.}}\sphinxbfcode{\sphinxupquote{resilience\_factor\_comparison}}}{\emph{\DUrole{n}{comparison\_table}}, \emph{\DUrole{n}{faults}\DUrole{o}{=}\DUrole{default_value}{'all'}}, \emph{\DUrole{n}{rows}\DUrole{o}{=}\DUrole{default_value}{1}}, \emph{\DUrole{n}{stat}\DUrole{o}{=}\DUrole{default_value}{'proportion'}}, \emph{\DUrole{n}{figsize}\DUrole{o}{=}\DUrole{default_value}{(12, 8)}}, \emph{\DUrole{n}{title}\DUrole{o}{=}\DUrole{default_value}{''}}, \emph{\DUrole{n}{maxy}\DUrole{o}{=}\DUrole{default_value}{'max'}}, \emph{\DUrole{n}{legend}\DUrole{o}{=}\DUrole{default_value}{'single'}}, \emph{\DUrole{n}{stack}\DUrole{o}{=}\DUrole{default_value}{False}}, \emph{\DUrole{n}{xlabel}\DUrole{o}{=}\DUrole{default_value}{True}}, \emph{\DUrole{n}{error\_bars}\DUrole{o}{=}\DUrole{default_value}{False}}}{}
\sphinxAtStartPar
Plots a comparison\_table from tabulate.resilience\_factor\_comparison as a bar plot for each fault scenario/set of fault scenarios.
\begin{quote}\begin{description}
\item[{Parameters}] \leavevmode\begin{itemize}
\item {} 
\sphinxAtStartPar
\sphinxstyleliteralstrong{\sphinxupquote{comparison\_table}} (\sphinxstyleliteralemphasis{\sphinxupquote{pandas table}}) – Table from tabulate.resilience\_factor\_test with factors as rows and fault scenarios as columns

\item {} 
\sphinxAtStartPar
\sphinxstyleliteralstrong{\sphinxupquote{faults}} (\sphinxstyleliteralemphasis{\sphinxupquote{list}}\sphinxstyleliteralemphasis{\sphinxupquote{, }}\sphinxstyleliteralemphasis{\sphinxupquote{optional}}) – iterable of faults/fault types to include in the bar plot (the columns of the table). The default is ‘all’.
a dictionary \{‘fault’:’title’\} will associate the given fault with a title (otherwise ‘fault’ is used)

\item {} 
\sphinxAtStartPar
\sphinxstyleliteralstrong{\sphinxupquote{rows}} (\sphinxstyleliteralemphasis{\sphinxupquote{int}}\sphinxstyleliteralemphasis{\sphinxupquote{, }}\sphinxstyleliteralemphasis{\sphinxupquote{optional}}) – Number of rows in the multplot. The default is 1.

\item {} 
\sphinxAtStartPar
\sphinxstyleliteralstrong{\sphinxupquote{stat}} (\sphinxstyleliteralemphasis{\sphinxupquote{str}}\sphinxstyleliteralemphasis{\sphinxupquote{, }}\sphinxstyleliteralemphasis{\sphinxupquote{optional}}) – Metric being presented in the table (for the y\sphinxhyphen{}axis). The default is ‘proportion’.

\item {} 
\sphinxAtStartPar
\sphinxstyleliteralstrong{\sphinxupquote{figsize}} (\sphinxstyleliteralemphasis{\sphinxupquote{tuple}}\sphinxstyleliteralemphasis{\sphinxupquote{(}}\sphinxstyleliteralemphasis{\sphinxupquote{int}}\sphinxstyleliteralemphasis{\sphinxupquote{, }}\sphinxstyleliteralemphasis{\sphinxupquote{int}}\sphinxstyleliteralemphasis{\sphinxupquote{)}}\sphinxstyleliteralemphasis{\sphinxupquote{, }}\sphinxstyleliteralemphasis{\sphinxupquote{optional}}) – Size of the figure in (width, height). The default is (12,8).

\item {} 
\sphinxAtStartPar
\sphinxstyleliteralstrong{\sphinxupquote{title}} (\sphinxstyleliteralemphasis{\sphinxupquote{string}}\sphinxstyleliteralemphasis{\sphinxupquote{, }}\sphinxstyleliteralemphasis{\sphinxupquote{optional}}) – Overall title for the plots. The default is ‘’.

\item {} 
\sphinxAtStartPar
\sphinxstyleliteralstrong{\sphinxupquote{maxy}} (\sphinxstyleliteralemphasis{\sphinxupquote{float}}\sphinxstyleliteralemphasis{\sphinxupquote{, }}\sphinxstyleliteralemphasis{\sphinxupquote{optional}}) – Maximum y\sphinxhyphen{}value (to ensure same scale). The default is ‘max’ (finds max value of table).

\item {} 
\sphinxAtStartPar
\sphinxstyleliteralstrong{\sphinxupquote{legend}} (\sphinxstyleliteralemphasis{\sphinxupquote{str}}\sphinxstyleliteralemphasis{\sphinxupquote{, }}\sphinxstyleliteralemphasis{\sphinxupquote{optional}}) – ‘all’/’single’/’none’. The default is “single”.

\item {} 
\sphinxAtStartPar
\sphinxstyleliteralstrong{\sphinxupquote{stack}} (\sphinxstyleliteralemphasis{\sphinxupquote{bool}}\sphinxstyleliteralemphasis{\sphinxupquote{, }}\sphinxstyleliteralemphasis{\sphinxupquote{optional}}) – Whether or not to stack the nominal and resilience plots. The default is False.

\item {} 
\sphinxAtStartPar
\sphinxstyleliteralstrong{\sphinxupquote{xlabel}} (\sphinxstyleliteralemphasis{\sphinxupquote{bool/str}}) – The x\sphinxhyphen{}label descriptor for the design factors. Defaults to the column values.

\item {} 
\sphinxAtStartPar
\sphinxstyleliteralstrong{\sphinxupquote{error\_bars}} (\sphinxstyleliteralemphasis{\sphinxupquote{bool}}) – Whether to include error bars for the factor. Requires comparison\_table to have lower and upper bound information

\end{itemize}

\item[{Returns}] \leavevmode
\sphinxAtStartPar
\sphinxstylestrong{figure} – Plot handle of the figure.

\item[{Return type}] \leavevmode
\sphinxAtStartPar
matplotlib figure

\end{description}\end{quote}

\end{fulllineitems}

\index{samplecost() (in module fmdtools.resultdisp.plot)@\spxentry{samplecost()}\spxextra{in module fmdtools.resultdisp.plot}}

\begin{fulllineitems}
\phantomsection\label{\detokenize{docs/fmdtools.resultdisp:fmdtools.resultdisp.plot.samplecost}}\pysiglinewithargsret{\sphinxcode{\sphinxupquote{fmdtools.resultdisp.plot.}}\sphinxbfcode{\sphinxupquote{samplecost}}}{\emph{\DUrole{n}{app}}, \emph{\DUrole{n}{endclasses}}, \emph{\DUrole{n}{fxnmode}}, \emph{\DUrole{n}{samptype}\DUrole{o}{=}\DUrole{default_value}{'std'}}, \emph{\DUrole{n}{title}\DUrole{o}{=}\DUrole{default_value}{''}}}{}
\sphinxAtStartPar
Plots the sample cost and rate of a given fault over the injection times defined in the app sampleapproach

\sphinxAtStartPar
(note: not currently compatible with joint fault modes)
\begin{quote}\begin{description}
\item[{Parameters}] \leavevmode\begin{itemize}
\item {} 
\sphinxAtStartPar
\sphinxstyleliteralstrong{\sphinxupquote{app}} (\sphinxstyleliteralemphasis{\sphinxupquote{sampleapproach}}) – Sample approach defining the underlying samples to take and probability model of the list of scenarios.

\item {} 
\sphinxAtStartPar
\sphinxstyleliteralstrong{\sphinxupquote{endclasses}} (\sphinxstyleliteralemphasis{\sphinxupquote{dict}}) – A dict with the end classification of each fault (costs, etc)

\item {} 
\sphinxAtStartPar
\sphinxstyleliteralstrong{\sphinxupquote{fxnmode}} (\sphinxstyleliteralemphasis{\sphinxupquote{tuple}}) – tuple (or tuple of tuples) with structure (‘function name’, ‘mode name’) defining the fault mode

\item {} 
\sphinxAtStartPar
\sphinxstyleliteralstrong{\sphinxupquote{samptype}} (\sphinxstyleliteralemphasis{\sphinxupquote{str}}\sphinxstyleliteralemphasis{\sphinxupquote{, }}\sphinxstyleliteralemphasis{\sphinxupquote{optional}}) – \begin{description}
\item[{The type of sample approach used:}] \leavevmode\begin{itemize}
\item {} 
\sphinxAtStartPar
’std’ for a single point for each interval

\item {} 
\sphinxAtStartPar
’quadrature’ for a set of points with weights defined by a quadrature

\item {} 
\sphinxAtStartPar
’pruned piecewise\sphinxhyphen{}linear’ for a set of points with weights defined by a pruned approach (from app.prune\_scenarios())

\item {} 
\sphinxAtStartPar
’fullint’ for the full integral (sampling every possible time)

\end{itemize}

\end{description}


\end{itemize}

\end{description}\end{quote}

\end{fulllineitems}

\index{samplecosts() (in module fmdtools.resultdisp.plot)@\spxentry{samplecosts()}\spxextra{in module fmdtools.resultdisp.plot}}

\begin{fulllineitems}
\phantomsection\label{\detokenize{docs/fmdtools.resultdisp:fmdtools.resultdisp.plot.samplecosts}}\pysiglinewithargsret{\sphinxcode{\sphinxupquote{fmdtools.resultdisp.plot.}}\sphinxbfcode{\sphinxupquote{samplecosts}}}{\emph{\DUrole{n}{app}}, \emph{\DUrole{n}{endclasses}}, \emph{\DUrole{n}{joint}\DUrole{o}{=}\DUrole{default_value}{False}}, \emph{\DUrole{n}{title}\DUrole{o}{=}\DUrole{default_value}{''}}}{}
\sphinxAtStartPar
Plots the costs and rates of a set of faults injected over time according to the approach app
\begin{quote}\begin{description}
\item[{Parameters}] \leavevmode\begin{itemize}
\item {} 
\sphinxAtStartPar
\sphinxstyleliteralstrong{\sphinxupquote{app}} (\sphinxstyleliteralemphasis{\sphinxupquote{sampleapproach}}) – The sample approach used to run the list of faults

\item {} 
\sphinxAtStartPar
\sphinxstyleliteralstrong{\sphinxupquote{endclasses}} (\sphinxstyleliteralemphasis{\sphinxupquote{dict}}) – A dict of results for each of the scenarios.

\item {} 
\sphinxAtStartPar
\sphinxstyleliteralstrong{\sphinxupquote{joint}} (\sphinxstyleliteralemphasis{\sphinxupquote{bool}}\sphinxstyleliteralemphasis{\sphinxupquote{, }}\sphinxstyleliteralemphasis{\sphinxupquote{optional}}) – Whether to include joint fault scenarios. The default is False.

\end{itemize}

\end{description}\end{quote}

\end{fulllineitems}



\subsubsection{fmdtools.resultdisp.process}
\label{\detokenize{docs/fmdtools.resultdisp:module-fmdtools.resultdisp.process}}\label{\detokenize{docs/fmdtools.resultdisp:fmdtools-resultdisp-process}}\index{module@\spxentry{module}!fmdtools.resultdisp.process@\spxentry{fmdtools.resultdisp.process}}\index{fmdtools.resultdisp.process@\spxentry{fmdtools.resultdisp.process}!module@\spxentry{module}}
\sphinxAtStartPar
Description: Processes model results for visualization
\begin{description}
\item[{Uses methods:}] \leavevmode\begin{itemize}
\item {} 
\sphinxAtStartPar
{\hyperref[\detokenize{docs/fmdtools.resultdisp:fmdtools.resultdisp.process.hists}]{\sphinxcrossref{\sphinxcode{\sphinxupquote{hists()}}}}}:                    Processes a model histories for each scenario into results histories by comparing the states over time in each scenario with the states in the nominal scenario.

\item {} \begin{description}
\item[{{\hyperref[\detokenize{docs/fmdtools.resultdisp:fmdtools.resultdisp.process.hist}]{\sphinxcrossref{\sphinxcode{\sphinxupquote{hist()}}}}}:                     Compares model history with the nominal model history over time to make a history of degradation.}] \leavevmode\begin{itemize}
\item {} 
\sphinxAtStartPar
{\hyperref[\detokenize{docs/fmdtools.resultdisp:fmdtools.resultdisp.process.fxnhist}]{\sphinxcrossref{\sphinxcode{\sphinxupquote{fxnhist()}}}}}:              Compares the history of function states in mdlhist over time.

\item {} 
\sphinxAtStartPar
{\hyperref[\detokenize{docs/fmdtools.resultdisp:fmdtools.resultdisp.process.flowhist}]{\sphinxcrossref{\sphinxcode{\sphinxupquote{flowhist()}}}}}:             Compares the history of flow states in mdlhist over time.

\end{itemize}

\end{description}

\item {} 
\sphinxAtStartPar
{\hyperref[\detokenize{docs/fmdtools.resultdisp:fmdtools.resultdisp.process.modephases}]{\sphinxcrossref{\sphinxcode{\sphinxupquote{modephases()}}}}}:               Identifies the phases of operation for the system based on a mdlhist with a history of its modes

\item {} 
\sphinxAtStartPar
{\hyperref[\detokenize{docs/fmdtools.resultdisp:fmdtools.resultdisp.process.graphflows}]{\sphinxcrossref{\sphinxcode{\sphinxupquote{graphflows()}}}}}:               Extracts non\sphinxhyphen{}nominal flows by comparing the a results graph with a nominal results graph.

\item {} 
\sphinxAtStartPar
{\hyperref[\detokenize{docs/fmdtools.resultdisp:fmdtools.resultdisp.process.resultsgraph}]{\sphinxcrossref{\sphinxcode{\sphinxupquote{resultsgraph()}}}}}:         Makes a dict history of results graphs given a dict history of the nominal and faulty graphs

\item {} 
\sphinxAtStartPar
{\hyperref[\detokenize{docs/fmdtools.resultdisp:fmdtools.resultdisp.process.resultsgraphs}]{\sphinxcrossref{\sphinxcode{\sphinxupquote{resultsgraphs()}}}}}:        Makes a dict history of results graphs given a dict history of the nominal and faulty graphs

\item {} 
\sphinxAtStartPar
{\hyperref[\detokenize{docs/fmdtools.resultdisp:fmdtools.resultdisp.process.totalcost}]{\sphinxcrossref{\sphinxcode{\sphinxupquote{totalcost()}}}}}:            Calculates the total host of a set of given end classifications

\item {} 
\sphinxAtStartPar
{\hyperref[\detokenize{docs/fmdtools.resultdisp:fmdtools.resultdisp.process.state_probabilities}]{\sphinxcrossref{\sphinxcode{\sphinxupquote{state\_probabilities()}}}}}:  Calculates the probabilities of given end\sphinxhyphen{}state classifications given an endclasses dictionary

\item {} 
\sphinxAtStartPar
{\hyperref[\detokenize{docs/fmdtools.resultdisp:fmdtools.resultdisp.process.bootstrap_confidence_interval}]{\sphinxcrossref{\sphinxcode{\sphinxupquote{bootstrap\_confidence\_interval()}}}}}: Convenience wrapper for scipy.bootstrap.

\item {} 
\sphinxAtStartPar
{\hyperref[\detokenize{docs/fmdtools.resultdisp:fmdtools.resultdisp.process.overall_diff}]{\sphinxcrossref{\sphinxcode{\sphinxupquote{overall\_diff()}}}}}:         Calculates the difference between the nominal and fault scenarios for a set of nested endclasses

\item {} 
\sphinxAtStartPar
{\hyperref[\detokenize{docs/fmdtools.resultdisp:fmdtools.resultdisp.process.end_diff}]{\sphinxcrossref{\sphinxcode{\sphinxupquote{end\_diff()}}}}}:             Calculates the difference between the nominal and fault scenarios for a set of endclasses

\item {} 
\sphinxAtStartPar
{\hyperref[\detokenize{docs/fmdtools.resultdisp:fmdtools.resultdisp.process.percent}]{\sphinxcrossref{\sphinxcode{\sphinxupquote{percent()}}}}}:              Calculates the percentage of a given indicator variable being True in endclasses

\item {} 
\sphinxAtStartPar
{\hyperref[\detokenize{docs/fmdtools.resultdisp:fmdtools.resultdisp.process.average}]{\sphinxcrossref{\sphinxcode{\sphinxupquote{average()}}}}}:              Calculates the average value of a given metric in endclasses

\item {} 
\sphinxAtStartPar
{\hyperref[\detokenize{docs/fmdtools.resultdisp:fmdtools.resultdisp.process.expected}]{\sphinxcrossref{\sphinxcode{\sphinxupquote{expected()}}}}}:             Calculates the expected value of a given metric in endclasses using the rate variable in endclasses

\item {} 
\sphinxAtStartPar
{\hyperref[\detokenize{docs/fmdtools.resultdisp:fmdtools.resultdisp.process.rate}]{\sphinxcrossref{\sphinxcode{\sphinxupquote{rate()}}}}}:                Calculates the rate of a given indicator variable being True in endclasses using the rate variable in endclasses

\end{itemize}

\item[{Also used for graph heatmaps, which use the results history to map results history statistics onto the graph, returning a dictonary with structure \{fxn/flow: value\}:}] \leavevmode\begin{itemize}
\item {} 
\sphinxAtStartPar
{\hyperref[\detokenize{docs/fmdtools.resultdisp:fmdtools.resultdisp.process.heatmaps}]{\sphinxcrossref{\sphinxcode{\sphinxupquote{heatmaps()}}}}}:            Makes a dict of heatmaps given a results history and a history of the differences between nominal and faulty models.

\item {} 
\sphinxAtStartPar
{\hyperref[\detokenize{docs/fmdtools.resultdisp:fmdtools.resultdisp.process.degtime_heatmap}]{\sphinxcrossref{\sphinxcode{\sphinxupquote{degtime\_heatmap()}}}}}:          Makes a heatmap dictionary of degraded time for functions given a result history

\item {} 
\sphinxAtStartPar
{\hyperref[\detokenize{docs/fmdtools.resultdisp:fmdtools.resultdisp.process.degtime_heatmaps}]{\sphinxcrossref{\sphinxcode{\sphinxupquote{degtime\_heatmaps()}}}}}:         Makes a dict of heatmap dictionaries of degraded time for functions given results histories

\item {} 
\sphinxAtStartPar
{\hyperref[\detokenize{docs/fmdtools.resultdisp:fmdtools.resultdisp.process.avg_degtime_heatmap}]{\sphinxcrossref{\sphinxcode{\sphinxupquote{avg\_degtime\_heatmap()}}}}}:   Makes a heatmap dictionary of the average degraded heat time over a list of scenarios in the dict of results histories.

\item {} 
\sphinxAtStartPar
{\hyperref[\detokenize{docs/fmdtools.resultdisp:fmdtools.resultdisp.process.exp_degtime_heatmap}]{\sphinxcrossref{\sphinxcode{\sphinxupquote{exp\_degtime\_heatmap()}}}}}:   Makes a heatmap dictionary of the expected degraded heat time over a list of scenarios in the dict of results histories based on the rates in endclasses.

\item {} 
\sphinxAtStartPar
{\hyperref[\detokenize{docs/fmdtools.resultdisp:fmdtools.resultdisp.process.fault_heatmap}]{\sphinxcrossref{\sphinxcode{\sphinxupquote{fault\_heatmap()}}}}}:            Makes a heatmap dictionary of faults given a results history.

\item {} 
\sphinxAtStartPar
{\hyperref[\detokenize{docs/fmdtools.resultdisp:fmdtools.resultdisp.process.fault_heatmaps}]{\sphinxcrossref{\sphinxcode{\sphinxupquote{fault\_heatmaps()}}}}}:           Makes dict of heatmaps dictionaries of resulting faults given a results history.

\item {} 
\sphinxAtStartPar
{\hyperref[\detokenize{docs/fmdtools.resultdisp:fmdtools.resultdisp.process.faults_heatmap}]{\sphinxcrossref{\sphinxcode{\sphinxupquote{faults\_heatmap()}}}}}:       Makes a heatmap dictionary of the average resulting faults over all scenarios

\item {} 
\sphinxAtStartPar
{\hyperref[\detokenize{docs/fmdtools.resultdisp:fmdtools.resultdisp.process.exp_faults_heatmap}]{\sphinxcrossref{\sphinxcode{\sphinxupquote{exp\_faults\_heatmap()}}}}}:    Makes a heatmap dictionary of the expected resulting faults over all scenarios

\end{itemize}

\end{description}
\index{average() (in module fmdtools.resultdisp.process)@\spxentry{average()}\spxextra{in module fmdtools.resultdisp.process}}

\begin{fulllineitems}
\phantomsection\label{\detokenize{docs/fmdtools.resultdisp:fmdtools.resultdisp.process.average}}\pysiglinewithargsret{\sphinxcode{\sphinxupquote{fmdtools.resultdisp.process.}}\sphinxbfcode{\sphinxupquote{average}}}{\emph{\DUrole{n}{endclasses}}, \emph{\DUrole{n}{metric}}}{}
\sphinxAtStartPar
Calculates the average value of a given metric in endclasses

\end{fulllineitems}

\index{avg\_degtime\_heatmap() (in module fmdtools.resultdisp.process)@\spxentry{avg\_degtime\_heatmap()}\spxextra{in module fmdtools.resultdisp.process}}

\begin{fulllineitems}
\phantomsection\label{\detokenize{docs/fmdtools.resultdisp:fmdtools.resultdisp.process.avg_degtime_heatmap}}\pysiglinewithargsret{\sphinxcode{\sphinxupquote{fmdtools.resultdisp.process.}}\sphinxbfcode{\sphinxupquote{avg\_degtime\_heatmap}}}{\emph{\DUrole{n}{reshists}}}{}
\sphinxAtStartPar
Makes a heatmap dictionary of the average degraded heat time over a list of scenarios in the dict of results histories.

\end{fulllineitems}

\index{bootstrap\_confidence\_interval() (in module fmdtools.resultdisp.process)@\spxentry{bootstrap\_confidence\_interval()}\spxextra{in module fmdtools.resultdisp.process}}

\begin{fulllineitems}
\phantomsection\label{\detokenize{docs/fmdtools.resultdisp:fmdtools.resultdisp.process.bootstrap_confidence_interval}}\pysiglinewithargsret{\sphinxcode{\sphinxupquote{fmdtools.resultdisp.process.}}\sphinxbfcode{\sphinxupquote{bootstrap\_confidence\_interval}}}{\emph{data}, \emph{method=<function mean>}, \emph{return\_anyway=False}, \emph{**kwargs}}{}
\sphinxAtStartPar
Convenience wrapper for scipy.bootstrap.
\begin{quote}\begin{description}
\item[{Parameters}] \leavevmode\begin{itemize}
\item {} 
\sphinxAtStartPar
\sphinxstyleliteralstrong{\sphinxupquote{data}} (\sphinxstyleliteralemphasis{\sphinxupquote{list/array/etc}}) – Iterable with the data. May be float (for mean) or indicator (for proportion)

\item {} 
\sphinxAtStartPar
\sphinxstyleliteralstrong{\sphinxupquote{method}} – numpy method to give scipy.bootstrap.

\item {} 
\sphinxAtStartPar
\sphinxstyleliteralstrong{\sphinxupquote{return\_anyway}} (\sphinxstyleliteralemphasis{\sphinxupquote{bool}}) – Gives a dummy interval of (stat, stat) if no . Used for plotting

\end{itemize}

\item[{Returns}] \leavevmode
\sphinxAtStartPar


\item[{Return type}] \leavevmode
\sphinxAtStartPar
statistic, lower bound, upper bound

\end{description}\end{quote}

\end{fulllineitems}

\index{degtime\_heatmap() (in module fmdtools.resultdisp.process)@\spxentry{degtime\_heatmap()}\spxextra{in module fmdtools.resultdisp.process}}

\begin{fulllineitems}
\phantomsection\label{\detokenize{docs/fmdtools.resultdisp:fmdtools.resultdisp.process.degtime_heatmap}}\pysiglinewithargsret{\sphinxcode{\sphinxupquote{fmdtools.resultdisp.process.}}\sphinxbfcode{\sphinxupquote{degtime\_heatmap}}}{\emph{\DUrole{n}{reshist}}}{}
\sphinxAtStartPar
Makes a heatmap dictionary of degraded time for functions given a result history

\end{fulllineitems}

\index{degtime\_heatmaps() (in module fmdtools.resultdisp.process)@\spxentry{degtime\_heatmaps()}\spxextra{in module fmdtools.resultdisp.process}}

\begin{fulllineitems}
\phantomsection\label{\detokenize{docs/fmdtools.resultdisp:fmdtools.resultdisp.process.degtime_heatmaps}}\pysiglinewithargsret{\sphinxcode{\sphinxupquote{fmdtools.resultdisp.process.}}\sphinxbfcode{\sphinxupquote{degtime\_heatmaps}}}{\emph{\DUrole{n}{reshists}}}{}
\sphinxAtStartPar
Makes a dict of heatmap dictionaries of degraded time for functions given results histories

\end{fulllineitems}

\index{end\_diff() (in module fmdtools.resultdisp.process)@\spxentry{end\_diff()}\spxextra{in module fmdtools.resultdisp.process}}

\begin{fulllineitems}
\phantomsection\label{\detokenize{docs/fmdtools.resultdisp:fmdtools.resultdisp.process.end_diff}}\pysiglinewithargsret{\sphinxcode{\sphinxupquote{fmdtools.resultdisp.process.}}\sphinxbfcode{\sphinxupquote{end\_diff}}}{\emph{\DUrole{n}{endclasses}}, \emph{\DUrole{n}{metric}}, \emph{\DUrole{n}{nan\_as}\DUrole{o}{=}\DUrole{default_value}{nan}}, \emph{\DUrole{n}{as\_ind}\DUrole{o}{=}\DUrole{default_value}{False}}, \emph{\DUrole{n}{no\_diff}\DUrole{o}{=}\DUrole{default_value}{False}}}{}
\sphinxAtStartPar
Calculates the difference between the nominal and fault scenarios for a set of endclasses
\begin{quote}\begin{description}
\item[{Parameters}] \leavevmode\begin{itemize}
\item {} 
\sphinxAtStartPar
\sphinxstyleliteralstrong{\sphinxupquote{endclasses}} (\sphinxstyleliteralemphasis{\sphinxupquote{dict}}) – endclass dictionary for the set \{scen:endclass\}, where endclass is a dict of metrics

\item {} 
\sphinxAtStartPar
\sphinxstyleliteralstrong{\sphinxupquote{metric}} (\sphinxstyleliteralemphasis{\sphinxupquote{str}}) – metric to calculate the difference of in the endclasses

\item {} 
\sphinxAtStartPar
\sphinxstyleliteralstrong{\sphinxupquote{nan\_as}} (\sphinxstyleliteralemphasis{\sphinxupquote{float}}\sphinxstyleliteralemphasis{\sphinxupquote{, }}\sphinxstyleliteralemphasis{\sphinxupquote{optional}}) – How do deal with nans in the difference. The default is np.nan.

\item {} 
\sphinxAtStartPar
\sphinxstyleliteralstrong{\sphinxupquote{as\_ind}} (\sphinxstyleliteralemphasis{\sphinxupquote{bool}}\sphinxstyleliteralemphasis{\sphinxupquote{, }}\sphinxstyleliteralemphasis{\sphinxupquote{optional}}) – Whether to return the difference as an indicator (1,\sphinxhyphen{}1,0) or real value. The default is False.

\item {} 
\sphinxAtStartPar
\sphinxstyleliteralstrong{\sphinxupquote{no\_diff}} (\sphinxstyleliteralemphasis{\sphinxupquote{bool}}\sphinxstyleliteralemphasis{\sphinxupquote{, }}\sphinxstyleliteralemphasis{\sphinxupquote{optional}}) – Option for not computing the difference (but still performing the processing here). The default is False.

\end{itemize}

\item[{Returns}] \leavevmode
\sphinxAtStartPar
\sphinxstylestrong{difference} – dictionary of differences over the set of scenarios

\item[{Return type}] \leavevmode
\sphinxAtStartPar
dict

\end{description}\end{quote}

\end{fulllineitems}

\index{exp\_degtime\_heatmap() (in module fmdtools.resultdisp.process)@\spxentry{exp\_degtime\_heatmap()}\spxextra{in module fmdtools.resultdisp.process}}

\begin{fulllineitems}
\phantomsection\label{\detokenize{docs/fmdtools.resultdisp:fmdtools.resultdisp.process.exp_degtime_heatmap}}\pysiglinewithargsret{\sphinxcode{\sphinxupquote{fmdtools.resultdisp.process.}}\sphinxbfcode{\sphinxupquote{exp\_degtime\_heatmap}}}{\emph{\DUrole{n}{reshists}}, \emph{\DUrole{n}{endclasses}}}{}
\sphinxAtStartPar
Makes a heatmap dictionary of the expected degraded heat time over a list of scenarios in the dict of results histories based on the rates in endclasses.

\end{fulllineitems}

\index{exp\_faults\_heatmap() (in module fmdtools.resultdisp.process)@\spxentry{exp\_faults\_heatmap()}\spxextra{in module fmdtools.resultdisp.process}}

\begin{fulllineitems}
\phantomsection\label{\detokenize{docs/fmdtools.resultdisp:fmdtools.resultdisp.process.exp_faults_heatmap}}\pysiglinewithargsret{\sphinxcode{\sphinxupquote{fmdtools.resultdisp.process.}}\sphinxbfcode{\sphinxupquote{exp\_faults\_heatmap}}}{\emph{\DUrole{n}{reshists}}, \emph{\DUrole{n}{endclasses}}}{}
\sphinxAtStartPar
Makes a heatmap dictionary of the expected resulting faults over all scenarios

\end{fulllineitems}

\index{expected() (in module fmdtools.resultdisp.process)@\spxentry{expected()}\spxextra{in module fmdtools.resultdisp.process}}

\begin{fulllineitems}
\phantomsection\label{\detokenize{docs/fmdtools.resultdisp:fmdtools.resultdisp.process.expected}}\pysiglinewithargsret{\sphinxcode{\sphinxupquote{fmdtools.resultdisp.process.}}\sphinxbfcode{\sphinxupquote{expected}}}{\emph{\DUrole{n}{endclasses}}, \emph{\DUrole{n}{metric}}}{}
\sphinxAtStartPar
Calculates the expected value of a given metric in endclasses using the rate variable in endclasses

\end{fulllineitems}

\index{fault\_heatmap() (in module fmdtools.resultdisp.process)@\spxentry{fault\_heatmap()}\spxextra{in module fmdtools.resultdisp.process}}

\begin{fulllineitems}
\phantomsection\label{\detokenize{docs/fmdtools.resultdisp:fmdtools.resultdisp.process.fault_heatmap}}\pysiglinewithargsret{\sphinxcode{\sphinxupquote{fmdtools.resultdisp.process.}}\sphinxbfcode{\sphinxupquote{fault\_heatmap}}}{\emph{\DUrole{n}{reshist}}}{}
\sphinxAtStartPar
Makes a heatmap dictionary of faults given a results history.

\end{fulllineitems}

\index{fault\_heatmaps() (in module fmdtools.resultdisp.process)@\spxentry{fault\_heatmaps()}\spxextra{in module fmdtools.resultdisp.process}}

\begin{fulllineitems}
\phantomsection\label{\detokenize{docs/fmdtools.resultdisp:fmdtools.resultdisp.process.fault_heatmaps}}\pysiglinewithargsret{\sphinxcode{\sphinxupquote{fmdtools.resultdisp.process.}}\sphinxbfcode{\sphinxupquote{fault\_heatmaps}}}{\emph{\DUrole{n}{reshists}}}{}
\sphinxAtStartPar
Makes dict of heatmaps dictionaries of resulting faults given a results history.

\end{fulllineitems}

\index{faults\_heatmap() (in module fmdtools.resultdisp.process)@\spxentry{faults\_heatmap()}\spxextra{in module fmdtools.resultdisp.process}}

\begin{fulllineitems}
\phantomsection\label{\detokenize{docs/fmdtools.resultdisp:fmdtools.resultdisp.process.faults_heatmap}}\pysiglinewithargsret{\sphinxcode{\sphinxupquote{fmdtools.resultdisp.process.}}\sphinxbfcode{\sphinxupquote{faults\_heatmap}}}{\emph{\DUrole{n}{reshists}}}{}
\sphinxAtStartPar
Makes a heatmap dictionary of the average resulting faults over all scenarios

\end{fulllineitems}

\index{flowhist() (in module fmdtools.resultdisp.process)@\spxentry{flowhist()}\spxextra{in module fmdtools.resultdisp.process}}

\begin{fulllineitems}
\phantomsection\label{\detokenize{docs/fmdtools.resultdisp:fmdtools.resultdisp.process.flowhist}}\pysiglinewithargsret{\sphinxcode{\sphinxupquote{fmdtools.resultdisp.process.}}\sphinxbfcode{\sphinxupquote{flowhist}}}{\emph{\DUrole{n}{mdlhist}}, \emph{\DUrole{n}{returndiff}\DUrole{o}{=}\DUrole{default_value}{True}}}{}
\sphinxAtStartPar
Compares the history of flow states in mdlhist over time.

\end{fulllineitems}

\index{fxnhist() (in module fmdtools.resultdisp.process)@\spxentry{fxnhist()}\spxextra{in module fmdtools.resultdisp.process}}

\begin{fulllineitems}
\phantomsection\label{\detokenize{docs/fmdtools.resultdisp:fmdtools.resultdisp.process.fxnhist}}\pysiglinewithargsret{\sphinxcode{\sphinxupquote{fmdtools.resultdisp.process.}}\sphinxbfcode{\sphinxupquote{fxnhist}}}{\emph{\DUrole{n}{mdlhist}}, \emph{\DUrole{n}{returndiff}\DUrole{o}{=}\DUrole{default_value}{True}}}{}
\sphinxAtStartPar
Compares the history of function states in mdlhist over time.

\end{fulllineitems}

\index{graphflows() (in module fmdtools.resultdisp.process)@\spxentry{graphflows()}\spxextra{in module fmdtools.resultdisp.process}}

\begin{fulllineitems}
\phantomsection\label{\detokenize{docs/fmdtools.resultdisp:fmdtools.resultdisp.process.graphflows}}\pysiglinewithargsret{\sphinxcode{\sphinxupquote{fmdtools.resultdisp.process.}}\sphinxbfcode{\sphinxupquote{graphflows}}}{\emph{\DUrole{n}{g}}, \emph{\DUrole{n}{nomg}}, \emph{\DUrole{n}{gtype}\DUrole{o}{=}\DUrole{default_value}{'bipartite'}}}{}
\sphinxAtStartPar
Extracts non\sphinxhyphen{}nominal flows by comparing the a results graph with a nominal results graph.
\begin{quote}\begin{description}
\item[{Parameters}] \leavevmode\begin{itemize}
\item {} 
\sphinxAtStartPar
\sphinxstyleliteralstrong{\sphinxupquote{g}} (\sphinxstyleliteralemphasis{\sphinxupquote{networkx graph}}) – The graph in the given fault scenario

\item {} 
\sphinxAtStartPar
\sphinxstyleliteralstrong{\sphinxupquote{nomg}} (\sphinxstyleliteralemphasis{\sphinxupquote{networkx graph}}) – The graph in the nominal fault scenario

\item {} 
\sphinxAtStartPar
\sphinxstyleliteralstrong{\sphinxupquote{gtype}} (\sphinxstyleliteralemphasis{\sphinxupquote{str}}\sphinxstyleliteralemphasis{\sphinxupquote{, }}\sphinxstyleliteralemphasis{\sphinxupquote{optional}}) – The type of graph to return (‘normal’ or ‘bipartite’) The default is ‘bipartite’.

\end{itemize}

\item[{Returns}] \leavevmode
\sphinxAtStartPar
\sphinxstylestrong{endflows} – A dictionary of degraded flows.

\item[{Return type}] \leavevmode
\sphinxAtStartPar
dict

\end{description}\end{quote}

\end{fulllineitems}

\index{heatmaps() (in module fmdtools.resultdisp.process)@\spxentry{heatmaps()}\spxextra{in module fmdtools.resultdisp.process}}

\begin{fulllineitems}
\phantomsection\label{\detokenize{docs/fmdtools.resultdisp:fmdtools.resultdisp.process.heatmaps}}\pysiglinewithargsret{\sphinxcode{\sphinxupquote{fmdtools.resultdisp.process.}}\sphinxbfcode{\sphinxupquote{heatmaps}}}{\emph{\DUrole{n}{reshist}}, \emph{\DUrole{n}{diff}}}{}
\sphinxAtStartPar
Makes a dict of heatmaps given a results history and a history of the differences between nominal and faulty models.
\begin{quote}\begin{description}
\item[{Parameters}] \leavevmode\begin{itemize}
\item {} 
\sphinxAtStartPar
\sphinxstyleliteralstrong{\sphinxupquote{reshist}} (\sphinxstyleliteralemphasis{\sphinxupquote{dict}}) – The model results history (e.g. from compare\_functionhist

\item {} 
\sphinxAtStartPar
\sphinxstyleliteralstrong{\sphinxupquote{diff}} (\sphinxstyleliteralemphasis{\sphinxupquote{dict}}) – The differences (e.g. from compare\_functionhist(s))

\end{itemize}

\item[{Returns}] \leavevmode
\sphinxAtStartPar

\sphinxAtStartPar
\sphinxstylestrong{heatmaps} –
\begin{description}
\item[{A dict of heatmaps based on the results history, including:}] \leavevmode\begin{itemize}
\item {} 
\sphinxAtStartPar
degtime, the time the function/flow was degraded

\item {} 
\sphinxAtStartPar
maxdeg, the maximum degradation experienced by the function

\item {} 
\sphinxAtStartPar
intdeg, the integral of degradation of the function over the time interval

\item {} 
\sphinxAtStartPar
maxfaults, the maximum number of faults in the function

\item {} 
\sphinxAtStartPar
intdiff, the integral of the differences between function/flow states of the nominal and faulty model over time.

\item {} 
\sphinxAtStartPar
maxdiff, the maximum difference between function/flow states of the nominal and faulty model over time.

\end{itemize}

\end{description}


\item[{Return type}] \leavevmode
\sphinxAtStartPar
dict

\end{description}\end{quote}

\end{fulllineitems}

\index{hist() (in module fmdtools.resultdisp.process)@\spxentry{hist()}\spxextra{in module fmdtools.resultdisp.process}}

\begin{fulllineitems}
\phantomsection\label{\detokenize{docs/fmdtools.resultdisp:fmdtools.resultdisp.process.hist}}\pysiglinewithargsret{\sphinxcode{\sphinxupquote{fmdtools.resultdisp.process.}}\sphinxbfcode{\sphinxupquote{hist}}}{\emph{\DUrole{n}{mdlhist}}, \emph{\DUrole{n}{nomhist}\DUrole{o}{=}\DUrole{default_value}{\{\}}}, \emph{\DUrole{n}{returndiff}\DUrole{o}{=}\DUrole{default_value}{True}}}{}
\sphinxAtStartPar
Compares model history with the nominal model history over time to make a history of degradation.
\begin{quote}\begin{description}
\item[{Parameters}] \leavevmode\begin{itemize}
\item {} 
\sphinxAtStartPar
\sphinxstyleliteralstrong{\sphinxupquote{mdlhist}} (\sphinxstyleliteralemphasis{\sphinxupquote{dict}}) – the model fault history or a dict of both the nominal and fault histories \{‘nominal’:nomhist, ‘faulty’:mdlhist\}

\item {} 
\sphinxAtStartPar
\sphinxstyleliteralstrong{\sphinxupquote{nomhist}} (\sphinxstyleliteralemphasis{\sphinxupquote{dict}}\sphinxstyleliteralemphasis{\sphinxupquote{, }}\sphinxstyleliteralemphasis{\sphinxupquote{optional}}) – The model history in the nominal scenario (if not provided in mdlhist) The default is \{\}.

\item {} 
\sphinxAtStartPar
\sphinxstyleliteralstrong{\sphinxupquote{returndiff}} (\sphinxstyleliteralemphasis{\sphinxupquote{bool}}\sphinxstyleliteralemphasis{\sphinxupquote{, }}\sphinxstyleliteralemphasis{\sphinxupquote{optional}}) – Whether to return diffs, a dict of the differences between the values of the states in the nominal scenario and fault scenario. The default is True.

\end{itemize}

\item[{Returns}] \leavevmode
\sphinxAtStartPar
\begin{itemize}
\item {} 
\sphinxAtStartPar
\sphinxstylestrong{reshist} (\sphinxstyleemphasis{dict}) – The results history over time.

\item {} 
\sphinxAtStartPar
\sphinxstylestrong{diff} (\sphinxstyleemphasis{dict}) – The difference between the nominal and fault scenario states (if returndiff is true–otherwise returns empty)

\item {} 
\sphinxAtStartPar
\sphinxstylestrong{summary} (\sphinxstyleemphasis{dict}) – A dict with all degraded functions and degraded flows.

\end{itemize}


\end{description}\end{quote}

\end{fulllineitems}

\index{hists() (in module fmdtools.resultdisp.process)@\spxentry{hists()}\spxextra{in module fmdtools.resultdisp.process}}

\begin{fulllineitems}
\phantomsection\label{\detokenize{docs/fmdtools.resultdisp:fmdtools.resultdisp.process.hists}}\pysiglinewithargsret{\sphinxcode{\sphinxupquote{fmdtools.resultdisp.process.}}\sphinxbfcode{\sphinxupquote{hists}}}{\emph{\DUrole{n}{mdlhists}}, \emph{\DUrole{n}{returndiff}\DUrole{o}{=}\DUrole{default_value}{True}}}{}
\sphinxAtStartPar
Processes a model histories for each scenario into results histories by comparing the states over time in each scenario with the states in the nominal scenario.
\begin{quote}\begin{description}
\item[{Parameters}] \leavevmode\begin{itemize}
\item {} 
\sphinxAtStartPar
\sphinxstyleliteralstrong{\sphinxupquote{mdlhists}} (\sphinxstyleliteralemphasis{\sphinxupquote{dict}}) – A dictionary of model histories for each scenario (e.g. from run\_list or run\_approach)

\item {} 
\sphinxAtStartPar
\sphinxstyleliteralstrong{\sphinxupquote{returndiff}} (\sphinxstyleliteralemphasis{\sphinxupquote{bool}}\sphinxstyleliteralemphasis{\sphinxupquote{, }}\sphinxstyleliteralemphasis{\sphinxupquote{optional}}) – Whether to return diffs, a dict of the differences between the values of the states in the nominal scenario and fault scenario. The default is True.

\end{itemize}

\item[{Returns}] \leavevmode
\sphinxAtStartPar
\begin{itemize}
\item {} 
\sphinxAtStartPar
\sphinxstylestrong{reshists} (\sphinxstyleemphasis{dict}) – A dictionary of the results histories of each scenario over time.

\item {} 
\sphinxAtStartPar
\sphinxstylestrong{diffs} (\sphinxstyleemphasis{dict}) – The difference between the nominal and fault scenario states (if returndiff is true–otherwise returns empty)

\item {} 
\sphinxAtStartPar
\sphinxstylestrong{summaries} (\sphinxstyleemphasis{dict}) – A dict with all degraded functions and degraded flows resulting from the fault scenarios.

\end{itemize}


\end{description}\end{quote}

\end{fulllineitems}

\index{modephases() (in module fmdtools.resultdisp.process)@\spxentry{modephases()}\spxextra{in module fmdtools.resultdisp.process}}

\begin{fulllineitems}
\phantomsection\label{\detokenize{docs/fmdtools.resultdisp:fmdtools.resultdisp.process.modephases}}\pysiglinewithargsret{\sphinxcode{\sphinxupquote{fmdtools.resultdisp.process.}}\sphinxbfcode{\sphinxupquote{modephases}}}{\emph{\DUrole{n}{mdlhist}}}{}
\sphinxAtStartPar
Identifies the phases of operation for the system based on its modes.
\begin{quote}\begin{description}
\item[{Parameters}] \leavevmode
\sphinxAtStartPar
\sphinxstyleliteralstrong{\sphinxupquote{mdlhist}} (\sphinxstyleliteralemphasis{\sphinxupquote{dict}}) – Model history from the nominal run

\item[{Returns}] \leavevmode
\sphinxAtStartPar
\begin{itemize}
\item {} 
\sphinxAtStartPar
\sphinxstylestrong{phases} (\sphinxstyleemphasis{dict}) –
\begin{description}
\item[{Dictionary of distict phases that the system functions pass through, of the form:}] \leavevmode
\sphinxAtStartPar
\{‘fxn’:\{‘phase1’:{[}beg, end{]}, phase2:{[}beg, end{]}\}\}
where each phase is defined by its corresponding mode in the modelhist
(numbered mode, mode1, mode2… for multiple modes)

\end{description}

\item {} 
\sphinxAtStartPar
\sphinxstylestrong{modephases} (\sphinxstyleemphasis{dict}) – Dictionary of phases that the system passes through, of the form: \{‘fxn’:\{‘mode1’:\{‘phase1’, ‘phase2’’\}\}\}

\end{itemize}


\end{description}\end{quote}

\end{fulllineitems}

\index{nan\_to\_x() (in module fmdtools.resultdisp.process)@\spxentry{nan\_to\_x()}\spxextra{in module fmdtools.resultdisp.process}}

\begin{fulllineitems}
\phantomsection\label{\detokenize{docs/fmdtools.resultdisp:fmdtools.resultdisp.process.nan_to_x}}\pysiglinewithargsret{\sphinxcode{\sphinxupquote{fmdtools.resultdisp.process.}}\sphinxbfcode{\sphinxupquote{nan\_to\_x}}}{\emph{\DUrole{n}{metric}}, \emph{\DUrole{n}{x}\DUrole{o}{=}\DUrole{default_value}{0.0}}}{}
\sphinxAtStartPar
returns nan as zero if present, otherwise returns the number

\end{fulllineitems}

\index{overall\_diff() (in module fmdtools.resultdisp.process)@\spxentry{overall\_diff()}\spxextra{in module fmdtools.resultdisp.process}}

\begin{fulllineitems}
\phantomsection\label{\detokenize{docs/fmdtools.resultdisp:fmdtools.resultdisp.process.overall_diff}}\pysiglinewithargsret{\sphinxcode{\sphinxupquote{fmdtools.resultdisp.process.}}\sphinxbfcode{\sphinxupquote{overall\_diff}}}{\emph{\DUrole{n}{nested\_endclasses}}, \emph{\DUrole{n}{metric}}, \emph{\DUrole{n}{nan\_as}\DUrole{o}{=}\DUrole{default_value}{nan}}, \emph{\DUrole{n}{as\_ind}\DUrole{o}{=}\DUrole{default_value}{False}}, \emph{\DUrole{n}{no\_diff}\DUrole{o}{=}\DUrole{default_value}{False}}}{}
\sphinxAtStartPar
Calculates the difference between the nominal and fault scenarios over a set of endclasses
\begin{quote}\begin{description}
\item[{Parameters}] \leavevmode\begin{itemize}
\item {} 
\sphinxAtStartPar
\sphinxstyleliteralstrong{\sphinxupquote{nested\_endclasses}} (\sphinxstyleliteralemphasis{\sphinxupquote{dict}}) – Nested dict of endclasses from propogate.nested

\item {} 
\sphinxAtStartPar
\sphinxstyleliteralstrong{\sphinxupquote{metric}} (\sphinxstyleliteralemphasis{\sphinxupquote{str}}) – metric to calculate the difference of in the endclasses

\item {} 
\sphinxAtStartPar
\sphinxstyleliteralstrong{\sphinxupquote{nan\_as}} (\sphinxstyleliteralemphasis{\sphinxupquote{float}}\sphinxstyleliteralemphasis{\sphinxupquote{, }}\sphinxstyleliteralemphasis{\sphinxupquote{optional}}) – How do deal with nans in the difference. The default is np.nan.

\item {} 
\sphinxAtStartPar
\sphinxstyleliteralstrong{\sphinxupquote{as\_ind}} (\sphinxstyleliteralemphasis{\sphinxupquote{bool}}\sphinxstyleliteralemphasis{\sphinxupquote{, }}\sphinxstyleliteralemphasis{\sphinxupquote{optional}}) – Whether to return the difference as an indicator (1,\sphinxhyphen{}1,0) or real value. The default is False.

\item {} 
\sphinxAtStartPar
\sphinxstyleliteralstrong{\sphinxupquote{no\_diff}} (\sphinxstyleliteralemphasis{\sphinxupquote{bool}}\sphinxstyleliteralemphasis{\sphinxupquote{, }}\sphinxstyleliteralemphasis{\sphinxupquote{optional}}) – Option for not computing the difference (but still performing the processing here). The default is False.

\end{itemize}

\item[{Returns}] \leavevmode
\sphinxAtStartPar
\sphinxstylestrong{differences} – nested dictionary of differences over the set of fault scenarios nested in nominal scenarios

\item[{Return type}] \leavevmode
\sphinxAtStartPar
dict

\end{description}\end{quote}

\end{fulllineitems}

\index{percent() (in module fmdtools.resultdisp.process)@\spxentry{percent()}\spxextra{in module fmdtools.resultdisp.process}}

\begin{fulllineitems}
\phantomsection\label{\detokenize{docs/fmdtools.resultdisp:fmdtools.resultdisp.process.percent}}\pysiglinewithargsret{\sphinxcode{\sphinxupquote{fmdtools.resultdisp.process.}}\sphinxbfcode{\sphinxupquote{percent}}}{\emph{\DUrole{n}{endclasses}}, \emph{\DUrole{n}{metric}}}{}
\sphinxAtStartPar
Calculates the percentage of a given indicator variable being True in endclasses

\end{fulllineitems}

\index{rate() (in module fmdtools.resultdisp.process)@\spxentry{rate()}\spxextra{in module fmdtools.resultdisp.process}}

\begin{fulllineitems}
\phantomsection\label{\detokenize{docs/fmdtools.resultdisp:fmdtools.resultdisp.process.rate}}\pysiglinewithargsret{\sphinxcode{\sphinxupquote{fmdtools.resultdisp.process.}}\sphinxbfcode{\sphinxupquote{rate}}}{\emph{\DUrole{n}{endclasses}}, \emph{\DUrole{n}{metric}}}{}
\sphinxAtStartPar
Calculates the rate of a given indicator variable being True in endclasses using the rate variable in endclasses

\end{fulllineitems}

\index{resultsgraph() (in module fmdtools.resultdisp.process)@\spxentry{resultsgraph()}\spxextra{in module fmdtools.resultdisp.process}}

\begin{fulllineitems}
\phantomsection\label{\detokenize{docs/fmdtools.resultdisp:fmdtools.resultdisp.process.resultsgraph}}\pysiglinewithargsret{\sphinxcode{\sphinxupquote{fmdtools.resultdisp.process.}}\sphinxbfcode{\sphinxupquote{resultsgraph}}}{\emph{\DUrole{n}{g}}, \emph{\DUrole{n}{nomg}}, \emph{\DUrole{n}{gtype}\DUrole{o}{=}\DUrole{default_value}{'bipartite'}}}{}
\sphinxAtStartPar
Makes a graph of nominal/non\sphinxhyphen{}nominal states by comparing the nominal graph states with the non\sphinxhyphen{}nominal graph states
\begin{quote}\begin{description}
\item[{Parameters}] \leavevmode\begin{itemize}
\item {} 
\sphinxAtStartPar
\sphinxstyleliteralstrong{\sphinxupquote{g}} (\sphinxstyleliteralemphasis{\sphinxupquote{networkx Graph}}) – graph for the fault scenario where the functions are nodes and flows are edges and with ‘faults’ and ‘states’ attributes

\item {} 
\sphinxAtStartPar
\sphinxstyleliteralstrong{\sphinxupquote{nomg}} (\sphinxstyleliteralemphasis{\sphinxupquote{networkx Graph}}) – graph for the nominal scenario where the functions are nodes and flows are edges and with ‘faults’ and ‘states’ attributes

\item {} 
\sphinxAtStartPar
\sphinxstyleliteralstrong{\sphinxupquote{gtype}} (\sphinxstyleliteralemphasis{\sphinxupquote{'normal'}}\sphinxstyleliteralemphasis{\sphinxupquote{ or }}\sphinxstyleliteralemphasis{\sphinxupquote{'bipartite'}}) – whether the graph is a normal multgraph, or a bipartite graph. the default is ‘bipartite’

\end{itemize}

\item[{Returns}] \leavevmode
\sphinxAtStartPar
\sphinxstylestrong{rg} – copy of g with ‘status’ attributes added for faulty/degraded functions/flows

\item[{Return type}] \leavevmode
\sphinxAtStartPar
networkx graph

\end{description}\end{quote}

\end{fulllineitems}

\index{resultsgraphs() (in module fmdtools.resultdisp.process)@\spxentry{resultsgraphs()}\spxextra{in module fmdtools.resultdisp.process}}

\begin{fulllineitems}
\phantomsection\label{\detokenize{docs/fmdtools.resultdisp:fmdtools.resultdisp.process.resultsgraphs}}\pysiglinewithargsret{\sphinxcode{\sphinxupquote{fmdtools.resultdisp.process.}}\sphinxbfcode{\sphinxupquote{resultsgraphs}}}{\emph{\DUrole{n}{ghist}}, \emph{\DUrole{n}{nomghist}}, \emph{\DUrole{n}{gtype}\DUrole{o}{=}\DUrole{default_value}{'bipartite'}}}{}
\sphinxAtStartPar
Makes a dict history of results graphs given a dict history of the nominal and faulty graphs
\begin{quote}\begin{description}
\item[{Parameters}] \leavevmode\begin{itemize}
\item {} 
\sphinxAtStartPar
\sphinxstyleliteralstrong{\sphinxupquote{ghist}} (\sphinxstyleliteralemphasis{\sphinxupquote{dict}}) – dict history of the faulty graph

\item {} 
\sphinxAtStartPar
\sphinxstyleliteralstrong{\sphinxupquote{nomghist}} (\sphinxstyleliteralemphasis{\sphinxupquote{dict}}) – dict history of the nominal graph

\item {} 
\sphinxAtStartPar
\sphinxstyleliteralstrong{\sphinxupquote{gtype}} (\sphinxstyleliteralemphasis{\sphinxupquote{str}}\sphinxstyleliteralemphasis{\sphinxupquote{, }}\sphinxstyleliteralemphasis{\sphinxupquote{optional}}) – Type of graph provided/returned (bipartite, component, or normal). The default is ‘bipartite’.

\end{itemize}

\item[{Returns}] \leavevmode
\sphinxAtStartPar
\sphinxstylestrong{rghist} – dict history of results graphs

\item[{Return type}] \leavevmode
\sphinxAtStartPar
dict

\end{description}\end{quote}

\end{fulllineitems}

\index{state\_probabilities() (in module fmdtools.resultdisp.process)@\spxentry{state\_probabilities()}\spxextra{in module fmdtools.resultdisp.process}}

\begin{fulllineitems}
\phantomsection\label{\detokenize{docs/fmdtools.resultdisp:fmdtools.resultdisp.process.state_probabilities}}\pysiglinewithargsret{\sphinxcode{\sphinxupquote{fmdtools.resultdisp.process.}}\sphinxbfcode{\sphinxupquote{state\_probabilities}}}{\emph{\DUrole{n}{endclasses}}}{}
\sphinxAtStartPar
Tabulates the probabilities of different states in endclasses.
\begin{quote}\begin{description}
\item[{Parameters}] \leavevmode
\sphinxAtStartPar
\sphinxstyleliteralstrong{\sphinxupquote{endclasses}} (\sphinxstyleliteralemphasis{\sphinxupquote{dict}}) – Dictionary of end\sphinxhyphen{}state classifications ‘classification’ and ‘prob’ attributes

\item[{Returns}] \leavevmode
\sphinxAtStartPar
\sphinxstylestrong{probabilities} – Dictionary of probabilities of different simulation classifications

\item[{Return type}] \leavevmode
\sphinxAtStartPar
dict

\end{description}\end{quote}

\end{fulllineitems}

\index{totalcost() (in module fmdtools.resultdisp.process)@\spxentry{totalcost()}\spxextra{in module fmdtools.resultdisp.process}}

\begin{fulllineitems}
\phantomsection\label{\detokenize{docs/fmdtools.resultdisp:fmdtools.resultdisp.process.totalcost}}\pysiglinewithargsret{\sphinxcode{\sphinxupquote{fmdtools.resultdisp.process.}}\sphinxbfcode{\sphinxupquote{totalcost}}}{\emph{\DUrole{n}{endclasses}}}{}
\sphinxAtStartPar
Tabulates the total expected cost of given endlcasses from a run.
\begin{quote}\begin{description}
\item[{Parameters}] \leavevmode
\sphinxAtStartPar
\sphinxstyleliteralstrong{\sphinxupquote{endclasses}} (\sphinxstyleliteralemphasis{\sphinxupquote{dict}}) – Dictionary of end\sphinxhyphen{}state classifications with ‘expected cost’ attributes

\item[{Returns}] \leavevmode
\sphinxAtStartPar
\sphinxstylestrong{totalcost} – The total expected cost of the scenarios.

\item[{Return type}] \leavevmode
\sphinxAtStartPar
Float

\end{description}\end{quote}

\end{fulllineitems}

\index{typehist() (in module fmdtools.resultdisp.process)@\spxentry{typehist()}\spxextra{in module fmdtools.resultdisp.process}}

\begin{fulllineitems}
\phantomsection\label{\detokenize{docs/fmdtools.resultdisp:fmdtools.resultdisp.process.typehist}}\pysiglinewithargsret{\sphinxcode{\sphinxupquote{fmdtools.resultdisp.process.}}\sphinxbfcode{\sphinxupquote{typehist}}}{\emph{\DUrole{n}{mdl}}, \emph{\DUrole{n}{reshist}}}{}
\sphinxAtStartPar
Summarizes results history reshist over model classes
\begin{quote}\begin{description}
\item[{Parameters}] \leavevmode\begin{itemize}
\item {} 
\sphinxAtStartPar
\sphinxstyleliteralstrong{\sphinxupquote{mdl}} ({\hyperref[\detokenize{docs/fmdtools:fmdtools.modeldef.Model}]{\sphinxcrossref{\sphinxstyleliteralemphasis{\sphinxupquote{Model}}}}}) – Model used in the simulation

\item {} 
\sphinxAtStartPar
\sphinxstyleliteralstrong{\sphinxupquote{reshist}} (\sphinxstyleliteralemphasis{\sphinxupquote{Dict}}) – Results history from rd.process.hist(mdlhist)

\end{itemize}

\item[{Returns}] \leavevmode
\sphinxAtStartPar

\sphinxAtStartPar
\sphinxstylestrong{typehist} –
\begin{description}
\item[{Results history of flow types/function classes with structure:}] \leavevmode
\sphinxAtStartPar
\{‘functions’:\{‘status’:{[}{]},’faults’:\{fxn1:{[}{]}, fxn2:{[}{]}\},’numfaults’:{[}{]}\}, ‘flows’:{[}{]}, ‘flowvals’\{‘flow1’:{[}{]}, ‘flow2’:{[}{]}\}\}

\end{description}


\item[{Return type}] \leavevmode
\sphinxAtStartPar
Dict

\end{description}\end{quote}

\end{fulllineitems}



\subsubsection{fmdtools.resultdisp.tabulate}
\label{\detokenize{docs/fmdtools.resultdisp:module-fmdtools.resultdisp.tabulate}}\label{\detokenize{docs/fmdtools.resultdisp:fmdtools-resultdisp-tabulate}}\index{module@\spxentry{module}!fmdtools.resultdisp.tabulate@\spxentry{fmdtools.resultdisp.tabulate}}\index{fmdtools.resultdisp.tabulate@\spxentry{fmdtools.resultdisp.tabulate}!module@\spxentry{module}}
\sphinxAtStartPar
Description: Translates simulation outputs to pandas tables for display, export, etc.
\begin{description}
\item[{Uses methods:}] \leavevmode\begin{itemize}
\item {} 
\sphinxAtStartPar
{\hyperref[\detokenize{docs/fmdtools.resultdisp:fmdtools.resultdisp.tabulate.hist}]{\sphinxcrossref{\sphinxcode{\sphinxupquote{hist()}}}}}:           Returns formatted pandas dataframe of model history

\item {} 
\sphinxAtStartPar
{\hyperref[\detokenize{docs/fmdtools.resultdisp:fmdtools.resultdisp.tabulate.objtab}]{\sphinxcrossref{\sphinxcode{\sphinxupquote{objtab()}}}}}:         Make table of function OR flow value attributes \sphinxhyphen{} objtype = ‘function’ or ‘flow’

\item {} 
\sphinxAtStartPar
{\hyperref[\detokenize{docs/fmdtools.resultdisp:fmdtools.resultdisp.tabulate.stats}]{\sphinxcrossref{\sphinxcode{\sphinxupquote{stats()}}}}}:          Makes a table of \#of degraded flows, \# of degraded functions, and \# of total faults over time given a single result history

\item {} 
\sphinxAtStartPar
{\hyperref[\detokenize{docs/fmdtools.resultdisp:fmdtools.resultdisp.tabulate.degflows}]{\sphinxcrossref{\sphinxcode{\sphinxupquote{degflows()}}}}}:       Makes a  of flows over time, where 0 is degraded and 1 is nominal

\item {} 
\sphinxAtStartPar
{\hyperref[\detokenize{docs/fmdtools.resultdisp:fmdtools.resultdisp.tabulate.degflowvals}]{\sphinxcrossref{\sphinxcode{\sphinxupquote{degflowvals()}}}}}:    Makes a table of individual flow state values over time, where 0 is degraded and 1 is nominal

\item {} 
\sphinxAtStartPar
{\hyperref[\detokenize{docs/fmdtools.resultdisp:fmdtools.resultdisp.tabulate.degfxns}]{\sphinxcrossref{\sphinxcode{\sphinxupquote{degfxns()}}}}}:        Makes a table showing which functions are degraded over time (0 for degraded, 1 for nominal)

\item {} 
\sphinxAtStartPar
{\hyperref[\detokenize{docs/fmdtools.resultdisp:fmdtools.resultdisp.tabulate.deghist}]{\sphinxcrossref{\sphinxcode{\sphinxupquote{deghist()}}}}}:        Makes a table of all funcitons and flows that are degraded over time. If withstats=True, the total \# of each type degraded is provided in the last columns

\item {} 
\sphinxAtStartPar
{\hyperref[\detokenize{docs/fmdtools.resultdisp:fmdtools.resultdisp.tabulate.heatmaps}]{\sphinxcrossref{\sphinxcode{\sphinxupquote{heatmaps()}}}}}:       Makes a table of a heatmap dictionary

\item {} 
\sphinxAtStartPar
{\hyperref[\detokenize{docs/fmdtools.resultdisp:fmdtools.resultdisp.tabulate.costovertime}]{\sphinxcrossref{\sphinxcode{\sphinxupquote{costovertime()}}}}}:   Makes a table of the total cost, rate, and expected cost of all faults over time

\item {} 
\sphinxAtStartPar
{\hyperref[\detokenize{docs/fmdtools.resultdisp:fmdtools.resultdisp.tabulate.samptime}]{\sphinxcrossref{\sphinxcode{\sphinxupquote{samptime()}}}}}:       Makes a table of the times sampled for each phase given a dict (i.e. app.sampletimes)

\item {} 
\sphinxAtStartPar
\sphinxcode{\sphinxupquote{summary:()}}        Makes a table of a summary dictionary from a given model run

\item {} 
\sphinxAtStartPar
{\hyperref[\detokenize{docs/fmdtools.resultdisp:fmdtools.resultdisp.tabulate.result}]{\sphinxcrossref{\sphinxcode{\sphinxupquote{result()}}}}}:         Makes a table of results (degraded functions/flows, cost, rate, expected cost) of a single run

\item {} 
\sphinxAtStartPar
{\hyperref[\detokenize{docs/fmdtools.resultdisp:fmdtools.resultdisp.tabulate.dicttab}]{\sphinxcrossref{\sphinxcode{\sphinxupquote{dicttab()}}}}}:           Makes table of a generic dictionary

\item {} 
\sphinxAtStartPar
{\hyperref[\detokenize{docs/fmdtools.resultdisp:fmdtools.resultdisp.tabulate.maptab}]{\sphinxcrossref{\sphinxcode{\sphinxupquote{maptab()}}}}}:            Makes table of a generic map

\item {} 
\sphinxAtStartPar
{\hyperref[\detokenize{docs/fmdtools.resultdisp:fmdtools.resultdisp.tabulate.nominal_stats}]{\sphinxcrossref{\sphinxcode{\sphinxupquote{nominal\_stats()}}}}}:  Makes a table of quantities of interest from endclasses from a nominal approach.

\item {} 
\sphinxAtStartPar
{\hyperref[\detokenize{docs/fmdtools.resultdisp:fmdtools.resultdisp.tabulate.nested_stats}]{\sphinxcrossref{\sphinxcode{\sphinxupquote{nested\_stats()}}}}}:   Makes a table of quantities of interest from endclasses from a nested approach.

\item {} 
\sphinxAtStartPar
{\hyperref[\detokenize{docs/fmdtools.resultdisp:fmdtools.resultdisp.tabulate.nominal_factor_comparison}]{\sphinxcrossref{\sphinxcode{\sphinxupquote{nominal\_factor\_comparison()}}}}}: Compares a metric for a given set of model parameters/factors over a set of nominal scenarios.

\item {} 
\sphinxAtStartPar
\sphinxcode{\sphinxupquote{nested\_factor\_comparison()}}: Compares a metric for a given set of model parameters/factors over a nested set of nominal and fault scenarios.

\end{itemize}

\item[{Also used for FMEA\sphinxhyphen{}like tables:}] \leavevmode\begin{itemize}
\item {} 
\sphinxAtStartPar
{\hyperref[\detokenize{docs/fmdtools.resultdisp:fmdtools.resultdisp.tabulate.simplefmea}]{\sphinxcrossref{\sphinxcode{\sphinxupquote{simplefmea()}}}}}:          Makes a simple fmea (rate, cost, expected cost) of the endclasses of a list of fault scenarios run

\item {} 
\sphinxAtStartPar
{\hyperref[\detokenize{docs/fmdtools.resultdisp:fmdtools.resultdisp.tabulate.phasefmea}]{\sphinxcrossref{\sphinxcode{\sphinxupquote{phasefmea()}}}}}:           Makes a simple fmea of the endclasses of a set of fault scenarios run grouped by phase.

\item {} 
\sphinxAtStartPar
{\hyperref[\detokenize{docs/fmdtools.resultdisp:fmdtools.resultdisp.tabulate.summfmea}]{\sphinxcrossref{\sphinxcode{\sphinxupquote{summfmea()}}}}}:            Makes a simple fmea of the endclasses of a set of fault scenarios run grouped by fault.

\item {} 
\sphinxAtStartPar
{\hyperref[\detokenize{docs/fmdtools.resultdisp:fmdtools.resultdisp.tabulate.fullfmea}]{\sphinxcrossref{\sphinxcode{\sphinxupquote{fullfmea()}}}}}:            Makes full fmea table (degraded functions/flows, cost, rate, expected cost) of scenarios given endclasses dict (cost, rate, expected cost) and summaries dict (degraded functions, degraded flows)

\end{itemize}

\end{description}
\index{costovertime() (in module fmdtools.resultdisp.tabulate)@\spxentry{costovertime()}\spxextra{in module fmdtools.resultdisp.tabulate}}

\begin{fulllineitems}
\phantomsection\label{\detokenize{docs/fmdtools.resultdisp:fmdtools.resultdisp.tabulate.costovertime}}\pysiglinewithargsret{\sphinxcode{\sphinxupquote{fmdtools.resultdisp.tabulate.}}\sphinxbfcode{\sphinxupquote{costovertime}}}{\emph{\DUrole{n}{endclasses}}, \emph{\DUrole{n}{app}}}{}
\sphinxAtStartPar
Makes a table of the total cost, rate, and expected cost of all faults over time
\begin{quote}\begin{description}
\item[{Parameters}] \leavevmode\begin{itemize}
\item {} 
\sphinxAtStartPar
\sphinxstyleliteralstrong{\sphinxupquote{endclasses}} (\sphinxstyleliteralemphasis{\sphinxupquote{dict}}) – dict with rate,cost, and expected cost for each injected scenario

\item {} 
\sphinxAtStartPar
\sphinxstyleliteralstrong{\sphinxupquote{app}} (\sphinxstyleliteralemphasis{\sphinxupquote{sampleapproach}}) – sample approach used to generate the list of scenarios

\end{itemize}

\item[{Returns}] \leavevmode
\sphinxAtStartPar
\sphinxstylestrong{costovertime} – pandas dataframe with the total cost, rate, and expected cost for the set of scenarios

\item[{Return type}] \leavevmode
\sphinxAtStartPar
dataframe

\end{description}\end{quote}

\end{fulllineitems}

\index{degflows() (in module fmdtools.resultdisp.tabulate)@\spxentry{degflows()}\spxextra{in module fmdtools.resultdisp.tabulate}}

\begin{fulllineitems}
\phantomsection\label{\detokenize{docs/fmdtools.resultdisp:fmdtools.resultdisp.tabulate.degflows}}\pysiglinewithargsret{\sphinxcode{\sphinxupquote{fmdtools.resultdisp.tabulate.}}\sphinxbfcode{\sphinxupquote{degflows}}}{\emph{\DUrole{n}{reshist}}}{}
\sphinxAtStartPar
Makes a table of flows over time, where 0 is degraded and 1 is nominal

\end{fulllineitems}

\index{degflowvals() (in module fmdtools.resultdisp.tabulate)@\spxentry{degflowvals()}\spxextra{in module fmdtools.resultdisp.tabulate}}

\begin{fulllineitems}
\phantomsection\label{\detokenize{docs/fmdtools.resultdisp:fmdtools.resultdisp.tabulate.degflowvals}}\pysiglinewithargsret{\sphinxcode{\sphinxupquote{fmdtools.resultdisp.tabulate.}}\sphinxbfcode{\sphinxupquote{degflowvals}}}{\emph{\DUrole{n}{reshist}}}{}
\sphinxAtStartPar
Makes a table of individual flow state values over time, where 0 is degraded and 1 is nominal

\end{fulllineitems}

\index{degfxns() (in module fmdtools.resultdisp.tabulate)@\spxentry{degfxns()}\spxextra{in module fmdtools.resultdisp.tabulate}}

\begin{fulllineitems}
\phantomsection\label{\detokenize{docs/fmdtools.resultdisp:fmdtools.resultdisp.tabulate.degfxns}}\pysiglinewithargsret{\sphinxcode{\sphinxupquote{fmdtools.resultdisp.tabulate.}}\sphinxbfcode{\sphinxupquote{degfxns}}}{\emph{\DUrole{n}{reshist}}}{}
\sphinxAtStartPar
Makes a table showing which functions are degraded over time (0 for degraded, 1 for nominal)

\end{fulllineitems}

\index{deghist() (in module fmdtools.resultdisp.tabulate)@\spxentry{deghist()}\spxextra{in module fmdtools.resultdisp.tabulate}}

\begin{fulllineitems}
\phantomsection\label{\detokenize{docs/fmdtools.resultdisp:fmdtools.resultdisp.tabulate.deghist}}\pysiglinewithargsret{\sphinxcode{\sphinxupquote{fmdtools.resultdisp.tabulate.}}\sphinxbfcode{\sphinxupquote{deghist}}}{\emph{\DUrole{n}{reshist}}, \emph{\DUrole{n}{withstats}\DUrole{o}{=}\DUrole{default_value}{False}}}{}
\sphinxAtStartPar
Makes a table of all funcitons and flows that are degraded over time. If withstats=True, the total \# of each type degraded is provided in the last columns

\end{fulllineitems}

\index{dicttab() (in module fmdtools.resultdisp.tabulate)@\spxentry{dicttab()}\spxextra{in module fmdtools.resultdisp.tabulate}}

\begin{fulllineitems}
\phantomsection\label{\detokenize{docs/fmdtools.resultdisp:fmdtools.resultdisp.tabulate.dicttab}}\pysiglinewithargsret{\sphinxcode{\sphinxupquote{fmdtools.resultdisp.tabulate.}}\sphinxbfcode{\sphinxupquote{dicttab}}}{\emph{\DUrole{n}{dictionary}}}{}
\sphinxAtStartPar
Makes table of a generic dictionary

\end{fulllineitems}

\index{fullfmea() (in module fmdtools.resultdisp.tabulate)@\spxentry{fullfmea()}\spxextra{in module fmdtools.resultdisp.tabulate}}

\begin{fulllineitems}
\phantomsection\label{\detokenize{docs/fmdtools.resultdisp:fmdtools.resultdisp.tabulate.fullfmea}}\pysiglinewithargsret{\sphinxcode{\sphinxupquote{fmdtools.resultdisp.tabulate.}}\sphinxbfcode{\sphinxupquote{fullfmea}}}{\emph{\DUrole{n}{endclasses}}, \emph{\DUrole{n}{summaries}}}{}
\sphinxAtStartPar
Makes full fmea table (degraded functions/flows, cost, rate, expected cost) of scenarios given endclasses dict (cost, rate, expected cost) and summaries dict (degraded functions, degraded flows)

\end{fulllineitems}

\index{heatmaps() (in module fmdtools.resultdisp.tabulate)@\spxentry{heatmaps()}\spxextra{in module fmdtools.resultdisp.tabulate}}

\begin{fulllineitems}
\phantomsection\label{\detokenize{docs/fmdtools.resultdisp:fmdtools.resultdisp.tabulate.heatmaps}}\pysiglinewithargsret{\sphinxcode{\sphinxupquote{fmdtools.resultdisp.tabulate.}}\sphinxbfcode{\sphinxupquote{heatmaps}}}{\emph{\DUrole{n}{heatmaps}}}{}
\sphinxAtStartPar
Makes a table of a heatmap dictionary

\end{fulllineitems}

\index{hist() (in module fmdtools.resultdisp.tabulate)@\spxentry{hist()}\spxextra{in module fmdtools.resultdisp.tabulate}}

\begin{fulllineitems}
\phantomsection\label{\detokenize{docs/fmdtools.resultdisp:fmdtools.resultdisp.tabulate.hist}}\pysiglinewithargsret{\sphinxcode{\sphinxupquote{fmdtools.resultdisp.tabulate.}}\sphinxbfcode{\sphinxupquote{hist}}}{\emph{\DUrole{n}{mdlhist}}}{}
\sphinxAtStartPar
Returns formatted pandas dataframe of model history

\end{fulllineitems}

\index{maptab() (in module fmdtools.resultdisp.tabulate)@\spxentry{maptab()}\spxextra{in module fmdtools.resultdisp.tabulate}}

\begin{fulllineitems}
\phantomsection\label{\detokenize{docs/fmdtools.resultdisp:fmdtools.resultdisp.tabulate.maptab}}\pysiglinewithargsret{\sphinxcode{\sphinxupquote{fmdtools.resultdisp.tabulate.}}\sphinxbfcode{\sphinxupquote{maptab}}}{\emph{\DUrole{n}{mapping}}}{}
\sphinxAtStartPar
Makes table of a generic map

\end{fulllineitems}

\index{nested\_stats() (in module fmdtools.resultdisp.tabulate)@\spxentry{nested\_stats()}\spxextra{in module fmdtools.resultdisp.tabulate}}

\begin{fulllineitems}
\phantomsection\label{\detokenize{docs/fmdtools.resultdisp:fmdtools.resultdisp.tabulate.nested_stats}}\pysiglinewithargsret{\sphinxcode{\sphinxupquote{fmdtools.resultdisp.tabulate.}}\sphinxbfcode{\sphinxupquote{nested\_stats}}}{\emph{\DUrole{n}{nomapp}}, \emph{\DUrole{n}{nested\_endclasses}}, \emph{\DUrole{n}{percent\_metrics}\DUrole{o}{=}\DUrole{default_value}{{[}{]}}}, \emph{\DUrole{n}{rate\_metrics}\DUrole{o}{=}\DUrole{default_value}{{[}{]}}}, \emph{\DUrole{n}{average\_metrics}\DUrole{o}{=}\DUrole{default_value}{{[}{]}}}, \emph{\DUrole{n}{expected\_metrics}\DUrole{o}{=}\DUrole{default_value}{{[}{]}}}, \emph{\DUrole{n}{inputparams}\DUrole{o}{=}\DUrole{default_value}{'from\_range'}}, \emph{\DUrole{n}{scenarios}\DUrole{o}{=}\DUrole{default_value}{'all'}}}{}
\sphinxAtStartPar
Makes a table of quantities of interest from endclasses.
\begin{quote}\begin{description}
\item[{Parameters}] \leavevmode\begin{itemize}
\item {} 
\sphinxAtStartPar
\sphinxstyleliteralstrong{\sphinxupquote{nomapp}} ({\hyperref[\detokenize{docs/fmdtools:fmdtools.modeldef.NominalApproach}]{\sphinxcrossref{\sphinxstyleliteralemphasis{\sphinxupquote{NominalApproach}}}}}) – NominalApproach used to generate the simulation.

\item {} 
\sphinxAtStartPar
\sphinxstyleliteralstrong{\sphinxupquote{endclasses}} (\sphinxstyleliteralemphasis{\sphinxupquote{dict}}) – End\sphinxhyphen{}state classifcations for the set of simulations from propagate.nested\_approach()

\item {} 
\sphinxAtStartPar
\sphinxstyleliteralstrong{\sphinxupquote{percent\_metrics}} (\sphinxstyleliteralemphasis{\sphinxupquote{list}}) – List of metrics to calculate a percent of (e.g. use with an indicator variable like failure=1/0 or True/False)

\item {} 
\sphinxAtStartPar
\sphinxstyleliteralstrong{\sphinxupquote{rate\_metrics}} (\sphinxstyleliteralemphasis{\sphinxupquote{list}}) – List of metrics to calculate the probability of using the rate variable in endclasses

\item {} 
\sphinxAtStartPar
\sphinxstyleliteralstrong{\sphinxupquote{average\_metrics}} (\sphinxstyleliteralemphasis{\sphinxupquote{list}}) – List of metrics to calculate an average of (e.g., use for float values like speed=25)

\item {} 
\sphinxAtStartPar
\sphinxstyleliteralstrong{\sphinxupquote{expected\_metrics}} (\sphinxstyleliteralemphasis{\sphinxupquote{list}}) – List of metrics to calculate the expected value of using the rate variable in endclasses

\item {} 
\sphinxAtStartPar
\sphinxstyleliteralstrong{\sphinxupquote{inputparams}} (\sphinxstyleliteralemphasis{\sphinxupquote{'from\_range'/'all'}}\sphinxstyleliteralemphasis{\sphinxupquote{,}}\sphinxstyleliteralemphasis{\sphinxupquote{list}}\sphinxstyleliteralemphasis{\sphinxupquote{, }}\sphinxstyleliteralemphasis{\sphinxupquote{optional}}) – Parameters to show on the table. The default is ‘from\_range’.

\item {} 
\sphinxAtStartPar
\sphinxstyleliteralstrong{\sphinxupquote{scenarios}} (\sphinxstyleliteralemphasis{\sphinxupquote{'all'}}\sphinxstyleliteralemphasis{\sphinxupquote{,}}\sphinxstyleliteralemphasis{\sphinxupquote{'range'/list}}\sphinxstyleliteralemphasis{\sphinxupquote{, }}\sphinxstyleliteralemphasis{\sphinxupquote{optional}}) – Scenarios to include in the table. ‘range’ is a given range\_id in the nominalapproach.

\end{itemize}

\item[{Returns}] \leavevmode
\sphinxAtStartPar
\sphinxstylestrong{table} – Table with the averages/percentages of interest layed out over the input parameters for the set of scenarios in endclasses

\item[{Return type}] \leavevmode
\sphinxAtStartPar
pandas DataFrame

\end{description}\end{quote}

\end{fulllineitems}

\index{nominal\_factor\_comparison() (in module fmdtools.resultdisp.tabulate)@\spxentry{nominal\_factor\_comparison()}\spxextra{in module fmdtools.resultdisp.tabulate}}

\begin{fulllineitems}
\phantomsection\label{\detokenize{docs/fmdtools.resultdisp:fmdtools.resultdisp.tabulate.nominal_factor_comparison}}\pysiglinewithargsret{\sphinxcode{\sphinxupquote{fmdtools.resultdisp.tabulate.}}\sphinxbfcode{\sphinxupquote{nominal\_factor\_comparison}}}{\emph{\DUrole{n}{nomapp}}, \emph{\DUrole{n}{endclasses}}, \emph{\DUrole{n}{params}}, \emph{\DUrole{n}{metrics}\DUrole{o}{=}\DUrole{default_value}{'all'}}, \emph{\DUrole{n}{rangeid}\DUrole{o}{=}\DUrole{default_value}{'default'}}, \emph{\DUrole{n}{nan\_as}\DUrole{o}{=}\DUrole{default_value}{nan}}, \emph{\DUrole{n}{percent}\DUrole{o}{=}\DUrole{default_value}{True}}, \emph{\DUrole{n}{difference}\DUrole{o}{=}\DUrole{default_value}{True}}, \emph{\DUrole{n}{give\_ci}\DUrole{o}{=}\DUrole{default_value}{False}}, \emph{\DUrole{o}{**}\DUrole{n}{kwargs}}}{}
\sphinxAtStartPar
Compares a metric for a given set of model parameters/factors over set of nominal scenarios.
\begin{quote}\begin{description}
\item[{Parameters}] \leavevmode\begin{itemize}
\item {} 
\sphinxAtStartPar
\sphinxstyleliteralstrong{\sphinxupquote{nomapp}} ({\hyperref[\detokenize{docs/fmdtools:fmdtools.modeldef.NominalApproach}]{\sphinxcrossref{\sphinxstyleliteralemphasis{\sphinxupquote{NominalApproach}}}}}) – Nominal Approach used to generate the simulations

\item {} 
\sphinxAtStartPar
\sphinxstyleliteralstrong{\sphinxupquote{endclasses}} (\sphinxstyleliteralemphasis{\sphinxupquote{dict}}) – \begin{description}
\item[{dict of endclasses from propagate.nominal\_approach or nested\_approach with structure:}] \leavevmode
\sphinxAtStartPar
\{scen\_x:\{metric1:x, metric2:x…\}\} or \{scen\_x:\{fault:\{metric1:x, metric2:x…\}\}\}

\end{description}


\item {} 
\sphinxAtStartPar
\sphinxstyleliteralstrong{\sphinxupquote{params}} (\sphinxstyleliteralemphasis{\sphinxupquote{list/str}}) – List of parameters (or parameter) to use for the factor levels in the comparison

\item {} 
\sphinxAtStartPar
\sphinxstyleliteralstrong{\sphinxupquote{metrics}} (\sphinxstyleliteralemphasis{\sphinxupquote{'all'/list}}\sphinxstyleliteralemphasis{\sphinxupquote{, }}\sphinxstyleliteralemphasis{\sphinxupquote{optional}}) – Metrics to show in the table. The default is ‘all’.

\item {} 
\sphinxAtStartPar
\sphinxstyleliteralstrong{\sphinxupquote{rangeid}} (\sphinxstyleliteralemphasis{\sphinxupquote{str}}\sphinxstyleliteralemphasis{\sphinxupquote{, }}\sphinxstyleliteralemphasis{\sphinxupquote{optional}}) – 
\sphinxAtStartPar
Nominal Approach range to use for the test, if run over a single range.
The default is ‘default’, which either:
\begin{itemize}
\item {} 
\sphinxAtStartPar
picks the only range (if there is only one), or

\item {} 
\sphinxAtStartPar
compares between ranges (if more than one)

\end{itemize}


\item {} 
\sphinxAtStartPar
\sphinxstyleliteralstrong{\sphinxupquote{nan\_as}} (\sphinxstyleliteralemphasis{\sphinxupquote{float}}\sphinxstyleliteralemphasis{\sphinxupquote{, }}\sphinxstyleliteralemphasis{\sphinxupquote{optional}}) – Number to parse NaNs as (if present). The default is np.nan.

\item {} 
\sphinxAtStartPar
\sphinxstyleliteralstrong{\sphinxupquote{percent}} (\sphinxstyleliteralemphasis{\sphinxupquote{bool}}\sphinxstyleliteralemphasis{\sphinxupquote{, }}\sphinxstyleliteralemphasis{\sphinxupquote{optional}}) – Whether to compare metrics as bools (True \sphinxhyphen{} results in a comparison of percentages of indicator variables)
or as averages (False \sphinxhyphen{} results in a comparison of average values of real valued variables). The default is True.

\item {} 
\sphinxAtStartPar
\sphinxstyleliteralstrong{\sphinxupquote{difference}} (\sphinxstyleliteralemphasis{\sphinxupquote{bool}}\sphinxstyleliteralemphasis{\sphinxupquote{, }}\sphinxstyleliteralemphasis{\sphinxupquote{optional}}) – Whether to tabulate the difference of the metric from the nominal over each scenario (True),
or the value of the metric over all (False). The default is True.

\item {} 
\sphinxAtStartPar
\sphinxstyleliteralstrong{\sphinxupquote{bool}} (\sphinxstyleliteralemphasis{\sphinxupquote{give\_ci =}}) – gives the bootstrap confidence interval for the given statistic using the given kwargs
‘combined’ combines the values as a strings in the table (for display)

\item {} 
\sphinxAtStartPar
\sphinxstyleliteralstrong{\sphinxupquote{kwargs}} (\sphinxstyleliteralemphasis{\sphinxupquote{keyword arguments for bootstrap\_confidence\_interval}}\sphinxstyleliteralemphasis{\sphinxupquote{ (}}\sphinxstyleliteralemphasis{\sphinxupquote{sample\_size}}\sphinxstyleliteralemphasis{\sphinxupquote{, }}\sphinxstyleliteralemphasis{\sphinxupquote{num\_samples}}\sphinxstyleliteralemphasis{\sphinxupquote{, }}\sphinxstyleliteralemphasis{\sphinxupquote{interval}}\sphinxstyleliteralemphasis{\sphinxupquote{, }}\sphinxstyleliteralemphasis{\sphinxupquote{seed}}\sphinxstyleliteralemphasis{\sphinxupquote{)}}) – 

\end{itemize}

\item[{Returns}] \leavevmode
\sphinxAtStartPar
\sphinxstylestrong{table} – Table with the metric statistic (percent or average) over the nominal scenario and each listed function/mode (as differences or averages)

\item[{Return type}] \leavevmode
\sphinxAtStartPar
pandas table

\end{description}\end{quote}

\end{fulllineitems}

\index{nominal\_stats() (in module fmdtools.resultdisp.tabulate)@\spxentry{nominal\_stats()}\spxextra{in module fmdtools.resultdisp.tabulate}}

\begin{fulllineitems}
\phantomsection\label{\detokenize{docs/fmdtools.resultdisp:fmdtools.resultdisp.tabulate.nominal_stats}}\pysiglinewithargsret{\sphinxcode{\sphinxupquote{fmdtools.resultdisp.tabulate.}}\sphinxbfcode{\sphinxupquote{nominal\_stats}}}{\emph{\DUrole{n}{nomapp}}, \emph{\DUrole{n}{nomapp\_endclasses}}, \emph{\DUrole{n}{metrics}\DUrole{o}{=}\DUrole{default_value}{'all'}}, \emph{\DUrole{n}{inputparams}\DUrole{o}{=}\DUrole{default_value}{'from\_range'}}, \emph{\DUrole{n}{scenarios}\DUrole{o}{=}\DUrole{default_value}{'all'}}}{}
\sphinxAtStartPar
Makes a table of quantities of interest from endclasses.
\begin{quote}\begin{description}
\item[{Parameters}] \leavevmode\begin{itemize}
\item {} 
\sphinxAtStartPar
\sphinxstyleliteralstrong{\sphinxupquote{nomapp}} ({\hyperref[\detokenize{docs/fmdtools:fmdtools.modeldef.NominalApproach}]{\sphinxcrossref{\sphinxstyleliteralemphasis{\sphinxupquote{NominalApproach}}}}}) – NominalApproach used to generate the simulation.

\item {} 
\sphinxAtStartPar
\sphinxstyleliteralstrong{\sphinxupquote{nomapp\_endclasses}} (\sphinxstyleliteralemphasis{\sphinxupquote{dict}}) – End\sphinxhyphen{}state classifcations for the set of simulations from propagate.nominalapproach()

\item {} 
\sphinxAtStartPar
\sphinxstyleliteralstrong{\sphinxupquote{metrics}} (\sphinxstyleliteralemphasis{\sphinxupquote{'all'/list}}\sphinxstyleliteralemphasis{\sphinxupquote{, }}\sphinxstyleliteralemphasis{\sphinxupquote{optional}}) – Metrics to show on the plot. The default is ‘all’.

\item {} 
\sphinxAtStartPar
\sphinxstyleliteralstrong{\sphinxupquote{inputparams}} (\sphinxstyleliteralemphasis{\sphinxupquote{'from\_range'/'all'}}\sphinxstyleliteralemphasis{\sphinxupquote{,}}\sphinxstyleliteralemphasis{\sphinxupquote{list}}\sphinxstyleliteralemphasis{\sphinxupquote{, }}\sphinxstyleliteralemphasis{\sphinxupquote{optional}}) – Parameters to show on the plot. The default is ‘from\_range’.

\item {} 
\sphinxAtStartPar
\sphinxstyleliteralstrong{\sphinxupquote{scenarios}} (\sphinxstyleliteralemphasis{\sphinxupquote{'all'}}\sphinxstyleliteralemphasis{\sphinxupquote{,}}\sphinxstyleliteralemphasis{\sphinxupquote{'range'/list}}\sphinxstyleliteralemphasis{\sphinxupquote{, }}\sphinxstyleliteralemphasis{\sphinxupquote{optional}}) – Scenarios to include in the plot. ‘range’ is a given range\_id in the nominalapproach.

\end{itemize}

\item[{Returns}] \leavevmode
\sphinxAtStartPar
\sphinxstylestrong{table} – Table with the metrics of interest layed out over the input parameters for the set of scenarios in endclasses

\item[{Return type}] \leavevmode
\sphinxAtStartPar
pandas DataFrame

\end{description}\end{quote}

\end{fulllineitems}

\index{objtab() (in module fmdtools.resultdisp.tabulate)@\spxentry{objtab()}\spxextra{in module fmdtools.resultdisp.tabulate}}

\begin{fulllineitems}
\phantomsection\label{\detokenize{docs/fmdtools.resultdisp:fmdtools.resultdisp.tabulate.objtab}}\pysiglinewithargsret{\sphinxcode{\sphinxupquote{fmdtools.resultdisp.tabulate.}}\sphinxbfcode{\sphinxupquote{objtab}}}{\emph{\DUrole{n}{hist}}, \emph{\DUrole{n}{objtype}}}{}
\sphinxAtStartPar
make table of function OR flow value attributes \sphinxhyphen{} objtype = ‘function’ or ‘flow’

\end{fulllineitems}

\index{phasefmea() (in module fmdtools.resultdisp.tabulate)@\spxentry{phasefmea()}\spxextra{in module fmdtools.resultdisp.tabulate}}

\begin{fulllineitems}
\phantomsection\label{\detokenize{docs/fmdtools.resultdisp:fmdtools.resultdisp.tabulate.phasefmea}}\pysiglinewithargsret{\sphinxcode{\sphinxupquote{fmdtools.resultdisp.tabulate.}}\sphinxbfcode{\sphinxupquote{phasefmea}}}{\emph{\DUrole{n}{endclasses}}, \emph{\DUrole{n}{app}}}{}
\sphinxAtStartPar
Makes a simple fmea of the endclasses of a set of fault scenarios run grouped by phase.
\begin{quote}\begin{description}
\item[{Parameters}] \leavevmode\begin{itemize}
\item {} 
\sphinxAtStartPar
\sphinxstyleliteralstrong{\sphinxupquote{endclasses}} (\sphinxstyleliteralemphasis{\sphinxupquote{dict}}) – dict of endclasses of the simulation runs

\item {} 
\sphinxAtStartPar
\sphinxstyleliteralstrong{\sphinxupquote{app}} (\sphinxstyleliteralemphasis{\sphinxupquote{sampleapproach}}) – sample approach used for the underlying probability model of the set of scenarios run

\end{itemize}

\item[{Returns}] \leavevmode
\sphinxAtStartPar
\sphinxstylestrong{table} – table with cost, rate, and expected cost of each fault in each phase

\item[{Return type}] \leavevmode
\sphinxAtStartPar
dataframe

\end{description}\end{quote}

\end{fulllineitems}

\index{resilience\_factor\_comparison() (in module fmdtools.resultdisp.tabulate)@\spxentry{resilience\_factor\_comparison()}\spxextra{in module fmdtools.resultdisp.tabulate}}

\begin{fulllineitems}
\phantomsection\label{\detokenize{docs/fmdtools.resultdisp:fmdtools.resultdisp.tabulate.resilience_factor_comparison}}\pysiglinewithargsret{\sphinxcode{\sphinxupquote{fmdtools.resultdisp.tabulate.}}\sphinxbfcode{\sphinxupquote{resilience\_factor\_comparison}}}{\emph{\DUrole{n}{nomapp}}, \emph{\DUrole{n}{nested\_endclasses}}, \emph{\DUrole{n}{params}}, \emph{\DUrole{n}{value}}, \emph{\DUrole{n}{faults}\DUrole{o}{=}\DUrole{default_value}{'functions'}}, \emph{\DUrole{n}{rangeid}\DUrole{o}{=}\DUrole{default_value}{'default'}}, \emph{\DUrole{n}{nan\_as}\DUrole{o}{=}\DUrole{default_value}{nan}}, \emph{\DUrole{n}{percent}\DUrole{o}{=}\DUrole{default_value}{True}}, \emph{\DUrole{n}{difference}\DUrole{o}{=}\DUrole{default_value}{True}}, \emph{\DUrole{n}{give\_ci}\DUrole{o}{=}\DUrole{default_value}{False}}, \emph{\DUrole{o}{**}\DUrole{n}{kwargs}}}{}
\sphinxAtStartPar
Compares a metric for a given set of model parameters/factors over a nested set of nominal and fault scenarios.
\begin{quote}\begin{description}
\item[{Parameters}] \leavevmode\begin{itemize}
\item {} 
\sphinxAtStartPar
\sphinxstyleliteralstrong{\sphinxupquote{nomapp}} ({\hyperref[\detokenize{docs/fmdtools:fmdtools.modeldef.NominalApproach}]{\sphinxcrossref{\sphinxstyleliteralemphasis{\sphinxupquote{NominalApproach}}}}}) – Nominal Approach used to generate the simulations

\item {} 
\sphinxAtStartPar
\sphinxstyleliteralstrong{\sphinxupquote{nested\_endclasses}} (\sphinxstyleliteralemphasis{\sphinxupquote{dict}}) – dict of endclasses from propagate.nested\_approach with structure: \{scen\_x:\{fault:\{metric1:x, metric2:x…\}\}\}

\item {} 
\sphinxAtStartPar
\sphinxstyleliteralstrong{\sphinxupquote{params}} (\sphinxstyleliteralemphasis{\sphinxupquote{list/str}}) – List of parameters (or parameter) to use for the factor levels in the comparison

\item {} 
\sphinxAtStartPar
\sphinxstyleliteralstrong{\sphinxupquote{value}} (\sphinxstyleliteralemphasis{\sphinxupquote{string}}) – metric of the endclass (returned by mdl.find\_classification) to use for the comparison.

\item {} 
\sphinxAtStartPar
\sphinxstyleliteralstrong{\sphinxupquote{faults}} (\sphinxstyleliteralemphasis{\sphinxupquote{str/list}}\sphinxstyleliteralemphasis{\sphinxupquote{, }}\sphinxstyleliteralemphasis{\sphinxupquote{optional}}) – \begin{description}
\item[{Set of faults to run the comparison over}] \leavevmode
\sphinxAtStartPar
–‘modes’ (all fault modes),
–‘functions’ (modes for each function are grouped)
–‘mode type’ (modes with the same name are grouped)
– or a set of specific modes/functions. The default is ‘functions’.

\end{description}


\item {} 
\sphinxAtStartPar
\sphinxstyleliteralstrong{\sphinxupquote{rangeid}} (\sphinxstyleliteralemphasis{\sphinxupquote{str}}\sphinxstyleliteralemphasis{\sphinxupquote{, }}\sphinxstyleliteralemphasis{\sphinxupquote{optional}}) – 
\sphinxAtStartPar
Nominal Approach range to use for the test, if run over a single range.
The default is ‘default’, which either:
\begin{itemize}
\item {} 
\sphinxAtStartPar
picks the only range (if there is only one), or

\item {} 
\sphinxAtStartPar
compares between ranges (if more than one)

\end{itemize}


\item {} 
\sphinxAtStartPar
\sphinxstyleliteralstrong{\sphinxupquote{nan\_as}} (\sphinxstyleliteralemphasis{\sphinxupquote{float}}\sphinxstyleliteralemphasis{\sphinxupquote{, }}\sphinxstyleliteralemphasis{\sphinxupquote{optional}}) – Number to parse NaNs as (if present). The default is np.nan.

\item {} 
\sphinxAtStartPar
\sphinxstyleliteralstrong{\sphinxupquote{percent}} (\sphinxstyleliteralemphasis{\sphinxupquote{bool}}\sphinxstyleliteralemphasis{\sphinxupquote{, }}\sphinxstyleliteralemphasis{\sphinxupquote{optional}}) – Whether to compare metrics as bools (True \sphinxhyphen{} results in a comparison of percentages of indicator variables)
or as averages (False \sphinxhyphen{} results in a comparison of average values of real valued variables). The default is True.

\item {} 
\sphinxAtStartPar
\sphinxstyleliteralstrong{\sphinxupquote{difference}} (\sphinxstyleliteralemphasis{\sphinxupquote{bool}}\sphinxstyleliteralemphasis{\sphinxupquote{, }}\sphinxstyleliteralemphasis{\sphinxupquote{optional}}) – Whether to tabulate the difference of the metric from the nominal over each scenario (True),
or the value of the metric over all (False). The default is True.

\item {} 
\sphinxAtStartPar
\sphinxstyleliteralstrong{\sphinxupquote{bool}} (\sphinxstyleliteralemphasis{\sphinxupquote{give\_ci =}}) – gives the bootstrap confidence interval for the given statistic using the given kwargs
‘combined’ combines the values as a strings in the table (for display)

\item {} 
\sphinxAtStartPar
\sphinxstyleliteralstrong{\sphinxupquote{kwargs}} (\sphinxstyleliteralemphasis{\sphinxupquote{keyword arguments for bootstrap\_confidence\_interval}}\sphinxstyleliteralemphasis{\sphinxupquote{ (}}\sphinxstyleliteralemphasis{\sphinxupquote{sample\_size}}\sphinxstyleliteralemphasis{\sphinxupquote{, }}\sphinxstyleliteralemphasis{\sphinxupquote{num\_samples}}\sphinxstyleliteralemphasis{\sphinxupquote{, }}\sphinxstyleliteralemphasis{\sphinxupquote{interval}}\sphinxstyleliteralemphasis{\sphinxupquote{, }}\sphinxstyleliteralemphasis{\sphinxupquote{seed}}\sphinxstyleliteralemphasis{\sphinxupquote{)}}) – 

\end{itemize}

\item[{Returns}] \leavevmode
\sphinxAtStartPar
\sphinxstylestrong{table} – Table with the metric statistic (percent or average) over the nominal scenario and each listed function/mode (as differences or averages)

\item[{Return type}] \leavevmode
\sphinxAtStartPar
pandas table

\end{description}\end{quote}

\end{fulllineitems}

\index{result() (in module fmdtools.resultdisp.tabulate)@\spxentry{result()}\spxextra{in module fmdtools.resultdisp.tabulate}}

\begin{fulllineitems}
\phantomsection\label{\detokenize{docs/fmdtools.resultdisp:fmdtools.resultdisp.tabulate.result}}\pysiglinewithargsret{\sphinxcode{\sphinxupquote{fmdtools.resultdisp.tabulate.}}\sphinxbfcode{\sphinxupquote{result}}}{\emph{\DUrole{n}{endresults}}, \emph{\DUrole{n}{summary}}}{}
\sphinxAtStartPar
Makes a table of results (degraded functions/flows, cost, rate, expected cost) of a single run

\end{fulllineitems}

\index{samptime() (in module fmdtools.resultdisp.tabulate)@\spxentry{samptime()}\spxextra{in module fmdtools.resultdisp.tabulate}}

\begin{fulllineitems}
\phantomsection\label{\detokenize{docs/fmdtools.resultdisp:fmdtools.resultdisp.tabulate.samptime}}\pysiglinewithargsret{\sphinxcode{\sphinxupquote{fmdtools.resultdisp.tabulate.}}\sphinxbfcode{\sphinxupquote{samptime}}}{\emph{\DUrole{n}{sampletimes}}}{}
\sphinxAtStartPar
Makes a table of the times sampled for each phase given a dict (i.e. app.sampletimes)

\end{fulllineitems}

\index{simplefmea() (in module fmdtools.resultdisp.tabulate)@\spxentry{simplefmea()}\spxextra{in module fmdtools.resultdisp.tabulate}}

\begin{fulllineitems}
\phantomsection\label{\detokenize{docs/fmdtools.resultdisp:fmdtools.resultdisp.tabulate.simplefmea}}\pysiglinewithargsret{\sphinxcode{\sphinxupquote{fmdtools.resultdisp.tabulate.}}\sphinxbfcode{\sphinxupquote{simplefmea}}}{\emph{\DUrole{n}{endclasses}}}{}
\sphinxAtStartPar
Makes a simple fmea (rate, cost, expected cost) of the endclasses of a list of fault scenarios run

\end{fulllineitems}

\index{stats() (in module fmdtools.resultdisp.tabulate)@\spxentry{stats()}\spxextra{in module fmdtools.resultdisp.tabulate}}

\begin{fulllineitems}
\phantomsection\label{\detokenize{docs/fmdtools.resultdisp:fmdtools.resultdisp.tabulate.stats}}\pysiglinewithargsret{\sphinxcode{\sphinxupquote{fmdtools.resultdisp.tabulate.}}\sphinxbfcode{\sphinxupquote{stats}}}{\emph{\DUrole{n}{reshist}}}{}
\sphinxAtStartPar
Makes a table of \#of degraded flows, \# of degraded functions, and \# of total faults over time given a single result history

\end{fulllineitems}

\index{summary() (in module fmdtools.resultdisp.tabulate)@\spxentry{summary()}\spxextra{in module fmdtools.resultdisp.tabulate}}

\begin{fulllineitems}
\phantomsection\label{\detokenize{docs/fmdtools.resultdisp:fmdtools.resultdisp.tabulate.summary}}\pysiglinewithargsret{\sphinxcode{\sphinxupquote{fmdtools.resultdisp.tabulate.}}\sphinxbfcode{\sphinxupquote{summary}}}{\emph{\DUrole{n}{summary}}}{}
\sphinxAtStartPar
Makes a table of a summary dictionary from a given model run

\end{fulllineitems}

\index{summfmea() (in module fmdtools.resultdisp.tabulate)@\spxentry{summfmea()}\spxextra{in module fmdtools.resultdisp.tabulate}}

\begin{fulllineitems}
\phantomsection\label{\detokenize{docs/fmdtools.resultdisp:fmdtools.resultdisp.tabulate.summfmea}}\pysiglinewithargsret{\sphinxcode{\sphinxupquote{fmdtools.resultdisp.tabulate.}}\sphinxbfcode{\sphinxupquote{summfmea}}}{\emph{\DUrole{n}{endclasses}}, \emph{\DUrole{n}{app}}}{}
\sphinxAtStartPar
Makes a simple fmea of the endclasses of a set of fault scenarios run grouped by fault.
\begin{quote}\begin{description}
\item[{Parameters}] \leavevmode\begin{itemize}
\item {} 
\sphinxAtStartPar
\sphinxstyleliteralstrong{\sphinxupquote{endclasses}} (\sphinxstyleliteralemphasis{\sphinxupquote{dict}}) – dict of endclasses of the simulation runs

\item {} 
\sphinxAtStartPar
\sphinxstyleliteralstrong{\sphinxupquote{app}} (\sphinxstyleliteralemphasis{\sphinxupquote{sampleapproach}}) – sample approach used for the underlying probability model of the set of scenarios run

\end{itemize}

\item[{Returns}] \leavevmode
\sphinxAtStartPar
\sphinxstylestrong{table} – table with cost, rate, and expected cost of each fault (over all phases)

\item[{Return type}] \leavevmode
\sphinxAtStartPar
dataframe

\end{description}\end{quote}

\end{fulllineitems}



\subsection{fmdtools.modeldef}
\label{\detokenize{docs/fmdtools:fmdtools-modeldef}}
\noindent\sphinxincludegraphics[width=800\sphinxpxdimen]{{model_definition}.png}

\sphinxAtStartPar
The {\hyperref[\detokenize{docs/fmdtools:module-fmdtools.modeldef}]{\sphinxcrossref{\sphinxcode{\sphinxupquote{fmdtools.modeldef}}}}} module provides constructs for model and simulation definition, as shown above. In general, to define a model, these classes are extended by the user in a model file to define the specific attributes of the functions, flows, components, etc. The module reference is provided below:

\phantomsection\label{\detokenize{docs/fmdtools:module-fmdtools.modeldef}}\index{module@\spxentry{module}!fmdtools.modeldef@\spxentry{fmdtools.modeldef}}\index{fmdtools.modeldef@\spxentry{fmdtools.modeldef}!module@\spxentry{module}}
\sphinxAtStartPar
Description: A module to define resilience models and simulations.
\begin{itemize}
\item {} 
\sphinxAtStartPar
{\hyperref[\detokenize{docs/fmdtools:fmdtools.modeldef.Common}]{\sphinxcrossref{\sphinxcode{\sphinxupquote{Common}}}}}:      Class defining common methods accessible by Function/Flow/Component Classes

\item {} 
\sphinxAtStartPar
{\hyperref[\detokenize{docs/fmdtools:fmdtools.modeldef.FxnBlock}]{\sphinxcrossref{\sphinxcode{\sphinxupquote{FxnBlock}}}}}:    Class defining Model Functions and their attributes

\item {} 
\sphinxAtStartPar
{\hyperref[\detokenize{docs/fmdtools:fmdtools.modeldef.Flow}]{\sphinxcrossref{\sphinxcode{\sphinxupquote{Flow}}}}}:        Class defining Model Flows and their attributes

\item {} 
\sphinxAtStartPar
{\hyperref[\detokenize{docs/fmdtools:fmdtools.modeldef.Component}]{\sphinxcrossref{\sphinxcode{\sphinxupquote{Component}}}}}:   Class defining Function Components and their attributes

\item {} 
\sphinxAtStartPar
{\hyperref[\detokenize{docs/fmdtools:fmdtools.modeldef.SampleApproach}]{\sphinxcrossref{\sphinxcode{\sphinxupquote{SampleApproach}}}}}:  Class defining fault sampling approaches

\item {} 
\sphinxAtStartPar
{\hyperref[\detokenize{docs/fmdtools:fmdtools.modeldef.NominalApproach}]{\sphinxcrossref{\sphinxcode{\sphinxupquote{NominalApproach}}}}}: Class defining parameter sampling approaches

\end{itemize}
\index{Block (class in fmdtools.modeldef)@\spxentry{Block}\spxextra{class in fmdtools.modeldef}}

\begin{fulllineitems}
\phantomsection\label{\detokenize{docs/fmdtools:fmdtools.modeldef.Block}}\pysiglinewithargsret{\sphinxbfcode{\sphinxupquote{class }}\sphinxcode{\sphinxupquote{fmdtools.modeldef.}}\sphinxbfcode{\sphinxupquote{Block}}}{\emph{\DUrole{n}{states}\DUrole{o}{=}\DUrole{default_value}{\{\}}}}{}
\sphinxAtStartPar
Bases: {\hyperref[\detokenize{docs/fmdtools:fmdtools.modeldef.Common}]{\sphinxcrossref{\sphinxcode{\sphinxupquote{fmdtools.modeldef.Common}}}}}

\sphinxAtStartPar
Superclass for FxnBlock and Component subclasses. Has functions for model setup, querying state, reseting the model
\index{failrate (fmdtools.modeldef.Block attribute)@\spxentry{failrate}\spxextra{fmdtools.modeldef.Block attribute}}

\begin{fulllineitems}
\phantomsection\label{\detokenize{docs/fmdtools:fmdtools.modeldef.Block.failrate}}\pysigline{\sphinxbfcode{\sphinxupquote{failrate}}}
\sphinxAtStartPar
Failure rate for the block
\begin{quote}\begin{description}
\item[{Type}] \leavevmode
\sphinxAtStartPar
float

\end{description}\end{quote}

\end{fulllineitems}

\index{time (fmdtools.modeldef.Block attribute)@\spxentry{time}\spxextra{fmdtools.modeldef.Block attribute}}

\begin{fulllineitems}
\phantomsection\label{\detokenize{docs/fmdtools:fmdtools.modeldef.Block.time}}\pysigline{\sphinxbfcode{\sphinxupquote{time}}}
\sphinxAtStartPar
internal time of the function
\begin{quote}\begin{description}
\item[{Type}] \leavevmode
\sphinxAtStartPar
float

\end{description}\end{quote}

\end{fulllineitems}

\index{faults (fmdtools.modeldef.Block attribute)@\spxentry{faults}\spxextra{fmdtools.modeldef.Block attribute}}

\begin{fulllineitems}
\phantomsection\label{\detokenize{docs/fmdtools:fmdtools.modeldef.Block.faults}}\pysigline{\sphinxbfcode{\sphinxupquote{faults}}}
\sphinxAtStartPar
faults currently present in the block. If the function is nominal, set is \{‘nom’\}
\begin{quote}\begin{description}
\item[{Type}] \leavevmode
\sphinxAtStartPar
set

\end{description}\end{quote}

\end{fulllineitems}

\index{faultmodes (fmdtools.modeldef.Block attribute)@\spxentry{faultmodes}\spxextra{fmdtools.modeldef.Block attribute}}

\begin{fulllineitems}
\phantomsection\label{\detokenize{docs/fmdtools:fmdtools.modeldef.Block.faultmodes}}\pysigline{\sphinxbfcode{\sphinxupquote{faultmodes}}}~\begin{description}
\item[{faults possible to inject in the block and their properties. Has structure:}] \leavevmode\begin{itemize}
\item {} \begin{description}
\item[{faultname :}] \leavevmode\begin{itemize}
\item {} 
\sphinxAtStartPar
dist : (float of \% failures due to this fualt)

\item {} 
\sphinxAtStartPar
oppvect : (list of relative probabilities of the fault occuring in each phase)

\item {} 
\sphinxAtStartPar
rcost : cost of repairing the fault

\end{itemize}

\end{description}

\end{itemize}

\end{description}
\begin{quote}\begin{description}
\item[{Type}] \leavevmode
\sphinxAtStartPar
dict

\end{description}\end{quote}

\end{fulllineitems}

\index{opermodes (fmdtools.modeldef.Block attribute)@\spxentry{opermodes}\spxextra{fmdtools.modeldef.Block attribute}}

\begin{fulllineitems}
\phantomsection\label{\detokenize{docs/fmdtools:fmdtools.modeldef.Block.opermodes}}\pysigline{\sphinxbfcode{\sphinxupquote{opermodes}}}
\sphinxAtStartPar
possible modes for the block to enter
\begin{quote}\begin{description}
\item[{Type}] \leavevmode
\sphinxAtStartPar
list

\end{description}\end{quote}

\end{fulllineitems}

\index{rngs (fmdtools.modeldef.Block attribute)@\spxentry{rngs}\spxextra{fmdtools.modeldef.Block attribute}}

\begin{fulllineitems}
\phantomsection\label{\detokenize{docs/fmdtools:fmdtools.modeldef.Block.rngs}}\pysigline{\sphinxbfcode{\sphinxupquote{rngs}}}
\sphinxAtStartPar
dictionary of random number generators for random states
\begin{quote}\begin{description}
\item[{Type}] \leavevmode
\sphinxAtStartPar
dict

\end{description}\end{quote}

\end{fulllineitems}

\index{seed (fmdtools.modeldef.Block attribute)@\spxentry{seed}\spxextra{fmdtools.modeldef.Block attribute}}

\begin{fulllineitems}
\phantomsection\label{\detokenize{docs/fmdtools:fmdtools.modeldef.Block.seed}}\pysigline{\sphinxbfcode{\sphinxupquote{seed}}}
\sphinxAtStartPar
seed sequence for internal random number generator
\begin{quote}\begin{description}
\item[{Type}] \leavevmode
\sphinxAtStartPar
int

\end{description}\end{quote}

\end{fulllineitems}

\index{mode (fmdtools.modeldef.Block attribute)@\spxentry{mode}\spxextra{fmdtools.modeldef.Block attribute}}

\begin{fulllineitems}
\phantomsection\label{\detokenize{docs/fmdtools:fmdtools.modeldef.Block.mode}}\pysigline{\sphinxbfcode{\sphinxupquote{mode}}}
\sphinxAtStartPar
current mode of block operation
\begin{quote}\begin{description}
\item[{Type}] \leavevmode
\sphinxAtStartPar
string

\end{description}\end{quote}

\end{fulllineitems}

\index{add\_fault() (fmdtools.modeldef.Block method)@\spxentry{add\_fault()}\spxextra{fmdtools.modeldef.Block method}}

\begin{fulllineitems}
\phantomsection\label{\detokenize{docs/fmdtools:fmdtools.modeldef.Block.add_fault}}\pysiglinewithargsret{\sphinxbfcode{\sphinxupquote{add\_fault}}}{\emph{\DUrole{o}{*}\DUrole{n}{faults}}}{}
\sphinxAtStartPar
Adds fault (a str) to the block
\begin{quote}\begin{description}
\item[{Parameters}] \leavevmode
\sphinxAtStartPar
\sphinxstyleliteralstrong{\sphinxupquote{*fault}} (\sphinxstyleliteralemphasis{\sphinxupquote{str}}\sphinxstyleliteralemphasis{\sphinxupquote{(}}\sphinxstyleliteralemphasis{\sphinxupquote{s}}\sphinxstyleliteralemphasis{\sphinxupquote{)}}) – name(s) of the fault to add to the black

\end{description}\end{quote}

\end{fulllineitems}

\index{add\_he\_rate() (fmdtools.modeldef.Block method)@\spxentry{add\_he\_rate()}\spxextra{fmdtools.modeldef.Block method}}

\begin{fulllineitems}
\phantomsection\label{\detokenize{docs/fmdtools:fmdtools.modeldef.Block.add_he_rate}}\pysiglinewithargsret{\sphinxbfcode{\sphinxupquote{add\_he\_rate}}}{\emph{\DUrole{n}{gtp}}, \emph{\DUrole{n}{EPCs}\DUrole{o}{=}\DUrole{default_value}{\{'na': {[}1, 0{]}\}}}}{}
\sphinxAtStartPar
Calculates self.failrate based on a human error probability model.
\begin{quote}\begin{description}
\item[{Parameters}] \leavevmode\begin{itemize}
\item {} 
\sphinxAtStartPar
\sphinxstyleliteralstrong{\sphinxupquote{gtp}} (\sphinxstyleliteralemphasis{\sphinxupquote{float}}) – Generic Task Probability. (from HEART)

\item {} 
\sphinxAtStartPar
\sphinxstyleliteralstrong{\sphinxupquote{EPCs}} (\sphinxstyleliteralemphasis{\sphinxupquote{Dict}}\sphinxstyleliteralemphasis{\sphinxupquote{ or }}\sphinxstyleliteralemphasis{\sphinxupquote{list}}) – Error producing conditions (and respective factors) for a given task (from HEART). Used in format:
Dict \{‘name’:{[}EPC factor, Effect proportion{]}\} or list {[}{[}EPC factor, Effect proportion{]},{[}{[}EPC factor, Effect proportion{]}{]}{]}

\end{itemize}

\end{description}\end{quote}

\end{fulllineitems}

\index{add\_params() (fmdtools.modeldef.Block method)@\spxentry{add\_params()}\spxextra{fmdtools.modeldef.Block method}}

\begin{fulllineitems}
\phantomsection\label{\detokenize{docs/fmdtools:fmdtools.modeldef.Block.add_params}}\pysiglinewithargsret{\sphinxbfcode{\sphinxupquote{add\_params}}}{\emph{\DUrole{o}{*}\DUrole{n}{params}}}{}
\sphinxAtStartPar
Adds given dictionary(s) of parameters to the function/block.
e.g. self.add\_params(\{‘x’:1,’y’:1\}) results in a block where:
\begin{quote}

\sphinxAtStartPar
self.x = 1, self.y = 1
\end{quote}

\end{fulllineitems}

\index{any\_faults() (fmdtools.modeldef.Block method)@\spxentry{any\_faults()}\spxextra{fmdtools.modeldef.Block method}}

\begin{fulllineitems}
\phantomsection\label{\detokenize{docs/fmdtools:fmdtools.modeldef.Block.any_faults}}\pysiglinewithargsret{\sphinxbfcode{\sphinxupquote{any\_faults}}}{}{}
\sphinxAtStartPar
check if the block has any fault modes

\end{fulllineitems}

\index{assoc\_healthstate\_modes() (fmdtools.modeldef.Block method)@\spxentry{assoc\_healthstate\_modes()}\spxextra{fmdtools.modeldef.Block method}}

\begin{fulllineitems}
\phantomsection\label{\detokenize{docs/fmdtools:fmdtools.modeldef.Block.assoc_healthstate_modes}}\pysiglinewithargsret{\sphinxbfcode{\sphinxupquote{assoc\_healthstate\_modes}}}{\emph{\DUrole{n}{hranges}\DUrole{o}{=}\DUrole{default_value}{\{\}}}, \emph{\DUrole{n}{mode\_app}\DUrole{o}{=}\DUrole{default_value}{'none'}}, \emph{\DUrole{n}{manual\_modes}\DUrole{o}{=}\DUrole{default_value}{\{\}}}, \emph{\DUrole{n}{probtype}\DUrole{o}{=}\DUrole{default_value}{'prob'}}, \emph{\DUrole{n}{units}\DUrole{o}{=}\DUrole{default_value}{'hr'}}, \emph{\DUrole{n}{key\_phases\_by}\DUrole{o}{=}\DUrole{default_value}{'global'}}}{}
\sphinxAtStartPar
Associates modes with given healthstates.
\begin{quote}\begin{description}
\item[{Parameters}] \leavevmode\begin{itemize}
\item {} 
\sphinxAtStartPar
\sphinxstyleliteralstrong{\sphinxupquote{hranges}} (\sphinxstyleliteralemphasis{\sphinxupquote{dict}}\sphinxstyleliteralemphasis{\sphinxupquote{, }}\sphinxstyleliteralemphasis{\sphinxupquote{optional}}) – Dictionary of form \{‘state’:\{val1, val2…\}) of ranges for each health state (if used to generate modes). The default is \{\}.

\item {} 
\sphinxAtStartPar
\sphinxstyleliteralstrong{\sphinxupquote{mode\_app}} (\sphinxstyleliteralemphasis{\sphinxupquote{str}}) – type of modes to elaborate from the given health states.

\item {} 
\sphinxAtStartPar
\sphinxstyleliteralstrong{\sphinxupquote{manual\_modes}} (\sphinxstyleliteralemphasis{\sphinxupquote{dict}}\sphinxstyleliteralemphasis{\sphinxupquote{, }}\sphinxstyleliteralemphasis{\sphinxupquote{optional}}) – \begin{description}
\item[{Dictionary/Set of faultmodes with structure, which has the form:}] \leavevmode\begin{itemize}
\item {} \begin{description}
\item[{dict \{‘fault1’: {[}atts{]}, ‘fault2’: atts\}, where atts may be of form:}] \leavevmode\begin{itemize}
\item {} 
\sphinxAtStartPar
states: \{state1: val1, state2, val2\}

\item {} 
\sphinxAtStartPar
{[}states, faultattributes{]}, where faultattributes is the same as in assoc\_modes

\end{itemize}

\end{description}

\end{itemize}

\end{description}


\item {} 
\sphinxAtStartPar
\sphinxstyleliteralstrong{\sphinxupquote{probtype}} (\sphinxstyleliteralemphasis{\sphinxupquote{str}}\sphinxstyleliteralemphasis{\sphinxupquote{, }}\sphinxstyleliteralemphasis{\sphinxupquote{optional}}) – Type of probability in the probability model, a per\sphinxhyphen{}time ‘rate’ or per\sphinxhyphen{}run ‘prob’.
The default is ‘rate’

\item {} 
\sphinxAtStartPar
\sphinxstyleliteralstrong{\sphinxupquote{units}} (\sphinxstyleliteralemphasis{\sphinxupquote{str}}\sphinxstyleliteralemphasis{\sphinxupquote{, }}\sphinxstyleliteralemphasis{\sphinxupquote{optional}}) – Type of units (‘sec’/’min’/’hr’/’day’) used for the rates. Default is ‘hr’

\end{itemize}

\end{description}\end{quote}

\end{fulllineitems}

\index{assoc\_healthstates() (fmdtools.modeldef.Block method)@\spxentry{assoc\_healthstates()}\spxextra{fmdtools.modeldef.Block method}}

\begin{fulllineitems}
\phantomsection\label{\detokenize{docs/fmdtools:fmdtools.modeldef.Block.assoc_healthstates}}\pysiglinewithargsret{\sphinxbfcode{\sphinxupquote{assoc\_healthstates}}}{\emph{\DUrole{n}{hstates}}, \emph{\DUrole{n}{mode\_app}\DUrole{o}{=}\DUrole{default_value}{'single\sphinxhyphen{}state'}}, \emph{\DUrole{n}{probtype}\DUrole{o}{=}\DUrole{default_value}{'prob'}}, \emph{\DUrole{n}{units}\DUrole{o}{=}\DUrole{default_value}{'hr'}}}{}
\sphinxAtStartPar
Adds health state attributes to the model (and a mode approach if desired).
\begin{quote}\begin{description}
\item[{Parameters}] \leavevmode\begin{itemize}
\item {} 
\sphinxAtStartPar
\sphinxstyleliteralstrong{\sphinxupquote{hstates}} (\sphinxstyleliteralemphasis{\sphinxupquote{Dict}}) – Health states to incorporate in the model and their respective values.
e.g., \{‘state’:{[}1,\{0,2,\sphinxhyphen{}1\}{]}\}, \{‘state’:\{0,2,\sphinxhyphen{}1\}\}

\item {} 
\sphinxAtStartPar
\sphinxstyleliteralstrong{\sphinxupquote{mode\_app}} (\sphinxstyleliteralemphasis{\sphinxupquote{str}}) – type of modes to elaborate from the given health states.

\end{itemize}

\end{description}\end{quote}

\end{fulllineitems}

\index{assoc\_modes() (fmdtools.modeldef.Block method)@\spxentry{assoc\_modes()}\spxextra{fmdtools.modeldef.Block method}}

\begin{fulllineitems}
\phantomsection\label{\detokenize{docs/fmdtools:fmdtools.modeldef.Block.assoc_modes}}\pysiglinewithargsret{\sphinxbfcode{\sphinxupquote{assoc\_modes}}}{\emph{\DUrole{n}{faultmodes}\DUrole{o}{=}\DUrole{default_value}{\{\}}}, \emph{\DUrole{n}{opermodes}\DUrole{o}{=}\DUrole{default_value}{{[}{]}}}, \emph{\DUrole{n}{initmode}\DUrole{o}{=}\DUrole{default_value}{'nom'}}, \emph{\DUrole{n}{name}\DUrole{o}{=}\DUrole{default_value}{''}}, \emph{\DUrole{n}{probtype}\DUrole{o}{=}\DUrole{default_value}{'rate'}}, \emph{\DUrole{n}{units}\DUrole{o}{=}\DUrole{default_value}{'hr'}}, \emph{\DUrole{n}{exclusive}\DUrole{o}{=}\DUrole{default_value}{False}}, \emph{\DUrole{n}{key\_phases\_by}\DUrole{o}{=}\DUrole{default_value}{'global'}}, \emph{\DUrole{n}{longnames}\DUrole{o}{=}\DUrole{default_value}{\{\}}}}{}
\sphinxAtStartPar
Associates fault and operational modes with the block when called in the function or component.
\begin{quote}\begin{description}
\item[{Parameters}] \leavevmode\begin{itemize}
\item {} 
\sphinxAtStartPar
\sphinxstyleliteralstrong{\sphinxupquote{faultmodes}} (\sphinxstyleliteralemphasis{\sphinxupquote{dict}}\sphinxstyleliteralemphasis{\sphinxupquote{, }}\sphinxstyleliteralemphasis{\sphinxupquote{optional}}) – \begin{description}
\item[{Dictionary/Set of faultmodes with structure, which can have the forms:}] \leavevmode\begin{itemize}
\item {} 
\sphinxAtStartPar
set \{‘fault1’, ‘fault2’, ‘fault3’\} (just the respective faults)

\item {} \begin{description}
\item[{dict \{‘fault1’: faultattributes, ‘fault2’: faultattributes\}, where faultattributes is:}] \leavevmode\begin{itemize}
\item {} 
\sphinxAtStartPar
float: rate for the mode

\item {} 
\sphinxAtStartPar
{[}float, float{]}: rate and repair cost for the mode

\item {} 
\sphinxAtStartPar
float, oppvect, float{]}: rate, opportunity vector, and repair cost for the mode

\end{itemize}
\begin{description}
\item[{opportunity vector can be specified as:}] \leavevmode
\sphinxAtStartPar
{[}float1, float2,…{]}, a vector of relative likelihoods for each phase, or
\{opermode:float1, opermode:float1\}, a dict of relative likelihoods for each phase/mode
the phases/modes to key by are defined in “key\_phases\_by”

\end{description}

\end{description}

\end{itemize}

\end{description}


\item {} 
\sphinxAtStartPar
\sphinxstyleliteralstrong{\sphinxupquote{opermodes}} (\sphinxstyleliteralemphasis{\sphinxupquote{list}}\sphinxstyleliteralemphasis{\sphinxupquote{, }}\sphinxstyleliteralemphasis{\sphinxupquote{optional}}) – List of operational modes

\item {} 
\sphinxAtStartPar
\sphinxstyleliteralstrong{\sphinxupquote{initmode}} (\sphinxstyleliteralemphasis{\sphinxupquote{str}}\sphinxstyleliteralemphasis{\sphinxupquote{, }}\sphinxstyleliteralemphasis{\sphinxupquote{optional}}) – Initial operational mode. Default is ‘nom’

\item {} 
\sphinxAtStartPar
\sphinxstyleliteralstrong{\sphinxupquote{name}} (\sphinxstyleliteralemphasis{\sphinxupquote{str}}\sphinxstyleliteralemphasis{\sphinxupquote{, }}\sphinxstyleliteralemphasis{\sphinxupquote{optional}}) – (for components only) Name of the component. The default is ‘’.

\item {} 
\sphinxAtStartPar
\sphinxstyleliteralstrong{\sphinxupquote{probtype}} (\sphinxstyleliteralemphasis{\sphinxupquote{str}}\sphinxstyleliteralemphasis{\sphinxupquote{, }}\sphinxstyleliteralemphasis{\sphinxupquote{optional}}) – Type of probability in the probability model, a per\sphinxhyphen{}time ‘rate’ or per\sphinxhyphen{}run ‘prob’.
The default is ‘rate’

\item {} 
\sphinxAtStartPar
\sphinxstyleliteralstrong{\sphinxupquote{units}} (\sphinxstyleliteralemphasis{\sphinxupquote{str}}\sphinxstyleliteralemphasis{\sphinxupquote{, }}\sphinxstyleliteralemphasis{\sphinxupquote{optional}}) – Type of units (‘sec’/’min’/’hr’/’day’) used for the rates. Default is ‘hr’

\item {} 
\sphinxAtStartPar
\sphinxstyleliteralstrong{\sphinxupquote{exclusive}} (\sphinxstyleliteralemphasis{\sphinxupquote{True/False}}) – Whether fault modes are exclusive of each other or not. Default is False (i.e. more than one can be present).

\item {} 
\sphinxAtStartPar
\sphinxstyleliteralstrong{\sphinxupquote{key\_phases\_by}} (\sphinxstyleliteralemphasis{\sphinxupquote{'self'/'none'/'global'/'fxnname'}}) – Phases to key the faultmodes by (using local, global, or an external function’s modes’). Default is ‘global’

\item {} 
\sphinxAtStartPar
\sphinxstyleliteralstrong{\sphinxupquote{longnames}} (\sphinxstyleliteralemphasis{\sphinxupquote{dict}}) – Longer names for the faults (if desired). \{faultname: longname\}

\end{itemize}

\end{description}\end{quote}

\end{fulllineitems}

\index{assoc\_rand\_state() (fmdtools.modeldef.Block method)@\spxentry{assoc\_rand\_state()}\spxextra{fmdtools.modeldef.Block method}}

\begin{fulllineitems}
\phantomsection\label{\detokenize{docs/fmdtools:fmdtools.modeldef.Block.assoc_rand_state}}\pysiglinewithargsret{\sphinxbfcode{\sphinxupquote{assoc\_rand\_state}}}{\emph{\DUrole{n}{name}}, \emph{\DUrole{n}{default}}, \emph{\DUrole{n}{seed}\DUrole{o}{=}\DUrole{default_value}{None}}, \emph{\DUrole{n}{auto\_update}\DUrole{o}{=}\DUrole{default_value}{{[}{]}}}}{}
\sphinxAtStartPar
Associate a stochastic state with the Block. Enables the simulation of stochastic behavior over time.
\begin{quote}\begin{description}
\item[{Parameters}] \leavevmode\begin{itemize}
\item {} 
\sphinxAtStartPar
\sphinxstyleliteralstrong{\sphinxupquote{name}} (\sphinxstyleliteralemphasis{\sphinxupquote{str}}) – name for the parameter to use in the model behavior.

\item {} 
\sphinxAtStartPar
\sphinxstyleliteralstrong{\sphinxupquote{default}} (\sphinxstyleliteralemphasis{\sphinxupquote{int/float/str/etc}}) – Default value for the parameter for the parameter

\item {} 
\sphinxAtStartPar
\sphinxstyleliteralstrong{\sphinxupquote{seed}} (\sphinxstyleliteralemphasis{\sphinxupquote{int}}) – seed for the random state generator to use. Defaults to None.

\item {} 
\sphinxAtStartPar
\sphinxstyleliteralstrong{\sphinxupquote{auto\_update}} (\sphinxstyleliteralemphasis{\sphinxupquote{list}}\sphinxstyleliteralemphasis{\sphinxupquote{, }}\sphinxstyleliteralemphasis{\sphinxupquote{optional}}) – 
\sphinxAtStartPar
If given, updates the state with the given numpy method at each time\sphinxhyphen{}step.
List is made up of two arguments:
\sphinxhyphen{} generator\_method : str
\begin{quote}

\sphinxAtStartPar
Name of the numpy random method to use.
see: \sphinxurl{https://numpy.org/doc/stable/reference/random/generator.html}
\end{quote}
\begin{itemize}
\item {} \begin{description}
\item[{generator\_params}] \leavevmode{[}tuple{]}
\sphinxAtStartPar
Parameter inputs for the numpy generator function

\end{description}

\end{itemize}


\end{itemize}

\end{description}\end{quote}

\end{fulllineitems}

\index{assoc\_rand\_states() (fmdtools.modeldef.Block method)@\spxentry{assoc\_rand\_states()}\spxextra{fmdtools.modeldef.Block method}}

\begin{fulllineitems}
\phantomsection\label{\detokenize{docs/fmdtools:fmdtools.modeldef.Block.assoc_rand_states}}\pysiglinewithargsret{\sphinxbfcode{\sphinxupquote{assoc\_rand\_states}}}{\emph{\DUrole{o}{*}\DUrole{n}{states}}}{}
\sphinxAtStartPar
Associates multiple random states with the model
\begin{quote}\begin{description}
\item[{Parameters}] \leavevmode
\sphinxAtStartPar
\sphinxstyleliteralstrong{\sphinxupquote{*states}} (\sphinxstyleliteralemphasis{\sphinxupquote{tuple}}) – 
\sphinxAtStartPar
can give any number of tuples for each of the states.
The tuple is of the form (name, default), where:
\begin{quote}
\begin{description}
\item[{name}] \leavevmode{[}str{]}
\sphinxAtStartPar
name for the parameter to use in the model behavior.

\item[{default}] \leavevmode{[}int/float/str/etc{]}
\sphinxAtStartPar
Default value for the parameter

\end{description}
\end{quote}


\end{description}\end{quote}

\end{fulllineitems}

\index{get\_flowtypes() (fmdtools.modeldef.Block method)@\spxentry{get\_flowtypes()}\spxextra{fmdtools.modeldef.Block method}}

\begin{fulllineitems}
\phantomsection\label{\detokenize{docs/fmdtools:fmdtools.modeldef.Block.get_flowtypes}}\pysiglinewithargsret{\sphinxbfcode{\sphinxupquote{get\_flowtypes}}}{}{}
\sphinxAtStartPar
Returns the names of the flow types in the model

\end{fulllineitems}

\index{has\_fault() (fmdtools.modeldef.Block method)@\spxentry{has\_fault()}\spxextra{fmdtools.modeldef.Block method}}

\begin{fulllineitems}
\phantomsection\label{\detokenize{docs/fmdtools:fmdtools.modeldef.Block.has_fault}}\pysiglinewithargsret{\sphinxbfcode{\sphinxupquote{has\_fault}}}{\emph{\DUrole{o}{*}\DUrole{n}{faults}}}{}
\sphinxAtStartPar
Check if the block has fault (a str)
\begin{quote}\begin{description}
\item[{Parameters}] \leavevmode
\sphinxAtStartPar
\sphinxstyleliteralstrong{\sphinxupquote{*faults}} (\sphinxstyleliteralemphasis{\sphinxupquote{strs}}) – names of the fault to check.

\end{description}\end{quote}

\end{fulllineitems}

\index{in\_mode() (fmdtools.modeldef.Block method)@\spxentry{in\_mode()}\spxextra{fmdtools.modeldef.Block method}}

\begin{fulllineitems}
\phantomsection\label{\detokenize{docs/fmdtools:fmdtools.modeldef.Block.in_mode}}\pysiglinewithargsret{\sphinxbfcode{\sphinxupquote{in\_mode}}}{\emph{\DUrole{o}{*}\DUrole{n}{modes}}}{}
\sphinxAtStartPar
Checks if the system is in a given operational mode
\begin{quote}\begin{description}
\item[{Parameters}] \leavevmode
\sphinxAtStartPar
\sphinxstyleliteralstrong{\sphinxupquote{*modes}} (\sphinxstyleliteralemphasis{\sphinxupquote{strs}}) – names of the mode to check

\end{description}\end{quote}

\end{fulllineitems}

\index{no\_fault() (fmdtools.modeldef.Block method)@\spxentry{no\_fault()}\spxextra{fmdtools.modeldef.Block method}}

\begin{fulllineitems}
\phantomsection\label{\detokenize{docs/fmdtools:fmdtools.modeldef.Block.no_fault}}\pysiglinewithargsret{\sphinxbfcode{\sphinxupquote{no\_fault}}}{\emph{\DUrole{n}{fault}}}{}
\sphinxAtStartPar
Check if the block does not have fault (a str)
\begin{quote}\begin{description}
\item[{Parameters}] \leavevmode
\sphinxAtStartPar
\sphinxstyleliteralstrong{\sphinxupquote{fault}} (\sphinxstyleliteralemphasis{\sphinxupquote{str}}) – name of the fault to check.

\end{description}\end{quote}

\end{fulllineitems}

\index{remove\_any\_faults() (fmdtools.modeldef.Block method)@\spxentry{remove\_any\_faults()}\spxextra{fmdtools.modeldef.Block method}}

\begin{fulllineitems}
\phantomsection\label{\detokenize{docs/fmdtools:fmdtools.modeldef.Block.remove_any_faults}}\pysiglinewithargsret{\sphinxbfcode{\sphinxupquote{remove\_any\_faults}}}{\emph{\DUrole{n}{opermode}\DUrole{o}{=}\DUrole{default_value}{False}}}{}
\sphinxAtStartPar
Resets fault mode to nominal and returns to the given operational mode
\begin{quote}\begin{description}
\item[{Parameters}] \leavevmode
\sphinxAtStartPar
\sphinxstyleliteralstrong{\sphinxupquote{opermode}} (\sphinxstyleliteralemphasis{\sphinxupquote{str}}\sphinxstyleliteralemphasis{\sphinxupquote{ (}}\sphinxstyleliteralemphasis{\sphinxupquote{optional}}\sphinxstyleliteralemphasis{\sphinxupquote{)}}) – operational mode to return to when the fault mode is removed

\end{description}\end{quote}

\end{fulllineitems}

\index{remove\_fault() (fmdtools.modeldef.Block method)@\spxentry{remove\_fault()}\spxextra{fmdtools.modeldef.Block method}}

\begin{fulllineitems}
\phantomsection\label{\detokenize{docs/fmdtools:fmdtools.modeldef.Block.remove_fault}}\pysiglinewithargsret{\sphinxbfcode{\sphinxupquote{remove\_fault}}}{\emph{\DUrole{n}{fault\_to\_remove}}, \emph{\DUrole{n}{opermode}\DUrole{o}{=}\DUrole{default_value}{False}}}{}
\sphinxAtStartPar
Removes fault in the set of faults and returns to given operational mode
\begin{quote}\begin{description}
\item[{Parameters}] \leavevmode\begin{itemize}
\item {} 
\sphinxAtStartPar
\sphinxstyleliteralstrong{\sphinxupquote{fault\_to\_replace}} (\sphinxstyleliteralemphasis{\sphinxupquote{str}}) – name of the fault to remove

\item {} 
\sphinxAtStartPar
\sphinxstyleliteralstrong{\sphinxupquote{opermode}} (\sphinxstyleliteralemphasis{\sphinxupquote{str}}\sphinxstyleliteralemphasis{\sphinxupquote{ (}}\sphinxstyleliteralemphasis{\sphinxupquote{optional}}\sphinxstyleliteralemphasis{\sphinxupquote{)}}) – operational mode to return to when the fault mode is removed

\end{itemize}

\end{description}\end{quote}

\end{fulllineitems}

\index{replace\_fault() (fmdtools.modeldef.Block method)@\spxentry{replace\_fault()}\spxextra{fmdtools.modeldef.Block method}}

\begin{fulllineitems}
\phantomsection\label{\detokenize{docs/fmdtools:fmdtools.modeldef.Block.replace_fault}}\pysiglinewithargsret{\sphinxbfcode{\sphinxupquote{replace\_fault}}}{\emph{\DUrole{n}{fault\_to\_replace}}, \emph{\DUrole{n}{fault\_to\_add}}}{}
\sphinxAtStartPar
Replaces fault\_to\_replace with fault\_to\_add in the set of faults
\begin{quote}\begin{description}
\item[{Parameters}] \leavevmode\begin{itemize}
\item {} 
\sphinxAtStartPar
\sphinxstyleliteralstrong{\sphinxupquote{fault\_to\_replace}} (\sphinxstyleliteralemphasis{\sphinxupquote{str}}) – name of the fault to replace

\item {} 
\sphinxAtStartPar
\sphinxstyleliteralstrong{\sphinxupquote{fault\_to\_add}} (\sphinxstyleliteralemphasis{\sphinxupquote{str}}) – name of the fault to add in its place

\end{itemize}

\end{description}\end{quote}

\end{fulllineitems}

\index{reset() (fmdtools.modeldef.Block method)@\spxentry{reset()}\spxextra{fmdtools.modeldef.Block method}}

\begin{fulllineitems}
\phantomsection\label{\detokenize{docs/fmdtools:fmdtools.modeldef.Block.reset}}\pysiglinewithargsret{\sphinxbfcode{\sphinxupquote{reset}}}{}{}
\sphinxAtStartPar
Resets the block to the initial state with no faults. Used by default in
derived objects when resetting the model. Requires associated flows to be cleared first.

\end{fulllineitems}

\index{return\_states() (fmdtools.modeldef.Block method)@\spxentry{return\_states()}\spxextra{fmdtools.modeldef.Block method}}

\begin{fulllineitems}
\phantomsection\label{\detokenize{docs/fmdtools:fmdtools.modeldef.Block.return_states}}\pysiglinewithargsret{\sphinxbfcode{\sphinxupquote{return\_states}}}{}{}
\sphinxAtStartPar
Returns states of the block at the current state. Used (iteratively) to record states over time.
\begin{quote}\begin{description}
\item[{Returns}] \leavevmode
\sphinxAtStartPar
\begin{itemize}
\item {} 
\sphinxAtStartPar
\sphinxstylestrong{states} (\sphinxstyleemphasis{dict}) – States (variables) of the block

\item {} 
\sphinxAtStartPar
\sphinxstylestrong{faults} (\sphinxstyleemphasis{set}) – Faults present in the block

\end{itemize}


\end{description}\end{quote}

\end{fulllineitems}

\index{set\_mode() (fmdtools.modeldef.Block method)@\spxentry{set\_mode()}\spxextra{fmdtools.modeldef.Block method}}

\begin{fulllineitems}
\phantomsection\label{\detokenize{docs/fmdtools:fmdtools.modeldef.Block.set_mode}}\pysiglinewithargsret{\sphinxbfcode{\sphinxupquote{set\_mode}}}{\emph{\DUrole{n}{mode}}}{}
\sphinxAtStartPar
Sets a mode in the block
\begin{quote}\begin{description}
\item[{Parameters}] \leavevmode
\sphinxAtStartPar
\sphinxstyleliteralstrong{\sphinxupquote{mode}} (\sphinxstyleliteralemphasis{\sphinxupquote{str}}) – name of the mode to enter.

\end{description}\end{quote}

\end{fulllineitems}

\index{set\_rand() (fmdtools.modeldef.Block method)@\spxentry{set\_rand()}\spxextra{fmdtools.modeldef.Block method}}

\begin{fulllineitems}
\phantomsection\label{\detokenize{docs/fmdtools:fmdtools.modeldef.Block.set_rand}}\pysiglinewithargsret{\sphinxbfcode{\sphinxupquote{set\_rand}}}{\emph{\DUrole{n}{statename}}, \emph{\DUrole{n}{methodname}}, \emph{\DUrole{o}{*}\DUrole{n}{args}}}{}
\sphinxAtStartPar
Update the given random state with a given method and arguments
\begin{quote}\begin{description}
\item[{Parameters}] \leavevmode\begin{itemize}
\item {} 
\sphinxAtStartPar
\sphinxstyleliteralstrong{\sphinxupquote{statename}} (\sphinxstyleliteralemphasis{\sphinxupquote{str}}) – name of the random state defined in assoc\_rand\_state(s)

\item {} 
\sphinxAtStartPar
\sphinxstyleliteralstrong{\sphinxupquote{methodname}} – str name of the numpy method to call in the rng

\item {} 
\sphinxAtStartPar
\sphinxstyleliteralstrong{\sphinxupquote{*args}} (\sphinxstyleliteralemphasis{\sphinxupquote{args}}) – arguments for the numpy method

\end{itemize}

\end{description}\end{quote}

\end{fulllineitems}

\index{to\_default() (fmdtools.modeldef.Block method)@\spxentry{to\_default()}\spxextra{fmdtools.modeldef.Block method}}

\begin{fulllineitems}
\phantomsection\label{\detokenize{docs/fmdtools:fmdtools.modeldef.Block.to_default}}\pysiglinewithargsret{\sphinxbfcode{\sphinxupquote{to\_default}}}{\emph{\DUrole{o}{*}\DUrole{n}{statenames}}}{}
\sphinxAtStartPar
Resets (given or all by default) random states to their default values
\begin{quote}\begin{description}
\item[{Parameters}] \leavevmode
\sphinxAtStartPar
\sphinxstyleliteralstrong{\sphinxupquote{*statenames}} (\sphinxstyleliteralemphasis{\sphinxupquote{str}}\sphinxstyleliteralemphasis{\sphinxupquote{, }}\sphinxstyleliteralemphasis{\sphinxupquote{str}}\sphinxstyleliteralemphasis{\sphinxupquote{, }}\sphinxstyleliteralemphasis{\sphinxupquote{str}}\sphinxstyleliteralemphasis{\sphinxupquote{...}}) – names of the random state defined in assoc\_rand\_state(s)

\end{description}\end{quote}

\end{fulllineitems}

\index{to\_fault() (fmdtools.modeldef.Block method)@\spxentry{to\_fault()}\spxextra{fmdtools.modeldef.Block method}}

\begin{fulllineitems}
\phantomsection\label{\detokenize{docs/fmdtools:fmdtools.modeldef.Block.to_fault}}\pysiglinewithargsret{\sphinxbfcode{\sphinxupquote{to\_fault}}}{\emph{\DUrole{n}{fault}}}{}
\sphinxAtStartPar
Moves from the current fault mode to a new fault mode
\begin{quote}\begin{description}
\item[{Parameters}] \leavevmode
\sphinxAtStartPar
\sphinxstyleliteralstrong{\sphinxupquote{fault}} (\sphinxstyleliteralemphasis{\sphinxupquote{str}}) – name of the fault mode to switch to

\end{description}\end{quote}

\end{fulllineitems}


\end{fulllineitems}

\index{Common (class in fmdtools.modeldef)@\spxentry{Common}\spxextra{class in fmdtools.modeldef}}

\begin{fulllineitems}
\phantomsection\label{\detokenize{docs/fmdtools:fmdtools.modeldef.Common}}\pysigline{\sphinxbfcode{\sphinxupquote{class }}\sphinxcode{\sphinxupquote{fmdtools.modeldef.}}\sphinxbfcode{\sphinxupquote{Common}}}
\sphinxAtStartPar
Bases: \sphinxcode{\sphinxupquote{object}}
\index{add() (fmdtools.modeldef.Common method)@\spxentry{add()}\spxextra{fmdtools.modeldef.Common method}}

\begin{fulllineitems}
\phantomsection\label{\detokenize{docs/fmdtools:fmdtools.modeldef.Common.add}}\pysiglinewithargsret{\sphinxbfcode{\sphinxupquote{add}}}{\emph{\DUrole{o}{*}\DUrole{n}{states}}}{}
\sphinxAtStartPar
Returns the addition of given attributes of the model construct
e.g.,   a = self.add(‘x’,’y’,’z’) is the same as
\begin{quote}

\sphinxAtStartPar
a = self.x+self.y+self.z
\end{quote}

\end{fulllineitems}

\index{assign() (fmdtools.modeldef.Common method)@\spxentry{assign()}\spxextra{fmdtools.modeldef.Common method}}

\begin{fulllineitems}
\phantomsection\label{\detokenize{docs/fmdtools:fmdtools.modeldef.Common.assign}}\pysiglinewithargsret{\sphinxbfcode{\sphinxupquote{assign}}}{\emph{\DUrole{n}{obj}}, \emph{\DUrole{o}{*}\DUrole{n}{states}}}{}
\sphinxAtStartPar
Sets the same\sphinxhyphen{}named values of the current flow/function object to those of a given flow.
Further arguments specify which values.
e.g. self.EE1.assign(EE2, ‘v’, ‘a’) is the same as saying
\begin{quote}

\sphinxAtStartPar
self.EE1.a = self.EE2.a; self.EE1.v = self.EE2.v
\end{quote}

\end{fulllineitems}

\index{different() (fmdtools.modeldef.Common method)@\spxentry{different()}\spxextra{fmdtools.modeldef.Common method}}

\begin{fulllineitems}
\phantomsection\label{\detokenize{docs/fmdtools:fmdtools.modeldef.Common.different}}\pysiglinewithargsret{\sphinxbfcode{\sphinxupquote{different}}}{\emph{\DUrole{n}{values}}, \emph{\DUrole{o}{*}\DUrole{n}{states}}}{}
\sphinxAtStartPar
Tests whether a given iterable values has any different value the
given states in the model construct.
e.g.,   self.same({[}1,2{]},’a’,’b’) is the same as
\begin{quote}

\sphinxAtStartPar
any({[}1,2{]}!={[}self.a, self.b{]})
\end{quote}

\end{fulllineitems}

\index{div() (fmdtools.modeldef.Common method)@\spxentry{div()}\spxextra{fmdtools.modeldef.Common method}}

\begin{fulllineitems}
\phantomsection\label{\detokenize{docs/fmdtools:fmdtools.modeldef.Common.div}}\pysiglinewithargsret{\sphinxbfcode{\sphinxupquote{div}}}{\emph{\DUrole{o}{*}\DUrole{n}{states}}}{}
\sphinxAtStartPar
Returns the division of given attributes of the model construct
e.g.,   a = self.div(‘x’,’y’,’z’) is the same as
\begin{quote}

\sphinxAtStartPar
a = (self.x/self.y)/self.z
\end{quote}

\end{fulllineitems}

\index{get() (fmdtools.modeldef.Common method)@\spxentry{get()}\spxextra{fmdtools.modeldef.Common method}}

\begin{fulllineitems}
\phantomsection\label{\detokenize{docs/fmdtools:fmdtools.modeldef.Common.get}}\pysiglinewithargsret{\sphinxbfcode{\sphinxupquote{get}}}{\emph{\DUrole{o}{*}\DUrole{n}{attnames}}, \emph{\DUrole{o}{**}\DUrole{n}{kwargs}}}{}
\sphinxAtStartPar
Returns the given attribute names (strings). Mainly useful for reducing length
of lines/adding clarity to assignment statements.
e.g., x,y = self.Pos.get(‘x’,’y’) is the same as
\begin{quote}

\sphinxAtStartPar
x,y = self.Pos.x, self.Pos.y, or
z = self.Pos.get(‘x’,’y’) is the same as
z = np.array({[}self.Pos.x, self.Pos.y{]})
\end{quote}

\end{fulllineitems}

\index{gett() (fmdtools.modeldef.Common method)@\spxentry{gett()}\spxextra{fmdtools.modeldef.Common method}}

\begin{fulllineitems}
\phantomsection\label{\detokenize{docs/fmdtools:fmdtools.modeldef.Common.gett}}\pysiglinewithargsret{\sphinxbfcode{\sphinxupquote{gett}}}{\emph{\DUrole{o}{*}\DUrole{n}{attnames}}}{}
\sphinxAtStartPar
Alternative to self.get that returns the given constructs as a tuple instead
of as an array. Useful when a numpy array would translate the underlying data types
poorly (e.g., np.array({[}1,’b’{]} would make 1 a string–using a tuple instead preserves
the data type)

\end{fulllineitems}

\index{inc() (fmdtools.modeldef.Common method)@\spxentry{inc()}\spxextra{fmdtools.modeldef.Common method}}

\begin{fulllineitems}
\phantomsection\label{\detokenize{docs/fmdtools:fmdtools.modeldef.Common.inc}}\pysiglinewithargsret{\sphinxbfcode{\sphinxupquote{inc}}}{\emph{\DUrole{o}{**}\DUrole{n}{kwargs}}}{}
\sphinxAtStartPar
Increments the given arguments by a given value. Mainly useful for
reducing length/adding clarity to increment statements.
e.g., self.Pos.inc(x=1,y=1) is the same as
\begin{quote}

\sphinxAtStartPar
self.Pos.x+=1; self.Pos.y+=1, or
self.Pos.x = self.Pos.x + 1; self.Pos.y = self.Pos.y +1
\end{quote}

\sphinxAtStartPar
Can additionally be provided with a second value denoting a limit on the increments
e.g. self.Pos.inc(x=(1,10)) will increment x by 1 until it reaches 10

\end{fulllineitems}

\index{limit() (fmdtools.modeldef.Common method)@\spxentry{limit()}\spxextra{fmdtools.modeldef.Common method}}

\begin{fulllineitems}
\phantomsection\label{\detokenize{docs/fmdtools:fmdtools.modeldef.Common.limit}}\pysiglinewithargsret{\sphinxbfcode{\sphinxupquote{limit}}}{\emph{\DUrole{o}{**}\DUrole{n}{kwargs}}}{}
\sphinxAtStartPar
Enforces limits on the value of a given property. Mainly useful for
reducing length/adding clarity to increment statements.
e.g., self.EE.limit(a=(0,100), v=(0,12)) is the same as
\begin{quote}

\sphinxAtStartPar
self.EE.a = min(100, max(0,self.EE.a));
self.EE.v = min(12, max(0,self.EE.v))
\end{quote}

\end{fulllineitems}

\index{mul() (fmdtools.modeldef.Common method)@\spxentry{mul()}\spxextra{fmdtools.modeldef.Common method}}

\begin{fulllineitems}
\phantomsection\label{\detokenize{docs/fmdtools:fmdtools.modeldef.Common.mul}}\pysiglinewithargsret{\sphinxbfcode{\sphinxupquote{mul}}}{\emph{\DUrole{o}{*}\DUrole{n}{states}}}{}
\sphinxAtStartPar
Returns the multiplication of given attributes of the model construct.
e.g.,   a = self.mul(‘x’,’y’,’z’) is the same as
\begin{quote}

\sphinxAtStartPar
a = self.x*self.y*self.z
\end{quote}

\end{fulllineitems}

\index{put() (fmdtools.modeldef.Common method)@\spxentry{put()}\spxextra{fmdtools.modeldef.Common method}}

\begin{fulllineitems}
\phantomsection\label{\detokenize{docs/fmdtools:fmdtools.modeldef.Common.put}}\pysiglinewithargsret{\sphinxbfcode{\sphinxupquote{put}}}{\emph{\DUrole{o}{**}\DUrole{n}{kwargs}}}{}
\sphinxAtStartPar
Sets the given arguments to a given value. Mainly useful for
reducing length/adding clarity to assignment statements.
e.g., self.EE.put(v=1, a=1) is the same as saying
\begin{quote}

\sphinxAtStartPar
self.EE.v=1; self.EE.a=1
\end{quote}

\end{fulllineitems}

\index{same() (fmdtools.modeldef.Common method)@\spxentry{same()}\spxextra{fmdtools.modeldef.Common method}}

\begin{fulllineitems}
\phantomsection\label{\detokenize{docs/fmdtools:fmdtools.modeldef.Common.same}}\pysiglinewithargsret{\sphinxbfcode{\sphinxupquote{same}}}{\emph{\DUrole{n}{values}}, \emph{\DUrole{o}{*}\DUrole{n}{states}}}{}
\sphinxAtStartPar
Tests whether a given iterable values has the same value as each
give state in the model construct.
e.g.,   self.same({[}1,2{]},’a’,’b’) is the same as
\begin{quote}

\sphinxAtStartPar
all({[}1,2{]}=={[}self.a, self.b{]})
\end{quote}

\end{fulllineitems}

\index{set\_atts() (fmdtools.modeldef.Common method)@\spxentry{set\_atts()}\spxextra{fmdtools.modeldef.Common method}}

\begin{fulllineitems}
\phantomsection\label{\detokenize{docs/fmdtools:fmdtools.modeldef.Common.set_atts}}\pysiglinewithargsret{\sphinxbfcode{\sphinxupquote{set\_atts}}}{\emph{\DUrole{o}{**}\DUrole{n}{kwargs}}}{}
\sphinxAtStartPar
Sets the given arguments to a given value. Mainly useful for
reducing length/adding clarity to assignment statements in \_\_init\_\_ methods
(self.put is reccomended otherwise so that the iteration is on function/flow \sphinxstyleemphasis{states})
e.g., self.set\_attr(maxpower=1, maxvoltage=1) is the same as saying
\begin{quote}

\sphinxAtStartPar
self.maxpower=1; self.maxvoltage=1
\end{quote}

\end{fulllineitems}

\index{sub() (fmdtools.modeldef.Common method)@\spxentry{sub()}\spxextra{fmdtools.modeldef.Common method}}

\begin{fulllineitems}
\phantomsection\label{\detokenize{docs/fmdtools:fmdtools.modeldef.Common.sub}}\pysiglinewithargsret{\sphinxbfcode{\sphinxupquote{sub}}}{\emph{\DUrole{o}{*}\DUrole{n}{states}}}{}
\sphinxAtStartPar
Returns the addition of given attributes of the model construct
e.g.,   a = self.div(‘x’,’y’,’z’) is the same as
\begin{quote}

\sphinxAtStartPar
a = (self.x\sphinxhyphen{}self.y)\sphinxhyphen{}self.z
\end{quote}

\end{fulllineitems}


\end{fulllineitems}

\index{Component (class in fmdtools.modeldef)@\spxentry{Component}\spxextra{class in fmdtools.modeldef}}

\begin{fulllineitems}
\phantomsection\label{\detokenize{docs/fmdtools:fmdtools.modeldef.Component}}\pysiglinewithargsret{\sphinxbfcode{\sphinxupquote{class }}\sphinxcode{\sphinxupquote{fmdtools.modeldef.}}\sphinxbfcode{\sphinxupquote{Component}}}{\emph{\DUrole{n}{name}}, \emph{\DUrole{n}{states}\DUrole{o}{=}\DUrole{default_value}{\{\}}}}{}
\sphinxAtStartPar
Bases: {\hyperref[\detokenize{docs/fmdtools:fmdtools.modeldef.Block}]{\sphinxcrossref{\sphinxcode{\sphinxupquote{fmdtools.modeldef.Block}}}}}

\sphinxAtStartPar
Superclass for components (most attributes and methods inherited from Block superclass)
\index{behavior() (fmdtools.modeldef.Component method)@\spxentry{behavior()}\spxextra{fmdtools.modeldef.Component method}}

\begin{fulllineitems}
\phantomsection\label{\detokenize{docs/fmdtools:fmdtools.modeldef.Component.behavior}}\pysiglinewithargsret{\sphinxbfcode{\sphinxupquote{behavior}}}{\emph{\DUrole{n}{time}}}{}
\sphinxAtStartPar
Placeholder for component behavior methods. Enables one to include components
without yet having a defined behavior for them.

\end{fulllineitems}


\end{fulllineitems}

\index{Flow (class in fmdtools.modeldef)@\spxentry{Flow}\spxextra{class in fmdtools.modeldef}}

\begin{fulllineitems}
\phantomsection\label{\detokenize{docs/fmdtools:fmdtools.modeldef.Flow}}\pysiglinewithargsret{\sphinxbfcode{\sphinxupquote{class }}\sphinxcode{\sphinxupquote{fmdtools.modeldef.}}\sphinxbfcode{\sphinxupquote{Flow}}}{\emph{\DUrole{n}{states}}, \emph{\DUrole{n}{name}}, \emph{\DUrole{n}{ftype}\DUrole{o}{=}\DUrole{default_value}{'generic'}}}{}
\sphinxAtStartPar
Bases: {\hyperref[\detokenize{docs/fmdtools:fmdtools.modeldef.Common}]{\sphinxcrossref{\sphinxcode{\sphinxupquote{fmdtools.modeldef.Common}}}}}

\sphinxAtStartPar
Superclass for flows. Instanced by Model.add\_flow but can also be used as a flow superclass if flow attributes are not easily definable as a dict.
\index{copy() (fmdtools.modeldef.Flow method)@\spxentry{copy()}\spxextra{fmdtools.modeldef.Flow method}}

\begin{fulllineitems}
\phantomsection\label{\detokenize{docs/fmdtools:fmdtools.modeldef.Flow.copy}}\pysiglinewithargsret{\sphinxbfcode{\sphinxupquote{copy}}}{}{}
\sphinxAtStartPar
Returns a copy of the flow object (used when copying the model)

\end{fulllineitems}

\index{reset() (fmdtools.modeldef.Flow method)@\spxentry{reset()}\spxextra{fmdtools.modeldef.Flow method}}

\begin{fulllineitems}
\phantomsection\label{\detokenize{docs/fmdtools:fmdtools.modeldef.Flow.reset}}\pysiglinewithargsret{\sphinxbfcode{\sphinxupquote{reset}}}{}{}
\sphinxAtStartPar
Resets the flow to the initial state

\end{fulllineitems}

\index{status() (fmdtools.modeldef.Flow method)@\spxentry{status()}\spxextra{fmdtools.modeldef.Flow method}}

\begin{fulllineitems}
\phantomsection\label{\detokenize{docs/fmdtools:fmdtools.modeldef.Flow.status}}\pysiglinewithargsret{\sphinxbfcode{\sphinxupquote{status}}}{}{}
\sphinxAtStartPar
Returns a dict with the current states of the flow.

\end{fulllineitems}


\end{fulllineitems}

\index{FxnBlock (class in fmdtools.modeldef)@\spxentry{FxnBlock}\spxextra{class in fmdtools.modeldef}}

\begin{fulllineitems}
\phantomsection\label{\detokenize{docs/fmdtools:fmdtools.modeldef.FxnBlock}}\pysiglinewithargsret{\sphinxbfcode{\sphinxupquote{class }}\sphinxcode{\sphinxupquote{fmdtools.modeldef.}}\sphinxbfcode{\sphinxupquote{FxnBlock}}}{\emph{\DUrole{n}{name}}, \emph{\DUrole{n}{flows}}, \emph{\DUrole{n}{flownames}\DUrole{o}{=}\DUrole{default_value}{{[}{]}}}, \emph{\DUrole{n}{states}\DUrole{o}{=}\DUrole{default_value}{\{\}}}, \emph{\DUrole{n}{components}\DUrole{o}{=}\DUrole{default_value}{\{\}}}, \emph{\DUrole{n}{timers}\DUrole{o}{=}\DUrole{default_value}{\{\}}}, \emph{\DUrole{n}{tstep}\DUrole{o}{=}\DUrole{default_value}{1.0}}, \emph{\DUrole{n}{seed}\DUrole{o}{=}\DUrole{default_value}{None}}}{}
\sphinxAtStartPar
Bases: {\hyperref[\detokenize{docs/fmdtools:fmdtools.modeldef.Block}]{\sphinxcrossref{\sphinxcode{\sphinxupquote{fmdtools.modeldef.Block}}}}}

\sphinxAtStartPar
Superclass for functions.
\begin{description}
\item[{type}] \leavevmode{[}str{]}
\sphinxAtStartPar
labels the function as a function (may not be necessary) Default is ‘function’

\item[{flows}] \leavevmode{[}dict{]}
\sphinxAtStartPar
flows associated with the function. structured \{flow:\{value:XX\}\}

\item[{components}] \leavevmode{[}dict{]}
\sphinxAtStartPar
component instantiations of the function (if any)

\item[{timers}] \leavevmode{[}set{]}
\sphinxAtStartPar
names of timers to be used in the function (if any)

\item[{tstep}] \leavevmode{[}float{]}
\sphinxAtStartPar
timestep of the model in the function (added/overridden by model definition)

\end{description}
\index{copy() (fmdtools.modeldef.FxnBlock method)@\spxentry{copy()}\spxextra{fmdtools.modeldef.FxnBlock method}}

\begin{fulllineitems}
\phantomsection\label{\detokenize{docs/fmdtools:fmdtools.modeldef.FxnBlock.copy}}\pysiglinewithargsret{\sphinxbfcode{\sphinxupquote{copy}}}{\emph{\DUrole{n}{newflows}}, \emph{\DUrole{o}{*}\DUrole{n}{attr}}}{}
\sphinxAtStartPar
Creates a copy of the function object with newflows and arbitrary parameters associated with the copy. Used when copying the model.
\begin{quote}\begin{description}
\item[{Parameters}] \leavevmode\begin{itemize}
\item {} 
\sphinxAtStartPar
\sphinxstyleliteralstrong{\sphinxupquote{newflows}} (\sphinxstyleliteralemphasis{\sphinxupquote{list}}) – list of new flow objects to be associated with the copy of the function

\item {} 
\sphinxAtStartPar
\sphinxstyleliteralstrong{\sphinxupquote{*attr}} (\sphinxstyleliteralemphasis{\sphinxupquote{any}}) – arbitrary parameters to add (if funciton takes in more than flows e.g. design variables)

\end{itemize}

\item[{Returns}] \leavevmode
\sphinxAtStartPar
\sphinxstylestrong{copy} – Copy of the given function with new flows

\item[{Return type}] \leavevmode
\sphinxAtStartPar
{\hyperref[\detokenize{docs/fmdtools:fmdtools.modeldef.FxnBlock}]{\sphinxcrossref{FxnBlock}}}

\end{description}\end{quote}

\end{fulllineitems}

\index{make\_flowdict() (fmdtools.modeldef.FxnBlock method)@\spxentry{make\_flowdict()}\spxextra{fmdtools.modeldef.FxnBlock method}}

\begin{fulllineitems}
\phantomsection\label{\detokenize{docs/fmdtools:fmdtools.modeldef.FxnBlock.make_flowdict}}\pysiglinewithargsret{\sphinxbfcode{\sphinxupquote{make\_flowdict}}}{\emph{\DUrole{n}{flownames}}, \emph{\DUrole{n}{flows}}}{}
\sphinxAtStartPar
Puts a list of flows with a list of flow names in a dictionary.
\begin{quote}\begin{description}
\item[{Parameters}] \leavevmode\begin{itemize}
\item {} 
\sphinxAtStartPar
\sphinxstyleliteralstrong{\sphinxupquote{flownames}} (\sphinxstyleliteralemphasis{\sphinxupquote{list}}\sphinxstyleliteralemphasis{\sphinxupquote{ or }}\sphinxstyleliteralemphasis{\sphinxupquote{dict}}\sphinxstyleliteralemphasis{\sphinxupquote{ or }}\sphinxstyleliteralemphasis{\sphinxupquote{empty}}) – names of flows corresponding to flows
using \{externalname: internalname\}

\item {} 
\sphinxAtStartPar
\sphinxstyleliteralstrong{\sphinxupquote{flows}} (\sphinxstyleliteralemphasis{\sphinxupquote{list}}) – flows

\end{itemize}

\item[{Returns}] \leavevmode
\sphinxAtStartPar
\sphinxstylestrong{flowdict} – dict of flows indexed by flownames

\item[{Return type}] \leavevmode
\sphinxAtStartPar
dict

\end{description}\end{quote}

\end{fulllineitems}

\index{update\_modestates() (fmdtools.modeldef.FxnBlock method)@\spxentry{update\_modestates()}\spxextra{fmdtools.modeldef.FxnBlock method}}

\begin{fulllineitems}
\phantomsection\label{\detokenize{docs/fmdtools:fmdtools.modeldef.FxnBlock.update_modestates}}\pysiglinewithargsret{\sphinxbfcode{\sphinxupquote{update\_modestates}}}{}{}
\sphinxAtStartPar
Updates states of the model associated with a specific fault mode (see assoc\_modes).

\end{fulllineitems}

\index{update\_stochastic\_states() (fmdtools.modeldef.FxnBlock method)@\spxentry{update\_stochastic\_states()}\spxextra{fmdtools.modeldef.FxnBlock method}}

\begin{fulllineitems}
\phantomsection\label{\detokenize{docs/fmdtools:fmdtools.modeldef.FxnBlock.update_stochastic_states}}\pysiglinewithargsret{\sphinxbfcode{\sphinxupquote{update\_stochastic\_states}}}{}{}
\sphinxAtStartPar
Updates the defined stochastic states defined to auto\sphinxhyphen{}update (see assoc\_randstates).

\end{fulllineitems}

\index{updatefxn() (fmdtools.modeldef.FxnBlock method)@\spxentry{updatefxn()}\spxextra{fmdtools.modeldef.FxnBlock method}}

\begin{fulllineitems}
\phantomsection\label{\detokenize{docs/fmdtools:fmdtools.modeldef.FxnBlock.updatefxn}}\pysiglinewithargsret{\sphinxbfcode{\sphinxupquote{updatefxn}}}{\emph{\DUrole{n}{proptype}}, \emph{\DUrole{n}{faults}\DUrole{o}{=}\DUrole{default_value}{{[}{]}}}, \emph{\DUrole{n}{time}\DUrole{o}{=}\DUrole{default_value}{0}}, \emph{\DUrole{n}{run\_stochastic}\DUrole{o}{=}\DUrole{default_value}{False}}}{}
\sphinxAtStartPar
Updates the state of the function at a given time and injects faults.
\begin{quote}\begin{description}
\item[{Parameters}] \leavevmode\begin{itemize}
\item {} 
\sphinxAtStartPar
\sphinxstyleliteralstrong{\sphinxupquote{faults}} (\sphinxstyleliteralemphasis{\sphinxupquote{list}}\sphinxstyleliteralemphasis{\sphinxupquote{, }}\sphinxstyleliteralemphasis{\sphinxupquote{optional}}) – Faults to inject in the function. The default is {[}‘nom’{]}.

\item {} 
\sphinxAtStartPar
\sphinxstyleliteralstrong{\sphinxupquote{time}} (\sphinxstyleliteralemphasis{\sphinxupquote{float}}\sphinxstyleliteralemphasis{\sphinxupquote{, }}\sphinxstyleliteralemphasis{\sphinxupquote{optional}}) – Model time. The default is 0.

\end{itemize}

\end{description}\end{quote}

\end{fulllineitems}


\end{fulllineitems}

\index{GenericFxn (class in fmdtools.modeldef)@\spxentry{GenericFxn}\spxextra{class in fmdtools.modeldef}}

\begin{fulllineitems}
\phantomsection\label{\detokenize{docs/fmdtools:fmdtools.modeldef.GenericFxn}}\pysiglinewithargsret{\sphinxbfcode{\sphinxupquote{class }}\sphinxcode{\sphinxupquote{fmdtools.modeldef.}}\sphinxbfcode{\sphinxupquote{GenericFxn}}}{\emph{\DUrole{n}{name}}, \emph{\DUrole{n}{flows}}}{}
\sphinxAtStartPar
Bases: {\hyperref[\detokenize{docs/fmdtools:fmdtools.modeldef.FxnBlock}]{\sphinxcrossref{\sphinxcode{\sphinxupquote{fmdtools.modeldef.FxnBlock}}}}}

\sphinxAtStartPar
Generic function block. For use when the user has not yet defined a class for the
given (to be implemented) function block. Acts as a placeholder that enables simulation.

\end{fulllineitems}

\index{Model (class in fmdtools.modeldef)@\spxentry{Model}\spxextra{class in fmdtools.modeldef}}

\begin{fulllineitems}
\phantomsection\label{\detokenize{docs/fmdtools:fmdtools.modeldef.Model}}\pysiglinewithargsret{\sphinxbfcode{\sphinxupquote{class }}\sphinxcode{\sphinxupquote{fmdtools.modeldef.}}\sphinxbfcode{\sphinxupquote{Model}}}{\emph{\DUrole{n}{params}\DUrole{o}{=}\DUrole{default_value}{\{\}}}, \emph{\DUrole{n}{modelparams}\DUrole{o}{=}\DUrole{default_value}{\{\}}}, \emph{\DUrole{n}{valparams}\DUrole{o}{=}\DUrole{default_value}{'all'}}}{}
\sphinxAtStartPar
Bases: \sphinxcode{\sphinxupquote{object}}

\sphinxAtStartPar
Model superclass used to construct the model, return representations of the model, and copy and reset the model when run.
\index{type (fmdtools.modeldef.Model attribute)@\spxentry{type}\spxextra{fmdtools.modeldef.Model attribute}}

\begin{fulllineitems}
\phantomsection\label{\detokenize{docs/fmdtools:fmdtools.modeldef.Model.type}}\pysigline{\sphinxbfcode{\sphinxupquote{type}}}
\sphinxAtStartPar
labels the model as a model (may not be necessary)
\begin{quote}\begin{description}
\item[{Type}] \leavevmode
\sphinxAtStartPar
str

\end{description}\end{quote}

\end{fulllineitems}

\index{flows (fmdtools.modeldef.Model attribute)@\spxentry{flows}\spxextra{fmdtools.modeldef.Model attribute}}

\begin{fulllineitems}
\phantomsection\label{\detokenize{docs/fmdtools:fmdtools.modeldef.Model.flows}}\pysigline{\sphinxbfcode{\sphinxupquote{flows}}}
\sphinxAtStartPar
dictionary of flows objects in the model indexed by name
\begin{quote}\begin{description}
\item[{Type}] \leavevmode
\sphinxAtStartPar
dict

\end{description}\end{quote}

\end{fulllineitems}

\index{fxns (fmdtools.modeldef.Model attribute)@\spxentry{fxns}\spxextra{fmdtools.modeldef.Model attribute}}

\begin{fulllineitems}
\phantomsection\label{\detokenize{docs/fmdtools:fmdtools.modeldef.Model.fxns}}\pysigline{\sphinxbfcode{\sphinxupquote{fxns}}}
\sphinxAtStartPar
dictionary of functions in the model indexed by name
\begin{quote}\begin{description}
\item[{Type}] \leavevmode
\sphinxAtStartPar
dict

\end{description}\end{quote}

\end{fulllineitems}



\begin{fulllineitems}
\pysigline{\sphinxbfcode{\sphinxupquote{params,modelparams,valparams}}}
\sphinxAtStartPar
dictionaries of (optional) parameters for a given instantiation of a model
\begin{quote}\begin{description}
\item[{Type}] \leavevmode
\sphinxAtStartPar
dict

\end{description}\end{quote}

\end{fulllineitems}

\index{modelparams (fmdtools.modeldef.Model attribute)@\spxentry{modelparams}\spxextra{fmdtools.modeldef.Model attribute}}

\begin{fulllineitems}
\phantomsection\label{\detokenize{docs/fmdtools:fmdtools.modeldef.Model.modelparams}}\pysigline{\sphinxbfcode{\sphinxupquote{modelparams}}}~\begin{description}
\item[{dictionary of parameters for running a simulation. defines these parameters in the model:}] \leavevmode\begin{description}
\item[{phases}] \leavevmode{[}dict{]}
\sphinxAtStartPar
phases \{‘name’:{[}start, end{]}\} that the simulation progresses through

\item[{times}] \leavevmode{[}array{]}
\sphinxAtStartPar
array of times to sample (if desired) {[}starttime, sampletime1, sampletime2,… endtime{]}

\item[{tstep}] \leavevmode{[}float{]}
\sphinxAtStartPar
timestep used in the simulation. default is 1.0

\item[{units}] \leavevmode{[}str{]}
\sphinxAtStartPar
time\sphinxhyphen{}units. default is hours

\item[{use\_end\_condition}] \leavevmode{[}bool{]}
\sphinxAtStartPar
whether to use an end\sphinxhyphen{}condition method (defined by user\sphinxhyphen{}defined end\_condition method)
or defined end time to end the simulation

\item[{seed}] \leavevmode{[}int{]}
\sphinxAtStartPar
seed used for the internal random number generator

\end{description}

\end{description}
\begin{quote}\begin{description}
\item[{Type}] \leavevmode
\sphinxAtStartPar
dict

\end{description}\end{quote}

\end{fulllineitems}

\index{valparams (fmdtools.modeldef.Model attribute)@\spxentry{valparams}\spxextra{fmdtools.modeldef.Model attribute}}

\begin{fulllineitems}
\phantomsection\label{\detokenize{docs/fmdtools:fmdtools.modeldef.Model.valparams}}\pysigline{\sphinxbfcode{\sphinxupquote{valparams}}}
\sphinxAtStartPar
dictionary of parameters for defining what simulation constructs to record for find\_classification

\end{fulllineitems}

\index{bipartite (fmdtools.modeldef.Model attribute)@\spxentry{bipartite}\spxextra{fmdtools.modeldef.Model attribute}}

\begin{fulllineitems}
\phantomsection\label{\detokenize{docs/fmdtools:fmdtools.modeldef.Model.bipartite}}\pysigline{\sphinxbfcode{\sphinxupquote{bipartite}}}
\sphinxAtStartPar
bipartite graph view of the functions and flows
\begin{quote}\begin{description}
\item[{Type}] \leavevmode
\sphinxAtStartPar
networkx graph

\end{description}\end{quote}

\end{fulllineitems}

\index{graph (fmdtools.modeldef.Model attribute)@\spxentry{graph}\spxextra{fmdtools.modeldef.Model attribute}}

\begin{fulllineitems}
\phantomsection\label{\detokenize{docs/fmdtools:fmdtools.modeldef.Model.graph}}\pysigline{\sphinxbfcode{\sphinxupquote{graph}}}
\sphinxAtStartPar
multigraph view of functions and flows
\begin{quote}\begin{description}
\item[{Type}] \leavevmode
\sphinxAtStartPar
networkx graph

\end{description}\end{quote}

\end{fulllineitems}

\index{add\_flow() (fmdtools.modeldef.Model method)@\spxentry{add\_flow()}\spxextra{fmdtools.modeldef.Model method}}

\begin{fulllineitems}
\phantomsection\label{\detokenize{docs/fmdtools:fmdtools.modeldef.Model.add_flow}}\pysiglinewithargsret{\sphinxbfcode{\sphinxupquote{add\_flow}}}{\emph{\DUrole{n}{flowname}}, \emph{\DUrole{n}{flowdict}\DUrole{o}{=}\DUrole{default_value}{\{\}}}, \emph{\DUrole{n}{flowtype}\DUrole{o}{=}\DUrole{default_value}{''}}}{}
\sphinxAtStartPar
Adds a flow with given attributes to the model.
\begin{quote}\begin{description}
\item[{Parameters}] \leavevmode\begin{itemize}
\item {} 
\sphinxAtStartPar
\sphinxstyleliteralstrong{\sphinxupquote{flowname}} (\sphinxstyleliteralemphasis{\sphinxupquote{str}}) – Unique flow name to give the flow in the model

\item {} 
\sphinxAtStartPar
\sphinxstyleliteralstrong{\sphinxupquote{flowattributes}} (\sphinxstyleliteralemphasis{\sphinxupquote{dict}}\sphinxstyleliteralemphasis{\sphinxupquote{, }}{\hyperref[\detokenize{docs/fmdtools:fmdtools.modeldef.Flow}]{\sphinxcrossref{\sphinxstyleliteralemphasis{\sphinxupquote{Flow}}}}}\sphinxstyleliteralemphasis{\sphinxupquote{, }}\sphinxstyleliteralemphasis{\sphinxupquote{set}}\sphinxstyleliteralemphasis{\sphinxupquote{ or }}\sphinxstyleliteralemphasis{\sphinxupquote{empty set}}) – Dictionary of flow attributes e.g. \{‘value’:XX\}, or the Flow object.
If a set of attribute names is provided, each will be given a value of 1
If an empty set is given, it will be represented w\sphinxhyphen{} \{flowname: 1\}

\end{itemize}

\end{description}\end{quote}

\end{fulllineitems}

\index{add\_flows() (fmdtools.modeldef.Model method)@\spxentry{add\_flows()}\spxextra{fmdtools.modeldef.Model method}}

\begin{fulllineitems}
\phantomsection\label{\detokenize{docs/fmdtools:fmdtools.modeldef.Model.add_flows}}\pysiglinewithargsret{\sphinxbfcode{\sphinxupquote{add\_flows}}}{\emph{\DUrole{n}{flownames}}, \emph{\DUrole{n}{flowdict}\DUrole{o}{=}\DUrole{default_value}{\{\}}}, \emph{\DUrole{n}{flowtype}\DUrole{o}{=}\DUrole{default_value}{'generic'}}}{}
\sphinxAtStartPar
Adds a set of flows with the same type and initial parameters
\begin{quote}\begin{description}
\item[{Parameters}] \leavevmode\begin{itemize}
\item {} 
\sphinxAtStartPar
\sphinxstyleliteralstrong{\sphinxupquote{flowname}} (\sphinxstyleliteralemphasis{\sphinxupquote{list}}) – Unique flow names to give the flows in the model

\item {} 
\sphinxAtStartPar
\sphinxstyleliteralstrong{\sphinxupquote{flowattributes}} (\sphinxstyleliteralemphasis{\sphinxupquote{dict}}\sphinxstyleliteralemphasis{\sphinxupquote{, }}{\hyperref[\detokenize{docs/fmdtools:fmdtools.modeldef.Flow}]{\sphinxcrossref{\sphinxstyleliteralemphasis{\sphinxupquote{Flow}}}}}\sphinxstyleliteralemphasis{\sphinxupquote{, }}\sphinxstyleliteralemphasis{\sphinxupquote{set}}\sphinxstyleliteralemphasis{\sphinxupquote{ or }}\sphinxstyleliteralemphasis{\sphinxupquote{empty set}}) – Dictionary of flow attributes e.g. \{‘value’:XX\}, or the Flow object.
If a set of attribute names is provided, each will be given a value of 1
If an empty set is given, it will be represented w\sphinxhyphen{} \{flowname: 1\}

\end{itemize}

\end{description}\end{quote}

\end{fulllineitems}

\index{add\_fxn() (fmdtools.modeldef.Model method)@\spxentry{add\_fxn()}\spxextra{fmdtools.modeldef.Model method}}

\begin{fulllineitems}
\phantomsection\label{\detokenize{docs/fmdtools:fmdtools.modeldef.Model.add_fxn}}\pysiglinewithargsret{\sphinxbfcode{\sphinxupquote{add\_fxn}}}{\emph{name}, \emph{flownames}, \emph{fclass=<class 'fmdtools.modeldef.GenericFxn'>}, \emph{fparams='None'}}{}
\sphinxAtStartPar
Instantiates a given function in the model.
\begin{quote}\begin{description}
\item[{Parameters}] \leavevmode\begin{itemize}
\item {} 
\sphinxAtStartPar
\sphinxstyleliteralstrong{\sphinxupquote{name}} (\sphinxstyleliteralemphasis{\sphinxupquote{str}}) – Name to give the function.

\item {} 
\sphinxAtStartPar
\sphinxstyleliteralstrong{\sphinxupquote{flownames}} (\sphinxstyleliteralemphasis{\sphinxupquote{list}}) – List of flows to associate with the function.

\item {} 
\sphinxAtStartPar
\sphinxstyleliteralstrong{\sphinxupquote{fclass}} (\sphinxstyleliteralemphasis{\sphinxupquote{Class}}) – Class to instantiate the function as.

\item {} 
\sphinxAtStartPar
\sphinxstyleliteralstrong{\sphinxupquote{fparams}} (\sphinxstyleliteralemphasis{\sphinxupquote{arbitrary float}}\sphinxstyleliteralemphasis{\sphinxupquote{, }}\sphinxstyleliteralemphasis{\sphinxupquote{dict}}\sphinxstyleliteralemphasis{\sphinxupquote{, }}\sphinxstyleliteralemphasis{\sphinxupquote{list}}\sphinxstyleliteralemphasis{\sphinxupquote{, }}\sphinxstyleliteralemphasis{\sphinxupquote{etc.}}) – Other parameters to send to the \_\_init\_\_ method of the function class

\end{itemize}

\end{description}\end{quote}

\end{fulllineitems}

\index{build\_model() (fmdtools.modeldef.Model method)@\spxentry{build\_model()}\spxextra{fmdtools.modeldef.Model method}}

\begin{fulllineitems}
\phantomsection\label{\detokenize{docs/fmdtools:fmdtools.modeldef.Model.build_model}}\pysiglinewithargsret{\sphinxbfcode{\sphinxupquote{build\_model}}}{\emph{\DUrole{n}{functionorder}\DUrole{o}{=}\DUrole{default_value}{{[}{]}}}, \emph{\DUrole{n}{graph\_pos}\DUrole{o}{=}\DUrole{default_value}{\{\}}}, \emph{\DUrole{n}{bipartite\_pos}\DUrole{o}{=}\DUrole{default_value}{\{\}}}}{}
\sphinxAtStartPar
Builds the model graph after the functions have been added.
\begin{quote}\begin{description}
\item[{Parameters}] \leavevmode\begin{itemize}
\item {} 
\sphinxAtStartPar
\sphinxstyleliteralstrong{\sphinxupquote{functionorder}} (\sphinxstyleliteralemphasis{\sphinxupquote{list}}\sphinxstyleliteralemphasis{\sphinxupquote{, }}\sphinxstyleliteralemphasis{\sphinxupquote{optional}}) – The order for the functions to be executed in. The default is {[}{]}.

\item {} 
\sphinxAtStartPar
\sphinxstyleliteralstrong{\sphinxupquote{graph\_pos}} (\sphinxstyleliteralemphasis{\sphinxupquote{dict}}\sphinxstyleliteralemphasis{\sphinxupquote{, }}\sphinxstyleliteralemphasis{\sphinxupquote{optional}}) – position of graph nodes. The default is \{\}.

\item {} 
\sphinxAtStartPar
\sphinxstyleliteralstrong{\sphinxupquote{bipartite\_pos}} (\sphinxstyleliteralemphasis{\sphinxupquote{dict}}\sphinxstyleliteralemphasis{\sphinxupquote{, }}\sphinxstyleliteralemphasis{\sphinxupquote{optional}}) – position of bipartite graph nodes. The default is \{\}.

\end{itemize}

\end{description}\end{quote}

\end{fulllineitems}

\index{calc\_repaircost() (fmdtools.modeldef.Model method)@\spxentry{calc\_repaircost()}\spxextra{fmdtools.modeldef.Model method}}

\begin{fulllineitems}
\phantomsection\label{\detokenize{docs/fmdtools:fmdtools.modeldef.Model.calc_repaircost}}\pysiglinewithargsret{\sphinxbfcode{\sphinxupquote{calc\_repaircost}}}{\emph{\DUrole{n}{additional\_cost}\DUrole{o}{=}\DUrole{default_value}{0}}, \emph{\DUrole{n}{default\_cost}\DUrole{o}{=}\DUrole{default_value}{0}}, \emph{\DUrole{n}{max\_cost}\DUrole{o}{=}\DUrole{default_value}{inf}}}{}
\sphinxAtStartPar
Calculates the repair cost of the fault modes in the model based on given
mode cost information for each function mode (in fxn.assoc\_faultmodes).
\begin{quote}\begin{description}
\item[{Parameters}] \leavevmode\begin{itemize}
\item {} 
\sphinxAtStartPar
\sphinxstyleliteralstrong{\sphinxupquote{additional\_cost}} (\sphinxstyleliteralemphasis{\sphinxupquote{int/float}}) – Additional cost to add if there are faults in the model. Default is 0.

\item {} 
\sphinxAtStartPar
\sphinxstyleliteralstrong{\sphinxupquote{default\_cost}} (\sphinxstyleliteralemphasis{\sphinxupquote{int/float}}) – Cost to use for each fault mode if no fault cost information given
in assoc\_faultmodes/ Default is 0.

\item {} 
\sphinxAtStartPar
\sphinxstyleliteralstrong{\sphinxupquote{max\_cost}} (\sphinxstyleliteralemphasis{\sphinxupquote{int/float}}) – Maximum cost of repair (e.g. cost of replacement). Default is np.inf

\end{itemize}

\item[{Returns}] \leavevmode
\sphinxAtStartPar
\sphinxstylestrong{repair\_cost} – Cost of repairing the fault modes in the given model

\item[{Return type}] \leavevmode
\sphinxAtStartPar
float

\end{description}\end{quote}

\end{fulllineitems}

\index{construct\_graph() (fmdtools.modeldef.Model method)@\spxentry{construct\_graph()}\spxextra{fmdtools.modeldef.Model method}}

\begin{fulllineitems}
\phantomsection\label{\detokenize{docs/fmdtools:fmdtools.modeldef.Model.construct_graph}}\pysiglinewithargsret{\sphinxbfcode{\sphinxupquote{construct\_graph}}}{\emph{\DUrole{n}{graph\_pos}\DUrole{o}{=}\DUrole{default_value}{\{\}}}, \emph{\DUrole{n}{bipartite\_pos}\DUrole{o}{=}\DUrole{default_value}{\{\}}}}{}
\sphinxAtStartPar
Creates and returns a graph representation of the model
\begin{quote}\begin{description}
\item[{Returns}] \leavevmode
\sphinxAtStartPar
\sphinxstylestrong{graph} – multgraph representation of the model functions and flows

\item[{Return type}] \leavevmode
\sphinxAtStartPar
networkx graph

\end{description}\end{quote}

\end{fulllineitems}

\index{copy() (fmdtools.modeldef.Model method)@\spxentry{copy()}\spxextra{fmdtools.modeldef.Model method}}

\begin{fulllineitems}
\phantomsection\label{\detokenize{docs/fmdtools:fmdtools.modeldef.Model.copy}}\pysiglinewithargsret{\sphinxbfcode{\sphinxupquote{copy}}}{}{}
\sphinxAtStartPar
Copies the model at the current state.
\begin{quote}\begin{description}
\item[{Returns}] \leavevmode
\sphinxAtStartPar
\sphinxstylestrong{copy} – Copy of the curent model.

\item[{Return type}] \leavevmode
\sphinxAtStartPar
{\hyperref[\detokenize{docs/fmdtools:fmdtools.modeldef.Model}]{\sphinxcrossref{Model}}}

\end{description}\end{quote}

\end{fulllineitems}

\index{find\_classification() (fmdtools.modeldef.Model method)@\spxentry{find\_classification()}\spxextra{fmdtools.modeldef.Model method}}

\begin{fulllineitems}
\phantomsection\label{\detokenize{docs/fmdtools:fmdtools.modeldef.Model.find_classification}}\pysiglinewithargsret{\sphinxbfcode{\sphinxupquote{find\_classification}}}{\emph{\DUrole{n}{scen}}, \emph{\DUrole{n}{mdlhists}}}{}
\sphinxAtStartPar
Placeholder for model find\_classification methods (for running nominal models)

\end{fulllineitems}

\index{flows\_of\_type() (fmdtools.modeldef.Model method)@\spxentry{flows\_of\_type()}\spxextra{fmdtools.modeldef.Model method}}

\begin{fulllineitems}
\phantomsection\label{\detokenize{docs/fmdtools:fmdtools.modeldef.Model.flows_of_type}}\pysiglinewithargsret{\sphinxbfcode{\sphinxupquote{flows\_of\_type}}}{\emph{\DUrole{n}{ftype}}}{}
\sphinxAtStartPar
Returns the set of flows for each flow type

\end{fulllineitems}

\index{flowtypes() (fmdtools.modeldef.Model method)@\spxentry{flowtypes()}\spxextra{fmdtools.modeldef.Model method}}

\begin{fulllineitems}
\phantomsection\label{\detokenize{docs/fmdtools:fmdtools.modeldef.Model.flowtypes}}\pysiglinewithargsret{\sphinxbfcode{\sphinxupquote{flowtypes}}}{}{}
\sphinxAtStartPar
Returns the set of flow types used in the model

\end{fulllineitems}

\index{flowtypes\_for\_fxnclasses() (fmdtools.modeldef.Model method)@\spxentry{flowtypes\_for\_fxnclasses()}\spxextra{fmdtools.modeldef.Model method}}

\begin{fulllineitems}
\phantomsection\label{\detokenize{docs/fmdtools:fmdtools.modeldef.Model.flowtypes_for_fxnclasses}}\pysiglinewithargsret{\sphinxbfcode{\sphinxupquote{flowtypes\_for\_fxnclasses}}}{}{}
\sphinxAtStartPar
Returns the flows required by each function class in the model (as a dict)

\end{fulllineitems}

\index{fxnclasses() (fmdtools.modeldef.Model method)@\spxentry{fxnclasses()}\spxextra{fmdtools.modeldef.Model method}}

\begin{fulllineitems}
\phantomsection\label{\detokenize{docs/fmdtools:fmdtools.modeldef.Model.fxnclasses}}\pysiglinewithargsret{\sphinxbfcode{\sphinxupquote{fxnclasses}}}{}{}
\sphinxAtStartPar
Returns the set of class names used in the model

\end{fulllineitems}

\index{fxns\_of\_class() (fmdtools.modeldef.Model method)@\spxentry{fxns\_of\_class()}\spxextra{fmdtools.modeldef.Model method}}

\begin{fulllineitems}
\phantomsection\label{\detokenize{docs/fmdtools:fmdtools.modeldef.Model.fxns_of_class}}\pysiglinewithargsret{\sphinxbfcode{\sphinxupquote{fxns\_of\_class}}}{\emph{\DUrole{n}{ftype}}}{}
\sphinxAtStartPar
Returns dict of functionname:functionobjects corresponding to the given class name ftype

\end{fulllineitems}

\index{get\_flows() (fmdtools.modeldef.Model method)@\spxentry{get\_flows()}\spxextra{fmdtools.modeldef.Model method}}

\begin{fulllineitems}
\phantomsection\label{\detokenize{docs/fmdtools:fmdtools.modeldef.Model.get_flows}}\pysiglinewithargsret{\sphinxbfcode{\sphinxupquote{get\_flows}}}{\emph{\DUrole{n}{flownames}}}{}
\sphinxAtStartPar
Returns a list of the model flow objects

\end{fulllineitems}

\index{reset() (fmdtools.modeldef.Model method)@\spxentry{reset()}\spxextra{fmdtools.modeldef.Model method}}

\begin{fulllineitems}
\phantomsection\label{\detokenize{docs/fmdtools:fmdtools.modeldef.Model.reset}}\pysiglinewithargsret{\sphinxbfcode{\sphinxupquote{reset}}}{}{}
\sphinxAtStartPar
Resets the model to the initial state (with no faults, etc)

\end{fulllineitems}

\index{return\_componentgraph() (fmdtools.modeldef.Model method)@\spxentry{return\_componentgraph()}\spxextra{fmdtools.modeldef.Model method}}

\begin{fulllineitems}
\phantomsection\label{\detokenize{docs/fmdtools:fmdtools.modeldef.Model.return_componentgraph}}\pysiglinewithargsret{\sphinxbfcode{\sphinxupquote{return\_componentgraph}}}{\emph{\DUrole{n}{fxnname}}}{}
\sphinxAtStartPar
Returns a graph representation of the components associated with a given funciton
\begin{quote}\begin{description}
\item[{Parameters}] \leavevmode
\sphinxAtStartPar
\sphinxstyleliteralstrong{\sphinxupquote{fxnname}} (\sphinxstyleliteralemphasis{\sphinxupquote{str}}) – Name of the function (e.g. in mdl.fxns)

\item[{Returns}] \leavevmode
\sphinxAtStartPar
\sphinxstylestrong{g} – Bipartite graph representation of the function with components.

\item[{Return type}] \leavevmode
\sphinxAtStartPar
networkx graph

\end{description}\end{quote}

\end{fulllineitems}

\index{return\_faultmodes() (fmdtools.modeldef.Model method)@\spxentry{return\_faultmodes()}\spxextra{fmdtools.modeldef.Model method}}

\begin{fulllineitems}
\phantomsection\label{\detokenize{docs/fmdtools:fmdtools.modeldef.Model.return_faultmodes}}\pysiglinewithargsret{\sphinxbfcode{\sphinxupquote{return\_faultmodes}}}{}{}
\sphinxAtStartPar
Returns faultmodes present in the model
\begin{quote}\begin{description}
\item[{Returns}] \leavevmode
\sphinxAtStartPar
\begin{itemize}
\item {} 
\sphinxAtStartPar
\sphinxstylestrong{modes} (\sphinxstyleemphasis{dict}) – Fault modes present in the model indexed by function name

\item {} 
\sphinxAtStartPar
\sphinxstylestrong{modeprops} (\sphinxstyleemphasis{dict}) – Fault mode properties (defined in the function definition) with structure \{fxn:mode:properties\}

\end{itemize}


\end{description}\end{quote}

\end{fulllineitems}

\index{return\_paramgraph() (fmdtools.modeldef.Model method)@\spxentry{return\_paramgraph()}\spxextra{fmdtools.modeldef.Model method}}

\begin{fulllineitems}
\phantomsection\label{\detokenize{docs/fmdtools:fmdtools.modeldef.Model.return_paramgraph}}\pysiglinewithargsret{\sphinxbfcode{\sphinxupquote{return\_paramgraph}}}{}{}
\sphinxAtStartPar
Returns a graph representation of the flows in the model, where flows are nodes and edges are
associations in functions

\end{fulllineitems}

\index{return\_stategraph() (fmdtools.modeldef.Model method)@\spxentry{return\_stategraph()}\spxextra{fmdtools.modeldef.Model method}}

\begin{fulllineitems}
\phantomsection\label{\detokenize{docs/fmdtools:fmdtools.modeldef.Model.return_stategraph}}\pysiglinewithargsret{\sphinxbfcode{\sphinxupquote{return\_stategraph}}}{\emph{\DUrole{n}{gtype}\DUrole{o}{=}\DUrole{default_value}{'bipartite'}}}{}
\sphinxAtStartPar
Returns a graph representation of the current state of the model.
\begin{quote}\begin{description}
\item[{Parameters}] \leavevmode
\sphinxAtStartPar
\sphinxstyleliteralstrong{\sphinxupquote{gtype}} (\sphinxstyleliteralemphasis{\sphinxupquote{str}}\sphinxstyleliteralemphasis{\sphinxupquote{, }}\sphinxstyleliteralemphasis{\sphinxupquote{optional}}) – Type of graph to return (normal, bipartite, component, or typegraph). The default is ‘bipartite’.

\item[{Returns}] \leavevmode
\sphinxAtStartPar
\sphinxstylestrong{graph} – Graph representation of the system with the modes and states added as attributes.

\item[{Return type}] \leavevmode
\sphinxAtStartPar
networkx graph

\end{description}\end{quote}

\end{fulllineitems}

\index{return\_typegraph() (fmdtools.modeldef.Model method)@\spxentry{return\_typegraph()}\spxextra{fmdtools.modeldef.Model method}}

\begin{fulllineitems}
\phantomsection\label{\detokenize{docs/fmdtools:fmdtools.modeldef.Model.return_typegraph}}\pysiglinewithargsret{\sphinxbfcode{\sphinxupquote{return\_typegraph}}}{\emph{\DUrole{n}{withflows}\DUrole{o}{=}\DUrole{default_value}{True}}}{}
\sphinxAtStartPar
Returns a graph with the type containment relationships of the different model constructs.
\begin{quote}\begin{description}
\item[{Parameters}] \leavevmode
\sphinxAtStartPar
\sphinxstyleliteralstrong{\sphinxupquote{withflows}} (\sphinxstyleliteralemphasis{\sphinxupquote{bool}}\sphinxstyleliteralemphasis{\sphinxupquote{, }}\sphinxstyleliteralemphasis{\sphinxupquote{optional}}) – Whether to include flows. The default is True.

\item[{Returns}] \leavevmode
\sphinxAtStartPar
\sphinxstylestrong{g} – networkx directed graph of the type relationships

\item[{Return type}] \leavevmode
\sphinxAtStartPar
nx.DiGraph

\end{description}\end{quote}

\end{fulllineitems}

\index{set\_functionorder() (fmdtools.modeldef.Model method)@\spxentry{set\_functionorder()}\spxextra{fmdtools.modeldef.Model method}}

\begin{fulllineitems}
\phantomsection\label{\detokenize{docs/fmdtools:fmdtools.modeldef.Model.set_functionorder}}\pysiglinewithargsret{\sphinxbfcode{\sphinxupquote{set\_functionorder}}}{\emph{\DUrole{n}{functionorder}}}{}
\sphinxAtStartPar
Manually sets the order of functions to be executed (otherwise it will be executed based on the sequence of add\_fxn calls)

\end{fulllineitems}


\end{fulllineitems}

\index{NominalApproach (class in fmdtools.modeldef)@\spxentry{NominalApproach}\spxextra{class in fmdtools.modeldef}}

\begin{fulllineitems}
\phantomsection\label{\detokenize{docs/fmdtools:fmdtools.modeldef.NominalApproach}}\pysigline{\sphinxbfcode{\sphinxupquote{class }}\sphinxcode{\sphinxupquote{fmdtools.modeldef.}}\sphinxbfcode{\sphinxupquote{NominalApproach}}}
\sphinxAtStartPar
Bases: \sphinxcode{\sphinxupquote{object}}

\sphinxAtStartPar
Class for defining sets of nominal simulations. To explain, a given system
may have a number of input situations (missions, terrain, etc) which the
user may want to simulate to ensure the system operates as desired. This
class (in conjunction with propagate.nominal\_approach()) can be used to
perform these simulations.
\index{scenarios (fmdtools.modeldef.NominalApproach attribute)@\spxentry{scenarios}\spxextra{fmdtools.modeldef.NominalApproach attribute}}

\begin{fulllineitems}
\phantomsection\label{\detokenize{docs/fmdtools:fmdtools.modeldef.NominalApproach.scenarios}}\pysigline{\sphinxbfcode{\sphinxupquote{scenarios}}}
\sphinxAtStartPar
scenarios to inject based on the approach
\begin{quote}\begin{description}
\item[{Type}] \leavevmode
\sphinxAtStartPar
dict

\end{description}\end{quote}

\end{fulllineitems}

\index{num\_scenarios (fmdtools.modeldef.NominalApproach attribute)@\spxentry{num\_scenarios}\spxextra{fmdtools.modeldef.NominalApproach attribute}}

\begin{fulllineitems}
\phantomsection\label{\detokenize{docs/fmdtools:fmdtools.modeldef.NominalApproach.num_scenarios}}\pysigline{\sphinxbfcode{\sphinxupquote{num\_scenarios}}}
\sphinxAtStartPar
number of scenarios in the approach
\begin{quote}\begin{description}
\item[{Type}] \leavevmode
\sphinxAtStartPar
int

\end{description}\end{quote}

\end{fulllineitems}

\index{ranges (fmdtools.modeldef.NominalApproach attribute)@\spxentry{ranges}\spxextra{fmdtools.modeldef.NominalApproach attribute}}

\begin{fulllineitems}
\phantomsection\label{\detokenize{docs/fmdtools:fmdtools.modeldef.NominalApproach.ranges}}\pysigline{\sphinxbfcode{\sphinxupquote{ranges}}}
\sphinxAtStartPar
dict of the parameters defined in each method for the approach
\begin{quote}\begin{description}
\item[{Type}] \leavevmode
\sphinxAtStartPar
dict

\end{description}\end{quote}

\end{fulllineitems}

\index{add\_param\_ranges() (fmdtools.modeldef.NominalApproach method)@\spxentry{add\_param\_ranges()}\spxextra{fmdtools.modeldef.NominalApproach method}}

\begin{fulllineitems}
\phantomsection\label{\detokenize{docs/fmdtools:fmdtools.modeldef.NominalApproach.add_param_ranges}}\pysiglinewithargsret{\sphinxbfcode{\sphinxupquote{add\_param\_ranges}}}{\emph{\DUrole{n}{paramfunc}}, \emph{\DUrole{n}{rangeid}}, \emph{\DUrole{o}{*}\DUrole{n}{args}}, \emph{\DUrole{n}{replicates}\DUrole{o}{=}\DUrole{default_value}{1}}, \emph{\DUrole{n}{seeds}\DUrole{o}{=}\DUrole{default_value}{'shared'}}, \emph{\DUrole{o}{**}\DUrole{n}{kwargs}}}{}
\sphinxAtStartPar
Adds a set of scenarios to the approach.
\begin{quote}\begin{description}
\item[{Parameters}] \leavevmode\begin{itemize}
\item {} 
\sphinxAtStartPar
\sphinxstyleliteralstrong{\sphinxupquote{paramfunc}} (\sphinxstyleliteralemphasis{\sphinxupquote{method}}) – Python method which generates a set of model parameters given the input arguments.
method should have form: method(fixedarg, fixedarg…, inputarg=X, inputarg=X)

\item {} 
\sphinxAtStartPar
\sphinxstyleliteralstrong{\sphinxupquote{rangeid}} (\sphinxstyleliteralemphasis{\sphinxupquote{str}}) – Name for the range being used. Default is ‘nominal’

\item {} 
\sphinxAtStartPar
\sphinxstyleliteralstrong{\sphinxupquote{*args}} (\sphinxstyleliteralemphasis{\sphinxupquote{specifies values for positional args of paramfunc.}}) – May be given as a fixed float/int/dict/str defining a set value for positional arguments

\item {} 
\sphinxAtStartPar
\sphinxstyleliteralstrong{\sphinxupquote{replicates}} (\sphinxstyleliteralemphasis{\sphinxupquote{int}}) – Number of points to take over each range (for random parameters). Default is 1.

\item {} 
\sphinxAtStartPar
\sphinxstyleliteralstrong{\sphinxupquote{seeds}} (\sphinxstyleliteralemphasis{\sphinxupquote{str/list}}) – \begin{description}
\item[{Options for seeding models/replicates: (Default is ‘shared’)}] \leavevmode\begin{itemize}
\item {} 
\sphinxAtStartPar
’shared’ creates random seeds and shares them between parameters and models

\item {} 
\sphinxAtStartPar
’independent’ creates separate random seeds for models and parameter generation

\item {} 
\sphinxAtStartPar
’keep\_model’ uses the seed provided in the model for all of the model

\end{itemize}

\end{description}

\sphinxAtStartPar
When a list is provided, these seeds are are used (and shared). Must be of length replicates.


\item {} 
\sphinxAtStartPar
\sphinxstyleliteralstrong{\sphinxupquote{**kwargs}} (\sphinxstyleliteralemphasis{\sphinxupquote{specifies range for keyword args of paramfunc}}) – May be given as a fixed float/int/dict/str (k=value) defining a set value for the range (if not the default) or
as a tuple k=(start, end, step)

\end{itemize}

\end{description}\end{quote}

\end{fulllineitems}

\index{add\_param\_replicates() (fmdtools.modeldef.NominalApproach method)@\spxentry{add\_param\_replicates()}\spxextra{fmdtools.modeldef.NominalApproach method}}

\begin{fulllineitems}
\phantomsection\label{\detokenize{docs/fmdtools:fmdtools.modeldef.NominalApproach.add_param_replicates}}\pysiglinewithargsret{\sphinxbfcode{\sphinxupquote{add\_param\_replicates}}}{\emph{\DUrole{n}{paramfunc}}, \emph{\DUrole{n}{rangeid}}, \emph{\DUrole{n}{replicates}}, \emph{\DUrole{o}{*}\DUrole{n}{args}}, \emph{\DUrole{n}{ind\_seeds}\DUrole{o}{=}\DUrole{default_value}{True}}, \emph{\DUrole{o}{**}\DUrole{n}{kwargs}}}{}
\sphinxAtStartPar
Adds a set of repeated scenarios to the approach. For use in (external) random scenario generation.
\begin{quote}\begin{description}
\item[{Parameters}] \leavevmode\begin{itemize}
\item {} 
\sphinxAtStartPar
\sphinxstyleliteralstrong{\sphinxupquote{paramfunc}} (\sphinxstyleliteralemphasis{\sphinxupquote{method}}) – Python method which generates a set of model parameters given the input arguments.
method should have form: method(fixedarg, fixedarg…, inputarg=X, inputarg=X)

\item {} 
\sphinxAtStartPar
\sphinxstyleliteralstrong{\sphinxupquote{rangeid}} (\sphinxstyleliteralemphasis{\sphinxupquote{str}}) – Name for the set of replicates

\item {} 
\sphinxAtStartPar
\sphinxstyleliteralstrong{\sphinxupquote{replicates}} (\sphinxstyleliteralemphasis{\sphinxupquote{int}}) – Number of replicates to use

\item {} 
\sphinxAtStartPar
\sphinxstyleliteralstrong{\sphinxupquote{*args}} (\sphinxstyleliteralemphasis{\sphinxupquote{any}}) – arguments to send to paramfunc

\item {} 
\sphinxAtStartPar
\sphinxstyleliteralstrong{\sphinxupquote{ind\_seeds}} (\sphinxstyleliteralemphasis{\sphinxupquote{Bool/list}}) – Whether the models should be run with different seeds (rather than the same seed). Default is True
When a list is provided, these seeds are are used. Must be of length replicates.

\item {} 
\sphinxAtStartPar
\sphinxstyleliteralstrong{\sphinxupquote{**kwargs}} (\sphinxstyleliteralemphasis{\sphinxupquote{any}}) – keyword arguments to send to paramfunc

\end{itemize}

\end{description}\end{quote}

\end{fulllineitems}

\index{add\_rand\_params() (fmdtools.modeldef.NominalApproach method)@\spxentry{add\_rand\_params()}\spxextra{fmdtools.modeldef.NominalApproach method}}

\begin{fulllineitems}
\phantomsection\label{\detokenize{docs/fmdtools:fmdtools.modeldef.NominalApproach.add_rand_params}}\pysiglinewithargsret{\sphinxbfcode{\sphinxupquote{add\_rand\_params}}}{\emph{\DUrole{n}{paramfunc}}, \emph{\DUrole{n}{rangeid}}, \emph{\DUrole{o}{*}\DUrole{n}{fixedargs}}, \emph{\DUrole{n}{prob\_weight}\DUrole{o}{=}\DUrole{default_value}{1.0}}, \emph{\DUrole{n}{replicates}\DUrole{o}{=}\DUrole{default_value}{1000}}, \emph{\DUrole{n}{seeds}\DUrole{o}{=}\DUrole{default_value}{'shared'}}, \emph{\DUrole{o}{**}\DUrole{n}{randvars}}}{}
\sphinxAtStartPar
Adds a set of random scenarios to the approach.
\begin{quote}\begin{description}
\item[{Parameters}] \leavevmode\begin{itemize}
\item {} 
\sphinxAtStartPar
\sphinxstyleliteralstrong{\sphinxupquote{paramfunc}} (\sphinxstyleliteralemphasis{\sphinxupquote{method}}) – Python method which generates a set of model parameters given the input arguments.
method should have form: method(fixedarg, fixedarg…, inputarg=X, inputarg=X)

\item {} 
\sphinxAtStartPar
\sphinxstyleliteralstrong{\sphinxupquote{rangeid}} (\sphinxstyleliteralemphasis{\sphinxupquote{str}}) – Name for the range being used. Default is ‘nominal’

\item {} 
\sphinxAtStartPar
\sphinxstyleliteralstrong{\sphinxupquote{prob\_weight}} (\sphinxstyleliteralemphasis{\sphinxupquote{float}}\sphinxstyleliteralemphasis{\sphinxupquote{ (}}\sphinxstyleliteralemphasis{\sphinxupquote{0\sphinxhyphen{}1}}\sphinxstyleliteralemphasis{\sphinxupquote{)}}) – Overall probability for the set of scenarios (to use if adding more ranges). Default is 1.0

\item {} 
\sphinxAtStartPar
\sphinxstyleliteralstrong{\sphinxupquote{*fixedargs}} (\sphinxstyleliteralemphasis{\sphinxupquote{any}}) – Fixed positional arguments in the parameter generator function.
Useful for discrete modes with different parameters.

\item {} 
\sphinxAtStartPar
\sphinxstyleliteralstrong{\sphinxupquote{seeds}} (\sphinxstyleliteralemphasis{\sphinxupquote{str/list}}) – \begin{description}
\item[{Options for seeding models/replicates: (Default is ‘shared’)}] \leavevmode\begin{itemize}
\item {} 
\sphinxAtStartPar
’shared’ creates random seeds and shares them between parameters and models

\item {} 
\sphinxAtStartPar
’independent’ creates separate random seeds for models and parameter generation

\item {} 
\sphinxAtStartPar
’keep\_model’ uses the seed provided in the model for all of the model

\end{itemize}

\end{description}

\sphinxAtStartPar
When a list is provided, these seeds are are used (and shared). Must be of length replicates.


\item {} 
\sphinxAtStartPar
\sphinxstyleliteralstrong{\sphinxupquote{**randvars}} (\sphinxstyleliteralemphasis{\sphinxupquote{key=tuple}}) – Specification for each random input parameter, specified as
input = (randfunc, param1, param2…)
where randfunc is the method producing random outputs (e.g. numpy.random.rand)
and the successive parameters param1, param2, etc are inputs to the method

\end{itemize}

\end{description}\end{quote}

\end{fulllineitems}

\index{add\_seed\_replicates() (fmdtools.modeldef.NominalApproach method)@\spxentry{add\_seed\_replicates()}\spxextra{fmdtools.modeldef.NominalApproach method}}

\begin{fulllineitems}
\phantomsection\label{\detokenize{docs/fmdtools:fmdtools.modeldef.NominalApproach.add_seed_replicates}}\pysiglinewithargsret{\sphinxbfcode{\sphinxupquote{add\_seed\_replicates}}}{\emph{\DUrole{n}{rangeid}}, \emph{\DUrole{n}{seeds}}}{}
\sphinxAtStartPar
Generates an approach with different seeds to use for the model’s internal stochastic behaviors
\begin{quote}\begin{description}
\item[{Parameters}] \leavevmode\begin{itemize}
\item {} 
\sphinxAtStartPar
\sphinxstyleliteralstrong{\sphinxupquote{rangeid}} (\sphinxstyleliteralemphasis{\sphinxupquote{str}}) – Name for the set of replicates

\item {} 
\sphinxAtStartPar
\sphinxstyleliteralstrong{\sphinxupquote{seeds}} (\sphinxstyleliteralemphasis{\sphinxupquote{int/list}}) – Number of seeds (if an int) or a list of seeds to use.

\end{itemize}

\end{description}\end{quote}

\end{fulllineitems}

\index{assoc\_probs() (fmdtools.modeldef.NominalApproach method)@\spxentry{assoc\_probs()}\spxextra{fmdtools.modeldef.NominalApproach method}}

\begin{fulllineitems}
\phantomsection\label{\detokenize{docs/fmdtools:fmdtools.modeldef.NominalApproach.assoc_probs}}\pysiglinewithargsret{\sphinxbfcode{\sphinxupquote{assoc\_probs}}}{\emph{\DUrole{n}{rangeid}}, \emph{\DUrole{n}{prob\_weight}\DUrole{o}{=}\DUrole{default_value}{1.0}}, \emph{\DUrole{o}{**}\DUrole{n}{inputpdfs}}}{}
\sphinxAtStartPar
Associates a probability model (assuming variable independence) with a
given previously\sphinxhyphen{}defined range of scenarios using given pdfs
\begin{quote}\begin{description}
\item[{Parameters}] \leavevmode\begin{itemize}
\item {} 
\sphinxAtStartPar
\sphinxstyleliteralstrong{\sphinxupquote{rangeid}} (\sphinxstyleliteralemphasis{\sphinxupquote{str}}) – Name of the range to apply the probability model to.

\item {} 
\sphinxAtStartPar
\sphinxstyleliteralstrong{\sphinxupquote{prob\_weight}} (\sphinxstyleliteralemphasis{\sphinxupquote{float}}\sphinxstyleliteralemphasis{\sphinxupquote{, }}\sphinxstyleliteralemphasis{\sphinxupquote{optional}}) – Overall probability for the set of scenarios (to use if adding more ranges
or if the range does not cover the space of probability). The default is 1.0.

\item {} 
\sphinxAtStartPar
\sphinxstyleliteralstrong{\sphinxupquote{**inputpdfs}} (\sphinxstyleliteralemphasis{\sphinxupquote{key=}}\sphinxstyleliteralemphasis{\sphinxupquote{(}}\sphinxstyleliteralemphasis{\sphinxupquote{pdf}}\sphinxstyleliteralemphasis{\sphinxupquote{, }}\sphinxstyleliteralemphasis{\sphinxupquote{params}}\sphinxstyleliteralemphasis{\sphinxupquote{)}}) – pdf to associate with the different variables of the model.
Where the pdf has form pdf(x, {\color{red}\bfseries{}**}kwargs) where x is the location and {\color{red}\bfseries{}**}kwargs is parameters
(for example, scipy.stats.norm.pdf)
and params is a dictionary of parameters (e.g., \{‘mu’:1,’std’:1\}) to use ‘
as the key/parameter inputs to the pdf

\end{itemize}

\end{description}\end{quote}

\end{fulllineitems}

\index{change\_params() (fmdtools.modeldef.NominalApproach method)@\spxentry{change\_params()}\spxextra{fmdtools.modeldef.NominalApproach method}}

\begin{fulllineitems}
\phantomsection\label{\detokenize{docs/fmdtools:fmdtools.modeldef.NominalApproach.change_params}}\pysiglinewithargsret{\sphinxbfcode{\sphinxupquote{change\_params}}}{\emph{\DUrole{n}{rangeid}\DUrole{o}{=}\DUrole{default_value}{'all'}}, \emph{\DUrole{o}{**}\DUrole{n}{kwargs}}}{}
\sphinxAtStartPar
Changes a given parameter accross all scenarios. Modifies ‘params’ (rather than regenerating params from the paramfunc).
\begin{quote}\begin{description}
\item[{Parameters}] \leavevmode\begin{itemize}
\item {} 
\sphinxAtStartPar
\sphinxstyleliteralstrong{\sphinxupquote{rangeid}} (\sphinxstyleliteralemphasis{\sphinxupquote{str}}) – Name of the range to modify. Optional. Defaults to “all”

\item {} 
\sphinxAtStartPar
\sphinxstyleliteralstrong{\sphinxupquote{**kwargs}} (\sphinxstyleliteralemphasis{\sphinxupquote{any}}) – Parameters to change stated as paramname=value or
as a dict paramname=\{‘sub\_param’:value\}, where ‘sub\_param’ is the parameter of the dictionary with name paramname to update

\end{itemize}

\end{description}\end{quote}

\end{fulllineitems}

\index{copy() (fmdtools.modeldef.NominalApproach method)@\spxentry{copy()}\spxextra{fmdtools.modeldef.NominalApproach method}}

\begin{fulllineitems}
\phantomsection\label{\detokenize{docs/fmdtools:fmdtools.modeldef.NominalApproach.copy}}\pysiglinewithargsret{\sphinxbfcode{\sphinxupquote{copy}}}{}{}
\sphinxAtStartPar
Copies the given sampleapproach. Used in nested scenario sampling.

\end{fulllineitems}

\index{get\_param\_scens() (fmdtools.modeldef.NominalApproach method)@\spxentry{get\_param\_scens()}\spxextra{fmdtools.modeldef.NominalApproach method}}

\begin{fulllineitems}
\phantomsection\label{\detokenize{docs/fmdtools:fmdtools.modeldef.NominalApproach.get_param_scens}}\pysiglinewithargsret{\sphinxbfcode{\sphinxupquote{get\_param\_scens}}}{\emph{\DUrole{n}{rangeid}}, \emph{\DUrole{o}{*}\DUrole{n}{level\_params}}}{}
\sphinxAtStartPar
Returns the scenarios of a range associated with given parameter ranges
\begin{quote}\begin{description}
\item[{Parameters}] \leavevmode\begin{itemize}
\item {} 
\sphinxAtStartPar
\sphinxstyleliteralstrong{\sphinxupquote{rangeid}} (\sphinxstyleliteralemphasis{\sphinxupquote{str}}) – Range id to check

\item {} 
\sphinxAtStartPar
\sphinxstyleliteralstrong{\sphinxupquote{level\_params}} (\sphinxstyleliteralemphasis{\sphinxupquote{str}}\sphinxstyleliteralemphasis{\sphinxupquote{ (}}\sphinxstyleliteralemphasis{\sphinxupquote{multiple}}\sphinxstyleliteralemphasis{\sphinxupquote{)}}) – Level parameters iterate over

\end{itemize}

\item[{Returns}] \leavevmode
\sphinxAtStartPar
\sphinxstylestrong{param\_scens} – The scenarios associated with each level of parameter (or joint parameters)

\item[{Return type}] \leavevmode
\sphinxAtStartPar
dict

\end{description}\end{quote}

\end{fulllineitems}


\end{fulllineitems}

\index{SampleApproach (class in fmdtools.modeldef)@\spxentry{SampleApproach}\spxextra{class in fmdtools.modeldef}}

\begin{fulllineitems}
\phantomsection\label{\detokenize{docs/fmdtools:fmdtools.modeldef.SampleApproach}}\pysiglinewithargsret{\sphinxbfcode{\sphinxupquote{class }}\sphinxcode{\sphinxupquote{fmdtools.modeldef.}}\sphinxbfcode{\sphinxupquote{SampleApproach}}}{\emph{\DUrole{n}{mdl}}, \emph{\DUrole{n}{faults}\DUrole{o}{=}\DUrole{default_value}{'all'}}, \emph{\DUrole{n}{phases}\DUrole{o}{=}\DUrole{default_value}{'global'}}, \emph{\DUrole{n}{modephases}\DUrole{o}{=}\DUrole{default_value}{\{\}}}, \emph{\DUrole{n}{jointfaults}\DUrole{o}{=}\DUrole{default_value}{\{'faults': 'None'\}}}, \emph{\DUrole{n}{sampparams}\DUrole{o}{=}\DUrole{default_value}{\{\}}}, \emph{\DUrole{n}{defaultsamp}\DUrole{o}{=}\DUrole{default_value}{\{'numpts': 1, 'samp': 'evenspacing'\}}}}{}
\sphinxAtStartPar
Bases: \sphinxcode{\sphinxupquote{object}}

\sphinxAtStartPar
Class for defining the sample approach to be used for a set of faults.
\index{phases (fmdtools.modeldef.SampleApproach attribute)@\spxentry{phases}\spxextra{fmdtools.modeldef.SampleApproach attribute}}

\begin{fulllineitems}
\phantomsection\label{\detokenize{docs/fmdtools:fmdtools.modeldef.SampleApproach.phases}}\pysigline{\sphinxbfcode{\sphinxupquote{phases}}}
\sphinxAtStartPar
phases given to sample the fault modes in
\begin{quote}\begin{description}
\item[{Type}] \leavevmode
\sphinxAtStartPar
dict

\end{description}\end{quote}

\end{fulllineitems}

\index{globalphases (fmdtools.modeldef.SampleApproach attribute)@\spxentry{globalphases}\spxextra{fmdtools.modeldef.SampleApproach attribute}}

\begin{fulllineitems}
\phantomsection\label{\detokenize{docs/fmdtools:fmdtools.modeldef.SampleApproach.globalphases}}\pysigline{\sphinxbfcode{\sphinxupquote{globalphases}}}
\sphinxAtStartPar
phases defined in the model
\begin{quote}\begin{description}
\item[{Type}] \leavevmode
\sphinxAtStartPar
dict

\end{description}\end{quote}

\end{fulllineitems}

\index{modephases (fmdtools.modeldef.SampleApproach attribute)@\spxentry{modephases}\spxextra{fmdtools.modeldef.SampleApproach attribute}}

\begin{fulllineitems}
\phantomsection\label{\detokenize{docs/fmdtools:fmdtools.modeldef.SampleApproach.modephases}}\pysigline{\sphinxbfcode{\sphinxupquote{modephases}}}
\sphinxAtStartPar
Dictionary of modes associated with each state
\begin{quote}\begin{description}
\item[{Type}] \leavevmode
\sphinxAtStartPar
dict

\end{description}\end{quote}

\end{fulllineitems}

\index{mode\_phase\_map (fmdtools.modeldef.SampleApproach attribute)@\spxentry{mode\_phase\_map}\spxextra{fmdtools.modeldef.SampleApproach attribute}}

\begin{fulllineitems}
\phantomsection\label{\detokenize{docs/fmdtools:fmdtools.modeldef.SampleApproach.mode_phase_map}}\pysigline{\sphinxbfcode{\sphinxupquote{mode\_phase\_map}}}
\sphinxAtStartPar
Mapping of modes to their corresponding phases
\begin{quote}\begin{description}
\item[{Type}] \leavevmode
\sphinxAtStartPar
dict

\end{description}\end{quote}

\end{fulllineitems}

\index{tstep (fmdtools.modeldef.SampleApproach attribute)@\spxentry{tstep}\spxextra{fmdtools.modeldef.SampleApproach attribute}}

\begin{fulllineitems}
\phantomsection\label{\detokenize{docs/fmdtools:fmdtools.modeldef.SampleApproach.tstep}}\pysigline{\sphinxbfcode{\sphinxupquote{tstep}}}
\sphinxAtStartPar
timestep defined in the model
\begin{quote}\begin{description}
\item[{Type}] \leavevmode
\sphinxAtStartPar
float

\end{description}\end{quote}

\end{fulllineitems}

\index{fxnrates (fmdtools.modeldef.SampleApproach attribute)@\spxentry{fxnrates}\spxextra{fmdtools.modeldef.SampleApproach attribute}}

\begin{fulllineitems}
\phantomsection\label{\detokenize{docs/fmdtools:fmdtools.modeldef.SampleApproach.fxnrates}}\pysigline{\sphinxbfcode{\sphinxupquote{fxnrates}}}
\sphinxAtStartPar
overall failure rates for each function
\begin{quote}\begin{description}
\item[{Type}] \leavevmode
\sphinxAtStartPar
dict

\end{description}\end{quote}

\end{fulllineitems}

\index{comprates (fmdtools.modeldef.SampleApproach attribute)@\spxentry{comprates}\spxextra{fmdtools.modeldef.SampleApproach attribute}}

\begin{fulllineitems}
\phantomsection\label{\detokenize{docs/fmdtools:fmdtools.modeldef.SampleApproach.comprates}}\pysigline{\sphinxbfcode{\sphinxupquote{comprates}}}
\sphinxAtStartPar
overall failure rates for each component
\begin{quote}\begin{description}
\item[{Type}] \leavevmode
\sphinxAtStartPar
dict

\end{description}\end{quote}

\end{fulllineitems}

\index{jointmodes (fmdtools.modeldef.SampleApproach attribute)@\spxentry{jointmodes}\spxextra{fmdtools.modeldef.SampleApproach attribute}}

\begin{fulllineitems}
\phantomsection\label{\detokenize{docs/fmdtools:fmdtools.modeldef.SampleApproach.jointmodes}}\pysigline{\sphinxbfcode{\sphinxupquote{jointmodes}}}
\sphinxAtStartPar
(if any) joint fault modes to be injected in the approach
\begin{quote}\begin{description}
\item[{Type}] \leavevmode
\sphinxAtStartPar
list

\end{description}\end{quote}

\end{fulllineitems}



\begin{fulllineitems}
\pysigline{\sphinxbfcode{\sphinxupquote{rates/comprates/rates\_timeless}}}
\sphinxAtStartPar
rates of each mode (fxn, mode) in each model phase, structured \{fxnmode: \{phaseid:rate\}\}
\begin{quote}\begin{description}
\item[{Type}] \leavevmode
\sphinxAtStartPar
dict

\end{description}\end{quote}

\end{fulllineitems}

\index{sampletimes (fmdtools.modeldef.SampleApproach attribute)@\spxentry{sampletimes}\spxextra{fmdtools.modeldef.SampleApproach attribute}}

\begin{fulllineitems}
\phantomsection\label{\detokenize{docs/fmdtools:fmdtools.modeldef.SampleApproach.sampletimes}}\pysigline{\sphinxbfcode{\sphinxupquote{sampletimes}}}
\sphinxAtStartPar
faults to inject at each time in each phase, structured \{phaseid:time:fnxmode\}
\begin{quote}\begin{description}
\item[{Type}] \leavevmode
\sphinxAtStartPar
dict

\end{description}\end{quote}

\end{fulllineitems}

\index{weights (fmdtools.modeldef.SampleApproach attribute)@\spxentry{weights}\spxextra{fmdtools.modeldef.SampleApproach attribute}}

\begin{fulllineitems}
\phantomsection\label{\detokenize{docs/fmdtools:fmdtools.modeldef.SampleApproach.weights}}\pysigline{\sphinxbfcode{\sphinxupquote{weights}}}
\sphinxAtStartPar
weight to put on each time each fault was injected, structured \{fxnmode:phaseid:time:weight\}
\begin{quote}\begin{description}
\item[{Type}] \leavevmode
\sphinxAtStartPar
dict

\end{description}\end{quote}

\end{fulllineitems}

\index{sampparams (fmdtools.modeldef.SampleApproach attribute)@\spxentry{sampparams}\spxextra{fmdtools.modeldef.SampleApproach attribute}}

\begin{fulllineitems}
\phantomsection\label{\detokenize{docs/fmdtools:fmdtools.modeldef.SampleApproach.sampparams}}\pysigline{\sphinxbfcode{\sphinxupquote{sampparams}}}
\sphinxAtStartPar
parameters used to sample each mode
\begin{quote}\begin{description}
\item[{Type}] \leavevmode
\sphinxAtStartPar
dict

\end{description}\end{quote}

\end{fulllineitems}

\index{scenlist (fmdtools.modeldef.SampleApproach attribute)@\spxentry{scenlist}\spxextra{fmdtools.modeldef.SampleApproach attribute}}

\begin{fulllineitems}
\phantomsection\label{\detokenize{docs/fmdtools:fmdtools.modeldef.SampleApproach.scenlist}}\pysigline{\sphinxbfcode{\sphinxupquote{scenlist}}}
\sphinxAtStartPar
list of fault scenarios (dicts of faults and properties) that fault propagation iterates through
\begin{quote}\begin{description}
\item[{Type}] \leavevmode
\sphinxAtStartPar
list

\end{description}\end{quote}

\end{fulllineitems}

\index{scenids (fmdtools.modeldef.SampleApproach attribute)@\spxentry{scenids}\spxextra{fmdtools.modeldef.SampleApproach attribute}}

\begin{fulllineitems}
\phantomsection\label{\detokenize{docs/fmdtools:fmdtools.modeldef.SampleApproach.scenids}}\pysigline{\sphinxbfcode{\sphinxupquote{scenids}}}
\sphinxAtStartPar
a list of scenario ids associated with a given fault in a given phase, structured \{(fxnmode,phaseid):listofnames\}
\begin{quote}\begin{description}
\item[{Type}] \leavevmode
\sphinxAtStartPar
dict

\end{description}\end{quote}

\end{fulllineitems}

\index{mode\_phase\_map (fmdtools.modeldef.SampleApproach attribute)@\spxentry{mode\_phase\_map}\spxextra{fmdtools.modeldef.SampleApproach attribute}}

\begin{fulllineitems}
\phantomsection\label{\detokenize{docs/fmdtools:id0}}\pysigline{\sphinxbfcode{\sphinxupquote{mode\_phase\_map}}}
\sphinxAtStartPar
a dict of modes and their respective phases to inject with structure \{fxnmode:\{mode\_phase\_map:{[}starttime, endtime{]}\}\}
\begin{quote}\begin{description}
\item[{Type}] \leavevmode
\sphinxAtStartPar
dict

\end{description}\end{quote}

\end{fulllineitems}

\index{units (fmdtools.modeldef.SampleApproach attribute)@\spxentry{units}\spxextra{fmdtools.modeldef.SampleApproach attribute}}

\begin{fulllineitems}
\phantomsection\label{\detokenize{docs/fmdtools:fmdtools.modeldef.SampleApproach.units}}\pysigline{\sphinxbfcode{\sphinxupquote{units}}}
\sphinxAtStartPar
time\sphinxhyphen{}units to use in the approach probability model
\begin{quote}\begin{description}
\item[{Type}] \leavevmode
\sphinxAtStartPar
str

\end{description}\end{quote}

\end{fulllineitems}

\index{unit\_factors (fmdtools.modeldef.SampleApproach attribute)@\spxentry{unit\_factors}\spxextra{fmdtools.modeldef.SampleApproach attribute}}

\begin{fulllineitems}
\phantomsection\label{\detokenize{docs/fmdtools:fmdtools.modeldef.SampleApproach.unit_factors}}\pysigline{\sphinxbfcode{\sphinxupquote{unit\_factors}}}
\sphinxAtStartPar
multiplication factors for converting some time units to others.
\begin{quote}\begin{description}
\item[{Type}] \leavevmode
\sphinxAtStartPar
dict

\end{description}\end{quote}

\end{fulllineitems}

\index{add\_phasetimes() (fmdtools.modeldef.SampleApproach method)@\spxentry{add\_phasetimes()}\spxextra{fmdtools.modeldef.SampleApproach method}}

\begin{fulllineitems}
\phantomsection\label{\detokenize{docs/fmdtools:fmdtools.modeldef.SampleApproach.add_phasetimes}}\pysiglinewithargsret{\sphinxbfcode{\sphinxupquote{add\_phasetimes}}}{\emph{\DUrole{n}{fxnmode}}, \emph{\DUrole{n}{phaseid}}, \emph{\DUrole{n}{phasetimes}}, \emph{\DUrole{n}{weights}\DUrole{o}{=}\DUrole{default_value}{{[}{]}}}}{}
\sphinxAtStartPar
Adds a set of times for a given mode to sampletimes

\end{fulllineitems}

\index{create\_nomscen() (fmdtools.modeldef.SampleApproach method)@\spxentry{create\_nomscen()}\spxextra{fmdtools.modeldef.SampleApproach method}}

\begin{fulllineitems}
\phantomsection\label{\detokenize{docs/fmdtools:fmdtools.modeldef.SampleApproach.create_nomscen}}\pysiglinewithargsret{\sphinxbfcode{\sphinxupquote{create\_nomscen}}}{\emph{\DUrole{n}{mdl}}}{}
\sphinxAtStartPar
Creates a nominal scenario

\end{fulllineitems}

\index{create\_sampletimes() (fmdtools.modeldef.SampleApproach method)@\spxentry{create\_sampletimes()}\spxextra{fmdtools.modeldef.SampleApproach method}}

\begin{fulllineitems}
\phantomsection\label{\detokenize{docs/fmdtools:fmdtools.modeldef.SampleApproach.create_sampletimes}}\pysiglinewithargsret{\sphinxbfcode{\sphinxupquote{create\_sampletimes}}}{\emph{\DUrole{n}{mdl}}, \emph{\DUrole{n}{params}\DUrole{o}{=}\DUrole{default_value}{\{\}}}, \emph{\DUrole{n}{default}\DUrole{o}{=}\DUrole{default_value}{\{'numpts': 1, 'samp': 'evenspacing'\}}}}{}
\sphinxAtStartPar
Initializes weights and sampletimes

\end{fulllineitems}

\index{create\_scenarios() (fmdtools.modeldef.SampleApproach method)@\spxentry{create\_scenarios()}\spxextra{fmdtools.modeldef.SampleApproach method}}

\begin{fulllineitems}
\phantomsection\label{\detokenize{docs/fmdtools:fmdtools.modeldef.SampleApproach.create_scenarios}}\pysiglinewithargsret{\sphinxbfcode{\sphinxupquote{create\_scenarios}}}{}{}
\sphinxAtStartPar
Creates list of scenarios to be iterated over in fault injection. Added as scenlist and scenids

\end{fulllineitems}

\index{init\_modelist() (fmdtools.modeldef.SampleApproach method)@\spxentry{init\_modelist()}\spxextra{fmdtools.modeldef.SampleApproach method}}

\begin{fulllineitems}
\phantomsection\label{\detokenize{docs/fmdtools:fmdtools.modeldef.SampleApproach.init_modelist}}\pysiglinewithargsret{\sphinxbfcode{\sphinxupquote{init\_modelist}}}{\emph{\DUrole{n}{mdl}}, \emph{\DUrole{n}{faults}}, \emph{\DUrole{n}{jointfaults}\DUrole{o}{=}\DUrole{default_value}{\{'faults': 'None'\}}}}{}
\sphinxAtStartPar
Initializes comprates, jointmodes internal list of modes

\end{fulllineitems}

\index{init\_rates() (fmdtools.modeldef.SampleApproach method)@\spxentry{init\_rates()}\spxextra{fmdtools.modeldef.SampleApproach method}}

\begin{fulllineitems}
\phantomsection\label{\detokenize{docs/fmdtools:fmdtools.modeldef.SampleApproach.init_rates}}\pysiglinewithargsret{\sphinxbfcode{\sphinxupquote{init\_rates}}}{\emph{\DUrole{n}{mdl}}, \emph{\DUrole{n}{jointfaults}\DUrole{o}{=}\DUrole{default_value}{\{'faults': 'None'\}}}, \emph{\DUrole{n}{modephases}\DUrole{o}{=}\DUrole{default_value}{\{\}}}}{}
\sphinxAtStartPar
Initializes rates, rates\_timeless

\end{fulllineitems}

\index{list\_moderates() (fmdtools.modeldef.SampleApproach method)@\spxentry{list\_moderates()}\spxextra{fmdtools.modeldef.SampleApproach method}}

\begin{fulllineitems}
\phantomsection\label{\detokenize{docs/fmdtools:fmdtools.modeldef.SampleApproach.list_moderates}}\pysiglinewithargsret{\sphinxbfcode{\sphinxupquote{list\_moderates}}}{}{}
\sphinxAtStartPar
Returns the rates for each mode

\end{fulllineitems}

\index{list\_modes() (fmdtools.modeldef.SampleApproach method)@\spxentry{list\_modes()}\spxextra{fmdtools.modeldef.SampleApproach method}}

\begin{fulllineitems}
\phantomsection\label{\detokenize{docs/fmdtools:fmdtools.modeldef.SampleApproach.list_modes}}\pysiglinewithargsret{\sphinxbfcode{\sphinxupquote{list\_modes}}}{\emph{\DUrole{n}{joint}\DUrole{o}{=}\DUrole{default_value}{False}}}{}
\sphinxAtStartPar
Returns a list of modes in the approach

\end{fulllineitems}

\index{prune\_scenarios() (fmdtools.modeldef.SampleApproach method)@\spxentry{prune\_scenarios()}\spxextra{fmdtools.modeldef.SampleApproach method}}

\begin{fulllineitems}
\phantomsection\label{\detokenize{docs/fmdtools:fmdtools.modeldef.SampleApproach.prune_scenarios}}\pysiglinewithargsret{\sphinxbfcode{\sphinxupquote{prune\_scenarios}}}{\emph{\DUrole{n}{endclasses}}, \emph{\DUrole{n}{samptype}\DUrole{o}{=}\DUrole{default_value}{'piecewise'}}, \emph{\DUrole{n}{threshold}\DUrole{o}{=}\DUrole{default_value}{0.1}}, \emph{\DUrole{n}{sampparam}\DUrole{o}{=}\DUrole{default_value}{\{'numpts': 1, 'samp': 'evenspacing'\}}}}{}
\sphinxAtStartPar
Finds the best sample approach to approximate the full integral (given the approach was the full integral).
\begin{quote}\begin{description}
\item[{Parameters}] \leavevmode\begin{itemize}
\item {} 
\sphinxAtStartPar
\sphinxstyleliteralstrong{\sphinxupquote{endclasses}} (\sphinxstyleliteralemphasis{\sphinxupquote{dict}}) – dict of results (cost, rate, expected cost) for the model run indexed by scenid

\item {} 
\sphinxAtStartPar
\sphinxstyleliteralstrong{\sphinxupquote{samptype}} (\sphinxstyleliteralemphasis{\sphinxupquote{str}}\sphinxstyleliteralemphasis{\sphinxupquote{ (}}\sphinxstyleliteralemphasis{\sphinxupquote{'piecewise'}}\sphinxstyleliteralemphasis{\sphinxupquote{ or }}\sphinxstyleliteralemphasis{\sphinxupquote{'bestpt'}}\sphinxstyleliteralemphasis{\sphinxupquote{)}}\sphinxstyleliteralemphasis{\sphinxupquote{, }}\sphinxstyleliteralemphasis{\sphinxupquote{optional}}) – Method to use.
If ‘bestpt’, finds the point in the interval that gives the average cost.
If ‘piecewise’, attempts to split the inverval into sub\sphinxhyphen{}intervals of continuity
The default is ‘piecewise’.

\item {} 
\sphinxAtStartPar
\sphinxstyleliteralstrong{\sphinxupquote{threshold}} (\sphinxstyleliteralemphasis{\sphinxupquote{float}}\sphinxstyleliteralemphasis{\sphinxupquote{, }}\sphinxstyleliteralemphasis{\sphinxupquote{optional}}) – If ‘piecewise,’ the threshold for detecting a discontinuity based on deviation from linearity. The default is 0.1.

\item {} 
\sphinxAtStartPar
\sphinxstyleliteralstrong{\sphinxupquote{sampparam}} (\sphinxstyleliteralemphasis{\sphinxupquote{float}}\sphinxstyleliteralemphasis{\sphinxupquote{, }}\sphinxstyleliteralemphasis{\sphinxupquote{optional}}) – If ‘piecewise,’ the sampparam sampparam to prune to. The default is \{‘samp’:’evenspacing’,’numpts’:1\}, which would be a single point (optimal for linear).

\end{itemize}

\end{description}\end{quote}

\end{fulllineitems}

\index{select\_points() (fmdtools.modeldef.SampleApproach method)@\spxentry{select\_points()}\spxextra{fmdtools.modeldef.SampleApproach method}}

\begin{fulllineitems}
\phantomsection\label{\detokenize{docs/fmdtools:fmdtools.modeldef.SampleApproach.select_points}}\pysiglinewithargsret{\sphinxbfcode{\sphinxupquote{select\_points}}}{\emph{\DUrole{n}{param}}, \emph{\DUrole{n}{possible\_pts}}}{}
\sphinxAtStartPar
Selects points in the list possible\_points according to a given sample strategy.
\begin{quote}\begin{description}
\item[{Parameters}] \leavevmode\begin{itemize}
\item {} 
\sphinxAtStartPar
\sphinxstyleliteralstrong{\sphinxupquote{param}} (\sphinxstyleliteralemphasis{\sphinxupquote{dict}}) – \begin{description}
\item[{Sample parameter. Has structure:}] \leavevmode\begin{itemize}
\item {} \begin{description}
\item[{’samp’}] \leavevmode{[}str (‘quad’, ‘fullint’, ‘evenspacing’,’randtimes’,’symrandtimes’){]}
\sphinxAtStartPar
sample strategy to use (quadrature, full integral, even spacing, random times, or symmetric random times)

\end{description}

\item {} \begin{description}
\item[{’numpts’}] \leavevmode{[}float{]}
\sphinxAtStartPar
number of points to use (for evenspacing, randtimes, and symrandtimes only)

\end{description}

\item {} \begin{description}
\item[{’quad’}] \leavevmode{[}quadpy quadrature{]}
\sphinxAtStartPar
quadrature object if the quadrature option is selected.

\end{description}

\end{itemize}

\end{description}


\item {} 
\sphinxAtStartPar
\sphinxstyleliteralstrong{\sphinxupquote{possible\_pts}} – list of possible points in time.

\end{itemize}

\item[{Returns}] \leavevmode
\sphinxAtStartPar
\begin{itemize}
\item {} 
\sphinxAtStartPar
\sphinxstylestrong{pts} (\sphinxstyleemphasis{list}) – selected points

\item {} 
\sphinxAtStartPar
\sphinxstylestrong{weights} (\sphinxstyleemphasis{list}) – weights for each point

\end{itemize}


\end{description}\end{quote}

\end{fulllineitems}


\end{fulllineitems}

\index{Timer (class in fmdtools.modeldef)@\spxentry{Timer}\spxextra{class in fmdtools.modeldef}}

\begin{fulllineitems}
\phantomsection\label{\detokenize{docs/fmdtools:fmdtools.modeldef.Timer}}\pysiglinewithargsret{\sphinxbfcode{\sphinxupquote{class }}\sphinxcode{\sphinxupquote{fmdtools.modeldef.}}\sphinxbfcode{\sphinxupquote{Timer}}}{\emph{\DUrole{n}{name}}}{}
\sphinxAtStartPar
Bases: \sphinxcode{\sphinxupquote{object}}

\sphinxAtStartPar
class for model timers used in functions (e.g. for conditional faults)
.. attribute:: name
\begin{quote}

\sphinxAtStartPar
timer name
\begin{quote}\begin{description}
\item[{type}] \leavevmode
\sphinxAtStartPar
str

\end{description}\end{quote}
\end{quote}
\index{time (fmdtools.modeldef.Timer attribute)@\spxentry{time}\spxextra{fmdtools.modeldef.Timer attribute}}

\begin{fulllineitems}
\phantomsection\label{\detokenize{docs/fmdtools:fmdtools.modeldef.Timer.time}}\pysigline{\sphinxbfcode{\sphinxupquote{time}}}
\sphinxAtStartPar
internal timer clock time
\begin{quote}\begin{description}
\item[{Type}] \leavevmode
\sphinxAtStartPar
float

\end{description}\end{quote}

\end{fulllineitems}

\index{tstep (fmdtools.modeldef.Timer attribute)@\spxentry{tstep}\spxextra{fmdtools.modeldef.Timer attribute}}

\begin{fulllineitems}
\phantomsection\label{\detokenize{docs/fmdtools:fmdtools.modeldef.Timer.tstep}}\pysigline{\sphinxbfcode{\sphinxupquote{tstep}}}
\sphinxAtStartPar
time to increment at each time\sphinxhyphen{}step
\begin{quote}\begin{description}
\item[{Type}] \leavevmode
\sphinxAtStartPar
float

\end{description}\end{quote}

\end{fulllineitems}

\index{mode (fmdtools.modeldef.Timer attribute)@\spxentry{mode}\spxextra{fmdtools.modeldef.Timer attribute}}

\begin{fulllineitems}
\phantomsection\label{\detokenize{docs/fmdtools:fmdtools.modeldef.Timer.mode}}\pysigline{\sphinxbfcode{\sphinxupquote{mode}}}
\sphinxAtStartPar
the internal state of the timer
\begin{quote}\begin{description}
\item[{Type}] \leavevmode
\sphinxAtStartPar
str (standby/ticking/complete)

\end{description}\end{quote}

\end{fulllineitems}

\index{in\_standby() (fmdtools.modeldef.Timer method)@\spxentry{in\_standby()}\spxextra{fmdtools.modeldef.Timer method}}

\begin{fulllineitems}
\phantomsection\label{\detokenize{docs/fmdtools:fmdtools.modeldef.Timer.in_standby}}\pysiglinewithargsret{\sphinxbfcode{\sphinxupquote{in\_standby}}}{}{}
\sphinxAtStartPar
Whether the timer is in standby (time has not been set)

\end{fulllineitems}

\index{inc() (fmdtools.modeldef.Timer method)@\spxentry{inc()}\spxextra{fmdtools.modeldef.Timer method}}

\begin{fulllineitems}
\phantomsection\label{\detokenize{docs/fmdtools:fmdtools.modeldef.Timer.inc}}\pysiglinewithargsret{\sphinxbfcode{\sphinxupquote{inc}}}{\emph{\DUrole{n}{tstep}\DUrole{o}{=}\DUrole{default_value}{{[}{]}}}}{}
\sphinxAtStartPar
Increments the time elapsed by tstep

\end{fulllineitems}

\index{is\_complete() (fmdtools.modeldef.Timer method)@\spxentry{is\_complete()}\spxextra{fmdtools.modeldef.Timer method}}

\begin{fulllineitems}
\phantomsection\label{\detokenize{docs/fmdtools:fmdtools.modeldef.Timer.is_complete}}\pysiglinewithargsret{\sphinxbfcode{\sphinxupquote{is\_complete}}}{}{}
\sphinxAtStartPar
Whether the timer is complete (after time is done incrementing)

\end{fulllineitems}

\index{is\_set() (fmdtools.modeldef.Timer method)@\spxentry{is\_set()}\spxextra{fmdtools.modeldef.Timer method}}

\begin{fulllineitems}
\phantomsection\label{\detokenize{docs/fmdtools:fmdtools.modeldef.Timer.is_set}}\pysiglinewithargsret{\sphinxbfcode{\sphinxupquote{is\_set}}}{}{}
\sphinxAtStartPar
Whether the timer is set (before time increments)

\end{fulllineitems}

\index{is\_ticking() (fmdtools.modeldef.Timer method)@\spxentry{is\_ticking()}\spxextra{fmdtools.modeldef.Timer method}}

\begin{fulllineitems}
\phantomsection\label{\detokenize{docs/fmdtools:fmdtools.modeldef.Timer.is_ticking}}\pysiglinewithargsret{\sphinxbfcode{\sphinxupquote{is\_ticking}}}{}{}
\sphinxAtStartPar
Whether the timer is ticking (time is incrementing)

\end{fulllineitems}

\index{reset() (fmdtools.modeldef.Timer method)@\spxentry{reset()}\spxextra{fmdtools.modeldef.Timer method}}

\begin{fulllineitems}
\phantomsection\label{\detokenize{docs/fmdtools:fmdtools.modeldef.Timer.reset}}\pysiglinewithargsret{\sphinxbfcode{\sphinxupquote{reset}}}{}{}
\sphinxAtStartPar
Resets the time to zero

\end{fulllineitems}

\index{set\_timer() (fmdtools.modeldef.Timer method)@\spxentry{set\_timer()}\spxextra{fmdtools.modeldef.Timer method}}

\begin{fulllineitems}
\phantomsection\label{\detokenize{docs/fmdtools:fmdtools.modeldef.Timer.set_timer}}\pysiglinewithargsret{\sphinxbfcode{\sphinxupquote{set\_timer}}}{\emph{\DUrole{n}{time}}, \emph{\DUrole{n}{tstep}\DUrole{o}{=}\DUrole{default_value}{\sphinxhyphen{} 1.0}}, \emph{\DUrole{n}{overwrite}\DUrole{o}{=}\DUrole{default_value}{'always'}}}{}
\sphinxAtStartPar
Sets timer to a given time
\begin{quote}\begin{description}
\item[{Parameters}] \leavevmode\begin{itemize}
\item {} 
\sphinxAtStartPar
\sphinxstyleliteralstrong{\sphinxupquote{time}} (\sphinxstyleliteralemphasis{\sphinxupquote{float}}) – set time to count down in the timer

\item {} 
\sphinxAtStartPar
\sphinxstyleliteralstrong{\sphinxupquote{tstep}} (\sphinxstyleliteralemphasis{\sphinxupquote{float}}\sphinxstyleliteralemphasis{\sphinxupquote{ (}}\sphinxstyleliteralemphasis{\sphinxupquote{default \sphinxhyphen{}1.0}}\sphinxstyleliteralemphasis{\sphinxupquote{)}}) – time to increment the timer at each time\sphinxhyphen{}step

\item {} 
\sphinxAtStartPar
\sphinxstyleliteralstrong{\sphinxupquote{overwrite}} (\sphinxstyleliteralemphasis{\sphinxupquote{str}}) – whether/how to overwrite the previous time
‘always’ (default) sets the time to the given time
‘if\_more’ only overwrites the old time if the new time is greater
‘if\_less’ only overwrites the old time if the new time is less
‘never’ doesn’t overwrite an existing timer unless it has reached 0.0
‘increment’ increments the previous time by the new time

\end{itemize}

\end{description}\end{quote}

\end{fulllineitems}

\index{t() (fmdtools.modeldef.Timer method)@\spxentry{t()}\spxextra{fmdtools.modeldef.Timer method}}

\begin{fulllineitems}
\phantomsection\label{\detokenize{docs/fmdtools:fmdtools.modeldef.Timer.t}}\pysiglinewithargsret{\sphinxbfcode{\sphinxupquote{t}}}{}{}
\sphinxAtStartPar
Returns the time elapsed

\end{fulllineitems}


\end{fulllineitems}

\index{accumulate() (in module fmdtools.modeldef)@\spxentry{accumulate()}\spxextra{in module fmdtools.modeldef}}

\begin{fulllineitems}
\phantomsection\label{\detokenize{docs/fmdtools:fmdtools.modeldef.accumulate}}\pysiglinewithargsret{\sphinxcode{\sphinxupquote{fmdtools.modeldef.}}\sphinxbfcode{\sphinxupquote{accumulate}}}{\emph{\DUrole{n}{vec}}}{}
\sphinxAtStartPar
Accummulates vector (e.g. if input ={[}1,1,1, 0, 1,1{]}, output = {[}1,2,3,3,4,5{]})

\end{fulllineitems}

\index{check\_model\_pickleability() (in module fmdtools.modeldef)@\spxentry{check\_model\_pickleability()}\spxextra{in module fmdtools.modeldef}}

\begin{fulllineitems}
\phantomsection\label{\detokenize{docs/fmdtools:fmdtools.modeldef.check_model_pickleability}}\pysiglinewithargsret{\sphinxcode{\sphinxupquote{fmdtools.modeldef.}}\sphinxbfcode{\sphinxupquote{check\_model\_pickleability}}}{\emph{\DUrole{n}{model}}}{}
\sphinxAtStartPar
Checks to see which attributes of a model object will pickle, providing more detail about functions/flows

\end{fulllineitems}

\index{check\_pickleability() (in module fmdtools.modeldef)@\spxentry{check\_pickleability()}\spxextra{in module fmdtools.modeldef}}

\begin{fulllineitems}
\phantomsection\label{\detokenize{docs/fmdtools:fmdtools.modeldef.check_pickleability}}\pysiglinewithargsret{\sphinxcode{\sphinxupquote{fmdtools.modeldef.}}\sphinxbfcode{\sphinxupquote{check\_pickleability}}}{\emph{\DUrole{n}{obj}}, \emph{\DUrole{n}{verbose}\DUrole{o}{=}\DUrole{default_value}{True}}}{}
\sphinxAtStartPar
Checks to see which attributes of an object will pickle (and thus parallelize)

\end{fulllineitems}

\index{find\_overlap\_n() (in module fmdtools.modeldef)@\spxentry{find\_overlap\_n()}\spxextra{in module fmdtools.modeldef}}

\begin{fulllineitems}
\phantomsection\label{\detokenize{docs/fmdtools:fmdtools.modeldef.find_overlap_n}}\pysiglinewithargsret{\sphinxcode{\sphinxupquote{fmdtools.modeldef.}}\sphinxbfcode{\sphinxupquote{find\_overlap\_n}}}{\emph{\DUrole{n}{intervals}}}{}
\sphinxAtStartPar
Finds the overlap between given intervals.
Used to sample joint fault modes with different (potentially overlapping) phases

\end{fulllineitems}

\index{is\_iter() (in module fmdtools.modeldef)@\spxentry{is\_iter()}\spxextra{in module fmdtools.modeldef}}

\begin{fulllineitems}
\phantomsection\label{\detokenize{docs/fmdtools:fmdtools.modeldef.is_iter}}\pysiglinewithargsret{\sphinxcode{\sphinxupquote{fmdtools.modeldef.}}\sphinxbfcode{\sphinxupquote{is\_iter}}}{\emph{\DUrole{n}{data}}}{}
\sphinxAtStartPar
Checks whether a data type should be interpreted as an iterable or not and returned
as a single value or tuple/array

\end{fulllineitems}

\index{m2to1() (in module fmdtools.modeldef)@\spxentry{m2to1()}\spxextra{in module fmdtools.modeldef}}

\begin{fulllineitems}
\phantomsection\label{\detokenize{docs/fmdtools:fmdtools.modeldef.m2to1}}\pysiglinewithargsret{\sphinxcode{\sphinxupquote{fmdtools.modeldef.}}\sphinxbfcode{\sphinxupquote{m2to1}}}{\emph{\DUrole{n}{x}}}{}
\sphinxAtStartPar
Multiplies a list of numbers which may take on the values infinity or zero. In deciding if num is inf or zero, the earlier values take precedence
\begin{quote}\begin{description}
\item[{Parameters}] \leavevmode
\sphinxAtStartPar
\sphinxstyleliteralstrong{\sphinxupquote{x}} (\sphinxstyleliteralemphasis{\sphinxupquote{list}}) – numbers to multiply

\item[{Returns}] \leavevmode
\sphinxAtStartPar
\sphinxstylestrong{y} – result of multiplication

\item[{Return type}] \leavevmode
\sphinxAtStartPar
float

\end{description}\end{quote}

\end{fulllineitems}

\index{phases() (in module fmdtools.modeldef)@\spxentry{phases()}\spxextra{in module fmdtools.modeldef}}

\begin{fulllineitems}
\phantomsection\label{\detokenize{docs/fmdtools:fmdtools.modeldef.phases}}\pysiglinewithargsret{\sphinxcode{\sphinxupquote{fmdtools.modeldef.}}\sphinxbfcode{\sphinxupquote{phases}}}{\emph{\DUrole{n}{times}}, \emph{\DUrole{n}{names}\DUrole{o}{=}\DUrole{default_value}{{[}{]}}}}{}
\sphinxAtStartPar
Creates named phases from a set of times defining the edges of the intervals

\end{fulllineitems}

\index{reseting\_accumulate() (in module fmdtools.modeldef)@\spxentry{reseting\_accumulate()}\spxextra{in module fmdtools.modeldef}}

\begin{fulllineitems}
\phantomsection\label{\detokenize{docs/fmdtools:fmdtools.modeldef.reseting_accumulate}}\pysiglinewithargsret{\sphinxcode{\sphinxupquote{fmdtools.modeldef.}}\sphinxbfcode{\sphinxupquote{reseting\_accumulate}}}{\emph{\DUrole{n}{vec}}}{}
\sphinxAtStartPar
Accummulates vector for all positive output (e.g. if input ={[}1,1,1, 0, 1,1{]}, output = {[}1,2,3,0,1,2{]})

\end{fulllineitems}

\index{trunc() (in module fmdtools.modeldef)@\spxentry{trunc()}\spxextra{in module fmdtools.modeldef}}

\begin{fulllineitems}
\phantomsection\label{\detokenize{docs/fmdtools:fmdtools.modeldef.trunc}}\pysiglinewithargsret{\sphinxcode{\sphinxupquote{fmdtools.modeldef.}}\sphinxbfcode{\sphinxupquote{trunc}}}{\emph{\DUrole{n}{x}}, \emph{\DUrole{n}{n}\DUrole{o}{=}\DUrole{default_value}{2.0}}, \emph{\DUrole{n}{truncif}\DUrole{o}{=}\DUrole{default_value}{'greater'}}}{}
\sphinxAtStartPar
truncates a value to a given number (useful if behavior unchanged by increases)
\begin{quote}\begin{description}
\item[{Parameters}] \leavevmode\begin{itemize}
\item {} 
\sphinxAtStartPar
\sphinxstyleliteralstrong{\sphinxupquote{x}} (\sphinxstyleliteralemphasis{\sphinxupquote{float/int}}) – number to truncate

\item {} 
\sphinxAtStartPar
\sphinxstyleliteralstrong{\sphinxupquote{n}} (\sphinxstyleliteralemphasis{\sphinxupquote{float/int}}\sphinxstyleliteralemphasis{\sphinxupquote{ (}}\sphinxstyleliteralemphasis{\sphinxupquote{optional}}\sphinxstyleliteralemphasis{\sphinxupquote{)}}) – number to truncate to if >= number

\item {} 
\sphinxAtStartPar
\sphinxstyleliteralstrong{\sphinxupquote{truncif}} (\sphinxstyleliteralemphasis{\sphinxupquote{'greater'/'less'}}) – whether to truncate if greater or less than the given number

\end{itemize}

\end{description}\end{quote}

\end{fulllineitems}

\index{union() (in module fmdtools.modeldef)@\spxentry{union()}\spxextra{in module fmdtools.modeldef}}

\begin{fulllineitems}
\phantomsection\label{\detokenize{docs/fmdtools:fmdtools.modeldef.union}}\pysiglinewithargsret{\sphinxcode{\sphinxupquote{fmdtools.modeldef.}}\sphinxbfcode{\sphinxupquote{union}}}{\emph{\DUrole{n}{probs}}}{}
\sphinxAtStartPar
Calculates the union of a list of probabilities {[}p\_1, p\_2, … p\_n{]} p = p\_1 U p\_2 U … U p\_n

\end{fulllineitems}



\renewcommand{\indexname}{Python Module Index}
\begin{sphinxtheindex}
\let\bigletter\sphinxstyleindexlettergroup
\bigletter{f}
\item\relax\sphinxstyleindexentry{fmdtools.faultsim.networks}\sphinxstyleindexpageref{docs/fmdtools.faultsim:\detokenize{module-fmdtools.faultsim.networks}}
\item\relax\sphinxstyleindexentry{fmdtools.faultsim.propagate}\sphinxstyleindexpageref{docs/fmdtools.faultsim:\detokenize{module-fmdtools.faultsim.propagate}}
\item\relax\sphinxstyleindexentry{fmdtools.modeldef}\sphinxstyleindexpageref{docs/fmdtools:\detokenize{module-fmdtools.modeldef}}
\item\relax\sphinxstyleindexentry{fmdtools.resultdisp.graph}\sphinxstyleindexpageref{docs/fmdtools.resultdisp:\detokenize{module-fmdtools.resultdisp.graph}}
\item\relax\sphinxstyleindexentry{fmdtools.resultdisp.plot}\sphinxstyleindexpageref{docs/fmdtools.resultdisp:\detokenize{module-fmdtools.resultdisp.plot}}
\item\relax\sphinxstyleindexentry{fmdtools.resultdisp.process}\sphinxstyleindexpageref{docs/fmdtools.resultdisp:\detokenize{module-fmdtools.resultdisp.process}}
\item\relax\sphinxstyleindexentry{fmdtools.resultdisp.tabulate}\sphinxstyleindexpageref{docs/fmdtools.resultdisp:\detokenize{module-fmdtools.resultdisp.tabulate}}
\end{sphinxtheindex}

\renewcommand{\indexname}{Index}
\printindex
\end{document}